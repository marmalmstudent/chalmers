\documentclass{article}
\usepackage[utf8]{inputenc}
\usepackage{amsmath}
\title{Förberedelser: K6}
\author{Marcus Malmquist}
\date{Maj 2016}

\begin{document}
\maketitle
\section{Räkneuppgifter}
\subsection*{Task 1}
\begin{equation}
  \begin{array}{l}
    ^{108}_{\text{  }47}\text{Ag} \text{ }(t_{1/2}=2.4\text{ min})\rightarrow ^{108}_{\text{  }46}\text{Cd}\text{ (stable)}+e^{-}+\bar{\nu} \\
     \\
    ^{110}_{\text{  }47}\text{Ag} \text{ }(t_{1/2}=25\text{ sec})\rightarrow ^{110}_{\text{  }46}\text{Cd}\text{ (stable)}+e^{-}+\bar{\nu}
    \end{array}
\end{equation}
\subsection*{Task 2}
\begin{center}
  \begin{tabular}{|l|l|}\hline
    m_{\text{Ag-plate}} & 10\cdot10^{-3}\text{ kg} \\
    \rho_{\text{Ag}} & 10.49\cdot10^{3}\text{ kg/m$^3$} \\
    V_{\text{Ag-plate}} & 9.53\cdot10^{-7}\text{ m}^{3} \\
    c_{^{107}\text{Ag}} & 51.8\text{ }\% \\
    c_{^{109}\text{Ag}} & 48.2\text{ }\% \\
    \Phi & 10^{10}\text{ neutrons/(m$^2$s)} \\
    \sigma(^{107}\text{Ag}) & 35\cdot10^{-28}\text{ m}^{2} \\
    \sigma(^{109}\text{Ag}) & 89\cdot10^{-28}\text{ m}^{2} \\
    M_{\text{Ag}} & 107.87\cdot10^{-3}\text{ kg/mol} \\
    N_{A} & 6.022\cdot10^{23}\\ \hline
  \end{tabular}
\end{center}\\
$N$ scattering nucleis per unit volume would then be
\begin{equation}
  N=\frac{n_{^{x}\text{Ag}}}{V_{^{x}\text{Ag}}}=\frac{m_{^{x}\text{Ag}}}{M_{\text{Ag}}}N_{A}\frac{\rho_{\text{Ag}}}{m_{^{x}\text{Ag}}}=\frac{\rho_{\text{Ag}}}{M_{\text{Ag}}}N_{A}
\end{equation}
The reaction rate is
\begin{equation}
  R=\Phi N\sigma
\end{equation}
Assuming all of the Ag atoms are taking part in the process, the production rate becomes
\begin{equation}
  \begin{array}{l}
    ^{108}\text{Ag}:c_{^{107}\text{Ag}}RV_{\text{Ag-plate}} = 1.68\cdot10^{-18} \text{ mol$\cdot$ s$^{-1}$} \\
    ^{110}\text{Ag}:c_{^{109}\text{Ag}}RV_{\text{Ag-plate}} = 3.98\cdot10^{-18} \text{ mol$\cdot$ s$^{-1}$}
  \end{array}
\end{equation}
\subsection*{Task 3}
If a neutron initially have 5 MeV and looses half of its energy each time it collides with a hydrogen nucleus, the number of collisions needed to slow down the neutron to 25 meV is
\begin{equation}
  n=\lceil\log_2{\frac{E_0}{E_n}}\rceil=\lceil\log_2{\frac{5\cdot10^{6}}{25\cdot10^{-3}}}\rceil=28
\end{equation}
The cemical formula for the paraffin used in the calculations are C$_{58}$H$_{118}$.
\begin{center}
  \begin{tabular}{|l|l|}\hline
    $n_{\text{H}}$ & 118 \\
    $\sigma$ & 20\cdot10^{-28}\text{ m$^2$} \\
    $M_{\text{C}}$ & 12.0107\text{ g/mol} \\
    $M_{\text{H}}$ & 1.00794\text{ g/mol} \\
    $M_{\text{C$_{58}$H$_{118}$}}$ & 0.81556\text{ kg/mol} \\
    $\rho_{\text{C$_{58}$H$_{118}$}}$ & 900\text{ kg/m$^3$} \\ \hline
  \end{tabular}
\end{center}\\
Using the values from the tabular above, the mean free path for a neutron
\begin{equation}
  l=\frac{1}{\Sigma}=\frac{1}{N\sigma}=\frac{M_{\text{paraffin}}}{n_{H}\rho_{\text{paraffin}}N_{A}\sigma}=6.4\text{ mm}
\end{equation}
There is no upper limit of paraffin thickness as an infinite thickness would guarantee that all of the incomming neutrons have had its energy decreased to 25 meV.
The upperlower limit, however, of the paraffin thickness would then logically be how far the average neutron would move untill its energy has decreased to 25 meV, which is
\begin{equation}
  \text{min}(\Delta x)=nl=179\text{ mm}
\end{equation}
\subsection*{Task 4}
The activation time to reach 90\% saturation is found when
\begin{equation}
  N_{act}(t)=\frac{R}{\lambda}(1-e^{-\lambda t})=0.9\frac{R}{\lambda}
\end{equation}
The half-life of the isotope used in these calculations are $t_{1/2}=180$ seconds, which makes the saturation time
\begin{equation}
  t_{90\%}=-\frac{\log{0.1}}{\lambda}=598\text{ s}=10\text{ min}
\end{equation}
\end{document}
