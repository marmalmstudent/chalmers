\documentclass[12pt,a4paper]{article}
% \documentclass[12pt,a4paper]{IEEEtran}

%\pdfoutput=1

\usepackage[utf8]{inputenc}
\usepackage[T1]{fontenc}
\usepackage[english]{babel}
\usepackage{amsmath}
\usepackage{mathabx}
\usepackage{lmodern}
\usepackage{units}
\usepackage{siunitx}
\usepackage{icomma}
\usepackage{graphicx}
\usepackage{caption}
\usepackage{subcaption}
\usepackage{color}
\usepackage{pgf}
\newcommand{\N}{\ensuremath{\mathbbm{N}}}
\newcommand{\Z}{\ensuremath{\mathbbm{Z}}}
\newcommand{\Q}{\ensuremath{\mathbbm{Q}}}
\newcommand{\R}{\ensuremath{\mathbbm{R}}}
\newcommand{\C}{\ensuremath{\mathbbm{C}}}
\newcommand{\rd}{\ensuremath{\mathrm{d}}}
\newcommand{\id}{\ensuremath{\,\rd}}
\usepackage{hyperref}
% \usepackage{a4wide} % puts the page numbering further down the page.
\usepackage{pdfpages}
\usepackage{epstopdf}
\DeclareGraphicsExtensions{.eps}

\title{Handin 1}
\author{Marcus Malmquist}
\date{\today}

\begin{document}
\maketitle
\section{Tasks 1-5}
The I-V characteristics can be found using (\ref{eq:diode_eq}) with help from the data from Table~\ref{tab:data}.
\begin{table}
  \centering
  \caption{Given data}
  \begin{tabular}{|l|l|} \hline
    $A$ & $\SI{0.1}{\milli\metre^2}$ \\\hline
    $N_{\text{epi}}$ & $1\cdot 10^{21}\SI{}{\metre^{-3}}$ \\\hline
    $N_{\text{sub,a}}$ & $1\cdot 10^{21}\SI{}{\metre^{-3}}$ \\\hline
    $N_{\text{sub,b}}$ & $1\cdot 10^{22}\SI{}{\metre^{-3}}$ \\\hline
    $N_{\text{sub,c}}$ & $1\cdot 10^{23}\SI{}{\metre^{-3}}$ \\\hline
    $L_{\text{epi}}$ & $\SI{400}{\micro\metre}$ \\\hline
    $T$ & $\SI{300}{\kelvin}$ \\\hline
    $t_{\text{p+}}$ & $\SI{200}{\nano\metre}$ \\\hline
    $t_{\text{epi}}$ & $\SI{4}{\micro\metre}$ \\\hline
    $t_{\text{sub}}$ & $\SI{250}{\micro\metre}$ \\\hline
  \end{tabular}
  \label{tab:data}
\end{table}
\begin{equation}
  \label{eq:diode_eq}
  I = AJ_0 \big(e^{\frac{qV}{kT}} - 1\big)
\end{equation}
The current density, $J_0$, for a pn-junction diode can be found using (\ref{eq:diode_cd})
\begin{equation}
  \label{eq:diode_cd}
  J_0 = q n^{2}_{i} \Bigg(\frac{D_N}{L_NN_A} + \frac{D_P}{L_PN_D}\Bigg)
\end{equation}

The diode can be modeled as it is depicted in Figure~\ref{fig:schematic}.
The substrate is modeled as a purely resistive component while there epitaxial layer and the p+ layer as a unit is modeled as an ideal diode and a resistive component in series.
\begin{figure}[!ht]
  \centering
  \noindent\makebox[\textwidth]{\scalebox{0.9}{\input{figures/schematic.pdf_t}}}
  \caption{The figure depicts a model of the diode, and how it can be described in terms of discrete components.}
  \label{fig:schematic}
\end{figure}

Since the p+ region has much higher doping concentration than the epitaxial layer, (\ref{eq:diode_eq}) is reduced to (\ref{eq:diode_red}).
\begin{equation}
  \label{eq:diode_red}
  J_0 = q n^{2}_{i} \frac{D_P}{L_PN_D}
\end{equation}
From lecture notes 1\&2 slide 20 the value for $n_{i}$ for Si at room temperature is $10^{16}\SI{}{m^{-3}}$.
From lecture notes 3\&4 slide 6 there values for $\mu_p$ and $\mu_n$ for Si can be seen in Table~\ref{tab:mobility}.
\begin{table}
  \centering
  \caption{Electron and hole mobility for some impurity concentrations.}
  \begin{tabular}{|l|l|l|} \hline
    Impurity, $N$ [$\SI{}{\metre^{-3}}$] & Hole mobility, $\mu_p$ [$\SI{}{\metre^{2}\volt^{-1}\second^{-1}}$] & Electron mobility, $\mu_n$ [$\SI{}{\metre^{2}\volt^{-1}\second^{-1}}$] \\\hline
    $10^{21}$ & $5\cdot 10^{-2}$ & $1.5\cdot 10^{-1}$ \\\hline
    $10^{22}$ & $4.5\cdot 10^{-2}$ & $1.3\cdot 10^{-1}$ \\\hline
    $10^{23}$ & $3\cdot 10^{-2}$ & $8\cdot 10^{-2}$ \\\hline
  \end{tabular}
  \label{tab:mobility}
\end{table}
$D_P$ can now be calculated using (\ref{eq:dp}).
\begin{equation}
  \label{eq:dp}
  D_P = \frac{k T}{q}\mu_n
\end{equation}
$J_0$ can now be computed using $L_D=L_{\text{epi}}$ and $N_D=N_{\text{epi}}$.

In order to compute the $I-V$ data for the diode, the voltage across the diode given an input voltage must be found.
The voltage can then be found using voltage division ($V_{\text{diode}}=\frac{R_i}{R_i+R_{\text{sub}}}V_{\text{in}}$).
The resistance for a doped semicontuctor can be found using (\ref{eq:resistance}).
\begin{equation}
  \label{eq:resistance}
  R=\rho\frac{t}{A}=\frac{t}{A q (\mu_nn + \mu_pp)}
\end{equation}
In the substrate we have $N_A=0$ and $N_D\gg n_i$ so $n \approx N_D$ and $p \ll n$, so (\ref{eq:resistance}) reduces to (\ref{eq:resistance_red}) for $R_{\text{sub}}$.

Some more reasoning is needed in order to find $R_i$ since both the epitaxial and the p+ layer contributes to $R_i$.
We note that $t_p \ll t_{\text{epi}}$ and that $N_p \gg N_{\text{epi}}$.
This means that the contribution to $R_i$ from the p+ layer should be negligible compared to that of the epitaxial layer.
Using this reasoning, (\ref{eq:resistance_red}) can also be used to find $R_{\text{epi}}$.

\begin{equation}
  \label{eq:resistance_red}
  R=\frac{t}{A q \mu_n n}
\end{equation}

The resulting $I-V$ curves for diode a, b, c and the ideal case can be seen in Figure~\ref{fig:iv_curves}.
From the $I-V$ curves in the figure it is evident that the $I-V$ curve approaches that of an ideal diode as the doping concentration increases in the substrate.
This can be explained by looking at (\ref{eq:resistance}) which states that the resistance is inversely proportional to the doping concentration.

While on the other hand the mobility decreases with higher doping concentration (lecture notes 3\&4 slide 6), the product $\mu_nn$/$\mu_pp$ increases with the doping concentration.
\begin{figure}[!ht]
  \begin{subfigure}{0.49\textwidth}
    \centering
    \noindent\makebox[\textwidth]{\scalebox{0.3}{\includegraphics{figures/task1.pdf}}}
    \caption{$I-V$ curves for diode a, b, c and ideal; lin-lin scale.}
    \label{fig:iv_linlin}
  \end{subfigure}
  \begin{subfigure}{0.49\textwidth}
    \centering
    \noindent\makebox[\textwidth]{\scalebox{0.3}{\includegraphics{figures/task2.pdf}}}
    \caption{$I-V$ curves for diode a, b, c and ideal; lin-log scale.}
    \label{fig:iv_linlog}
  \end{subfigure}
  \caption{The figures depics the resulting $I-V$ curves for diode a, b, c and ideal using both linear and logarithmic scaling on the I-axis.}
  \label{fig:iv_curves}
\end{figure}

The breakdown voltage for the diode can be found in Figure~16a of chapter 2 in the course book, where it is estimated to be $\SI{300}{\volt}$.
This value is valid for $V_{\text{diode}}$ and not for $V_{\text{in}}$ (except in the ideal case where they are the same);
In order to get the breakdown voltage for $V_{\text{diode}}$, one should multiply by a factor $\frac{R_i+R_{\text{sub}}}{R_i}$.
The breakdown voltages are described in Table~\ref{tab:breakdown}.
\begin{table}
  \centering
  \caption{Estimated breakdown voltages for some doping concentrations.}
  \begin{tabular}{|l|l|} \hline
    Impurity, $N$ [$\SI{}{\metre^{-3}}$] & Breakdown voltage [$\SI{}{\kilo\volt}$] \\\hline
    $10^{21}$ & $19$ \\\hline
    $10^{22}$ & $2.5$ \\\hline
    $10^{23}$ & $0.65$ \\\hline
  \end{tabular}
  \label{tab:breakdown}
\end{table}

\subsection{Task 4}
When the substrate width is much larger than the epitaxial and p+ layer, we see in (\ref{eq:resistance}) that the resistance should decreases dramatically.
As such, the $I-V$ curve should more closely resemble that ideal $I-V$ curve.

\end{document}