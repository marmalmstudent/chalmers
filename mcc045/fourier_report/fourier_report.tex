\documentclass[12pt,a4paper]{article}

%\pdfoutput=1

\usepackage{pagecolor}
%\definecolor{bgcolor}{HTML}{3F3F3F}
%\definecolor{fgcolor}{HTML}{DCDCCC}
%\pagecolor{bgcolor}
%\color{fgcolor}
\usepackage[utf8]{inputenc}
\usepackage[T1]{fontenc}
\usepackage[english]{babel}
\usepackage{lmodern}
%\renewcommand*\rmdefault{ptm}
\usepackage{amsmath}
\usepackage{mathabx}
\usepackage{units}
\usepackage{siunitx}
\usepackage{icomma}
\usepackage{graphicx}
\usepackage{caption}
\usepackage{subcaption}
\usepackage{color}
\usepackage{pgf}
\DeclareMathOperator{\acosh}{arccosh}
\newcommand{\N}{\ensuremath{\mathbbm{N}}}
\newcommand{\Z}{\ensuremath{\mathbbm{Z}}}
\newcommand{\Q}{\ensuremath{\mathbbm{Q}}}
\newcommand{\R}{\ensuremath{\mathbbm{R}}}
\newcommand{\C}{\ensuremath{\mathbbm{C}}}
\newcommand{\rd}{\ensuremath{\mathrm{d}}}
\newcommand{\id}{\ensuremath{\,\rd}}
\usepackage{hyperref}
%\usepackage{a4wide} % puts the page numbering further down the page.
\usepackage{pdfpages}
\usepackage{epstopdf}
\DeclareGraphicsExtensions{.eps}
\title{Fundamantals of Photonics - lab report}
\author{Marcus Malmquist, marmalm}
\date{\today}

\begin{document}
\pagenumbering{gobble}
\maketitle
\newpage
\pagenumbering{roman}
\tableofcontents
\newpage
\pagenumbering{arabic}

\section{Introduction}
This report is a summary of three separate labs. In the first lab the effects of diffraction was studied and in the second lab some characteristics of a gaussian beam (such as beam width and beam waist) was studied. In the third lab a setup to study defferent image filtering techniques was created and used to filter images both in the spatial domain and the frequency domain.

Since these labs cover separate topics the structure of this report will be somewhat unconventional and focus on explaining the topics as they are brought up and in the order that they were performed.
\section{Wavelength measurement
  through diffraction}
The wavelength of the lasers can be found using the equation for diffraction from a grating (\ref{eq:grating}), where $\Lambda$ is the separation of the grating lines, $\theta_i$ is the incident angle and $\lambda$ is the wavelength. The wavelength of a red and an orange laser was measured.
\begin{equation}
  \label{eq:grating}
  \Lambda(\sin\theta_i+\sin\theta_q)=q\lambda \text{, }q=0,1,2,...
\end{equation}
\subsection{Red laser}
The wavelength of the red laser was measured using the setup in Figure~\ref{fig:diffraction_red} and the measured quantities can be found in Table~\ref{tab:diffraction_red_vals}.

Using the values from Table~\ref{tab:diffraction_red_vals}, the angles depicted in Figure~\ref{fig:diffraction_red} could be calculated and finally the wavelength. These can be found in Table~\ref{tab:diffraction_red_calcs}. The calculated value of $\lambda$ ($\SI{627}{\nano\metre}$) was quite close to the wavelength of the laser which was $\SI{633}{\nano\metre}$.

\begin{table}
  \centering
  \begin{tabular}{|l|l|}\hline
    $L$ & $\SI{3.5}{\metre}$ \\
    $2x$ & $\SI{76.8}{\centi\metre}$ \\
    $y-x$ & $\SI{5}{\centi\metre}$ \\ 
    $\Lambda$ & $\frac{1}{64}$''$\approx\SI{400}{\micro\metre}$ \\ \hline
  \end{tabular}
  \caption{The quantities measured during the measurement of the wavelength of the red laser.}
  \label{tab:diffraction_red_vals}
\end{table}
\begin{table}
  \centering
  \begin{tabular}{|l|l|}\hline
    $\theta_0$ & $6.3^\text{o}$ \\
    $\theta_1$ & $7.1^\text{o}$ \\
    $\theta_i$ & $83.7^\text{o}$ \\
    $\lambda$ & $\SI{627}{\nano\metre}$ \\ \hline
  \end{tabular}
  \caption{The quantities calculated during the measurement of the wavelength of the red laser.}
  \label{tab:diffraction_red_calcs}
\end{table}
\begin{figure}[h]
  \centering
  \noindent\makebox[\textwidth]{\scalebox{0.80}{\input{figures/diffraction_red.pdf_t}}}
  \caption{The setup used when measuring the wavelength of the red laser.}
  \label{fig:diffraction_red}
\end{figure}
\subsection{Orange laser}
The wavelength of the orange laser was measured using the setup in Figure~\ref{fig:diffraction_orange} and the measured quantities can be found in Table~\ref{tab:diffraction_orange_vals}. The separation of the grating lines ($\Lambda$), was calculated using the red laser. When $\Lambda$ was known the orange laser was instead used and the calculated angles along with the grating line separation and the wavelength can be seen in Table~\ref{tab:diffraction_orange_calcs}.

The grating line separation was not as close to the actual value as one would have wanted which could have to do with the scaling of the setup. The distance between the disk and the board was much smaller than in the setup used to measure the wavelength of the red laser and the setup was so small that is was rather difficult to accurately measure distances. The actual value of $\Lambda$ was $\SI{740}{\nano\metre}$.

The error in $\Lambda$ affected the error in $\lambda$ which resulted in a calculated wavelength for the orange laser that was higher than the wavelength of the red lased (which is impossible). When using $\Lambda=\SI{740}{\nano\metre}$, however, $\lambda_\text{orange}$ was $\SI{620}{\nano\metre}$ which is a lot closer to the actual value ($\SI{613}{\nano\metre}$) measured using a spectrometer.

The accuracy of the measurement could have increased if a lower grating line separation was used (by using a DVD or blu-ray disk) because that would have meant more lines which in turn increases the resolution.
\begin{table}
  \centering
  \begin{tabular}{|l|l|}\hline
    $L_\text{red}$ & $\SI{21.5}{\centi\metre}$ \\
    $x_\text{red}$ & $\SI{32.5}{\centi\metre}$ \\
    $L_\text{orange}$ & $\SI{25.4}{\centi\metre}$ \\
    $x_\text{orange}$ & $\SI{38.9}{\centi\metre}$ \\ \hline
  \end{tabular}
  \caption{The quantities measured during the measurement of the wavelength of the orange laser.}
  \label{tab:diffraction_orange_vals}
\end{table}
\begin{table}
  \centering
  \begin{tabular}{|l|l|}\hline
    $\theta_{1,\text{red}}$ & $56.5^\text{o}$ \\
    $\Lambda_\text{CD}$ & $\SI{759}{\nano\metre}$ \\
    $\theta_{1,\text{orange}}$ & $56.9^\text{o}$ \\
    $\lambda$ & $\SI{636}{\nano\metre}$ \\ \hline
  \end{tabular}
  \caption{The quantities calculated during the measurement of the wavelength of the orange laser.}
  \label{tab:diffraction_orange_calcs}
\end{table}
\begin{figure}[h]
  \centering
  \noindent\makebox[\textwidth]{\scalebox{0.90}{\input{figures/diffraction_orange.pdf_t}}}
  \caption{The setup used when measuring the wavelength of the orange laser.}
  \label{fig:diffraction_orange}
\end{figure}
\section{Gaussian beam propagation}
In this section two methods for measuring the beam diameter will be used. The first method uses a knife edge to partially block the beam. A central part of this method is the rule of thumb (\ref{eq:rot}), where $x_{90\%-10\%}$ is the beam diameter that starts where their knife blocks 10\% of the beam and ends where the knife blocks 90\% of the beam (and thus contains 80\% of the beam).
\begin{equation}
  \label{eq:rot}
  x_{90\%-10\%}=1.28w_0
\end{equation}
The second method uses a circular aperture uses a circular aperture to partially block the beam. This method uses a relation between the radius of the aperture and the radius of the beam as a function of how much of the beam the aperture blocks. This relation, when the aperture blocks 50\% of the beam, can be seen in (\ref{eq:cap}).
\begin{equation}
  \label{eq:cap}
  r_{50\%}=0.59w
\end{equation}
Finally, the (half-angle) divergence as well as the beam diameter after propagating $\SI{1}{\kilo\metre}$ will beam calculated. The half-angle divergence can be calculated using (\ref{eq:ff}) which requires that the far-field approximations is valid (\ref{eq:ff_def}).
\begin{equation}
  \label{eq:ff}
  \theta_0=\lim_{z\rightarrow 0}\frac{w(z)}{z}=\frac{\lambda}{\pi w_0}
\end{equation}
\begin{equation}
  \label{eq:ff_def}
  z \gg z_0=\frac{\pi w_0^2}{\lambda}=\SI{700}{\milli\metre}
\end{equation}
The beam radius can be calculated using (\ref{eq:bw}).
\begin{equation}
  \label{eq:bw}
  w(z)=w_0\sqrt{1+\Big(\frac{z}{z_0}\Big)^2}
\end{equation}
\begin{figure}[h]
  \centering
  \noindent\makebox[\textwidth]{\scalebox{0.90}{\input{figures/gauss_setup.pdf_t}}}
  \caption{The setup used when measuring the beam width on a gaussian beam.}
  \label{fig:gauss_setup}
\end{figure}
\subsection{Without a beam expander}
When the beam expander was not present it means that $L_1$ and $L_2$ in Figure~\ref{fig:gauss_setup} are not present.
\subsubsection{Knife-edge}
$x_{90\%-10\%}$ was measured at the output of the laser using a knife attached to a mm-screw to be $\SI{475}{\micro\metre}$. Using (\ref{eq:rot}) the beam diameter was $\SI{740}{\micro\metre}$. A (very) rough estimate of the beam diameter was measured to be \textasciitilde$\SI{2}{\milli\metre}$ using a ruler which indicates that the knife-edge method works.
\subsection{Circular aperture}
The aperture diameter when the intensity had dropped to 50\% ($d_{50\%}$) was measured to be $\SI{1.03}{\centi\metre}$. Using (\ref{eq:cap}) the beam diameter was $\SI{1.76}{\milli\metre}$. A rough estimate of the beam diameter was measured to be \textasciitilde$\SI{2}{\centi\metre}$ using a ruler which indicates that the circular aperture method works.
\subsubsection{Divergence}
The far-field approximations are valid if (\ref{eq:ff_def}) is fulfilled. The half-angle, which is valid in the far-field was computed to be $5.45\cdot 10^{-4}\corresponds 0.0047^\circ$ using (\ref{eq:ff}) and the beam diameter $\SI{1}{\kilo\metre}$ from the source was computed to be $\SI{16.4}{\centi\metre}$ using (\ref{eq:bw}). This means that using a larger beam actually makes the beam smaller in the far field. This has to do with how (\ref{eq:bw}) scales with distance and the fact that increasing the beam diameter is equivalend to decreasing the distance.

\subsection{With a beam expander}
When the beam expander was present $L_1$ and $L_2$ in Figure~\ref{fig:gauss_setup} are present and $f_1=\SI{-50}{\milli\metre}$ and $f_2=\SI{300}{\milli\metre}$.
\subsubsection{Knife-edge}
$x_{90\%-10\%}$ was measured at the output of the laser using a knife attached to a mm-screw to be $\SI{3.15}{\milli\metre}$. Using (\ref{eq:rot}) the beam diameter was $\SI{4.92}{\milli\metre}$.
\subsection{Circular aperture}
The aperture diameter when the intensity had dropped to 50\% ($d_{50\%}$) was measured to be $\SI{5.5}{\centi\metre}$. Using (\ref{eq:cap}) the beam diameter was $\SI{9.32}{\centi\metre}$.
\subsubsection{Divergence}
The half-angle, which is valid in the far-field was computed to be $8.19\cdot 10^{-5}\corresponds 0.031^\circ$ using (\ref{eq:ff}) and the beam diameter $\SI{1}{\kilo\metre}$ from the source was computed to be $\SI{1.09}{\metre}$ using (\ref{eq:bw}).

\section{Fourier Optics}
This lab focused on setting up the equipment and then using different filters to get an understanding of what they do with the ultimate goal of being able to manipulate images today filter out unwanted content or impurities.
\subsection{Experimental setup}\label{sec:setup}
The available lenses that was used in this lab can be seen in Table~\ref{tab:lenses} and the setup that was initially used by combining the lenses can be seen in Figure~\ref{fig:setup0}.

The Beam expander was comprised of the two that had the highest ratio between their focal lengths. For this reason $L_1$ and $L_2$ was used, separated by the sum of their focal lengths ($\SI{350}{\milli\metre}$), expanding the beam by a factor of 8.

The object lens in the $4f$-system was chosen to have the longest focal length (i.e. $L_4$) to create the largest fourier plane. The image lens was a compound lens made from the remaining two lenses ($L_2$ and $L_3$) separated by $\SI{50}{\milli\metre}$ (from (\ref{eq:clens_sep})) to create a compound lens with a focal length of $\SI{400}{\milli\metre}$. The ``position'' of the compound lens was set to be directly inbetween $L_2$ andhelp $L_3$.

In order to see a larger image on the screen, $L_2$ was removed and $L_3$ was moved to be $\SI{147}{\milli\metre}$ from the fourier plane while the image plane was kept $\SI{800}{\milli\metre}$ (from (\ref{eq:new_lens_pos})). This results in a magnification of $44.4$. The final setup can be seen in Figure~\ref{fig:setup_final}.
\begin{equation}
  \label{eq:clens_sep}
  d=f_1+f_2-\frac{f_1f_2}{f}
\end{equation}
\begin{equation}
  \label{eq:new_lens_pos}
  \begin{cases}
    \dfrac{1}{z_1}+\dfrac{1}{z_2}=\dfrac{1}{f_3} \\
    z_1+z_2=\SI{800}{\milli\metre}
  \end{cases}
\end{equation}
\begin{table}
  \centering
  \begin{tabular}{|l|l|}\hline
    Lens & Focal length \\
    $L_1$ & $f_1=\SI{-50}{\milli\metre}$ \\
    $L_2$ & $f_2=\SI{-100}{\milli\metre}$ \\
    $L_3$ & $f_3=\SI{120}{\milli\metre}$ \\
    $L_4$ & $f_4=\SI{200}{\milli\metre}$ \\
    $L_5$ & $f_5=\SI{400}{\milli\metre}$ \\ \hline
  \end{tabular}
  \caption{The Available lenses and their focal lengths.}
  \label{tab:lenses}
\end{table}
\begin{figure}[h]
  \centering
  \noindent\makebox[\textwidth]{\scalebox{0.70}{\input{figures/lense_setup.pdf_t}}}
  \caption{The inital setup using a compound lens between the fourier plane and the image plane. Values for the focal lengths can be seen in Table~\ref{tab:lenses}.}
  \label{fig:setup0}
\end{figure}
\begin{figure}[h]
  \centering
  \noindent\makebox[\textwidth]{\scalebox{0.70}{\input{figures/lense_setup2.pdf_t}}}
  \caption{The final setup using a compound lens between the fourier plane and the image plane. Values for the focal lengths can be seen in Table~\ref{tab:lenses}.}
  \label{fig:setup_final}
\end{figure}

\subsection{Amplitude modulation}\label{sec:ampmod}
Amplitude modulation simply means that an object is placed in front of the beam that have parts that let through varying intensity. In these experiments the amplitude modulation was either full transmission or no transmission.
\subsubsection{Linear grating}
When a linear grating with vertical lines was placed in the object plane dots appeared along the horizontal axis on the image plane. The $0^\text{th}$ order was brightest and higher order spots less so. The even order spots had notable less intensity than their odd order modes. This is caused by the linewidth being very close to the line separation and the lines act as a bandpass filter. The object and the image plane can be seen in Figure~\ref{fig:linear_grating}.
\begin{figure}[h]
  \centering
  \noindent\makebox[\textwidth]{\scalebox{0.70}{\input{figures/linear_grating.pdf_t}}}
  \caption{Linear grating.}
  \label{fig:linear_grating}
\end{figure}

\subsubsection{Circular grating}
When a circular grating was placed in the object plane circles appeared on the image plane. Much like the case with the linear grating the $0^\text{th}$ order was brightest and the brightness decreased with higher order. The even order rings also had notably less intensity compared to the odd order rings and the reason is the same; their width of the rings is very close to the separation of the rings. The object and the image plane can be sen inbetween Figure~\ref{fig:circular_grating}.
\begin{figure}[h]
  \centering
  \noindent\makebox[\textwidth]{\scalebox{0.70}{\input{figures/circular_grating.pdf_t}}}
  \caption{circular grating.}
  \label{fig:circular_grating}
\end{figure}

\subsubsection{Chess-board grating}
When a chess-board grating was placed in the object plane both horizontal and vertical dots appeared on the image plane. The $0^\text{th}$ order was brightest and the brightness decreased with higher order. Unlike the other gratings the even orders was not visibly less bright than the odd orders. This is because the chess board is not quite a bandpass filter. The object and the image plane can be seen in Figure~\ref{fig:chess_board}
\begin{figure}[h]
  \centering
  \noindent\makebox[\textwidth]{\scalebox{0.70}{\input{figures/chess_board.pdf_t}}}
  \caption{chess borad.}
  \label{fig:chess_board}
\end{figure}

\subsubsection{Two gratings}
When two gratings were placed in the object plane (linear with vertical lines and circular) rings appeared along the horizontal axis. Much like the case when linear and circular gratings were used the even order spots andhelp even order circles was less bright than odd orders. This is because placing two gratings in their object plane yields a grating that is the convolution between the gratings. A convolution in the spatial domain corresponds to multiplication on the frequency domains which results in the circular pattern from the circular grating appears where the spots from the linear grating would be. The object and the image plane can be seen in Figure~\ref{fig:lin_circ_grating}.
\begin{figure}[h]
  \centering
  \noindent\makebox[\textwidth]{\scalebox{0.70}{\input{figures/lin_circ_grating.pdf_t}}}
  \caption{linear and circular grating.}
  \label{fig:lin_circ_grating}
\end{figure}

\subsection{Spatial filtering}
If gratings are placed in the fourier plane one can manipulate the spatial image.

\subsubsection{One- and two-dimensional low pass and high pass}
One-dimensional filters are created by blocking the light some distance from the center of the beam along one axis and out if it is a low pass filter. For a one-dimensional high-pass filter the light should be blocked some distance from the center of the beam along one axis and in.
The resulting image will have the sharp edges between bright and dark areas (high contrast) removed along the axis of the filter for low pass filter but for high-pass filter the areas with very low variation in brightness (low contrast) will be removed.

Two-dimensional filters are created by blocking light some distance from the center of the beam radially isotropic and out if it is a low pass filter. Forall a two-dimensional high-pass filter the light should be blocked some distance from the center of the beam radially isotropic and in.
The resulting image will have the sharp edges between bright and dark areas (high contrast) for low pass filter but for high-pass filter the areas with very low variation in brightness (low contrast) will be removed.

The filters can be seen in Figure~\ref{fig:filters}.
\begin{figure}[h]
  \centering
  \begin{subfigure}[b]{0.45\textwidth}
    \includegraphics[width=0.9\textwidth]{figures/1dlp.pdf}
    \subcaption{One-dimensional low pass filter.}
  \end{subfigure}
  \begin{subfigure}[b]{0.45\textwidth}
    \includegraphics[width=0.9\textwidth]{figures/1dhp.pdf}
    \subcaption{One-dimensional high pass filter.}
  \end{subfigure}\\
  \begin{subfigure}[b]{0.45\textwidth}
    \includegraphics[width=0.9\textwidth]{figures/2dlp.pdf}
    \subcaption{Two-dimensional low pass filter.}
  \end{subfigure}
  \begin{subfigure}[b]{0.45\textwidth}
    \includegraphics[width=0.9\textwidth]{figures/2dhp.pdf}
    \subcaption{Two-dimensional high pass filter.}
  \end{subfigure}
  \caption{One- and two-dimensional high and low pass filters.}
  \label{fig:filters}
\end{figure}

\subsubsection{Image filtering}
An image of a brick building was placed in the object plane. In order to filter out the bricks a two-dimensional low pass filter was used. This is because the bricks were rather small which means that high contrast is required to see them. Analogously a high pass filter was used to emphasize the bricks and filter out their rest of the image.

\subsubsection{Disturbance filtering}
A linear grating with some disturbance was placed in the object plane. As expected the bright spots (intensity decreasing with higher order spots) appeared in the image plane. The disturbance formed forward- and backward-slanting lines intersecting at the bright spots (forming an X-patten with the bright spot in the center). The disturbance  therefor could therefor be thin and near-horizontally aligned.
In order to filter out the disturbances a one-dimensional low pass filter was used.
In order to filter out the linear grating a one-dimensional slit was placed horizontally.

The filters can be seen in Figure~\ref{fig:filters2}.
\begin{figure}[h]
  \centering
  \noindent\makebox[\textwidth]{\scalebox{0.90}{\input{figures/disturbance.pdf_t}}}
  \caption{linear grating with some disturbance.}
  \label{fig:disturbance}
\end{figure}
\begin{figure}[h]
  \centering
  \begin{subfigure}[b]{0.45\textwidth}
    \includegraphics[width=0.9\textwidth]{figures/1dlp.pdf}
    \subcaption{One-dimensional low pass filter.}
  \end{subfigure}
  \begin{subfigure}[b]{0.45\textwidth}
    \includegraphics[width=0.9\textwidth]{figures/slit.pdf}
    \subcaption{One-dimensional slit.}
  \end{subfigure}
  \caption{One-dimensional low pass filter and a one-dimensional slit.}
  \label{fig:filters2}
\end{figure}

\end{document}