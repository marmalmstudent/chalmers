\documentclass[12pt,a4paper]{article}

%\pdfoutput=1

\usepackage[utf8]{inputenc}
\usepackage[T1]{fontenc}
\usepackage[english]{babel}
\usepackage{amsmath}
\usepackage{mathabx}
\usepackage{lmodern}
\usepackage{listings}
\usepackage{units}
\usepackage{siunitx}
\usepackage{icomma}
\usepackage{graphicx}
\usepackage{caption}
\usepackage{subcaption}
\usepackage{color}
\usepackage{pgf}
\DeclareMathOperator{\acosh}{arccosh}
\newcommand{\N}{\ensuremath{\mathbbm{N}}}
\newcommand{\Z}{\ensuremath{\mathbbm{Z}}}
\newcommand{\Q}{\ensuremath{\mathbbm{Q}}}
\newcommand{\R}{\ensuremath{\mathbbm{R}}}
\newcommand{\C}{\ensuremath{\mathbbm{C}}}
\newcommand{\rd}{\ensuremath{\mathrm{d}}}
\newcommand{\id}{\ensuremath{\,\rd}}
\usepackage{hyperref}
%\usepackage{a4wide} % puts the page numbering further down the page.
\usepackage{pdfpages}
\usepackage{epstopdf}
\DeclareGraphicsExtensions{.eps}
\def\changemargin#1#2{\list{}{\rightmargin#2\leftmargin#1}\item[]}
\let\endchangemargin=\endlist

\title{Handin 5}
\author{Marcus Malmquist, marmalm}
\date{\today}

\begin{document}
\maketitle

\section{Task 1}\label{sec:1}
%\begin{figure}
%  \centering
%  \noindent\makebox[\textwidth]{\scalebox{0.90}{%% Creator: Matplotlib, PGF backend
%%
%% To include the figure in your LaTeX document, write
%%   \input{<filename>.pgf}
%%
%% Make sure the required packages are loaded in your preamble
%%   \usepackage{pgf}
%%
%% Figures using additional raster images can only be included by \input if
%% they are in the same directory as the main LaTeX file. For loading figures
%% from other directories you can use the `import` package
%%   \usepackage{import}
%% and then include the figures with
%%   \import{<path to file>}{<filename>.pgf}
%%
%% Matplotlib used the following preamble
%%   \usepackage{fontspec}
%%   \setmainfont{DejaVu Serif}
%%   \setsansfont{DejaVu Sans}
%%   \setmonofont{DejaVu Sans Mono}
%%
\begingroup%
\makeatletter%
\begin{pgfpicture}%
\pgfpathrectangle{\pgfpointorigin}{\pgfqpoint{6.400000in}{4.800000in}}%
\pgfusepath{use as bounding box, clip}%
\begin{pgfscope}%
\pgfsetbuttcap%
\pgfsetmiterjoin%
\definecolor{currentfill}{rgb}{1.000000,1.000000,1.000000}%
\pgfsetfillcolor{currentfill}%
\pgfsetlinewidth{0.000000pt}%
\definecolor{currentstroke}{rgb}{1.000000,1.000000,1.000000}%
\pgfsetstrokecolor{currentstroke}%
\pgfsetdash{}{0pt}%
\pgfpathmoveto{\pgfqpoint{0.000000in}{0.000000in}}%
\pgfpathlineto{\pgfqpoint{6.400000in}{0.000000in}}%
\pgfpathlineto{\pgfqpoint{6.400000in}{4.800000in}}%
\pgfpathlineto{\pgfqpoint{0.000000in}{4.800000in}}%
\pgfpathclose%
\pgfusepath{fill}%
\end{pgfscope}%
\begin{pgfscope}%
\pgfsetbuttcap%
\pgfsetmiterjoin%
\definecolor{currentfill}{rgb}{1.000000,1.000000,1.000000}%
\pgfsetfillcolor{currentfill}%
\pgfsetlinewidth{0.000000pt}%
\definecolor{currentstroke}{rgb}{0.000000,0.000000,0.000000}%
\pgfsetstrokecolor{currentstroke}%
\pgfsetstrokeopacity{0.000000}%
\pgfsetdash{}{0pt}%
\pgfpathmoveto{\pgfqpoint{0.800000in}{0.528000in}}%
\pgfpathlineto{\pgfqpoint{5.760000in}{0.528000in}}%
\pgfpathlineto{\pgfqpoint{5.760000in}{4.224000in}}%
\pgfpathlineto{\pgfqpoint{0.800000in}{4.224000in}}%
\pgfpathclose%
\pgfusepath{fill}%
\end{pgfscope}%
\begin{pgfscope}%
\pgfsetbuttcap%
\pgfsetroundjoin%
\definecolor{currentfill}{rgb}{0.000000,0.000000,0.000000}%
\pgfsetfillcolor{currentfill}%
\pgfsetlinewidth{0.803000pt}%
\definecolor{currentstroke}{rgb}{0.000000,0.000000,0.000000}%
\pgfsetstrokecolor{currentstroke}%
\pgfsetdash{}{0pt}%
\pgfsys@defobject{currentmarker}{\pgfqpoint{0.000000in}{-0.048611in}}{\pgfqpoint{0.000000in}{0.000000in}}{%
\pgfpathmoveto{\pgfqpoint{0.000000in}{0.000000in}}%
\pgfpathlineto{\pgfqpoint{0.000000in}{-0.048611in}}%
\pgfusepath{stroke,fill}%
}%
\begin{pgfscope}%
\pgfsys@transformshift{0.800000in}{0.528000in}%
\pgfsys@useobject{currentmarker}{}%
\end{pgfscope}%
\end{pgfscope}%
\begin{pgfscope}%
\pgftext[x=0.800000in,y=0.430778in,,top]{\sffamily\fontsize{10.000000}{12.000000}\selectfont 0}%
\end{pgfscope}%
\begin{pgfscope}%
\pgfsetbuttcap%
\pgfsetroundjoin%
\definecolor{currentfill}{rgb}{0.000000,0.000000,0.000000}%
\pgfsetfillcolor{currentfill}%
\pgfsetlinewidth{0.803000pt}%
\definecolor{currentstroke}{rgb}{0.000000,0.000000,0.000000}%
\pgfsetstrokecolor{currentstroke}%
\pgfsetdash{}{0pt}%
\pgfsys@defobject{currentmarker}{\pgfqpoint{0.000000in}{-0.048611in}}{\pgfqpoint{0.000000in}{0.000000in}}{%
\pgfpathmoveto{\pgfqpoint{0.000000in}{0.000000in}}%
\pgfpathlineto{\pgfqpoint{0.000000in}{-0.048611in}}%
\pgfusepath{stroke,fill}%
}%
\begin{pgfscope}%
\pgfsys@transformshift{1.387678in}{0.528000in}%
\pgfsys@useobject{currentmarker}{}%
\end{pgfscope}%
\end{pgfscope}%
\begin{pgfscope}%
\pgftext[x=1.387678in,y=0.430778in,,top]{\sffamily\fontsize{10.000000}{12.000000}\selectfont 5}%
\end{pgfscope}%
\begin{pgfscope}%
\pgfsetbuttcap%
\pgfsetroundjoin%
\definecolor{currentfill}{rgb}{0.000000,0.000000,0.000000}%
\pgfsetfillcolor{currentfill}%
\pgfsetlinewidth{0.803000pt}%
\definecolor{currentstroke}{rgb}{0.000000,0.000000,0.000000}%
\pgfsetstrokecolor{currentstroke}%
\pgfsetdash{}{0pt}%
\pgfsys@defobject{currentmarker}{\pgfqpoint{0.000000in}{-0.048611in}}{\pgfqpoint{0.000000in}{0.000000in}}{%
\pgfpathmoveto{\pgfqpoint{0.000000in}{0.000000in}}%
\pgfpathlineto{\pgfqpoint{0.000000in}{-0.048611in}}%
\pgfusepath{stroke,fill}%
}%
\begin{pgfscope}%
\pgfsys@transformshift{1.975355in}{0.528000in}%
\pgfsys@useobject{currentmarker}{}%
\end{pgfscope}%
\end{pgfscope}%
\begin{pgfscope}%
\pgftext[x=1.975355in,y=0.430778in,,top]{\sffamily\fontsize{10.000000}{12.000000}\selectfont 10}%
\end{pgfscope}%
\begin{pgfscope}%
\pgfsetbuttcap%
\pgfsetroundjoin%
\definecolor{currentfill}{rgb}{0.000000,0.000000,0.000000}%
\pgfsetfillcolor{currentfill}%
\pgfsetlinewidth{0.803000pt}%
\definecolor{currentstroke}{rgb}{0.000000,0.000000,0.000000}%
\pgfsetstrokecolor{currentstroke}%
\pgfsetdash{}{0pt}%
\pgfsys@defobject{currentmarker}{\pgfqpoint{0.000000in}{-0.048611in}}{\pgfqpoint{0.000000in}{0.000000in}}{%
\pgfpathmoveto{\pgfqpoint{0.000000in}{0.000000in}}%
\pgfpathlineto{\pgfqpoint{0.000000in}{-0.048611in}}%
\pgfusepath{stroke,fill}%
}%
\begin{pgfscope}%
\pgfsys@transformshift{2.563033in}{0.528000in}%
\pgfsys@useobject{currentmarker}{}%
\end{pgfscope}%
\end{pgfscope}%
\begin{pgfscope}%
\pgftext[x=2.563033in,y=0.430778in,,top]{\sffamily\fontsize{10.000000}{12.000000}\selectfont 15}%
\end{pgfscope}%
\begin{pgfscope}%
\pgfsetbuttcap%
\pgfsetroundjoin%
\definecolor{currentfill}{rgb}{0.000000,0.000000,0.000000}%
\pgfsetfillcolor{currentfill}%
\pgfsetlinewidth{0.803000pt}%
\definecolor{currentstroke}{rgb}{0.000000,0.000000,0.000000}%
\pgfsetstrokecolor{currentstroke}%
\pgfsetdash{}{0pt}%
\pgfsys@defobject{currentmarker}{\pgfqpoint{0.000000in}{-0.048611in}}{\pgfqpoint{0.000000in}{0.000000in}}{%
\pgfpathmoveto{\pgfqpoint{0.000000in}{0.000000in}}%
\pgfpathlineto{\pgfqpoint{0.000000in}{-0.048611in}}%
\pgfusepath{stroke,fill}%
}%
\begin{pgfscope}%
\pgfsys@transformshift{3.150711in}{0.528000in}%
\pgfsys@useobject{currentmarker}{}%
\end{pgfscope}%
\end{pgfscope}%
\begin{pgfscope}%
\pgftext[x=3.150711in,y=0.430778in,,top]{\sffamily\fontsize{10.000000}{12.000000}\selectfont 20}%
\end{pgfscope}%
\begin{pgfscope}%
\pgfsetbuttcap%
\pgfsetroundjoin%
\definecolor{currentfill}{rgb}{0.000000,0.000000,0.000000}%
\pgfsetfillcolor{currentfill}%
\pgfsetlinewidth{0.803000pt}%
\definecolor{currentstroke}{rgb}{0.000000,0.000000,0.000000}%
\pgfsetstrokecolor{currentstroke}%
\pgfsetdash{}{0pt}%
\pgfsys@defobject{currentmarker}{\pgfqpoint{0.000000in}{-0.048611in}}{\pgfqpoint{0.000000in}{0.000000in}}{%
\pgfpathmoveto{\pgfqpoint{0.000000in}{0.000000in}}%
\pgfpathlineto{\pgfqpoint{0.000000in}{-0.048611in}}%
\pgfusepath{stroke,fill}%
}%
\begin{pgfscope}%
\pgfsys@transformshift{3.738389in}{0.528000in}%
\pgfsys@useobject{currentmarker}{}%
\end{pgfscope}%
\end{pgfscope}%
\begin{pgfscope}%
\pgftext[x=3.738389in,y=0.430778in,,top]{\sffamily\fontsize{10.000000}{12.000000}\selectfont 25}%
\end{pgfscope}%
\begin{pgfscope}%
\pgfsetbuttcap%
\pgfsetroundjoin%
\definecolor{currentfill}{rgb}{0.000000,0.000000,0.000000}%
\pgfsetfillcolor{currentfill}%
\pgfsetlinewidth{0.803000pt}%
\definecolor{currentstroke}{rgb}{0.000000,0.000000,0.000000}%
\pgfsetstrokecolor{currentstroke}%
\pgfsetdash{}{0pt}%
\pgfsys@defobject{currentmarker}{\pgfqpoint{0.000000in}{-0.048611in}}{\pgfqpoint{0.000000in}{0.000000in}}{%
\pgfpathmoveto{\pgfqpoint{0.000000in}{0.000000in}}%
\pgfpathlineto{\pgfqpoint{0.000000in}{-0.048611in}}%
\pgfusepath{stroke,fill}%
}%
\begin{pgfscope}%
\pgfsys@transformshift{4.326066in}{0.528000in}%
\pgfsys@useobject{currentmarker}{}%
\end{pgfscope}%
\end{pgfscope}%
\begin{pgfscope}%
\pgftext[x=4.326066in,y=0.430778in,,top]{\sffamily\fontsize{10.000000}{12.000000}\selectfont 30}%
\end{pgfscope}%
\begin{pgfscope}%
\pgfsetbuttcap%
\pgfsetroundjoin%
\definecolor{currentfill}{rgb}{0.000000,0.000000,0.000000}%
\pgfsetfillcolor{currentfill}%
\pgfsetlinewidth{0.803000pt}%
\definecolor{currentstroke}{rgb}{0.000000,0.000000,0.000000}%
\pgfsetstrokecolor{currentstroke}%
\pgfsetdash{}{0pt}%
\pgfsys@defobject{currentmarker}{\pgfqpoint{0.000000in}{-0.048611in}}{\pgfqpoint{0.000000in}{0.000000in}}{%
\pgfpathmoveto{\pgfqpoint{0.000000in}{0.000000in}}%
\pgfpathlineto{\pgfqpoint{0.000000in}{-0.048611in}}%
\pgfusepath{stroke,fill}%
}%
\begin{pgfscope}%
\pgfsys@transformshift{4.913744in}{0.528000in}%
\pgfsys@useobject{currentmarker}{}%
\end{pgfscope}%
\end{pgfscope}%
\begin{pgfscope}%
\pgftext[x=4.913744in,y=0.430778in,,top]{\sffamily\fontsize{10.000000}{12.000000}\selectfont 35}%
\end{pgfscope}%
\begin{pgfscope}%
\pgfsetbuttcap%
\pgfsetroundjoin%
\definecolor{currentfill}{rgb}{0.000000,0.000000,0.000000}%
\pgfsetfillcolor{currentfill}%
\pgfsetlinewidth{0.803000pt}%
\definecolor{currentstroke}{rgb}{0.000000,0.000000,0.000000}%
\pgfsetstrokecolor{currentstroke}%
\pgfsetdash{}{0pt}%
\pgfsys@defobject{currentmarker}{\pgfqpoint{0.000000in}{-0.048611in}}{\pgfqpoint{0.000000in}{0.000000in}}{%
\pgfpathmoveto{\pgfqpoint{0.000000in}{0.000000in}}%
\pgfpathlineto{\pgfqpoint{0.000000in}{-0.048611in}}%
\pgfusepath{stroke,fill}%
}%
\begin{pgfscope}%
\pgfsys@transformshift{5.501422in}{0.528000in}%
\pgfsys@useobject{currentmarker}{}%
\end{pgfscope}%
\end{pgfscope}%
\begin{pgfscope}%
\pgftext[x=5.501422in,y=0.430778in,,top]{\sffamily\fontsize{10.000000}{12.000000}\selectfont 40}%
\end{pgfscope}%
\begin{pgfscope}%
\pgftext[x=3.280000in,y=0.240809in,,top]{\sffamily\fontsize{16.000000}{19.200000}\selectfont \(\displaystyle z-position [\mu m]\)}%
\end{pgfscope}%
\begin{pgfscope}%
\pgfsetbuttcap%
\pgfsetroundjoin%
\definecolor{currentfill}{rgb}{0.000000,0.000000,0.000000}%
\pgfsetfillcolor{currentfill}%
\pgfsetlinewidth{0.803000pt}%
\definecolor{currentstroke}{rgb}{0.000000,0.000000,0.000000}%
\pgfsetstrokecolor{currentstroke}%
\pgfsetdash{}{0pt}%
\pgfsys@defobject{currentmarker}{\pgfqpoint{-0.048611in}{0.000000in}}{\pgfqpoint{0.000000in}{0.000000in}}{%
\pgfpathmoveto{\pgfqpoint{0.000000in}{0.000000in}}%
\pgfpathlineto{\pgfqpoint{-0.048611in}{0.000000in}}%
\pgfusepath{stroke,fill}%
}%
\begin{pgfscope}%
\pgfsys@transformshift{0.800000in}{0.528000in}%
\pgfsys@useobject{currentmarker}{}%
\end{pgfscope}%
\end{pgfscope}%
\begin{pgfscope}%
\pgftext[x=0.365525in,y=0.475238in,left,base]{\sffamily\fontsize{10.000000}{12.000000}\selectfont -2.0}%
\end{pgfscope}%
\begin{pgfscope}%
\pgfsetbuttcap%
\pgfsetroundjoin%
\definecolor{currentfill}{rgb}{0.000000,0.000000,0.000000}%
\pgfsetfillcolor{currentfill}%
\pgfsetlinewidth{0.803000pt}%
\definecolor{currentstroke}{rgb}{0.000000,0.000000,0.000000}%
\pgfsetstrokecolor{currentstroke}%
\pgfsetdash{}{0pt}%
\pgfsys@defobject{currentmarker}{\pgfqpoint{-0.048611in}{0.000000in}}{\pgfqpoint{0.000000in}{0.000000in}}{%
\pgfpathmoveto{\pgfqpoint{0.000000in}{0.000000in}}%
\pgfpathlineto{\pgfqpoint{-0.048611in}{0.000000in}}%
\pgfusepath{stroke,fill}%
}%
\begin{pgfscope}%
\pgfsys@transformshift{0.800000in}{0.990000in}%
\pgfsys@useobject{currentmarker}{}%
\end{pgfscope}%
\end{pgfscope}%
\begin{pgfscope}%
\pgftext[x=0.365525in,y=0.937238in,left,base]{\sffamily\fontsize{10.000000}{12.000000}\selectfont -1.5}%
\end{pgfscope}%
\begin{pgfscope}%
\pgfsetbuttcap%
\pgfsetroundjoin%
\definecolor{currentfill}{rgb}{0.000000,0.000000,0.000000}%
\pgfsetfillcolor{currentfill}%
\pgfsetlinewidth{0.803000pt}%
\definecolor{currentstroke}{rgb}{0.000000,0.000000,0.000000}%
\pgfsetstrokecolor{currentstroke}%
\pgfsetdash{}{0pt}%
\pgfsys@defobject{currentmarker}{\pgfqpoint{-0.048611in}{0.000000in}}{\pgfqpoint{0.000000in}{0.000000in}}{%
\pgfpathmoveto{\pgfqpoint{0.000000in}{0.000000in}}%
\pgfpathlineto{\pgfqpoint{-0.048611in}{0.000000in}}%
\pgfusepath{stroke,fill}%
}%
\begin{pgfscope}%
\pgfsys@transformshift{0.800000in}{1.452000in}%
\pgfsys@useobject{currentmarker}{}%
\end{pgfscope}%
\end{pgfscope}%
\begin{pgfscope}%
\pgftext[x=0.365525in,y=1.399238in,left,base]{\sffamily\fontsize{10.000000}{12.000000}\selectfont -1.0}%
\end{pgfscope}%
\begin{pgfscope}%
\pgfsetbuttcap%
\pgfsetroundjoin%
\definecolor{currentfill}{rgb}{0.000000,0.000000,0.000000}%
\pgfsetfillcolor{currentfill}%
\pgfsetlinewidth{0.803000pt}%
\definecolor{currentstroke}{rgb}{0.000000,0.000000,0.000000}%
\pgfsetstrokecolor{currentstroke}%
\pgfsetdash{}{0pt}%
\pgfsys@defobject{currentmarker}{\pgfqpoint{-0.048611in}{0.000000in}}{\pgfqpoint{0.000000in}{0.000000in}}{%
\pgfpathmoveto{\pgfqpoint{0.000000in}{0.000000in}}%
\pgfpathlineto{\pgfqpoint{-0.048611in}{0.000000in}}%
\pgfusepath{stroke,fill}%
}%
\begin{pgfscope}%
\pgfsys@transformshift{0.800000in}{1.914000in}%
\pgfsys@useobject{currentmarker}{}%
\end{pgfscope}%
\end{pgfscope}%
\begin{pgfscope}%
\pgftext[x=0.365525in,y=1.861238in,left,base]{\sffamily\fontsize{10.000000}{12.000000}\selectfont -0.5}%
\end{pgfscope}%
\begin{pgfscope}%
\pgfsetbuttcap%
\pgfsetroundjoin%
\definecolor{currentfill}{rgb}{0.000000,0.000000,0.000000}%
\pgfsetfillcolor{currentfill}%
\pgfsetlinewidth{0.803000pt}%
\definecolor{currentstroke}{rgb}{0.000000,0.000000,0.000000}%
\pgfsetstrokecolor{currentstroke}%
\pgfsetdash{}{0pt}%
\pgfsys@defobject{currentmarker}{\pgfqpoint{-0.048611in}{0.000000in}}{\pgfqpoint{0.000000in}{0.000000in}}{%
\pgfpathmoveto{\pgfqpoint{0.000000in}{0.000000in}}%
\pgfpathlineto{\pgfqpoint{-0.048611in}{0.000000in}}%
\pgfusepath{stroke,fill}%
}%
\begin{pgfscope}%
\pgfsys@transformshift{0.800000in}{2.376000in}%
\pgfsys@useobject{currentmarker}{}%
\end{pgfscope}%
\end{pgfscope}%
\begin{pgfscope}%
\pgftext[x=0.481898in,y=2.323238in,left,base]{\sffamily\fontsize{10.000000}{12.000000}\selectfont 0.0}%
\end{pgfscope}%
\begin{pgfscope}%
\pgfsetbuttcap%
\pgfsetroundjoin%
\definecolor{currentfill}{rgb}{0.000000,0.000000,0.000000}%
\pgfsetfillcolor{currentfill}%
\pgfsetlinewidth{0.803000pt}%
\definecolor{currentstroke}{rgb}{0.000000,0.000000,0.000000}%
\pgfsetstrokecolor{currentstroke}%
\pgfsetdash{}{0pt}%
\pgfsys@defobject{currentmarker}{\pgfqpoint{-0.048611in}{0.000000in}}{\pgfqpoint{0.000000in}{0.000000in}}{%
\pgfpathmoveto{\pgfqpoint{0.000000in}{0.000000in}}%
\pgfpathlineto{\pgfqpoint{-0.048611in}{0.000000in}}%
\pgfusepath{stroke,fill}%
}%
\begin{pgfscope}%
\pgfsys@transformshift{0.800000in}{2.838000in}%
\pgfsys@useobject{currentmarker}{}%
\end{pgfscope}%
\end{pgfscope}%
\begin{pgfscope}%
\pgftext[x=0.481898in,y=2.785238in,left,base]{\sffamily\fontsize{10.000000}{12.000000}\selectfont 0.5}%
\end{pgfscope}%
\begin{pgfscope}%
\pgfsetbuttcap%
\pgfsetroundjoin%
\definecolor{currentfill}{rgb}{0.000000,0.000000,0.000000}%
\pgfsetfillcolor{currentfill}%
\pgfsetlinewidth{0.803000pt}%
\definecolor{currentstroke}{rgb}{0.000000,0.000000,0.000000}%
\pgfsetstrokecolor{currentstroke}%
\pgfsetdash{}{0pt}%
\pgfsys@defobject{currentmarker}{\pgfqpoint{-0.048611in}{0.000000in}}{\pgfqpoint{0.000000in}{0.000000in}}{%
\pgfpathmoveto{\pgfqpoint{0.000000in}{0.000000in}}%
\pgfpathlineto{\pgfqpoint{-0.048611in}{0.000000in}}%
\pgfusepath{stroke,fill}%
}%
\begin{pgfscope}%
\pgfsys@transformshift{0.800000in}{3.300000in}%
\pgfsys@useobject{currentmarker}{}%
\end{pgfscope}%
\end{pgfscope}%
\begin{pgfscope}%
\pgftext[x=0.481898in,y=3.247238in,left,base]{\sffamily\fontsize{10.000000}{12.000000}\selectfont 1.0}%
\end{pgfscope}%
\begin{pgfscope}%
\pgfsetbuttcap%
\pgfsetroundjoin%
\definecolor{currentfill}{rgb}{0.000000,0.000000,0.000000}%
\pgfsetfillcolor{currentfill}%
\pgfsetlinewidth{0.803000pt}%
\definecolor{currentstroke}{rgb}{0.000000,0.000000,0.000000}%
\pgfsetstrokecolor{currentstroke}%
\pgfsetdash{}{0pt}%
\pgfsys@defobject{currentmarker}{\pgfqpoint{-0.048611in}{0.000000in}}{\pgfqpoint{0.000000in}{0.000000in}}{%
\pgfpathmoveto{\pgfqpoint{0.000000in}{0.000000in}}%
\pgfpathlineto{\pgfqpoint{-0.048611in}{0.000000in}}%
\pgfusepath{stroke,fill}%
}%
\begin{pgfscope}%
\pgfsys@transformshift{0.800000in}{3.762000in}%
\pgfsys@useobject{currentmarker}{}%
\end{pgfscope}%
\end{pgfscope}%
\begin{pgfscope}%
\pgftext[x=0.481898in,y=3.709238in,left,base]{\sffamily\fontsize{10.000000}{12.000000}\selectfont 1.5}%
\end{pgfscope}%
\begin{pgfscope}%
\pgfsetbuttcap%
\pgfsetroundjoin%
\definecolor{currentfill}{rgb}{0.000000,0.000000,0.000000}%
\pgfsetfillcolor{currentfill}%
\pgfsetlinewidth{0.803000pt}%
\definecolor{currentstroke}{rgb}{0.000000,0.000000,0.000000}%
\pgfsetstrokecolor{currentstroke}%
\pgfsetdash{}{0pt}%
\pgfsys@defobject{currentmarker}{\pgfqpoint{-0.048611in}{0.000000in}}{\pgfqpoint{0.000000in}{0.000000in}}{%
\pgfpathmoveto{\pgfqpoint{0.000000in}{0.000000in}}%
\pgfpathlineto{\pgfqpoint{-0.048611in}{0.000000in}}%
\pgfusepath{stroke,fill}%
}%
\begin{pgfscope}%
\pgfsys@transformshift{0.800000in}{4.224000in}%
\pgfsys@useobject{currentmarker}{}%
\end{pgfscope}%
\end{pgfscope}%
\begin{pgfscope}%
\pgftext[x=0.481898in,y=4.171238in,left,base]{\sffamily\fontsize{10.000000}{12.000000}\selectfont 2.0}%
\end{pgfscope}%
\begin{pgfscope}%
\pgftext[x=0.309969in,y=2.376000in,,bottom,rotate=90.000000]{\sffamily\fontsize{16.000000}{19.200000}\selectfont \(\displaystyle E-field\)}%
\end{pgfscope}%
\begin{pgfscope}%
\pgfpathrectangle{\pgfqpoint{0.800000in}{0.528000in}}{\pgfqpoint{4.960000in}{3.696000in}} %
\pgfusepath{clip}%
\pgfsetrectcap%
\pgfsetroundjoin%
\pgfsetlinewidth{1.505625pt}%
\definecolor{currentstroke}{rgb}{0.000000,0.000000,0.000000}%
\pgfsetstrokecolor{currentstroke}%
\pgfsetdash{}{0pt}%
\pgfpathmoveto{\pgfqpoint{0.800000in}{2.376000in}}%
\pgfpathlineto{\pgfqpoint{4.182720in}{2.377067in}}%
\pgfpathlineto{\pgfqpoint{4.190160in}{2.381524in}}%
\pgfpathlineto{\pgfqpoint{4.205040in}{2.393425in}}%
\pgfpathlineto{\pgfqpoint{4.207520in}{2.393024in}}%
\pgfpathlineto{\pgfqpoint{4.210000in}{2.390930in}}%
\pgfpathlineto{\pgfqpoint{4.212480in}{2.386761in}}%
\pgfpathlineto{\pgfqpoint{4.217440in}{2.371137in}}%
\pgfpathlineto{\pgfqpoint{4.222400in}{2.345622in}}%
\pgfpathlineto{\pgfqpoint{4.234800in}{2.267676in}}%
\pgfpathlineto{\pgfqpoint{4.237280in}{2.258694in}}%
\pgfpathlineto{\pgfqpoint{4.239760in}{2.255222in}}%
\pgfpathlineto{\pgfqpoint{4.242240in}{2.258570in}}%
\pgfpathlineto{\pgfqpoint{4.244720in}{2.269786in}}%
\pgfpathlineto{\pgfqpoint{4.249680in}{2.317996in}}%
\pgfpathlineto{\pgfqpoint{4.254640in}{2.398876in}}%
\pgfpathlineto{\pgfqpoint{4.269520in}{2.690827in}}%
\pgfpathlineto{\pgfqpoint{4.272000in}{2.715921in}}%
\pgfpathlineto{\pgfqpoint{4.274480in}{2.726276in}}%
\pgfpathlineto{\pgfqpoint{4.276960in}{2.719817in}}%
\pgfpathlineto{\pgfqpoint{4.279440in}{2.695263in}}%
\pgfpathlineto{\pgfqpoint{4.284400in}{2.591532in}}%
\pgfpathlineto{\pgfqpoint{4.289360in}{2.424820in}}%
\pgfpathlineto{\pgfqpoint{4.304240in}{1.855870in}}%
\pgfpathlineto{\pgfqpoint{4.306720in}{1.803525in}}%
\pgfpathlineto{\pgfqpoint{4.309200in}{1.775210in}}%
\pgfpathlineto{\pgfqpoint{4.311680in}{1.773548in}}%
\pgfpathlineto{\pgfqpoint{4.314160in}{1.799905in}}%
\pgfpathlineto{\pgfqpoint{4.316640in}{1.854274in}}%
\pgfpathlineto{\pgfqpoint{4.321600in}{2.039943in}}%
\pgfpathlineto{\pgfqpoint{4.329040in}{2.450170in}}%
\pgfpathlineto{\pgfqpoint{4.338960in}{2.985332in}}%
\pgfpathlineto{\pgfqpoint{4.343920in}{3.131461in}}%
\pgfpathlineto{\pgfqpoint{4.346400in}{3.157285in}}%
\pgfpathlineto{\pgfqpoint{4.348880in}{3.148560in}}%
\pgfpathlineto{\pgfqpoint{4.351360in}{3.104965in}}%
\pgfpathlineto{\pgfqpoint{4.356320in}{2.920037in}}%
\pgfpathlineto{\pgfqpoint{4.363760in}{2.463068in}}%
\pgfpathlineto{\pgfqpoint{4.373680in}{1.803971in}}%
\pgfpathlineto{\pgfqpoint{4.378640in}{1.586619in}}%
\pgfpathlineto{\pgfqpoint{4.381120in}{1.526687in}}%
\pgfpathlineto{\pgfqpoint{4.383600in}{1.503760in}}%
\pgfpathlineto{\pgfqpoint{4.386080in}{1.519189in}}%
\pgfpathlineto{\pgfqpoint{4.388560in}{1.572592in}}%
\pgfpathlineto{\pgfqpoint{4.393520in}{1.783302in}}%
\pgfpathlineto{\pgfqpoint{4.400960in}{2.282569in}}%
\pgfpathlineto{\pgfqpoint{4.410880in}{2.979745in}}%
\pgfpathlineto{\pgfqpoint{4.415840in}{3.203381in}}%
\pgfpathlineto{\pgfqpoint{4.418320in}{3.263376in}}%
\pgfpathlineto{\pgfqpoint{4.420800in}{3.284610in}}%
\pgfpathlineto{\pgfqpoint{4.423280in}{3.266032in}}%
\pgfpathlineto{\pgfqpoint{4.425760in}{3.208354in}}%
\pgfpathlineto{\pgfqpoint{4.430720in}{2.987100in}}%
\pgfpathlineto{\pgfqpoint{4.438160in}{2.471739in}}%
\pgfpathlineto{\pgfqpoint{4.448080in}{1.761372in}}%
\pgfpathlineto{\pgfqpoint{4.453040in}{1.535944in}}%
\pgfpathlineto{\pgfqpoint{4.455520in}{1.476108in}}%
\pgfpathlineto{\pgfqpoint{4.458000in}{1.455602in}}%
\pgfpathlineto{\pgfqpoint{4.460480in}{1.475357in}}%
\pgfpathlineto{\pgfqpoint{4.462960in}{1.534536in}}%
\pgfpathlineto{\pgfqpoint{4.467920in}{1.759279in}}%
\pgfpathlineto{\pgfqpoint{4.475360in}{2.279588in}}%
\pgfpathlineto{\pgfqpoint{4.485280in}{2.993598in}}%
\pgfpathlineto{\pgfqpoint{4.490240in}{3.219397in}}%
\pgfpathlineto{\pgfqpoint{4.492720in}{3.279133in}}%
\pgfpathlineto{\pgfqpoint{4.495200in}{3.299395in}}%
\pgfpathlineto{\pgfqpoint{4.497680in}{3.279291in}}%
\pgfpathlineto{\pgfqpoint{4.500160in}{3.219694in}}%
\pgfpathlineto{\pgfqpoint{4.505120in}{2.994043in}}%
\pgfpathlineto{\pgfqpoint{4.512560in}{2.472561in}}%
\pgfpathlineto{\pgfqpoint{4.522480in}{1.757803in}}%
\pgfpathlineto{\pgfqpoint{4.527440in}{1.531962in}}%
\pgfpathlineto{\pgfqpoint{4.529920in}{1.472262in}}%
\pgfpathlineto{\pgfqpoint{4.532400in}{1.452061in}}%
\pgfpathlineto{\pgfqpoint{4.534880in}{1.472241in}}%
\pgfpathlineto{\pgfqpoint{4.537360in}{1.531922in}}%
\pgfpathlineto{\pgfqpoint{4.542320in}{1.757743in}}%
\pgfpathlineto{\pgfqpoint{4.549760in}{2.279417in}}%
\pgfpathlineto{\pgfqpoint{4.559680in}{2.994272in}}%
\pgfpathlineto{\pgfqpoint{4.564640in}{3.220112in}}%
\pgfpathlineto{\pgfqpoint{4.567120in}{3.279805in}}%
\pgfpathlineto{\pgfqpoint{4.569600in}{3.299998in}}%
\pgfpathlineto{\pgfqpoint{4.572080in}{3.279807in}}%
\pgfpathlineto{\pgfqpoint{4.574560in}{3.220115in}}%
\pgfpathlineto{\pgfqpoint{4.579520in}{2.994276in}}%
\pgfpathlineto{\pgfqpoint{4.586960in}{2.472584in}}%
\pgfpathlineto{\pgfqpoint{4.596880in}{1.757723in}}%
\pgfpathlineto{\pgfqpoint{4.601840in}{1.531884in}}%
\pgfpathlineto{\pgfqpoint{4.604320in}{1.472192in}}%
\pgfpathlineto{\pgfqpoint{4.606800in}{1.452000in}}%
\pgfpathlineto{\pgfqpoint{4.609280in}{1.472192in}}%
\pgfpathlineto{\pgfqpoint{4.611760in}{1.531884in}}%
\pgfpathlineto{\pgfqpoint{4.616720in}{1.757723in}}%
\pgfpathlineto{\pgfqpoint{4.624160in}{2.279416in}}%
\pgfpathlineto{\pgfqpoint{4.634080in}{2.994277in}}%
\pgfpathlineto{\pgfqpoint{4.639040in}{3.220116in}}%
\pgfpathlineto{\pgfqpoint{4.641520in}{3.279808in}}%
\pgfpathlineto{\pgfqpoint{4.644000in}{3.300000in}}%
\pgfpathlineto{\pgfqpoint{4.646480in}{3.279808in}}%
\pgfpathlineto{\pgfqpoint{4.648960in}{3.220116in}}%
\pgfpathlineto{\pgfqpoint{4.653920in}{2.994277in}}%
\pgfpathlineto{\pgfqpoint{4.661360in}{2.472584in}}%
\pgfpathlineto{\pgfqpoint{4.671280in}{1.757723in}}%
\pgfpathlineto{\pgfqpoint{4.676240in}{1.531884in}}%
\pgfpathlineto{\pgfqpoint{4.678720in}{1.472192in}}%
\pgfpathlineto{\pgfqpoint{4.681200in}{1.452000in}}%
\pgfpathlineto{\pgfqpoint{4.683680in}{1.472192in}}%
\pgfpathlineto{\pgfqpoint{4.686160in}{1.531884in}}%
\pgfpathlineto{\pgfqpoint{4.691120in}{1.757723in}}%
\pgfpathlineto{\pgfqpoint{4.698560in}{2.279416in}}%
\pgfpathlineto{\pgfqpoint{4.708480in}{2.994276in}}%
\pgfpathlineto{\pgfqpoint{4.713440in}{3.220115in}}%
\pgfpathlineto{\pgfqpoint{4.715920in}{3.279807in}}%
\pgfpathlineto{\pgfqpoint{4.718400in}{3.299998in}}%
\pgfpathlineto{\pgfqpoint{4.720880in}{3.279805in}}%
\pgfpathlineto{\pgfqpoint{4.723360in}{3.220112in}}%
\pgfpathlineto{\pgfqpoint{4.728320in}{2.994272in}}%
\pgfpathlineto{\pgfqpoint{4.735760in}{2.472583in}}%
\pgfpathlineto{\pgfqpoint{4.745680in}{1.757743in}}%
\pgfpathlineto{\pgfqpoint{4.750640in}{1.531922in}}%
\pgfpathlineto{\pgfqpoint{4.753120in}{1.472241in}}%
\pgfpathlineto{\pgfqpoint{4.755600in}{1.452061in}}%
\pgfpathlineto{\pgfqpoint{4.758080in}{1.472262in}}%
\pgfpathlineto{\pgfqpoint{4.760560in}{1.531962in}}%
\pgfpathlineto{\pgfqpoint{4.765520in}{1.757803in}}%
\pgfpathlineto{\pgfqpoint{4.772960in}{2.279436in}}%
\pgfpathlineto{\pgfqpoint{4.782880in}{2.994043in}}%
\pgfpathlineto{\pgfqpoint{4.787840in}{3.219694in}}%
\pgfpathlineto{\pgfqpoint{4.790320in}{3.279291in}}%
\pgfpathlineto{\pgfqpoint{4.792800in}{3.299395in}}%
\pgfpathlineto{\pgfqpoint{4.795280in}{3.279133in}}%
\pgfpathlineto{\pgfqpoint{4.797760in}{3.219397in}}%
\pgfpathlineto{\pgfqpoint{4.802720in}{2.993598in}}%
\pgfpathlineto{\pgfqpoint{4.810160in}{2.472431in}}%
\pgfpathlineto{\pgfqpoint{4.820080in}{1.759279in}}%
\pgfpathlineto{\pgfqpoint{4.825040in}{1.534536in}}%
\pgfpathlineto{\pgfqpoint{4.827520in}{1.475357in}}%
\pgfpathlineto{\pgfqpoint{4.830000in}{1.455602in}}%
\pgfpathlineto{\pgfqpoint{4.832480in}{1.476108in}}%
\pgfpathlineto{\pgfqpoint{4.834960in}{1.535944in}}%
\pgfpathlineto{\pgfqpoint{4.839920in}{1.761372in}}%
\pgfpathlineto{\pgfqpoint{4.847360in}{2.280183in}}%
\pgfpathlineto{\pgfqpoint{4.857280in}{2.987100in}}%
\pgfpathlineto{\pgfqpoint{4.862240in}{3.208354in}}%
\pgfpathlineto{\pgfqpoint{4.864720in}{3.266032in}}%
\pgfpathlineto{\pgfqpoint{4.867200in}{3.284610in}}%
\pgfpathlineto{\pgfqpoint{4.869680in}{3.263376in}}%
\pgfpathlineto{\pgfqpoint{4.872160in}{3.203381in}}%
\pgfpathlineto{\pgfqpoint{4.877120in}{2.979745in}}%
\pgfpathlineto{\pgfqpoint{4.884560in}{2.469675in}}%
\pgfpathlineto{\pgfqpoint{4.894480in}{1.783302in}}%
\pgfpathlineto{\pgfqpoint{4.899440in}{1.572592in}}%
\pgfpathlineto{\pgfqpoint{4.901920in}{1.519189in}}%
\pgfpathlineto{\pgfqpoint{4.904400in}{1.503760in}}%
\pgfpathlineto{\pgfqpoint{4.906880in}{1.526687in}}%
\pgfpathlineto{\pgfqpoint{4.909360in}{1.586619in}}%
\pgfpathlineto{\pgfqpoint{4.914320in}{1.803971in}}%
\pgfpathlineto{\pgfqpoint{4.921760in}{2.288310in}}%
\pgfpathlineto{\pgfqpoint{4.931680in}{2.920037in}}%
\pgfpathlineto{\pgfqpoint{4.936640in}{3.104965in}}%
\pgfpathlineto{\pgfqpoint{4.939120in}{3.148560in}}%
\pgfpathlineto{\pgfqpoint{4.941600in}{3.157285in}}%
\pgfpathlineto{\pgfqpoint{4.944080in}{3.131461in}}%
\pgfpathlineto{\pgfqpoint{4.946560in}{3.073009in}}%
\pgfpathlineto{\pgfqpoint{4.951520in}{2.873125in}}%
\pgfpathlineto{\pgfqpoint{4.971360in}{1.854274in}}%
\pgfpathlineto{\pgfqpoint{4.973840in}{1.799905in}}%
\pgfpathlineto{\pgfqpoint{4.976320in}{1.773548in}}%
\pgfpathlineto{\pgfqpoint{4.978800in}{1.775210in}}%
\pgfpathlineto{\pgfqpoint{4.981280in}{1.803525in}}%
\pgfpathlineto{\pgfqpoint{4.986240in}{1.928553in}}%
\pgfpathlineto{\pgfqpoint{4.993680in}{2.220789in}}%
\pgfpathlineto{\pgfqpoint{5.001120in}{2.514790in}}%
\pgfpathlineto{\pgfqpoint{5.006080in}{2.652272in}}%
\pgfpathlineto{\pgfqpoint{5.011040in}{2.719817in}}%
\pgfpathlineto{\pgfqpoint{5.013520in}{2.726276in}}%
\pgfpathlineto{\pgfqpoint{5.016000in}{2.715921in}}%
\pgfpathlineto{\pgfqpoint{5.018480in}{2.690827in}}%
\pgfpathlineto{\pgfqpoint{5.023440in}{2.607624in}}%
\pgfpathlineto{\pgfqpoint{5.038320in}{2.317996in}}%
\pgfpathlineto{\pgfqpoint{5.043280in}{2.269786in}}%
\pgfpathlineto{\pgfqpoint{5.045760in}{2.258570in}}%
\pgfpathlineto{\pgfqpoint{5.048240in}{2.255222in}}%
\pgfpathlineto{\pgfqpoint{5.050720in}{2.258694in}}%
\pgfpathlineto{\pgfqpoint{5.055680in}{2.280716in}}%
\pgfpathlineto{\pgfqpoint{5.070560in}{2.371137in}}%
\pgfpathlineto{\pgfqpoint{5.075520in}{2.386761in}}%
\pgfpathlineto{\pgfqpoint{5.080480in}{2.393024in}}%
\pgfpathlineto{\pgfqpoint{5.082960in}{2.393425in}}%
\pgfpathlineto{\pgfqpoint{5.087920in}{2.390794in}}%
\pgfpathlineto{\pgfqpoint{5.102800in}{2.378186in}}%
\pgfpathlineto{\pgfqpoint{5.110240in}{2.375781in}}%
\pgfpathlineto{\pgfqpoint{5.120160in}{2.375445in}}%
\pgfpathlineto{\pgfqpoint{5.162320in}{2.376001in}}%
\pgfpathlineto{\pgfqpoint{5.757520in}{2.376000in}}%
\pgfpathlineto{\pgfqpoint{5.757520in}{2.376000in}}%
\pgfusepath{stroke}%
\end{pgfscope}%
\begin{pgfscope}%
\pgfsetrectcap%
\pgfsetmiterjoin%
\pgfsetlinewidth{0.803000pt}%
\definecolor{currentstroke}{rgb}{0.000000,0.000000,0.000000}%
\pgfsetstrokecolor{currentstroke}%
\pgfsetdash{}{0pt}%
\pgfpathmoveto{\pgfqpoint{0.800000in}{0.528000in}}%
\pgfpathlineto{\pgfqpoint{0.800000in}{4.224000in}}%
\pgfusepath{stroke}%
\end{pgfscope}%
\begin{pgfscope}%
\pgfsetrectcap%
\pgfsetmiterjoin%
\pgfsetlinewidth{0.803000pt}%
\definecolor{currentstroke}{rgb}{0.000000,0.000000,0.000000}%
\pgfsetstrokecolor{currentstroke}%
\pgfsetdash{}{0pt}%
\pgfpathmoveto{\pgfqpoint{5.760000in}{0.528000in}}%
\pgfpathlineto{\pgfqpoint{5.760000in}{4.224000in}}%
\pgfusepath{stroke}%
\end{pgfscope}%
\begin{pgfscope}%
\pgfsetrectcap%
\pgfsetmiterjoin%
\pgfsetlinewidth{0.803000pt}%
\definecolor{currentstroke}{rgb}{0.000000,0.000000,0.000000}%
\pgfsetstrokecolor{currentstroke}%
\pgfsetdash{}{0pt}%
\pgfpathmoveto{\pgfqpoint{0.800000in}{0.528000in}}%
\pgfpathlineto{\pgfqpoint{5.760000in}{0.528000in}}%
\pgfusepath{stroke}%
\end{pgfscope}%
\begin{pgfscope}%
\pgfsetrectcap%
\pgfsetmiterjoin%
\pgfsetlinewidth{0.803000pt}%
\definecolor{currentstroke}{rgb}{0.000000,0.000000,0.000000}%
\pgfsetstrokecolor{currentstroke}%
\pgfsetdash{}{0pt}%
\pgfpathmoveto{\pgfqpoint{0.800000in}{4.224000in}}%
\pgfpathlineto{\pgfqpoint{5.760000in}{4.224000in}}%
\pgfusepath{stroke}%
\end{pgfscope}%
\begin{pgfscope}%
\pgftext[x=0.800000in,y=4.408800in,left,base]{\sffamily\fontsize{10.000000}{12.000000}\selectfont Iterations: 1990, Time: 0.140 ps, velocity: 300 Mm/s, imp: 377 ohm}%
\end{pgfscope}%
\end{pgfpicture}%
\makeatother%
\endgroup%
}}
%  \caption{Simulation result from task 1.}
%  \label{fig:task1}
%\end{figure}

%\begin{figure}
%  \centering
%    \begin{subfigure}[b]{0.5\textwidth}
%        \noindent\makebox[\textwidth]{\scalebox{0.5}{%% Creator: Matplotlib, PGF backend
%%
%% To include the figure in your LaTeX document, write
%%   \input{<filename>.pgf}
%%
%% Make sure the required packages are loaded in your preamble
%%   \usepackage{pgf}
%%
%% Figures using additional raster images can only be included by \input if
%% they are in the same directory as the main LaTeX file. For loading figures
%% from other directories you can use the `import` package
%%   \usepackage{import}
%% and then include the figures with
%%   \import{<path to file>}{<filename>.pgf}
%%
%% Matplotlib used the following preamble
%%   \usepackage{fontspec}
%%   \setmainfont{DejaVu Serif}
%%   \setsansfont{DejaVu Sans}
%%   \setmonofont{DejaVu Sans Mono}
%%
\begingroup%
\makeatletter%
\begin{pgfpicture}%
\pgfpathrectangle{\pgfpointorigin}{\pgfqpoint{13.660000in}{6.570000in}}%
\pgfusepath{use as bounding box, clip}%
\begin{pgfscope}%
\pgfsetbuttcap%
\pgfsetmiterjoin%
\definecolor{currentfill}{rgb}{1.000000,1.000000,1.000000}%
\pgfsetfillcolor{currentfill}%
\pgfsetlinewidth{0.000000pt}%
\definecolor{currentstroke}{rgb}{1.000000,1.000000,1.000000}%
\pgfsetstrokecolor{currentstroke}%
\pgfsetdash{}{0pt}%
\pgfpathmoveto{\pgfqpoint{0.000000in}{0.000000in}}%
\pgfpathlineto{\pgfqpoint{13.660000in}{0.000000in}}%
\pgfpathlineto{\pgfqpoint{13.660000in}{6.570000in}}%
\pgfpathlineto{\pgfqpoint{0.000000in}{6.570000in}}%
\pgfpathclose%
\pgfusepath{fill}%
\end{pgfscope}%
\begin{pgfscope}%
\pgfsetbuttcap%
\pgfsetmiterjoin%
\definecolor{currentfill}{rgb}{1.000000,1.000000,1.000000}%
\pgfsetfillcolor{currentfill}%
\pgfsetlinewidth{0.000000pt}%
\definecolor{currentstroke}{rgb}{0.000000,0.000000,0.000000}%
\pgfsetstrokecolor{currentstroke}%
\pgfsetstrokeopacity{0.000000}%
\pgfsetdash{}{0pt}%
\pgfpathmoveto{\pgfqpoint{1.707500in}{0.722700in}}%
\pgfpathlineto{\pgfqpoint{6.519545in}{0.722700in}}%
\pgfpathlineto{\pgfqpoint{6.519545in}{5.781600in}}%
\pgfpathlineto{\pgfqpoint{1.707500in}{5.781600in}}%
\pgfpathclose%
\pgfusepath{fill}%
\end{pgfscope}%
\begin{pgfscope}%
\pgfsetbuttcap%
\pgfsetroundjoin%
\definecolor{currentfill}{rgb}{0.000000,0.000000,0.000000}%
\pgfsetfillcolor{currentfill}%
\pgfsetlinewidth{0.803000pt}%
\definecolor{currentstroke}{rgb}{0.000000,0.000000,0.000000}%
\pgfsetstrokecolor{currentstroke}%
\pgfsetdash{}{0pt}%
\pgfsys@defobject{currentmarker}{\pgfqpoint{0.000000in}{-0.048611in}}{\pgfqpoint{0.000000in}{0.000000in}}{%
\pgfpathmoveto{\pgfqpoint{0.000000in}{0.000000in}}%
\pgfpathlineto{\pgfqpoint{0.000000in}{-0.048611in}}%
\pgfusepath{stroke,fill}%
}%
\begin{pgfscope}%
\pgfsys@transformshift{1.707500in}{0.722700in}%
\pgfsys@useobject{currentmarker}{}%
\end{pgfscope}%
\end{pgfscope}%
\begin{pgfscope}%
\pgftext[x=1.707500in,y=0.625478in,,top]{\sffamily\fontsize{10.000000}{12.000000}\selectfont 0}%
\end{pgfscope}%
\begin{pgfscope}%
\pgfsetbuttcap%
\pgfsetroundjoin%
\definecolor{currentfill}{rgb}{0.000000,0.000000,0.000000}%
\pgfsetfillcolor{currentfill}%
\pgfsetlinewidth{0.803000pt}%
\definecolor{currentstroke}{rgb}{0.000000,0.000000,0.000000}%
\pgfsetstrokecolor{currentstroke}%
\pgfsetdash{}{0pt}%
\pgfsys@defobject{currentmarker}{\pgfqpoint{0.000000in}{-0.048611in}}{\pgfqpoint{0.000000in}{0.000000in}}{%
\pgfpathmoveto{\pgfqpoint{0.000000in}{0.000000in}}%
\pgfpathlineto{\pgfqpoint{0.000000in}{-0.048611in}}%
\pgfusepath{stroke,fill}%
}%
\begin{pgfscope}%
\pgfsys@transformshift{2.277648in}{0.722700in}%
\pgfsys@useobject{currentmarker}{}%
\end{pgfscope}%
\end{pgfscope}%
\begin{pgfscope}%
\pgftext[x=2.277648in,y=0.625478in,,top]{\sffamily\fontsize{10.000000}{12.000000}\selectfont 5}%
\end{pgfscope}%
\begin{pgfscope}%
\pgfsetbuttcap%
\pgfsetroundjoin%
\definecolor{currentfill}{rgb}{0.000000,0.000000,0.000000}%
\pgfsetfillcolor{currentfill}%
\pgfsetlinewidth{0.803000pt}%
\definecolor{currentstroke}{rgb}{0.000000,0.000000,0.000000}%
\pgfsetstrokecolor{currentstroke}%
\pgfsetdash{}{0pt}%
\pgfsys@defobject{currentmarker}{\pgfqpoint{0.000000in}{-0.048611in}}{\pgfqpoint{0.000000in}{0.000000in}}{%
\pgfpathmoveto{\pgfqpoint{0.000000in}{0.000000in}}%
\pgfpathlineto{\pgfqpoint{0.000000in}{-0.048611in}}%
\pgfusepath{stroke,fill}%
}%
\begin{pgfscope}%
\pgfsys@transformshift{2.847795in}{0.722700in}%
\pgfsys@useobject{currentmarker}{}%
\end{pgfscope}%
\end{pgfscope}%
\begin{pgfscope}%
\pgftext[x=2.847795in,y=0.625478in,,top]{\sffamily\fontsize{10.000000}{12.000000}\selectfont 10}%
\end{pgfscope}%
\begin{pgfscope}%
\pgfsetbuttcap%
\pgfsetroundjoin%
\definecolor{currentfill}{rgb}{0.000000,0.000000,0.000000}%
\pgfsetfillcolor{currentfill}%
\pgfsetlinewidth{0.803000pt}%
\definecolor{currentstroke}{rgb}{0.000000,0.000000,0.000000}%
\pgfsetstrokecolor{currentstroke}%
\pgfsetdash{}{0pt}%
\pgfsys@defobject{currentmarker}{\pgfqpoint{0.000000in}{-0.048611in}}{\pgfqpoint{0.000000in}{0.000000in}}{%
\pgfpathmoveto{\pgfqpoint{0.000000in}{0.000000in}}%
\pgfpathlineto{\pgfqpoint{0.000000in}{-0.048611in}}%
\pgfusepath{stroke,fill}%
}%
\begin{pgfscope}%
\pgfsys@transformshift{3.417943in}{0.722700in}%
\pgfsys@useobject{currentmarker}{}%
\end{pgfscope}%
\end{pgfscope}%
\begin{pgfscope}%
\pgftext[x=3.417943in,y=0.625478in,,top]{\sffamily\fontsize{10.000000}{12.000000}\selectfont 15}%
\end{pgfscope}%
\begin{pgfscope}%
\pgfsetbuttcap%
\pgfsetroundjoin%
\definecolor{currentfill}{rgb}{0.000000,0.000000,0.000000}%
\pgfsetfillcolor{currentfill}%
\pgfsetlinewidth{0.803000pt}%
\definecolor{currentstroke}{rgb}{0.000000,0.000000,0.000000}%
\pgfsetstrokecolor{currentstroke}%
\pgfsetdash{}{0pt}%
\pgfsys@defobject{currentmarker}{\pgfqpoint{0.000000in}{-0.048611in}}{\pgfqpoint{0.000000in}{0.000000in}}{%
\pgfpathmoveto{\pgfqpoint{0.000000in}{0.000000in}}%
\pgfpathlineto{\pgfqpoint{0.000000in}{-0.048611in}}%
\pgfusepath{stroke,fill}%
}%
\begin{pgfscope}%
\pgfsys@transformshift{3.988090in}{0.722700in}%
\pgfsys@useobject{currentmarker}{}%
\end{pgfscope}%
\end{pgfscope}%
\begin{pgfscope}%
\pgftext[x=3.988090in,y=0.625478in,,top]{\sffamily\fontsize{10.000000}{12.000000}\selectfont 20}%
\end{pgfscope}%
\begin{pgfscope}%
\pgfsetbuttcap%
\pgfsetroundjoin%
\definecolor{currentfill}{rgb}{0.000000,0.000000,0.000000}%
\pgfsetfillcolor{currentfill}%
\pgfsetlinewidth{0.803000pt}%
\definecolor{currentstroke}{rgb}{0.000000,0.000000,0.000000}%
\pgfsetstrokecolor{currentstroke}%
\pgfsetdash{}{0pt}%
\pgfsys@defobject{currentmarker}{\pgfqpoint{0.000000in}{-0.048611in}}{\pgfqpoint{0.000000in}{0.000000in}}{%
\pgfpathmoveto{\pgfqpoint{0.000000in}{0.000000in}}%
\pgfpathlineto{\pgfqpoint{0.000000in}{-0.048611in}}%
\pgfusepath{stroke,fill}%
}%
\begin{pgfscope}%
\pgfsys@transformshift{4.558238in}{0.722700in}%
\pgfsys@useobject{currentmarker}{}%
\end{pgfscope}%
\end{pgfscope}%
\begin{pgfscope}%
\pgftext[x=4.558238in,y=0.625478in,,top]{\sffamily\fontsize{10.000000}{12.000000}\selectfont 25}%
\end{pgfscope}%
\begin{pgfscope}%
\pgfsetbuttcap%
\pgfsetroundjoin%
\definecolor{currentfill}{rgb}{0.000000,0.000000,0.000000}%
\pgfsetfillcolor{currentfill}%
\pgfsetlinewidth{0.803000pt}%
\definecolor{currentstroke}{rgb}{0.000000,0.000000,0.000000}%
\pgfsetstrokecolor{currentstroke}%
\pgfsetdash{}{0pt}%
\pgfsys@defobject{currentmarker}{\pgfqpoint{0.000000in}{-0.048611in}}{\pgfqpoint{0.000000in}{0.000000in}}{%
\pgfpathmoveto{\pgfqpoint{0.000000in}{0.000000in}}%
\pgfpathlineto{\pgfqpoint{0.000000in}{-0.048611in}}%
\pgfusepath{stroke,fill}%
}%
\begin{pgfscope}%
\pgfsys@transformshift{5.128385in}{0.722700in}%
\pgfsys@useobject{currentmarker}{}%
\end{pgfscope}%
\end{pgfscope}%
\begin{pgfscope}%
\pgftext[x=5.128385in,y=0.625478in,,top]{\sffamily\fontsize{10.000000}{12.000000}\selectfont 30}%
\end{pgfscope}%
\begin{pgfscope}%
\pgfsetbuttcap%
\pgfsetroundjoin%
\definecolor{currentfill}{rgb}{0.000000,0.000000,0.000000}%
\pgfsetfillcolor{currentfill}%
\pgfsetlinewidth{0.803000pt}%
\definecolor{currentstroke}{rgb}{0.000000,0.000000,0.000000}%
\pgfsetstrokecolor{currentstroke}%
\pgfsetdash{}{0pt}%
\pgfsys@defobject{currentmarker}{\pgfqpoint{0.000000in}{-0.048611in}}{\pgfqpoint{0.000000in}{0.000000in}}{%
\pgfpathmoveto{\pgfqpoint{0.000000in}{0.000000in}}%
\pgfpathlineto{\pgfqpoint{0.000000in}{-0.048611in}}%
\pgfusepath{stroke,fill}%
}%
\begin{pgfscope}%
\pgfsys@transformshift{5.698533in}{0.722700in}%
\pgfsys@useobject{currentmarker}{}%
\end{pgfscope}%
\end{pgfscope}%
\begin{pgfscope}%
\pgftext[x=5.698533in,y=0.625478in,,top]{\sffamily\fontsize{10.000000}{12.000000}\selectfont 35}%
\end{pgfscope}%
\begin{pgfscope}%
\pgfsetbuttcap%
\pgfsetroundjoin%
\definecolor{currentfill}{rgb}{0.000000,0.000000,0.000000}%
\pgfsetfillcolor{currentfill}%
\pgfsetlinewidth{0.803000pt}%
\definecolor{currentstroke}{rgb}{0.000000,0.000000,0.000000}%
\pgfsetstrokecolor{currentstroke}%
\pgfsetdash{}{0pt}%
\pgfsys@defobject{currentmarker}{\pgfqpoint{0.000000in}{-0.048611in}}{\pgfqpoint{0.000000in}{0.000000in}}{%
\pgfpathmoveto{\pgfqpoint{0.000000in}{0.000000in}}%
\pgfpathlineto{\pgfqpoint{0.000000in}{-0.048611in}}%
\pgfusepath{stroke,fill}%
}%
\begin{pgfscope}%
\pgfsys@transformshift{6.268681in}{0.722700in}%
\pgfsys@useobject{currentmarker}{}%
\end{pgfscope}%
\end{pgfscope}%
\begin{pgfscope}%
\pgftext[x=6.268681in,y=0.625478in,,top]{\sffamily\fontsize{10.000000}{12.000000}\selectfont 40}%
\end{pgfscope}%
\begin{pgfscope}%
\pgftext[x=4.113523in,y=0.435509in,,top]{\sffamily\fontsize{16.000000}{19.200000}\selectfont \(\displaystyle z-position [\mu m]\)}%
\end{pgfscope}%
\begin{pgfscope}%
\pgfsetbuttcap%
\pgfsetroundjoin%
\definecolor{currentfill}{rgb}{0.000000,0.000000,0.000000}%
\pgfsetfillcolor{currentfill}%
\pgfsetlinewidth{0.803000pt}%
\definecolor{currentstroke}{rgb}{0.000000,0.000000,0.000000}%
\pgfsetstrokecolor{currentstroke}%
\pgfsetdash{}{0pt}%
\pgfsys@defobject{currentmarker}{\pgfqpoint{-0.048611in}{0.000000in}}{\pgfqpoint{0.000000in}{0.000000in}}{%
\pgfpathmoveto{\pgfqpoint{0.000000in}{0.000000in}}%
\pgfpathlineto{\pgfqpoint{-0.048611in}{0.000000in}}%
\pgfusepath{stroke,fill}%
}%
\begin{pgfscope}%
\pgfsys@transformshift{1.707500in}{0.722700in}%
\pgfsys@useobject{currentmarker}{}%
\end{pgfscope}%
\end{pgfscope}%
\begin{pgfscope}%
\pgftext[x=1.273025in,y=0.669938in,left,base]{\sffamily\fontsize{10.000000}{12.000000}\selectfont -2.0}%
\end{pgfscope}%
\begin{pgfscope}%
\pgfsetbuttcap%
\pgfsetroundjoin%
\definecolor{currentfill}{rgb}{0.000000,0.000000,0.000000}%
\pgfsetfillcolor{currentfill}%
\pgfsetlinewidth{0.803000pt}%
\definecolor{currentstroke}{rgb}{0.000000,0.000000,0.000000}%
\pgfsetstrokecolor{currentstroke}%
\pgfsetdash{}{0pt}%
\pgfsys@defobject{currentmarker}{\pgfqpoint{-0.048611in}{0.000000in}}{\pgfqpoint{0.000000in}{0.000000in}}{%
\pgfpathmoveto{\pgfqpoint{0.000000in}{0.000000in}}%
\pgfpathlineto{\pgfqpoint{-0.048611in}{0.000000in}}%
\pgfusepath{stroke,fill}%
}%
\begin{pgfscope}%
\pgfsys@transformshift{1.707500in}{1.355062in}%
\pgfsys@useobject{currentmarker}{}%
\end{pgfscope}%
\end{pgfscope}%
\begin{pgfscope}%
\pgftext[x=1.273025in,y=1.302301in,left,base]{\sffamily\fontsize{10.000000}{12.000000}\selectfont -1.5}%
\end{pgfscope}%
\begin{pgfscope}%
\pgfsetbuttcap%
\pgfsetroundjoin%
\definecolor{currentfill}{rgb}{0.000000,0.000000,0.000000}%
\pgfsetfillcolor{currentfill}%
\pgfsetlinewidth{0.803000pt}%
\definecolor{currentstroke}{rgb}{0.000000,0.000000,0.000000}%
\pgfsetstrokecolor{currentstroke}%
\pgfsetdash{}{0pt}%
\pgfsys@defobject{currentmarker}{\pgfqpoint{-0.048611in}{0.000000in}}{\pgfqpoint{0.000000in}{0.000000in}}{%
\pgfpathmoveto{\pgfqpoint{0.000000in}{0.000000in}}%
\pgfpathlineto{\pgfqpoint{-0.048611in}{0.000000in}}%
\pgfusepath{stroke,fill}%
}%
\begin{pgfscope}%
\pgfsys@transformshift{1.707500in}{1.987425in}%
\pgfsys@useobject{currentmarker}{}%
\end{pgfscope}%
\end{pgfscope}%
\begin{pgfscope}%
\pgftext[x=1.273025in,y=1.934663in,left,base]{\sffamily\fontsize{10.000000}{12.000000}\selectfont -1.0}%
\end{pgfscope}%
\begin{pgfscope}%
\pgfsetbuttcap%
\pgfsetroundjoin%
\definecolor{currentfill}{rgb}{0.000000,0.000000,0.000000}%
\pgfsetfillcolor{currentfill}%
\pgfsetlinewidth{0.803000pt}%
\definecolor{currentstroke}{rgb}{0.000000,0.000000,0.000000}%
\pgfsetstrokecolor{currentstroke}%
\pgfsetdash{}{0pt}%
\pgfsys@defobject{currentmarker}{\pgfqpoint{-0.048611in}{0.000000in}}{\pgfqpoint{0.000000in}{0.000000in}}{%
\pgfpathmoveto{\pgfqpoint{0.000000in}{0.000000in}}%
\pgfpathlineto{\pgfqpoint{-0.048611in}{0.000000in}}%
\pgfusepath{stroke,fill}%
}%
\begin{pgfscope}%
\pgfsys@transformshift{1.707500in}{2.619788in}%
\pgfsys@useobject{currentmarker}{}%
\end{pgfscope}%
\end{pgfscope}%
\begin{pgfscope}%
\pgftext[x=1.273025in,y=2.567026in,left,base]{\sffamily\fontsize{10.000000}{12.000000}\selectfont -0.5}%
\end{pgfscope}%
\begin{pgfscope}%
\pgfsetbuttcap%
\pgfsetroundjoin%
\definecolor{currentfill}{rgb}{0.000000,0.000000,0.000000}%
\pgfsetfillcolor{currentfill}%
\pgfsetlinewidth{0.803000pt}%
\definecolor{currentstroke}{rgb}{0.000000,0.000000,0.000000}%
\pgfsetstrokecolor{currentstroke}%
\pgfsetdash{}{0pt}%
\pgfsys@defobject{currentmarker}{\pgfqpoint{-0.048611in}{0.000000in}}{\pgfqpoint{0.000000in}{0.000000in}}{%
\pgfpathmoveto{\pgfqpoint{0.000000in}{0.000000in}}%
\pgfpathlineto{\pgfqpoint{-0.048611in}{0.000000in}}%
\pgfusepath{stroke,fill}%
}%
\begin{pgfscope}%
\pgfsys@transformshift{1.707500in}{3.252150in}%
\pgfsys@useobject{currentmarker}{}%
\end{pgfscope}%
\end{pgfscope}%
\begin{pgfscope}%
\pgftext[x=1.389398in,y=3.199388in,left,base]{\sffamily\fontsize{10.000000}{12.000000}\selectfont 0.0}%
\end{pgfscope}%
\begin{pgfscope}%
\pgfsetbuttcap%
\pgfsetroundjoin%
\definecolor{currentfill}{rgb}{0.000000,0.000000,0.000000}%
\pgfsetfillcolor{currentfill}%
\pgfsetlinewidth{0.803000pt}%
\definecolor{currentstroke}{rgb}{0.000000,0.000000,0.000000}%
\pgfsetstrokecolor{currentstroke}%
\pgfsetdash{}{0pt}%
\pgfsys@defobject{currentmarker}{\pgfqpoint{-0.048611in}{0.000000in}}{\pgfqpoint{0.000000in}{0.000000in}}{%
\pgfpathmoveto{\pgfqpoint{0.000000in}{0.000000in}}%
\pgfpathlineto{\pgfqpoint{-0.048611in}{0.000000in}}%
\pgfusepath{stroke,fill}%
}%
\begin{pgfscope}%
\pgfsys@transformshift{1.707500in}{3.884512in}%
\pgfsys@useobject{currentmarker}{}%
\end{pgfscope}%
\end{pgfscope}%
\begin{pgfscope}%
\pgftext[x=1.389398in,y=3.831751in,left,base]{\sffamily\fontsize{10.000000}{12.000000}\selectfont 0.5}%
\end{pgfscope}%
\begin{pgfscope}%
\pgfsetbuttcap%
\pgfsetroundjoin%
\definecolor{currentfill}{rgb}{0.000000,0.000000,0.000000}%
\pgfsetfillcolor{currentfill}%
\pgfsetlinewidth{0.803000pt}%
\definecolor{currentstroke}{rgb}{0.000000,0.000000,0.000000}%
\pgfsetstrokecolor{currentstroke}%
\pgfsetdash{}{0pt}%
\pgfsys@defobject{currentmarker}{\pgfqpoint{-0.048611in}{0.000000in}}{\pgfqpoint{0.000000in}{0.000000in}}{%
\pgfpathmoveto{\pgfqpoint{0.000000in}{0.000000in}}%
\pgfpathlineto{\pgfqpoint{-0.048611in}{0.000000in}}%
\pgfusepath{stroke,fill}%
}%
\begin{pgfscope}%
\pgfsys@transformshift{1.707500in}{4.516875in}%
\pgfsys@useobject{currentmarker}{}%
\end{pgfscope}%
\end{pgfscope}%
\begin{pgfscope}%
\pgftext[x=1.389398in,y=4.464113in,left,base]{\sffamily\fontsize{10.000000}{12.000000}\selectfont 1.0}%
\end{pgfscope}%
\begin{pgfscope}%
\pgfsetbuttcap%
\pgfsetroundjoin%
\definecolor{currentfill}{rgb}{0.000000,0.000000,0.000000}%
\pgfsetfillcolor{currentfill}%
\pgfsetlinewidth{0.803000pt}%
\definecolor{currentstroke}{rgb}{0.000000,0.000000,0.000000}%
\pgfsetstrokecolor{currentstroke}%
\pgfsetdash{}{0pt}%
\pgfsys@defobject{currentmarker}{\pgfqpoint{-0.048611in}{0.000000in}}{\pgfqpoint{0.000000in}{0.000000in}}{%
\pgfpathmoveto{\pgfqpoint{0.000000in}{0.000000in}}%
\pgfpathlineto{\pgfqpoint{-0.048611in}{0.000000in}}%
\pgfusepath{stroke,fill}%
}%
\begin{pgfscope}%
\pgfsys@transformshift{1.707500in}{5.149237in}%
\pgfsys@useobject{currentmarker}{}%
\end{pgfscope}%
\end{pgfscope}%
\begin{pgfscope}%
\pgftext[x=1.389398in,y=5.096476in,left,base]{\sffamily\fontsize{10.000000}{12.000000}\selectfont 1.5}%
\end{pgfscope}%
\begin{pgfscope}%
\pgfsetbuttcap%
\pgfsetroundjoin%
\definecolor{currentfill}{rgb}{0.000000,0.000000,0.000000}%
\pgfsetfillcolor{currentfill}%
\pgfsetlinewidth{0.803000pt}%
\definecolor{currentstroke}{rgb}{0.000000,0.000000,0.000000}%
\pgfsetstrokecolor{currentstroke}%
\pgfsetdash{}{0pt}%
\pgfsys@defobject{currentmarker}{\pgfqpoint{-0.048611in}{0.000000in}}{\pgfqpoint{0.000000in}{0.000000in}}{%
\pgfpathmoveto{\pgfqpoint{0.000000in}{0.000000in}}%
\pgfpathlineto{\pgfqpoint{-0.048611in}{0.000000in}}%
\pgfusepath{stroke,fill}%
}%
\begin{pgfscope}%
\pgfsys@transformshift{1.707500in}{5.781600in}%
\pgfsys@useobject{currentmarker}{}%
\end{pgfscope}%
\end{pgfscope}%
\begin{pgfscope}%
\pgftext[x=1.389398in,y=5.728838in,left,base]{\sffamily\fontsize{10.000000}{12.000000}\selectfont 2.0}%
\end{pgfscope}%
\begin{pgfscope}%
\pgftext[x=1.217469in,y=3.252150in,,bottom,rotate=90.000000]{\sffamily\fontsize{16.000000}{19.200000}\selectfont \(\displaystyle E-field\)}%
\end{pgfscope}%
\begin{pgfscope}%
\pgfpathrectangle{\pgfqpoint{1.707500in}{0.722700in}}{\pgfqpoint{4.812045in}{5.058900in}} %
\pgfusepath{clip}%
\pgfsetrectcap%
\pgfsetroundjoin%
\pgfsetlinewidth{1.505625pt}%
\definecolor{currentstroke}{rgb}{0.000000,0.000000,0.000000}%
\pgfsetstrokecolor{currentstroke}%
\pgfsetdash{}{0pt}%
\pgfpathmoveto{\pgfqpoint{1.707500in}{3.252150in}}%
\pgfpathlineto{\pgfqpoint{4.779259in}{3.253188in}}%
\pgfpathlineto{\pgfqpoint{4.784698in}{3.256978in}}%
\pgfpathlineto{\pgfqpoint{4.790975in}{3.264373in}}%
\pgfpathlineto{\pgfqpoint{4.800181in}{3.275379in}}%
\pgfpathlineto{\pgfqpoint{4.803110in}{3.275943in}}%
\pgfpathlineto{\pgfqpoint{4.805620in}{3.274018in}}%
\pgfpathlineto{\pgfqpoint{4.808549in}{3.268095in}}%
\pgfpathlineto{\pgfqpoint{4.811897in}{3.255523in}}%
\pgfpathlineto{\pgfqpoint{4.816081in}{3.230361in}}%
\pgfpathlineto{\pgfqpoint{4.822358in}{3.176192in}}%
\pgfpathlineto{\pgfqpoint{4.831564in}{3.098298in}}%
\pgfpathlineto{\pgfqpoint{4.834493in}{3.087676in}}%
\pgfpathlineto{\pgfqpoint{4.836166in}{3.087167in}}%
\pgfpathlineto{\pgfqpoint{4.837840in}{3.091417in}}%
\pgfpathlineto{\pgfqpoint{4.839932in}{3.104118in}}%
\pgfpathlineto{\pgfqpoint{4.842861in}{3.136718in}}%
\pgfpathlineto{\pgfqpoint{4.847046in}{3.213578in}}%
\pgfpathlineto{\pgfqpoint{4.852904in}{3.370300in}}%
\pgfpathlineto{\pgfqpoint{4.864202in}{3.681173in}}%
\pgfpathlineto{\pgfqpoint{4.867549in}{3.724800in}}%
\pgfpathlineto{\pgfqpoint{4.869642in}{3.731705in}}%
\pgfpathlineto{\pgfqpoint{4.870897in}{3.727394in}}%
\pgfpathlineto{\pgfqpoint{4.872989in}{3.705302in}}%
\pgfpathlineto{\pgfqpoint{4.875918in}{3.642321in}}%
\pgfpathlineto{\pgfqpoint{4.880102in}{3.490268in}}%
\pgfpathlineto{\pgfqpoint{4.886379in}{3.158217in}}%
\pgfpathlineto{\pgfqpoint{4.897677in}{2.551742in}}%
\pgfpathlineto{\pgfqpoint{4.901443in}{2.447435in}}%
\pgfpathlineto{\pgfqpoint{4.903953in}{2.424022in}}%
\pgfpathlineto{\pgfqpoint{4.904372in}{2.424043in}}%
\pgfpathlineto{\pgfqpoint{4.905209in}{2.427544in}}%
\pgfpathlineto{\pgfqpoint{4.906883in}{2.448543in}}%
\pgfpathlineto{\pgfqpoint{4.909393in}{2.515003in}}%
\pgfpathlineto{\pgfqpoint{4.913159in}{2.689248in}}%
\pgfpathlineto{\pgfqpoint{4.918599in}{3.067850in}}%
\pgfpathlineto{\pgfqpoint{4.933663in}{4.188024in}}%
\pgfpathlineto{\pgfqpoint{4.937429in}{4.305530in}}%
\pgfpathlineto{\pgfqpoint{4.939521in}{4.323034in}}%
\pgfpathlineto{\pgfqpoint{4.940358in}{4.319974in}}%
\pgfpathlineto{\pgfqpoint{4.942031in}{4.296443in}}%
\pgfpathlineto{\pgfqpoint{4.944542in}{4.218242in}}%
\pgfpathlineto{\pgfqpoint{4.948308in}{4.011154in}}%
\pgfpathlineto{\pgfqpoint{4.953748in}{3.562804in}}%
\pgfpathlineto{\pgfqpoint{4.968812in}{2.238989in}}%
\pgfpathlineto{\pgfqpoint{4.972577in}{2.089653in}}%
\pgfpathlineto{\pgfqpoint{4.975088in}{2.058094in}}%
\pgfpathlineto{\pgfqpoint{4.975507in}{2.058378in}}%
\pgfpathlineto{\pgfqpoint{4.976762in}{2.068806in}}%
\pgfpathlineto{\pgfqpoint{4.978854in}{2.117824in}}%
\pgfpathlineto{\pgfqpoint{4.981783in}{2.250105in}}%
\pgfpathlineto{\pgfqpoint{4.985967in}{2.551684in}}%
\pgfpathlineto{\pgfqpoint{4.993081in}{3.257728in}}%
\pgfpathlineto{\pgfqpoint{5.002705in}{4.174641in}}%
\pgfpathlineto{\pgfqpoint{5.007308in}{4.426660in}}%
\pgfpathlineto{\pgfqpoint{5.010237in}{4.491859in}}%
\pgfpathlineto{\pgfqpoint{5.011074in}{4.495810in}}%
\pgfpathlineto{\pgfqpoint{5.011492in}{4.495306in}}%
\pgfpathlineto{\pgfqpoint{5.012748in}{4.483878in}}%
\pgfpathlineto{\pgfqpoint{5.014840in}{4.432174in}}%
\pgfpathlineto{\pgfqpoint{5.017769in}{4.294403in}}%
\pgfpathlineto{\pgfqpoint{5.021953in}{3.982701in}}%
\pgfpathlineto{\pgfqpoint{5.029485in}{3.212180in}}%
\pgfpathlineto{\pgfqpoint{5.038691in}{2.321402in}}%
\pgfpathlineto{\pgfqpoint{5.043294in}{2.064046in}}%
\pgfpathlineto{\pgfqpoint{5.046223in}{1.996793in}}%
\pgfpathlineto{\pgfqpoint{5.047060in}{1.992431in}}%
\pgfpathlineto{\pgfqpoint{5.047478in}{1.992758in}}%
\pgfpathlineto{\pgfqpoint{5.048733in}{2.003760in}}%
\pgfpathlineto{\pgfqpoint{5.050825in}{2.055092in}}%
\pgfpathlineto{\pgfqpoint{5.053755in}{2.192974in}}%
\pgfpathlineto{\pgfqpoint{5.057939in}{2.505896in}}%
\pgfpathlineto{\pgfqpoint{5.065052in}{3.234905in}}%
\pgfpathlineto{\pgfqpoint{5.074677in}{4.179412in}}%
\pgfpathlineto{\pgfqpoint{5.079279in}{4.440942in}}%
\pgfpathlineto{\pgfqpoint{5.082208in}{4.510761in}}%
\pgfpathlineto{\pgfqpoint{5.083464in}{4.515846in}}%
\pgfpathlineto{\pgfqpoint{5.084301in}{4.510857in}}%
\pgfpathlineto{\pgfqpoint{5.085974in}{4.480902in}}%
\pgfpathlineto{\pgfqpoint{5.088485in}{4.387481in}}%
\pgfpathlineto{\pgfqpoint{5.092251in}{4.148128in}}%
\pgfpathlineto{\pgfqpoint{5.098109in}{3.598815in}}%
\pgfpathlineto{\pgfqpoint{5.111918in}{2.242786in}}%
\pgfpathlineto{\pgfqpoint{5.116102in}{2.037583in}}%
\pgfpathlineto{\pgfqpoint{5.118613in}{1.990081in}}%
\pgfpathlineto{\pgfqpoint{5.119449in}{1.987560in}}%
\pgfpathlineto{\pgfqpoint{5.119868in}{1.988815in}}%
\pgfpathlineto{\pgfqpoint{5.121123in}{2.002621in}}%
\pgfpathlineto{\pgfqpoint{5.123215in}{2.058592in}}%
\pgfpathlineto{\pgfqpoint{5.126563in}{2.229005in}}%
\pgfpathlineto{\pgfqpoint{5.131166in}{2.599963in}}%
\pgfpathlineto{\pgfqpoint{5.140790in}{3.620998in}}%
\pgfpathlineto{\pgfqpoint{5.147903in}{4.254641in}}%
\pgfpathlineto{\pgfqpoint{5.152088in}{4.463555in}}%
\pgfpathlineto{\pgfqpoint{5.154598in}{4.513517in}}%
\pgfpathlineto{\pgfqpoint{5.155435in}{4.516872in}}%
\pgfpathlineto{\pgfqpoint{5.155854in}{4.516033in}}%
\pgfpathlineto{\pgfqpoint{5.157109in}{4.503473in}}%
\pgfpathlineto{\pgfqpoint{5.159201in}{4.449523in}}%
\pgfpathlineto{\pgfqpoint{5.162130in}{4.308107in}}%
\pgfpathlineto{\pgfqpoint{5.166733in}{3.952997in}}%
\pgfpathlineto{\pgfqpoint{5.175102in}{3.074234in}}%
\pgfpathlineto{\pgfqpoint{5.183052in}{2.316139in}}%
\pgfpathlineto{\pgfqpoint{5.187655in}{2.058533in}}%
\pgfpathlineto{\pgfqpoint{5.190584in}{1.991670in}}%
\pgfpathlineto{\pgfqpoint{5.191421in}{1.987477in}}%
\pgfpathlineto{\pgfqpoint{5.191839in}{1.987897in}}%
\pgfpathlineto{\pgfqpoint{5.193095in}{1.999205in}}%
\pgfpathlineto{\pgfqpoint{5.195187in}{2.051117in}}%
\pgfpathlineto{\pgfqpoint{5.198116in}{2.189897in}}%
\pgfpathlineto{\pgfqpoint{5.202300in}{2.504112in}}%
\pgfpathlineto{\pgfqpoint{5.209832in}{3.280939in}}%
\pgfpathlineto{\pgfqpoint{5.219038in}{4.180377in}}%
\pgfpathlineto{\pgfqpoint{5.223641in}{4.441910in}}%
\pgfpathlineto{\pgfqpoint{5.226570in}{4.511635in}}%
\pgfpathlineto{\pgfqpoint{5.227825in}{4.516665in}}%
\pgfpathlineto{\pgfqpoint{5.228662in}{4.511635in}}%
\pgfpathlineto{\pgfqpoint{5.230336in}{4.481594in}}%
\pgfpathlineto{\pgfqpoint{5.232846in}{4.388035in}}%
\pgfpathlineto{\pgfqpoint{5.236612in}{4.148479in}}%
\pgfpathlineto{\pgfqpoint{5.242470in}{3.598910in}}%
\pgfpathlineto{\pgfqpoint{5.256279in}{2.242674in}}%
\pgfpathlineto{\pgfqpoint{5.260463in}{2.037482in}}%
\pgfpathlineto{\pgfqpoint{5.262974in}{1.989993in}}%
\pgfpathlineto{\pgfqpoint{5.263811in}{1.987477in}}%
\pgfpathlineto{\pgfqpoint{5.264229in}{1.988736in}}%
\pgfpathlineto{\pgfqpoint{5.265485in}{2.002549in}}%
\pgfpathlineto{\pgfqpoint{5.267577in}{2.058533in}}%
\pgfpathlineto{\pgfqpoint{5.270924in}{2.228966in}}%
\pgfpathlineto{\pgfqpoint{5.275527in}{2.599946in}}%
\pgfpathlineto{\pgfqpoint{5.285151in}{3.621002in}}%
\pgfpathlineto{\pgfqpoint{5.292265in}{4.254646in}}%
\pgfpathlineto{\pgfqpoint{5.296449in}{4.463558in}}%
\pgfpathlineto{\pgfqpoint{5.298960in}{4.513518in}}%
\pgfpathlineto{\pgfqpoint{5.299797in}{4.516872in}}%
\pgfpathlineto{\pgfqpoint{5.300215in}{4.516033in}}%
\pgfpathlineto{\pgfqpoint{5.301470in}{4.503471in}}%
\pgfpathlineto{\pgfqpoint{5.303562in}{4.449520in}}%
\pgfpathlineto{\pgfqpoint{5.306492in}{4.308103in}}%
\pgfpathlineto{\pgfqpoint{5.311094in}{3.952991in}}%
\pgfpathlineto{\pgfqpoint{5.319463in}{3.074237in}}%
\pgfpathlineto{\pgfqpoint{5.327413in}{2.316171in}}%
\pgfpathlineto{\pgfqpoint{5.332016in}{2.058592in}}%
\pgfpathlineto{\pgfqpoint{5.334945in}{1.991747in}}%
\pgfpathlineto{\pgfqpoint{5.335782in}{1.987560in}}%
\pgfpathlineto{\pgfqpoint{5.336201in}{1.987982in}}%
\pgfpathlineto{\pgfqpoint{5.337456in}{1.999297in}}%
\pgfpathlineto{\pgfqpoint{5.339548in}{2.051220in}}%
\pgfpathlineto{\pgfqpoint{5.342477in}{2.190009in}}%
\pgfpathlineto{\pgfqpoint{5.346662in}{2.504217in}}%
\pgfpathlineto{\pgfqpoint{5.354194in}{3.280933in}}%
\pgfpathlineto{\pgfqpoint{5.363399in}{4.180005in}}%
\pgfpathlineto{\pgfqpoint{5.368002in}{4.441287in}}%
\pgfpathlineto{\pgfqpoint{5.370931in}{4.510857in}}%
\pgfpathlineto{\pgfqpoint{5.372186in}{4.515827in}}%
\pgfpathlineto{\pgfqpoint{5.373023in}{4.510761in}}%
\pgfpathlineto{\pgfqpoint{5.374697in}{4.480659in}}%
\pgfpathlineto{\pgfqpoint{5.377208in}{4.387045in}}%
\pgfpathlineto{\pgfqpoint{5.380974in}{4.147527in}}%
\pgfpathlineto{\pgfqpoint{5.386832in}{3.598413in}}%
\pgfpathlineto{\pgfqpoint{5.400640in}{2.245487in}}%
\pgfpathlineto{\pgfqpoint{5.404825in}{2.041579in}}%
\pgfpathlineto{\pgfqpoint{5.407335in}{1.994757in}}%
\pgfpathlineto{\pgfqpoint{5.408172in}{1.992431in}}%
\pgfpathlineto{\pgfqpoint{5.408591in}{1.993776in}}%
\pgfpathlineto{\pgfqpoint{5.409846in}{2.007817in}}%
\pgfpathlineto{\pgfqpoint{5.411938in}{2.064046in}}%
\pgfpathlineto{\pgfqpoint{5.415286in}{2.234421in}}%
\pgfpathlineto{\pgfqpoint{5.419888in}{2.604165in}}%
\pgfpathlineto{\pgfqpoint{5.429931in}{3.660807in}}%
\pgfpathlineto{\pgfqpoint{5.436626in}{4.242042in}}%
\pgfpathlineto{\pgfqpoint{5.440810in}{4.445729in}}%
\pgfpathlineto{\pgfqpoint{5.443321in}{4.493147in}}%
\pgfpathlineto{\pgfqpoint{5.444158in}{4.495810in}}%
\pgfpathlineto{\pgfqpoint{5.444576in}{4.494660in}}%
\pgfpathlineto{\pgfqpoint{5.445832in}{4.481321in}}%
\pgfpathlineto{\pgfqpoint{5.447924in}{4.426660in}}%
\pgfpathlineto{\pgfqpoint{5.451271in}{4.259900in}}%
\pgfpathlineto{\pgfqpoint{5.455874in}{3.897459in}}%
\pgfpathlineto{\pgfqpoint{5.466335in}{2.823004in}}%
\pgfpathlineto{\pgfqpoint{5.473030in}{2.274689in}}%
\pgfpathlineto{\pgfqpoint{5.477215in}{2.093522in}}%
\pgfpathlineto{\pgfqpoint{5.479725in}{2.058378in}}%
\pgfpathlineto{\pgfqpoint{5.480144in}{2.058094in}}%
\pgfpathlineto{\pgfqpoint{5.480981in}{2.062305in}}%
\pgfpathlineto{\pgfqpoint{5.482654in}{2.089653in}}%
\pgfpathlineto{\pgfqpoint{5.485165in}{2.176345in}}%
\pgfpathlineto{\pgfqpoint{5.488931in}{2.398748in}}%
\pgfpathlineto{\pgfqpoint{5.494789in}{2.905061in}}%
\pgfpathlineto{\pgfqpoint{5.507761in}{4.065752in}}%
\pgfpathlineto{\pgfqpoint{5.511945in}{4.263676in}}%
\pgfpathlineto{\pgfqpoint{5.514874in}{4.319974in}}%
\pgfpathlineto{\pgfqpoint{5.515711in}{4.323034in}}%
\pgfpathlineto{\pgfqpoint{5.516129in}{4.322394in}}%
\pgfpathlineto{\pgfqpoint{5.517385in}{4.311871in}}%
\pgfpathlineto{\pgfqpoint{5.519477in}{4.266389in}}%
\pgfpathlineto{\pgfqpoint{5.522824in}{4.126578in}}%
\pgfpathlineto{\pgfqpoint{5.527846in}{3.794022in}}%
\pgfpathlineto{\pgfqpoint{5.545002in}{2.546271in}}%
\pgfpathlineto{\pgfqpoint{5.548768in}{2.441537in}}%
\pgfpathlineto{\pgfqpoint{5.550860in}{2.424043in}}%
\pgfpathlineto{\pgfqpoint{5.551278in}{2.424022in}}%
\pgfpathlineto{\pgfqpoint{5.552115in}{2.427396in}}%
\pgfpathlineto{\pgfqpoint{5.553789in}{2.447435in}}%
\pgfpathlineto{\pgfqpoint{5.556299in}{2.508562in}}%
\pgfpathlineto{\pgfqpoint{5.560065in}{2.659406in}}%
\pgfpathlineto{\pgfqpoint{5.567179in}{3.058522in}}%
\pgfpathlineto{\pgfqpoint{5.575966in}{3.526078in}}%
\pgfpathlineto{\pgfqpoint{5.580569in}{3.673888in}}%
\pgfpathlineto{\pgfqpoint{5.583916in}{3.724485in}}%
\pgfpathlineto{\pgfqpoint{5.586009in}{3.731702in}}%
\pgfpathlineto{\pgfqpoint{5.587264in}{3.727531in}}%
\pgfpathlineto{\pgfqpoint{5.589356in}{3.707585in}}%
\pgfpathlineto{\pgfqpoint{5.592704in}{3.646858in}}%
\pgfpathlineto{\pgfqpoint{5.598143in}{3.498436in}}%
\pgfpathlineto{\pgfqpoint{5.609441in}{3.186921in}}%
\pgfpathlineto{\pgfqpoint{5.614044in}{3.115926in}}%
\pgfpathlineto{\pgfqpoint{5.617392in}{3.091417in}}%
\pgfpathlineto{\pgfqpoint{5.619484in}{3.086868in}}%
\pgfpathlineto{\pgfqpoint{5.620739in}{3.087676in}}%
\pgfpathlineto{\pgfqpoint{5.622413in}{3.092385in}}%
\pgfpathlineto{\pgfqpoint{5.625342in}{3.108820in}}%
\pgfpathlineto{\pgfqpoint{5.630782in}{3.156030in}}%
\pgfpathlineto{\pgfqpoint{5.640406in}{3.238998in}}%
\pgfpathlineto{\pgfqpoint{5.645427in}{3.264149in}}%
\pgfpathlineto{\pgfqpoint{5.649193in}{3.273433in}}%
\pgfpathlineto{\pgfqpoint{5.652122in}{3.275943in}}%
\pgfpathlineto{\pgfqpoint{5.654633in}{3.275609in}}%
\pgfpathlineto{\pgfqpoint{5.658399in}{3.272276in}}%
\pgfpathlineto{\pgfqpoint{5.675136in}{3.253610in}}%
\pgfpathlineto{\pgfqpoint{5.681413in}{3.251587in}}%
\pgfpathlineto{\pgfqpoint{5.690618in}{3.251443in}}%
\pgfpathlineto{\pgfqpoint{5.724093in}{3.252152in}}%
\pgfpathlineto{\pgfqpoint{6.519545in}{3.252150in}}%
\pgfpathlineto{\pgfqpoint{6.519545in}{3.252150in}}%
\pgfusepath{stroke}%
\end{pgfscope}%
\begin{pgfscope}%
\pgfsetrectcap%
\pgfsetmiterjoin%
\pgfsetlinewidth{0.803000pt}%
\definecolor{currentstroke}{rgb}{0.000000,0.000000,0.000000}%
\pgfsetstrokecolor{currentstroke}%
\pgfsetdash{}{0pt}%
\pgfpathmoveto{\pgfqpoint{1.707500in}{0.722700in}}%
\pgfpathlineto{\pgfqpoint{1.707500in}{5.781600in}}%
\pgfusepath{stroke}%
\end{pgfscope}%
\begin{pgfscope}%
\pgfsetrectcap%
\pgfsetmiterjoin%
\pgfsetlinewidth{0.803000pt}%
\definecolor{currentstroke}{rgb}{0.000000,0.000000,0.000000}%
\pgfsetstrokecolor{currentstroke}%
\pgfsetdash{}{0pt}%
\pgfpathmoveto{\pgfqpoint{6.519545in}{0.722700in}}%
\pgfpathlineto{\pgfqpoint{6.519545in}{5.781600in}}%
\pgfusepath{stroke}%
\end{pgfscope}%
\begin{pgfscope}%
\pgfsetrectcap%
\pgfsetmiterjoin%
\pgfsetlinewidth{0.803000pt}%
\definecolor{currentstroke}{rgb}{0.000000,0.000000,0.000000}%
\pgfsetstrokecolor{currentstroke}%
\pgfsetdash{}{0pt}%
\pgfpathmoveto{\pgfqpoint{1.707500in}{0.722700in}}%
\pgfpathlineto{\pgfqpoint{6.519545in}{0.722700in}}%
\pgfusepath{stroke}%
\end{pgfscope}%
\begin{pgfscope}%
\pgfsetrectcap%
\pgfsetmiterjoin%
\pgfsetlinewidth{0.803000pt}%
\definecolor{currentstroke}{rgb}{0.000000,0.000000,0.000000}%
\pgfsetstrokecolor{currentstroke}%
\pgfsetdash{}{0pt}%
\pgfpathmoveto{\pgfqpoint{1.707500in}{5.781600in}}%
\pgfpathlineto{\pgfqpoint{6.519545in}{5.781600in}}%
\pgfusepath{stroke}%
\end{pgfscope}%
\begin{pgfscope}%
\pgftext[x=1.707500in,y=6.034545in,left,base]{\sffamily\fontsize{10.000000}{12.000000}\selectfont Iterations: 10945, Time: 0.134 ps, imp: 377 ohm}%
\end{pgfscope}%
\begin{pgfscope}%
\pgfsetbuttcap%
\pgfsetmiterjoin%
\definecolor{currentfill}{rgb}{1.000000,1.000000,1.000000}%
\pgfsetfillcolor{currentfill}%
\pgfsetlinewidth{0.000000pt}%
\definecolor{currentstroke}{rgb}{0.000000,0.000000,0.000000}%
\pgfsetstrokecolor{currentstroke}%
\pgfsetstrokeopacity{0.000000}%
\pgfsetdash{}{0pt}%
\pgfpathmoveto{\pgfqpoint{7.481955in}{0.722700in}}%
\pgfpathlineto{\pgfqpoint{12.294000in}{0.722700in}}%
\pgfpathlineto{\pgfqpoint{12.294000in}{5.781600in}}%
\pgfpathlineto{\pgfqpoint{7.481955in}{5.781600in}}%
\pgfpathclose%
\pgfusepath{fill}%
\end{pgfscope}%
\begin{pgfscope}%
\pgfsetbuttcap%
\pgfsetroundjoin%
\definecolor{currentfill}{rgb}{0.000000,0.000000,0.000000}%
\pgfsetfillcolor{currentfill}%
\pgfsetlinewidth{0.803000pt}%
\definecolor{currentstroke}{rgb}{0.000000,0.000000,0.000000}%
\pgfsetstrokecolor{currentstroke}%
\pgfsetdash{}{0pt}%
\pgfsys@defobject{currentmarker}{\pgfqpoint{0.000000in}{-0.048611in}}{\pgfqpoint{0.000000in}{0.000000in}}{%
\pgfpathmoveto{\pgfqpoint{0.000000in}{0.000000in}}%
\pgfpathlineto{\pgfqpoint{0.000000in}{-0.048611in}}%
\pgfusepath{stroke,fill}%
}%
\begin{pgfscope}%
\pgfsys@transformshift{7.481955in}{0.722700in}%
\pgfsys@useobject{currentmarker}{}%
\end{pgfscope}%
\end{pgfscope}%
\begin{pgfscope}%
\pgftext[x=7.481955in,y=0.625478in,,top]{\sffamily\fontsize{10.000000}{12.000000}\selectfont 0}%
\end{pgfscope}%
\begin{pgfscope}%
\pgfsetbuttcap%
\pgfsetroundjoin%
\definecolor{currentfill}{rgb}{0.000000,0.000000,0.000000}%
\pgfsetfillcolor{currentfill}%
\pgfsetlinewidth{0.803000pt}%
\definecolor{currentstroke}{rgb}{0.000000,0.000000,0.000000}%
\pgfsetstrokecolor{currentstroke}%
\pgfsetdash{}{0pt}%
\pgfsys@defobject{currentmarker}{\pgfqpoint{0.000000in}{-0.048611in}}{\pgfqpoint{0.000000in}{0.000000in}}{%
\pgfpathmoveto{\pgfqpoint{0.000000in}{0.000000in}}%
\pgfpathlineto{\pgfqpoint{0.000000in}{-0.048611in}}%
\pgfusepath{stroke,fill}%
}%
\begin{pgfscope}%
\pgfsys@transformshift{8.052102in}{0.722700in}%
\pgfsys@useobject{currentmarker}{}%
\end{pgfscope}%
\end{pgfscope}%
\begin{pgfscope}%
\pgftext[x=8.052102in,y=0.625478in,,top]{\sffamily\fontsize{10.000000}{12.000000}\selectfont 5}%
\end{pgfscope}%
\begin{pgfscope}%
\pgfsetbuttcap%
\pgfsetroundjoin%
\definecolor{currentfill}{rgb}{0.000000,0.000000,0.000000}%
\pgfsetfillcolor{currentfill}%
\pgfsetlinewidth{0.803000pt}%
\definecolor{currentstroke}{rgb}{0.000000,0.000000,0.000000}%
\pgfsetstrokecolor{currentstroke}%
\pgfsetdash{}{0pt}%
\pgfsys@defobject{currentmarker}{\pgfqpoint{0.000000in}{-0.048611in}}{\pgfqpoint{0.000000in}{0.000000in}}{%
\pgfpathmoveto{\pgfqpoint{0.000000in}{0.000000in}}%
\pgfpathlineto{\pgfqpoint{0.000000in}{-0.048611in}}%
\pgfusepath{stroke,fill}%
}%
\begin{pgfscope}%
\pgfsys@transformshift{8.622250in}{0.722700in}%
\pgfsys@useobject{currentmarker}{}%
\end{pgfscope}%
\end{pgfscope}%
\begin{pgfscope}%
\pgftext[x=8.622250in,y=0.625478in,,top]{\sffamily\fontsize{10.000000}{12.000000}\selectfont 10}%
\end{pgfscope}%
\begin{pgfscope}%
\pgfsetbuttcap%
\pgfsetroundjoin%
\definecolor{currentfill}{rgb}{0.000000,0.000000,0.000000}%
\pgfsetfillcolor{currentfill}%
\pgfsetlinewidth{0.803000pt}%
\definecolor{currentstroke}{rgb}{0.000000,0.000000,0.000000}%
\pgfsetstrokecolor{currentstroke}%
\pgfsetdash{}{0pt}%
\pgfsys@defobject{currentmarker}{\pgfqpoint{0.000000in}{-0.048611in}}{\pgfqpoint{0.000000in}{0.000000in}}{%
\pgfpathmoveto{\pgfqpoint{0.000000in}{0.000000in}}%
\pgfpathlineto{\pgfqpoint{0.000000in}{-0.048611in}}%
\pgfusepath{stroke,fill}%
}%
\begin{pgfscope}%
\pgfsys@transformshift{9.192397in}{0.722700in}%
\pgfsys@useobject{currentmarker}{}%
\end{pgfscope}%
\end{pgfscope}%
\begin{pgfscope}%
\pgftext[x=9.192397in,y=0.625478in,,top]{\sffamily\fontsize{10.000000}{12.000000}\selectfont 15}%
\end{pgfscope}%
\begin{pgfscope}%
\pgfsetbuttcap%
\pgfsetroundjoin%
\definecolor{currentfill}{rgb}{0.000000,0.000000,0.000000}%
\pgfsetfillcolor{currentfill}%
\pgfsetlinewidth{0.803000pt}%
\definecolor{currentstroke}{rgb}{0.000000,0.000000,0.000000}%
\pgfsetstrokecolor{currentstroke}%
\pgfsetdash{}{0pt}%
\pgfsys@defobject{currentmarker}{\pgfqpoint{0.000000in}{-0.048611in}}{\pgfqpoint{0.000000in}{0.000000in}}{%
\pgfpathmoveto{\pgfqpoint{0.000000in}{0.000000in}}%
\pgfpathlineto{\pgfqpoint{0.000000in}{-0.048611in}}%
\pgfusepath{stroke,fill}%
}%
\begin{pgfscope}%
\pgfsys@transformshift{9.762545in}{0.722700in}%
\pgfsys@useobject{currentmarker}{}%
\end{pgfscope}%
\end{pgfscope}%
\begin{pgfscope}%
\pgftext[x=9.762545in,y=0.625478in,,top]{\sffamily\fontsize{10.000000}{12.000000}\selectfont 20}%
\end{pgfscope}%
\begin{pgfscope}%
\pgfsetbuttcap%
\pgfsetroundjoin%
\definecolor{currentfill}{rgb}{0.000000,0.000000,0.000000}%
\pgfsetfillcolor{currentfill}%
\pgfsetlinewidth{0.803000pt}%
\definecolor{currentstroke}{rgb}{0.000000,0.000000,0.000000}%
\pgfsetstrokecolor{currentstroke}%
\pgfsetdash{}{0pt}%
\pgfsys@defobject{currentmarker}{\pgfqpoint{0.000000in}{-0.048611in}}{\pgfqpoint{0.000000in}{0.000000in}}{%
\pgfpathmoveto{\pgfqpoint{0.000000in}{0.000000in}}%
\pgfpathlineto{\pgfqpoint{0.000000in}{-0.048611in}}%
\pgfusepath{stroke,fill}%
}%
\begin{pgfscope}%
\pgfsys@transformshift{10.332692in}{0.722700in}%
\pgfsys@useobject{currentmarker}{}%
\end{pgfscope}%
\end{pgfscope}%
\begin{pgfscope}%
\pgftext[x=10.332692in,y=0.625478in,,top]{\sffamily\fontsize{10.000000}{12.000000}\selectfont 25}%
\end{pgfscope}%
\begin{pgfscope}%
\pgfsetbuttcap%
\pgfsetroundjoin%
\definecolor{currentfill}{rgb}{0.000000,0.000000,0.000000}%
\pgfsetfillcolor{currentfill}%
\pgfsetlinewidth{0.803000pt}%
\definecolor{currentstroke}{rgb}{0.000000,0.000000,0.000000}%
\pgfsetstrokecolor{currentstroke}%
\pgfsetdash{}{0pt}%
\pgfsys@defobject{currentmarker}{\pgfqpoint{0.000000in}{-0.048611in}}{\pgfqpoint{0.000000in}{0.000000in}}{%
\pgfpathmoveto{\pgfqpoint{0.000000in}{0.000000in}}%
\pgfpathlineto{\pgfqpoint{0.000000in}{-0.048611in}}%
\pgfusepath{stroke,fill}%
}%
\begin{pgfscope}%
\pgfsys@transformshift{10.902840in}{0.722700in}%
\pgfsys@useobject{currentmarker}{}%
\end{pgfscope}%
\end{pgfscope}%
\begin{pgfscope}%
\pgftext[x=10.902840in,y=0.625478in,,top]{\sffamily\fontsize{10.000000}{12.000000}\selectfont 30}%
\end{pgfscope}%
\begin{pgfscope}%
\pgfsetbuttcap%
\pgfsetroundjoin%
\definecolor{currentfill}{rgb}{0.000000,0.000000,0.000000}%
\pgfsetfillcolor{currentfill}%
\pgfsetlinewidth{0.803000pt}%
\definecolor{currentstroke}{rgb}{0.000000,0.000000,0.000000}%
\pgfsetstrokecolor{currentstroke}%
\pgfsetdash{}{0pt}%
\pgfsys@defobject{currentmarker}{\pgfqpoint{0.000000in}{-0.048611in}}{\pgfqpoint{0.000000in}{0.000000in}}{%
\pgfpathmoveto{\pgfqpoint{0.000000in}{0.000000in}}%
\pgfpathlineto{\pgfqpoint{0.000000in}{-0.048611in}}%
\pgfusepath{stroke,fill}%
}%
\begin{pgfscope}%
\pgfsys@transformshift{11.472988in}{0.722700in}%
\pgfsys@useobject{currentmarker}{}%
\end{pgfscope}%
\end{pgfscope}%
\begin{pgfscope}%
\pgftext[x=11.472988in,y=0.625478in,,top]{\sffamily\fontsize{10.000000}{12.000000}\selectfont 35}%
\end{pgfscope}%
\begin{pgfscope}%
\pgfsetbuttcap%
\pgfsetroundjoin%
\definecolor{currentfill}{rgb}{0.000000,0.000000,0.000000}%
\pgfsetfillcolor{currentfill}%
\pgfsetlinewidth{0.803000pt}%
\definecolor{currentstroke}{rgb}{0.000000,0.000000,0.000000}%
\pgfsetstrokecolor{currentstroke}%
\pgfsetdash{}{0pt}%
\pgfsys@defobject{currentmarker}{\pgfqpoint{0.000000in}{-0.048611in}}{\pgfqpoint{0.000000in}{0.000000in}}{%
\pgfpathmoveto{\pgfqpoint{0.000000in}{0.000000in}}%
\pgfpathlineto{\pgfqpoint{0.000000in}{-0.048611in}}%
\pgfusepath{stroke,fill}%
}%
\begin{pgfscope}%
\pgfsys@transformshift{12.043135in}{0.722700in}%
\pgfsys@useobject{currentmarker}{}%
\end{pgfscope}%
\end{pgfscope}%
\begin{pgfscope}%
\pgftext[x=12.043135in,y=0.625478in,,top]{\sffamily\fontsize{10.000000}{12.000000}\selectfont 40}%
\end{pgfscope}%
\begin{pgfscope}%
\pgftext[x=9.887977in,y=0.435509in,,top]{\sffamily\fontsize{16.000000}{19.200000}\selectfont \(\displaystyle z-position [\mu m]\)}%
\end{pgfscope}%
\begin{pgfscope}%
\pgfsetbuttcap%
\pgfsetroundjoin%
\definecolor{currentfill}{rgb}{0.000000,0.000000,0.000000}%
\pgfsetfillcolor{currentfill}%
\pgfsetlinewidth{0.803000pt}%
\definecolor{currentstroke}{rgb}{0.000000,0.000000,0.000000}%
\pgfsetstrokecolor{currentstroke}%
\pgfsetdash{}{0pt}%
\pgfsys@defobject{currentmarker}{\pgfqpoint{-0.048611in}{0.000000in}}{\pgfqpoint{0.000000in}{0.000000in}}{%
\pgfpathmoveto{\pgfqpoint{0.000000in}{0.000000in}}%
\pgfpathlineto{\pgfqpoint{-0.048611in}{0.000000in}}%
\pgfusepath{stroke,fill}%
}%
\begin{pgfscope}%
\pgfsys@transformshift{7.481955in}{0.722700in}%
\pgfsys@useobject{currentmarker}{}%
\end{pgfscope}%
\end{pgfscope}%
\begin{pgfscope}%
\pgftext[x=6.870748in,y=0.669938in,left,base]{\sffamily\fontsize{10.000000}{12.000000}\selectfont -0.003}%
\end{pgfscope}%
\begin{pgfscope}%
\pgfsetbuttcap%
\pgfsetroundjoin%
\definecolor{currentfill}{rgb}{0.000000,0.000000,0.000000}%
\pgfsetfillcolor{currentfill}%
\pgfsetlinewidth{0.803000pt}%
\definecolor{currentstroke}{rgb}{0.000000,0.000000,0.000000}%
\pgfsetstrokecolor{currentstroke}%
\pgfsetdash{}{0pt}%
\pgfsys@defobject{currentmarker}{\pgfqpoint{-0.048611in}{0.000000in}}{\pgfqpoint{0.000000in}{0.000000in}}{%
\pgfpathmoveto{\pgfqpoint{0.000000in}{0.000000in}}%
\pgfpathlineto{\pgfqpoint{-0.048611in}{0.000000in}}%
\pgfusepath{stroke,fill}%
}%
\begin{pgfscope}%
\pgfsys@transformshift{7.481955in}{1.565850in}%
\pgfsys@useobject{currentmarker}{}%
\end{pgfscope}%
\end{pgfscope}%
\begin{pgfscope}%
\pgftext[x=6.870748in,y=1.513088in,left,base]{\sffamily\fontsize{10.000000}{12.000000}\selectfont -0.002}%
\end{pgfscope}%
\begin{pgfscope}%
\pgfsetbuttcap%
\pgfsetroundjoin%
\definecolor{currentfill}{rgb}{0.000000,0.000000,0.000000}%
\pgfsetfillcolor{currentfill}%
\pgfsetlinewidth{0.803000pt}%
\definecolor{currentstroke}{rgb}{0.000000,0.000000,0.000000}%
\pgfsetstrokecolor{currentstroke}%
\pgfsetdash{}{0pt}%
\pgfsys@defobject{currentmarker}{\pgfqpoint{-0.048611in}{0.000000in}}{\pgfqpoint{0.000000in}{0.000000in}}{%
\pgfpathmoveto{\pgfqpoint{0.000000in}{0.000000in}}%
\pgfpathlineto{\pgfqpoint{-0.048611in}{0.000000in}}%
\pgfusepath{stroke,fill}%
}%
\begin{pgfscope}%
\pgfsys@transformshift{7.481955in}{2.409000in}%
\pgfsys@useobject{currentmarker}{}%
\end{pgfscope}%
\end{pgfscope}%
\begin{pgfscope}%
\pgftext[x=6.870748in,y=2.356238in,left,base]{\sffamily\fontsize{10.000000}{12.000000}\selectfont -0.001}%
\end{pgfscope}%
\begin{pgfscope}%
\pgfsetbuttcap%
\pgfsetroundjoin%
\definecolor{currentfill}{rgb}{0.000000,0.000000,0.000000}%
\pgfsetfillcolor{currentfill}%
\pgfsetlinewidth{0.803000pt}%
\definecolor{currentstroke}{rgb}{0.000000,0.000000,0.000000}%
\pgfsetstrokecolor{currentstroke}%
\pgfsetdash{}{0pt}%
\pgfsys@defobject{currentmarker}{\pgfqpoint{-0.048611in}{0.000000in}}{\pgfqpoint{0.000000in}{0.000000in}}{%
\pgfpathmoveto{\pgfqpoint{0.000000in}{0.000000in}}%
\pgfpathlineto{\pgfqpoint{-0.048611in}{0.000000in}}%
\pgfusepath{stroke,fill}%
}%
\begin{pgfscope}%
\pgfsys@transformshift{7.481955in}{3.252150in}%
\pgfsys@useobject{currentmarker}{}%
\end{pgfscope}%
\end{pgfscope}%
\begin{pgfscope}%
\pgftext[x=6.987122in,y=3.199388in,left,base]{\sffamily\fontsize{10.000000}{12.000000}\selectfont 0.000}%
\end{pgfscope}%
\begin{pgfscope}%
\pgfsetbuttcap%
\pgfsetroundjoin%
\definecolor{currentfill}{rgb}{0.000000,0.000000,0.000000}%
\pgfsetfillcolor{currentfill}%
\pgfsetlinewidth{0.803000pt}%
\definecolor{currentstroke}{rgb}{0.000000,0.000000,0.000000}%
\pgfsetstrokecolor{currentstroke}%
\pgfsetdash{}{0pt}%
\pgfsys@defobject{currentmarker}{\pgfqpoint{-0.048611in}{0.000000in}}{\pgfqpoint{0.000000in}{0.000000in}}{%
\pgfpathmoveto{\pgfqpoint{0.000000in}{0.000000in}}%
\pgfpathlineto{\pgfqpoint{-0.048611in}{0.000000in}}%
\pgfusepath{stroke,fill}%
}%
\begin{pgfscope}%
\pgfsys@transformshift{7.481955in}{4.095300in}%
\pgfsys@useobject{currentmarker}{}%
\end{pgfscope}%
\end{pgfscope}%
\begin{pgfscope}%
\pgftext[x=6.987122in,y=4.042538in,left,base]{\sffamily\fontsize{10.000000}{12.000000}\selectfont 0.001}%
\end{pgfscope}%
\begin{pgfscope}%
\pgfsetbuttcap%
\pgfsetroundjoin%
\definecolor{currentfill}{rgb}{0.000000,0.000000,0.000000}%
\pgfsetfillcolor{currentfill}%
\pgfsetlinewidth{0.803000pt}%
\definecolor{currentstroke}{rgb}{0.000000,0.000000,0.000000}%
\pgfsetstrokecolor{currentstroke}%
\pgfsetdash{}{0pt}%
\pgfsys@defobject{currentmarker}{\pgfqpoint{-0.048611in}{0.000000in}}{\pgfqpoint{0.000000in}{0.000000in}}{%
\pgfpathmoveto{\pgfqpoint{0.000000in}{0.000000in}}%
\pgfpathlineto{\pgfqpoint{-0.048611in}{0.000000in}}%
\pgfusepath{stroke,fill}%
}%
\begin{pgfscope}%
\pgfsys@transformshift{7.481955in}{4.938450in}%
\pgfsys@useobject{currentmarker}{}%
\end{pgfscope}%
\end{pgfscope}%
\begin{pgfscope}%
\pgftext[x=6.987122in,y=4.885688in,left,base]{\sffamily\fontsize{10.000000}{12.000000}\selectfont 0.002}%
\end{pgfscope}%
\begin{pgfscope}%
\pgfsetbuttcap%
\pgfsetroundjoin%
\definecolor{currentfill}{rgb}{0.000000,0.000000,0.000000}%
\pgfsetfillcolor{currentfill}%
\pgfsetlinewidth{0.803000pt}%
\definecolor{currentstroke}{rgb}{0.000000,0.000000,0.000000}%
\pgfsetstrokecolor{currentstroke}%
\pgfsetdash{}{0pt}%
\pgfsys@defobject{currentmarker}{\pgfqpoint{-0.048611in}{0.000000in}}{\pgfqpoint{0.000000in}{0.000000in}}{%
\pgfpathmoveto{\pgfqpoint{0.000000in}{0.000000in}}%
\pgfpathlineto{\pgfqpoint{-0.048611in}{0.000000in}}%
\pgfusepath{stroke,fill}%
}%
\begin{pgfscope}%
\pgfsys@transformshift{7.481955in}{5.781600in}%
\pgfsys@useobject{currentmarker}{}%
\end{pgfscope}%
\end{pgfscope}%
\begin{pgfscope}%
\pgftext[x=6.987122in,y=5.728838in,left,base]{\sffamily\fontsize{10.000000}{12.000000}\selectfont 0.003}%
\end{pgfscope}%
\begin{pgfscope}%
\pgftext[x=6.815193in,y=3.252150in,,bottom,rotate=90.000000]{\sffamily\fontsize{16.000000}{19.200000}\selectfont \(\displaystyle Poynting\) \(\displaystyle vector\)}%
\end{pgfscope}%
\begin{pgfscope}%
\pgfpathrectangle{\pgfqpoint{7.481955in}{0.722700in}}{\pgfqpoint{4.812045in}{5.058900in}} %
\pgfusepath{clip}%
\pgfsetrectcap%
\pgfsetroundjoin%
\pgfsetlinewidth{1.505625pt}%
\definecolor{currentstroke}{rgb}{0.000000,0.000000,0.000000}%
\pgfsetstrokecolor{currentstroke}%
\pgfsetdash{}{0pt}%
\pgfpathmoveto{\pgfqpoint{7.481955in}{3.252150in}}%
\pgfpathlineto{\pgfqpoint{10.590954in}{3.253125in}}%
\pgfpathlineto{\pgfqpoint{10.594302in}{3.256292in}}%
\pgfpathlineto{\pgfqpoint{10.598068in}{3.263496in}}%
\pgfpathlineto{\pgfqpoint{10.609784in}{3.290376in}}%
\pgfpathlineto{\pgfqpoint{10.611458in}{3.289269in}}%
\pgfpathlineto{\pgfqpoint{10.613550in}{3.284772in}}%
\pgfpathlineto{\pgfqpoint{10.617316in}{3.269808in}}%
\pgfpathlineto{\pgfqpoint{10.621919in}{3.252931in}}%
\pgfpathlineto{\pgfqpoint{10.623174in}{3.252172in}}%
\pgfpathlineto{\pgfqpoint{10.623593in}{3.252535in}}%
\pgfpathlineto{\pgfqpoint{10.624848in}{3.255796in}}%
\pgfpathlineto{\pgfqpoint{10.626940in}{3.269616in}}%
\pgfpathlineto{\pgfqpoint{10.629869in}{3.308835in}}%
\pgfpathlineto{\pgfqpoint{10.634472in}{3.411217in}}%
\pgfpathlineto{\pgfqpoint{10.641167in}{3.557649in}}%
\pgfpathlineto{\pgfqpoint{10.643678in}{3.573928in}}%
\pgfpathlineto{\pgfqpoint{10.644515in}{3.572018in}}%
\pgfpathlineto{\pgfqpoint{10.646188in}{3.556602in}}%
\pgfpathlineto{\pgfqpoint{10.648699in}{3.506330in}}%
\pgfpathlineto{\pgfqpoint{10.659160in}{3.252063in}}%
\pgfpathlineto{\pgfqpoint{10.659997in}{3.256517in}}%
\pgfpathlineto{\pgfqpoint{10.661670in}{3.286013in}}%
\pgfpathlineto{\pgfqpoint{10.664181in}{3.383122in}}%
\pgfpathlineto{\pgfqpoint{10.668366in}{3.661199in}}%
\pgfpathlineto{\pgfqpoint{10.675897in}{4.165698in}}%
\pgfpathlineto{\pgfqpoint{10.677990in}{4.210421in}}%
\pgfpathlineto{\pgfqpoint{10.678408in}{4.211697in}}%
\pgfpathlineto{\pgfqpoint{10.678826in}{4.210316in}}%
\pgfpathlineto{\pgfqpoint{10.680082in}{4.190171in}}%
\pgfpathlineto{\pgfqpoint{10.682174in}{4.105203in}}%
\pgfpathlineto{\pgfqpoint{10.685940in}{3.826120in}}%
\pgfpathlineto{\pgfqpoint{10.693053in}{3.291039in}}%
\pgfpathlineto{\pgfqpoint{10.695146in}{3.251746in}}%
\pgfpathlineto{\pgfqpoint{10.695982in}{3.259571in}}%
\pgfpathlineto{\pgfqpoint{10.697656in}{3.317520in}}%
\pgfpathlineto{\pgfqpoint{10.700167in}{3.505479in}}%
\pgfpathlineto{\pgfqpoint{10.704770in}{4.069022in}}%
\pgfpathlineto{\pgfqpoint{10.710628in}{4.740258in}}%
\pgfpathlineto{\pgfqpoint{10.713138in}{4.851719in}}%
\pgfpathlineto{\pgfqpoint{10.713557in}{4.855793in}}%
\pgfpathlineto{\pgfqpoint{10.713975in}{4.855545in}}%
\pgfpathlineto{\pgfqpoint{10.714812in}{4.842054in}}%
\pgfpathlineto{\pgfqpoint{10.716486in}{4.764181in}}%
\pgfpathlineto{\pgfqpoint{10.719415in}{4.484656in}}%
\pgfpathlineto{\pgfqpoint{10.730294in}{3.262441in}}%
\pgfpathlineto{\pgfqpoint{10.731131in}{3.251581in}}%
\pgfpathlineto{\pgfqpoint{10.731550in}{3.253405in}}%
\pgfpathlineto{\pgfqpoint{10.732805in}{3.288133in}}%
\pgfpathlineto{\pgfqpoint{10.734897in}{3.440308in}}%
\pgfpathlineto{\pgfqpoint{10.738245in}{3.884456in}}%
\pgfpathlineto{\pgfqpoint{10.747032in}{5.157967in}}%
\pgfpathlineto{\pgfqpoint{10.749124in}{5.244909in}}%
\pgfpathlineto{\pgfqpoint{10.749543in}{5.246622in}}%
\pgfpathlineto{\pgfqpoint{10.750380in}{5.234071in}}%
\pgfpathlineto{\pgfqpoint{10.752053in}{5.146383in}}%
\pgfpathlineto{\pgfqpoint{10.754982in}{4.816096in}}%
\pgfpathlineto{\pgfqpoint{10.766280in}{3.267096in}}%
\pgfpathlineto{\pgfqpoint{10.767117in}{3.251845in}}%
\pgfpathlineto{\pgfqpoint{10.767535in}{3.252542in}}%
\pgfpathlineto{\pgfqpoint{10.768791in}{3.287966in}}%
\pgfpathlineto{\pgfqpoint{10.770883in}{3.453466in}}%
\pgfpathlineto{\pgfqpoint{10.774231in}{3.941169in}}%
\pgfpathlineto{\pgfqpoint{10.783018in}{5.323721in}}%
\pgfpathlineto{\pgfqpoint{10.785110in}{5.414289in}}%
\pgfpathlineto{\pgfqpoint{10.785528in}{5.415412in}}%
\pgfpathlineto{\pgfqpoint{10.786365in}{5.400415in}}%
\pgfpathlineto{\pgfqpoint{10.788039in}{5.303127in}}%
\pgfpathlineto{\pgfqpoint{10.790968in}{4.943927in}}%
\pgfpathlineto{\pgfqpoint{10.802266in}{3.271465in}}%
\pgfpathlineto{\pgfqpoint{10.803521in}{3.251831in}}%
\pgfpathlineto{\pgfqpoint{10.804358in}{3.267860in}}%
\pgfpathlineto{\pgfqpoint{10.806032in}{3.368172in}}%
\pgfpathlineto{\pgfqpoint{10.808961in}{3.735393in}}%
\pgfpathlineto{\pgfqpoint{10.820259in}{5.448967in}}%
\pgfpathlineto{\pgfqpoint{10.821514in}{5.471965in}}%
\pgfpathlineto{\pgfqpoint{10.821933in}{5.467866in}}%
\pgfpathlineto{\pgfqpoint{10.823188in}{5.420567in}}%
\pgfpathlineto{\pgfqpoint{10.825280in}{5.231842in}}%
\pgfpathlineto{\pgfqpoint{10.829046in}{4.630228in}}%
\pgfpathlineto{\pgfqpoint{10.836578in}{3.388317in}}%
\pgfpathlineto{\pgfqpoint{10.839089in}{3.253675in}}%
\pgfpathlineto{\pgfqpoint{10.839507in}{3.251456in}}%
\pgfpathlineto{\pgfqpoint{10.839507in}{3.251456in}}%
\pgfpathlineto{\pgfqpoint{10.839507in}{3.251456in}}%
\pgfpathlineto{\pgfqpoint{10.840344in}{3.264749in}}%
\pgfpathlineto{\pgfqpoint{10.842018in}{3.360646in}}%
\pgfpathlineto{\pgfqpoint{10.844947in}{3.723912in}}%
\pgfpathlineto{\pgfqpoint{10.856663in}{5.474572in}}%
\pgfpathlineto{\pgfqpoint{10.857500in}{5.486550in}}%
\pgfpathlineto{\pgfqpoint{10.857918in}{5.483653in}}%
\pgfpathlineto{\pgfqpoint{10.859174in}{5.439694in}}%
\pgfpathlineto{\pgfqpoint{10.861266in}{5.255333in}}%
\pgfpathlineto{\pgfqpoint{10.865032in}{4.656411in}}%
\pgfpathlineto{\pgfqpoint{10.872564in}{3.398713in}}%
\pgfpathlineto{\pgfqpoint{10.875074in}{3.255162in}}%
\pgfpathlineto{\pgfqpoint{10.875493in}{3.251454in}}%
\pgfpathlineto{\pgfqpoint{10.875911in}{3.253680in}}%
\pgfpathlineto{\pgfqpoint{10.877166in}{3.295681in}}%
\pgfpathlineto{\pgfqpoint{10.879259in}{3.477173in}}%
\pgfpathlineto{\pgfqpoint{10.883025in}{4.073214in}}%
\pgfpathlineto{\pgfqpoint{10.890975in}{5.375145in}}%
\pgfpathlineto{\pgfqpoint{10.893486in}{5.488999in}}%
\pgfpathlineto{\pgfqpoint{10.893904in}{5.487524in}}%
\pgfpathlineto{\pgfqpoint{10.895159in}{5.447728in}}%
\pgfpathlineto{\pgfqpoint{10.897252in}{5.269536in}}%
\pgfpathlineto{\pgfqpoint{10.901017in}{4.677154in}}%
\pgfpathlineto{\pgfqpoint{10.908968in}{3.369925in}}%
\pgfpathlineto{\pgfqpoint{10.911478in}{3.251825in}}%
\pgfpathlineto{\pgfqpoint{10.911897in}{3.252567in}}%
\pgfpathlineto{\pgfqpoint{10.913152in}{3.290199in}}%
\pgfpathlineto{\pgfqpoint{10.915244in}{3.465120in}}%
\pgfpathlineto{\pgfqpoint{10.918592in}{3.976550in}}%
\pgfpathlineto{\pgfqpoint{10.927379in}{5.400840in}}%
\pgfpathlineto{\pgfqpoint{10.929471in}{5.488727in}}%
\pgfpathlineto{\pgfqpoint{10.929890in}{5.488727in}}%
\pgfpathlineto{\pgfqpoint{10.930727in}{5.470964in}}%
\pgfpathlineto{\pgfqpoint{10.932400in}{5.366346in}}%
\pgfpathlineto{\pgfqpoint{10.935329in}{4.990554in}}%
\pgfpathlineto{\pgfqpoint{10.946627in}{3.271806in}}%
\pgfpathlineto{\pgfqpoint{10.947883in}{3.251825in}}%
\pgfpathlineto{\pgfqpoint{10.948719in}{3.268115in}}%
\pgfpathlineto{\pgfqpoint{10.950393in}{3.369929in}}%
\pgfpathlineto{\pgfqpoint{10.953322in}{3.741874in}}%
\pgfpathlineto{\pgfqpoint{10.964620in}{5.467100in}}%
\pgfpathlineto{\pgfqpoint{10.965875in}{5.489295in}}%
\pgfpathlineto{\pgfqpoint{10.966712in}{5.474482in}}%
\pgfpathlineto{\pgfqpoint{10.968386in}{5.375482in}}%
\pgfpathlineto{\pgfqpoint{10.971315in}{5.007419in}}%
\pgfpathlineto{\pgfqpoint{10.983031in}{3.261831in}}%
\pgfpathlineto{\pgfqpoint{10.983868in}{3.251454in}}%
\pgfpathlineto{\pgfqpoint{10.984287in}{3.255164in}}%
\pgfpathlineto{\pgfqpoint{10.985542in}{3.301553in}}%
\pgfpathlineto{\pgfqpoint{10.987634in}{3.489678in}}%
\pgfpathlineto{\pgfqpoint{10.991400in}{4.093138in}}%
\pgfpathlineto{\pgfqpoint{10.998932in}{5.347110in}}%
\pgfpathlineto{\pgfqpoint{11.001443in}{5.486513in}}%
\pgfpathlineto{\pgfqpoint{11.001861in}{5.489481in}}%
\pgfpathlineto{\pgfqpoint{11.002280in}{5.486513in}}%
\pgfpathlineto{\pgfqpoint{11.003535in}{5.442307in}}%
\pgfpathlineto{\pgfqpoint{11.005627in}{5.257460in}}%
\pgfpathlineto{\pgfqpoint{11.009393in}{4.657611in}}%
\pgfpathlineto{\pgfqpoint{11.016925in}{3.398793in}}%
\pgfpathlineto{\pgfqpoint{11.019436in}{3.255164in}}%
\pgfpathlineto{\pgfqpoint{11.019854in}{3.251454in}}%
\pgfpathlineto{\pgfqpoint{11.020272in}{3.253680in}}%
\pgfpathlineto{\pgfqpoint{11.021528in}{3.295699in}}%
\pgfpathlineto{\pgfqpoint{11.023620in}{3.477253in}}%
\pgfpathlineto{\pgfqpoint{11.027386in}{4.073441in}}%
\pgfpathlineto{\pgfqpoint{11.035336in}{5.375482in}}%
\pgfpathlineto{\pgfqpoint{11.037847in}{5.489295in}}%
\pgfpathlineto{\pgfqpoint{11.038265in}{5.487811in}}%
\pgfpathlineto{\pgfqpoint{11.039521in}{5.447984in}}%
\pgfpathlineto{\pgfqpoint{11.041613in}{5.269737in}}%
\pgfpathlineto{\pgfqpoint{11.045379in}{4.677260in}}%
\pgfpathlineto{\pgfqpoint{11.053329in}{3.369929in}}%
\pgfpathlineto{\pgfqpoint{11.055840in}{3.251825in}}%
\pgfpathlineto{\pgfqpoint{11.056258in}{3.252567in}}%
\pgfpathlineto{\pgfqpoint{11.057514in}{3.290200in}}%
\pgfpathlineto{\pgfqpoint{11.059606in}{3.465125in}}%
\pgfpathlineto{\pgfqpoint{11.062953in}{3.976561in}}%
\pgfpathlineto{\pgfqpoint{11.071740in}{5.400846in}}%
\pgfpathlineto{\pgfqpoint{11.073833in}{5.488727in}}%
\pgfpathlineto{\pgfqpoint{11.074251in}{5.488727in}}%
\pgfpathlineto{\pgfqpoint{11.075088in}{5.470961in}}%
\pgfpathlineto{\pgfqpoint{11.076762in}{5.366340in}}%
\pgfpathlineto{\pgfqpoint{11.079691in}{4.990542in}}%
\pgfpathlineto{\pgfqpoint{11.090989in}{3.271805in}}%
\pgfpathlineto{\pgfqpoint{11.092244in}{3.251825in}}%
\pgfpathlineto{\pgfqpoint{11.093081in}{3.268114in}}%
\pgfpathlineto{\pgfqpoint{11.094755in}{3.369925in}}%
\pgfpathlineto{\pgfqpoint{11.097684in}{3.741850in}}%
\pgfpathlineto{\pgfqpoint{11.108982in}{5.466834in}}%
\pgfpathlineto{\pgfqpoint{11.110237in}{5.488999in}}%
\pgfpathlineto{\pgfqpoint{11.111074in}{5.474170in}}%
\pgfpathlineto{\pgfqpoint{11.112747in}{5.375145in}}%
\pgfpathlineto{\pgfqpoint{11.115677in}{5.007076in}}%
\pgfpathlineto{\pgfqpoint{11.127393in}{3.261826in}}%
\pgfpathlineto{\pgfqpoint{11.128230in}{3.251454in}}%
\pgfpathlineto{\pgfqpoint{11.128648in}{3.255162in}}%
\pgfpathlineto{\pgfqpoint{11.129903in}{3.301529in}}%
\pgfpathlineto{\pgfqpoint{11.131996in}{3.489542in}}%
\pgfpathlineto{\pgfqpoint{11.135762in}{4.092535in}}%
\pgfpathlineto{\pgfqpoint{11.143293in}{5.344781in}}%
\pgfpathlineto{\pgfqpoint{11.145804in}{5.483653in}}%
\pgfpathlineto{\pgfqpoint{11.146223in}{5.486550in}}%
\pgfpathlineto{\pgfqpoint{11.146223in}{5.486550in}}%
\pgfpathlineto{\pgfqpoint{11.146223in}{5.486550in}}%
\pgfpathlineto{\pgfqpoint{11.147059in}{5.474572in}}%
\pgfpathlineto{\pgfqpoint{11.148733in}{5.381071in}}%
\pgfpathlineto{\pgfqpoint{11.151244in}{5.085284in}}%
\pgfpathlineto{\pgfqpoint{11.156684in}{4.091211in}}%
\pgfpathlineto{\pgfqpoint{11.161705in}{3.360646in}}%
\pgfpathlineto{\pgfqpoint{11.164215in}{3.251456in}}%
\pgfpathlineto{\pgfqpoint{11.165052in}{3.261797in}}%
\pgfpathlineto{\pgfqpoint{11.166726in}{3.351961in}}%
\pgfpathlineto{\pgfqpoint{11.169237in}{3.642981in}}%
\pgfpathlineto{\pgfqpoint{11.174676in}{4.630228in}}%
\pgfpathlineto{\pgfqpoint{11.179698in}{5.360773in}}%
\pgfpathlineto{\pgfqpoint{11.182208in}{5.471965in}}%
\pgfpathlineto{\pgfqpoint{11.182627in}{5.470170in}}%
\pgfpathlineto{\pgfqpoint{11.183882in}{5.429675in}}%
\pgfpathlineto{\pgfqpoint{11.185974in}{5.251311in}}%
\pgfpathlineto{\pgfqpoint{11.189740in}{4.661968in}}%
\pgfpathlineto{\pgfqpoint{11.197691in}{3.368172in}}%
\pgfpathlineto{\pgfqpoint{11.200201in}{3.251831in}}%
\pgfpathlineto{\pgfqpoint{11.200620in}{3.252560in}}%
\pgfpathlineto{\pgfqpoint{11.201875in}{3.289529in}}%
\pgfpathlineto{\pgfqpoint{11.203967in}{3.461045in}}%
\pgfpathlineto{\pgfqpoint{11.207315in}{3.960756in}}%
\pgfpathlineto{\pgfqpoint{11.215683in}{5.303127in}}%
\pgfpathlineto{\pgfqpoint{11.218194in}{5.415412in}}%
\pgfpathlineto{\pgfqpoint{11.218612in}{5.414289in}}%
\pgfpathlineto{\pgfqpoint{11.219868in}{5.376630in}}%
\pgfpathlineto{\pgfqpoint{11.221960in}{5.205513in}}%
\pgfpathlineto{\pgfqpoint{11.225307in}{4.709538in}}%
\pgfpathlineto{\pgfqpoint{11.234095in}{3.338804in}}%
\pgfpathlineto{\pgfqpoint{11.236187in}{3.252542in}}%
\pgfpathlineto{\pgfqpoint{11.236605in}{3.251845in}}%
\pgfpathlineto{\pgfqpoint{11.237442in}{3.267096in}}%
\pgfpathlineto{\pgfqpoint{11.239116in}{3.361992in}}%
\pgfpathlineto{\pgfqpoint{11.242045in}{3.705599in}}%
\pgfpathlineto{\pgfqpoint{11.252924in}{5.219859in}}%
\pgfpathlineto{\pgfqpoint{11.254180in}{5.246622in}}%
\pgfpathlineto{\pgfqpoint{11.254598in}{5.244909in}}%
\pgfpathlineto{\pgfqpoint{11.255853in}{5.208097in}}%
\pgfpathlineto{\pgfqpoint{11.257946in}{5.047207in}}%
\pgfpathlineto{\pgfqpoint{11.261712in}{4.519330in}}%
\pgfpathlineto{\pgfqpoint{11.269662in}{3.365891in}}%
\pgfpathlineto{\pgfqpoint{11.272173in}{3.253405in}}%
\pgfpathlineto{\pgfqpoint{11.272591in}{3.251581in}}%
\pgfpathlineto{\pgfqpoint{11.272591in}{3.251581in}}%
\pgfpathlineto{\pgfqpoint{11.272591in}{3.251581in}}%
\pgfpathlineto{\pgfqpoint{11.273428in}{3.262441in}}%
\pgfpathlineto{\pgfqpoint{11.275102in}{3.339836in}}%
\pgfpathlineto{\pgfqpoint{11.278031in}{3.626015in}}%
\pgfpathlineto{\pgfqpoint{11.288492in}{4.828853in}}%
\pgfpathlineto{\pgfqpoint{11.290165in}{4.855793in}}%
\pgfpathlineto{\pgfqpoint{11.291002in}{4.843363in}}%
\pgfpathlineto{\pgfqpoint{11.292676in}{4.768669in}}%
\pgfpathlineto{\pgfqpoint{11.295605in}{4.501437in}}%
\pgfpathlineto{\pgfqpoint{11.306903in}{3.281494in}}%
\pgfpathlineto{\pgfqpoint{11.308577in}{3.251746in}}%
\pgfpathlineto{\pgfqpoint{11.309414in}{3.257700in}}%
\pgfpathlineto{\pgfqpoint{11.311087in}{3.308195in}}%
\pgfpathlineto{\pgfqpoint{11.314016in}{3.496108in}}%
\pgfpathlineto{\pgfqpoint{11.323222in}{4.178178in}}%
\pgfpathlineto{\pgfqpoint{11.325314in}{4.211697in}}%
\pgfpathlineto{\pgfqpoint{11.326151in}{4.206517in}}%
\pgfpathlineto{\pgfqpoint{11.327825in}{4.165698in}}%
\pgfpathlineto{\pgfqpoint{11.330754in}{4.011735in}}%
\pgfpathlineto{\pgfqpoint{11.342470in}{3.275992in}}%
\pgfpathlineto{\pgfqpoint{11.344563in}{3.252063in}}%
\pgfpathlineto{\pgfqpoint{11.344981in}{3.252260in}}%
\pgfpathlineto{\pgfqpoint{11.346236in}{3.261775in}}%
\pgfpathlineto{\pgfqpoint{11.348328in}{3.302469in}}%
\pgfpathlineto{\pgfqpoint{11.359626in}{3.573454in}}%
\pgfpathlineto{\pgfqpoint{11.360045in}{3.573928in}}%
\pgfpathlineto{\pgfqpoint{11.360463in}{3.573452in}}%
\pgfpathlineto{\pgfqpoint{11.361718in}{3.566537in}}%
\pgfpathlineto{\pgfqpoint{11.363811in}{3.538686in}}%
\pgfpathlineto{\pgfqpoint{11.367995in}{3.443491in}}%
\pgfpathlineto{\pgfqpoint{11.374690in}{3.295224in}}%
\pgfpathlineto{\pgfqpoint{11.378038in}{3.259983in}}%
\pgfpathlineto{\pgfqpoint{11.380548in}{3.252172in}}%
\pgfpathlineto{\pgfqpoint{11.381804in}{3.252931in}}%
\pgfpathlineto{\pgfqpoint{11.383896in}{3.258854in}}%
\pgfpathlineto{\pgfqpoint{11.392683in}{3.289753in}}%
\pgfpathlineto{\pgfqpoint{11.394775in}{3.290126in}}%
\pgfpathlineto{\pgfqpoint{11.396867in}{3.287471in}}%
\pgfpathlineto{\pgfqpoint{11.400633in}{3.277829in}}%
\pgfpathlineto{\pgfqpoint{11.408165in}{3.258245in}}%
\pgfpathlineto{\pgfqpoint{11.412350in}{3.253377in}}%
\pgfpathlineto{\pgfqpoint{11.416116in}{3.252173in}}%
\pgfpathlineto{\pgfqpoint{11.424484in}{3.252864in}}%
\pgfpathlineto{\pgfqpoint{11.436201in}{3.252483in}}%
\pgfpathlineto{\pgfqpoint{11.460470in}{3.252151in}}%
\pgfpathlineto{\pgfqpoint{12.293582in}{3.252150in}}%
\pgfpathlineto{\pgfqpoint{12.293582in}{3.252150in}}%
\pgfusepath{stroke}%
\end{pgfscope}%
\begin{pgfscope}%
\pgfsetrectcap%
\pgfsetmiterjoin%
\pgfsetlinewidth{0.803000pt}%
\definecolor{currentstroke}{rgb}{0.000000,0.000000,0.000000}%
\pgfsetstrokecolor{currentstroke}%
\pgfsetdash{}{0pt}%
\pgfpathmoveto{\pgfqpoint{7.481955in}{0.722700in}}%
\pgfpathlineto{\pgfqpoint{7.481955in}{5.781600in}}%
\pgfusepath{stroke}%
\end{pgfscope}%
\begin{pgfscope}%
\pgfsetrectcap%
\pgfsetmiterjoin%
\pgfsetlinewidth{0.803000pt}%
\definecolor{currentstroke}{rgb}{0.000000,0.000000,0.000000}%
\pgfsetstrokecolor{currentstroke}%
\pgfsetdash{}{0pt}%
\pgfpathmoveto{\pgfqpoint{12.294000in}{0.722700in}}%
\pgfpathlineto{\pgfqpoint{12.294000in}{5.781600in}}%
\pgfusepath{stroke}%
\end{pgfscope}%
\begin{pgfscope}%
\pgfsetrectcap%
\pgfsetmiterjoin%
\pgfsetlinewidth{0.803000pt}%
\definecolor{currentstroke}{rgb}{0.000000,0.000000,0.000000}%
\pgfsetstrokecolor{currentstroke}%
\pgfsetdash{}{0pt}%
\pgfpathmoveto{\pgfqpoint{7.481955in}{0.722700in}}%
\pgfpathlineto{\pgfqpoint{12.294000in}{0.722700in}}%
\pgfusepath{stroke}%
\end{pgfscope}%
\begin{pgfscope}%
\pgfsetrectcap%
\pgfsetmiterjoin%
\pgfsetlinewidth{0.803000pt}%
\definecolor{currentstroke}{rgb}{0.000000,0.000000,0.000000}%
\pgfsetstrokecolor{currentstroke}%
\pgfsetdash{}{0pt}%
\pgfpathmoveto{\pgfqpoint{7.481955in}{5.781600in}}%
\pgfpathlineto{\pgfqpoint{12.294000in}{5.781600in}}%
\pgfusepath{stroke}%
\end{pgfscope}%
\end{pgfpicture}%
\makeatother%
\endgroup%
}}
%        \subcaption{Simulation before the wave hit the right side of the numerical window.}
%        \label{fig:task2_1}
%    \end{subfigure}\\
%    \begin{subfigure}[b]{0.5\textwidth}
%        \noindent\makebox[\textwidth]{\scalebox{0.5}{%% Creator: Matplotlib, PGF backend
%%
%% To include the figure in your LaTeX document, write
%%   \input{<filename>.pgf}
%%
%% Make sure the required packages are loaded in your preamble
%%   \usepackage{pgf}
%%
%% Figures using additional raster images can only be included by \input if
%% they are in the same directory as the main LaTeX file. For loading figures
%% from other directories you can use the `import` package
%%   \usepackage{import}
%% and then include the figures with
%%   \import{<path to file>}{<filename>.pgf}
%%
%% Matplotlib used the following preamble
%%   \usepackage{fontspec}
%%   \setmainfont{DejaVu Serif}
%%   \setsansfont{DejaVu Sans}
%%   \setmonofont{DejaVu Sans Mono}
%%
\begingroup%
\makeatletter%
\begin{pgfpicture}%
\pgfpathrectangle{\pgfpointorigin}{\pgfqpoint{13.660000in}{6.570000in}}%
\pgfusepath{use as bounding box, clip}%
\begin{pgfscope}%
\pgfsetbuttcap%
\pgfsetmiterjoin%
\definecolor{currentfill}{rgb}{1.000000,1.000000,1.000000}%
\pgfsetfillcolor{currentfill}%
\pgfsetlinewidth{0.000000pt}%
\definecolor{currentstroke}{rgb}{1.000000,1.000000,1.000000}%
\pgfsetstrokecolor{currentstroke}%
\pgfsetdash{}{0pt}%
\pgfpathmoveto{\pgfqpoint{0.000000in}{0.000000in}}%
\pgfpathlineto{\pgfqpoint{13.660000in}{0.000000in}}%
\pgfpathlineto{\pgfqpoint{13.660000in}{6.570000in}}%
\pgfpathlineto{\pgfqpoint{0.000000in}{6.570000in}}%
\pgfpathclose%
\pgfusepath{fill}%
\end{pgfscope}%
\begin{pgfscope}%
\pgfsetbuttcap%
\pgfsetmiterjoin%
\definecolor{currentfill}{rgb}{1.000000,1.000000,1.000000}%
\pgfsetfillcolor{currentfill}%
\pgfsetlinewidth{0.000000pt}%
\definecolor{currentstroke}{rgb}{0.000000,0.000000,0.000000}%
\pgfsetstrokecolor{currentstroke}%
\pgfsetstrokeopacity{0.000000}%
\pgfsetdash{}{0pt}%
\pgfpathmoveto{\pgfqpoint{1.707500in}{0.722700in}}%
\pgfpathlineto{\pgfqpoint{6.519545in}{0.722700in}}%
\pgfpathlineto{\pgfqpoint{6.519545in}{5.781600in}}%
\pgfpathlineto{\pgfqpoint{1.707500in}{5.781600in}}%
\pgfpathclose%
\pgfusepath{fill}%
\end{pgfscope}%
\begin{pgfscope}%
\pgfsetbuttcap%
\pgfsetroundjoin%
\definecolor{currentfill}{rgb}{0.000000,0.000000,0.000000}%
\pgfsetfillcolor{currentfill}%
\pgfsetlinewidth{0.803000pt}%
\definecolor{currentstroke}{rgb}{0.000000,0.000000,0.000000}%
\pgfsetstrokecolor{currentstroke}%
\pgfsetdash{}{0pt}%
\pgfsys@defobject{currentmarker}{\pgfqpoint{0.000000in}{-0.048611in}}{\pgfqpoint{0.000000in}{0.000000in}}{%
\pgfpathmoveto{\pgfqpoint{0.000000in}{0.000000in}}%
\pgfpathlineto{\pgfqpoint{0.000000in}{-0.048611in}}%
\pgfusepath{stroke,fill}%
}%
\begin{pgfscope}%
\pgfsys@transformshift{1.707500in}{0.722700in}%
\pgfsys@useobject{currentmarker}{}%
\end{pgfscope}%
\end{pgfscope}%
\begin{pgfscope}%
\pgftext[x=1.707500in,y=0.625478in,,top]{\sffamily\fontsize{10.000000}{12.000000}\selectfont 0}%
\end{pgfscope}%
\begin{pgfscope}%
\pgfsetbuttcap%
\pgfsetroundjoin%
\definecolor{currentfill}{rgb}{0.000000,0.000000,0.000000}%
\pgfsetfillcolor{currentfill}%
\pgfsetlinewidth{0.803000pt}%
\definecolor{currentstroke}{rgb}{0.000000,0.000000,0.000000}%
\pgfsetstrokecolor{currentstroke}%
\pgfsetdash{}{0pt}%
\pgfsys@defobject{currentmarker}{\pgfqpoint{0.000000in}{-0.048611in}}{\pgfqpoint{0.000000in}{0.000000in}}{%
\pgfpathmoveto{\pgfqpoint{0.000000in}{0.000000in}}%
\pgfpathlineto{\pgfqpoint{0.000000in}{-0.048611in}}%
\pgfusepath{stroke,fill}%
}%
\begin{pgfscope}%
\pgfsys@transformshift{2.277648in}{0.722700in}%
\pgfsys@useobject{currentmarker}{}%
\end{pgfscope}%
\end{pgfscope}%
\begin{pgfscope}%
\pgftext[x=2.277648in,y=0.625478in,,top]{\sffamily\fontsize{10.000000}{12.000000}\selectfont 5}%
\end{pgfscope}%
\begin{pgfscope}%
\pgfsetbuttcap%
\pgfsetroundjoin%
\definecolor{currentfill}{rgb}{0.000000,0.000000,0.000000}%
\pgfsetfillcolor{currentfill}%
\pgfsetlinewidth{0.803000pt}%
\definecolor{currentstroke}{rgb}{0.000000,0.000000,0.000000}%
\pgfsetstrokecolor{currentstroke}%
\pgfsetdash{}{0pt}%
\pgfsys@defobject{currentmarker}{\pgfqpoint{0.000000in}{-0.048611in}}{\pgfqpoint{0.000000in}{0.000000in}}{%
\pgfpathmoveto{\pgfqpoint{0.000000in}{0.000000in}}%
\pgfpathlineto{\pgfqpoint{0.000000in}{-0.048611in}}%
\pgfusepath{stroke,fill}%
}%
\begin{pgfscope}%
\pgfsys@transformshift{2.847795in}{0.722700in}%
\pgfsys@useobject{currentmarker}{}%
\end{pgfscope}%
\end{pgfscope}%
\begin{pgfscope}%
\pgftext[x=2.847795in,y=0.625478in,,top]{\sffamily\fontsize{10.000000}{12.000000}\selectfont 10}%
\end{pgfscope}%
\begin{pgfscope}%
\pgfsetbuttcap%
\pgfsetroundjoin%
\definecolor{currentfill}{rgb}{0.000000,0.000000,0.000000}%
\pgfsetfillcolor{currentfill}%
\pgfsetlinewidth{0.803000pt}%
\definecolor{currentstroke}{rgb}{0.000000,0.000000,0.000000}%
\pgfsetstrokecolor{currentstroke}%
\pgfsetdash{}{0pt}%
\pgfsys@defobject{currentmarker}{\pgfqpoint{0.000000in}{-0.048611in}}{\pgfqpoint{0.000000in}{0.000000in}}{%
\pgfpathmoveto{\pgfqpoint{0.000000in}{0.000000in}}%
\pgfpathlineto{\pgfqpoint{0.000000in}{-0.048611in}}%
\pgfusepath{stroke,fill}%
}%
\begin{pgfscope}%
\pgfsys@transformshift{3.417943in}{0.722700in}%
\pgfsys@useobject{currentmarker}{}%
\end{pgfscope}%
\end{pgfscope}%
\begin{pgfscope}%
\pgftext[x=3.417943in,y=0.625478in,,top]{\sffamily\fontsize{10.000000}{12.000000}\selectfont 15}%
\end{pgfscope}%
\begin{pgfscope}%
\pgfsetbuttcap%
\pgfsetroundjoin%
\definecolor{currentfill}{rgb}{0.000000,0.000000,0.000000}%
\pgfsetfillcolor{currentfill}%
\pgfsetlinewidth{0.803000pt}%
\definecolor{currentstroke}{rgb}{0.000000,0.000000,0.000000}%
\pgfsetstrokecolor{currentstroke}%
\pgfsetdash{}{0pt}%
\pgfsys@defobject{currentmarker}{\pgfqpoint{0.000000in}{-0.048611in}}{\pgfqpoint{0.000000in}{0.000000in}}{%
\pgfpathmoveto{\pgfqpoint{0.000000in}{0.000000in}}%
\pgfpathlineto{\pgfqpoint{0.000000in}{-0.048611in}}%
\pgfusepath{stroke,fill}%
}%
\begin{pgfscope}%
\pgfsys@transformshift{3.988090in}{0.722700in}%
\pgfsys@useobject{currentmarker}{}%
\end{pgfscope}%
\end{pgfscope}%
\begin{pgfscope}%
\pgftext[x=3.988090in,y=0.625478in,,top]{\sffamily\fontsize{10.000000}{12.000000}\selectfont 20}%
\end{pgfscope}%
\begin{pgfscope}%
\pgfsetbuttcap%
\pgfsetroundjoin%
\definecolor{currentfill}{rgb}{0.000000,0.000000,0.000000}%
\pgfsetfillcolor{currentfill}%
\pgfsetlinewidth{0.803000pt}%
\definecolor{currentstroke}{rgb}{0.000000,0.000000,0.000000}%
\pgfsetstrokecolor{currentstroke}%
\pgfsetdash{}{0pt}%
\pgfsys@defobject{currentmarker}{\pgfqpoint{0.000000in}{-0.048611in}}{\pgfqpoint{0.000000in}{0.000000in}}{%
\pgfpathmoveto{\pgfqpoint{0.000000in}{0.000000in}}%
\pgfpathlineto{\pgfqpoint{0.000000in}{-0.048611in}}%
\pgfusepath{stroke,fill}%
}%
\begin{pgfscope}%
\pgfsys@transformshift{4.558238in}{0.722700in}%
\pgfsys@useobject{currentmarker}{}%
\end{pgfscope}%
\end{pgfscope}%
\begin{pgfscope}%
\pgftext[x=4.558238in,y=0.625478in,,top]{\sffamily\fontsize{10.000000}{12.000000}\selectfont 25}%
\end{pgfscope}%
\begin{pgfscope}%
\pgfsetbuttcap%
\pgfsetroundjoin%
\definecolor{currentfill}{rgb}{0.000000,0.000000,0.000000}%
\pgfsetfillcolor{currentfill}%
\pgfsetlinewidth{0.803000pt}%
\definecolor{currentstroke}{rgb}{0.000000,0.000000,0.000000}%
\pgfsetstrokecolor{currentstroke}%
\pgfsetdash{}{0pt}%
\pgfsys@defobject{currentmarker}{\pgfqpoint{0.000000in}{-0.048611in}}{\pgfqpoint{0.000000in}{0.000000in}}{%
\pgfpathmoveto{\pgfqpoint{0.000000in}{0.000000in}}%
\pgfpathlineto{\pgfqpoint{0.000000in}{-0.048611in}}%
\pgfusepath{stroke,fill}%
}%
\begin{pgfscope}%
\pgfsys@transformshift{5.128385in}{0.722700in}%
\pgfsys@useobject{currentmarker}{}%
\end{pgfscope}%
\end{pgfscope}%
\begin{pgfscope}%
\pgftext[x=5.128385in,y=0.625478in,,top]{\sffamily\fontsize{10.000000}{12.000000}\selectfont 30}%
\end{pgfscope}%
\begin{pgfscope}%
\pgfsetbuttcap%
\pgfsetroundjoin%
\definecolor{currentfill}{rgb}{0.000000,0.000000,0.000000}%
\pgfsetfillcolor{currentfill}%
\pgfsetlinewidth{0.803000pt}%
\definecolor{currentstroke}{rgb}{0.000000,0.000000,0.000000}%
\pgfsetstrokecolor{currentstroke}%
\pgfsetdash{}{0pt}%
\pgfsys@defobject{currentmarker}{\pgfqpoint{0.000000in}{-0.048611in}}{\pgfqpoint{0.000000in}{0.000000in}}{%
\pgfpathmoveto{\pgfqpoint{0.000000in}{0.000000in}}%
\pgfpathlineto{\pgfqpoint{0.000000in}{-0.048611in}}%
\pgfusepath{stroke,fill}%
}%
\begin{pgfscope}%
\pgfsys@transformshift{5.698533in}{0.722700in}%
\pgfsys@useobject{currentmarker}{}%
\end{pgfscope}%
\end{pgfscope}%
\begin{pgfscope}%
\pgftext[x=5.698533in,y=0.625478in,,top]{\sffamily\fontsize{10.000000}{12.000000}\selectfont 35}%
\end{pgfscope}%
\begin{pgfscope}%
\pgfsetbuttcap%
\pgfsetroundjoin%
\definecolor{currentfill}{rgb}{0.000000,0.000000,0.000000}%
\pgfsetfillcolor{currentfill}%
\pgfsetlinewidth{0.803000pt}%
\definecolor{currentstroke}{rgb}{0.000000,0.000000,0.000000}%
\pgfsetstrokecolor{currentstroke}%
\pgfsetdash{}{0pt}%
\pgfsys@defobject{currentmarker}{\pgfqpoint{0.000000in}{-0.048611in}}{\pgfqpoint{0.000000in}{0.000000in}}{%
\pgfpathmoveto{\pgfqpoint{0.000000in}{0.000000in}}%
\pgfpathlineto{\pgfqpoint{0.000000in}{-0.048611in}}%
\pgfusepath{stroke,fill}%
}%
\begin{pgfscope}%
\pgfsys@transformshift{6.268681in}{0.722700in}%
\pgfsys@useobject{currentmarker}{}%
\end{pgfscope}%
\end{pgfscope}%
\begin{pgfscope}%
\pgftext[x=6.268681in,y=0.625478in,,top]{\sffamily\fontsize{10.000000}{12.000000}\selectfont 40}%
\end{pgfscope}%
\begin{pgfscope}%
\pgftext[x=4.113523in,y=0.435509in,,top]{\sffamily\fontsize{16.000000}{19.200000}\selectfont \(\displaystyle z-position [\mu m]\)}%
\end{pgfscope}%
\begin{pgfscope}%
\pgfsetbuttcap%
\pgfsetroundjoin%
\definecolor{currentfill}{rgb}{0.000000,0.000000,0.000000}%
\pgfsetfillcolor{currentfill}%
\pgfsetlinewidth{0.803000pt}%
\definecolor{currentstroke}{rgb}{0.000000,0.000000,0.000000}%
\pgfsetstrokecolor{currentstroke}%
\pgfsetdash{}{0pt}%
\pgfsys@defobject{currentmarker}{\pgfqpoint{-0.048611in}{0.000000in}}{\pgfqpoint{0.000000in}{0.000000in}}{%
\pgfpathmoveto{\pgfqpoint{0.000000in}{0.000000in}}%
\pgfpathlineto{\pgfqpoint{-0.048611in}{0.000000in}}%
\pgfusepath{stroke,fill}%
}%
\begin{pgfscope}%
\pgfsys@transformshift{1.707500in}{0.722700in}%
\pgfsys@useobject{currentmarker}{}%
\end{pgfscope}%
\end{pgfscope}%
\begin{pgfscope}%
\pgftext[x=1.273025in,y=0.669938in,left,base]{\sffamily\fontsize{10.000000}{12.000000}\selectfont -2.0}%
\end{pgfscope}%
\begin{pgfscope}%
\pgfsetbuttcap%
\pgfsetroundjoin%
\definecolor{currentfill}{rgb}{0.000000,0.000000,0.000000}%
\pgfsetfillcolor{currentfill}%
\pgfsetlinewidth{0.803000pt}%
\definecolor{currentstroke}{rgb}{0.000000,0.000000,0.000000}%
\pgfsetstrokecolor{currentstroke}%
\pgfsetdash{}{0pt}%
\pgfsys@defobject{currentmarker}{\pgfqpoint{-0.048611in}{0.000000in}}{\pgfqpoint{0.000000in}{0.000000in}}{%
\pgfpathmoveto{\pgfqpoint{0.000000in}{0.000000in}}%
\pgfpathlineto{\pgfqpoint{-0.048611in}{0.000000in}}%
\pgfusepath{stroke,fill}%
}%
\begin{pgfscope}%
\pgfsys@transformshift{1.707500in}{1.355062in}%
\pgfsys@useobject{currentmarker}{}%
\end{pgfscope}%
\end{pgfscope}%
\begin{pgfscope}%
\pgftext[x=1.273025in,y=1.302301in,left,base]{\sffamily\fontsize{10.000000}{12.000000}\selectfont -1.5}%
\end{pgfscope}%
\begin{pgfscope}%
\pgfsetbuttcap%
\pgfsetroundjoin%
\definecolor{currentfill}{rgb}{0.000000,0.000000,0.000000}%
\pgfsetfillcolor{currentfill}%
\pgfsetlinewidth{0.803000pt}%
\definecolor{currentstroke}{rgb}{0.000000,0.000000,0.000000}%
\pgfsetstrokecolor{currentstroke}%
\pgfsetdash{}{0pt}%
\pgfsys@defobject{currentmarker}{\pgfqpoint{-0.048611in}{0.000000in}}{\pgfqpoint{0.000000in}{0.000000in}}{%
\pgfpathmoveto{\pgfqpoint{0.000000in}{0.000000in}}%
\pgfpathlineto{\pgfqpoint{-0.048611in}{0.000000in}}%
\pgfusepath{stroke,fill}%
}%
\begin{pgfscope}%
\pgfsys@transformshift{1.707500in}{1.987425in}%
\pgfsys@useobject{currentmarker}{}%
\end{pgfscope}%
\end{pgfscope}%
\begin{pgfscope}%
\pgftext[x=1.273025in,y=1.934663in,left,base]{\sffamily\fontsize{10.000000}{12.000000}\selectfont -1.0}%
\end{pgfscope}%
\begin{pgfscope}%
\pgfsetbuttcap%
\pgfsetroundjoin%
\definecolor{currentfill}{rgb}{0.000000,0.000000,0.000000}%
\pgfsetfillcolor{currentfill}%
\pgfsetlinewidth{0.803000pt}%
\definecolor{currentstroke}{rgb}{0.000000,0.000000,0.000000}%
\pgfsetstrokecolor{currentstroke}%
\pgfsetdash{}{0pt}%
\pgfsys@defobject{currentmarker}{\pgfqpoint{-0.048611in}{0.000000in}}{\pgfqpoint{0.000000in}{0.000000in}}{%
\pgfpathmoveto{\pgfqpoint{0.000000in}{0.000000in}}%
\pgfpathlineto{\pgfqpoint{-0.048611in}{0.000000in}}%
\pgfusepath{stroke,fill}%
}%
\begin{pgfscope}%
\pgfsys@transformshift{1.707500in}{2.619788in}%
\pgfsys@useobject{currentmarker}{}%
\end{pgfscope}%
\end{pgfscope}%
\begin{pgfscope}%
\pgftext[x=1.273025in,y=2.567026in,left,base]{\sffamily\fontsize{10.000000}{12.000000}\selectfont -0.5}%
\end{pgfscope}%
\begin{pgfscope}%
\pgfsetbuttcap%
\pgfsetroundjoin%
\definecolor{currentfill}{rgb}{0.000000,0.000000,0.000000}%
\pgfsetfillcolor{currentfill}%
\pgfsetlinewidth{0.803000pt}%
\definecolor{currentstroke}{rgb}{0.000000,0.000000,0.000000}%
\pgfsetstrokecolor{currentstroke}%
\pgfsetdash{}{0pt}%
\pgfsys@defobject{currentmarker}{\pgfqpoint{-0.048611in}{0.000000in}}{\pgfqpoint{0.000000in}{0.000000in}}{%
\pgfpathmoveto{\pgfqpoint{0.000000in}{0.000000in}}%
\pgfpathlineto{\pgfqpoint{-0.048611in}{0.000000in}}%
\pgfusepath{stroke,fill}%
}%
\begin{pgfscope}%
\pgfsys@transformshift{1.707500in}{3.252150in}%
\pgfsys@useobject{currentmarker}{}%
\end{pgfscope}%
\end{pgfscope}%
\begin{pgfscope}%
\pgftext[x=1.389398in,y=3.199388in,left,base]{\sffamily\fontsize{10.000000}{12.000000}\selectfont 0.0}%
\end{pgfscope}%
\begin{pgfscope}%
\pgfsetbuttcap%
\pgfsetroundjoin%
\definecolor{currentfill}{rgb}{0.000000,0.000000,0.000000}%
\pgfsetfillcolor{currentfill}%
\pgfsetlinewidth{0.803000pt}%
\definecolor{currentstroke}{rgb}{0.000000,0.000000,0.000000}%
\pgfsetstrokecolor{currentstroke}%
\pgfsetdash{}{0pt}%
\pgfsys@defobject{currentmarker}{\pgfqpoint{-0.048611in}{0.000000in}}{\pgfqpoint{0.000000in}{0.000000in}}{%
\pgfpathmoveto{\pgfqpoint{0.000000in}{0.000000in}}%
\pgfpathlineto{\pgfqpoint{-0.048611in}{0.000000in}}%
\pgfusepath{stroke,fill}%
}%
\begin{pgfscope}%
\pgfsys@transformshift{1.707500in}{3.884512in}%
\pgfsys@useobject{currentmarker}{}%
\end{pgfscope}%
\end{pgfscope}%
\begin{pgfscope}%
\pgftext[x=1.389398in,y=3.831751in,left,base]{\sffamily\fontsize{10.000000}{12.000000}\selectfont 0.5}%
\end{pgfscope}%
\begin{pgfscope}%
\pgfsetbuttcap%
\pgfsetroundjoin%
\definecolor{currentfill}{rgb}{0.000000,0.000000,0.000000}%
\pgfsetfillcolor{currentfill}%
\pgfsetlinewidth{0.803000pt}%
\definecolor{currentstroke}{rgb}{0.000000,0.000000,0.000000}%
\pgfsetstrokecolor{currentstroke}%
\pgfsetdash{}{0pt}%
\pgfsys@defobject{currentmarker}{\pgfqpoint{-0.048611in}{0.000000in}}{\pgfqpoint{0.000000in}{0.000000in}}{%
\pgfpathmoveto{\pgfqpoint{0.000000in}{0.000000in}}%
\pgfpathlineto{\pgfqpoint{-0.048611in}{0.000000in}}%
\pgfusepath{stroke,fill}%
}%
\begin{pgfscope}%
\pgfsys@transformshift{1.707500in}{4.516875in}%
\pgfsys@useobject{currentmarker}{}%
\end{pgfscope}%
\end{pgfscope}%
\begin{pgfscope}%
\pgftext[x=1.389398in,y=4.464113in,left,base]{\sffamily\fontsize{10.000000}{12.000000}\selectfont 1.0}%
\end{pgfscope}%
\begin{pgfscope}%
\pgfsetbuttcap%
\pgfsetroundjoin%
\definecolor{currentfill}{rgb}{0.000000,0.000000,0.000000}%
\pgfsetfillcolor{currentfill}%
\pgfsetlinewidth{0.803000pt}%
\definecolor{currentstroke}{rgb}{0.000000,0.000000,0.000000}%
\pgfsetstrokecolor{currentstroke}%
\pgfsetdash{}{0pt}%
\pgfsys@defobject{currentmarker}{\pgfqpoint{-0.048611in}{0.000000in}}{\pgfqpoint{0.000000in}{0.000000in}}{%
\pgfpathmoveto{\pgfqpoint{0.000000in}{0.000000in}}%
\pgfpathlineto{\pgfqpoint{-0.048611in}{0.000000in}}%
\pgfusepath{stroke,fill}%
}%
\begin{pgfscope}%
\pgfsys@transformshift{1.707500in}{5.149237in}%
\pgfsys@useobject{currentmarker}{}%
\end{pgfscope}%
\end{pgfscope}%
\begin{pgfscope}%
\pgftext[x=1.389398in,y=5.096476in,left,base]{\sffamily\fontsize{10.000000}{12.000000}\selectfont 1.5}%
\end{pgfscope}%
\begin{pgfscope}%
\pgfsetbuttcap%
\pgfsetroundjoin%
\definecolor{currentfill}{rgb}{0.000000,0.000000,0.000000}%
\pgfsetfillcolor{currentfill}%
\pgfsetlinewidth{0.803000pt}%
\definecolor{currentstroke}{rgb}{0.000000,0.000000,0.000000}%
\pgfsetstrokecolor{currentstroke}%
\pgfsetdash{}{0pt}%
\pgfsys@defobject{currentmarker}{\pgfqpoint{-0.048611in}{0.000000in}}{\pgfqpoint{0.000000in}{0.000000in}}{%
\pgfpathmoveto{\pgfqpoint{0.000000in}{0.000000in}}%
\pgfpathlineto{\pgfqpoint{-0.048611in}{0.000000in}}%
\pgfusepath{stroke,fill}%
}%
\begin{pgfscope}%
\pgfsys@transformshift{1.707500in}{5.781600in}%
\pgfsys@useobject{currentmarker}{}%
\end{pgfscope}%
\end{pgfscope}%
\begin{pgfscope}%
\pgftext[x=1.389398in,y=5.728838in,left,base]{\sffamily\fontsize{10.000000}{12.000000}\selectfont 2.0}%
\end{pgfscope}%
\begin{pgfscope}%
\pgftext[x=1.217469in,y=3.252150in,,bottom,rotate=90.000000]{\sffamily\fontsize{16.000000}{19.200000}\selectfont \(\displaystyle E-field\)}%
\end{pgfscope}%
\begin{pgfscope}%
\pgfpathrectangle{\pgfqpoint{1.707500in}{0.722700in}}{\pgfqpoint{4.812045in}{5.058900in}} %
\pgfusepath{clip}%
\pgfsetrectcap%
\pgfsetroundjoin%
\pgfsetlinewidth{1.505625pt}%
\definecolor{currentstroke}{rgb}{0.000000,0.000000,0.000000}%
\pgfsetstrokecolor{currentstroke}%
\pgfsetdash{}{0pt}%
\pgfpathmoveto{\pgfqpoint{1.707500in}{3.252150in}}%
\pgfpathlineto{\pgfqpoint{4.852486in}{3.251112in}}%
\pgfpathlineto{\pgfqpoint{4.857925in}{3.247322in}}%
\pgfpathlineto{\pgfqpoint{4.864202in}{3.239927in}}%
\pgfpathlineto{\pgfqpoint{4.873407in}{3.228921in}}%
\pgfpathlineto{\pgfqpoint{4.876337in}{3.228357in}}%
\pgfpathlineto{\pgfqpoint{4.878847in}{3.230282in}}%
\pgfpathlineto{\pgfqpoint{4.881776in}{3.236205in}}%
\pgfpathlineto{\pgfqpoint{4.885124in}{3.248777in}}%
\pgfpathlineto{\pgfqpoint{4.889308in}{3.273939in}}%
\pgfpathlineto{\pgfqpoint{4.895585in}{3.328108in}}%
\pgfpathlineto{\pgfqpoint{4.904790in}{3.406002in}}%
\pgfpathlineto{\pgfqpoint{4.907719in}{3.416624in}}%
\pgfpathlineto{\pgfqpoint{4.909393in}{3.417133in}}%
\pgfpathlineto{\pgfqpoint{4.911067in}{3.412883in}}%
\pgfpathlineto{\pgfqpoint{4.913159in}{3.400182in}}%
\pgfpathlineto{\pgfqpoint{4.916088in}{3.367582in}}%
\pgfpathlineto{\pgfqpoint{4.920273in}{3.290722in}}%
\pgfpathlineto{\pgfqpoint{4.926131in}{3.134000in}}%
\pgfpathlineto{\pgfqpoint{4.937429in}{2.823127in}}%
\pgfpathlineto{\pgfqpoint{4.940776in}{2.779500in}}%
\pgfpathlineto{\pgfqpoint{4.942868in}{2.772595in}}%
\pgfpathlineto{\pgfqpoint{4.944124in}{2.776906in}}%
\pgfpathlineto{\pgfqpoint{4.946216in}{2.798998in}}%
\pgfpathlineto{\pgfqpoint{4.949145in}{2.861979in}}%
\pgfpathlineto{\pgfqpoint{4.953329in}{3.014032in}}%
\pgfpathlineto{\pgfqpoint{4.959606in}{3.346083in}}%
\pgfpathlineto{\pgfqpoint{4.970904in}{3.952558in}}%
\pgfpathlineto{\pgfqpoint{4.974670in}{4.056865in}}%
\pgfpathlineto{\pgfqpoint{4.977180in}{4.080278in}}%
\pgfpathlineto{\pgfqpoint{4.977599in}{4.080257in}}%
\pgfpathlineto{\pgfqpoint{4.978436in}{4.076756in}}%
\pgfpathlineto{\pgfqpoint{4.980109in}{4.055757in}}%
\pgfpathlineto{\pgfqpoint{4.982620in}{3.989297in}}%
\pgfpathlineto{\pgfqpoint{4.986386in}{3.815052in}}%
\pgfpathlineto{\pgfqpoint{4.991826in}{3.436450in}}%
\pgfpathlineto{\pgfqpoint{5.006889in}{2.316276in}}%
\pgfpathlineto{\pgfqpoint{5.010655in}{2.198770in}}%
\pgfpathlineto{\pgfqpoint{5.012748in}{2.181266in}}%
\pgfpathlineto{\pgfqpoint{5.013584in}{2.184326in}}%
\pgfpathlineto{\pgfqpoint{5.015258in}{2.207857in}}%
\pgfpathlineto{\pgfqpoint{5.017769in}{2.286058in}}%
\pgfpathlineto{\pgfqpoint{5.021535in}{2.493146in}}%
\pgfpathlineto{\pgfqpoint{5.026974in}{2.941496in}}%
\pgfpathlineto{\pgfqpoint{5.042038in}{4.265311in}}%
\pgfpathlineto{\pgfqpoint{5.045804in}{4.414647in}}%
\pgfpathlineto{\pgfqpoint{5.048315in}{4.446206in}}%
\pgfpathlineto{\pgfqpoint{5.048733in}{4.445922in}}%
\pgfpathlineto{\pgfqpoint{5.049989in}{4.435494in}}%
\pgfpathlineto{\pgfqpoint{5.052081in}{4.386476in}}%
\pgfpathlineto{\pgfqpoint{5.055010in}{4.254195in}}%
\pgfpathlineto{\pgfqpoint{5.059194in}{3.952616in}}%
\pgfpathlineto{\pgfqpoint{5.066308in}{3.246572in}}%
\pgfpathlineto{\pgfqpoint{5.075932in}{2.329659in}}%
\pgfpathlineto{\pgfqpoint{5.080535in}{2.077640in}}%
\pgfpathlineto{\pgfqpoint{5.083464in}{2.012441in}}%
\pgfpathlineto{\pgfqpoint{5.084301in}{2.008490in}}%
\pgfpathlineto{\pgfqpoint{5.084719in}{2.008994in}}%
\pgfpathlineto{\pgfqpoint{5.085974in}{2.020422in}}%
\pgfpathlineto{\pgfqpoint{5.088067in}{2.072126in}}%
\pgfpathlineto{\pgfqpoint{5.090996in}{2.209897in}}%
\pgfpathlineto{\pgfqpoint{5.095180in}{2.521599in}}%
\pgfpathlineto{\pgfqpoint{5.102712in}{3.292120in}}%
\pgfpathlineto{\pgfqpoint{5.111918in}{4.182898in}}%
\pgfpathlineto{\pgfqpoint{5.116520in}{4.440254in}}%
\pgfpathlineto{\pgfqpoint{5.119449in}{4.507507in}}%
\pgfpathlineto{\pgfqpoint{5.120286in}{4.511869in}}%
\pgfpathlineto{\pgfqpoint{5.120705in}{4.511542in}}%
\pgfpathlineto{\pgfqpoint{5.121960in}{4.500540in}}%
\pgfpathlineto{\pgfqpoint{5.124052in}{4.449208in}}%
\pgfpathlineto{\pgfqpoint{5.126981in}{4.311326in}}%
\pgfpathlineto{\pgfqpoint{5.131166in}{3.998404in}}%
\pgfpathlineto{\pgfqpoint{5.138279in}{3.269395in}}%
\pgfpathlineto{\pgfqpoint{5.147903in}{2.324888in}}%
\pgfpathlineto{\pgfqpoint{5.152506in}{2.063358in}}%
\pgfpathlineto{\pgfqpoint{5.155435in}{1.993539in}}%
\pgfpathlineto{\pgfqpoint{5.156690in}{1.988454in}}%
\pgfpathlineto{\pgfqpoint{5.157527in}{1.993443in}}%
\pgfpathlineto{\pgfqpoint{5.159201in}{2.023398in}}%
\pgfpathlineto{\pgfqpoint{5.161712in}{2.116819in}}%
\pgfpathlineto{\pgfqpoint{5.165478in}{2.356172in}}%
\pgfpathlineto{\pgfqpoint{5.171336in}{2.905485in}}%
\pgfpathlineto{\pgfqpoint{5.185144in}{4.261514in}}%
\pgfpathlineto{\pgfqpoint{5.189329in}{4.466717in}}%
\pgfpathlineto{\pgfqpoint{5.191839in}{4.514219in}}%
\pgfpathlineto{\pgfqpoint{5.192676in}{4.516740in}}%
\pgfpathlineto{\pgfqpoint{5.193095in}{4.515485in}}%
\pgfpathlineto{\pgfqpoint{5.194350in}{4.501679in}}%
\pgfpathlineto{\pgfqpoint{5.196442in}{4.445708in}}%
\pgfpathlineto{\pgfqpoint{5.199790in}{4.275295in}}%
\pgfpathlineto{\pgfqpoint{5.204393in}{3.904337in}}%
\pgfpathlineto{\pgfqpoint{5.214017in}{2.883302in}}%
\pgfpathlineto{\pgfqpoint{5.221130in}{2.249659in}}%
\pgfpathlineto{\pgfqpoint{5.225314in}{2.040745in}}%
\pgfpathlineto{\pgfqpoint{5.227825in}{1.990783in}}%
\pgfpathlineto{\pgfqpoint{5.228662in}{1.987428in}}%
\pgfpathlineto{\pgfqpoint{5.229080in}{1.988267in}}%
\pgfpathlineto{\pgfqpoint{5.230336in}{2.000827in}}%
\pgfpathlineto{\pgfqpoint{5.232428in}{2.054777in}}%
\pgfpathlineto{\pgfqpoint{5.235357in}{2.196193in}}%
\pgfpathlineto{\pgfqpoint{5.239960in}{2.551303in}}%
\pgfpathlineto{\pgfqpoint{5.248329in}{3.430066in}}%
\pgfpathlineto{\pgfqpoint{5.256279in}{4.188161in}}%
\pgfpathlineto{\pgfqpoint{5.260882in}{4.445767in}}%
\pgfpathlineto{\pgfqpoint{5.263811in}{4.512630in}}%
\pgfpathlineto{\pgfqpoint{5.264648in}{4.516823in}}%
\pgfpathlineto{\pgfqpoint{5.265066in}{4.516403in}}%
\pgfpathlineto{\pgfqpoint{5.266321in}{4.505095in}}%
\pgfpathlineto{\pgfqpoint{5.268414in}{4.453183in}}%
\pgfpathlineto{\pgfqpoint{5.271343in}{4.314403in}}%
\pgfpathlineto{\pgfqpoint{5.275527in}{4.000188in}}%
\pgfpathlineto{\pgfqpoint{5.283059in}{3.223361in}}%
\pgfpathlineto{\pgfqpoint{5.292265in}{2.323923in}}%
\pgfpathlineto{\pgfqpoint{5.296867in}{2.062390in}}%
\pgfpathlineto{\pgfqpoint{5.299797in}{1.992665in}}%
\pgfpathlineto{\pgfqpoint{5.301052in}{1.987635in}}%
\pgfpathlineto{\pgfqpoint{5.301889in}{1.992665in}}%
\pgfpathlineto{\pgfqpoint{5.303562in}{2.022706in}}%
\pgfpathlineto{\pgfqpoint{5.306073in}{2.116265in}}%
\pgfpathlineto{\pgfqpoint{5.309839in}{2.355821in}}%
\pgfpathlineto{\pgfqpoint{5.315697in}{2.905390in}}%
\pgfpathlineto{\pgfqpoint{5.329506in}{4.261626in}}%
\pgfpathlineto{\pgfqpoint{5.333690in}{4.466818in}}%
\pgfpathlineto{\pgfqpoint{5.336201in}{4.514307in}}%
\pgfpathlineto{\pgfqpoint{5.337038in}{4.516823in}}%
\pgfpathlineto{\pgfqpoint{5.337456in}{4.515564in}}%
\pgfpathlineto{\pgfqpoint{5.338711in}{4.501751in}}%
\pgfpathlineto{\pgfqpoint{5.340804in}{4.445767in}}%
\pgfpathlineto{\pgfqpoint{5.344151in}{4.275334in}}%
\pgfpathlineto{\pgfqpoint{5.348754in}{3.904354in}}%
\pgfpathlineto{\pgfqpoint{5.358378in}{2.883298in}}%
\pgfpathlineto{\pgfqpoint{5.365491in}{2.249654in}}%
\pgfpathlineto{\pgfqpoint{5.369676in}{2.040742in}}%
\pgfpathlineto{\pgfqpoint{5.372186in}{1.990782in}}%
\pgfpathlineto{\pgfqpoint{5.373023in}{1.987428in}}%
\pgfpathlineto{\pgfqpoint{5.373442in}{1.988267in}}%
\pgfpathlineto{\pgfqpoint{5.374697in}{2.000829in}}%
\pgfpathlineto{\pgfqpoint{5.376789in}{2.054780in}}%
\pgfpathlineto{\pgfqpoint{5.379718in}{2.196197in}}%
\pgfpathlineto{\pgfqpoint{5.384321in}{2.551309in}}%
\pgfpathlineto{\pgfqpoint{5.392690in}{3.430063in}}%
\pgfpathlineto{\pgfqpoint{5.400640in}{4.188129in}}%
\pgfpathlineto{\pgfqpoint{5.405243in}{4.445708in}}%
\pgfpathlineto{\pgfqpoint{5.408172in}{4.512553in}}%
\pgfpathlineto{\pgfqpoint{5.409009in}{4.516740in}}%
\pgfpathlineto{\pgfqpoint{5.409427in}{4.516318in}}%
\pgfpathlineto{\pgfqpoint{5.410683in}{4.505003in}}%
\pgfpathlineto{\pgfqpoint{5.412775in}{4.453080in}}%
\pgfpathlineto{\pgfqpoint{5.415704in}{4.314291in}}%
\pgfpathlineto{\pgfqpoint{5.419888in}{4.000083in}}%
\pgfpathlineto{\pgfqpoint{5.427420in}{3.223367in}}%
\pgfpathlineto{\pgfqpoint{5.436626in}{2.324295in}}%
\pgfpathlineto{\pgfqpoint{5.441229in}{2.063013in}}%
\pgfpathlineto{\pgfqpoint{5.444158in}{1.993443in}}%
\pgfpathlineto{\pgfqpoint{5.445413in}{1.988473in}}%
\pgfpathlineto{\pgfqpoint{5.446250in}{1.993539in}}%
\pgfpathlineto{\pgfqpoint{5.447924in}{2.023641in}}%
\pgfpathlineto{\pgfqpoint{5.450434in}{2.117255in}}%
\pgfpathlineto{\pgfqpoint{5.454200in}{2.356773in}}%
\pgfpathlineto{\pgfqpoint{5.460059in}{2.905887in}}%
\pgfpathlineto{\pgfqpoint{5.473867in}{4.258813in}}%
\pgfpathlineto{\pgfqpoint{5.478051in}{4.462721in}}%
\pgfpathlineto{\pgfqpoint{5.480562in}{4.509543in}}%
\pgfpathlineto{\pgfqpoint{5.481399in}{4.511869in}}%
\pgfpathlineto{\pgfqpoint{5.481817in}{4.510524in}}%
\pgfpathlineto{\pgfqpoint{5.483073in}{4.496483in}}%
\pgfpathlineto{\pgfqpoint{5.485165in}{4.440254in}}%
\pgfpathlineto{\pgfqpoint{5.488512in}{4.269879in}}%
\pgfpathlineto{\pgfqpoint{5.493115in}{3.900135in}}%
\pgfpathlineto{\pgfqpoint{5.503158in}{2.843493in}}%
\pgfpathlineto{\pgfqpoint{5.509853in}{2.262258in}}%
\pgfpathlineto{\pgfqpoint{5.514037in}{2.058571in}}%
\pgfpathlineto{\pgfqpoint{5.516548in}{2.011153in}}%
\pgfpathlineto{\pgfqpoint{5.517385in}{2.008490in}}%
\pgfpathlineto{\pgfqpoint{5.517803in}{2.009640in}}%
\pgfpathlineto{\pgfqpoint{5.519058in}{2.022979in}}%
\pgfpathlineto{\pgfqpoint{5.521151in}{2.077640in}}%
\pgfpathlineto{\pgfqpoint{5.524498in}{2.244400in}}%
\pgfpathlineto{\pgfqpoint{5.529101in}{2.606841in}}%
\pgfpathlineto{\pgfqpoint{5.539562in}{3.681296in}}%
\pgfpathlineto{\pgfqpoint{5.546257in}{4.229611in}}%
\pgfpathlineto{\pgfqpoint{5.550441in}{4.410778in}}%
\pgfpathlineto{\pgfqpoint{5.552952in}{4.445922in}}%
\pgfpathlineto{\pgfqpoint{5.553370in}{4.446206in}}%
\pgfpathlineto{\pgfqpoint{5.554207in}{4.441995in}}%
\pgfpathlineto{\pgfqpoint{5.555881in}{4.414647in}}%
\pgfpathlineto{\pgfqpoint{5.558392in}{4.327955in}}%
\pgfpathlineto{\pgfqpoint{5.562158in}{4.105552in}}%
\pgfpathlineto{\pgfqpoint{5.568016in}{3.599239in}}%
\pgfpathlineto{\pgfqpoint{5.580987in}{2.438548in}}%
\pgfpathlineto{\pgfqpoint{5.585172in}{2.240624in}}%
\pgfpathlineto{\pgfqpoint{5.588101in}{2.184326in}}%
\pgfpathlineto{\pgfqpoint{5.588938in}{2.181266in}}%
\pgfpathlineto{\pgfqpoint{5.589356in}{2.181906in}}%
\pgfpathlineto{\pgfqpoint{5.590611in}{2.192429in}}%
\pgfpathlineto{\pgfqpoint{5.592704in}{2.237911in}}%
\pgfpathlineto{\pgfqpoint{5.596051in}{2.377722in}}%
\pgfpathlineto{\pgfqpoint{5.601072in}{2.710278in}}%
\pgfpathlineto{\pgfqpoint{5.618228in}{3.958029in}}%
\pgfpathlineto{\pgfqpoint{5.621994in}{4.062763in}}%
\pgfpathlineto{\pgfqpoint{5.624087in}{4.080257in}}%
\pgfpathlineto{\pgfqpoint{5.624505in}{4.080278in}}%
\pgfpathlineto{\pgfqpoint{5.625342in}{4.076904in}}%
\pgfpathlineto{\pgfqpoint{5.627016in}{4.056865in}}%
\pgfpathlineto{\pgfqpoint{5.629526in}{3.995738in}}%
\pgfpathlineto{\pgfqpoint{5.633292in}{3.844894in}}%
\pgfpathlineto{\pgfqpoint{5.640406in}{3.445778in}}%
\pgfpathlineto{\pgfqpoint{5.649193in}{2.978222in}}%
\pgfpathlineto{\pgfqpoint{5.653796in}{2.830412in}}%
\pgfpathlineto{\pgfqpoint{5.657143in}{2.779815in}}%
\pgfpathlineto{\pgfqpoint{5.659235in}{2.772598in}}%
\pgfpathlineto{\pgfqpoint{5.660491in}{2.776769in}}%
\pgfpathlineto{\pgfqpoint{5.662583in}{2.796715in}}%
\pgfpathlineto{\pgfqpoint{5.665930in}{2.857442in}}%
\pgfpathlineto{\pgfqpoint{5.671370in}{3.005864in}}%
\pgfpathlineto{\pgfqpoint{5.682668in}{3.317379in}}%
\pgfpathlineto{\pgfqpoint{5.687271in}{3.388374in}}%
\pgfpathlineto{\pgfqpoint{5.690618in}{3.412883in}}%
\pgfpathlineto{\pgfqpoint{5.692711in}{3.417432in}}%
\pgfpathlineto{\pgfqpoint{5.693966in}{3.416624in}}%
\pgfpathlineto{\pgfqpoint{5.695640in}{3.411915in}}%
\pgfpathlineto{\pgfqpoint{5.698569in}{3.395480in}}%
\pgfpathlineto{\pgfqpoint{5.704008in}{3.348270in}}%
\pgfpathlineto{\pgfqpoint{5.713632in}{3.265302in}}%
\pgfpathlineto{\pgfqpoint{5.718654in}{3.240151in}}%
\pgfpathlineto{\pgfqpoint{5.722420in}{3.230867in}}%
\pgfpathlineto{\pgfqpoint{5.725349in}{3.228357in}}%
\pgfpathlineto{\pgfqpoint{5.727859in}{3.228691in}}%
\pgfpathlineto{\pgfqpoint{5.731625in}{3.232024in}}%
\pgfpathlineto{\pgfqpoint{5.748363in}{3.250690in}}%
\pgfpathlineto{\pgfqpoint{5.754639in}{3.252713in}}%
\pgfpathlineto{\pgfqpoint{5.763845in}{3.252857in}}%
\pgfpathlineto{\pgfqpoint{5.797320in}{3.252148in}}%
\pgfpathlineto{\pgfqpoint{6.519545in}{3.252150in}}%
\pgfpathlineto{\pgfqpoint{6.519545in}{3.252150in}}%
\pgfusepath{stroke}%
\end{pgfscope}%
\begin{pgfscope}%
\pgfsetrectcap%
\pgfsetmiterjoin%
\pgfsetlinewidth{0.803000pt}%
\definecolor{currentstroke}{rgb}{0.000000,0.000000,0.000000}%
\pgfsetstrokecolor{currentstroke}%
\pgfsetdash{}{0pt}%
\pgfpathmoveto{\pgfqpoint{1.707500in}{0.722700in}}%
\pgfpathlineto{\pgfqpoint{1.707500in}{5.781600in}}%
\pgfusepath{stroke}%
\end{pgfscope}%
\begin{pgfscope}%
\pgfsetrectcap%
\pgfsetmiterjoin%
\pgfsetlinewidth{0.803000pt}%
\definecolor{currentstroke}{rgb}{0.000000,0.000000,0.000000}%
\pgfsetstrokecolor{currentstroke}%
\pgfsetdash{}{0pt}%
\pgfpathmoveto{\pgfqpoint{6.519545in}{0.722700in}}%
\pgfpathlineto{\pgfqpoint{6.519545in}{5.781600in}}%
\pgfusepath{stroke}%
\end{pgfscope}%
\begin{pgfscope}%
\pgfsetrectcap%
\pgfsetmiterjoin%
\pgfsetlinewidth{0.803000pt}%
\definecolor{currentstroke}{rgb}{0.000000,0.000000,0.000000}%
\pgfsetstrokecolor{currentstroke}%
\pgfsetdash{}{0pt}%
\pgfpathmoveto{\pgfqpoint{1.707500in}{0.722700in}}%
\pgfpathlineto{\pgfqpoint{6.519545in}{0.722700in}}%
\pgfusepath{stroke}%
\end{pgfscope}%
\begin{pgfscope}%
\pgfsetrectcap%
\pgfsetmiterjoin%
\pgfsetlinewidth{0.803000pt}%
\definecolor{currentstroke}{rgb}{0.000000,0.000000,0.000000}%
\pgfsetstrokecolor{currentstroke}%
\pgfsetdash{}{0pt}%
\pgfpathmoveto{\pgfqpoint{1.707500in}{5.781600in}}%
\pgfpathlineto{\pgfqpoint{6.519545in}{5.781600in}}%
\pgfusepath{stroke}%
\end{pgfscope}%
\begin{pgfscope}%
\pgftext[x=1.707500in,y=6.034545in,left,base]{\sffamily\fontsize{10.000000}{12.000000}\selectfont Iterations: 16915, Time: 0.207 ps, imp: 377 ohm}%
\end{pgfscope}%
\begin{pgfscope}%
\pgfsetbuttcap%
\pgfsetmiterjoin%
\definecolor{currentfill}{rgb}{1.000000,1.000000,1.000000}%
\pgfsetfillcolor{currentfill}%
\pgfsetlinewidth{0.000000pt}%
\definecolor{currentstroke}{rgb}{0.000000,0.000000,0.000000}%
\pgfsetstrokecolor{currentstroke}%
\pgfsetstrokeopacity{0.000000}%
\pgfsetdash{}{0pt}%
\pgfpathmoveto{\pgfqpoint{7.481955in}{0.722700in}}%
\pgfpathlineto{\pgfqpoint{12.294000in}{0.722700in}}%
\pgfpathlineto{\pgfqpoint{12.294000in}{5.781600in}}%
\pgfpathlineto{\pgfqpoint{7.481955in}{5.781600in}}%
\pgfpathclose%
\pgfusepath{fill}%
\end{pgfscope}%
\begin{pgfscope}%
\pgfsetbuttcap%
\pgfsetroundjoin%
\definecolor{currentfill}{rgb}{0.000000,0.000000,0.000000}%
\pgfsetfillcolor{currentfill}%
\pgfsetlinewidth{0.803000pt}%
\definecolor{currentstroke}{rgb}{0.000000,0.000000,0.000000}%
\pgfsetstrokecolor{currentstroke}%
\pgfsetdash{}{0pt}%
\pgfsys@defobject{currentmarker}{\pgfqpoint{0.000000in}{-0.048611in}}{\pgfqpoint{0.000000in}{0.000000in}}{%
\pgfpathmoveto{\pgfqpoint{0.000000in}{0.000000in}}%
\pgfpathlineto{\pgfqpoint{0.000000in}{-0.048611in}}%
\pgfusepath{stroke,fill}%
}%
\begin{pgfscope}%
\pgfsys@transformshift{7.481955in}{0.722700in}%
\pgfsys@useobject{currentmarker}{}%
\end{pgfscope}%
\end{pgfscope}%
\begin{pgfscope}%
\pgftext[x=7.481955in,y=0.625478in,,top]{\sffamily\fontsize{10.000000}{12.000000}\selectfont 0}%
\end{pgfscope}%
\begin{pgfscope}%
\pgfsetbuttcap%
\pgfsetroundjoin%
\definecolor{currentfill}{rgb}{0.000000,0.000000,0.000000}%
\pgfsetfillcolor{currentfill}%
\pgfsetlinewidth{0.803000pt}%
\definecolor{currentstroke}{rgb}{0.000000,0.000000,0.000000}%
\pgfsetstrokecolor{currentstroke}%
\pgfsetdash{}{0pt}%
\pgfsys@defobject{currentmarker}{\pgfqpoint{0.000000in}{-0.048611in}}{\pgfqpoint{0.000000in}{0.000000in}}{%
\pgfpathmoveto{\pgfqpoint{0.000000in}{0.000000in}}%
\pgfpathlineto{\pgfqpoint{0.000000in}{-0.048611in}}%
\pgfusepath{stroke,fill}%
}%
\begin{pgfscope}%
\pgfsys@transformshift{8.052102in}{0.722700in}%
\pgfsys@useobject{currentmarker}{}%
\end{pgfscope}%
\end{pgfscope}%
\begin{pgfscope}%
\pgftext[x=8.052102in,y=0.625478in,,top]{\sffamily\fontsize{10.000000}{12.000000}\selectfont 5}%
\end{pgfscope}%
\begin{pgfscope}%
\pgfsetbuttcap%
\pgfsetroundjoin%
\definecolor{currentfill}{rgb}{0.000000,0.000000,0.000000}%
\pgfsetfillcolor{currentfill}%
\pgfsetlinewidth{0.803000pt}%
\definecolor{currentstroke}{rgb}{0.000000,0.000000,0.000000}%
\pgfsetstrokecolor{currentstroke}%
\pgfsetdash{}{0pt}%
\pgfsys@defobject{currentmarker}{\pgfqpoint{0.000000in}{-0.048611in}}{\pgfqpoint{0.000000in}{0.000000in}}{%
\pgfpathmoveto{\pgfqpoint{0.000000in}{0.000000in}}%
\pgfpathlineto{\pgfqpoint{0.000000in}{-0.048611in}}%
\pgfusepath{stroke,fill}%
}%
\begin{pgfscope}%
\pgfsys@transformshift{8.622250in}{0.722700in}%
\pgfsys@useobject{currentmarker}{}%
\end{pgfscope}%
\end{pgfscope}%
\begin{pgfscope}%
\pgftext[x=8.622250in,y=0.625478in,,top]{\sffamily\fontsize{10.000000}{12.000000}\selectfont 10}%
\end{pgfscope}%
\begin{pgfscope}%
\pgfsetbuttcap%
\pgfsetroundjoin%
\definecolor{currentfill}{rgb}{0.000000,0.000000,0.000000}%
\pgfsetfillcolor{currentfill}%
\pgfsetlinewidth{0.803000pt}%
\definecolor{currentstroke}{rgb}{0.000000,0.000000,0.000000}%
\pgfsetstrokecolor{currentstroke}%
\pgfsetdash{}{0pt}%
\pgfsys@defobject{currentmarker}{\pgfqpoint{0.000000in}{-0.048611in}}{\pgfqpoint{0.000000in}{0.000000in}}{%
\pgfpathmoveto{\pgfqpoint{0.000000in}{0.000000in}}%
\pgfpathlineto{\pgfqpoint{0.000000in}{-0.048611in}}%
\pgfusepath{stroke,fill}%
}%
\begin{pgfscope}%
\pgfsys@transformshift{9.192397in}{0.722700in}%
\pgfsys@useobject{currentmarker}{}%
\end{pgfscope}%
\end{pgfscope}%
\begin{pgfscope}%
\pgftext[x=9.192397in,y=0.625478in,,top]{\sffamily\fontsize{10.000000}{12.000000}\selectfont 15}%
\end{pgfscope}%
\begin{pgfscope}%
\pgfsetbuttcap%
\pgfsetroundjoin%
\definecolor{currentfill}{rgb}{0.000000,0.000000,0.000000}%
\pgfsetfillcolor{currentfill}%
\pgfsetlinewidth{0.803000pt}%
\definecolor{currentstroke}{rgb}{0.000000,0.000000,0.000000}%
\pgfsetstrokecolor{currentstroke}%
\pgfsetdash{}{0pt}%
\pgfsys@defobject{currentmarker}{\pgfqpoint{0.000000in}{-0.048611in}}{\pgfqpoint{0.000000in}{0.000000in}}{%
\pgfpathmoveto{\pgfqpoint{0.000000in}{0.000000in}}%
\pgfpathlineto{\pgfqpoint{0.000000in}{-0.048611in}}%
\pgfusepath{stroke,fill}%
}%
\begin{pgfscope}%
\pgfsys@transformshift{9.762545in}{0.722700in}%
\pgfsys@useobject{currentmarker}{}%
\end{pgfscope}%
\end{pgfscope}%
\begin{pgfscope}%
\pgftext[x=9.762545in,y=0.625478in,,top]{\sffamily\fontsize{10.000000}{12.000000}\selectfont 20}%
\end{pgfscope}%
\begin{pgfscope}%
\pgfsetbuttcap%
\pgfsetroundjoin%
\definecolor{currentfill}{rgb}{0.000000,0.000000,0.000000}%
\pgfsetfillcolor{currentfill}%
\pgfsetlinewidth{0.803000pt}%
\definecolor{currentstroke}{rgb}{0.000000,0.000000,0.000000}%
\pgfsetstrokecolor{currentstroke}%
\pgfsetdash{}{0pt}%
\pgfsys@defobject{currentmarker}{\pgfqpoint{0.000000in}{-0.048611in}}{\pgfqpoint{0.000000in}{0.000000in}}{%
\pgfpathmoveto{\pgfqpoint{0.000000in}{0.000000in}}%
\pgfpathlineto{\pgfqpoint{0.000000in}{-0.048611in}}%
\pgfusepath{stroke,fill}%
}%
\begin{pgfscope}%
\pgfsys@transformshift{10.332692in}{0.722700in}%
\pgfsys@useobject{currentmarker}{}%
\end{pgfscope}%
\end{pgfscope}%
\begin{pgfscope}%
\pgftext[x=10.332692in,y=0.625478in,,top]{\sffamily\fontsize{10.000000}{12.000000}\selectfont 25}%
\end{pgfscope}%
\begin{pgfscope}%
\pgfsetbuttcap%
\pgfsetroundjoin%
\definecolor{currentfill}{rgb}{0.000000,0.000000,0.000000}%
\pgfsetfillcolor{currentfill}%
\pgfsetlinewidth{0.803000pt}%
\definecolor{currentstroke}{rgb}{0.000000,0.000000,0.000000}%
\pgfsetstrokecolor{currentstroke}%
\pgfsetdash{}{0pt}%
\pgfsys@defobject{currentmarker}{\pgfqpoint{0.000000in}{-0.048611in}}{\pgfqpoint{0.000000in}{0.000000in}}{%
\pgfpathmoveto{\pgfqpoint{0.000000in}{0.000000in}}%
\pgfpathlineto{\pgfqpoint{0.000000in}{-0.048611in}}%
\pgfusepath{stroke,fill}%
}%
\begin{pgfscope}%
\pgfsys@transformshift{10.902840in}{0.722700in}%
\pgfsys@useobject{currentmarker}{}%
\end{pgfscope}%
\end{pgfscope}%
\begin{pgfscope}%
\pgftext[x=10.902840in,y=0.625478in,,top]{\sffamily\fontsize{10.000000}{12.000000}\selectfont 30}%
\end{pgfscope}%
\begin{pgfscope}%
\pgfsetbuttcap%
\pgfsetroundjoin%
\definecolor{currentfill}{rgb}{0.000000,0.000000,0.000000}%
\pgfsetfillcolor{currentfill}%
\pgfsetlinewidth{0.803000pt}%
\definecolor{currentstroke}{rgb}{0.000000,0.000000,0.000000}%
\pgfsetstrokecolor{currentstroke}%
\pgfsetdash{}{0pt}%
\pgfsys@defobject{currentmarker}{\pgfqpoint{0.000000in}{-0.048611in}}{\pgfqpoint{0.000000in}{0.000000in}}{%
\pgfpathmoveto{\pgfqpoint{0.000000in}{0.000000in}}%
\pgfpathlineto{\pgfqpoint{0.000000in}{-0.048611in}}%
\pgfusepath{stroke,fill}%
}%
\begin{pgfscope}%
\pgfsys@transformshift{11.472988in}{0.722700in}%
\pgfsys@useobject{currentmarker}{}%
\end{pgfscope}%
\end{pgfscope}%
\begin{pgfscope}%
\pgftext[x=11.472988in,y=0.625478in,,top]{\sffamily\fontsize{10.000000}{12.000000}\selectfont 35}%
\end{pgfscope}%
\begin{pgfscope}%
\pgfsetbuttcap%
\pgfsetroundjoin%
\definecolor{currentfill}{rgb}{0.000000,0.000000,0.000000}%
\pgfsetfillcolor{currentfill}%
\pgfsetlinewidth{0.803000pt}%
\definecolor{currentstroke}{rgb}{0.000000,0.000000,0.000000}%
\pgfsetstrokecolor{currentstroke}%
\pgfsetdash{}{0pt}%
\pgfsys@defobject{currentmarker}{\pgfqpoint{0.000000in}{-0.048611in}}{\pgfqpoint{0.000000in}{0.000000in}}{%
\pgfpathmoveto{\pgfqpoint{0.000000in}{0.000000in}}%
\pgfpathlineto{\pgfqpoint{0.000000in}{-0.048611in}}%
\pgfusepath{stroke,fill}%
}%
\begin{pgfscope}%
\pgfsys@transformshift{12.043135in}{0.722700in}%
\pgfsys@useobject{currentmarker}{}%
\end{pgfscope}%
\end{pgfscope}%
\begin{pgfscope}%
\pgftext[x=12.043135in,y=0.625478in,,top]{\sffamily\fontsize{10.000000}{12.000000}\selectfont 40}%
\end{pgfscope}%
\begin{pgfscope}%
\pgftext[x=9.887977in,y=0.435509in,,top]{\sffamily\fontsize{16.000000}{19.200000}\selectfont \(\displaystyle z-position [\mu m]\)}%
\end{pgfscope}%
\begin{pgfscope}%
\pgfsetbuttcap%
\pgfsetroundjoin%
\definecolor{currentfill}{rgb}{0.000000,0.000000,0.000000}%
\pgfsetfillcolor{currentfill}%
\pgfsetlinewidth{0.803000pt}%
\definecolor{currentstroke}{rgb}{0.000000,0.000000,0.000000}%
\pgfsetstrokecolor{currentstroke}%
\pgfsetdash{}{0pt}%
\pgfsys@defobject{currentmarker}{\pgfqpoint{-0.048611in}{0.000000in}}{\pgfqpoint{0.000000in}{0.000000in}}{%
\pgfpathmoveto{\pgfqpoint{0.000000in}{0.000000in}}%
\pgfpathlineto{\pgfqpoint{-0.048611in}{0.000000in}}%
\pgfusepath{stroke,fill}%
}%
\begin{pgfscope}%
\pgfsys@transformshift{7.481955in}{0.722700in}%
\pgfsys@useobject{currentmarker}{}%
\end{pgfscope}%
\end{pgfscope}%
\begin{pgfscope}%
\pgftext[x=6.870748in,y=0.669938in,left,base]{\sffamily\fontsize{10.000000}{12.000000}\selectfont -0.003}%
\end{pgfscope}%
\begin{pgfscope}%
\pgfsetbuttcap%
\pgfsetroundjoin%
\definecolor{currentfill}{rgb}{0.000000,0.000000,0.000000}%
\pgfsetfillcolor{currentfill}%
\pgfsetlinewidth{0.803000pt}%
\definecolor{currentstroke}{rgb}{0.000000,0.000000,0.000000}%
\pgfsetstrokecolor{currentstroke}%
\pgfsetdash{}{0pt}%
\pgfsys@defobject{currentmarker}{\pgfqpoint{-0.048611in}{0.000000in}}{\pgfqpoint{0.000000in}{0.000000in}}{%
\pgfpathmoveto{\pgfqpoint{0.000000in}{0.000000in}}%
\pgfpathlineto{\pgfqpoint{-0.048611in}{0.000000in}}%
\pgfusepath{stroke,fill}%
}%
\begin{pgfscope}%
\pgfsys@transformshift{7.481955in}{1.565850in}%
\pgfsys@useobject{currentmarker}{}%
\end{pgfscope}%
\end{pgfscope}%
\begin{pgfscope}%
\pgftext[x=6.870748in,y=1.513088in,left,base]{\sffamily\fontsize{10.000000}{12.000000}\selectfont -0.002}%
\end{pgfscope}%
\begin{pgfscope}%
\pgfsetbuttcap%
\pgfsetroundjoin%
\definecolor{currentfill}{rgb}{0.000000,0.000000,0.000000}%
\pgfsetfillcolor{currentfill}%
\pgfsetlinewidth{0.803000pt}%
\definecolor{currentstroke}{rgb}{0.000000,0.000000,0.000000}%
\pgfsetstrokecolor{currentstroke}%
\pgfsetdash{}{0pt}%
\pgfsys@defobject{currentmarker}{\pgfqpoint{-0.048611in}{0.000000in}}{\pgfqpoint{0.000000in}{0.000000in}}{%
\pgfpathmoveto{\pgfqpoint{0.000000in}{0.000000in}}%
\pgfpathlineto{\pgfqpoint{-0.048611in}{0.000000in}}%
\pgfusepath{stroke,fill}%
}%
\begin{pgfscope}%
\pgfsys@transformshift{7.481955in}{2.409000in}%
\pgfsys@useobject{currentmarker}{}%
\end{pgfscope}%
\end{pgfscope}%
\begin{pgfscope}%
\pgftext[x=6.870748in,y=2.356238in,left,base]{\sffamily\fontsize{10.000000}{12.000000}\selectfont -0.001}%
\end{pgfscope}%
\begin{pgfscope}%
\pgfsetbuttcap%
\pgfsetroundjoin%
\definecolor{currentfill}{rgb}{0.000000,0.000000,0.000000}%
\pgfsetfillcolor{currentfill}%
\pgfsetlinewidth{0.803000pt}%
\definecolor{currentstroke}{rgb}{0.000000,0.000000,0.000000}%
\pgfsetstrokecolor{currentstroke}%
\pgfsetdash{}{0pt}%
\pgfsys@defobject{currentmarker}{\pgfqpoint{-0.048611in}{0.000000in}}{\pgfqpoint{0.000000in}{0.000000in}}{%
\pgfpathmoveto{\pgfqpoint{0.000000in}{0.000000in}}%
\pgfpathlineto{\pgfqpoint{-0.048611in}{0.000000in}}%
\pgfusepath{stroke,fill}%
}%
\begin{pgfscope}%
\pgfsys@transformshift{7.481955in}{3.252150in}%
\pgfsys@useobject{currentmarker}{}%
\end{pgfscope}%
\end{pgfscope}%
\begin{pgfscope}%
\pgftext[x=6.987122in,y=3.199388in,left,base]{\sffamily\fontsize{10.000000}{12.000000}\selectfont 0.000}%
\end{pgfscope}%
\begin{pgfscope}%
\pgfsetbuttcap%
\pgfsetroundjoin%
\definecolor{currentfill}{rgb}{0.000000,0.000000,0.000000}%
\pgfsetfillcolor{currentfill}%
\pgfsetlinewidth{0.803000pt}%
\definecolor{currentstroke}{rgb}{0.000000,0.000000,0.000000}%
\pgfsetstrokecolor{currentstroke}%
\pgfsetdash{}{0pt}%
\pgfsys@defobject{currentmarker}{\pgfqpoint{-0.048611in}{0.000000in}}{\pgfqpoint{0.000000in}{0.000000in}}{%
\pgfpathmoveto{\pgfqpoint{0.000000in}{0.000000in}}%
\pgfpathlineto{\pgfqpoint{-0.048611in}{0.000000in}}%
\pgfusepath{stroke,fill}%
}%
\begin{pgfscope}%
\pgfsys@transformshift{7.481955in}{4.095300in}%
\pgfsys@useobject{currentmarker}{}%
\end{pgfscope}%
\end{pgfscope}%
\begin{pgfscope}%
\pgftext[x=6.987122in,y=4.042538in,left,base]{\sffamily\fontsize{10.000000}{12.000000}\selectfont 0.001}%
\end{pgfscope}%
\begin{pgfscope}%
\pgfsetbuttcap%
\pgfsetroundjoin%
\definecolor{currentfill}{rgb}{0.000000,0.000000,0.000000}%
\pgfsetfillcolor{currentfill}%
\pgfsetlinewidth{0.803000pt}%
\definecolor{currentstroke}{rgb}{0.000000,0.000000,0.000000}%
\pgfsetstrokecolor{currentstroke}%
\pgfsetdash{}{0pt}%
\pgfsys@defobject{currentmarker}{\pgfqpoint{-0.048611in}{0.000000in}}{\pgfqpoint{0.000000in}{0.000000in}}{%
\pgfpathmoveto{\pgfqpoint{0.000000in}{0.000000in}}%
\pgfpathlineto{\pgfqpoint{-0.048611in}{0.000000in}}%
\pgfusepath{stroke,fill}%
}%
\begin{pgfscope}%
\pgfsys@transformshift{7.481955in}{4.938450in}%
\pgfsys@useobject{currentmarker}{}%
\end{pgfscope}%
\end{pgfscope}%
\begin{pgfscope}%
\pgftext[x=6.987122in,y=4.885688in,left,base]{\sffamily\fontsize{10.000000}{12.000000}\selectfont 0.002}%
\end{pgfscope}%
\begin{pgfscope}%
\pgfsetbuttcap%
\pgfsetroundjoin%
\definecolor{currentfill}{rgb}{0.000000,0.000000,0.000000}%
\pgfsetfillcolor{currentfill}%
\pgfsetlinewidth{0.803000pt}%
\definecolor{currentstroke}{rgb}{0.000000,0.000000,0.000000}%
\pgfsetstrokecolor{currentstroke}%
\pgfsetdash{}{0pt}%
\pgfsys@defobject{currentmarker}{\pgfqpoint{-0.048611in}{0.000000in}}{\pgfqpoint{0.000000in}{0.000000in}}{%
\pgfpathmoveto{\pgfqpoint{0.000000in}{0.000000in}}%
\pgfpathlineto{\pgfqpoint{-0.048611in}{0.000000in}}%
\pgfusepath{stroke,fill}%
}%
\begin{pgfscope}%
\pgfsys@transformshift{7.481955in}{5.781600in}%
\pgfsys@useobject{currentmarker}{}%
\end{pgfscope}%
\end{pgfscope}%
\begin{pgfscope}%
\pgftext[x=6.987122in,y=5.728838in,left,base]{\sffamily\fontsize{10.000000}{12.000000}\selectfont 0.003}%
\end{pgfscope}%
\begin{pgfscope}%
\pgftext[x=6.815193in,y=3.252150in,,bottom,rotate=90.000000]{\sffamily\fontsize{16.000000}{19.200000}\selectfont \(\displaystyle Poynting\) \(\displaystyle vector\)}%
\end{pgfscope}%
\begin{pgfscope}%
\pgfpathrectangle{\pgfqpoint{7.481955in}{0.722700in}}{\pgfqpoint{4.812045in}{5.058900in}} %
\pgfusepath{clip}%
\pgfsetrectcap%
\pgfsetroundjoin%
\pgfsetlinewidth{1.505625pt}%
\definecolor{currentstroke}{rgb}{0.000000,0.000000,0.000000}%
\pgfsetstrokecolor{currentstroke}%
\pgfsetdash{}{0pt}%
\pgfpathmoveto{\pgfqpoint{7.481955in}{3.252150in}}%
\pgfpathlineto{\pgfqpoint{10.664600in}{3.251051in}}%
\pgfpathlineto{\pgfqpoint{10.667947in}{3.247710in}}%
\pgfpathlineto{\pgfqpoint{10.671713in}{3.240284in}}%
\pgfpathlineto{\pgfqpoint{10.683011in}{3.213922in}}%
\pgfpathlineto{\pgfqpoint{10.684685in}{3.214753in}}%
\pgfpathlineto{\pgfqpoint{10.686777in}{3.218907in}}%
\pgfpathlineto{\pgfqpoint{10.690124in}{3.231605in}}%
\pgfpathlineto{\pgfqpoint{10.695564in}{3.251633in}}%
\pgfpathlineto{\pgfqpoint{10.696819in}{3.251949in}}%
\pgfpathlineto{\pgfqpoint{10.698075in}{3.249246in}}%
\pgfpathlineto{\pgfqpoint{10.700167in}{3.236532in}}%
\pgfpathlineto{\pgfqpoint{10.703096in}{3.198985in}}%
\pgfpathlineto{\pgfqpoint{10.707280in}{3.108898in}}%
\pgfpathlineto{\pgfqpoint{10.714812in}{2.944021in}}%
\pgfpathlineto{\pgfqpoint{10.716904in}{2.930374in}}%
\pgfpathlineto{\pgfqpoint{10.717741in}{2.931321in}}%
\pgfpathlineto{\pgfqpoint{10.718997in}{2.939987in}}%
\pgfpathlineto{\pgfqpoint{10.721089in}{2.973265in}}%
\pgfpathlineto{\pgfqpoint{10.724855in}{3.078887in}}%
\pgfpathlineto{\pgfqpoint{10.730713in}{3.239550in}}%
\pgfpathlineto{\pgfqpoint{10.732387in}{3.252138in}}%
\pgfpathlineto{\pgfqpoint{10.733224in}{3.249317in}}%
\pgfpathlineto{\pgfqpoint{10.734479in}{3.232441in}}%
\pgfpathlineto{\pgfqpoint{10.736571in}{3.168983in}}%
\pgfpathlineto{\pgfqpoint{10.739919in}{2.981839in}}%
\pgfpathlineto{\pgfqpoint{10.750380in}{2.308144in}}%
\pgfpathlineto{\pgfqpoint{10.751635in}{2.292580in}}%
\pgfpathlineto{\pgfqpoint{10.752053in}{2.292627in}}%
\pgfpathlineto{\pgfqpoint{10.752890in}{2.300724in}}%
\pgfpathlineto{\pgfqpoint{10.754564in}{2.348565in}}%
\pgfpathlineto{\pgfqpoint{10.757493in}{2.522626in}}%
\pgfpathlineto{\pgfqpoint{10.767954in}{3.248924in}}%
\pgfpathlineto{\pgfqpoint{10.768372in}{3.251908in}}%
\pgfpathlineto{\pgfqpoint{10.768791in}{3.251475in}}%
\pgfpathlineto{\pgfqpoint{10.769628in}{3.240138in}}%
\pgfpathlineto{\pgfqpoint{10.771301in}{3.175152in}}%
\pgfpathlineto{\pgfqpoint{10.774231in}{2.934698in}}%
\pgfpathlineto{\pgfqpoint{10.780089in}{2.182281in}}%
\pgfpathlineto{\pgfqpoint{10.784691in}{1.722296in}}%
\pgfpathlineto{\pgfqpoint{10.786784in}{1.649466in}}%
\pgfpathlineto{\pgfqpoint{10.787202in}{1.647548in}}%
\pgfpathlineto{\pgfqpoint{10.787202in}{1.647548in}}%
\pgfpathlineto{\pgfqpoint{10.787202in}{1.647548in}}%
\pgfpathlineto{\pgfqpoint{10.788039in}{1.656707in}}%
\pgfpathlineto{\pgfqpoint{10.789713in}{1.726246in}}%
\pgfpathlineto{\pgfqpoint{10.792223in}{1.946223in}}%
\pgfpathlineto{\pgfqpoint{10.798082in}{2.742072in}}%
\pgfpathlineto{\pgfqpoint{10.802266in}{3.177398in}}%
\pgfpathlineto{\pgfqpoint{10.804358in}{3.251202in}}%
\pgfpathlineto{\pgfqpoint{10.804777in}{3.251808in}}%
\pgfpathlineto{\pgfqpoint{10.805613in}{3.238395in}}%
\pgfpathlineto{\pgfqpoint{10.807287in}{3.153644in}}%
\pgfpathlineto{\pgfqpoint{10.810216in}{2.838115in}}%
\pgfpathlineto{\pgfqpoint{10.822351in}{1.261578in}}%
\pgfpathlineto{\pgfqpoint{10.822769in}{1.257203in}}%
\pgfpathlineto{\pgfqpoint{10.823188in}{1.258152in}}%
\pgfpathlineto{\pgfqpoint{10.824443in}{1.292837in}}%
\pgfpathlineto{\pgfqpoint{10.826535in}{1.451797in}}%
\pgfpathlineto{\pgfqpoint{10.829883in}{1.915306in}}%
\pgfpathlineto{\pgfqpoint{10.838252in}{3.157972in}}%
\pgfpathlineto{\pgfqpoint{10.840344in}{3.250019in}}%
\pgfpathlineto{\pgfqpoint{10.840762in}{3.252106in}}%
\pgfpathlineto{\pgfqpoint{10.840762in}{3.252106in}}%
\pgfpathlineto{\pgfqpoint{10.840762in}{3.252106in}}%
\pgfpathlineto{\pgfqpoint{10.841599in}{3.239571in}}%
\pgfpathlineto{\pgfqpoint{10.843273in}{3.148802in}}%
\pgfpathlineto{\pgfqpoint{10.845784in}{2.864570in}}%
\pgfpathlineto{\pgfqpoint{10.851223in}{1.906694in}}%
\pgfpathlineto{\pgfqpoint{10.856245in}{1.196421in}}%
\pgfpathlineto{\pgfqpoint{10.858755in}{1.088010in}}%
\pgfpathlineto{\pgfqpoint{10.859174in}{1.089765in}}%
\pgfpathlineto{\pgfqpoint{10.860429in}{1.129338in}}%
\pgfpathlineto{\pgfqpoint{10.862521in}{1.303815in}}%
\pgfpathlineto{\pgfqpoint{10.866287in}{1.881087in}}%
\pgfpathlineto{\pgfqpoint{10.874237in}{3.144605in}}%
\pgfpathlineto{\pgfqpoint{10.876748in}{3.252104in}}%
\pgfpathlineto{\pgfqpoint{10.877585in}{3.241895in}}%
\pgfpathlineto{\pgfqpoint{10.879259in}{3.152736in}}%
\pgfpathlineto{\pgfqpoint{10.881769in}{2.864241in}}%
\pgfpathlineto{\pgfqpoint{10.887209in}{1.880700in}}%
\pgfpathlineto{\pgfqpoint{10.892230in}{1.146613in}}%
\pgfpathlineto{\pgfqpoint{10.894741in}{1.031758in}}%
\pgfpathlineto{\pgfqpoint{10.895159in}{1.032911in}}%
\pgfpathlineto{\pgfqpoint{10.896415in}{1.071518in}}%
\pgfpathlineto{\pgfqpoint{10.898507in}{1.247160in}}%
\pgfpathlineto{\pgfqpoint{10.901854in}{1.757316in}}%
\pgfpathlineto{\pgfqpoint{10.910223in}{3.134082in}}%
\pgfpathlineto{\pgfqpoint{10.912734in}{3.251734in}}%
\pgfpathlineto{\pgfqpoint{10.913152in}{3.250994in}}%
\pgfpathlineto{\pgfqpoint{10.914407in}{3.213478in}}%
\pgfpathlineto{\pgfqpoint{10.916500in}{3.039033in}}%
\pgfpathlineto{\pgfqpoint{10.919847in}{2.528712in}}%
\pgfpathlineto{\pgfqpoint{10.928634in}{1.105898in}}%
\pgfpathlineto{\pgfqpoint{10.930727in}{1.017784in}}%
\pgfpathlineto{\pgfqpoint{10.931145in}{1.017716in}}%
\pgfpathlineto{\pgfqpoint{10.931982in}{1.035327in}}%
\pgfpathlineto{\pgfqpoint{10.933656in}{1.139583in}}%
\pgfpathlineto{\pgfqpoint{10.936585in}{1.514655in}}%
\pgfpathlineto{\pgfqpoint{10.947883in}{3.231762in}}%
\pgfpathlineto{\pgfqpoint{10.949138in}{3.251733in}}%
\pgfpathlineto{\pgfqpoint{10.949975in}{3.235450in}}%
\pgfpathlineto{\pgfqpoint{10.951649in}{3.133674in}}%
\pgfpathlineto{\pgfqpoint{10.954578in}{2.761838in}}%
\pgfpathlineto{\pgfqpoint{10.965875in}{1.036774in}}%
\pgfpathlineto{\pgfqpoint{10.967131in}{1.014554in}}%
\pgfpathlineto{\pgfqpoint{10.967968in}{1.029347in}}%
\pgfpathlineto{\pgfqpoint{10.969641in}{1.128305in}}%
\pgfpathlineto{\pgfqpoint{10.972570in}{1.496290in}}%
\pgfpathlineto{\pgfqpoint{10.984287in}{3.241728in}}%
\pgfpathlineto{\pgfqpoint{10.985124in}{3.252104in}}%
\pgfpathlineto{\pgfqpoint{10.985542in}{3.248394in}}%
\pgfpathlineto{\pgfqpoint{10.986797in}{3.202006in}}%
\pgfpathlineto{\pgfqpoint{10.988890in}{3.013885in}}%
\pgfpathlineto{\pgfqpoint{10.992656in}{2.410432in}}%
\pgfpathlineto{\pgfqpoint{11.000187in}{1.156462in}}%
\pgfpathlineto{\pgfqpoint{11.002698in}{1.017057in}}%
\pgfpathlineto{\pgfqpoint{11.003117in}{1.014088in}}%
\pgfpathlineto{\pgfqpoint{11.003535in}{1.017056in}}%
\pgfpathlineto{\pgfqpoint{11.004790in}{1.061260in}}%
\pgfpathlineto{\pgfqpoint{11.006882in}{1.246105in}}%
\pgfpathlineto{\pgfqpoint{11.010648in}{1.845950in}}%
\pgfpathlineto{\pgfqpoint{11.018180in}{3.104765in}}%
\pgfpathlineto{\pgfqpoint{11.020691in}{3.248394in}}%
\pgfpathlineto{\pgfqpoint{11.021109in}{3.252104in}}%
\pgfpathlineto{\pgfqpoint{11.021528in}{3.249877in}}%
\pgfpathlineto{\pgfqpoint{11.022783in}{3.207859in}}%
\pgfpathlineto{\pgfqpoint{11.024875in}{3.026305in}}%
\pgfpathlineto{\pgfqpoint{11.028641in}{2.430117in}}%
\pgfpathlineto{\pgfqpoint{11.036592in}{1.128076in}}%
\pgfpathlineto{\pgfqpoint{11.039102in}{1.014262in}}%
\pgfpathlineto{\pgfqpoint{11.039521in}{1.015747in}}%
\pgfpathlineto{\pgfqpoint{11.040776in}{1.055574in}}%
\pgfpathlineto{\pgfqpoint{11.042868in}{1.233821in}}%
\pgfpathlineto{\pgfqpoint{11.046634in}{1.826298in}}%
\pgfpathlineto{\pgfqpoint{11.054584in}{3.133629in}}%
\pgfpathlineto{\pgfqpoint{11.057095in}{3.251732in}}%
\pgfpathlineto{\pgfqpoint{11.057514in}{3.250990in}}%
\pgfpathlineto{\pgfqpoint{11.058769in}{3.213358in}}%
\pgfpathlineto{\pgfqpoint{11.060861in}{3.038433in}}%
\pgfpathlineto{\pgfqpoint{11.064209in}{2.526996in}}%
\pgfpathlineto{\pgfqpoint{11.072996in}{1.102703in}}%
\pgfpathlineto{\pgfqpoint{11.075088in}{1.014819in}}%
\pgfpathlineto{\pgfqpoint{11.075506in}{1.014819in}}%
\pgfpathlineto{\pgfqpoint{11.076343in}{1.032584in}}%
\pgfpathlineto{\pgfqpoint{11.078017in}{1.137204in}}%
\pgfpathlineto{\pgfqpoint{11.080946in}{1.513000in}}%
\pgfpathlineto{\pgfqpoint{11.092244in}{3.231752in}}%
\pgfpathlineto{\pgfqpoint{11.093499in}{3.251732in}}%
\pgfpathlineto{\pgfqpoint{11.094336in}{3.235443in}}%
\pgfpathlineto{\pgfqpoint{11.096010in}{3.133629in}}%
\pgfpathlineto{\pgfqpoint{11.098939in}{2.761684in}}%
\pgfpathlineto{\pgfqpoint{11.110237in}{1.036457in}}%
\pgfpathlineto{\pgfqpoint{11.111492in}{1.014262in}}%
\pgfpathlineto{\pgfqpoint{11.112329in}{1.029075in}}%
\pgfpathlineto{\pgfqpoint{11.114003in}{1.128076in}}%
\pgfpathlineto{\pgfqpoint{11.116932in}{1.496139in}}%
\pgfpathlineto{\pgfqpoint{11.128648in}{3.241727in}}%
\pgfpathlineto{\pgfqpoint{11.129485in}{3.252104in}}%
\pgfpathlineto{\pgfqpoint{11.129903in}{3.248394in}}%
\pgfpathlineto{\pgfqpoint{11.131159in}{3.202004in}}%
\pgfpathlineto{\pgfqpoint{11.133251in}{3.013880in}}%
\pgfpathlineto{\pgfqpoint{11.137017in}{2.410421in}}%
\pgfpathlineto{\pgfqpoint{11.144549in}{1.156455in}}%
\pgfpathlineto{\pgfqpoint{11.147059in}{1.017056in}}%
\pgfpathlineto{\pgfqpoint{11.147478in}{1.014088in}}%
\pgfpathlineto{\pgfqpoint{11.147896in}{1.017057in}}%
\pgfpathlineto{\pgfqpoint{11.149152in}{1.061264in}}%
\pgfpathlineto{\pgfqpoint{11.151244in}{1.246113in}}%
\pgfpathlineto{\pgfqpoint{11.155010in}{1.845963in}}%
\pgfpathlineto{\pgfqpoint{11.162542in}{3.104768in}}%
\pgfpathlineto{\pgfqpoint{11.165052in}{3.248394in}}%
\pgfpathlineto{\pgfqpoint{11.165471in}{3.252104in}}%
\pgfpathlineto{\pgfqpoint{11.165889in}{3.249877in}}%
\pgfpathlineto{\pgfqpoint{11.167144in}{3.207861in}}%
\pgfpathlineto{\pgfqpoint{11.169237in}{3.026314in}}%
\pgfpathlineto{\pgfqpoint{11.173003in}{2.430164in}}%
\pgfpathlineto{\pgfqpoint{11.180953in}{1.128305in}}%
\pgfpathlineto{\pgfqpoint{11.183464in}{1.014554in}}%
\pgfpathlineto{\pgfqpoint{11.183882in}{1.016047in}}%
\pgfpathlineto{\pgfqpoint{11.185137in}{1.055897in}}%
\pgfpathlineto{\pgfqpoint{11.187230in}{1.234166in}}%
\pgfpathlineto{\pgfqpoint{11.190995in}{1.826616in}}%
\pgfpathlineto{\pgfqpoint{11.198946in}{3.133674in}}%
\pgfpathlineto{\pgfqpoint{11.201456in}{3.251733in}}%
\pgfpathlineto{\pgfqpoint{11.201875in}{3.250991in}}%
\pgfpathlineto{\pgfqpoint{11.203130in}{3.213377in}}%
\pgfpathlineto{\pgfqpoint{11.205222in}{3.038554in}}%
\pgfpathlineto{\pgfqpoint{11.208570in}{2.527497in}}%
\pgfpathlineto{\pgfqpoint{11.217357in}{1.105179in}}%
\pgfpathlineto{\pgfqpoint{11.219449in}{1.017716in}}%
\pgfpathlineto{\pgfqpoint{11.219868in}{1.017784in}}%
\pgfpathlineto{\pgfqpoint{11.220705in}{1.035664in}}%
\pgfpathlineto{\pgfqpoint{11.222378in}{1.140420in}}%
\pgfpathlineto{\pgfqpoint{11.225307in}{1.516096in}}%
\pgfpathlineto{\pgfqpoint{11.236605in}{3.231817in}}%
\pgfpathlineto{\pgfqpoint{11.237861in}{3.251734in}}%
\pgfpathlineto{\pgfqpoint{11.238698in}{3.235502in}}%
\pgfpathlineto{\pgfqpoint{11.240371in}{3.134082in}}%
\pgfpathlineto{\pgfqpoint{11.243300in}{2.763842in}}%
\pgfpathlineto{\pgfqpoint{11.254598in}{1.052843in}}%
\pgfpathlineto{\pgfqpoint{11.255853in}{1.031758in}}%
\pgfpathlineto{\pgfqpoint{11.256690in}{1.047108in}}%
\pgfpathlineto{\pgfqpoint{11.258364in}{1.146613in}}%
\pgfpathlineto{\pgfqpoint{11.261293in}{1.513529in}}%
\pgfpathlineto{\pgfqpoint{11.272591in}{3.228083in}}%
\pgfpathlineto{\pgfqpoint{11.273846in}{3.252104in}}%
\pgfpathlineto{\pgfqpoint{11.274265in}{3.248458in}}%
\pgfpathlineto{\pgfqpoint{11.275520in}{3.202896in}}%
\pgfpathlineto{\pgfqpoint{11.277612in}{3.018482in}}%
\pgfpathlineto{\pgfqpoint{11.281378in}{2.429233in}}%
\pgfpathlineto{\pgfqpoint{11.288910in}{1.218638in}}%
\pgfpathlineto{\pgfqpoint{11.291421in}{1.089765in}}%
\pgfpathlineto{\pgfqpoint{11.291839in}{1.088010in}}%
\pgfpathlineto{\pgfqpoint{11.292676in}{1.101740in}}%
\pgfpathlineto{\pgfqpoint{11.294350in}{1.196421in}}%
\pgfpathlineto{\pgfqpoint{11.297279in}{1.550643in}}%
\pgfpathlineto{\pgfqpoint{11.308995in}{3.239571in}}%
\pgfpathlineto{\pgfqpoint{11.309832in}{3.252106in}}%
\pgfpathlineto{\pgfqpoint{11.310251in}{3.250019in}}%
\pgfpathlineto{\pgfqpoint{11.311506in}{3.210744in}}%
\pgfpathlineto{\pgfqpoint{11.313598in}{3.042049in}}%
\pgfpathlineto{\pgfqpoint{11.317364in}{2.494772in}}%
\pgfpathlineto{\pgfqpoint{11.324896in}{1.373828in}}%
\pgfpathlineto{\pgfqpoint{11.327407in}{1.258152in}}%
\pgfpathlineto{\pgfqpoint{11.327825in}{1.257203in}}%
\pgfpathlineto{\pgfqpoint{11.328662in}{1.271247in}}%
\pgfpathlineto{\pgfqpoint{11.330336in}{1.361261in}}%
\pgfpathlineto{\pgfqpoint{11.333265in}{1.690985in}}%
\pgfpathlineto{\pgfqpoint{11.344563in}{3.224377in}}%
\pgfpathlineto{\pgfqpoint{11.345818in}{3.251808in}}%
\pgfpathlineto{\pgfqpoint{11.346236in}{3.251202in}}%
\pgfpathlineto{\pgfqpoint{11.347492in}{3.220686in}}%
\pgfpathlineto{\pgfqpoint{11.349584in}{3.081085in}}%
\pgfpathlineto{\pgfqpoint{11.353350in}{2.626129in}}%
\pgfpathlineto{\pgfqpoint{11.360463in}{1.753866in}}%
\pgfpathlineto{\pgfqpoint{11.362974in}{1.649961in}}%
\pgfpathlineto{\pgfqpoint{11.363392in}{1.647548in}}%
\pgfpathlineto{\pgfqpoint{11.363811in}{1.649466in}}%
\pgfpathlineto{\pgfqpoint{11.365066in}{1.680828in}}%
\pgfpathlineto{\pgfqpoint{11.367158in}{1.812812in}}%
\pgfpathlineto{\pgfqpoint{11.370924in}{2.237096in}}%
\pgfpathlineto{\pgfqpoint{11.378874in}{3.150232in}}%
\pgfpathlineto{\pgfqpoint{11.381385in}{3.247565in}}%
\pgfpathlineto{\pgfqpoint{11.382222in}{3.251908in}}%
\pgfpathlineto{\pgfqpoint{11.382640in}{3.248924in}}%
\pgfpathlineto{\pgfqpoint{11.383896in}{3.220329in}}%
\pgfpathlineto{\pgfqpoint{11.386406in}{3.086567in}}%
\pgfpathlineto{\pgfqpoint{11.393101in}{2.522626in}}%
\pgfpathlineto{\pgfqpoint{11.396867in}{2.319457in}}%
\pgfpathlineto{\pgfqpoint{11.398960in}{2.292580in}}%
\pgfpathlineto{\pgfqpoint{11.399796in}{2.300381in}}%
\pgfpathlineto{\pgfqpoint{11.401470in}{2.346070in}}%
\pgfpathlineto{\pgfqpoint{11.404399in}{2.506377in}}%
\pgfpathlineto{\pgfqpoint{11.415697in}{3.223308in}}%
\pgfpathlineto{\pgfqpoint{11.417789in}{3.251540in}}%
\pgfpathlineto{\pgfqpoint{11.418208in}{3.252138in}}%
\pgfpathlineto{\pgfqpoint{11.418626in}{3.251168in}}%
\pgfpathlineto{\pgfqpoint{11.419881in}{3.239550in}}%
\pgfpathlineto{\pgfqpoint{11.422392in}{3.184862in}}%
\pgfpathlineto{\pgfqpoint{11.432016in}{2.936129in}}%
\pgfpathlineto{\pgfqpoint{11.433690in}{2.930374in}}%
\pgfpathlineto{\pgfqpoint{11.434945in}{2.935947in}}%
\pgfpathlineto{\pgfqpoint{11.437037in}{2.961925in}}%
\pgfpathlineto{\pgfqpoint{11.440803in}{3.044689in}}%
\pgfpathlineto{\pgfqpoint{11.448335in}{3.212126in}}%
\pgfpathlineto{\pgfqpoint{11.451683in}{3.245479in}}%
\pgfpathlineto{\pgfqpoint{11.453775in}{3.251949in}}%
\pgfpathlineto{\pgfqpoint{11.455030in}{3.251633in}}%
\pgfpathlineto{\pgfqpoint{11.456704in}{3.247656in}}%
\pgfpathlineto{\pgfqpoint{11.461725in}{3.226185in}}%
\pgfpathlineto{\pgfqpoint{11.465491in}{3.215307in}}%
\pgfpathlineto{\pgfqpoint{11.467583in}{3.213922in}}%
\pgfpathlineto{\pgfqpoint{11.469257in}{3.215154in}}%
\pgfpathlineto{\pgfqpoint{11.472186in}{3.221114in}}%
\pgfpathlineto{\pgfqpoint{11.483484in}{3.248814in}}%
\pgfpathlineto{\pgfqpoint{11.487250in}{3.251655in}}%
\pgfpathlineto{\pgfqpoint{11.491853in}{3.252079in}}%
\pgfpathlineto{\pgfqpoint{11.507754in}{3.251696in}}%
\pgfpathlineto{\pgfqpoint{11.531186in}{3.252149in}}%
\pgfpathlineto{\pgfqpoint{12.293582in}{3.252150in}}%
\pgfpathlineto{\pgfqpoint{12.293582in}{3.252150in}}%
\pgfusepath{stroke}%
\end{pgfscope}%
\begin{pgfscope}%
\pgfsetrectcap%
\pgfsetmiterjoin%
\pgfsetlinewidth{0.803000pt}%
\definecolor{currentstroke}{rgb}{0.000000,0.000000,0.000000}%
\pgfsetstrokecolor{currentstroke}%
\pgfsetdash{}{0pt}%
\pgfpathmoveto{\pgfqpoint{7.481955in}{0.722700in}}%
\pgfpathlineto{\pgfqpoint{7.481955in}{5.781600in}}%
\pgfusepath{stroke}%
\end{pgfscope}%
\begin{pgfscope}%
\pgfsetrectcap%
\pgfsetmiterjoin%
\pgfsetlinewidth{0.803000pt}%
\definecolor{currentstroke}{rgb}{0.000000,0.000000,0.000000}%
\pgfsetstrokecolor{currentstroke}%
\pgfsetdash{}{0pt}%
\pgfpathmoveto{\pgfqpoint{12.294000in}{0.722700in}}%
\pgfpathlineto{\pgfqpoint{12.294000in}{5.781600in}}%
\pgfusepath{stroke}%
\end{pgfscope}%
\begin{pgfscope}%
\pgfsetrectcap%
\pgfsetmiterjoin%
\pgfsetlinewidth{0.803000pt}%
\definecolor{currentstroke}{rgb}{0.000000,0.000000,0.000000}%
\pgfsetstrokecolor{currentstroke}%
\pgfsetdash{}{0pt}%
\pgfpathmoveto{\pgfqpoint{7.481955in}{0.722700in}}%
\pgfpathlineto{\pgfqpoint{12.294000in}{0.722700in}}%
\pgfusepath{stroke}%
\end{pgfscope}%
\begin{pgfscope}%
\pgfsetrectcap%
\pgfsetmiterjoin%
\pgfsetlinewidth{0.803000pt}%
\definecolor{currentstroke}{rgb}{0.000000,0.000000,0.000000}%
\pgfsetstrokecolor{currentstroke}%
\pgfsetdash{}{0pt}%
\pgfpathmoveto{\pgfqpoint{7.481955in}{5.781600in}}%
\pgfpathlineto{\pgfqpoint{12.294000in}{5.781600in}}%
\pgfusepath{stroke}%
\end{pgfscope}%
\end{pgfpicture}%
\makeatother%
\endgroup%
}}
%        \subcaption{Simulation after the wave hit the right side of the numerical window.}
%        \label{fig:task2_2}
%    \end{subfigure}
%  \caption{Simulation results from task 2.}
%  \label{fig:task2}
%\end{figure}

\newpage
\appendix
\section{Code}
In order to do this task efficiently, I had to write the matrix multiplication functions as a C-extension library. This was done in Cython, which allows you to add type definitions and compile the code. The generated C-code itself is about 10~000 lines so it is not included but the steps to generate it are.
\subsection{Python code}
\begin{changemargin}{-3cm}{0.5cm}
\lstinputlisting[language=Python]{calcs.py}
\end{changemargin}
\newpage
\begin{changemargin}{-3cm}{0.5cm}
\lstinputlisting[language=Python]{setup.py}
\end{changemargin}
\newpage
\subsection{Cython code}
\begin{changemargin}{-3cm}{0.5cm}
\lstinputlisting[language=Python]{ha5utils.pyx}
\end{changemargin}
\end{document}