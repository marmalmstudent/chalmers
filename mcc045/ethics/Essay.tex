
\documentclass[12pt,a4paper]{article}

%\pdfoutput=1

\usepackage[utf8]{inputenc}
\usepackage[T1]{fontenc}
\usepackage[english]{babel}
\usepackage{amsmath}
\usepackage{mathabx}
\usepackage{lmodern}
\usepackage{listings}
\usepackage{units}
\usepackage{siunitx}
\usepackage{icomma}
\usepackage{graphicx}
\usepackage{caption}
\usepackage{subcaption}
\usepackage{color}
\usepackage{pgf}
\DeclareMathOperator{\acosh}{arccosh}
\newcommand{\N}{\ensuremath{\mathbbm{N}}}
\newcommand{\Z}{\ensuremath{\mathbbm{Z}}}
\newcommand{\Q}{\ensuremath{\mathbbm{Q}}}
\newcommand{\R}{\ensuremath{\mathbbm{R}}}
\newcommand{\C}{\ensuremath{\mathbbm{C}}}
\newcommand{\rd}{\ensuremath{\mathrm{d}}}
\newcommand{\id}{\ensuremath{\,\rd}}
\usepackage{hyperref}
%\usepackage{a4wide} % puts the page numbering further down the page.
\usepackage{pdfpages}
\usepackage{epstopdf}
\DeclareGraphicsExtensions{.eps}
\def\changemargin#1#2{\list{}{\rightmargin#2\leftmargin#1}\item[]}
\let\endchangemargin=\endlist

\title{Free/libre and open source versus prorpietary software in engineering}
\author{Marcus Malmquist, marmalm}
\date{\today}

\begin{document}
\maketitle

\section{Introduciton}
In order to talk about free and open source software versus proprietary software we must define what they are.
\subsection{Free/libre and open source software}\label{sec:free}
When I say ``Free software,'' I mean free as in freedom, not free as in price. That is why I sometimes write ``free/libre software'' to distinguish it from ``gratis software'' meaning software that is free of charge.

Free software is software that respects the users freedom. A program that respects its user's freedom must have the four essential freedoms\footnote{https://www.gnu.org/philosophy/free-sw.html}:
\begin{itemize}
\item The freedom to run the program as you wish, for any purpose (freedom~0).
\item The freedom to study how the program works, and change it so it does your computing as you wish (freedom~1). Access to the source code is a precondition for this.
\item The freedom to redistribute copies so you can help your neighbor (freedom~2).
\item The freedom to distribute copies of your modified versions to others (freedom~3). By doing this you can give the whole community a chance to benefit from your changes. Access to the source code is a precondition for this.
\end{itemize}
Some people use the term ``Open source software\footnote{https://opensource.org/docs/osd}.'' The difference between free software and open source software is mainly the moral stand point. While the free software community says that it is your moral duty to share the source code while the open source community says that it might be in your practical interest to share the source code so that the users can improve it. Typically companies like to talk about open source because it focuses on the practical benefits of sharing the source code.
\subsection{Proprietary software}
Proprietary software is software that does not include the definitions for free software or open source software. While this does not necessarily mean that the software is any different from free or open source software there is one big difference: You can not control or even know what the software does.

\subsection{Why you should care}
With software there are two posibilities and it is inevetably one or the other: Either the users control the program, and they can make it do what they want and need to do, or the program controls the users, and the users have to change what they want and need to fit what the program allows them to do.

With free software you have control over the program so you can make it do what you want. If you do not know how to program you can pay someone who does to implement new features. You could also have a third person go through the changes to verify that the program does what you want.

With proprietary software you do not control the program and instead the proprietary developer controls the program. This means that the developer have power over its users and they can change the program to control what the users do. This is a system of unjust power and the developer will be tempted to use this power over the users because it can likely be profitable. Sometimes this power can be used through a back-door which allows the developer to forcebly apply changes to the software.
\\\\
Many gratis and commercial proprietary companies profit by collecting data from its users and then sell that information to the highest bidder or give it to the government. Every proprietary program is therefore either known to be malware or possible malware but it can never be known to not be malware.

Free software is not a guarantee that the program is not malware in the same way that public contracts does not guarantee that the procurement was fair. But since they are public contracts we can read them and detect unfair procurements and correct them. In the same way if we use free software we have access to the source code and we can detect malware. We can report the malware and remove it to retain our freedom. \\
It is for this reason that free versus proprietary software is a bit like democracy versus dictatorship. Which one do you prefer?

\section{Software in engineering}
If you have ever read the license that comes with a proprietary software, you will typically find that you do not purchase the software, only the right to use it for a limited period of time. The owner of the proprietary software can typically also revoke the license if they want. Sharing any information about a proprietary program is typically also prohibited. You can find some excerpts from a proprietary software EULA with some comments in Appendix~\ref{apdx:eula}.
\\\\
How does this issue apply to engineers? You may not care as much about privacy when you are doing work for your employer on a computer provided by your employer as you would doing personal work on your own computer so it may be more interesting to talk about how free and proprietary software affect companies and their customers. You can find two standpoints: the ethical and the practical.

\subsection{The ethical standpoint}
If a company creates products that do not include any software, the customers are not affected by this problem. This is becomming increasingly rare as integrated circuits are used a lot and many of them can be programmed. If this company uses proprietary programs to develop their products then this give the developer of the proprietary program power over this company.

If a company creates products that do include software then it must be free software or else this company does not respect the users freedom. Even if you did not put any malware into the program the users can not know this, and if you use a proprietary program to write the software then you can not know if the code you write is the only code you get. The proprietary program can add things to your program without asking for permission and without telling you it did so. In essence your code can contain malware even if you did not explicitly include any, if you use proprietary software to write it.

\subsection{The practical standpoint}
When you use a proprietary program to do your work, this work is typically saved in a format that only can be read and understood by that program. Since you do not own the program, in practice you do not even own the work you do in that program because without it your work can not be accessed. This means that your company depends on the owner of the software you use and if that company goes out of business, so will yours if this software is a significant enough part of what your company do. You might not even be legally allowed to continue to use that software but even if you do you will not be able to modify it so it will quickly become obsolete.

With free software you own the program just as much as anyone else. This means that the work you do with that program is yours and that your company does not depend on the developer of the software. If the developer goes out of business, you are still allowed to use it and you can pay someone to add the features you need to that program.

An important thing to remember is that you are never forced to publish the changes you make to a free program. If you deem that the changes are an important part of your business you can keep them to yourself. The free software licenses (sometimes known as copyleft licenses) only states that \textit{if} you release your program, you \textit{must} give the users of that program the four essential freedoms defined in the free software definition.

\section{Solution}
Many software developers might ask themselves how would someone make money if they give away their program for free. The solution is to change the idea of what you provide as a software developer; it is not the software itself but the service you provide as a programmer.

How do lawyers make money? Anyone can buy a book that contains the laws of your country cheap or even find them free of charge online. Still companies typically have a team of lawyers which they pay a lot of money to assist them in legal cases. Instead they could hire a lawyer for every case, take notes on what the lawyers says and do in every legal case, and publish the notes online for other people to use in their legal cases. If everybody did this, soon enough nobody would need a lawyer, saving companies a lot of money so why is this not being done? Because you will rarely find two legal cases that are the same.\\
The same goes for software. A program is a solution and you will rarely find two problems that have the same solution. Over time the community will have a large pool of software that anyone can use but new problems will always arise as society develops which requires new ideas and solutions.
\newpage
\appendix
\section{Excerpts from a proprietary EULA}\label{apdx:eula}
In this section some excerpts from the end user license agreement for \textit{Advanced System Design (ADS)} by \textit{Keysight Technologies, Inc.} are presented. These excerpts do not include the entire paragraph in which they are presented so they could seem to be taken out of context and are only ment to prove that a proprietary license is very restrictive.
\\\\
You do not own the software and any improvements you make become the property of \textit{Keysight}.
\begin{quote}
  ``... Keysight Technologies, Inc. ("Keysight") grants you a \textbf{limited}, non-exclusive \textbf{license to use}, in accordance with one of the license types listed below, the Software, for the Term (as defined below), subject to the terms and conditions herein ... The Software and all copies thereof are licensed and \textbf{not sold to you}. The Software and all copies thereof are \textbf{owned and copyrighted by Keysight} or its third party suppliers ... Any \textbf{comments, suggestions, improvements} or other communications from you to Keysight regarding the Software ("Feed-back") \textbf{will become the sole property of Keysight} ..''
\end{quote}
You may not modify this program to make it do what you need.
\begin{quote}
  ``... You may not decompile, reverse engineer, disassemble, attempt to derive the source code of, decrypt, \textbf{modify}, create derivative works or \textbf{disable features of the Software} ...''
\end{quote}
Chalmers potentially violates this part of the license since you can remotely access \textit{ADS} using a secure shell connection.
\begin{quote}
  ``... You may not use the Software on any network that permits \textbf{remote access} to the Software outside of the network site designated or expressly agreed to in writing by Keysight ...''
\end{quote}
If you deliberatly show this program to anyone who have not accepted the same EULA, you are violating the license.
\begin{quote}
  ``... Except as expressly provided herein, \textbf{you may not permit} any third party, or anyone other than your employees whose duties require such access or \textbf{use}, to have \textbf{access} to, \textbf{view} the operations of, or otherwise use the Software, or \textbf{any information learned or observed} during the use of the Software ...''
\end{quote}
You also accept any other license included here.
\begin{quote}
  ``... The Software may contain third party software subject to third party notices and/or additional terms and conditions ... You agree that agreement to this EULA also \textbf{confirms your acceptance of any applicable third party software licenses} ...''
\end{quote}
... Keysight collects personal data from your and can share it with third parties if you violate the EULA.
\begin{quote}
  ``By using the Software, you agree that Keysight may \textbf{remotely monitor, transfer and collect data} about the unlicensed use including Internet Protocol addresses (IP addresses) and media access control addresses (MAC addresses) that \textbf{could be personally identifying data} ... Keysight may \textbf{use and share} the data with third parties ...''
\end{quote}
Your license can be revoked if you violate the EULA.
\begin{quote}
  ``... Keysight may \textbf{terminate this license} upon notice for breach of this EULA ...''
\end{quote}
\end{document}