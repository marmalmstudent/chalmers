\documentclass[12pt,a4paper]{article}

%\pdfoutput=1

\usepackage{pagecolor}
\definecolor{bgcolor}{HTML}{3F3F3F}
\definecolor{fgcolor}{HTML}{DCDCCC}
\pagecolor{bgcolor}
\color{fgcolor}
\usepackage[utf8]{inputenc}
\usepackage[T1]{fontenc}
\usepackage[english]{babel}
\usepackage{lmodern}
%\renewcommand*\rmdefault{ptm}
\usepackage{amsmath}
\usepackage{mathabx}
\usepackage{units}
\usepackage{siunitx}
\usepackage{icomma}
\usepackage{graphicx}
\usepackage{caption}
\usepackage{subcaption}
\usepackage{color}
\usepackage{pgf}
\DeclareMathOperator{\acosh}{arccosh}
\newcommand{\N}{\ensuremath{\mathbbm{N}}}
\newcommand{\Z}{\ensuremath{\mathbbm{Z}}}
\newcommand{\Q}{\ensuremath{\mathbbm{Q}}}
\newcommand{\R}{\ensuremath{\mathbbm{R}}}
\newcommand{\C}{\ensuremath{\mathbbm{C}}}
\newcommand{\rd}{\ensuremath{\mathrm{d}}}
\newcommand{\id}{\ensuremath{\,\rd}}
\usepackage{hyperref}
%\usepackage{a4wide} % puts the page numbering further down the page.
\usepackage{pdfpages}
\usepackage{epstopdf}
\DeclareGraphicsExtensions{.eps}
\title{Laser safety}
\author{Marcus Malmquist, marmalm, 941022}
\date{\today}

\begin{document}
\maketitle

\section{Answers: laser safety}\label{sec:1}

\subsection{1}
It is OK to put your hand in the laser beam for two reasons:
\begin{enumerate}
\item The beam is not focused on your hand, but it will be ($>10^5$ times higher) on the retina.
\item You can easely cope with a damaged hand but not with a damaged eye...
\end{enumerate}
\subsection{2}
According to the lecture notes it take about one fourth of a second.
\subsection{3}
The laser beam has a diameter of approximately $\SI{2}{\milli\metre}$, which is less than the diameter of a pupil, so all of the incident power ($\SI{2.5}{\milli\watt}$) will hit the retina. The pupil can be approximated to be a circular disk with a diameter of approximately $\SI{7}{\milli\metre}$ so the incident irradiance is therefor $I=\SI{65}{\watt\metre^{-2}}$.

The laser is assumed to be (nearly) a point source so the solid angle is very small. The maximum premissible exposure time can be calculated from \ref{eq:mpe}, where $C_6=1$ since the solid angle is small.
\begin{equation}
  \label{eq:mpe}
  It=18t^{0.75}C_6 \Leftrightarrow t=(18/I)^{\frac{1}{0.25}}=\SI{5.9}{\milli\second}
\end{equation}
\subsection{4}
No because $\SI{5.9}{\milli\second}<\SI{250}{\milli\second}$.
\subsection{5}
The exposure time would be roughly 40 times higher than the maximum allowed so it would be likely to cause permanent damage to the retina.
\subsection{6}
This $\SI{2.5}{\milli\watt}$ laser should be labeled as \textbf{Class 3R} as it allows up to $\SI{5}{\milli\watt}$ and the more restrictive Class 2(M) allows up to $\SI{1}{\milli\watt}$.

\end{document}