\documentclass[12pt,a4paper]{article}

%\pdfoutput=1

\usepackage[utf8]{inputenc}
\usepackage[T1]{fontenc}
\usepackage[english]{babel}
\usepackage{amsmath}
\usepackage{mathabx}
\usepackage{lmodern}
\usepackage{listings}
\usepackage{units}
\usepackage{siunitx}
\usepackage{icomma}
\usepackage{graphicx}
\usepackage{caption}
\usepackage{subcaption}
\usepackage{color}
\usepackage{pgf}
\DeclareMathOperator{\acosh}{arccosh}
\newcommand{\N}{\ensuremath{\mathbbm{N}}}
\newcommand{\Z}{\ensuremath{\mathbbm{Z}}}
\newcommand{\Q}{\ensuremath{\mathbbm{Q}}}
\newcommand{\R}{\ensuremath{\mathbbm{R}}}
\newcommand{\C}{\ensuremath{\mathbbm{C}}}
\newcommand{\rd}{\ensuremath{\mathrm{d}}}
\newcommand{\id}{\ensuremath{\,\rd}}
\usepackage{hyperref}
%\usepackage{a4wide} % puts the page numbering further down the page.
\usepackage{pdfpages}
\usepackage{epstopdf}
\DeclareGraphicsExtensions{.eps}
\def\changemargin#1#2{\list{}{\rightmargin#2\leftmargin#1}\item[]}
\let\endchangemargin=\endlist

\title{Handin 4}
\author{Marcus Malmquist, marmalm}
\date{\today}

\begin{document}
\maketitle

\section{Task 1}\label{sec:1}
All of the requested data can be read from Figure~\ref{fig:task1}. The velocity and the wave impeedance seem to be correct.
\begin{figure}
  \centering
  \noindent\makebox[\textwidth]{\scalebox{0.90}{%% Creator: Matplotlib, PGF backend
%%
%% To include the figure in your LaTeX document, write
%%   \input{<filename>.pgf}
%%
%% Make sure the required packages are loaded in your preamble
%%   \usepackage{pgf}
%%
%% Figures using additional raster images can only be included by \input if
%% they are in the same directory as the main LaTeX file. For loading figures
%% from other directories you can use the `import` package
%%   \usepackage{import}
%% and then include the figures with
%%   \import{<path to file>}{<filename>.pgf}
%%
%% Matplotlib used the following preamble
%%   \usepackage{fontspec}
%%   \setmainfont{DejaVu Serif}
%%   \setsansfont{DejaVu Sans}
%%   \setmonofont{DejaVu Sans Mono}
%%
\begingroup%
\makeatletter%
\begin{pgfpicture}%
\pgfpathrectangle{\pgfpointorigin}{\pgfqpoint{6.400000in}{4.800000in}}%
\pgfusepath{use as bounding box, clip}%
\begin{pgfscope}%
\pgfsetbuttcap%
\pgfsetmiterjoin%
\definecolor{currentfill}{rgb}{1.000000,1.000000,1.000000}%
\pgfsetfillcolor{currentfill}%
\pgfsetlinewidth{0.000000pt}%
\definecolor{currentstroke}{rgb}{1.000000,1.000000,1.000000}%
\pgfsetstrokecolor{currentstroke}%
\pgfsetdash{}{0pt}%
\pgfpathmoveto{\pgfqpoint{0.000000in}{0.000000in}}%
\pgfpathlineto{\pgfqpoint{6.400000in}{0.000000in}}%
\pgfpathlineto{\pgfqpoint{6.400000in}{4.800000in}}%
\pgfpathlineto{\pgfqpoint{0.000000in}{4.800000in}}%
\pgfpathclose%
\pgfusepath{fill}%
\end{pgfscope}%
\begin{pgfscope}%
\pgfsetbuttcap%
\pgfsetmiterjoin%
\definecolor{currentfill}{rgb}{1.000000,1.000000,1.000000}%
\pgfsetfillcolor{currentfill}%
\pgfsetlinewidth{0.000000pt}%
\definecolor{currentstroke}{rgb}{0.000000,0.000000,0.000000}%
\pgfsetstrokecolor{currentstroke}%
\pgfsetstrokeopacity{0.000000}%
\pgfsetdash{}{0pt}%
\pgfpathmoveto{\pgfqpoint{0.800000in}{0.528000in}}%
\pgfpathlineto{\pgfqpoint{5.760000in}{0.528000in}}%
\pgfpathlineto{\pgfqpoint{5.760000in}{4.224000in}}%
\pgfpathlineto{\pgfqpoint{0.800000in}{4.224000in}}%
\pgfpathclose%
\pgfusepath{fill}%
\end{pgfscope}%
\begin{pgfscope}%
\pgfsetbuttcap%
\pgfsetroundjoin%
\definecolor{currentfill}{rgb}{0.000000,0.000000,0.000000}%
\pgfsetfillcolor{currentfill}%
\pgfsetlinewidth{0.803000pt}%
\definecolor{currentstroke}{rgb}{0.000000,0.000000,0.000000}%
\pgfsetstrokecolor{currentstroke}%
\pgfsetdash{}{0pt}%
\pgfsys@defobject{currentmarker}{\pgfqpoint{0.000000in}{-0.048611in}}{\pgfqpoint{0.000000in}{0.000000in}}{%
\pgfpathmoveto{\pgfqpoint{0.000000in}{0.000000in}}%
\pgfpathlineto{\pgfqpoint{0.000000in}{-0.048611in}}%
\pgfusepath{stroke,fill}%
}%
\begin{pgfscope}%
\pgfsys@transformshift{0.800000in}{0.528000in}%
\pgfsys@useobject{currentmarker}{}%
\end{pgfscope}%
\end{pgfscope}%
\begin{pgfscope}%
\pgftext[x=0.800000in,y=0.430778in,,top]{\sffamily\fontsize{10.000000}{12.000000}\selectfont 0}%
\end{pgfscope}%
\begin{pgfscope}%
\pgfsetbuttcap%
\pgfsetroundjoin%
\definecolor{currentfill}{rgb}{0.000000,0.000000,0.000000}%
\pgfsetfillcolor{currentfill}%
\pgfsetlinewidth{0.803000pt}%
\definecolor{currentstroke}{rgb}{0.000000,0.000000,0.000000}%
\pgfsetstrokecolor{currentstroke}%
\pgfsetdash{}{0pt}%
\pgfsys@defobject{currentmarker}{\pgfqpoint{0.000000in}{-0.048611in}}{\pgfqpoint{0.000000in}{0.000000in}}{%
\pgfpathmoveto{\pgfqpoint{0.000000in}{0.000000in}}%
\pgfpathlineto{\pgfqpoint{0.000000in}{-0.048611in}}%
\pgfusepath{stroke,fill}%
}%
\begin{pgfscope}%
\pgfsys@transformshift{1.387678in}{0.528000in}%
\pgfsys@useobject{currentmarker}{}%
\end{pgfscope}%
\end{pgfscope}%
\begin{pgfscope}%
\pgftext[x=1.387678in,y=0.430778in,,top]{\sffamily\fontsize{10.000000}{12.000000}\selectfont 5}%
\end{pgfscope}%
\begin{pgfscope}%
\pgfsetbuttcap%
\pgfsetroundjoin%
\definecolor{currentfill}{rgb}{0.000000,0.000000,0.000000}%
\pgfsetfillcolor{currentfill}%
\pgfsetlinewidth{0.803000pt}%
\definecolor{currentstroke}{rgb}{0.000000,0.000000,0.000000}%
\pgfsetstrokecolor{currentstroke}%
\pgfsetdash{}{0pt}%
\pgfsys@defobject{currentmarker}{\pgfqpoint{0.000000in}{-0.048611in}}{\pgfqpoint{0.000000in}{0.000000in}}{%
\pgfpathmoveto{\pgfqpoint{0.000000in}{0.000000in}}%
\pgfpathlineto{\pgfqpoint{0.000000in}{-0.048611in}}%
\pgfusepath{stroke,fill}%
}%
\begin{pgfscope}%
\pgfsys@transformshift{1.975355in}{0.528000in}%
\pgfsys@useobject{currentmarker}{}%
\end{pgfscope}%
\end{pgfscope}%
\begin{pgfscope}%
\pgftext[x=1.975355in,y=0.430778in,,top]{\sffamily\fontsize{10.000000}{12.000000}\selectfont 10}%
\end{pgfscope}%
\begin{pgfscope}%
\pgfsetbuttcap%
\pgfsetroundjoin%
\definecolor{currentfill}{rgb}{0.000000,0.000000,0.000000}%
\pgfsetfillcolor{currentfill}%
\pgfsetlinewidth{0.803000pt}%
\definecolor{currentstroke}{rgb}{0.000000,0.000000,0.000000}%
\pgfsetstrokecolor{currentstroke}%
\pgfsetdash{}{0pt}%
\pgfsys@defobject{currentmarker}{\pgfqpoint{0.000000in}{-0.048611in}}{\pgfqpoint{0.000000in}{0.000000in}}{%
\pgfpathmoveto{\pgfqpoint{0.000000in}{0.000000in}}%
\pgfpathlineto{\pgfqpoint{0.000000in}{-0.048611in}}%
\pgfusepath{stroke,fill}%
}%
\begin{pgfscope}%
\pgfsys@transformshift{2.563033in}{0.528000in}%
\pgfsys@useobject{currentmarker}{}%
\end{pgfscope}%
\end{pgfscope}%
\begin{pgfscope}%
\pgftext[x=2.563033in,y=0.430778in,,top]{\sffamily\fontsize{10.000000}{12.000000}\selectfont 15}%
\end{pgfscope}%
\begin{pgfscope}%
\pgfsetbuttcap%
\pgfsetroundjoin%
\definecolor{currentfill}{rgb}{0.000000,0.000000,0.000000}%
\pgfsetfillcolor{currentfill}%
\pgfsetlinewidth{0.803000pt}%
\definecolor{currentstroke}{rgb}{0.000000,0.000000,0.000000}%
\pgfsetstrokecolor{currentstroke}%
\pgfsetdash{}{0pt}%
\pgfsys@defobject{currentmarker}{\pgfqpoint{0.000000in}{-0.048611in}}{\pgfqpoint{0.000000in}{0.000000in}}{%
\pgfpathmoveto{\pgfqpoint{0.000000in}{0.000000in}}%
\pgfpathlineto{\pgfqpoint{0.000000in}{-0.048611in}}%
\pgfusepath{stroke,fill}%
}%
\begin{pgfscope}%
\pgfsys@transformshift{3.150711in}{0.528000in}%
\pgfsys@useobject{currentmarker}{}%
\end{pgfscope}%
\end{pgfscope}%
\begin{pgfscope}%
\pgftext[x=3.150711in,y=0.430778in,,top]{\sffamily\fontsize{10.000000}{12.000000}\selectfont 20}%
\end{pgfscope}%
\begin{pgfscope}%
\pgfsetbuttcap%
\pgfsetroundjoin%
\definecolor{currentfill}{rgb}{0.000000,0.000000,0.000000}%
\pgfsetfillcolor{currentfill}%
\pgfsetlinewidth{0.803000pt}%
\definecolor{currentstroke}{rgb}{0.000000,0.000000,0.000000}%
\pgfsetstrokecolor{currentstroke}%
\pgfsetdash{}{0pt}%
\pgfsys@defobject{currentmarker}{\pgfqpoint{0.000000in}{-0.048611in}}{\pgfqpoint{0.000000in}{0.000000in}}{%
\pgfpathmoveto{\pgfqpoint{0.000000in}{0.000000in}}%
\pgfpathlineto{\pgfqpoint{0.000000in}{-0.048611in}}%
\pgfusepath{stroke,fill}%
}%
\begin{pgfscope}%
\pgfsys@transformshift{3.738389in}{0.528000in}%
\pgfsys@useobject{currentmarker}{}%
\end{pgfscope}%
\end{pgfscope}%
\begin{pgfscope}%
\pgftext[x=3.738389in,y=0.430778in,,top]{\sffamily\fontsize{10.000000}{12.000000}\selectfont 25}%
\end{pgfscope}%
\begin{pgfscope}%
\pgfsetbuttcap%
\pgfsetroundjoin%
\definecolor{currentfill}{rgb}{0.000000,0.000000,0.000000}%
\pgfsetfillcolor{currentfill}%
\pgfsetlinewidth{0.803000pt}%
\definecolor{currentstroke}{rgb}{0.000000,0.000000,0.000000}%
\pgfsetstrokecolor{currentstroke}%
\pgfsetdash{}{0pt}%
\pgfsys@defobject{currentmarker}{\pgfqpoint{0.000000in}{-0.048611in}}{\pgfqpoint{0.000000in}{0.000000in}}{%
\pgfpathmoveto{\pgfqpoint{0.000000in}{0.000000in}}%
\pgfpathlineto{\pgfqpoint{0.000000in}{-0.048611in}}%
\pgfusepath{stroke,fill}%
}%
\begin{pgfscope}%
\pgfsys@transformshift{4.326066in}{0.528000in}%
\pgfsys@useobject{currentmarker}{}%
\end{pgfscope}%
\end{pgfscope}%
\begin{pgfscope}%
\pgftext[x=4.326066in,y=0.430778in,,top]{\sffamily\fontsize{10.000000}{12.000000}\selectfont 30}%
\end{pgfscope}%
\begin{pgfscope}%
\pgfsetbuttcap%
\pgfsetroundjoin%
\definecolor{currentfill}{rgb}{0.000000,0.000000,0.000000}%
\pgfsetfillcolor{currentfill}%
\pgfsetlinewidth{0.803000pt}%
\definecolor{currentstroke}{rgb}{0.000000,0.000000,0.000000}%
\pgfsetstrokecolor{currentstroke}%
\pgfsetdash{}{0pt}%
\pgfsys@defobject{currentmarker}{\pgfqpoint{0.000000in}{-0.048611in}}{\pgfqpoint{0.000000in}{0.000000in}}{%
\pgfpathmoveto{\pgfqpoint{0.000000in}{0.000000in}}%
\pgfpathlineto{\pgfqpoint{0.000000in}{-0.048611in}}%
\pgfusepath{stroke,fill}%
}%
\begin{pgfscope}%
\pgfsys@transformshift{4.913744in}{0.528000in}%
\pgfsys@useobject{currentmarker}{}%
\end{pgfscope}%
\end{pgfscope}%
\begin{pgfscope}%
\pgftext[x=4.913744in,y=0.430778in,,top]{\sffamily\fontsize{10.000000}{12.000000}\selectfont 35}%
\end{pgfscope}%
\begin{pgfscope}%
\pgfsetbuttcap%
\pgfsetroundjoin%
\definecolor{currentfill}{rgb}{0.000000,0.000000,0.000000}%
\pgfsetfillcolor{currentfill}%
\pgfsetlinewidth{0.803000pt}%
\definecolor{currentstroke}{rgb}{0.000000,0.000000,0.000000}%
\pgfsetstrokecolor{currentstroke}%
\pgfsetdash{}{0pt}%
\pgfsys@defobject{currentmarker}{\pgfqpoint{0.000000in}{-0.048611in}}{\pgfqpoint{0.000000in}{0.000000in}}{%
\pgfpathmoveto{\pgfqpoint{0.000000in}{0.000000in}}%
\pgfpathlineto{\pgfqpoint{0.000000in}{-0.048611in}}%
\pgfusepath{stroke,fill}%
}%
\begin{pgfscope}%
\pgfsys@transformshift{5.501422in}{0.528000in}%
\pgfsys@useobject{currentmarker}{}%
\end{pgfscope}%
\end{pgfscope}%
\begin{pgfscope}%
\pgftext[x=5.501422in,y=0.430778in,,top]{\sffamily\fontsize{10.000000}{12.000000}\selectfont 40}%
\end{pgfscope}%
\begin{pgfscope}%
\pgftext[x=3.280000in,y=0.240809in,,top]{\sffamily\fontsize{16.000000}{19.200000}\selectfont \(\displaystyle z-position [\mu m]\)}%
\end{pgfscope}%
\begin{pgfscope}%
\pgfsetbuttcap%
\pgfsetroundjoin%
\definecolor{currentfill}{rgb}{0.000000,0.000000,0.000000}%
\pgfsetfillcolor{currentfill}%
\pgfsetlinewidth{0.803000pt}%
\definecolor{currentstroke}{rgb}{0.000000,0.000000,0.000000}%
\pgfsetstrokecolor{currentstroke}%
\pgfsetdash{}{0pt}%
\pgfsys@defobject{currentmarker}{\pgfqpoint{-0.048611in}{0.000000in}}{\pgfqpoint{0.000000in}{0.000000in}}{%
\pgfpathmoveto{\pgfqpoint{0.000000in}{0.000000in}}%
\pgfpathlineto{\pgfqpoint{-0.048611in}{0.000000in}}%
\pgfusepath{stroke,fill}%
}%
\begin{pgfscope}%
\pgfsys@transformshift{0.800000in}{0.528000in}%
\pgfsys@useobject{currentmarker}{}%
\end{pgfscope}%
\end{pgfscope}%
\begin{pgfscope}%
\pgftext[x=0.365525in,y=0.475238in,left,base]{\sffamily\fontsize{10.000000}{12.000000}\selectfont -2.0}%
\end{pgfscope}%
\begin{pgfscope}%
\pgfsetbuttcap%
\pgfsetroundjoin%
\definecolor{currentfill}{rgb}{0.000000,0.000000,0.000000}%
\pgfsetfillcolor{currentfill}%
\pgfsetlinewidth{0.803000pt}%
\definecolor{currentstroke}{rgb}{0.000000,0.000000,0.000000}%
\pgfsetstrokecolor{currentstroke}%
\pgfsetdash{}{0pt}%
\pgfsys@defobject{currentmarker}{\pgfqpoint{-0.048611in}{0.000000in}}{\pgfqpoint{0.000000in}{0.000000in}}{%
\pgfpathmoveto{\pgfqpoint{0.000000in}{0.000000in}}%
\pgfpathlineto{\pgfqpoint{-0.048611in}{0.000000in}}%
\pgfusepath{stroke,fill}%
}%
\begin{pgfscope}%
\pgfsys@transformshift{0.800000in}{0.990000in}%
\pgfsys@useobject{currentmarker}{}%
\end{pgfscope}%
\end{pgfscope}%
\begin{pgfscope}%
\pgftext[x=0.365525in,y=0.937238in,left,base]{\sffamily\fontsize{10.000000}{12.000000}\selectfont -1.5}%
\end{pgfscope}%
\begin{pgfscope}%
\pgfsetbuttcap%
\pgfsetroundjoin%
\definecolor{currentfill}{rgb}{0.000000,0.000000,0.000000}%
\pgfsetfillcolor{currentfill}%
\pgfsetlinewidth{0.803000pt}%
\definecolor{currentstroke}{rgb}{0.000000,0.000000,0.000000}%
\pgfsetstrokecolor{currentstroke}%
\pgfsetdash{}{0pt}%
\pgfsys@defobject{currentmarker}{\pgfqpoint{-0.048611in}{0.000000in}}{\pgfqpoint{0.000000in}{0.000000in}}{%
\pgfpathmoveto{\pgfqpoint{0.000000in}{0.000000in}}%
\pgfpathlineto{\pgfqpoint{-0.048611in}{0.000000in}}%
\pgfusepath{stroke,fill}%
}%
\begin{pgfscope}%
\pgfsys@transformshift{0.800000in}{1.452000in}%
\pgfsys@useobject{currentmarker}{}%
\end{pgfscope}%
\end{pgfscope}%
\begin{pgfscope}%
\pgftext[x=0.365525in,y=1.399238in,left,base]{\sffamily\fontsize{10.000000}{12.000000}\selectfont -1.0}%
\end{pgfscope}%
\begin{pgfscope}%
\pgfsetbuttcap%
\pgfsetroundjoin%
\definecolor{currentfill}{rgb}{0.000000,0.000000,0.000000}%
\pgfsetfillcolor{currentfill}%
\pgfsetlinewidth{0.803000pt}%
\definecolor{currentstroke}{rgb}{0.000000,0.000000,0.000000}%
\pgfsetstrokecolor{currentstroke}%
\pgfsetdash{}{0pt}%
\pgfsys@defobject{currentmarker}{\pgfqpoint{-0.048611in}{0.000000in}}{\pgfqpoint{0.000000in}{0.000000in}}{%
\pgfpathmoveto{\pgfqpoint{0.000000in}{0.000000in}}%
\pgfpathlineto{\pgfqpoint{-0.048611in}{0.000000in}}%
\pgfusepath{stroke,fill}%
}%
\begin{pgfscope}%
\pgfsys@transformshift{0.800000in}{1.914000in}%
\pgfsys@useobject{currentmarker}{}%
\end{pgfscope}%
\end{pgfscope}%
\begin{pgfscope}%
\pgftext[x=0.365525in,y=1.861238in,left,base]{\sffamily\fontsize{10.000000}{12.000000}\selectfont -0.5}%
\end{pgfscope}%
\begin{pgfscope}%
\pgfsetbuttcap%
\pgfsetroundjoin%
\definecolor{currentfill}{rgb}{0.000000,0.000000,0.000000}%
\pgfsetfillcolor{currentfill}%
\pgfsetlinewidth{0.803000pt}%
\definecolor{currentstroke}{rgb}{0.000000,0.000000,0.000000}%
\pgfsetstrokecolor{currentstroke}%
\pgfsetdash{}{0pt}%
\pgfsys@defobject{currentmarker}{\pgfqpoint{-0.048611in}{0.000000in}}{\pgfqpoint{0.000000in}{0.000000in}}{%
\pgfpathmoveto{\pgfqpoint{0.000000in}{0.000000in}}%
\pgfpathlineto{\pgfqpoint{-0.048611in}{0.000000in}}%
\pgfusepath{stroke,fill}%
}%
\begin{pgfscope}%
\pgfsys@transformshift{0.800000in}{2.376000in}%
\pgfsys@useobject{currentmarker}{}%
\end{pgfscope}%
\end{pgfscope}%
\begin{pgfscope}%
\pgftext[x=0.481898in,y=2.323238in,left,base]{\sffamily\fontsize{10.000000}{12.000000}\selectfont 0.0}%
\end{pgfscope}%
\begin{pgfscope}%
\pgfsetbuttcap%
\pgfsetroundjoin%
\definecolor{currentfill}{rgb}{0.000000,0.000000,0.000000}%
\pgfsetfillcolor{currentfill}%
\pgfsetlinewidth{0.803000pt}%
\definecolor{currentstroke}{rgb}{0.000000,0.000000,0.000000}%
\pgfsetstrokecolor{currentstroke}%
\pgfsetdash{}{0pt}%
\pgfsys@defobject{currentmarker}{\pgfqpoint{-0.048611in}{0.000000in}}{\pgfqpoint{0.000000in}{0.000000in}}{%
\pgfpathmoveto{\pgfqpoint{0.000000in}{0.000000in}}%
\pgfpathlineto{\pgfqpoint{-0.048611in}{0.000000in}}%
\pgfusepath{stroke,fill}%
}%
\begin{pgfscope}%
\pgfsys@transformshift{0.800000in}{2.838000in}%
\pgfsys@useobject{currentmarker}{}%
\end{pgfscope}%
\end{pgfscope}%
\begin{pgfscope}%
\pgftext[x=0.481898in,y=2.785238in,left,base]{\sffamily\fontsize{10.000000}{12.000000}\selectfont 0.5}%
\end{pgfscope}%
\begin{pgfscope}%
\pgfsetbuttcap%
\pgfsetroundjoin%
\definecolor{currentfill}{rgb}{0.000000,0.000000,0.000000}%
\pgfsetfillcolor{currentfill}%
\pgfsetlinewidth{0.803000pt}%
\definecolor{currentstroke}{rgb}{0.000000,0.000000,0.000000}%
\pgfsetstrokecolor{currentstroke}%
\pgfsetdash{}{0pt}%
\pgfsys@defobject{currentmarker}{\pgfqpoint{-0.048611in}{0.000000in}}{\pgfqpoint{0.000000in}{0.000000in}}{%
\pgfpathmoveto{\pgfqpoint{0.000000in}{0.000000in}}%
\pgfpathlineto{\pgfqpoint{-0.048611in}{0.000000in}}%
\pgfusepath{stroke,fill}%
}%
\begin{pgfscope}%
\pgfsys@transformshift{0.800000in}{3.300000in}%
\pgfsys@useobject{currentmarker}{}%
\end{pgfscope}%
\end{pgfscope}%
\begin{pgfscope}%
\pgftext[x=0.481898in,y=3.247238in,left,base]{\sffamily\fontsize{10.000000}{12.000000}\selectfont 1.0}%
\end{pgfscope}%
\begin{pgfscope}%
\pgfsetbuttcap%
\pgfsetroundjoin%
\definecolor{currentfill}{rgb}{0.000000,0.000000,0.000000}%
\pgfsetfillcolor{currentfill}%
\pgfsetlinewidth{0.803000pt}%
\definecolor{currentstroke}{rgb}{0.000000,0.000000,0.000000}%
\pgfsetstrokecolor{currentstroke}%
\pgfsetdash{}{0pt}%
\pgfsys@defobject{currentmarker}{\pgfqpoint{-0.048611in}{0.000000in}}{\pgfqpoint{0.000000in}{0.000000in}}{%
\pgfpathmoveto{\pgfqpoint{0.000000in}{0.000000in}}%
\pgfpathlineto{\pgfqpoint{-0.048611in}{0.000000in}}%
\pgfusepath{stroke,fill}%
}%
\begin{pgfscope}%
\pgfsys@transformshift{0.800000in}{3.762000in}%
\pgfsys@useobject{currentmarker}{}%
\end{pgfscope}%
\end{pgfscope}%
\begin{pgfscope}%
\pgftext[x=0.481898in,y=3.709238in,left,base]{\sffamily\fontsize{10.000000}{12.000000}\selectfont 1.5}%
\end{pgfscope}%
\begin{pgfscope}%
\pgfsetbuttcap%
\pgfsetroundjoin%
\definecolor{currentfill}{rgb}{0.000000,0.000000,0.000000}%
\pgfsetfillcolor{currentfill}%
\pgfsetlinewidth{0.803000pt}%
\definecolor{currentstroke}{rgb}{0.000000,0.000000,0.000000}%
\pgfsetstrokecolor{currentstroke}%
\pgfsetdash{}{0pt}%
\pgfsys@defobject{currentmarker}{\pgfqpoint{-0.048611in}{0.000000in}}{\pgfqpoint{0.000000in}{0.000000in}}{%
\pgfpathmoveto{\pgfqpoint{0.000000in}{0.000000in}}%
\pgfpathlineto{\pgfqpoint{-0.048611in}{0.000000in}}%
\pgfusepath{stroke,fill}%
}%
\begin{pgfscope}%
\pgfsys@transformshift{0.800000in}{4.224000in}%
\pgfsys@useobject{currentmarker}{}%
\end{pgfscope}%
\end{pgfscope}%
\begin{pgfscope}%
\pgftext[x=0.481898in,y=4.171238in,left,base]{\sffamily\fontsize{10.000000}{12.000000}\selectfont 2.0}%
\end{pgfscope}%
\begin{pgfscope}%
\pgftext[x=0.309969in,y=2.376000in,,bottom,rotate=90.000000]{\sffamily\fontsize{16.000000}{19.200000}\selectfont \(\displaystyle E-field\)}%
\end{pgfscope}%
\begin{pgfscope}%
\pgfpathrectangle{\pgfqpoint{0.800000in}{0.528000in}}{\pgfqpoint{4.960000in}{3.696000in}} %
\pgfusepath{clip}%
\pgfsetrectcap%
\pgfsetroundjoin%
\pgfsetlinewidth{1.505625pt}%
\definecolor{currentstroke}{rgb}{0.000000,0.000000,0.000000}%
\pgfsetstrokecolor{currentstroke}%
\pgfsetdash{}{0pt}%
\pgfpathmoveto{\pgfqpoint{0.800000in}{2.376000in}}%
\pgfpathlineto{\pgfqpoint{4.182720in}{2.377067in}}%
\pgfpathlineto{\pgfqpoint{4.190160in}{2.381524in}}%
\pgfpathlineto{\pgfqpoint{4.205040in}{2.393425in}}%
\pgfpathlineto{\pgfqpoint{4.207520in}{2.393024in}}%
\pgfpathlineto{\pgfqpoint{4.210000in}{2.390930in}}%
\pgfpathlineto{\pgfqpoint{4.212480in}{2.386761in}}%
\pgfpathlineto{\pgfqpoint{4.217440in}{2.371137in}}%
\pgfpathlineto{\pgfqpoint{4.222400in}{2.345622in}}%
\pgfpathlineto{\pgfqpoint{4.234800in}{2.267676in}}%
\pgfpathlineto{\pgfqpoint{4.237280in}{2.258694in}}%
\pgfpathlineto{\pgfqpoint{4.239760in}{2.255222in}}%
\pgfpathlineto{\pgfqpoint{4.242240in}{2.258570in}}%
\pgfpathlineto{\pgfqpoint{4.244720in}{2.269786in}}%
\pgfpathlineto{\pgfqpoint{4.249680in}{2.317996in}}%
\pgfpathlineto{\pgfqpoint{4.254640in}{2.398876in}}%
\pgfpathlineto{\pgfqpoint{4.269520in}{2.690827in}}%
\pgfpathlineto{\pgfqpoint{4.272000in}{2.715921in}}%
\pgfpathlineto{\pgfqpoint{4.274480in}{2.726276in}}%
\pgfpathlineto{\pgfqpoint{4.276960in}{2.719817in}}%
\pgfpathlineto{\pgfqpoint{4.279440in}{2.695263in}}%
\pgfpathlineto{\pgfqpoint{4.284400in}{2.591532in}}%
\pgfpathlineto{\pgfqpoint{4.289360in}{2.424820in}}%
\pgfpathlineto{\pgfqpoint{4.304240in}{1.855870in}}%
\pgfpathlineto{\pgfqpoint{4.306720in}{1.803525in}}%
\pgfpathlineto{\pgfqpoint{4.309200in}{1.775210in}}%
\pgfpathlineto{\pgfqpoint{4.311680in}{1.773548in}}%
\pgfpathlineto{\pgfqpoint{4.314160in}{1.799905in}}%
\pgfpathlineto{\pgfqpoint{4.316640in}{1.854274in}}%
\pgfpathlineto{\pgfqpoint{4.321600in}{2.039943in}}%
\pgfpathlineto{\pgfqpoint{4.329040in}{2.450170in}}%
\pgfpathlineto{\pgfqpoint{4.338960in}{2.985332in}}%
\pgfpathlineto{\pgfqpoint{4.343920in}{3.131461in}}%
\pgfpathlineto{\pgfqpoint{4.346400in}{3.157285in}}%
\pgfpathlineto{\pgfqpoint{4.348880in}{3.148560in}}%
\pgfpathlineto{\pgfqpoint{4.351360in}{3.104965in}}%
\pgfpathlineto{\pgfqpoint{4.356320in}{2.920037in}}%
\pgfpathlineto{\pgfqpoint{4.363760in}{2.463068in}}%
\pgfpathlineto{\pgfqpoint{4.373680in}{1.803971in}}%
\pgfpathlineto{\pgfqpoint{4.378640in}{1.586619in}}%
\pgfpathlineto{\pgfqpoint{4.381120in}{1.526687in}}%
\pgfpathlineto{\pgfqpoint{4.383600in}{1.503760in}}%
\pgfpathlineto{\pgfqpoint{4.386080in}{1.519189in}}%
\pgfpathlineto{\pgfqpoint{4.388560in}{1.572592in}}%
\pgfpathlineto{\pgfqpoint{4.393520in}{1.783302in}}%
\pgfpathlineto{\pgfqpoint{4.400960in}{2.282569in}}%
\pgfpathlineto{\pgfqpoint{4.410880in}{2.979745in}}%
\pgfpathlineto{\pgfqpoint{4.415840in}{3.203381in}}%
\pgfpathlineto{\pgfqpoint{4.418320in}{3.263376in}}%
\pgfpathlineto{\pgfqpoint{4.420800in}{3.284610in}}%
\pgfpathlineto{\pgfqpoint{4.423280in}{3.266032in}}%
\pgfpathlineto{\pgfqpoint{4.425760in}{3.208354in}}%
\pgfpathlineto{\pgfqpoint{4.430720in}{2.987100in}}%
\pgfpathlineto{\pgfqpoint{4.438160in}{2.471739in}}%
\pgfpathlineto{\pgfqpoint{4.448080in}{1.761372in}}%
\pgfpathlineto{\pgfqpoint{4.453040in}{1.535944in}}%
\pgfpathlineto{\pgfqpoint{4.455520in}{1.476108in}}%
\pgfpathlineto{\pgfqpoint{4.458000in}{1.455602in}}%
\pgfpathlineto{\pgfqpoint{4.460480in}{1.475357in}}%
\pgfpathlineto{\pgfqpoint{4.462960in}{1.534536in}}%
\pgfpathlineto{\pgfqpoint{4.467920in}{1.759279in}}%
\pgfpathlineto{\pgfqpoint{4.475360in}{2.279588in}}%
\pgfpathlineto{\pgfqpoint{4.485280in}{2.993598in}}%
\pgfpathlineto{\pgfqpoint{4.490240in}{3.219397in}}%
\pgfpathlineto{\pgfqpoint{4.492720in}{3.279133in}}%
\pgfpathlineto{\pgfqpoint{4.495200in}{3.299395in}}%
\pgfpathlineto{\pgfqpoint{4.497680in}{3.279291in}}%
\pgfpathlineto{\pgfqpoint{4.500160in}{3.219694in}}%
\pgfpathlineto{\pgfqpoint{4.505120in}{2.994043in}}%
\pgfpathlineto{\pgfqpoint{4.512560in}{2.472561in}}%
\pgfpathlineto{\pgfqpoint{4.522480in}{1.757803in}}%
\pgfpathlineto{\pgfqpoint{4.527440in}{1.531962in}}%
\pgfpathlineto{\pgfqpoint{4.529920in}{1.472262in}}%
\pgfpathlineto{\pgfqpoint{4.532400in}{1.452061in}}%
\pgfpathlineto{\pgfqpoint{4.534880in}{1.472241in}}%
\pgfpathlineto{\pgfqpoint{4.537360in}{1.531922in}}%
\pgfpathlineto{\pgfqpoint{4.542320in}{1.757743in}}%
\pgfpathlineto{\pgfqpoint{4.549760in}{2.279417in}}%
\pgfpathlineto{\pgfqpoint{4.559680in}{2.994272in}}%
\pgfpathlineto{\pgfqpoint{4.564640in}{3.220112in}}%
\pgfpathlineto{\pgfqpoint{4.567120in}{3.279805in}}%
\pgfpathlineto{\pgfqpoint{4.569600in}{3.299998in}}%
\pgfpathlineto{\pgfqpoint{4.572080in}{3.279807in}}%
\pgfpathlineto{\pgfqpoint{4.574560in}{3.220115in}}%
\pgfpathlineto{\pgfqpoint{4.579520in}{2.994276in}}%
\pgfpathlineto{\pgfqpoint{4.586960in}{2.472584in}}%
\pgfpathlineto{\pgfqpoint{4.596880in}{1.757723in}}%
\pgfpathlineto{\pgfqpoint{4.601840in}{1.531884in}}%
\pgfpathlineto{\pgfqpoint{4.604320in}{1.472192in}}%
\pgfpathlineto{\pgfqpoint{4.606800in}{1.452000in}}%
\pgfpathlineto{\pgfqpoint{4.609280in}{1.472192in}}%
\pgfpathlineto{\pgfqpoint{4.611760in}{1.531884in}}%
\pgfpathlineto{\pgfqpoint{4.616720in}{1.757723in}}%
\pgfpathlineto{\pgfqpoint{4.624160in}{2.279416in}}%
\pgfpathlineto{\pgfqpoint{4.634080in}{2.994277in}}%
\pgfpathlineto{\pgfqpoint{4.639040in}{3.220116in}}%
\pgfpathlineto{\pgfqpoint{4.641520in}{3.279808in}}%
\pgfpathlineto{\pgfqpoint{4.644000in}{3.300000in}}%
\pgfpathlineto{\pgfqpoint{4.646480in}{3.279808in}}%
\pgfpathlineto{\pgfqpoint{4.648960in}{3.220116in}}%
\pgfpathlineto{\pgfqpoint{4.653920in}{2.994277in}}%
\pgfpathlineto{\pgfqpoint{4.661360in}{2.472584in}}%
\pgfpathlineto{\pgfqpoint{4.671280in}{1.757723in}}%
\pgfpathlineto{\pgfqpoint{4.676240in}{1.531884in}}%
\pgfpathlineto{\pgfqpoint{4.678720in}{1.472192in}}%
\pgfpathlineto{\pgfqpoint{4.681200in}{1.452000in}}%
\pgfpathlineto{\pgfqpoint{4.683680in}{1.472192in}}%
\pgfpathlineto{\pgfqpoint{4.686160in}{1.531884in}}%
\pgfpathlineto{\pgfqpoint{4.691120in}{1.757723in}}%
\pgfpathlineto{\pgfqpoint{4.698560in}{2.279416in}}%
\pgfpathlineto{\pgfqpoint{4.708480in}{2.994276in}}%
\pgfpathlineto{\pgfqpoint{4.713440in}{3.220115in}}%
\pgfpathlineto{\pgfqpoint{4.715920in}{3.279807in}}%
\pgfpathlineto{\pgfqpoint{4.718400in}{3.299998in}}%
\pgfpathlineto{\pgfqpoint{4.720880in}{3.279805in}}%
\pgfpathlineto{\pgfqpoint{4.723360in}{3.220112in}}%
\pgfpathlineto{\pgfqpoint{4.728320in}{2.994272in}}%
\pgfpathlineto{\pgfqpoint{4.735760in}{2.472583in}}%
\pgfpathlineto{\pgfqpoint{4.745680in}{1.757743in}}%
\pgfpathlineto{\pgfqpoint{4.750640in}{1.531922in}}%
\pgfpathlineto{\pgfqpoint{4.753120in}{1.472241in}}%
\pgfpathlineto{\pgfqpoint{4.755600in}{1.452061in}}%
\pgfpathlineto{\pgfqpoint{4.758080in}{1.472262in}}%
\pgfpathlineto{\pgfqpoint{4.760560in}{1.531962in}}%
\pgfpathlineto{\pgfqpoint{4.765520in}{1.757803in}}%
\pgfpathlineto{\pgfqpoint{4.772960in}{2.279436in}}%
\pgfpathlineto{\pgfqpoint{4.782880in}{2.994043in}}%
\pgfpathlineto{\pgfqpoint{4.787840in}{3.219694in}}%
\pgfpathlineto{\pgfqpoint{4.790320in}{3.279291in}}%
\pgfpathlineto{\pgfqpoint{4.792800in}{3.299395in}}%
\pgfpathlineto{\pgfqpoint{4.795280in}{3.279133in}}%
\pgfpathlineto{\pgfqpoint{4.797760in}{3.219397in}}%
\pgfpathlineto{\pgfqpoint{4.802720in}{2.993598in}}%
\pgfpathlineto{\pgfqpoint{4.810160in}{2.472431in}}%
\pgfpathlineto{\pgfqpoint{4.820080in}{1.759279in}}%
\pgfpathlineto{\pgfqpoint{4.825040in}{1.534536in}}%
\pgfpathlineto{\pgfqpoint{4.827520in}{1.475357in}}%
\pgfpathlineto{\pgfqpoint{4.830000in}{1.455602in}}%
\pgfpathlineto{\pgfqpoint{4.832480in}{1.476108in}}%
\pgfpathlineto{\pgfqpoint{4.834960in}{1.535944in}}%
\pgfpathlineto{\pgfqpoint{4.839920in}{1.761372in}}%
\pgfpathlineto{\pgfqpoint{4.847360in}{2.280183in}}%
\pgfpathlineto{\pgfqpoint{4.857280in}{2.987100in}}%
\pgfpathlineto{\pgfqpoint{4.862240in}{3.208354in}}%
\pgfpathlineto{\pgfqpoint{4.864720in}{3.266032in}}%
\pgfpathlineto{\pgfqpoint{4.867200in}{3.284610in}}%
\pgfpathlineto{\pgfqpoint{4.869680in}{3.263376in}}%
\pgfpathlineto{\pgfqpoint{4.872160in}{3.203381in}}%
\pgfpathlineto{\pgfqpoint{4.877120in}{2.979745in}}%
\pgfpathlineto{\pgfqpoint{4.884560in}{2.469675in}}%
\pgfpathlineto{\pgfqpoint{4.894480in}{1.783302in}}%
\pgfpathlineto{\pgfqpoint{4.899440in}{1.572592in}}%
\pgfpathlineto{\pgfqpoint{4.901920in}{1.519189in}}%
\pgfpathlineto{\pgfqpoint{4.904400in}{1.503760in}}%
\pgfpathlineto{\pgfqpoint{4.906880in}{1.526687in}}%
\pgfpathlineto{\pgfqpoint{4.909360in}{1.586619in}}%
\pgfpathlineto{\pgfqpoint{4.914320in}{1.803971in}}%
\pgfpathlineto{\pgfqpoint{4.921760in}{2.288310in}}%
\pgfpathlineto{\pgfqpoint{4.931680in}{2.920037in}}%
\pgfpathlineto{\pgfqpoint{4.936640in}{3.104965in}}%
\pgfpathlineto{\pgfqpoint{4.939120in}{3.148560in}}%
\pgfpathlineto{\pgfqpoint{4.941600in}{3.157285in}}%
\pgfpathlineto{\pgfqpoint{4.944080in}{3.131461in}}%
\pgfpathlineto{\pgfqpoint{4.946560in}{3.073009in}}%
\pgfpathlineto{\pgfqpoint{4.951520in}{2.873125in}}%
\pgfpathlineto{\pgfqpoint{4.971360in}{1.854274in}}%
\pgfpathlineto{\pgfqpoint{4.973840in}{1.799905in}}%
\pgfpathlineto{\pgfqpoint{4.976320in}{1.773548in}}%
\pgfpathlineto{\pgfqpoint{4.978800in}{1.775210in}}%
\pgfpathlineto{\pgfqpoint{4.981280in}{1.803525in}}%
\pgfpathlineto{\pgfqpoint{4.986240in}{1.928553in}}%
\pgfpathlineto{\pgfqpoint{4.993680in}{2.220789in}}%
\pgfpathlineto{\pgfqpoint{5.001120in}{2.514790in}}%
\pgfpathlineto{\pgfqpoint{5.006080in}{2.652272in}}%
\pgfpathlineto{\pgfqpoint{5.011040in}{2.719817in}}%
\pgfpathlineto{\pgfqpoint{5.013520in}{2.726276in}}%
\pgfpathlineto{\pgfqpoint{5.016000in}{2.715921in}}%
\pgfpathlineto{\pgfqpoint{5.018480in}{2.690827in}}%
\pgfpathlineto{\pgfqpoint{5.023440in}{2.607624in}}%
\pgfpathlineto{\pgfqpoint{5.038320in}{2.317996in}}%
\pgfpathlineto{\pgfqpoint{5.043280in}{2.269786in}}%
\pgfpathlineto{\pgfqpoint{5.045760in}{2.258570in}}%
\pgfpathlineto{\pgfqpoint{5.048240in}{2.255222in}}%
\pgfpathlineto{\pgfqpoint{5.050720in}{2.258694in}}%
\pgfpathlineto{\pgfqpoint{5.055680in}{2.280716in}}%
\pgfpathlineto{\pgfqpoint{5.070560in}{2.371137in}}%
\pgfpathlineto{\pgfqpoint{5.075520in}{2.386761in}}%
\pgfpathlineto{\pgfqpoint{5.080480in}{2.393024in}}%
\pgfpathlineto{\pgfqpoint{5.082960in}{2.393425in}}%
\pgfpathlineto{\pgfqpoint{5.087920in}{2.390794in}}%
\pgfpathlineto{\pgfqpoint{5.102800in}{2.378186in}}%
\pgfpathlineto{\pgfqpoint{5.110240in}{2.375781in}}%
\pgfpathlineto{\pgfqpoint{5.120160in}{2.375445in}}%
\pgfpathlineto{\pgfqpoint{5.162320in}{2.376001in}}%
\pgfpathlineto{\pgfqpoint{5.757520in}{2.376000in}}%
\pgfpathlineto{\pgfqpoint{5.757520in}{2.376000in}}%
\pgfusepath{stroke}%
\end{pgfscope}%
\begin{pgfscope}%
\pgfsetrectcap%
\pgfsetmiterjoin%
\pgfsetlinewidth{0.803000pt}%
\definecolor{currentstroke}{rgb}{0.000000,0.000000,0.000000}%
\pgfsetstrokecolor{currentstroke}%
\pgfsetdash{}{0pt}%
\pgfpathmoveto{\pgfqpoint{0.800000in}{0.528000in}}%
\pgfpathlineto{\pgfqpoint{0.800000in}{4.224000in}}%
\pgfusepath{stroke}%
\end{pgfscope}%
\begin{pgfscope}%
\pgfsetrectcap%
\pgfsetmiterjoin%
\pgfsetlinewidth{0.803000pt}%
\definecolor{currentstroke}{rgb}{0.000000,0.000000,0.000000}%
\pgfsetstrokecolor{currentstroke}%
\pgfsetdash{}{0pt}%
\pgfpathmoveto{\pgfqpoint{5.760000in}{0.528000in}}%
\pgfpathlineto{\pgfqpoint{5.760000in}{4.224000in}}%
\pgfusepath{stroke}%
\end{pgfscope}%
\begin{pgfscope}%
\pgfsetrectcap%
\pgfsetmiterjoin%
\pgfsetlinewidth{0.803000pt}%
\definecolor{currentstroke}{rgb}{0.000000,0.000000,0.000000}%
\pgfsetstrokecolor{currentstroke}%
\pgfsetdash{}{0pt}%
\pgfpathmoveto{\pgfqpoint{0.800000in}{0.528000in}}%
\pgfpathlineto{\pgfqpoint{5.760000in}{0.528000in}}%
\pgfusepath{stroke}%
\end{pgfscope}%
\begin{pgfscope}%
\pgfsetrectcap%
\pgfsetmiterjoin%
\pgfsetlinewidth{0.803000pt}%
\definecolor{currentstroke}{rgb}{0.000000,0.000000,0.000000}%
\pgfsetstrokecolor{currentstroke}%
\pgfsetdash{}{0pt}%
\pgfpathmoveto{\pgfqpoint{0.800000in}{4.224000in}}%
\pgfpathlineto{\pgfqpoint{5.760000in}{4.224000in}}%
\pgfusepath{stroke}%
\end{pgfscope}%
\begin{pgfscope}%
\pgftext[x=0.800000in,y=4.408800in,left,base]{\sffamily\fontsize{10.000000}{12.000000}\selectfont Iterations: 1990, Time: 0.140 ps, velocity: 300 Mm/s, imp: 377 ohm}%
\end{pgfscope}%
\end{pgfpicture}%
\makeatother%
\endgroup%
}}
  \caption{Simulation result from task 1.}
  \label{fig:task1}
\end{figure}

\section{Task 2}\label{sec:2}
The direction in which the pulse moved can be read from the poynting vector. In Figure\ref{fig:task2} a Poynting vector with positive amplitude move from left to right. As seen when comparing Figure\ref{fig:task2_1} (before reaching the right end) to Figure\ref{fig:task2} (after reaching the right end), it appears that the wave has bounced back and is now propagating from right to left. This is because the final index in the $E$-field is never changed so it remains 0. This is equivalend to the right wall being an infinite impedance so the wave is reflected and the amplitude is negated.
\begin{figure}
  \centering
    \begin{subfigure}[b]{0.5\textwidth}
        \noindent\makebox[\textwidth]{\scalebox{0.5}{%% Creator: Matplotlib, PGF backend
%%
%% To include the figure in your LaTeX document, write
%%   \input{<filename>.pgf}
%%
%% Make sure the required packages are loaded in your preamble
%%   \usepackage{pgf}
%%
%% Figures using additional raster images can only be included by \input if
%% they are in the same directory as the main LaTeX file. For loading figures
%% from other directories you can use the `import` package
%%   \usepackage{import}
%% and then include the figures with
%%   \import{<path to file>}{<filename>.pgf}
%%
%% Matplotlib used the following preamble
%%   \usepackage{fontspec}
%%   \setmainfont{DejaVu Serif}
%%   \setsansfont{DejaVu Sans}
%%   \setmonofont{DejaVu Sans Mono}
%%
\begingroup%
\makeatletter%
\begin{pgfpicture}%
\pgfpathrectangle{\pgfpointorigin}{\pgfqpoint{13.660000in}{6.570000in}}%
\pgfusepath{use as bounding box, clip}%
\begin{pgfscope}%
\pgfsetbuttcap%
\pgfsetmiterjoin%
\definecolor{currentfill}{rgb}{1.000000,1.000000,1.000000}%
\pgfsetfillcolor{currentfill}%
\pgfsetlinewidth{0.000000pt}%
\definecolor{currentstroke}{rgb}{1.000000,1.000000,1.000000}%
\pgfsetstrokecolor{currentstroke}%
\pgfsetdash{}{0pt}%
\pgfpathmoveto{\pgfqpoint{0.000000in}{0.000000in}}%
\pgfpathlineto{\pgfqpoint{13.660000in}{0.000000in}}%
\pgfpathlineto{\pgfqpoint{13.660000in}{6.570000in}}%
\pgfpathlineto{\pgfqpoint{0.000000in}{6.570000in}}%
\pgfpathclose%
\pgfusepath{fill}%
\end{pgfscope}%
\begin{pgfscope}%
\pgfsetbuttcap%
\pgfsetmiterjoin%
\definecolor{currentfill}{rgb}{1.000000,1.000000,1.000000}%
\pgfsetfillcolor{currentfill}%
\pgfsetlinewidth{0.000000pt}%
\definecolor{currentstroke}{rgb}{0.000000,0.000000,0.000000}%
\pgfsetstrokecolor{currentstroke}%
\pgfsetstrokeopacity{0.000000}%
\pgfsetdash{}{0pt}%
\pgfpathmoveto{\pgfqpoint{1.707500in}{0.722700in}}%
\pgfpathlineto{\pgfqpoint{6.519545in}{0.722700in}}%
\pgfpathlineto{\pgfqpoint{6.519545in}{5.781600in}}%
\pgfpathlineto{\pgfqpoint{1.707500in}{5.781600in}}%
\pgfpathclose%
\pgfusepath{fill}%
\end{pgfscope}%
\begin{pgfscope}%
\pgfsetbuttcap%
\pgfsetroundjoin%
\definecolor{currentfill}{rgb}{0.000000,0.000000,0.000000}%
\pgfsetfillcolor{currentfill}%
\pgfsetlinewidth{0.803000pt}%
\definecolor{currentstroke}{rgb}{0.000000,0.000000,0.000000}%
\pgfsetstrokecolor{currentstroke}%
\pgfsetdash{}{0pt}%
\pgfsys@defobject{currentmarker}{\pgfqpoint{0.000000in}{-0.048611in}}{\pgfqpoint{0.000000in}{0.000000in}}{%
\pgfpathmoveto{\pgfqpoint{0.000000in}{0.000000in}}%
\pgfpathlineto{\pgfqpoint{0.000000in}{-0.048611in}}%
\pgfusepath{stroke,fill}%
}%
\begin{pgfscope}%
\pgfsys@transformshift{1.707500in}{0.722700in}%
\pgfsys@useobject{currentmarker}{}%
\end{pgfscope}%
\end{pgfscope}%
\begin{pgfscope}%
\pgftext[x=1.707500in,y=0.625478in,,top]{\sffamily\fontsize{10.000000}{12.000000}\selectfont 0}%
\end{pgfscope}%
\begin{pgfscope}%
\pgfsetbuttcap%
\pgfsetroundjoin%
\definecolor{currentfill}{rgb}{0.000000,0.000000,0.000000}%
\pgfsetfillcolor{currentfill}%
\pgfsetlinewidth{0.803000pt}%
\definecolor{currentstroke}{rgb}{0.000000,0.000000,0.000000}%
\pgfsetstrokecolor{currentstroke}%
\pgfsetdash{}{0pt}%
\pgfsys@defobject{currentmarker}{\pgfqpoint{0.000000in}{-0.048611in}}{\pgfqpoint{0.000000in}{0.000000in}}{%
\pgfpathmoveto{\pgfqpoint{0.000000in}{0.000000in}}%
\pgfpathlineto{\pgfqpoint{0.000000in}{-0.048611in}}%
\pgfusepath{stroke,fill}%
}%
\begin{pgfscope}%
\pgfsys@transformshift{2.277648in}{0.722700in}%
\pgfsys@useobject{currentmarker}{}%
\end{pgfscope}%
\end{pgfscope}%
\begin{pgfscope}%
\pgftext[x=2.277648in,y=0.625478in,,top]{\sffamily\fontsize{10.000000}{12.000000}\selectfont 5}%
\end{pgfscope}%
\begin{pgfscope}%
\pgfsetbuttcap%
\pgfsetroundjoin%
\definecolor{currentfill}{rgb}{0.000000,0.000000,0.000000}%
\pgfsetfillcolor{currentfill}%
\pgfsetlinewidth{0.803000pt}%
\definecolor{currentstroke}{rgb}{0.000000,0.000000,0.000000}%
\pgfsetstrokecolor{currentstroke}%
\pgfsetdash{}{0pt}%
\pgfsys@defobject{currentmarker}{\pgfqpoint{0.000000in}{-0.048611in}}{\pgfqpoint{0.000000in}{0.000000in}}{%
\pgfpathmoveto{\pgfqpoint{0.000000in}{0.000000in}}%
\pgfpathlineto{\pgfqpoint{0.000000in}{-0.048611in}}%
\pgfusepath{stroke,fill}%
}%
\begin{pgfscope}%
\pgfsys@transformshift{2.847795in}{0.722700in}%
\pgfsys@useobject{currentmarker}{}%
\end{pgfscope}%
\end{pgfscope}%
\begin{pgfscope}%
\pgftext[x=2.847795in,y=0.625478in,,top]{\sffamily\fontsize{10.000000}{12.000000}\selectfont 10}%
\end{pgfscope}%
\begin{pgfscope}%
\pgfsetbuttcap%
\pgfsetroundjoin%
\definecolor{currentfill}{rgb}{0.000000,0.000000,0.000000}%
\pgfsetfillcolor{currentfill}%
\pgfsetlinewidth{0.803000pt}%
\definecolor{currentstroke}{rgb}{0.000000,0.000000,0.000000}%
\pgfsetstrokecolor{currentstroke}%
\pgfsetdash{}{0pt}%
\pgfsys@defobject{currentmarker}{\pgfqpoint{0.000000in}{-0.048611in}}{\pgfqpoint{0.000000in}{0.000000in}}{%
\pgfpathmoveto{\pgfqpoint{0.000000in}{0.000000in}}%
\pgfpathlineto{\pgfqpoint{0.000000in}{-0.048611in}}%
\pgfusepath{stroke,fill}%
}%
\begin{pgfscope}%
\pgfsys@transformshift{3.417943in}{0.722700in}%
\pgfsys@useobject{currentmarker}{}%
\end{pgfscope}%
\end{pgfscope}%
\begin{pgfscope}%
\pgftext[x=3.417943in,y=0.625478in,,top]{\sffamily\fontsize{10.000000}{12.000000}\selectfont 15}%
\end{pgfscope}%
\begin{pgfscope}%
\pgfsetbuttcap%
\pgfsetroundjoin%
\definecolor{currentfill}{rgb}{0.000000,0.000000,0.000000}%
\pgfsetfillcolor{currentfill}%
\pgfsetlinewidth{0.803000pt}%
\definecolor{currentstroke}{rgb}{0.000000,0.000000,0.000000}%
\pgfsetstrokecolor{currentstroke}%
\pgfsetdash{}{0pt}%
\pgfsys@defobject{currentmarker}{\pgfqpoint{0.000000in}{-0.048611in}}{\pgfqpoint{0.000000in}{0.000000in}}{%
\pgfpathmoveto{\pgfqpoint{0.000000in}{0.000000in}}%
\pgfpathlineto{\pgfqpoint{0.000000in}{-0.048611in}}%
\pgfusepath{stroke,fill}%
}%
\begin{pgfscope}%
\pgfsys@transformshift{3.988090in}{0.722700in}%
\pgfsys@useobject{currentmarker}{}%
\end{pgfscope}%
\end{pgfscope}%
\begin{pgfscope}%
\pgftext[x=3.988090in,y=0.625478in,,top]{\sffamily\fontsize{10.000000}{12.000000}\selectfont 20}%
\end{pgfscope}%
\begin{pgfscope}%
\pgfsetbuttcap%
\pgfsetroundjoin%
\definecolor{currentfill}{rgb}{0.000000,0.000000,0.000000}%
\pgfsetfillcolor{currentfill}%
\pgfsetlinewidth{0.803000pt}%
\definecolor{currentstroke}{rgb}{0.000000,0.000000,0.000000}%
\pgfsetstrokecolor{currentstroke}%
\pgfsetdash{}{0pt}%
\pgfsys@defobject{currentmarker}{\pgfqpoint{0.000000in}{-0.048611in}}{\pgfqpoint{0.000000in}{0.000000in}}{%
\pgfpathmoveto{\pgfqpoint{0.000000in}{0.000000in}}%
\pgfpathlineto{\pgfqpoint{0.000000in}{-0.048611in}}%
\pgfusepath{stroke,fill}%
}%
\begin{pgfscope}%
\pgfsys@transformshift{4.558238in}{0.722700in}%
\pgfsys@useobject{currentmarker}{}%
\end{pgfscope}%
\end{pgfscope}%
\begin{pgfscope}%
\pgftext[x=4.558238in,y=0.625478in,,top]{\sffamily\fontsize{10.000000}{12.000000}\selectfont 25}%
\end{pgfscope}%
\begin{pgfscope}%
\pgfsetbuttcap%
\pgfsetroundjoin%
\definecolor{currentfill}{rgb}{0.000000,0.000000,0.000000}%
\pgfsetfillcolor{currentfill}%
\pgfsetlinewidth{0.803000pt}%
\definecolor{currentstroke}{rgb}{0.000000,0.000000,0.000000}%
\pgfsetstrokecolor{currentstroke}%
\pgfsetdash{}{0pt}%
\pgfsys@defobject{currentmarker}{\pgfqpoint{0.000000in}{-0.048611in}}{\pgfqpoint{0.000000in}{0.000000in}}{%
\pgfpathmoveto{\pgfqpoint{0.000000in}{0.000000in}}%
\pgfpathlineto{\pgfqpoint{0.000000in}{-0.048611in}}%
\pgfusepath{stroke,fill}%
}%
\begin{pgfscope}%
\pgfsys@transformshift{5.128385in}{0.722700in}%
\pgfsys@useobject{currentmarker}{}%
\end{pgfscope}%
\end{pgfscope}%
\begin{pgfscope}%
\pgftext[x=5.128385in,y=0.625478in,,top]{\sffamily\fontsize{10.000000}{12.000000}\selectfont 30}%
\end{pgfscope}%
\begin{pgfscope}%
\pgfsetbuttcap%
\pgfsetroundjoin%
\definecolor{currentfill}{rgb}{0.000000,0.000000,0.000000}%
\pgfsetfillcolor{currentfill}%
\pgfsetlinewidth{0.803000pt}%
\definecolor{currentstroke}{rgb}{0.000000,0.000000,0.000000}%
\pgfsetstrokecolor{currentstroke}%
\pgfsetdash{}{0pt}%
\pgfsys@defobject{currentmarker}{\pgfqpoint{0.000000in}{-0.048611in}}{\pgfqpoint{0.000000in}{0.000000in}}{%
\pgfpathmoveto{\pgfqpoint{0.000000in}{0.000000in}}%
\pgfpathlineto{\pgfqpoint{0.000000in}{-0.048611in}}%
\pgfusepath{stroke,fill}%
}%
\begin{pgfscope}%
\pgfsys@transformshift{5.698533in}{0.722700in}%
\pgfsys@useobject{currentmarker}{}%
\end{pgfscope}%
\end{pgfscope}%
\begin{pgfscope}%
\pgftext[x=5.698533in,y=0.625478in,,top]{\sffamily\fontsize{10.000000}{12.000000}\selectfont 35}%
\end{pgfscope}%
\begin{pgfscope}%
\pgfsetbuttcap%
\pgfsetroundjoin%
\definecolor{currentfill}{rgb}{0.000000,0.000000,0.000000}%
\pgfsetfillcolor{currentfill}%
\pgfsetlinewidth{0.803000pt}%
\definecolor{currentstroke}{rgb}{0.000000,0.000000,0.000000}%
\pgfsetstrokecolor{currentstroke}%
\pgfsetdash{}{0pt}%
\pgfsys@defobject{currentmarker}{\pgfqpoint{0.000000in}{-0.048611in}}{\pgfqpoint{0.000000in}{0.000000in}}{%
\pgfpathmoveto{\pgfqpoint{0.000000in}{0.000000in}}%
\pgfpathlineto{\pgfqpoint{0.000000in}{-0.048611in}}%
\pgfusepath{stroke,fill}%
}%
\begin{pgfscope}%
\pgfsys@transformshift{6.268681in}{0.722700in}%
\pgfsys@useobject{currentmarker}{}%
\end{pgfscope}%
\end{pgfscope}%
\begin{pgfscope}%
\pgftext[x=6.268681in,y=0.625478in,,top]{\sffamily\fontsize{10.000000}{12.000000}\selectfont 40}%
\end{pgfscope}%
\begin{pgfscope}%
\pgftext[x=4.113523in,y=0.435509in,,top]{\sffamily\fontsize{16.000000}{19.200000}\selectfont \(\displaystyle z-position [\mu m]\)}%
\end{pgfscope}%
\begin{pgfscope}%
\pgfsetbuttcap%
\pgfsetroundjoin%
\definecolor{currentfill}{rgb}{0.000000,0.000000,0.000000}%
\pgfsetfillcolor{currentfill}%
\pgfsetlinewidth{0.803000pt}%
\definecolor{currentstroke}{rgb}{0.000000,0.000000,0.000000}%
\pgfsetstrokecolor{currentstroke}%
\pgfsetdash{}{0pt}%
\pgfsys@defobject{currentmarker}{\pgfqpoint{-0.048611in}{0.000000in}}{\pgfqpoint{0.000000in}{0.000000in}}{%
\pgfpathmoveto{\pgfqpoint{0.000000in}{0.000000in}}%
\pgfpathlineto{\pgfqpoint{-0.048611in}{0.000000in}}%
\pgfusepath{stroke,fill}%
}%
\begin{pgfscope}%
\pgfsys@transformshift{1.707500in}{0.722700in}%
\pgfsys@useobject{currentmarker}{}%
\end{pgfscope}%
\end{pgfscope}%
\begin{pgfscope}%
\pgftext[x=1.273025in,y=0.669938in,left,base]{\sffamily\fontsize{10.000000}{12.000000}\selectfont -2.0}%
\end{pgfscope}%
\begin{pgfscope}%
\pgfsetbuttcap%
\pgfsetroundjoin%
\definecolor{currentfill}{rgb}{0.000000,0.000000,0.000000}%
\pgfsetfillcolor{currentfill}%
\pgfsetlinewidth{0.803000pt}%
\definecolor{currentstroke}{rgb}{0.000000,0.000000,0.000000}%
\pgfsetstrokecolor{currentstroke}%
\pgfsetdash{}{0pt}%
\pgfsys@defobject{currentmarker}{\pgfqpoint{-0.048611in}{0.000000in}}{\pgfqpoint{0.000000in}{0.000000in}}{%
\pgfpathmoveto{\pgfqpoint{0.000000in}{0.000000in}}%
\pgfpathlineto{\pgfqpoint{-0.048611in}{0.000000in}}%
\pgfusepath{stroke,fill}%
}%
\begin{pgfscope}%
\pgfsys@transformshift{1.707500in}{1.355062in}%
\pgfsys@useobject{currentmarker}{}%
\end{pgfscope}%
\end{pgfscope}%
\begin{pgfscope}%
\pgftext[x=1.273025in,y=1.302301in,left,base]{\sffamily\fontsize{10.000000}{12.000000}\selectfont -1.5}%
\end{pgfscope}%
\begin{pgfscope}%
\pgfsetbuttcap%
\pgfsetroundjoin%
\definecolor{currentfill}{rgb}{0.000000,0.000000,0.000000}%
\pgfsetfillcolor{currentfill}%
\pgfsetlinewidth{0.803000pt}%
\definecolor{currentstroke}{rgb}{0.000000,0.000000,0.000000}%
\pgfsetstrokecolor{currentstroke}%
\pgfsetdash{}{0pt}%
\pgfsys@defobject{currentmarker}{\pgfqpoint{-0.048611in}{0.000000in}}{\pgfqpoint{0.000000in}{0.000000in}}{%
\pgfpathmoveto{\pgfqpoint{0.000000in}{0.000000in}}%
\pgfpathlineto{\pgfqpoint{-0.048611in}{0.000000in}}%
\pgfusepath{stroke,fill}%
}%
\begin{pgfscope}%
\pgfsys@transformshift{1.707500in}{1.987425in}%
\pgfsys@useobject{currentmarker}{}%
\end{pgfscope}%
\end{pgfscope}%
\begin{pgfscope}%
\pgftext[x=1.273025in,y=1.934663in,left,base]{\sffamily\fontsize{10.000000}{12.000000}\selectfont -1.0}%
\end{pgfscope}%
\begin{pgfscope}%
\pgfsetbuttcap%
\pgfsetroundjoin%
\definecolor{currentfill}{rgb}{0.000000,0.000000,0.000000}%
\pgfsetfillcolor{currentfill}%
\pgfsetlinewidth{0.803000pt}%
\definecolor{currentstroke}{rgb}{0.000000,0.000000,0.000000}%
\pgfsetstrokecolor{currentstroke}%
\pgfsetdash{}{0pt}%
\pgfsys@defobject{currentmarker}{\pgfqpoint{-0.048611in}{0.000000in}}{\pgfqpoint{0.000000in}{0.000000in}}{%
\pgfpathmoveto{\pgfqpoint{0.000000in}{0.000000in}}%
\pgfpathlineto{\pgfqpoint{-0.048611in}{0.000000in}}%
\pgfusepath{stroke,fill}%
}%
\begin{pgfscope}%
\pgfsys@transformshift{1.707500in}{2.619788in}%
\pgfsys@useobject{currentmarker}{}%
\end{pgfscope}%
\end{pgfscope}%
\begin{pgfscope}%
\pgftext[x=1.273025in,y=2.567026in,left,base]{\sffamily\fontsize{10.000000}{12.000000}\selectfont -0.5}%
\end{pgfscope}%
\begin{pgfscope}%
\pgfsetbuttcap%
\pgfsetroundjoin%
\definecolor{currentfill}{rgb}{0.000000,0.000000,0.000000}%
\pgfsetfillcolor{currentfill}%
\pgfsetlinewidth{0.803000pt}%
\definecolor{currentstroke}{rgb}{0.000000,0.000000,0.000000}%
\pgfsetstrokecolor{currentstroke}%
\pgfsetdash{}{0pt}%
\pgfsys@defobject{currentmarker}{\pgfqpoint{-0.048611in}{0.000000in}}{\pgfqpoint{0.000000in}{0.000000in}}{%
\pgfpathmoveto{\pgfqpoint{0.000000in}{0.000000in}}%
\pgfpathlineto{\pgfqpoint{-0.048611in}{0.000000in}}%
\pgfusepath{stroke,fill}%
}%
\begin{pgfscope}%
\pgfsys@transformshift{1.707500in}{3.252150in}%
\pgfsys@useobject{currentmarker}{}%
\end{pgfscope}%
\end{pgfscope}%
\begin{pgfscope}%
\pgftext[x=1.389398in,y=3.199388in,left,base]{\sffamily\fontsize{10.000000}{12.000000}\selectfont 0.0}%
\end{pgfscope}%
\begin{pgfscope}%
\pgfsetbuttcap%
\pgfsetroundjoin%
\definecolor{currentfill}{rgb}{0.000000,0.000000,0.000000}%
\pgfsetfillcolor{currentfill}%
\pgfsetlinewidth{0.803000pt}%
\definecolor{currentstroke}{rgb}{0.000000,0.000000,0.000000}%
\pgfsetstrokecolor{currentstroke}%
\pgfsetdash{}{0pt}%
\pgfsys@defobject{currentmarker}{\pgfqpoint{-0.048611in}{0.000000in}}{\pgfqpoint{0.000000in}{0.000000in}}{%
\pgfpathmoveto{\pgfqpoint{0.000000in}{0.000000in}}%
\pgfpathlineto{\pgfqpoint{-0.048611in}{0.000000in}}%
\pgfusepath{stroke,fill}%
}%
\begin{pgfscope}%
\pgfsys@transformshift{1.707500in}{3.884512in}%
\pgfsys@useobject{currentmarker}{}%
\end{pgfscope}%
\end{pgfscope}%
\begin{pgfscope}%
\pgftext[x=1.389398in,y=3.831751in,left,base]{\sffamily\fontsize{10.000000}{12.000000}\selectfont 0.5}%
\end{pgfscope}%
\begin{pgfscope}%
\pgfsetbuttcap%
\pgfsetroundjoin%
\definecolor{currentfill}{rgb}{0.000000,0.000000,0.000000}%
\pgfsetfillcolor{currentfill}%
\pgfsetlinewidth{0.803000pt}%
\definecolor{currentstroke}{rgb}{0.000000,0.000000,0.000000}%
\pgfsetstrokecolor{currentstroke}%
\pgfsetdash{}{0pt}%
\pgfsys@defobject{currentmarker}{\pgfqpoint{-0.048611in}{0.000000in}}{\pgfqpoint{0.000000in}{0.000000in}}{%
\pgfpathmoveto{\pgfqpoint{0.000000in}{0.000000in}}%
\pgfpathlineto{\pgfqpoint{-0.048611in}{0.000000in}}%
\pgfusepath{stroke,fill}%
}%
\begin{pgfscope}%
\pgfsys@transformshift{1.707500in}{4.516875in}%
\pgfsys@useobject{currentmarker}{}%
\end{pgfscope}%
\end{pgfscope}%
\begin{pgfscope}%
\pgftext[x=1.389398in,y=4.464113in,left,base]{\sffamily\fontsize{10.000000}{12.000000}\selectfont 1.0}%
\end{pgfscope}%
\begin{pgfscope}%
\pgfsetbuttcap%
\pgfsetroundjoin%
\definecolor{currentfill}{rgb}{0.000000,0.000000,0.000000}%
\pgfsetfillcolor{currentfill}%
\pgfsetlinewidth{0.803000pt}%
\definecolor{currentstroke}{rgb}{0.000000,0.000000,0.000000}%
\pgfsetstrokecolor{currentstroke}%
\pgfsetdash{}{0pt}%
\pgfsys@defobject{currentmarker}{\pgfqpoint{-0.048611in}{0.000000in}}{\pgfqpoint{0.000000in}{0.000000in}}{%
\pgfpathmoveto{\pgfqpoint{0.000000in}{0.000000in}}%
\pgfpathlineto{\pgfqpoint{-0.048611in}{0.000000in}}%
\pgfusepath{stroke,fill}%
}%
\begin{pgfscope}%
\pgfsys@transformshift{1.707500in}{5.149237in}%
\pgfsys@useobject{currentmarker}{}%
\end{pgfscope}%
\end{pgfscope}%
\begin{pgfscope}%
\pgftext[x=1.389398in,y=5.096476in,left,base]{\sffamily\fontsize{10.000000}{12.000000}\selectfont 1.5}%
\end{pgfscope}%
\begin{pgfscope}%
\pgfsetbuttcap%
\pgfsetroundjoin%
\definecolor{currentfill}{rgb}{0.000000,0.000000,0.000000}%
\pgfsetfillcolor{currentfill}%
\pgfsetlinewidth{0.803000pt}%
\definecolor{currentstroke}{rgb}{0.000000,0.000000,0.000000}%
\pgfsetstrokecolor{currentstroke}%
\pgfsetdash{}{0pt}%
\pgfsys@defobject{currentmarker}{\pgfqpoint{-0.048611in}{0.000000in}}{\pgfqpoint{0.000000in}{0.000000in}}{%
\pgfpathmoveto{\pgfqpoint{0.000000in}{0.000000in}}%
\pgfpathlineto{\pgfqpoint{-0.048611in}{0.000000in}}%
\pgfusepath{stroke,fill}%
}%
\begin{pgfscope}%
\pgfsys@transformshift{1.707500in}{5.781600in}%
\pgfsys@useobject{currentmarker}{}%
\end{pgfscope}%
\end{pgfscope}%
\begin{pgfscope}%
\pgftext[x=1.389398in,y=5.728838in,left,base]{\sffamily\fontsize{10.000000}{12.000000}\selectfont 2.0}%
\end{pgfscope}%
\begin{pgfscope}%
\pgftext[x=1.217469in,y=3.252150in,,bottom,rotate=90.000000]{\sffamily\fontsize{16.000000}{19.200000}\selectfont \(\displaystyle E-field\)}%
\end{pgfscope}%
\begin{pgfscope}%
\pgfpathrectangle{\pgfqpoint{1.707500in}{0.722700in}}{\pgfqpoint{4.812045in}{5.058900in}} %
\pgfusepath{clip}%
\pgfsetrectcap%
\pgfsetroundjoin%
\pgfsetlinewidth{1.505625pt}%
\definecolor{currentstroke}{rgb}{0.000000,0.000000,0.000000}%
\pgfsetstrokecolor{currentstroke}%
\pgfsetdash{}{0pt}%
\pgfpathmoveto{\pgfqpoint{1.707500in}{3.252150in}}%
\pgfpathlineto{\pgfqpoint{4.779259in}{3.253188in}}%
\pgfpathlineto{\pgfqpoint{4.784698in}{3.256978in}}%
\pgfpathlineto{\pgfqpoint{4.790975in}{3.264373in}}%
\pgfpathlineto{\pgfqpoint{4.800181in}{3.275379in}}%
\pgfpathlineto{\pgfqpoint{4.803110in}{3.275943in}}%
\pgfpathlineto{\pgfqpoint{4.805620in}{3.274018in}}%
\pgfpathlineto{\pgfqpoint{4.808549in}{3.268095in}}%
\pgfpathlineto{\pgfqpoint{4.811897in}{3.255523in}}%
\pgfpathlineto{\pgfqpoint{4.816081in}{3.230361in}}%
\pgfpathlineto{\pgfqpoint{4.822358in}{3.176192in}}%
\pgfpathlineto{\pgfqpoint{4.831564in}{3.098298in}}%
\pgfpathlineto{\pgfqpoint{4.834493in}{3.087676in}}%
\pgfpathlineto{\pgfqpoint{4.836166in}{3.087167in}}%
\pgfpathlineto{\pgfqpoint{4.837840in}{3.091417in}}%
\pgfpathlineto{\pgfqpoint{4.839932in}{3.104118in}}%
\pgfpathlineto{\pgfqpoint{4.842861in}{3.136718in}}%
\pgfpathlineto{\pgfqpoint{4.847046in}{3.213578in}}%
\pgfpathlineto{\pgfqpoint{4.852904in}{3.370300in}}%
\pgfpathlineto{\pgfqpoint{4.864202in}{3.681173in}}%
\pgfpathlineto{\pgfqpoint{4.867549in}{3.724800in}}%
\pgfpathlineto{\pgfqpoint{4.869642in}{3.731705in}}%
\pgfpathlineto{\pgfqpoint{4.870897in}{3.727394in}}%
\pgfpathlineto{\pgfqpoint{4.872989in}{3.705302in}}%
\pgfpathlineto{\pgfqpoint{4.875918in}{3.642321in}}%
\pgfpathlineto{\pgfqpoint{4.880102in}{3.490268in}}%
\pgfpathlineto{\pgfqpoint{4.886379in}{3.158217in}}%
\pgfpathlineto{\pgfqpoint{4.897677in}{2.551742in}}%
\pgfpathlineto{\pgfqpoint{4.901443in}{2.447435in}}%
\pgfpathlineto{\pgfqpoint{4.903953in}{2.424022in}}%
\pgfpathlineto{\pgfqpoint{4.904372in}{2.424043in}}%
\pgfpathlineto{\pgfqpoint{4.905209in}{2.427544in}}%
\pgfpathlineto{\pgfqpoint{4.906883in}{2.448543in}}%
\pgfpathlineto{\pgfqpoint{4.909393in}{2.515003in}}%
\pgfpathlineto{\pgfqpoint{4.913159in}{2.689248in}}%
\pgfpathlineto{\pgfqpoint{4.918599in}{3.067850in}}%
\pgfpathlineto{\pgfqpoint{4.933663in}{4.188024in}}%
\pgfpathlineto{\pgfqpoint{4.937429in}{4.305530in}}%
\pgfpathlineto{\pgfqpoint{4.939521in}{4.323034in}}%
\pgfpathlineto{\pgfqpoint{4.940358in}{4.319974in}}%
\pgfpathlineto{\pgfqpoint{4.942031in}{4.296443in}}%
\pgfpathlineto{\pgfqpoint{4.944542in}{4.218242in}}%
\pgfpathlineto{\pgfqpoint{4.948308in}{4.011154in}}%
\pgfpathlineto{\pgfqpoint{4.953748in}{3.562804in}}%
\pgfpathlineto{\pgfqpoint{4.968812in}{2.238989in}}%
\pgfpathlineto{\pgfqpoint{4.972577in}{2.089653in}}%
\pgfpathlineto{\pgfqpoint{4.975088in}{2.058094in}}%
\pgfpathlineto{\pgfqpoint{4.975507in}{2.058378in}}%
\pgfpathlineto{\pgfqpoint{4.976762in}{2.068806in}}%
\pgfpathlineto{\pgfqpoint{4.978854in}{2.117824in}}%
\pgfpathlineto{\pgfqpoint{4.981783in}{2.250105in}}%
\pgfpathlineto{\pgfqpoint{4.985967in}{2.551684in}}%
\pgfpathlineto{\pgfqpoint{4.993081in}{3.257728in}}%
\pgfpathlineto{\pgfqpoint{5.002705in}{4.174641in}}%
\pgfpathlineto{\pgfqpoint{5.007308in}{4.426660in}}%
\pgfpathlineto{\pgfqpoint{5.010237in}{4.491859in}}%
\pgfpathlineto{\pgfqpoint{5.011074in}{4.495810in}}%
\pgfpathlineto{\pgfqpoint{5.011492in}{4.495306in}}%
\pgfpathlineto{\pgfqpoint{5.012748in}{4.483878in}}%
\pgfpathlineto{\pgfqpoint{5.014840in}{4.432174in}}%
\pgfpathlineto{\pgfqpoint{5.017769in}{4.294403in}}%
\pgfpathlineto{\pgfqpoint{5.021953in}{3.982701in}}%
\pgfpathlineto{\pgfqpoint{5.029485in}{3.212180in}}%
\pgfpathlineto{\pgfqpoint{5.038691in}{2.321402in}}%
\pgfpathlineto{\pgfqpoint{5.043294in}{2.064046in}}%
\pgfpathlineto{\pgfqpoint{5.046223in}{1.996793in}}%
\pgfpathlineto{\pgfqpoint{5.047060in}{1.992431in}}%
\pgfpathlineto{\pgfqpoint{5.047478in}{1.992758in}}%
\pgfpathlineto{\pgfqpoint{5.048733in}{2.003760in}}%
\pgfpathlineto{\pgfqpoint{5.050825in}{2.055092in}}%
\pgfpathlineto{\pgfqpoint{5.053755in}{2.192974in}}%
\pgfpathlineto{\pgfqpoint{5.057939in}{2.505896in}}%
\pgfpathlineto{\pgfqpoint{5.065052in}{3.234905in}}%
\pgfpathlineto{\pgfqpoint{5.074677in}{4.179412in}}%
\pgfpathlineto{\pgfqpoint{5.079279in}{4.440942in}}%
\pgfpathlineto{\pgfqpoint{5.082208in}{4.510761in}}%
\pgfpathlineto{\pgfqpoint{5.083464in}{4.515846in}}%
\pgfpathlineto{\pgfqpoint{5.084301in}{4.510857in}}%
\pgfpathlineto{\pgfqpoint{5.085974in}{4.480902in}}%
\pgfpathlineto{\pgfqpoint{5.088485in}{4.387481in}}%
\pgfpathlineto{\pgfqpoint{5.092251in}{4.148128in}}%
\pgfpathlineto{\pgfqpoint{5.098109in}{3.598815in}}%
\pgfpathlineto{\pgfqpoint{5.111918in}{2.242786in}}%
\pgfpathlineto{\pgfqpoint{5.116102in}{2.037583in}}%
\pgfpathlineto{\pgfqpoint{5.118613in}{1.990081in}}%
\pgfpathlineto{\pgfqpoint{5.119449in}{1.987560in}}%
\pgfpathlineto{\pgfqpoint{5.119868in}{1.988815in}}%
\pgfpathlineto{\pgfqpoint{5.121123in}{2.002621in}}%
\pgfpathlineto{\pgfqpoint{5.123215in}{2.058592in}}%
\pgfpathlineto{\pgfqpoint{5.126563in}{2.229005in}}%
\pgfpathlineto{\pgfqpoint{5.131166in}{2.599963in}}%
\pgfpathlineto{\pgfqpoint{5.140790in}{3.620998in}}%
\pgfpathlineto{\pgfqpoint{5.147903in}{4.254641in}}%
\pgfpathlineto{\pgfqpoint{5.152088in}{4.463555in}}%
\pgfpathlineto{\pgfqpoint{5.154598in}{4.513517in}}%
\pgfpathlineto{\pgfqpoint{5.155435in}{4.516872in}}%
\pgfpathlineto{\pgfqpoint{5.155854in}{4.516033in}}%
\pgfpathlineto{\pgfqpoint{5.157109in}{4.503473in}}%
\pgfpathlineto{\pgfqpoint{5.159201in}{4.449523in}}%
\pgfpathlineto{\pgfqpoint{5.162130in}{4.308107in}}%
\pgfpathlineto{\pgfqpoint{5.166733in}{3.952997in}}%
\pgfpathlineto{\pgfqpoint{5.175102in}{3.074234in}}%
\pgfpathlineto{\pgfqpoint{5.183052in}{2.316139in}}%
\pgfpathlineto{\pgfqpoint{5.187655in}{2.058533in}}%
\pgfpathlineto{\pgfqpoint{5.190584in}{1.991670in}}%
\pgfpathlineto{\pgfqpoint{5.191421in}{1.987477in}}%
\pgfpathlineto{\pgfqpoint{5.191839in}{1.987897in}}%
\pgfpathlineto{\pgfqpoint{5.193095in}{1.999205in}}%
\pgfpathlineto{\pgfqpoint{5.195187in}{2.051117in}}%
\pgfpathlineto{\pgfqpoint{5.198116in}{2.189897in}}%
\pgfpathlineto{\pgfqpoint{5.202300in}{2.504112in}}%
\pgfpathlineto{\pgfqpoint{5.209832in}{3.280939in}}%
\pgfpathlineto{\pgfqpoint{5.219038in}{4.180377in}}%
\pgfpathlineto{\pgfqpoint{5.223641in}{4.441910in}}%
\pgfpathlineto{\pgfqpoint{5.226570in}{4.511635in}}%
\pgfpathlineto{\pgfqpoint{5.227825in}{4.516665in}}%
\pgfpathlineto{\pgfqpoint{5.228662in}{4.511635in}}%
\pgfpathlineto{\pgfqpoint{5.230336in}{4.481594in}}%
\pgfpathlineto{\pgfqpoint{5.232846in}{4.388035in}}%
\pgfpathlineto{\pgfqpoint{5.236612in}{4.148479in}}%
\pgfpathlineto{\pgfqpoint{5.242470in}{3.598910in}}%
\pgfpathlineto{\pgfqpoint{5.256279in}{2.242674in}}%
\pgfpathlineto{\pgfqpoint{5.260463in}{2.037482in}}%
\pgfpathlineto{\pgfqpoint{5.262974in}{1.989993in}}%
\pgfpathlineto{\pgfqpoint{5.263811in}{1.987477in}}%
\pgfpathlineto{\pgfqpoint{5.264229in}{1.988736in}}%
\pgfpathlineto{\pgfqpoint{5.265485in}{2.002549in}}%
\pgfpathlineto{\pgfqpoint{5.267577in}{2.058533in}}%
\pgfpathlineto{\pgfqpoint{5.270924in}{2.228966in}}%
\pgfpathlineto{\pgfqpoint{5.275527in}{2.599946in}}%
\pgfpathlineto{\pgfqpoint{5.285151in}{3.621002in}}%
\pgfpathlineto{\pgfqpoint{5.292265in}{4.254646in}}%
\pgfpathlineto{\pgfqpoint{5.296449in}{4.463558in}}%
\pgfpathlineto{\pgfqpoint{5.298960in}{4.513518in}}%
\pgfpathlineto{\pgfqpoint{5.299797in}{4.516872in}}%
\pgfpathlineto{\pgfqpoint{5.300215in}{4.516033in}}%
\pgfpathlineto{\pgfqpoint{5.301470in}{4.503471in}}%
\pgfpathlineto{\pgfqpoint{5.303562in}{4.449520in}}%
\pgfpathlineto{\pgfqpoint{5.306492in}{4.308103in}}%
\pgfpathlineto{\pgfqpoint{5.311094in}{3.952991in}}%
\pgfpathlineto{\pgfqpoint{5.319463in}{3.074237in}}%
\pgfpathlineto{\pgfqpoint{5.327413in}{2.316171in}}%
\pgfpathlineto{\pgfqpoint{5.332016in}{2.058592in}}%
\pgfpathlineto{\pgfqpoint{5.334945in}{1.991747in}}%
\pgfpathlineto{\pgfqpoint{5.335782in}{1.987560in}}%
\pgfpathlineto{\pgfqpoint{5.336201in}{1.987982in}}%
\pgfpathlineto{\pgfqpoint{5.337456in}{1.999297in}}%
\pgfpathlineto{\pgfqpoint{5.339548in}{2.051220in}}%
\pgfpathlineto{\pgfqpoint{5.342477in}{2.190009in}}%
\pgfpathlineto{\pgfqpoint{5.346662in}{2.504217in}}%
\pgfpathlineto{\pgfqpoint{5.354194in}{3.280933in}}%
\pgfpathlineto{\pgfqpoint{5.363399in}{4.180005in}}%
\pgfpathlineto{\pgfqpoint{5.368002in}{4.441287in}}%
\pgfpathlineto{\pgfqpoint{5.370931in}{4.510857in}}%
\pgfpathlineto{\pgfqpoint{5.372186in}{4.515827in}}%
\pgfpathlineto{\pgfqpoint{5.373023in}{4.510761in}}%
\pgfpathlineto{\pgfqpoint{5.374697in}{4.480659in}}%
\pgfpathlineto{\pgfqpoint{5.377208in}{4.387045in}}%
\pgfpathlineto{\pgfqpoint{5.380974in}{4.147527in}}%
\pgfpathlineto{\pgfqpoint{5.386832in}{3.598413in}}%
\pgfpathlineto{\pgfqpoint{5.400640in}{2.245487in}}%
\pgfpathlineto{\pgfqpoint{5.404825in}{2.041579in}}%
\pgfpathlineto{\pgfqpoint{5.407335in}{1.994757in}}%
\pgfpathlineto{\pgfqpoint{5.408172in}{1.992431in}}%
\pgfpathlineto{\pgfqpoint{5.408591in}{1.993776in}}%
\pgfpathlineto{\pgfqpoint{5.409846in}{2.007817in}}%
\pgfpathlineto{\pgfqpoint{5.411938in}{2.064046in}}%
\pgfpathlineto{\pgfqpoint{5.415286in}{2.234421in}}%
\pgfpathlineto{\pgfqpoint{5.419888in}{2.604165in}}%
\pgfpathlineto{\pgfqpoint{5.429931in}{3.660807in}}%
\pgfpathlineto{\pgfqpoint{5.436626in}{4.242042in}}%
\pgfpathlineto{\pgfqpoint{5.440810in}{4.445729in}}%
\pgfpathlineto{\pgfqpoint{5.443321in}{4.493147in}}%
\pgfpathlineto{\pgfqpoint{5.444158in}{4.495810in}}%
\pgfpathlineto{\pgfqpoint{5.444576in}{4.494660in}}%
\pgfpathlineto{\pgfqpoint{5.445832in}{4.481321in}}%
\pgfpathlineto{\pgfqpoint{5.447924in}{4.426660in}}%
\pgfpathlineto{\pgfqpoint{5.451271in}{4.259900in}}%
\pgfpathlineto{\pgfqpoint{5.455874in}{3.897459in}}%
\pgfpathlineto{\pgfqpoint{5.466335in}{2.823004in}}%
\pgfpathlineto{\pgfqpoint{5.473030in}{2.274689in}}%
\pgfpathlineto{\pgfqpoint{5.477215in}{2.093522in}}%
\pgfpathlineto{\pgfqpoint{5.479725in}{2.058378in}}%
\pgfpathlineto{\pgfqpoint{5.480144in}{2.058094in}}%
\pgfpathlineto{\pgfqpoint{5.480981in}{2.062305in}}%
\pgfpathlineto{\pgfqpoint{5.482654in}{2.089653in}}%
\pgfpathlineto{\pgfqpoint{5.485165in}{2.176345in}}%
\pgfpathlineto{\pgfqpoint{5.488931in}{2.398748in}}%
\pgfpathlineto{\pgfqpoint{5.494789in}{2.905061in}}%
\pgfpathlineto{\pgfqpoint{5.507761in}{4.065752in}}%
\pgfpathlineto{\pgfqpoint{5.511945in}{4.263676in}}%
\pgfpathlineto{\pgfqpoint{5.514874in}{4.319974in}}%
\pgfpathlineto{\pgfqpoint{5.515711in}{4.323034in}}%
\pgfpathlineto{\pgfqpoint{5.516129in}{4.322394in}}%
\pgfpathlineto{\pgfqpoint{5.517385in}{4.311871in}}%
\pgfpathlineto{\pgfqpoint{5.519477in}{4.266389in}}%
\pgfpathlineto{\pgfqpoint{5.522824in}{4.126578in}}%
\pgfpathlineto{\pgfqpoint{5.527846in}{3.794022in}}%
\pgfpathlineto{\pgfqpoint{5.545002in}{2.546271in}}%
\pgfpathlineto{\pgfqpoint{5.548768in}{2.441537in}}%
\pgfpathlineto{\pgfqpoint{5.550860in}{2.424043in}}%
\pgfpathlineto{\pgfqpoint{5.551278in}{2.424022in}}%
\pgfpathlineto{\pgfqpoint{5.552115in}{2.427396in}}%
\pgfpathlineto{\pgfqpoint{5.553789in}{2.447435in}}%
\pgfpathlineto{\pgfqpoint{5.556299in}{2.508562in}}%
\pgfpathlineto{\pgfqpoint{5.560065in}{2.659406in}}%
\pgfpathlineto{\pgfqpoint{5.567179in}{3.058522in}}%
\pgfpathlineto{\pgfqpoint{5.575966in}{3.526078in}}%
\pgfpathlineto{\pgfqpoint{5.580569in}{3.673888in}}%
\pgfpathlineto{\pgfqpoint{5.583916in}{3.724485in}}%
\pgfpathlineto{\pgfqpoint{5.586009in}{3.731702in}}%
\pgfpathlineto{\pgfqpoint{5.587264in}{3.727531in}}%
\pgfpathlineto{\pgfqpoint{5.589356in}{3.707585in}}%
\pgfpathlineto{\pgfqpoint{5.592704in}{3.646858in}}%
\pgfpathlineto{\pgfqpoint{5.598143in}{3.498436in}}%
\pgfpathlineto{\pgfqpoint{5.609441in}{3.186921in}}%
\pgfpathlineto{\pgfqpoint{5.614044in}{3.115926in}}%
\pgfpathlineto{\pgfqpoint{5.617392in}{3.091417in}}%
\pgfpathlineto{\pgfqpoint{5.619484in}{3.086868in}}%
\pgfpathlineto{\pgfqpoint{5.620739in}{3.087676in}}%
\pgfpathlineto{\pgfqpoint{5.622413in}{3.092385in}}%
\pgfpathlineto{\pgfqpoint{5.625342in}{3.108820in}}%
\pgfpathlineto{\pgfqpoint{5.630782in}{3.156030in}}%
\pgfpathlineto{\pgfqpoint{5.640406in}{3.238998in}}%
\pgfpathlineto{\pgfqpoint{5.645427in}{3.264149in}}%
\pgfpathlineto{\pgfqpoint{5.649193in}{3.273433in}}%
\pgfpathlineto{\pgfqpoint{5.652122in}{3.275943in}}%
\pgfpathlineto{\pgfqpoint{5.654633in}{3.275609in}}%
\pgfpathlineto{\pgfqpoint{5.658399in}{3.272276in}}%
\pgfpathlineto{\pgfqpoint{5.675136in}{3.253610in}}%
\pgfpathlineto{\pgfqpoint{5.681413in}{3.251587in}}%
\pgfpathlineto{\pgfqpoint{5.690618in}{3.251443in}}%
\pgfpathlineto{\pgfqpoint{5.724093in}{3.252152in}}%
\pgfpathlineto{\pgfqpoint{6.519545in}{3.252150in}}%
\pgfpathlineto{\pgfqpoint{6.519545in}{3.252150in}}%
\pgfusepath{stroke}%
\end{pgfscope}%
\begin{pgfscope}%
\pgfsetrectcap%
\pgfsetmiterjoin%
\pgfsetlinewidth{0.803000pt}%
\definecolor{currentstroke}{rgb}{0.000000,0.000000,0.000000}%
\pgfsetstrokecolor{currentstroke}%
\pgfsetdash{}{0pt}%
\pgfpathmoveto{\pgfqpoint{1.707500in}{0.722700in}}%
\pgfpathlineto{\pgfqpoint{1.707500in}{5.781600in}}%
\pgfusepath{stroke}%
\end{pgfscope}%
\begin{pgfscope}%
\pgfsetrectcap%
\pgfsetmiterjoin%
\pgfsetlinewidth{0.803000pt}%
\definecolor{currentstroke}{rgb}{0.000000,0.000000,0.000000}%
\pgfsetstrokecolor{currentstroke}%
\pgfsetdash{}{0pt}%
\pgfpathmoveto{\pgfqpoint{6.519545in}{0.722700in}}%
\pgfpathlineto{\pgfqpoint{6.519545in}{5.781600in}}%
\pgfusepath{stroke}%
\end{pgfscope}%
\begin{pgfscope}%
\pgfsetrectcap%
\pgfsetmiterjoin%
\pgfsetlinewidth{0.803000pt}%
\definecolor{currentstroke}{rgb}{0.000000,0.000000,0.000000}%
\pgfsetstrokecolor{currentstroke}%
\pgfsetdash{}{0pt}%
\pgfpathmoveto{\pgfqpoint{1.707500in}{0.722700in}}%
\pgfpathlineto{\pgfqpoint{6.519545in}{0.722700in}}%
\pgfusepath{stroke}%
\end{pgfscope}%
\begin{pgfscope}%
\pgfsetrectcap%
\pgfsetmiterjoin%
\pgfsetlinewidth{0.803000pt}%
\definecolor{currentstroke}{rgb}{0.000000,0.000000,0.000000}%
\pgfsetstrokecolor{currentstroke}%
\pgfsetdash{}{0pt}%
\pgfpathmoveto{\pgfqpoint{1.707500in}{5.781600in}}%
\pgfpathlineto{\pgfqpoint{6.519545in}{5.781600in}}%
\pgfusepath{stroke}%
\end{pgfscope}%
\begin{pgfscope}%
\pgftext[x=1.707500in,y=6.034545in,left,base]{\sffamily\fontsize{10.000000}{12.000000}\selectfont Iterations: 10945, Time: 0.134 ps, imp: 377 ohm}%
\end{pgfscope}%
\begin{pgfscope}%
\pgfsetbuttcap%
\pgfsetmiterjoin%
\definecolor{currentfill}{rgb}{1.000000,1.000000,1.000000}%
\pgfsetfillcolor{currentfill}%
\pgfsetlinewidth{0.000000pt}%
\definecolor{currentstroke}{rgb}{0.000000,0.000000,0.000000}%
\pgfsetstrokecolor{currentstroke}%
\pgfsetstrokeopacity{0.000000}%
\pgfsetdash{}{0pt}%
\pgfpathmoveto{\pgfqpoint{7.481955in}{0.722700in}}%
\pgfpathlineto{\pgfqpoint{12.294000in}{0.722700in}}%
\pgfpathlineto{\pgfqpoint{12.294000in}{5.781600in}}%
\pgfpathlineto{\pgfqpoint{7.481955in}{5.781600in}}%
\pgfpathclose%
\pgfusepath{fill}%
\end{pgfscope}%
\begin{pgfscope}%
\pgfsetbuttcap%
\pgfsetroundjoin%
\definecolor{currentfill}{rgb}{0.000000,0.000000,0.000000}%
\pgfsetfillcolor{currentfill}%
\pgfsetlinewidth{0.803000pt}%
\definecolor{currentstroke}{rgb}{0.000000,0.000000,0.000000}%
\pgfsetstrokecolor{currentstroke}%
\pgfsetdash{}{0pt}%
\pgfsys@defobject{currentmarker}{\pgfqpoint{0.000000in}{-0.048611in}}{\pgfqpoint{0.000000in}{0.000000in}}{%
\pgfpathmoveto{\pgfqpoint{0.000000in}{0.000000in}}%
\pgfpathlineto{\pgfqpoint{0.000000in}{-0.048611in}}%
\pgfusepath{stroke,fill}%
}%
\begin{pgfscope}%
\pgfsys@transformshift{7.481955in}{0.722700in}%
\pgfsys@useobject{currentmarker}{}%
\end{pgfscope}%
\end{pgfscope}%
\begin{pgfscope}%
\pgftext[x=7.481955in,y=0.625478in,,top]{\sffamily\fontsize{10.000000}{12.000000}\selectfont 0}%
\end{pgfscope}%
\begin{pgfscope}%
\pgfsetbuttcap%
\pgfsetroundjoin%
\definecolor{currentfill}{rgb}{0.000000,0.000000,0.000000}%
\pgfsetfillcolor{currentfill}%
\pgfsetlinewidth{0.803000pt}%
\definecolor{currentstroke}{rgb}{0.000000,0.000000,0.000000}%
\pgfsetstrokecolor{currentstroke}%
\pgfsetdash{}{0pt}%
\pgfsys@defobject{currentmarker}{\pgfqpoint{0.000000in}{-0.048611in}}{\pgfqpoint{0.000000in}{0.000000in}}{%
\pgfpathmoveto{\pgfqpoint{0.000000in}{0.000000in}}%
\pgfpathlineto{\pgfqpoint{0.000000in}{-0.048611in}}%
\pgfusepath{stroke,fill}%
}%
\begin{pgfscope}%
\pgfsys@transformshift{8.052102in}{0.722700in}%
\pgfsys@useobject{currentmarker}{}%
\end{pgfscope}%
\end{pgfscope}%
\begin{pgfscope}%
\pgftext[x=8.052102in,y=0.625478in,,top]{\sffamily\fontsize{10.000000}{12.000000}\selectfont 5}%
\end{pgfscope}%
\begin{pgfscope}%
\pgfsetbuttcap%
\pgfsetroundjoin%
\definecolor{currentfill}{rgb}{0.000000,0.000000,0.000000}%
\pgfsetfillcolor{currentfill}%
\pgfsetlinewidth{0.803000pt}%
\definecolor{currentstroke}{rgb}{0.000000,0.000000,0.000000}%
\pgfsetstrokecolor{currentstroke}%
\pgfsetdash{}{0pt}%
\pgfsys@defobject{currentmarker}{\pgfqpoint{0.000000in}{-0.048611in}}{\pgfqpoint{0.000000in}{0.000000in}}{%
\pgfpathmoveto{\pgfqpoint{0.000000in}{0.000000in}}%
\pgfpathlineto{\pgfqpoint{0.000000in}{-0.048611in}}%
\pgfusepath{stroke,fill}%
}%
\begin{pgfscope}%
\pgfsys@transformshift{8.622250in}{0.722700in}%
\pgfsys@useobject{currentmarker}{}%
\end{pgfscope}%
\end{pgfscope}%
\begin{pgfscope}%
\pgftext[x=8.622250in,y=0.625478in,,top]{\sffamily\fontsize{10.000000}{12.000000}\selectfont 10}%
\end{pgfscope}%
\begin{pgfscope}%
\pgfsetbuttcap%
\pgfsetroundjoin%
\definecolor{currentfill}{rgb}{0.000000,0.000000,0.000000}%
\pgfsetfillcolor{currentfill}%
\pgfsetlinewidth{0.803000pt}%
\definecolor{currentstroke}{rgb}{0.000000,0.000000,0.000000}%
\pgfsetstrokecolor{currentstroke}%
\pgfsetdash{}{0pt}%
\pgfsys@defobject{currentmarker}{\pgfqpoint{0.000000in}{-0.048611in}}{\pgfqpoint{0.000000in}{0.000000in}}{%
\pgfpathmoveto{\pgfqpoint{0.000000in}{0.000000in}}%
\pgfpathlineto{\pgfqpoint{0.000000in}{-0.048611in}}%
\pgfusepath{stroke,fill}%
}%
\begin{pgfscope}%
\pgfsys@transformshift{9.192397in}{0.722700in}%
\pgfsys@useobject{currentmarker}{}%
\end{pgfscope}%
\end{pgfscope}%
\begin{pgfscope}%
\pgftext[x=9.192397in,y=0.625478in,,top]{\sffamily\fontsize{10.000000}{12.000000}\selectfont 15}%
\end{pgfscope}%
\begin{pgfscope}%
\pgfsetbuttcap%
\pgfsetroundjoin%
\definecolor{currentfill}{rgb}{0.000000,0.000000,0.000000}%
\pgfsetfillcolor{currentfill}%
\pgfsetlinewidth{0.803000pt}%
\definecolor{currentstroke}{rgb}{0.000000,0.000000,0.000000}%
\pgfsetstrokecolor{currentstroke}%
\pgfsetdash{}{0pt}%
\pgfsys@defobject{currentmarker}{\pgfqpoint{0.000000in}{-0.048611in}}{\pgfqpoint{0.000000in}{0.000000in}}{%
\pgfpathmoveto{\pgfqpoint{0.000000in}{0.000000in}}%
\pgfpathlineto{\pgfqpoint{0.000000in}{-0.048611in}}%
\pgfusepath{stroke,fill}%
}%
\begin{pgfscope}%
\pgfsys@transformshift{9.762545in}{0.722700in}%
\pgfsys@useobject{currentmarker}{}%
\end{pgfscope}%
\end{pgfscope}%
\begin{pgfscope}%
\pgftext[x=9.762545in,y=0.625478in,,top]{\sffamily\fontsize{10.000000}{12.000000}\selectfont 20}%
\end{pgfscope}%
\begin{pgfscope}%
\pgfsetbuttcap%
\pgfsetroundjoin%
\definecolor{currentfill}{rgb}{0.000000,0.000000,0.000000}%
\pgfsetfillcolor{currentfill}%
\pgfsetlinewidth{0.803000pt}%
\definecolor{currentstroke}{rgb}{0.000000,0.000000,0.000000}%
\pgfsetstrokecolor{currentstroke}%
\pgfsetdash{}{0pt}%
\pgfsys@defobject{currentmarker}{\pgfqpoint{0.000000in}{-0.048611in}}{\pgfqpoint{0.000000in}{0.000000in}}{%
\pgfpathmoveto{\pgfqpoint{0.000000in}{0.000000in}}%
\pgfpathlineto{\pgfqpoint{0.000000in}{-0.048611in}}%
\pgfusepath{stroke,fill}%
}%
\begin{pgfscope}%
\pgfsys@transformshift{10.332692in}{0.722700in}%
\pgfsys@useobject{currentmarker}{}%
\end{pgfscope}%
\end{pgfscope}%
\begin{pgfscope}%
\pgftext[x=10.332692in,y=0.625478in,,top]{\sffamily\fontsize{10.000000}{12.000000}\selectfont 25}%
\end{pgfscope}%
\begin{pgfscope}%
\pgfsetbuttcap%
\pgfsetroundjoin%
\definecolor{currentfill}{rgb}{0.000000,0.000000,0.000000}%
\pgfsetfillcolor{currentfill}%
\pgfsetlinewidth{0.803000pt}%
\definecolor{currentstroke}{rgb}{0.000000,0.000000,0.000000}%
\pgfsetstrokecolor{currentstroke}%
\pgfsetdash{}{0pt}%
\pgfsys@defobject{currentmarker}{\pgfqpoint{0.000000in}{-0.048611in}}{\pgfqpoint{0.000000in}{0.000000in}}{%
\pgfpathmoveto{\pgfqpoint{0.000000in}{0.000000in}}%
\pgfpathlineto{\pgfqpoint{0.000000in}{-0.048611in}}%
\pgfusepath{stroke,fill}%
}%
\begin{pgfscope}%
\pgfsys@transformshift{10.902840in}{0.722700in}%
\pgfsys@useobject{currentmarker}{}%
\end{pgfscope}%
\end{pgfscope}%
\begin{pgfscope}%
\pgftext[x=10.902840in,y=0.625478in,,top]{\sffamily\fontsize{10.000000}{12.000000}\selectfont 30}%
\end{pgfscope}%
\begin{pgfscope}%
\pgfsetbuttcap%
\pgfsetroundjoin%
\definecolor{currentfill}{rgb}{0.000000,0.000000,0.000000}%
\pgfsetfillcolor{currentfill}%
\pgfsetlinewidth{0.803000pt}%
\definecolor{currentstroke}{rgb}{0.000000,0.000000,0.000000}%
\pgfsetstrokecolor{currentstroke}%
\pgfsetdash{}{0pt}%
\pgfsys@defobject{currentmarker}{\pgfqpoint{0.000000in}{-0.048611in}}{\pgfqpoint{0.000000in}{0.000000in}}{%
\pgfpathmoveto{\pgfqpoint{0.000000in}{0.000000in}}%
\pgfpathlineto{\pgfqpoint{0.000000in}{-0.048611in}}%
\pgfusepath{stroke,fill}%
}%
\begin{pgfscope}%
\pgfsys@transformshift{11.472988in}{0.722700in}%
\pgfsys@useobject{currentmarker}{}%
\end{pgfscope}%
\end{pgfscope}%
\begin{pgfscope}%
\pgftext[x=11.472988in,y=0.625478in,,top]{\sffamily\fontsize{10.000000}{12.000000}\selectfont 35}%
\end{pgfscope}%
\begin{pgfscope}%
\pgfsetbuttcap%
\pgfsetroundjoin%
\definecolor{currentfill}{rgb}{0.000000,0.000000,0.000000}%
\pgfsetfillcolor{currentfill}%
\pgfsetlinewidth{0.803000pt}%
\definecolor{currentstroke}{rgb}{0.000000,0.000000,0.000000}%
\pgfsetstrokecolor{currentstroke}%
\pgfsetdash{}{0pt}%
\pgfsys@defobject{currentmarker}{\pgfqpoint{0.000000in}{-0.048611in}}{\pgfqpoint{0.000000in}{0.000000in}}{%
\pgfpathmoveto{\pgfqpoint{0.000000in}{0.000000in}}%
\pgfpathlineto{\pgfqpoint{0.000000in}{-0.048611in}}%
\pgfusepath{stroke,fill}%
}%
\begin{pgfscope}%
\pgfsys@transformshift{12.043135in}{0.722700in}%
\pgfsys@useobject{currentmarker}{}%
\end{pgfscope}%
\end{pgfscope}%
\begin{pgfscope}%
\pgftext[x=12.043135in,y=0.625478in,,top]{\sffamily\fontsize{10.000000}{12.000000}\selectfont 40}%
\end{pgfscope}%
\begin{pgfscope}%
\pgftext[x=9.887977in,y=0.435509in,,top]{\sffamily\fontsize{16.000000}{19.200000}\selectfont \(\displaystyle z-position [\mu m]\)}%
\end{pgfscope}%
\begin{pgfscope}%
\pgfsetbuttcap%
\pgfsetroundjoin%
\definecolor{currentfill}{rgb}{0.000000,0.000000,0.000000}%
\pgfsetfillcolor{currentfill}%
\pgfsetlinewidth{0.803000pt}%
\definecolor{currentstroke}{rgb}{0.000000,0.000000,0.000000}%
\pgfsetstrokecolor{currentstroke}%
\pgfsetdash{}{0pt}%
\pgfsys@defobject{currentmarker}{\pgfqpoint{-0.048611in}{0.000000in}}{\pgfqpoint{0.000000in}{0.000000in}}{%
\pgfpathmoveto{\pgfqpoint{0.000000in}{0.000000in}}%
\pgfpathlineto{\pgfqpoint{-0.048611in}{0.000000in}}%
\pgfusepath{stroke,fill}%
}%
\begin{pgfscope}%
\pgfsys@transformshift{7.481955in}{0.722700in}%
\pgfsys@useobject{currentmarker}{}%
\end{pgfscope}%
\end{pgfscope}%
\begin{pgfscope}%
\pgftext[x=6.870748in,y=0.669938in,left,base]{\sffamily\fontsize{10.000000}{12.000000}\selectfont -0.003}%
\end{pgfscope}%
\begin{pgfscope}%
\pgfsetbuttcap%
\pgfsetroundjoin%
\definecolor{currentfill}{rgb}{0.000000,0.000000,0.000000}%
\pgfsetfillcolor{currentfill}%
\pgfsetlinewidth{0.803000pt}%
\definecolor{currentstroke}{rgb}{0.000000,0.000000,0.000000}%
\pgfsetstrokecolor{currentstroke}%
\pgfsetdash{}{0pt}%
\pgfsys@defobject{currentmarker}{\pgfqpoint{-0.048611in}{0.000000in}}{\pgfqpoint{0.000000in}{0.000000in}}{%
\pgfpathmoveto{\pgfqpoint{0.000000in}{0.000000in}}%
\pgfpathlineto{\pgfqpoint{-0.048611in}{0.000000in}}%
\pgfusepath{stroke,fill}%
}%
\begin{pgfscope}%
\pgfsys@transformshift{7.481955in}{1.565850in}%
\pgfsys@useobject{currentmarker}{}%
\end{pgfscope}%
\end{pgfscope}%
\begin{pgfscope}%
\pgftext[x=6.870748in,y=1.513088in,left,base]{\sffamily\fontsize{10.000000}{12.000000}\selectfont -0.002}%
\end{pgfscope}%
\begin{pgfscope}%
\pgfsetbuttcap%
\pgfsetroundjoin%
\definecolor{currentfill}{rgb}{0.000000,0.000000,0.000000}%
\pgfsetfillcolor{currentfill}%
\pgfsetlinewidth{0.803000pt}%
\definecolor{currentstroke}{rgb}{0.000000,0.000000,0.000000}%
\pgfsetstrokecolor{currentstroke}%
\pgfsetdash{}{0pt}%
\pgfsys@defobject{currentmarker}{\pgfqpoint{-0.048611in}{0.000000in}}{\pgfqpoint{0.000000in}{0.000000in}}{%
\pgfpathmoveto{\pgfqpoint{0.000000in}{0.000000in}}%
\pgfpathlineto{\pgfqpoint{-0.048611in}{0.000000in}}%
\pgfusepath{stroke,fill}%
}%
\begin{pgfscope}%
\pgfsys@transformshift{7.481955in}{2.409000in}%
\pgfsys@useobject{currentmarker}{}%
\end{pgfscope}%
\end{pgfscope}%
\begin{pgfscope}%
\pgftext[x=6.870748in,y=2.356238in,left,base]{\sffamily\fontsize{10.000000}{12.000000}\selectfont -0.001}%
\end{pgfscope}%
\begin{pgfscope}%
\pgfsetbuttcap%
\pgfsetroundjoin%
\definecolor{currentfill}{rgb}{0.000000,0.000000,0.000000}%
\pgfsetfillcolor{currentfill}%
\pgfsetlinewidth{0.803000pt}%
\definecolor{currentstroke}{rgb}{0.000000,0.000000,0.000000}%
\pgfsetstrokecolor{currentstroke}%
\pgfsetdash{}{0pt}%
\pgfsys@defobject{currentmarker}{\pgfqpoint{-0.048611in}{0.000000in}}{\pgfqpoint{0.000000in}{0.000000in}}{%
\pgfpathmoveto{\pgfqpoint{0.000000in}{0.000000in}}%
\pgfpathlineto{\pgfqpoint{-0.048611in}{0.000000in}}%
\pgfusepath{stroke,fill}%
}%
\begin{pgfscope}%
\pgfsys@transformshift{7.481955in}{3.252150in}%
\pgfsys@useobject{currentmarker}{}%
\end{pgfscope}%
\end{pgfscope}%
\begin{pgfscope}%
\pgftext[x=6.987122in,y=3.199388in,left,base]{\sffamily\fontsize{10.000000}{12.000000}\selectfont 0.000}%
\end{pgfscope}%
\begin{pgfscope}%
\pgfsetbuttcap%
\pgfsetroundjoin%
\definecolor{currentfill}{rgb}{0.000000,0.000000,0.000000}%
\pgfsetfillcolor{currentfill}%
\pgfsetlinewidth{0.803000pt}%
\definecolor{currentstroke}{rgb}{0.000000,0.000000,0.000000}%
\pgfsetstrokecolor{currentstroke}%
\pgfsetdash{}{0pt}%
\pgfsys@defobject{currentmarker}{\pgfqpoint{-0.048611in}{0.000000in}}{\pgfqpoint{0.000000in}{0.000000in}}{%
\pgfpathmoveto{\pgfqpoint{0.000000in}{0.000000in}}%
\pgfpathlineto{\pgfqpoint{-0.048611in}{0.000000in}}%
\pgfusepath{stroke,fill}%
}%
\begin{pgfscope}%
\pgfsys@transformshift{7.481955in}{4.095300in}%
\pgfsys@useobject{currentmarker}{}%
\end{pgfscope}%
\end{pgfscope}%
\begin{pgfscope}%
\pgftext[x=6.987122in,y=4.042538in,left,base]{\sffamily\fontsize{10.000000}{12.000000}\selectfont 0.001}%
\end{pgfscope}%
\begin{pgfscope}%
\pgfsetbuttcap%
\pgfsetroundjoin%
\definecolor{currentfill}{rgb}{0.000000,0.000000,0.000000}%
\pgfsetfillcolor{currentfill}%
\pgfsetlinewidth{0.803000pt}%
\definecolor{currentstroke}{rgb}{0.000000,0.000000,0.000000}%
\pgfsetstrokecolor{currentstroke}%
\pgfsetdash{}{0pt}%
\pgfsys@defobject{currentmarker}{\pgfqpoint{-0.048611in}{0.000000in}}{\pgfqpoint{0.000000in}{0.000000in}}{%
\pgfpathmoveto{\pgfqpoint{0.000000in}{0.000000in}}%
\pgfpathlineto{\pgfqpoint{-0.048611in}{0.000000in}}%
\pgfusepath{stroke,fill}%
}%
\begin{pgfscope}%
\pgfsys@transformshift{7.481955in}{4.938450in}%
\pgfsys@useobject{currentmarker}{}%
\end{pgfscope}%
\end{pgfscope}%
\begin{pgfscope}%
\pgftext[x=6.987122in,y=4.885688in,left,base]{\sffamily\fontsize{10.000000}{12.000000}\selectfont 0.002}%
\end{pgfscope}%
\begin{pgfscope}%
\pgfsetbuttcap%
\pgfsetroundjoin%
\definecolor{currentfill}{rgb}{0.000000,0.000000,0.000000}%
\pgfsetfillcolor{currentfill}%
\pgfsetlinewidth{0.803000pt}%
\definecolor{currentstroke}{rgb}{0.000000,0.000000,0.000000}%
\pgfsetstrokecolor{currentstroke}%
\pgfsetdash{}{0pt}%
\pgfsys@defobject{currentmarker}{\pgfqpoint{-0.048611in}{0.000000in}}{\pgfqpoint{0.000000in}{0.000000in}}{%
\pgfpathmoveto{\pgfqpoint{0.000000in}{0.000000in}}%
\pgfpathlineto{\pgfqpoint{-0.048611in}{0.000000in}}%
\pgfusepath{stroke,fill}%
}%
\begin{pgfscope}%
\pgfsys@transformshift{7.481955in}{5.781600in}%
\pgfsys@useobject{currentmarker}{}%
\end{pgfscope}%
\end{pgfscope}%
\begin{pgfscope}%
\pgftext[x=6.987122in,y=5.728838in,left,base]{\sffamily\fontsize{10.000000}{12.000000}\selectfont 0.003}%
\end{pgfscope}%
\begin{pgfscope}%
\pgftext[x=6.815193in,y=3.252150in,,bottom,rotate=90.000000]{\sffamily\fontsize{16.000000}{19.200000}\selectfont \(\displaystyle Poynting\) \(\displaystyle vector\)}%
\end{pgfscope}%
\begin{pgfscope}%
\pgfpathrectangle{\pgfqpoint{7.481955in}{0.722700in}}{\pgfqpoint{4.812045in}{5.058900in}} %
\pgfusepath{clip}%
\pgfsetrectcap%
\pgfsetroundjoin%
\pgfsetlinewidth{1.505625pt}%
\definecolor{currentstroke}{rgb}{0.000000,0.000000,0.000000}%
\pgfsetstrokecolor{currentstroke}%
\pgfsetdash{}{0pt}%
\pgfpathmoveto{\pgfqpoint{7.481955in}{3.252150in}}%
\pgfpathlineto{\pgfqpoint{10.590954in}{3.253125in}}%
\pgfpathlineto{\pgfqpoint{10.594302in}{3.256292in}}%
\pgfpathlineto{\pgfqpoint{10.598068in}{3.263496in}}%
\pgfpathlineto{\pgfqpoint{10.609784in}{3.290376in}}%
\pgfpathlineto{\pgfqpoint{10.611458in}{3.289269in}}%
\pgfpathlineto{\pgfqpoint{10.613550in}{3.284772in}}%
\pgfpathlineto{\pgfqpoint{10.617316in}{3.269808in}}%
\pgfpathlineto{\pgfqpoint{10.621919in}{3.252931in}}%
\pgfpathlineto{\pgfqpoint{10.623174in}{3.252172in}}%
\pgfpathlineto{\pgfqpoint{10.623593in}{3.252535in}}%
\pgfpathlineto{\pgfqpoint{10.624848in}{3.255796in}}%
\pgfpathlineto{\pgfqpoint{10.626940in}{3.269616in}}%
\pgfpathlineto{\pgfqpoint{10.629869in}{3.308835in}}%
\pgfpathlineto{\pgfqpoint{10.634472in}{3.411217in}}%
\pgfpathlineto{\pgfqpoint{10.641167in}{3.557649in}}%
\pgfpathlineto{\pgfqpoint{10.643678in}{3.573928in}}%
\pgfpathlineto{\pgfqpoint{10.644515in}{3.572018in}}%
\pgfpathlineto{\pgfqpoint{10.646188in}{3.556602in}}%
\pgfpathlineto{\pgfqpoint{10.648699in}{3.506330in}}%
\pgfpathlineto{\pgfqpoint{10.659160in}{3.252063in}}%
\pgfpathlineto{\pgfqpoint{10.659997in}{3.256517in}}%
\pgfpathlineto{\pgfqpoint{10.661670in}{3.286013in}}%
\pgfpathlineto{\pgfqpoint{10.664181in}{3.383122in}}%
\pgfpathlineto{\pgfqpoint{10.668366in}{3.661199in}}%
\pgfpathlineto{\pgfqpoint{10.675897in}{4.165698in}}%
\pgfpathlineto{\pgfqpoint{10.677990in}{4.210421in}}%
\pgfpathlineto{\pgfqpoint{10.678408in}{4.211697in}}%
\pgfpathlineto{\pgfqpoint{10.678826in}{4.210316in}}%
\pgfpathlineto{\pgfqpoint{10.680082in}{4.190171in}}%
\pgfpathlineto{\pgfqpoint{10.682174in}{4.105203in}}%
\pgfpathlineto{\pgfqpoint{10.685940in}{3.826120in}}%
\pgfpathlineto{\pgfqpoint{10.693053in}{3.291039in}}%
\pgfpathlineto{\pgfqpoint{10.695146in}{3.251746in}}%
\pgfpathlineto{\pgfqpoint{10.695982in}{3.259571in}}%
\pgfpathlineto{\pgfqpoint{10.697656in}{3.317520in}}%
\pgfpathlineto{\pgfqpoint{10.700167in}{3.505479in}}%
\pgfpathlineto{\pgfqpoint{10.704770in}{4.069022in}}%
\pgfpathlineto{\pgfqpoint{10.710628in}{4.740258in}}%
\pgfpathlineto{\pgfqpoint{10.713138in}{4.851719in}}%
\pgfpathlineto{\pgfqpoint{10.713557in}{4.855793in}}%
\pgfpathlineto{\pgfqpoint{10.713975in}{4.855545in}}%
\pgfpathlineto{\pgfqpoint{10.714812in}{4.842054in}}%
\pgfpathlineto{\pgfqpoint{10.716486in}{4.764181in}}%
\pgfpathlineto{\pgfqpoint{10.719415in}{4.484656in}}%
\pgfpathlineto{\pgfqpoint{10.730294in}{3.262441in}}%
\pgfpathlineto{\pgfqpoint{10.731131in}{3.251581in}}%
\pgfpathlineto{\pgfqpoint{10.731550in}{3.253405in}}%
\pgfpathlineto{\pgfqpoint{10.732805in}{3.288133in}}%
\pgfpathlineto{\pgfqpoint{10.734897in}{3.440308in}}%
\pgfpathlineto{\pgfqpoint{10.738245in}{3.884456in}}%
\pgfpathlineto{\pgfqpoint{10.747032in}{5.157967in}}%
\pgfpathlineto{\pgfqpoint{10.749124in}{5.244909in}}%
\pgfpathlineto{\pgfqpoint{10.749543in}{5.246622in}}%
\pgfpathlineto{\pgfqpoint{10.750380in}{5.234071in}}%
\pgfpathlineto{\pgfqpoint{10.752053in}{5.146383in}}%
\pgfpathlineto{\pgfqpoint{10.754982in}{4.816096in}}%
\pgfpathlineto{\pgfqpoint{10.766280in}{3.267096in}}%
\pgfpathlineto{\pgfqpoint{10.767117in}{3.251845in}}%
\pgfpathlineto{\pgfqpoint{10.767535in}{3.252542in}}%
\pgfpathlineto{\pgfqpoint{10.768791in}{3.287966in}}%
\pgfpathlineto{\pgfqpoint{10.770883in}{3.453466in}}%
\pgfpathlineto{\pgfqpoint{10.774231in}{3.941169in}}%
\pgfpathlineto{\pgfqpoint{10.783018in}{5.323721in}}%
\pgfpathlineto{\pgfqpoint{10.785110in}{5.414289in}}%
\pgfpathlineto{\pgfqpoint{10.785528in}{5.415412in}}%
\pgfpathlineto{\pgfqpoint{10.786365in}{5.400415in}}%
\pgfpathlineto{\pgfqpoint{10.788039in}{5.303127in}}%
\pgfpathlineto{\pgfqpoint{10.790968in}{4.943927in}}%
\pgfpathlineto{\pgfqpoint{10.802266in}{3.271465in}}%
\pgfpathlineto{\pgfqpoint{10.803521in}{3.251831in}}%
\pgfpathlineto{\pgfqpoint{10.804358in}{3.267860in}}%
\pgfpathlineto{\pgfqpoint{10.806032in}{3.368172in}}%
\pgfpathlineto{\pgfqpoint{10.808961in}{3.735393in}}%
\pgfpathlineto{\pgfqpoint{10.820259in}{5.448967in}}%
\pgfpathlineto{\pgfqpoint{10.821514in}{5.471965in}}%
\pgfpathlineto{\pgfqpoint{10.821933in}{5.467866in}}%
\pgfpathlineto{\pgfqpoint{10.823188in}{5.420567in}}%
\pgfpathlineto{\pgfqpoint{10.825280in}{5.231842in}}%
\pgfpathlineto{\pgfqpoint{10.829046in}{4.630228in}}%
\pgfpathlineto{\pgfqpoint{10.836578in}{3.388317in}}%
\pgfpathlineto{\pgfqpoint{10.839089in}{3.253675in}}%
\pgfpathlineto{\pgfqpoint{10.839507in}{3.251456in}}%
\pgfpathlineto{\pgfqpoint{10.839507in}{3.251456in}}%
\pgfpathlineto{\pgfqpoint{10.839507in}{3.251456in}}%
\pgfpathlineto{\pgfqpoint{10.840344in}{3.264749in}}%
\pgfpathlineto{\pgfqpoint{10.842018in}{3.360646in}}%
\pgfpathlineto{\pgfqpoint{10.844947in}{3.723912in}}%
\pgfpathlineto{\pgfqpoint{10.856663in}{5.474572in}}%
\pgfpathlineto{\pgfqpoint{10.857500in}{5.486550in}}%
\pgfpathlineto{\pgfqpoint{10.857918in}{5.483653in}}%
\pgfpathlineto{\pgfqpoint{10.859174in}{5.439694in}}%
\pgfpathlineto{\pgfqpoint{10.861266in}{5.255333in}}%
\pgfpathlineto{\pgfqpoint{10.865032in}{4.656411in}}%
\pgfpathlineto{\pgfqpoint{10.872564in}{3.398713in}}%
\pgfpathlineto{\pgfqpoint{10.875074in}{3.255162in}}%
\pgfpathlineto{\pgfqpoint{10.875493in}{3.251454in}}%
\pgfpathlineto{\pgfqpoint{10.875911in}{3.253680in}}%
\pgfpathlineto{\pgfqpoint{10.877166in}{3.295681in}}%
\pgfpathlineto{\pgfqpoint{10.879259in}{3.477173in}}%
\pgfpathlineto{\pgfqpoint{10.883025in}{4.073214in}}%
\pgfpathlineto{\pgfqpoint{10.890975in}{5.375145in}}%
\pgfpathlineto{\pgfqpoint{10.893486in}{5.488999in}}%
\pgfpathlineto{\pgfqpoint{10.893904in}{5.487524in}}%
\pgfpathlineto{\pgfqpoint{10.895159in}{5.447728in}}%
\pgfpathlineto{\pgfqpoint{10.897252in}{5.269536in}}%
\pgfpathlineto{\pgfqpoint{10.901017in}{4.677154in}}%
\pgfpathlineto{\pgfqpoint{10.908968in}{3.369925in}}%
\pgfpathlineto{\pgfqpoint{10.911478in}{3.251825in}}%
\pgfpathlineto{\pgfqpoint{10.911897in}{3.252567in}}%
\pgfpathlineto{\pgfqpoint{10.913152in}{3.290199in}}%
\pgfpathlineto{\pgfqpoint{10.915244in}{3.465120in}}%
\pgfpathlineto{\pgfqpoint{10.918592in}{3.976550in}}%
\pgfpathlineto{\pgfqpoint{10.927379in}{5.400840in}}%
\pgfpathlineto{\pgfqpoint{10.929471in}{5.488727in}}%
\pgfpathlineto{\pgfqpoint{10.929890in}{5.488727in}}%
\pgfpathlineto{\pgfqpoint{10.930727in}{5.470964in}}%
\pgfpathlineto{\pgfqpoint{10.932400in}{5.366346in}}%
\pgfpathlineto{\pgfqpoint{10.935329in}{4.990554in}}%
\pgfpathlineto{\pgfqpoint{10.946627in}{3.271806in}}%
\pgfpathlineto{\pgfqpoint{10.947883in}{3.251825in}}%
\pgfpathlineto{\pgfqpoint{10.948719in}{3.268115in}}%
\pgfpathlineto{\pgfqpoint{10.950393in}{3.369929in}}%
\pgfpathlineto{\pgfqpoint{10.953322in}{3.741874in}}%
\pgfpathlineto{\pgfqpoint{10.964620in}{5.467100in}}%
\pgfpathlineto{\pgfqpoint{10.965875in}{5.489295in}}%
\pgfpathlineto{\pgfqpoint{10.966712in}{5.474482in}}%
\pgfpathlineto{\pgfqpoint{10.968386in}{5.375482in}}%
\pgfpathlineto{\pgfqpoint{10.971315in}{5.007419in}}%
\pgfpathlineto{\pgfqpoint{10.983031in}{3.261831in}}%
\pgfpathlineto{\pgfqpoint{10.983868in}{3.251454in}}%
\pgfpathlineto{\pgfqpoint{10.984287in}{3.255164in}}%
\pgfpathlineto{\pgfqpoint{10.985542in}{3.301553in}}%
\pgfpathlineto{\pgfqpoint{10.987634in}{3.489678in}}%
\pgfpathlineto{\pgfqpoint{10.991400in}{4.093138in}}%
\pgfpathlineto{\pgfqpoint{10.998932in}{5.347110in}}%
\pgfpathlineto{\pgfqpoint{11.001443in}{5.486513in}}%
\pgfpathlineto{\pgfqpoint{11.001861in}{5.489481in}}%
\pgfpathlineto{\pgfqpoint{11.002280in}{5.486513in}}%
\pgfpathlineto{\pgfqpoint{11.003535in}{5.442307in}}%
\pgfpathlineto{\pgfqpoint{11.005627in}{5.257460in}}%
\pgfpathlineto{\pgfqpoint{11.009393in}{4.657611in}}%
\pgfpathlineto{\pgfqpoint{11.016925in}{3.398793in}}%
\pgfpathlineto{\pgfqpoint{11.019436in}{3.255164in}}%
\pgfpathlineto{\pgfqpoint{11.019854in}{3.251454in}}%
\pgfpathlineto{\pgfqpoint{11.020272in}{3.253680in}}%
\pgfpathlineto{\pgfqpoint{11.021528in}{3.295699in}}%
\pgfpathlineto{\pgfqpoint{11.023620in}{3.477253in}}%
\pgfpathlineto{\pgfqpoint{11.027386in}{4.073441in}}%
\pgfpathlineto{\pgfqpoint{11.035336in}{5.375482in}}%
\pgfpathlineto{\pgfqpoint{11.037847in}{5.489295in}}%
\pgfpathlineto{\pgfqpoint{11.038265in}{5.487811in}}%
\pgfpathlineto{\pgfqpoint{11.039521in}{5.447984in}}%
\pgfpathlineto{\pgfqpoint{11.041613in}{5.269737in}}%
\pgfpathlineto{\pgfqpoint{11.045379in}{4.677260in}}%
\pgfpathlineto{\pgfqpoint{11.053329in}{3.369929in}}%
\pgfpathlineto{\pgfqpoint{11.055840in}{3.251825in}}%
\pgfpathlineto{\pgfqpoint{11.056258in}{3.252567in}}%
\pgfpathlineto{\pgfqpoint{11.057514in}{3.290200in}}%
\pgfpathlineto{\pgfqpoint{11.059606in}{3.465125in}}%
\pgfpathlineto{\pgfqpoint{11.062953in}{3.976561in}}%
\pgfpathlineto{\pgfqpoint{11.071740in}{5.400846in}}%
\pgfpathlineto{\pgfqpoint{11.073833in}{5.488727in}}%
\pgfpathlineto{\pgfqpoint{11.074251in}{5.488727in}}%
\pgfpathlineto{\pgfqpoint{11.075088in}{5.470961in}}%
\pgfpathlineto{\pgfqpoint{11.076762in}{5.366340in}}%
\pgfpathlineto{\pgfqpoint{11.079691in}{4.990542in}}%
\pgfpathlineto{\pgfqpoint{11.090989in}{3.271805in}}%
\pgfpathlineto{\pgfqpoint{11.092244in}{3.251825in}}%
\pgfpathlineto{\pgfqpoint{11.093081in}{3.268114in}}%
\pgfpathlineto{\pgfqpoint{11.094755in}{3.369925in}}%
\pgfpathlineto{\pgfqpoint{11.097684in}{3.741850in}}%
\pgfpathlineto{\pgfqpoint{11.108982in}{5.466834in}}%
\pgfpathlineto{\pgfqpoint{11.110237in}{5.488999in}}%
\pgfpathlineto{\pgfqpoint{11.111074in}{5.474170in}}%
\pgfpathlineto{\pgfqpoint{11.112747in}{5.375145in}}%
\pgfpathlineto{\pgfqpoint{11.115677in}{5.007076in}}%
\pgfpathlineto{\pgfqpoint{11.127393in}{3.261826in}}%
\pgfpathlineto{\pgfqpoint{11.128230in}{3.251454in}}%
\pgfpathlineto{\pgfqpoint{11.128648in}{3.255162in}}%
\pgfpathlineto{\pgfqpoint{11.129903in}{3.301529in}}%
\pgfpathlineto{\pgfqpoint{11.131996in}{3.489542in}}%
\pgfpathlineto{\pgfqpoint{11.135762in}{4.092535in}}%
\pgfpathlineto{\pgfqpoint{11.143293in}{5.344781in}}%
\pgfpathlineto{\pgfqpoint{11.145804in}{5.483653in}}%
\pgfpathlineto{\pgfqpoint{11.146223in}{5.486550in}}%
\pgfpathlineto{\pgfqpoint{11.146223in}{5.486550in}}%
\pgfpathlineto{\pgfqpoint{11.146223in}{5.486550in}}%
\pgfpathlineto{\pgfqpoint{11.147059in}{5.474572in}}%
\pgfpathlineto{\pgfqpoint{11.148733in}{5.381071in}}%
\pgfpathlineto{\pgfqpoint{11.151244in}{5.085284in}}%
\pgfpathlineto{\pgfqpoint{11.156684in}{4.091211in}}%
\pgfpathlineto{\pgfqpoint{11.161705in}{3.360646in}}%
\pgfpathlineto{\pgfqpoint{11.164215in}{3.251456in}}%
\pgfpathlineto{\pgfqpoint{11.165052in}{3.261797in}}%
\pgfpathlineto{\pgfqpoint{11.166726in}{3.351961in}}%
\pgfpathlineto{\pgfqpoint{11.169237in}{3.642981in}}%
\pgfpathlineto{\pgfqpoint{11.174676in}{4.630228in}}%
\pgfpathlineto{\pgfqpoint{11.179698in}{5.360773in}}%
\pgfpathlineto{\pgfqpoint{11.182208in}{5.471965in}}%
\pgfpathlineto{\pgfqpoint{11.182627in}{5.470170in}}%
\pgfpathlineto{\pgfqpoint{11.183882in}{5.429675in}}%
\pgfpathlineto{\pgfqpoint{11.185974in}{5.251311in}}%
\pgfpathlineto{\pgfqpoint{11.189740in}{4.661968in}}%
\pgfpathlineto{\pgfqpoint{11.197691in}{3.368172in}}%
\pgfpathlineto{\pgfqpoint{11.200201in}{3.251831in}}%
\pgfpathlineto{\pgfqpoint{11.200620in}{3.252560in}}%
\pgfpathlineto{\pgfqpoint{11.201875in}{3.289529in}}%
\pgfpathlineto{\pgfqpoint{11.203967in}{3.461045in}}%
\pgfpathlineto{\pgfqpoint{11.207315in}{3.960756in}}%
\pgfpathlineto{\pgfqpoint{11.215683in}{5.303127in}}%
\pgfpathlineto{\pgfqpoint{11.218194in}{5.415412in}}%
\pgfpathlineto{\pgfqpoint{11.218612in}{5.414289in}}%
\pgfpathlineto{\pgfqpoint{11.219868in}{5.376630in}}%
\pgfpathlineto{\pgfqpoint{11.221960in}{5.205513in}}%
\pgfpathlineto{\pgfqpoint{11.225307in}{4.709538in}}%
\pgfpathlineto{\pgfqpoint{11.234095in}{3.338804in}}%
\pgfpathlineto{\pgfqpoint{11.236187in}{3.252542in}}%
\pgfpathlineto{\pgfqpoint{11.236605in}{3.251845in}}%
\pgfpathlineto{\pgfqpoint{11.237442in}{3.267096in}}%
\pgfpathlineto{\pgfqpoint{11.239116in}{3.361992in}}%
\pgfpathlineto{\pgfqpoint{11.242045in}{3.705599in}}%
\pgfpathlineto{\pgfqpoint{11.252924in}{5.219859in}}%
\pgfpathlineto{\pgfqpoint{11.254180in}{5.246622in}}%
\pgfpathlineto{\pgfqpoint{11.254598in}{5.244909in}}%
\pgfpathlineto{\pgfqpoint{11.255853in}{5.208097in}}%
\pgfpathlineto{\pgfqpoint{11.257946in}{5.047207in}}%
\pgfpathlineto{\pgfqpoint{11.261712in}{4.519330in}}%
\pgfpathlineto{\pgfqpoint{11.269662in}{3.365891in}}%
\pgfpathlineto{\pgfqpoint{11.272173in}{3.253405in}}%
\pgfpathlineto{\pgfqpoint{11.272591in}{3.251581in}}%
\pgfpathlineto{\pgfqpoint{11.272591in}{3.251581in}}%
\pgfpathlineto{\pgfqpoint{11.272591in}{3.251581in}}%
\pgfpathlineto{\pgfqpoint{11.273428in}{3.262441in}}%
\pgfpathlineto{\pgfqpoint{11.275102in}{3.339836in}}%
\pgfpathlineto{\pgfqpoint{11.278031in}{3.626015in}}%
\pgfpathlineto{\pgfqpoint{11.288492in}{4.828853in}}%
\pgfpathlineto{\pgfqpoint{11.290165in}{4.855793in}}%
\pgfpathlineto{\pgfqpoint{11.291002in}{4.843363in}}%
\pgfpathlineto{\pgfqpoint{11.292676in}{4.768669in}}%
\pgfpathlineto{\pgfqpoint{11.295605in}{4.501437in}}%
\pgfpathlineto{\pgfqpoint{11.306903in}{3.281494in}}%
\pgfpathlineto{\pgfqpoint{11.308577in}{3.251746in}}%
\pgfpathlineto{\pgfqpoint{11.309414in}{3.257700in}}%
\pgfpathlineto{\pgfqpoint{11.311087in}{3.308195in}}%
\pgfpathlineto{\pgfqpoint{11.314016in}{3.496108in}}%
\pgfpathlineto{\pgfqpoint{11.323222in}{4.178178in}}%
\pgfpathlineto{\pgfqpoint{11.325314in}{4.211697in}}%
\pgfpathlineto{\pgfqpoint{11.326151in}{4.206517in}}%
\pgfpathlineto{\pgfqpoint{11.327825in}{4.165698in}}%
\pgfpathlineto{\pgfqpoint{11.330754in}{4.011735in}}%
\pgfpathlineto{\pgfqpoint{11.342470in}{3.275992in}}%
\pgfpathlineto{\pgfqpoint{11.344563in}{3.252063in}}%
\pgfpathlineto{\pgfqpoint{11.344981in}{3.252260in}}%
\pgfpathlineto{\pgfqpoint{11.346236in}{3.261775in}}%
\pgfpathlineto{\pgfqpoint{11.348328in}{3.302469in}}%
\pgfpathlineto{\pgfqpoint{11.359626in}{3.573454in}}%
\pgfpathlineto{\pgfqpoint{11.360045in}{3.573928in}}%
\pgfpathlineto{\pgfqpoint{11.360463in}{3.573452in}}%
\pgfpathlineto{\pgfqpoint{11.361718in}{3.566537in}}%
\pgfpathlineto{\pgfqpoint{11.363811in}{3.538686in}}%
\pgfpathlineto{\pgfqpoint{11.367995in}{3.443491in}}%
\pgfpathlineto{\pgfqpoint{11.374690in}{3.295224in}}%
\pgfpathlineto{\pgfqpoint{11.378038in}{3.259983in}}%
\pgfpathlineto{\pgfqpoint{11.380548in}{3.252172in}}%
\pgfpathlineto{\pgfqpoint{11.381804in}{3.252931in}}%
\pgfpathlineto{\pgfqpoint{11.383896in}{3.258854in}}%
\pgfpathlineto{\pgfqpoint{11.392683in}{3.289753in}}%
\pgfpathlineto{\pgfqpoint{11.394775in}{3.290126in}}%
\pgfpathlineto{\pgfqpoint{11.396867in}{3.287471in}}%
\pgfpathlineto{\pgfqpoint{11.400633in}{3.277829in}}%
\pgfpathlineto{\pgfqpoint{11.408165in}{3.258245in}}%
\pgfpathlineto{\pgfqpoint{11.412350in}{3.253377in}}%
\pgfpathlineto{\pgfqpoint{11.416116in}{3.252173in}}%
\pgfpathlineto{\pgfqpoint{11.424484in}{3.252864in}}%
\pgfpathlineto{\pgfqpoint{11.436201in}{3.252483in}}%
\pgfpathlineto{\pgfqpoint{11.460470in}{3.252151in}}%
\pgfpathlineto{\pgfqpoint{12.293582in}{3.252150in}}%
\pgfpathlineto{\pgfqpoint{12.293582in}{3.252150in}}%
\pgfusepath{stroke}%
\end{pgfscope}%
\begin{pgfscope}%
\pgfsetrectcap%
\pgfsetmiterjoin%
\pgfsetlinewidth{0.803000pt}%
\definecolor{currentstroke}{rgb}{0.000000,0.000000,0.000000}%
\pgfsetstrokecolor{currentstroke}%
\pgfsetdash{}{0pt}%
\pgfpathmoveto{\pgfqpoint{7.481955in}{0.722700in}}%
\pgfpathlineto{\pgfqpoint{7.481955in}{5.781600in}}%
\pgfusepath{stroke}%
\end{pgfscope}%
\begin{pgfscope}%
\pgfsetrectcap%
\pgfsetmiterjoin%
\pgfsetlinewidth{0.803000pt}%
\definecolor{currentstroke}{rgb}{0.000000,0.000000,0.000000}%
\pgfsetstrokecolor{currentstroke}%
\pgfsetdash{}{0pt}%
\pgfpathmoveto{\pgfqpoint{12.294000in}{0.722700in}}%
\pgfpathlineto{\pgfqpoint{12.294000in}{5.781600in}}%
\pgfusepath{stroke}%
\end{pgfscope}%
\begin{pgfscope}%
\pgfsetrectcap%
\pgfsetmiterjoin%
\pgfsetlinewidth{0.803000pt}%
\definecolor{currentstroke}{rgb}{0.000000,0.000000,0.000000}%
\pgfsetstrokecolor{currentstroke}%
\pgfsetdash{}{0pt}%
\pgfpathmoveto{\pgfqpoint{7.481955in}{0.722700in}}%
\pgfpathlineto{\pgfqpoint{12.294000in}{0.722700in}}%
\pgfusepath{stroke}%
\end{pgfscope}%
\begin{pgfscope}%
\pgfsetrectcap%
\pgfsetmiterjoin%
\pgfsetlinewidth{0.803000pt}%
\definecolor{currentstroke}{rgb}{0.000000,0.000000,0.000000}%
\pgfsetstrokecolor{currentstroke}%
\pgfsetdash{}{0pt}%
\pgfpathmoveto{\pgfqpoint{7.481955in}{5.781600in}}%
\pgfpathlineto{\pgfqpoint{12.294000in}{5.781600in}}%
\pgfusepath{stroke}%
\end{pgfscope}%
\end{pgfpicture}%
\makeatother%
\endgroup%
}}
        \subcaption{Simulation before the wave hit the right side of the numerical window.}
        \label{fig:task2_1}
    \end{subfigure}\\
    \begin{subfigure}[b]{0.5\textwidth}
        \noindent\makebox[\textwidth]{\scalebox{0.5}{%% Creator: Matplotlib, PGF backend
%%
%% To include the figure in your LaTeX document, write
%%   \input{<filename>.pgf}
%%
%% Make sure the required packages are loaded in your preamble
%%   \usepackage{pgf}
%%
%% Figures using additional raster images can only be included by \input if
%% they are in the same directory as the main LaTeX file. For loading figures
%% from other directories you can use the `import` package
%%   \usepackage{import}
%% and then include the figures with
%%   \import{<path to file>}{<filename>.pgf}
%%
%% Matplotlib used the following preamble
%%   \usepackage{fontspec}
%%   \setmainfont{DejaVu Serif}
%%   \setsansfont{DejaVu Sans}
%%   \setmonofont{DejaVu Sans Mono}
%%
\begingroup%
\makeatletter%
\begin{pgfpicture}%
\pgfpathrectangle{\pgfpointorigin}{\pgfqpoint{13.660000in}{6.570000in}}%
\pgfusepath{use as bounding box, clip}%
\begin{pgfscope}%
\pgfsetbuttcap%
\pgfsetmiterjoin%
\definecolor{currentfill}{rgb}{1.000000,1.000000,1.000000}%
\pgfsetfillcolor{currentfill}%
\pgfsetlinewidth{0.000000pt}%
\definecolor{currentstroke}{rgb}{1.000000,1.000000,1.000000}%
\pgfsetstrokecolor{currentstroke}%
\pgfsetdash{}{0pt}%
\pgfpathmoveto{\pgfqpoint{0.000000in}{0.000000in}}%
\pgfpathlineto{\pgfqpoint{13.660000in}{0.000000in}}%
\pgfpathlineto{\pgfqpoint{13.660000in}{6.570000in}}%
\pgfpathlineto{\pgfqpoint{0.000000in}{6.570000in}}%
\pgfpathclose%
\pgfusepath{fill}%
\end{pgfscope}%
\begin{pgfscope}%
\pgfsetbuttcap%
\pgfsetmiterjoin%
\definecolor{currentfill}{rgb}{1.000000,1.000000,1.000000}%
\pgfsetfillcolor{currentfill}%
\pgfsetlinewidth{0.000000pt}%
\definecolor{currentstroke}{rgb}{0.000000,0.000000,0.000000}%
\pgfsetstrokecolor{currentstroke}%
\pgfsetstrokeopacity{0.000000}%
\pgfsetdash{}{0pt}%
\pgfpathmoveto{\pgfqpoint{1.707500in}{0.722700in}}%
\pgfpathlineto{\pgfqpoint{6.519545in}{0.722700in}}%
\pgfpathlineto{\pgfqpoint{6.519545in}{5.781600in}}%
\pgfpathlineto{\pgfqpoint{1.707500in}{5.781600in}}%
\pgfpathclose%
\pgfusepath{fill}%
\end{pgfscope}%
\begin{pgfscope}%
\pgfsetbuttcap%
\pgfsetroundjoin%
\definecolor{currentfill}{rgb}{0.000000,0.000000,0.000000}%
\pgfsetfillcolor{currentfill}%
\pgfsetlinewidth{0.803000pt}%
\definecolor{currentstroke}{rgb}{0.000000,0.000000,0.000000}%
\pgfsetstrokecolor{currentstroke}%
\pgfsetdash{}{0pt}%
\pgfsys@defobject{currentmarker}{\pgfqpoint{0.000000in}{-0.048611in}}{\pgfqpoint{0.000000in}{0.000000in}}{%
\pgfpathmoveto{\pgfqpoint{0.000000in}{0.000000in}}%
\pgfpathlineto{\pgfqpoint{0.000000in}{-0.048611in}}%
\pgfusepath{stroke,fill}%
}%
\begin{pgfscope}%
\pgfsys@transformshift{1.707500in}{0.722700in}%
\pgfsys@useobject{currentmarker}{}%
\end{pgfscope}%
\end{pgfscope}%
\begin{pgfscope}%
\pgftext[x=1.707500in,y=0.625478in,,top]{\sffamily\fontsize{10.000000}{12.000000}\selectfont 0}%
\end{pgfscope}%
\begin{pgfscope}%
\pgfsetbuttcap%
\pgfsetroundjoin%
\definecolor{currentfill}{rgb}{0.000000,0.000000,0.000000}%
\pgfsetfillcolor{currentfill}%
\pgfsetlinewidth{0.803000pt}%
\definecolor{currentstroke}{rgb}{0.000000,0.000000,0.000000}%
\pgfsetstrokecolor{currentstroke}%
\pgfsetdash{}{0pt}%
\pgfsys@defobject{currentmarker}{\pgfqpoint{0.000000in}{-0.048611in}}{\pgfqpoint{0.000000in}{0.000000in}}{%
\pgfpathmoveto{\pgfqpoint{0.000000in}{0.000000in}}%
\pgfpathlineto{\pgfqpoint{0.000000in}{-0.048611in}}%
\pgfusepath{stroke,fill}%
}%
\begin{pgfscope}%
\pgfsys@transformshift{2.277648in}{0.722700in}%
\pgfsys@useobject{currentmarker}{}%
\end{pgfscope}%
\end{pgfscope}%
\begin{pgfscope}%
\pgftext[x=2.277648in,y=0.625478in,,top]{\sffamily\fontsize{10.000000}{12.000000}\selectfont 5}%
\end{pgfscope}%
\begin{pgfscope}%
\pgfsetbuttcap%
\pgfsetroundjoin%
\definecolor{currentfill}{rgb}{0.000000,0.000000,0.000000}%
\pgfsetfillcolor{currentfill}%
\pgfsetlinewidth{0.803000pt}%
\definecolor{currentstroke}{rgb}{0.000000,0.000000,0.000000}%
\pgfsetstrokecolor{currentstroke}%
\pgfsetdash{}{0pt}%
\pgfsys@defobject{currentmarker}{\pgfqpoint{0.000000in}{-0.048611in}}{\pgfqpoint{0.000000in}{0.000000in}}{%
\pgfpathmoveto{\pgfqpoint{0.000000in}{0.000000in}}%
\pgfpathlineto{\pgfqpoint{0.000000in}{-0.048611in}}%
\pgfusepath{stroke,fill}%
}%
\begin{pgfscope}%
\pgfsys@transformshift{2.847795in}{0.722700in}%
\pgfsys@useobject{currentmarker}{}%
\end{pgfscope}%
\end{pgfscope}%
\begin{pgfscope}%
\pgftext[x=2.847795in,y=0.625478in,,top]{\sffamily\fontsize{10.000000}{12.000000}\selectfont 10}%
\end{pgfscope}%
\begin{pgfscope}%
\pgfsetbuttcap%
\pgfsetroundjoin%
\definecolor{currentfill}{rgb}{0.000000,0.000000,0.000000}%
\pgfsetfillcolor{currentfill}%
\pgfsetlinewidth{0.803000pt}%
\definecolor{currentstroke}{rgb}{0.000000,0.000000,0.000000}%
\pgfsetstrokecolor{currentstroke}%
\pgfsetdash{}{0pt}%
\pgfsys@defobject{currentmarker}{\pgfqpoint{0.000000in}{-0.048611in}}{\pgfqpoint{0.000000in}{0.000000in}}{%
\pgfpathmoveto{\pgfqpoint{0.000000in}{0.000000in}}%
\pgfpathlineto{\pgfqpoint{0.000000in}{-0.048611in}}%
\pgfusepath{stroke,fill}%
}%
\begin{pgfscope}%
\pgfsys@transformshift{3.417943in}{0.722700in}%
\pgfsys@useobject{currentmarker}{}%
\end{pgfscope}%
\end{pgfscope}%
\begin{pgfscope}%
\pgftext[x=3.417943in,y=0.625478in,,top]{\sffamily\fontsize{10.000000}{12.000000}\selectfont 15}%
\end{pgfscope}%
\begin{pgfscope}%
\pgfsetbuttcap%
\pgfsetroundjoin%
\definecolor{currentfill}{rgb}{0.000000,0.000000,0.000000}%
\pgfsetfillcolor{currentfill}%
\pgfsetlinewidth{0.803000pt}%
\definecolor{currentstroke}{rgb}{0.000000,0.000000,0.000000}%
\pgfsetstrokecolor{currentstroke}%
\pgfsetdash{}{0pt}%
\pgfsys@defobject{currentmarker}{\pgfqpoint{0.000000in}{-0.048611in}}{\pgfqpoint{0.000000in}{0.000000in}}{%
\pgfpathmoveto{\pgfqpoint{0.000000in}{0.000000in}}%
\pgfpathlineto{\pgfqpoint{0.000000in}{-0.048611in}}%
\pgfusepath{stroke,fill}%
}%
\begin{pgfscope}%
\pgfsys@transformshift{3.988090in}{0.722700in}%
\pgfsys@useobject{currentmarker}{}%
\end{pgfscope}%
\end{pgfscope}%
\begin{pgfscope}%
\pgftext[x=3.988090in,y=0.625478in,,top]{\sffamily\fontsize{10.000000}{12.000000}\selectfont 20}%
\end{pgfscope}%
\begin{pgfscope}%
\pgfsetbuttcap%
\pgfsetroundjoin%
\definecolor{currentfill}{rgb}{0.000000,0.000000,0.000000}%
\pgfsetfillcolor{currentfill}%
\pgfsetlinewidth{0.803000pt}%
\definecolor{currentstroke}{rgb}{0.000000,0.000000,0.000000}%
\pgfsetstrokecolor{currentstroke}%
\pgfsetdash{}{0pt}%
\pgfsys@defobject{currentmarker}{\pgfqpoint{0.000000in}{-0.048611in}}{\pgfqpoint{0.000000in}{0.000000in}}{%
\pgfpathmoveto{\pgfqpoint{0.000000in}{0.000000in}}%
\pgfpathlineto{\pgfqpoint{0.000000in}{-0.048611in}}%
\pgfusepath{stroke,fill}%
}%
\begin{pgfscope}%
\pgfsys@transformshift{4.558238in}{0.722700in}%
\pgfsys@useobject{currentmarker}{}%
\end{pgfscope}%
\end{pgfscope}%
\begin{pgfscope}%
\pgftext[x=4.558238in,y=0.625478in,,top]{\sffamily\fontsize{10.000000}{12.000000}\selectfont 25}%
\end{pgfscope}%
\begin{pgfscope}%
\pgfsetbuttcap%
\pgfsetroundjoin%
\definecolor{currentfill}{rgb}{0.000000,0.000000,0.000000}%
\pgfsetfillcolor{currentfill}%
\pgfsetlinewidth{0.803000pt}%
\definecolor{currentstroke}{rgb}{0.000000,0.000000,0.000000}%
\pgfsetstrokecolor{currentstroke}%
\pgfsetdash{}{0pt}%
\pgfsys@defobject{currentmarker}{\pgfqpoint{0.000000in}{-0.048611in}}{\pgfqpoint{0.000000in}{0.000000in}}{%
\pgfpathmoveto{\pgfqpoint{0.000000in}{0.000000in}}%
\pgfpathlineto{\pgfqpoint{0.000000in}{-0.048611in}}%
\pgfusepath{stroke,fill}%
}%
\begin{pgfscope}%
\pgfsys@transformshift{5.128385in}{0.722700in}%
\pgfsys@useobject{currentmarker}{}%
\end{pgfscope}%
\end{pgfscope}%
\begin{pgfscope}%
\pgftext[x=5.128385in,y=0.625478in,,top]{\sffamily\fontsize{10.000000}{12.000000}\selectfont 30}%
\end{pgfscope}%
\begin{pgfscope}%
\pgfsetbuttcap%
\pgfsetroundjoin%
\definecolor{currentfill}{rgb}{0.000000,0.000000,0.000000}%
\pgfsetfillcolor{currentfill}%
\pgfsetlinewidth{0.803000pt}%
\definecolor{currentstroke}{rgb}{0.000000,0.000000,0.000000}%
\pgfsetstrokecolor{currentstroke}%
\pgfsetdash{}{0pt}%
\pgfsys@defobject{currentmarker}{\pgfqpoint{0.000000in}{-0.048611in}}{\pgfqpoint{0.000000in}{0.000000in}}{%
\pgfpathmoveto{\pgfqpoint{0.000000in}{0.000000in}}%
\pgfpathlineto{\pgfqpoint{0.000000in}{-0.048611in}}%
\pgfusepath{stroke,fill}%
}%
\begin{pgfscope}%
\pgfsys@transformshift{5.698533in}{0.722700in}%
\pgfsys@useobject{currentmarker}{}%
\end{pgfscope}%
\end{pgfscope}%
\begin{pgfscope}%
\pgftext[x=5.698533in,y=0.625478in,,top]{\sffamily\fontsize{10.000000}{12.000000}\selectfont 35}%
\end{pgfscope}%
\begin{pgfscope}%
\pgfsetbuttcap%
\pgfsetroundjoin%
\definecolor{currentfill}{rgb}{0.000000,0.000000,0.000000}%
\pgfsetfillcolor{currentfill}%
\pgfsetlinewidth{0.803000pt}%
\definecolor{currentstroke}{rgb}{0.000000,0.000000,0.000000}%
\pgfsetstrokecolor{currentstroke}%
\pgfsetdash{}{0pt}%
\pgfsys@defobject{currentmarker}{\pgfqpoint{0.000000in}{-0.048611in}}{\pgfqpoint{0.000000in}{0.000000in}}{%
\pgfpathmoveto{\pgfqpoint{0.000000in}{0.000000in}}%
\pgfpathlineto{\pgfqpoint{0.000000in}{-0.048611in}}%
\pgfusepath{stroke,fill}%
}%
\begin{pgfscope}%
\pgfsys@transformshift{6.268681in}{0.722700in}%
\pgfsys@useobject{currentmarker}{}%
\end{pgfscope}%
\end{pgfscope}%
\begin{pgfscope}%
\pgftext[x=6.268681in,y=0.625478in,,top]{\sffamily\fontsize{10.000000}{12.000000}\selectfont 40}%
\end{pgfscope}%
\begin{pgfscope}%
\pgftext[x=4.113523in,y=0.435509in,,top]{\sffamily\fontsize{16.000000}{19.200000}\selectfont \(\displaystyle z-position [\mu m]\)}%
\end{pgfscope}%
\begin{pgfscope}%
\pgfsetbuttcap%
\pgfsetroundjoin%
\definecolor{currentfill}{rgb}{0.000000,0.000000,0.000000}%
\pgfsetfillcolor{currentfill}%
\pgfsetlinewidth{0.803000pt}%
\definecolor{currentstroke}{rgb}{0.000000,0.000000,0.000000}%
\pgfsetstrokecolor{currentstroke}%
\pgfsetdash{}{0pt}%
\pgfsys@defobject{currentmarker}{\pgfqpoint{-0.048611in}{0.000000in}}{\pgfqpoint{0.000000in}{0.000000in}}{%
\pgfpathmoveto{\pgfqpoint{0.000000in}{0.000000in}}%
\pgfpathlineto{\pgfqpoint{-0.048611in}{0.000000in}}%
\pgfusepath{stroke,fill}%
}%
\begin{pgfscope}%
\pgfsys@transformshift{1.707500in}{0.722700in}%
\pgfsys@useobject{currentmarker}{}%
\end{pgfscope}%
\end{pgfscope}%
\begin{pgfscope}%
\pgftext[x=1.273025in,y=0.669938in,left,base]{\sffamily\fontsize{10.000000}{12.000000}\selectfont -2.0}%
\end{pgfscope}%
\begin{pgfscope}%
\pgfsetbuttcap%
\pgfsetroundjoin%
\definecolor{currentfill}{rgb}{0.000000,0.000000,0.000000}%
\pgfsetfillcolor{currentfill}%
\pgfsetlinewidth{0.803000pt}%
\definecolor{currentstroke}{rgb}{0.000000,0.000000,0.000000}%
\pgfsetstrokecolor{currentstroke}%
\pgfsetdash{}{0pt}%
\pgfsys@defobject{currentmarker}{\pgfqpoint{-0.048611in}{0.000000in}}{\pgfqpoint{0.000000in}{0.000000in}}{%
\pgfpathmoveto{\pgfqpoint{0.000000in}{0.000000in}}%
\pgfpathlineto{\pgfqpoint{-0.048611in}{0.000000in}}%
\pgfusepath{stroke,fill}%
}%
\begin{pgfscope}%
\pgfsys@transformshift{1.707500in}{1.355062in}%
\pgfsys@useobject{currentmarker}{}%
\end{pgfscope}%
\end{pgfscope}%
\begin{pgfscope}%
\pgftext[x=1.273025in,y=1.302301in,left,base]{\sffamily\fontsize{10.000000}{12.000000}\selectfont -1.5}%
\end{pgfscope}%
\begin{pgfscope}%
\pgfsetbuttcap%
\pgfsetroundjoin%
\definecolor{currentfill}{rgb}{0.000000,0.000000,0.000000}%
\pgfsetfillcolor{currentfill}%
\pgfsetlinewidth{0.803000pt}%
\definecolor{currentstroke}{rgb}{0.000000,0.000000,0.000000}%
\pgfsetstrokecolor{currentstroke}%
\pgfsetdash{}{0pt}%
\pgfsys@defobject{currentmarker}{\pgfqpoint{-0.048611in}{0.000000in}}{\pgfqpoint{0.000000in}{0.000000in}}{%
\pgfpathmoveto{\pgfqpoint{0.000000in}{0.000000in}}%
\pgfpathlineto{\pgfqpoint{-0.048611in}{0.000000in}}%
\pgfusepath{stroke,fill}%
}%
\begin{pgfscope}%
\pgfsys@transformshift{1.707500in}{1.987425in}%
\pgfsys@useobject{currentmarker}{}%
\end{pgfscope}%
\end{pgfscope}%
\begin{pgfscope}%
\pgftext[x=1.273025in,y=1.934663in,left,base]{\sffamily\fontsize{10.000000}{12.000000}\selectfont -1.0}%
\end{pgfscope}%
\begin{pgfscope}%
\pgfsetbuttcap%
\pgfsetroundjoin%
\definecolor{currentfill}{rgb}{0.000000,0.000000,0.000000}%
\pgfsetfillcolor{currentfill}%
\pgfsetlinewidth{0.803000pt}%
\definecolor{currentstroke}{rgb}{0.000000,0.000000,0.000000}%
\pgfsetstrokecolor{currentstroke}%
\pgfsetdash{}{0pt}%
\pgfsys@defobject{currentmarker}{\pgfqpoint{-0.048611in}{0.000000in}}{\pgfqpoint{0.000000in}{0.000000in}}{%
\pgfpathmoveto{\pgfqpoint{0.000000in}{0.000000in}}%
\pgfpathlineto{\pgfqpoint{-0.048611in}{0.000000in}}%
\pgfusepath{stroke,fill}%
}%
\begin{pgfscope}%
\pgfsys@transformshift{1.707500in}{2.619788in}%
\pgfsys@useobject{currentmarker}{}%
\end{pgfscope}%
\end{pgfscope}%
\begin{pgfscope}%
\pgftext[x=1.273025in,y=2.567026in,left,base]{\sffamily\fontsize{10.000000}{12.000000}\selectfont -0.5}%
\end{pgfscope}%
\begin{pgfscope}%
\pgfsetbuttcap%
\pgfsetroundjoin%
\definecolor{currentfill}{rgb}{0.000000,0.000000,0.000000}%
\pgfsetfillcolor{currentfill}%
\pgfsetlinewidth{0.803000pt}%
\definecolor{currentstroke}{rgb}{0.000000,0.000000,0.000000}%
\pgfsetstrokecolor{currentstroke}%
\pgfsetdash{}{0pt}%
\pgfsys@defobject{currentmarker}{\pgfqpoint{-0.048611in}{0.000000in}}{\pgfqpoint{0.000000in}{0.000000in}}{%
\pgfpathmoveto{\pgfqpoint{0.000000in}{0.000000in}}%
\pgfpathlineto{\pgfqpoint{-0.048611in}{0.000000in}}%
\pgfusepath{stroke,fill}%
}%
\begin{pgfscope}%
\pgfsys@transformshift{1.707500in}{3.252150in}%
\pgfsys@useobject{currentmarker}{}%
\end{pgfscope}%
\end{pgfscope}%
\begin{pgfscope}%
\pgftext[x=1.389398in,y=3.199388in,left,base]{\sffamily\fontsize{10.000000}{12.000000}\selectfont 0.0}%
\end{pgfscope}%
\begin{pgfscope}%
\pgfsetbuttcap%
\pgfsetroundjoin%
\definecolor{currentfill}{rgb}{0.000000,0.000000,0.000000}%
\pgfsetfillcolor{currentfill}%
\pgfsetlinewidth{0.803000pt}%
\definecolor{currentstroke}{rgb}{0.000000,0.000000,0.000000}%
\pgfsetstrokecolor{currentstroke}%
\pgfsetdash{}{0pt}%
\pgfsys@defobject{currentmarker}{\pgfqpoint{-0.048611in}{0.000000in}}{\pgfqpoint{0.000000in}{0.000000in}}{%
\pgfpathmoveto{\pgfqpoint{0.000000in}{0.000000in}}%
\pgfpathlineto{\pgfqpoint{-0.048611in}{0.000000in}}%
\pgfusepath{stroke,fill}%
}%
\begin{pgfscope}%
\pgfsys@transformshift{1.707500in}{3.884512in}%
\pgfsys@useobject{currentmarker}{}%
\end{pgfscope}%
\end{pgfscope}%
\begin{pgfscope}%
\pgftext[x=1.389398in,y=3.831751in,left,base]{\sffamily\fontsize{10.000000}{12.000000}\selectfont 0.5}%
\end{pgfscope}%
\begin{pgfscope}%
\pgfsetbuttcap%
\pgfsetroundjoin%
\definecolor{currentfill}{rgb}{0.000000,0.000000,0.000000}%
\pgfsetfillcolor{currentfill}%
\pgfsetlinewidth{0.803000pt}%
\definecolor{currentstroke}{rgb}{0.000000,0.000000,0.000000}%
\pgfsetstrokecolor{currentstroke}%
\pgfsetdash{}{0pt}%
\pgfsys@defobject{currentmarker}{\pgfqpoint{-0.048611in}{0.000000in}}{\pgfqpoint{0.000000in}{0.000000in}}{%
\pgfpathmoveto{\pgfqpoint{0.000000in}{0.000000in}}%
\pgfpathlineto{\pgfqpoint{-0.048611in}{0.000000in}}%
\pgfusepath{stroke,fill}%
}%
\begin{pgfscope}%
\pgfsys@transformshift{1.707500in}{4.516875in}%
\pgfsys@useobject{currentmarker}{}%
\end{pgfscope}%
\end{pgfscope}%
\begin{pgfscope}%
\pgftext[x=1.389398in,y=4.464113in,left,base]{\sffamily\fontsize{10.000000}{12.000000}\selectfont 1.0}%
\end{pgfscope}%
\begin{pgfscope}%
\pgfsetbuttcap%
\pgfsetroundjoin%
\definecolor{currentfill}{rgb}{0.000000,0.000000,0.000000}%
\pgfsetfillcolor{currentfill}%
\pgfsetlinewidth{0.803000pt}%
\definecolor{currentstroke}{rgb}{0.000000,0.000000,0.000000}%
\pgfsetstrokecolor{currentstroke}%
\pgfsetdash{}{0pt}%
\pgfsys@defobject{currentmarker}{\pgfqpoint{-0.048611in}{0.000000in}}{\pgfqpoint{0.000000in}{0.000000in}}{%
\pgfpathmoveto{\pgfqpoint{0.000000in}{0.000000in}}%
\pgfpathlineto{\pgfqpoint{-0.048611in}{0.000000in}}%
\pgfusepath{stroke,fill}%
}%
\begin{pgfscope}%
\pgfsys@transformshift{1.707500in}{5.149237in}%
\pgfsys@useobject{currentmarker}{}%
\end{pgfscope}%
\end{pgfscope}%
\begin{pgfscope}%
\pgftext[x=1.389398in,y=5.096476in,left,base]{\sffamily\fontsize{10.000000}{12.000000}\selectfont 1.5}%
\end{pgfscope}%
\begin{pgfscope}%
\pgfsetbuttcap%
\pgfsetroundjoin%
\definecolor{currentfill}{rgb}{0.000000,0.000000,0.000000}%
\pgfsetfillcolor{currentfill}%
\pgfsetlinewidth{0.803000pt}%
\definecolor{currentstroke}{rgb}{0.000000,0.000000,0.000000}%
\pgfsetstrokecolor{currentstroke}%
\pgfsetdash{}{0pt}%
\pgfsys@defobject{currentmarker}{\pgfqpoint{-0.048611in}{0.000000in}}{\pgfqpoint{0.000000in}{0.000000in}}{%
\pgfpathmoveto{\pgfqpoint{0.000000in}{0.000000in}}%
\pgfpathlineto{\pgfqpoint{-0.048611in}{0.000000in}}%
\pgfusepath{stroke,fill}%
}%
\begin{pgfscope}%
\pgfsys@transformshift{1.707500in}{5.781600in}%
\pgfsys@useobject{currentmarker}{}%
\end{pgfscope}%
\end{pgfscope}%
\begin{pgfscope}%
\pgftext[x=1.389398in,y=5.728838in,left,base]{\sffamily\fontsize{10.000000}{12.000000}\selectfont 2.0}%
\end{pgfscope}%
\begin{pgfscope}%
\pgftext[x=1.217469in,y=3.252150in,,bottom,rotate=90.000000]{\sffamily\fontsize{16.000000}{19.200000}\selectfont \(\displaystyle E-field\)}%
\end{pgfscope}%
\begin{pgfscope}%
\pgfpathrectangle{\pgfqpoint{1.707500in}{0.722700in}}{\pgfqpoint{4.812045in}{5.058900in}} %
\pgfusepath{clip}%
\pgfsetrectcap%
\pgfsetroundjoin%
\pgfsetlinewidth{1.505625pt}%
\definecolor{currentstroke}{rgb}{0.000000,0.000000,0.000000}%
\pgfsetstrokecolor{currentstroke}%
\pgfsetdash{}{0pt}%
\pgfpathmoveto{\pgfqpoint{1.707500in}{3.252150in}}%
\pgfpathlineto{\pgfqpoint{4.852486in}{3.251112in}}%
\pgfpathlineto{\pgfqpoint{4.857925in}{3.247322in}}%
\pgfpathlineto{\pgfqpoint{4.864202in}{3.239927in}}%
\pgfpathlineto{\pgfqpoint{4.873407in}{3.228921in}}%
\pgfpathlineto{\pgfqpoint{4.876337in}{3.228357in}}%
\pgfpathlineto{\pgfqpoint{4.878847in}{3.230282in}}%
\pgfpathlineto{\pgfqpoint{4.881776in}{3.236205in}}%
\pgfpathlineto{\pgfqpoint{4.885124in}{3.248777in}}%
\pgfpathlineto{\pgfqpoint{4.889308in}{3.273939in}}%
\pgfpathlineto{\pgfqpoint{4.895585in}{3.328108in}}%
\pgfpathlineto{\pgfqpoint{4.904790in}{3.406002in}}%
\pgfpathlineto{\pgfqpoint{4.907719in}{3.416624in}}%
\pgfpathlineto{\pgfqpoint{4.909393in}{3.417133in}}%
\pgfpathlineto{\pgfqpoint{4.911067in}{3.412883in}}%
\pgfpathlineto{\pgfqpoint{4.913159in}{3.400182in}}%
\pgfpathlineto{\pgfqpoint{4.916088in}{3.367582in}}%
\pgfpathlineto{\pgfqpoint{4.920273in}{3.290722in}}%
\pgfpathlineto{\pgfqpoint{4.926131in}{3.134000in}}%
\pgfpathlineto{\pgfqpoint{4.937429in}{2.823127in}}%
\pgfpathlineto{\pgfqpoint{4.940776in}{2.779500in}}%
\pgfpathlineto{\pgfqpoint{4.942868in}{2.772595in}}%
\pgfpathlineto{\pgfqpoint{4.944124in}{2.776906in}}%
\pgfpathlineto{\pgfqpoint{4.946216in}{2.798998in}}%
\pgfpathlineto{\pgfqpoint{4.949145in}{2.861979in}}%
\pgfpathlineto{\pgfqpoint{4.953329in}{3.014032in}}%
\pgfpathlineto{\pgfqpoint{4.959606in}{3.346083in}}%
\pgfpathlineto{\pgfqpoint{4.970904in}{3.952558in}}%
\pgfpathlineto{\pgfqpoint{4.974670in}{4.056865in}}%
\pgfpathlineto{\pgfqpoint{4.977180in}{4.080278in}}%
\pgfpathlineto{\pgfqpoint{4.977599in}{4.080257in}}%
\pgfpathlineto{\pgfqpoint{4.978436in}{4.076756in}}%
\pgfpathlineto{\pgfqpoint{4.980109in}{4.055757in}}%
\pgfpathlineto{\pgfqpoint{4.982620in}{3.989297in}}%
\pgfpathlineto{\pgfqpoint{4.986386in}{3.815052in}}%
\pgfpathlineto{\pgfqpoint{4.991826in}{3.436450in}}%
\pgfpathlineto{\pgfqpoint{5.006889in}{2.316276in}}%
\pgfpathlineto{\pgfqpoint{5.010655in}{2.198770in}}%
\pgfpathlineto{\pgfqpoint{5.012748in}{2.181266in}}%
\pgfpathlineto{\pgfqpoint{5.013584in}{2.184326in}}%
\pgfpathlineto{\pgfqpoint{5.015258in}{2.207857in}}%
\pgfpathlineto{\pgfqpoint{5.017769in}{2.286058in}}%
\pgfpathlineto{\pgfqpoint{5.021535in}{2.493146in}}%
\pgfpathlineto{\pgfqpoint{5.026974in}{2.941496in}}%
\pgfpathlineto{\pgfqpoint{5.042038in}{4.265311in}}%
\pgfpathlineto{\pgfqpoint{5.045804in}{4.414647in}}%
\pgfpathlineto{\pgfqpoint{5.048315in}{4.446206in}}%
\pgfpathlineto{\pgfqpoint{5.048733in}{4.445922in}}%
\pgfpathlineto{\pgfqpoint{5.049989in}{4.435494in}}%
\pgfpathlineto{\pgfqpoint{5.052081in}{4.386476in}}%
\pgfpathlineto{\pgfqpoint{5.055010in}{4.254195in}}%
\pgfpathlineto{\pgfqpoint{5.059194in}{3.952616in}}%
\pgfpathlineto{\pgfqpoint{5.066308in}{3.246572in}}%
\pgfpathlineto{\pgfqpoint{5.075932in}{2.329659in}}%
\pgfpathlineto{\pgfqpoint{5.080535in}{2.077640in}}%
\pgfpathlineto{\pgfqpoint{5.083464in}{2.012441in}}%
\pgfpathlineto{\pgfqpoint{5.084301in}{2.008490in}}%
\pgfpathlineto{\pgfqpoint{5.084719in}{2.008994in}}%
\pgfpathlineto{\pgfqpoint{5.085974in}{2.020422in}}%
\pgfpathlineto{\pgfqpoint{5.088067in}{2.072126in}}%
\pgfpathlineto{\pgfqpoint{5.090996in}{2.209897in}}%
\pgfpathlineto{\pgfqpoint{5.095180in}{2.521599in}}%
\pgfpathlineto{\pgfqpoint{5.102712in}{3.292120in}}%
\pgfpathlineto{\pgfqpoint{5.111918in}{4.182898in}}%
\pgfpathlineto{\pgfqpoint{5.116520in}{4.440254in}}%
\pgfpathlineto{\pgfqpoint{5.119449in}{4.507507in}}%
\pgfpathlineto{\pgfqpoint{5.120286in}{4.511869in}}%
\pgfpathlineto{\pgfqpoint{5.120705in}{4.511542in}}%
\pgfpathlineto{\pgfqpoint{5.121960in}{4.500540in}}%
\pgfpathlineto{\pgfqpoint{5.124052in}{4.449208in}}%
\pgfpathlineto{\pgfqpoint{5.126981in}{4.311326in}}%
\pgfpathlineto{\pgfqpoint{5.131166in}{3.998404in}}%
\pgfpathlineto{\pgfqpoint{5.138279in}{3.269395in}}%
\pgfpathlineto{\pgfqpoint{5.147903in}{2.324888in}}%
\pgfpathlineto{\pgfqpoint{5.152506in}{2.063358in}}%
\pgfpathlineto{\pgfqpoint{5.155435in}{1.993539in}}%
\pgfpathlineto{\pgfqpoint{5.156690in}{1.988454in}}%
\pgfpathlineto{\pgfqpoint{5.157527in}{1.993443in}}%
\pgfpathlineto{\pgfqpoint{5.159201in}{2.023398in}}%
\pgfpathlineto{\pgfqpoint{5.161712in}{2.116819in}}%
\pgfpathlineto{\pgfqpoint{5.165478in}{2.356172in}}%
\pgfpathlineto{\pgfqpoint{5.171336in}{2.905485in}}%
\pgfpathlineto{\pgfqpoint{5.185144in}{4.261514in}}%
\pgfpathlineto{\pgfqpoint{5.189329in}{4.466717in}}%
\pgfpathlineto{\pgfqpoint{5.191839in}{4.514219in}}%
\pgfpathlineto{\pgfqpoint{5.192676in}{4.516740in}}%
\pgfpathlineto{\pgfqpoint{5.193095in}{4.515485in}}%
\pgfpathlineto{\pgfqpoint{5.194350in}{4.501679in}}%
\pgfpathlineto{\pgfqpoint{5.196442in}{4.445708in}}%
\pgfpathlineto{\pgfqpoint{5.199790in}{4.275295in}}%
\pgfpathlineto{\pgfqpoint{5.204393in}{3.904337in}}%
\pgfpathlineto{\pgfqpoint{5.214017in}{2.883302in}}%
\pgfpathlineto{\pgfqpoint{5.221130in}{2.249659in}}%
\pgfpathlineto{\pgfqpoint{5.225314in}{2.040745in}}%
\pgfpathlineto{\pgfqpoint{5.227825in}{1.990783in}}%
\pgfpathlineto{\pgfqpoint{5.228662in}{1.987428in}}%
\pgfpathlineto{\pgfqpoint{5.229080in}{1.988267in}}%
\pgfpathlineto{\pgfqpoint{5.230336in}{2.000827in}}%
\pgfpathlineto{\pgfqpoint{5.232428in}{2.054777in}}%
\pgfpathlineto{\pgfqpoint{5.235357in}{2.196193in}}%
\pgfpathlineto{\pgfqpoint{5.239960in}{2.551303in}}%
\pgfpathlineto{\pgfqpoint{5.248329in}{3.430066in}}%
\pgfpathlineto{\pgfqpoint{5.256279in}{4.188161in}}%
\pgfpathlineto{\pgfqpoint{5.260882in}{4.445767in}}%
\pgfpathlineto{\pgfqpoint{5.263811in}{4.512630in}}%
\pgfpathlineto{\pgfqpoint{5.264648in}{4.516823in}}%
\pgfpathlineto{\pgfqpoint{5.265066in}{4.516403in}}%
\pgfpathlineto{\pgfqpoint{5.266321in}{4.505095in}}%
\pgfpathlineto{\pgfqpoint{5.268414in}{4.453183in}}%
\pgfpathlineto{\pgfqpoint{5.271343in}{4.314403in}}%
\pgfpathlineto{\pgfqpoint{5.275527in}{4.000188in}}%
\pgfpathlineto{\pgfqpoint{5.283059in}{3.223361in}}%
\pgfpathlineto{\pgfqpoint{5.292265in}{2.323923in}}%
\pgfpathlineto{\pgfqpoint{5.296867in}{2.062390in}}%
\pgfpathlineto{\pgfqpoint{5.299797in}{1.992665in}}%
\pgfpathlineto{\pgfqpoint{5.301052in}{1.987635in}}%
\pgfpathlineto{\pgfqpoint{5.301889in}{1.992665in}}%
\pgfpathlineto{\pgfqpoint{5.303562in}{2.022706in}}%
\pgfpathlineto{\pgfqpoint{5.306073in}{2.116265in}}%
\pgfpathlineto{\pgfqpoint{5.309839in}{2.355821in}}%
\pgfpathlineto{\pgfqpoint{5.315697in}{2.905390in}}%
\pgfpathlineto{\pgfqpoint{5.329506in}{4.261626in}}%
\pgfpathlineto{\pgfqpoint{5.333690in}{4.466818in}}%
\pgfpathlineto{\pgfqpoint{5.336201in}{4.514307in}}%
\pgfpathlineto{\pgfqpoint{5.337038in}{4.516823in}}%
\pgfpathlineto{\pgfqpoint{5.337456in}{4.515564in}}%
\pgfpathlineto{\pgfqpoint{5.338711in}{4.501751in}}%
\pgfpathlineto{\pgfqpoint{5.340804in}{4.445767in}}%
\pgfpathlineto{\pgfqpoint{5.344151in}{4.275334in}}%
\pgfpathlineto{\pgfqpoint{5.348754in}{3.904354in}}%
\pgfpathlineto{\pgfqpoint{5.358378in}{2.883298in}}%
\pgfpathlineto{\pgfqpoint{5.365491in}{2.249654in}}%
\pgfpathlineto{\pgfqpoint{5.369676in}{2.040742in}}%
\pgfpathlineto{\pgfqpoint{5.372186in}{1.990782in}}%
\pgfpathlineto{\pgfqpoint{5.373023in}{1.987428in}}%
\pgfpathlineto{\pgfqpoint{5.373442in}{1.988267in}}%
\pgfpathlineto{\pgfqpoint{5.374697in}{2.000829in}}%
\pgfpathlineto{\pgfqpoint{5.376789in}{2.054780in}}%
\pgfpathlineto{\pgfqpoint{5.379718in}{2.196197in}}%
\pgfpathlineto{\pgfqpoint{5.384321in}{2.551309in}}%
\pgfpathlineto{\pgfqpoint{5.392690in}{3.430063in}}%
\pgfpathlineto{\pgfqpoint{5.400640in}{4.188129in}}%
\pgfpathlineto{\pgfqpoint{5.405243in}{4.445708in}}%
\pgfpathlineto{\pgfqpoint{5.408172in}{4.512553in}}%
\pgfpathlineto{\pgfqpoint{5.409009in}{4.516740in}}%
\pgfpathlineto{\pgfqpoint{5.409427in}{4.516318in}}%
\pgfpathlineto{\pgfqpoint{5.410683in}{4.505003in}}%
\pgfpathlineto{\pgfqpoint{5.412775in}{4.453080in}}%
\pgfpathlineto{\pgfqpoint{5.415704in}{4.314291in}}%
\pgfpathlineto{\pgfqpoint{5.419888in}{4.000083in}}%
\pgfpathlineto{\pgfqpoint{5.427420in}{3.223367in}}%
\pgfpathlineto{\pgfqpoint{5.436626in}{2.324295in}}%
\pgfpathlineto{\pgfqpoint{5.441229in}{2.063013in}}%
\pgfpathlineto{\pgfqpoint{5.444158in}{1.993443in}}%
\pgfpathlineto{\pgfqpoint{5.445413in}{1.988473in}}%
\pgfpathlineto{\pgfqpoint{5.446250in}{1.993539in}}%
\pgfpathlineto{\pgfqpoint{5.447924in}{2.023641in}}%
\pgfpathlineto{\pgfqpoint{5.450434in}{2.117255in}}%
\pgfpathlineto{\pgfqpoint{5.454200in}{2.356773in}}%
\pgfpathlineto{\pgfqpoint{5.460059in}{2.905887in}}%
\pgfpathlineto{\pgfqpoint{5.473867in}{4.258813in}}%
\pgfpathlineto{\pgfqpoint{5.478051in}{4.462721in}}%
\pgfpathlineto{\pgfqpoint{5.480562in}{4.509543in}}%
\pgfpathlineto{\pgfqpoint{5.481399in}{4.511869in}}%
\pgfpathlineto{\pgfqpoint{5.481817in}{4.510524in}}%
\pgfpathlineto{\pgfqpoint{5.483073in}{4.496483in}}%
\pgfpathlineto{\pgfqpoint{5.485165in}{4.440254in}}%
\pgfpathlineto{\pgfqpoint{5.488512in}{4.269879in}}%
\pgfpathlineto{\pgfqpoint{5.493115in}{3.900135in}}%
\pgfpathlineto{\pgfqpoint{5.503158in}{2.843493in}}%
\pgfpathlineto{\pgfqpoint{5.509853in}{2.262258in}}%
\pgfpathlineto{\pgfqpoint{5.514037in}{2.058571in}}%
\pgfpathlineto{\pgfqpoint{5.516548in}{2.011153in}}%
\pgfpathlineto{\pgfqpoint{5.517385in}{2.008490in}}%
\pgfpathlineto{\pgfqpoint{5.517803in}{2.009640in}}%
\pgfpathlineto{\pgfqpoint{5.519058in}{2.022979in}}%
\pgfpathlineto{\pgfqpoint{5.521151in}{2.077640in}}%
\pgfpathlineto{\pgfqpoint{5.524498in}{2.244400in}}%
\pgfpathlineto{\pgfqpoint{5.529101in}{2.606841in}}%
\pgfpathlineto{\pgfqpoint{5.539562in}{3.681296in}}%
\pgfpathlineto{\pgfqpoint{5.546257in}{4.229611in}}%
\pgfpathlineto{\pgfqpoint{5.550441in}{4.410778in}}%
\pgfpathlineto{\pgfqpoint{5.552952in}{4.445922in}}%
\pgfpathlineto{\pgfqpoint{5.553370in}{4.446206in}}%
\pgfpathlineto{\pgfqpoint{5.554207in}{4.441995in}}%
\pgfpathlineto{\pgfqpoint{5.555881in}{4.414647in}}%
\pgfpathlineto{\pgfqpoint{5.558392in}{4.327955in}}%
\pgfpathlineto{\pgfqpoint{5.562158in}{4.105552in}}%
\pgfpathlineto{\pgfqpoint{5.568016in}{3.599239in}}%
\pgfpathlineto{\pgfqpoint{5.580987in}{2.438548in}}%
\pgfpathlineto{\pgfqpoint{5.585172in}{2.240624in}}%
\pgfpathlineto{\pgfqpoint{5.588101in}{2.184326in}}%
\pgfpathlineto{\pgfqpoint{5.588938in}{2.181266in}}%
\pgfpathlineto{\pgfqpoint{5.589356in}{2.181906in}}%
\pgfpathlineto{\pgfqpoint{5.590611in}{2.192429in}}%
\pgfpathlineto{\pgfqpoint{5.592704in}{2.237911in}}%
\pgfpathlineto{\pgfqpoint{5.596051in}{2.377722in}}%
\pgfpathlineto{\pgfqpoint{5.601072in}{2.710278in}}%
\pgfpathlineto{\pgfqpoint{5.618228in}{3.958029in}}%
\pgfpathlineto{\pgfqpoint{5.621994in}{4.062763in}}%
\pgfpathlineto{\pgfqpoint{5.624087in}{4.080257in}}%
\pgfpathlineto{\pgfqpoint{5.624505in}{4.080278in}}%
\pgfpathlineto{\pgfqpoint{5.625342in}{4.076904in}}%
\pgfpathlineto{\pgfqpoint{5.627016in}{4.056865in}}%
\pgfpathlineto{\pgfqpoint{5.629526in}{3.995738in}}%
\pgfpathlineto{\pgfqpoint{5.633292in}{3.844894in}}%
\pgfpathlineto{\pgfqpoint{5.640406in}{3.445778in}}%
\pgfpathlineto{\pgfqpoint{5.649193in}{2.978222in}}%
\pgfpathlineto{\pgfqpoint{5.653796in}{2.830412in}}%
\pgfpathlineto{\pgfqpoint{5.657143in}{2.779815in}}%
\pgfpathlineto{\pgfqpoint{5.659235in}{2.772598in}}%
\pgfpathlineto{\pgfqpoint{5.660491in}{2.776769in}}%
\pgfpathlineto{\pgfqpoint{5.662583in}{2.796715in}}%
\pgfpathlineto{\pgfqpoint{5.665930in}{2.857442in}}%
\pgfpathlineto{\pgfqpoint{5.671370in}{3.005864in}}%
\pgfpathlineto{\pgfqpoint{5.682668in}{3.317379in}}%
\pgfpathlineto{\pgfqpoint{5.687271in}{3.388374in}}%
\pgfpathlineto{\pgfqpoint{5.690618in}{3.412883in}}%
\pgfpathlineto{\pgfqpoint{5.692711in}{3.417432in}}%
\pgfpathlineto{\pgfqpoint{5.693966in}{3.416624in}}%
\pgfpathlineto{\pgfqpoint{5.695640in}{3.411915in}}%
\pgfpathlineto{\pgfqpoint{5.698569in}{3.395480in}}%
\pgfpathlineto{\pgfqpoint{5.704008in}{3.348270in}}%
\pgfpathlineto{\pgfqpoint{5.713632in}{3.265302in}}%
\pgfpathlineto{\pgfqpoint{5.718654in}{3.240151in}}%
\pgfpathlineto{\pgfqpoint{5.722420in}{3.230867in}}%
\pgfpathlineto{\pgfqpoint{5.725349in}{3.228357in}}%
\pgfpathlineto{\pgfqpoint{5.727859in}{3.228691in}}%
\pgfpathlineto{\pgfqpoint{5.731625in}{3.232024in}}%
\pgfpathlineto{\pgfqpoint{5.748363in}{3.250690in}}%
\pgfpathlineto{\pgfqpoint{5.754639in}{3.252713in}}%
\pgfpathlineto{\pgfqpoint{5.763845in}{3.252857in}}%
\pgfpathlineto{\pgfqpoint{5.797320in}{3.252148in}}%
\pgfpathlineto{\pgfqpoint{6.519545in}{3.252150in}}%
\pgfpathlineto{\pgfqpoint{6.519545in}{3.252150in}}%
\pgfusepath{stroke}%
\end{pgfscope}%
\begin{pgfscope}%
\pgfsetrectcap%
\pgfsetmiterjoin%
\pgfsetlinewidth{0.803000pt}%
\definecolor{currentstroke}{rgb}{0.000000,0.000000,0.000000}%
\pgfsetstrokecolor{currentstroke}%
\pgfsetdash{}{0pt}%
\pgfpathmoveto{\pgfqpoint{1.707500in}{0.722700in}}%
\pgfpathlineto{\pgfqpoint{1.707500in}{5.781600in}}%
\pgfusepath{stroke}%
\end{pgfscope}%
\begin{pgfscope}%
\pgfsetrectcap%
\pgfsetmiterjoin%
\pgfsetlinewidth{0.803000pt}%
\definecolor{currentstroke}{rgb}{0.000000,0.000000,0.000000}%
\pgfsetstrokecolor{currentstroke}%
\pgfsetdash{}{0pt}%
\pgfpathmoveto{\pgfqpoint{6.519545in}{0.722700in}}%
\pgfpathlineto{\pgfqpoint{6.519545in}{5.781600in}}%
\pgfusepath{stroke}%
\end{pgfscope}%
\begin{pgfscope}%
\pgfsetrectcap%
\pgfsetmiterjoin%
\pgfsetlinewidth{0.803000pt}%
\definecolor{currentstroke}{rgb}{0.000000,0.000000,0.000000}%
\pgfsetstrokecolor{currentstroke}%
\pgfsetdash{}{0pt}%
\pgfpathmoveto{\pgfqpoint{1.707500in}{0.722700in}}%
\pgfpathlineto{\pgfqpoint{6.519545in}{0.722700in}}%
\pgfusepath{stroke}%
\end{pgfscope}%
\begin{pgfscope}%
\pgfsetrectcap%
\pgfsetmiterjoin%
\pgfsetlinewidth{0.803000pt}%
\definecolor{currentstroke}{rgb}{0.000000,0.000000,0.000000}%
\pgfsetstrokecolor{currentstroke}%
\pgfsetdash{}{0pt}%
\pgfpathmoveto{\pgfqpoint{1.707500in}{5.781600in}}%
\pgfpathlineto{\pgfqpoint{6.519545in}{5.781600in}}%
\pgfusepath{stroke}%
\end{pgfscope}%
\begin{pgfscope}%
\pgftext[x=1.707500in,y=6.034545in,left,base]{\sffamily\fontsize{10.000000}{12.000000}\selectfont Iterations: 16915, Time: 0.207 ps, imp: 377 ohm}%
\end{pgfscope}%
\begin{pgfscope}%
\pgfsetbuttcap%
\pgfsetmiterjoin%
\definecolor{currentfill}{rgb}{1.000000,1.000000,1.000000}%
\pgfsetfillcolor{currentfill}%
\pgfsetlinewidth{0.000000pt}%
\definecolor{currentstroke}{rgb}{0.000000,0.000000,0.000000}%
\pgfsetstrokecolor{currentstroke}%
\pgfsetstrokeopacity{0.000000}%
\pgfsetdash{}{0pt}%
\pgfpathmoveto{\pgfqpoint{7.481955in}{0.722700in}}%
\pgfpathlineto{\pgfqpoint{12.294000in}{0.722700in}}%
\pgfpathlineto{\pgfqpoint{12.294000in}{5.781600in}}%
\pgfpathlineto{\pgfqpoint{7.481955in}{5.781600in}}%
\pgfpathclose%
\pgfusepath{fill}%
\end{pgfscope}%
\begin{pgfscope}%
\pgfsetbuttcap%
\pgfsetroundjoin%
\definecolor{currentfill}{rgb}{0.000000,0.000000,0.000000}%
\pgfsetfillcolor{currentfill}%
\pgfsetlinewidth{0.803000pt}%
\definecolor{currentstroke}{rgb}{0.000000,0.000000,0.000000}%
\pgfsetstrokecolor{currentstroke}%
\pgfsetdash{}{0pt}%
\pgfsys@defobject{currentmarker}{\pgfqpoint{0.000000in}{-0.048611in}}{\pgfqpoint{0.000000in}{0.000000in}}{%
\pgfpathmoveto{\pgfqpoint{0.000000in}{0.000000in}}%
\pgfpathlineto{\pgfqpoint{0.000000in}{-0.048611in}}%
\pgfusepath{stroke,fill}%
}%
\begin{pgfscope}%
\pgfsys@transformshift{7.481955in}{0.722700in}%
\pgfsys@useobject{currentmarker}{}%
\end{pgfscope}%
\end{pgfscope}%
\begin{pgfscope}%
\pgftext[x=7.481955in,y=0.625478in,,top]{\sffamily\fontsize{10.000000}{12.000000}\selectfont 0}%
\end{pgfscope}%
\begin{pgfscope}%
\pgfsetbuttcap%
\pgfsetroundjoin%
\definecolor{currentfill}{rgb}{0.000000,0.000000,0.000000}%
\pgfsetfillcolor{currentfill}%
\pgfsetlinewidth{0.803000pt}%
\definecolor{currentstroke}{rgb}{0.000000,0.000000,0.000000}%
\pgfsetstrokecolor{currentstroke}%
\pgfsetdash{}{0pt}%
\pgfsys@defobject{currentmarker}{\pgfqpoint{0.000000in}{-0.048611in}}{\pgfqpoint{0.000000in}{0.000000in}}{%
\pgfpathmoveto{\pgfqpoint{0.000000in}{0.000000in}}%
\pgfpathlineto{\pgfqpoint{0.000000in}{-0.048611in}}%
\pgfusepath{stroke,fill}%
}%
\begin{pgfscope}%
\pgfsys@transformshift{8.052102in}{0.722700in}%
\pgfsys@useobject{currentmarker}{}%
\end{pgfscope}%
\end{pgfscope}%
\begin{pgfscope}%
\pgftext[x=8.052102in,y=0.625478in,,top]{\sffamily\fontsize{10.000000}{12.000000}\selectfont 5}%
\end{pgfscope}%
\begin{pgfscope}%
\pgfsetbuttcap%
\pgfsetroundjoin%
\definecolor{currentfill}{rgb}{0.000000,0.000000,0.000000}%
\pgfsetfillcolor{currentfill}%
\pgfsetlinewidth{0.803000pt}%
\definecolor{currentstroke}{rgb}{0.000000,0.000000,0.000000}%
\pgfsetstrokecolor{currentstroke}%
\pgfsetdash{}{0pt}%
\pgfsys@defobject{currentmarker}{\pgfqpoint{0.000000in}{-0.048611in}}{\pgfqpoint{0.000000in}{0.000000in}}{%
\pgfpathmoveto{\pgfqpoint{0.000000in}{0.000000in}}%
\pgfpathlineto{\pgfqpoint{0.000000in}{-0.048611in}}%
\pgfusepath{stroke,fill}%
}%
\begin{pgfscope}%
\pgfsys@transformshift{8.622250in}{0.722700in}%
\pgfsys@useobject{currentmarker}{}%
\end{pgfscope}%
\end{pgfscope}%
\begin{pgfscope}%
\pgftext[x=8.622250in,y=0.625478in,,top]{\sffamily\fontsize{10.000000}{12.000000}\selectfont 10}%
\end{pgfscope}%
\begin{pgfscope}%
\pgfsetbuttcap%
\pgfsetroundjoin%
\definecolor{currentfill}{rgb}{0.000000,0.000000,0.000000}%
\pgfsetfillcolor{currentfill}%
\pgfsetlinewidth{0.803000pt}%
\definecolor{currentstroke}{rgb}{0.000000,0.000000,0.000000}%
\pgfsetstrokecolor{currentstroke}%
\pgfsetdash{}{0pt}%
\pgfsys@defobject{currentmarker}{\pgfqpoint{0.000000in}{-0.048611in}}{\pgfqpoint{0.000000in}{0.000000in}}{%
\pgfpathmoveto{\pgfqpoint{0.000000in}{0.000000in}}%
\pgfpathlineto{\pgfqpoint{0.000000in}{-0.048611in}}%
\pgfusepath{stroke,fill}%
}%
\begin{pgfscope}%
\pgfsys@transformshift{9.192397in}{0.722700in}%
\pgfsys@useobject{currentmarker}{}%
\end{pgfscope}%
\end{pgfscope}%
\begin{pgfscope}%
\pgftext[x=9.192397in,y=0.625478in,,top]{\sffamily\fontsize{10.000000}{12.000000}\selectfont 15}%
\end{pgfscope}%
\begin{pgfscope}%
\pgfsetbuttcap%
\pgfsetroundjoin%
\definecolor{currentfill}{rgb}{0.000000,0.000000,0.000000}%
\pgfsetfillcolor{currentfill}%
\pgfsetlinewidth{0.803000pt}%
\definecolor{currentstroke}{rgb}{0.000000,0.000000,0.000000}%
\pgfsetstrokecolor{currentstroke}%
\pgfsetdash{}{0pt}%
\pgfsys@defobject{currentmarker}{\pgfqpoint{0.000000in}{-0.048611in}}{\pgfqpoint{0.000000in}{0.000000in}}{%
\pgfpathmoveto{\pgfqpoint{0.000000in}{0.000000in}}%
\pgfpathlineto{\pgfqpoint{0.000000in}{-0.048611in}}%
\pgfusepath{stroke,fill}%
}%
\begin{pgfscope}%
\pgfsys@transformshift{9.762545in}{0.722700in}%
\pgfsys@useobject{currentmarker}{}%
\end{pgfscope}%
\end{pgfscope}%
\begin{pgfscope}%
\pgftext[x=9.762545in,y=0.625478in,,top]{\sffamily\fontsize{10.000000}{12.000000}\selectfont 20}%
\end{pgfscope}%
\begin{pgfscope}%
\pgfsetbuttcap%
\pgfsetroundjoin%
\definecolor{currentfill}{rgb}{0.000000,0.000000,0.000000}%
\pgfsetfillcolor{currentfill}%
\pgfsetlinewidth{0.803000pt}%
\definecolor{currentstroke}{rgb}{0.000000,0.000000,0.000000}%
\pgfsetstrokecolor{currentstroke}%
\pgfsetdash{}{0pt}%
\pgfsys@defobject{currentmarker}{\pgfqpoint{0.000000in}{-0.048611in}}{\pgfqpoint{0.000000in}{0.000000in}}{%
\pgfpathmoveto{\pgfqpoint{0.000000in}{0.000000in}}%
\pgfpathlineto{\pgfqpoint{0.000000in}{-0.048611in}}%
\pgfusepath{stroke,fill}%
}%
\begin{pgfscope}%
\pgfsys@transformshift{10.332692in}{0.722700in}%
\pgfsys@useobject{currentmarker}{}%
\end{pgfscope}%
\end{pgfscope}%
\begin{pgfscope}%
\pgftext[x=10.332692in,y=0.625478in,,top]{\sffamily\fontsize{10.000000}{12.000000}\selectfont 25}%
\end{pgfscope}%
\begin{pgfscope}%
\pgfsetbuttcap%
\pgfsetroundjoin%
\definecolor{currentfill}{rgb}{0.000000,0.000000,0.000000}%
\pgfsetfillcolor{currentfill}%
\pgfsetlinewidth{0.803000pt}%
\definecolor{currentstroke}{rgb}{0.000000,0.000000,0.000000}%
\pgfsetstrokecolor{currentstroke}%
\pgfsetdash{}{0pt}%
\pgfsys@defobject{currentmarker}{\pgfqpoint{0.000000in}{-0.048611in}}{\pgfqpoint{0.000000in}{0.000000in}}{%
\pgfpathmoveto{\pgfqpoint{0.000000in}{0.000000in}}%
\pgfpathlineto{\pgfqpoint{0.000000in}{-0.048611in}}%
\pgfusepath{stroke,fill}%
}%
\begin{pgfscope}%
\pgfsys@transformshift{10.902840in}{0.722700in}%
\pgfsys@useobject{currentmarker}{}%
\end{pgfscope}%
\end{pgfscope}%
\begin{pgfscope}%
\pgftext[x=10.902840in,y=0.625478in,,top]{\sffamily\fontsize{10.000000}{12.000000}\selectfont 30}%
\end{pgfscope}%
\begin{pgfscope}%
\pgfsetbuttcap%
\pgfsetroundjoin%
\definecolor{currentfill}{rgb}{0.000000,0.000000,0.000000}%
\pgfsetfillcolor{currentfill}%
\pgfsetlinewidth{0.803000pt}%
\definecolor{currentstroke}{rgb}{0.000000,0.000000,0.000000}%
\pgfsetstrokecolor{currentstroke}%
\pgfsetdash{}{0pt}%
\pgfsys@defobject{currentmarker}{\pgfqpoint{0.000000in}{-0.048611in}}{\pgfqpoint{0.000000in}{0.000000in}}{%
\pgfpathmoveto{\pgfqpoint{0.000000in}{0.000000in}}%
\pgfpathlineto{\pgfqpoint{0.000000in}{-0.048611in}}%
\pgfusepath{stroke,fill}%
}%
\begin{pgfscope}%
\pgfsys@transformshift{11.472988in}{0.722700in}%
\pgfsys@useobject{currentmarker}{}%
\end{pgfscope}%
\end{pgfscope}%
\begin{pgfscope}%
\pgftext[x=11.472988in,y=0.625478in,,top]{\sffamily\fontsize{10.000000}{12.000000}\selectfont 35}%
\end{pgfscope}%
\begin{pgfscope}%
\pgfsetbuttcap%
\pgfsetroundjoin%
\definecolor{currentfill}{rgb}{0.000000,0.000000,0.000000}%
\pgfsetfillcolor{currentfill}%
\pgfsetlinewidth{0.803000pt}%
\definecolor{currentstroke}{rgb}{0.000000,0.000000,0.000000}%
\pgfsetstrokecolor{currentstroke}%
\pgfsetdash{}{0pt}%
\pgfsys@defobject{currentmarker}{\pgfqpoint{0.000000in}{-0.048611in}}{\pgfqpoint{0.000000in}{0.000000in}}{%
\pgfpathmoveto{\pgfqpoint{0.000000in}{0.000000in}}%
\pgfpathlineto{\pgfqpoint{0.000000in}{-0.048611in}}%
\pgfusepath{stroke,fill}%
}%
\begin{pgfscope}%
\pgfsys@transformshift{12.043135in}{0.722700in}%
\pgfsys@useobject{currentmarker}{}%
\end{pgfscope}%
\end{pgfscope}%
\begin{pgfscope}%
\pgftext[x=12.043135in,y=0.625478in,,top]{\sffamily\fontsize{10.000000}{12.000000}\selectfont 40}%
\end{pgfscope}%
\begin{pgfscope}%
\pgftext[x=9.887977in,y=0.435509in,,top]{\sffamily\fontsize{16.000000}{19.200000}\selectfont \(\displaystyle z-position [\mu m]\)}%
\end{pgfscope}%
\begin{pgfscope}%
\pgfsetbuttcap%
\pgfsetroundjoin%
\definecolor{currentfill}{rgb}{0.000000,0.000000,0.000000}%
\pgfsetfillcolor{currentfill}%
\pgfsetlinewidth{0.803000pt}%
\definecolor{currentstroke}{rgb}{0.000000,0.000000,0.000000}%
\pgfsetstrokecolor{currentstroke}%
\pgfsetdash{}{0pt}%
\pgfsys@defobject{currentmarker}{\pgfqpoint{-0.048611in}{0.000000in}}{\pgfqpoint{0.000000in}{0.000000in}}{%
\pgfpathmoveto{\pgfqpoint{0.000000in}{0.000000in}}%
\pgfpathlineto{\pgfqpoint{-0.048611in}{0.000000in}}%
\pgfusepath{stroke,fill}%
}%
\begin{pgfscope}%
\pgfsys@transformshift{7.481955in}{0.722700in}%
\pgfsys@useobject{currentmarker}{}%
\end{pgfscope}%
\end{pgfscope}%
\begin{pgfscope}%
\pgftext[x=6.870748in,y=0.669938in,left,base]{\sffamily\fontsize{10.000000}{12.000000}\selectfont -0.003}%
\end{pgfscope}%
\begin{pgfscope}%
\pgfsetbuttcap%
\pgfsetroundjoin%
\definecolor{currentfill}{rgb}{0.000000,0.000000,0.000000}%
\pgfsetfillcolor{currentfill}%
\pgfsetlinewidth{0.803000pt}%
\definecolor{currentstroke}{rgb}{0.000000,0.000000,0.000000}%
\pgfsetstrokecolor{currentstroke}%
\pgfsetdash{}{0pt}%
\pgfsys@defobject{currentmarker}{\pgfqpoint{-0.048611in}{0.000000in}}{\pgfqpoint{0.000000in}{0.000000in}}{%
\pgfpathmoveto{\pgfqpoint{0.000000in}{0.000000in}}%
\pgfpathlineto{\pgfqpoint{-0.048611in}{0.000000in}}%
\pgfusepath{stroke,fill}%
}%
\begin{pgfscope}%
\pgfsys@transformshift{7.481955in}{1.565850in}%
\pgfsys@useobject{currentmarker}{}%
\end{pgfscope}%
\end{pgfscope}%
\begin{pgfscope}%
\pgftext[x=6.870748in,y=1.513088in,left,base]{\sffamily\fontsize{10.000000}{12.000000}\selectfont -0.002}%
\end{pgfscope}%
\begin{pgfscope}%
\pgfsetbuttcap%
\pgfsetroundjoin%
\definecolor{currentfill}{rgb}{0.000000,0.000000,0.000000}%
\pgfsetfillcolor{currentfill}%
\pgfsetlinewidth{0.803000pt}%
\definecolor{currentstroke}{rgb}{0.000000,0.000000,0.000000}%
\pgfsetstrokecolor{currentstroke}%
\pgfsetdash{}{0pt}%
\pgfsys@defobject{currentmarker}{\pgfqpoint{-0.048611in}{0.000000in}}{\pgfqpoint{0.000000in}{0.000000in}}{%
\pgfpathmoveto{\pgfqpoint{0.000000in}{0.000000in}}%
\pgfpathlineto{\pgfqpoint{-0.048611in}{0.000000in}}%
\pgfusepath{stroke,fill}%
}%
\begin{pgfscope}%
\pgfsys@transformshift{7.481955in}{2.409000in}%
\pgfsys@useobject{currentmarker}{}%
\end{pgfscope}%
\end{pgfscope}%
\begin{pgfscope}%
\pgftext[x=6.870748in,y=2.356238in,left,base]{\sffamily\fontsize{10.000000}{12.000000}\selectfont -0.001}%
\end{pgfscope}%
\begin{pgfscope}%
\pgfsetbuttcap%
\pgfsetroundjoin%
\definecolor{currentfill}{rgb}{0.000000,0.000000,0.000000}%
\pgfsetfillcolor{currentfill}%
\pgfsetlinewidth{0.803000pt}%
\definecolor{currentstroke}{rgb}{0.000000,0.000000,0.000000}%
\pgfsetstrokecolor{currentstroke}%
\pgfsetdash{}{0pt}%
\pgfsys@defobject{currentmarker}{\pgfqpoint{-0.048611in}{0.000000in}}{\pgfqpoint{0.000000in}{0.000000in}}{%
\pgfpathmoveto{\pgfqpoint{0.000000in}{0.000000in}}%
\pgfpathlineto{\pgfqpoint{-0.048611in}{0.000000in}}%
\pgfusepath{stroke,fill}%
}%
\begin{pgfscope}%
\pgfsys@transformshift{7.481955in}{3.252150in}%
\pgfsys@useobject{currentmarker}{}%
\end{pgfscope}%
\end{pgfscope}%
\begin{pgfscope}%
\pgftext[x=6.987122in,y=3.199388in,left,base]{\sffamily\fontsize{10.000000}{12.000000}\selectfont 0.000}%
\end{pgfscope}%
\begin{pgfscope}%
\pgfsetbuttcap%
\pgfsetroundjoin%
\definecolor{currentfill}{rgb}{0.000000,0.000000,0.000000}%
\pgfsetfillcolor{currentfill}%
\pgfsetlinewidth{0.803000pt}%
\definecolor{currentstroke}{rgb}{0.000000,0.000000,0.000000}%
\pgfsetstrokecolor{currentstroke}%
\pgfsetdash{}{0pt}%
\pgfsys@defobject{currentmarker}{\pgfqpoint{-0.048611in}{0.000000in}}{\pgfqpoint{0.000000in}{0.000000in}}{%
\pgfpathmoveto{\pgfqpoint{0.000000in}{0.000000in}}%
\pgfpathlineto{\pgfqpoint{-0.048611in}{0.000000in}}%
\pgfusepath{stroke,fill}%
}%
\begin{pgfscope}%
\pgfsys@transformshift{7.481955in}{4.095300in}%
\pgfsys@useobject{currentmarker}{}%
\end{pgfscope}%
\end{pgfscope}%
\begin{pgfscope}%
\pgftext[x=6.987122in,y=4.042538in,left,base]{\sffamily\fontsize{10.000000}{12.000000}\selectfont 0.001}%
\end{pgfscope}%
\begin{pgfscope}%
\pgfsetbuttcap%
\pgfsetroundjoin%
\definecolor{currentfill}{rgb}{0.000000,0.000000,0.000000}%
\pgfsetfillcolor{currentfill}%
\pgfsetlinewidth{0.803000pt}%
\definecolor{currentstroke}{rgb}{0.000000,0.000000,0.000000}%
\pgfsetstrokecolor{currentstroke}%
\pgfsetdash{}{0pt}%
\pgfsys@defobject{currentmarker}{\pgfqpoint{-0.048611in}{0.000000in}}{\pgfqpoint{0.000000in}{0.000000in}}{%
\pgfpathmoveto{\pgfqpoint{0.000000in}{0.000000in}}%
\pgfpathlineto{\pgfqpoint{-0.048611in}{0.000000in}}%
\pgfusepath{stroke,fill}%
}%
\begin{pgfscope}%
\pgfsys@transformshift{7.481955in}{4.938450in}%
\pgfsys@useobject{currentmarker}{}%
\end{pgfscope}%
\end{pgfscope}%
\begin{pgfscope}%
\pgftext[x=6.987122in,y=4.885688in,left,base]{\sffamily\fontsize{10.000000}{12.000000}\selectfont 0.002}%
\end{pgfscope}%
\begin{pgfscope}%
\pgfsetbuttcap%
\pgfsetroundjoin%
\definecolor{currentfill}{rgb}{0.000000,0.000000,0.000000}%
\pgfsetfillcolor{currentfill}%
\pgfsetlinewidth{0.803000pt}%
\definecolor{currentstroke}{rgb}{0.000000,0.000000,0.000000}%
\pgfsetstrokecolor{currentstroke}%
\pgfsetdash{}{0pt}%
\pgfsys@defobject{currentmarker}{\pgfqpoint{-0.048611in}{0.000000in}}{\pgfqpoint{0.000000in}{0.000000in}}{%
\pgfpathmoveto{\pgfqpoint{0.000000in}{0.000000in}}%
\pgfpathlineto{\pgfqpoint{-0.048611in}{0.000000in}}%
\pgfusepath{stroke,fill}%
}%
\begin{pgfscope}%
\pgfsys@transformshift{7.481955in}{5.781600in}%
\pgfsys@useobject{currentmarker}{}%
\end{pgfscope}%
\end{pgfscope}%
\begin{pgfscope}%
\pgftext[x=6.987122in,y=5.728838in,left,base]{\sffamily\fontsize{10.000000}{12.000000}\selectfont 0.003}%
\end{pgfscope}%
\begin{pgfscope}%
\pgftext[x=6.815193in,y=3.252150in,,bottom,rotate=90.000000]{\sffamily\fontsize{16.000000}{19.200000}\selectfont \(\displaystyle Poynting\) \(\displaystyle vector\)}%
\end{pgfscope}%
\begin{pgfscope}%
\pgfpathrectangle{\pgfqpoint{7.481955in}{0.722700in}}{\pgfqpoint{4.812045in}{5.058900in}} %
\pgfusepath{clip}%
\pgfsetrectcap%
\pgfsetroundjoin%
\pgfsetlinewidth{1.505625pt}%
\definecolor{currentstroke}{rgb}{0.000000,0.000000,0.000000}%
\pgfsetstrokecolor{currentstroke}%
\pgfsetdash{}{0pt}%
\pgfpathmoveto{\pgfqpoint{7.481955in}{3.252150in}}%
\pgfpathlineto{\pgfqpoint{10.664600in}{3.251051in}}%
\pgfpathlineto{\pgfqpoint{10.667947in}{3.247710in}}%
\pgfpathlineto{\pgfqpoint{10.671713in}{3.240284in}}%
\pgfpathlineto{\pgfqpoint{10.683011in}{3.213922in}}%
\pgfpathlineto{\pgfqpoint{10.684685in}{3.214753in}}%
\pgfpathlineto{\pgfqpoint{10.686777in}{3.218907in}}%
\pgfpathlineto{\pgfqpoint{10.690124in}{3.231605in}}%
\pgfpathlineto{\pgfqpoint{10.695564in}{3.251633in}}%
\pgfpathlineto{\pgfqpoint{10.696819in}{3.251949in}}%
\pgfpathlineto{\pgfqpoint{10.698075in}{3.249246in}}%
\pgfpathlineto{\pgfqpoint{10.700167in}{3.236532in}}%
\pgfpathlineto{\pgfqpoint{10.703096in}{3.198985in}}%
\pgfpathlineto{\pgfqpoint{10.707280in}{3.108898in}}%
\pgfpathlineto{\pgfqpoint{10.714812in}{2.944021in}}%
\pgfpathlineto{\pgfqpoint{10.716904in}{2.930374in}}%
\pgfpathlineto{\pgfqpoint{10.717741in}{2.931321in}}%
\pgfpathlineto{\pgfqpoint{10.718997in}{2.939987in}}%
\pgfpathlineto{\pgfqpoint{10.721089in}{2.973265in}}%
\pgfpathlineto{\pgfqpoint{10.724855in}{3.078887in}}%
\pgfpathlineto{\pgfqpoint{10.730713in}{3.239550in}}%
\pgfpathlineto{\pgfqpoint{10.732387in}{3.252138in}}%
\pgfpathlineto{\pgfqpoint{10.733224in}{3.249317in}}%
\pgfpathlineto{\pgfqpoint{10.734479in}{3.232441in}}%
\pgfpathlineto{\pgfqpoint{10.736571in}{3.168983in}}%
\pgfpathlineto{\pgfqpoint{10.739919in}{2.981839in}}%
\pgfpathlineto{\pgfqpoint{10.750380in}{2.308144in}}%
\pgfpathlineto{\pgfqpoint{10.751635in}{2.292580in}}%
\pgfpathlineto{\pgfqpoint{10.752053in}{2.292627in}}%
\pgfpathlineto{\pgfqpoint{10.752890in}{2.300724in}}%
\pgfpathlineto{\pgfqpoint{10.754564in}{2.348565in}}%
\pgfpathlineto{\pgfqpoint{10.757493in}{2.522626in}}%
\pgfpathlineto{\pgfqpoint{10.767954in}{3.248924in}}%
\pgfpathlineto{\pgfqpoint{10.768372in}{3.251908in}}%
\pgfpathlineto{\pgfqpoint{10.768791in}{3.251475in}}%
\pgfpathlineto{\pgfqpoint{10.769628in}{3.240138in}}%
\pgfpathlineto{\pgfqpoint{10.771301in}{3.175152in}}%
\pgfpathlineto{\pgfqpoint{10.774231in}{2.934698in}}%
\pgfpathlineto{\pgfqpoint{10.780089in}{2.182281in}}%
\pgfpathlineto{\pgfqpoint{10.784691in}{1.722296in}}%
\pgfpathlineto{\pgfqpoint{10.786784in}{1.649466in}}%
\pgfpathlineto{\pgfqpoint{10.787202in}{1.647548in}}%
\pgfpathlineto{\pgfqpoint{10.787202in}{1.647548in}}%
\pgfpathlineto{\pgfqpoint{10.787202in}{1.647548in}}%
\pgfpathlineto{\pgfqpoint{10.788039in}{1.656707in}}%
\pgfpathlineto{\pgfqpoint{10.789713in}{1.726246in}}%
\pgfpathlineto{\pgfqpoint{10.792223in}{1.946223in}}%
\pgfpathlineto{\pgfqpoint{10.798082in}{2.742072in}}%
\pgfpathlineto{\pgfqpoint{10.802266in}{3.177398in}}%
\pgfpathlineto{\pgfqpoint{10.804358in}{3.251202in}}%
\pgfpathlineto{\pgfqpoint{10.804777in}{3.251808in}}%
\pgfpathlineto{\pgfqpoint{10.805613in}{3.238395in}}%
\pgfpathlineto{\pgfqpoint{10.807287in}{3.153644in}}%
\pgfpathlineto{\pgfqpoint{10.810216in}{2.838115in}}%
\pgfpathlineto{\pgfqpoint{10.822351in}{1.261578in}}%
\pgfpathlineto{\pgfqpoint{10.822769in}{1.257203in}}%
\pgfpathlineto{\pgfqpoint{10.823188in}{1.258152in}}%
\pgfpathlineto{\pgfqpoint{10.824443in}{1.292837in}}%
\pgfpathlineto{\pgfqpoint{10.826535in}{1.451797in}}%
\pgfpathlineto{\pgfqpoint{10.829883in}{1.915306in}}%
\pgfpathlineto{\pgfqpoint{10.838252in}{3.157972in}}%
\pgfpathlineto{\pgfqpoint{10.840344in}{3.250019in}}%
\pgfpathlineto{\pgfqpoint{10.840762in}{3.252106in}}%
\pgfpathlineto{\pgfqpoint{10.840762in}{3.252106in}}%
\pgfpathlineto{\pgfqpoint{10.840762in}{3.252106in}}%
\pgfpathlineto{\pgfqpoint{10.841599in}{3.239571in}}%
\pgfpathlineto{\pgfqpoint{10.843273in}{3.148802in}}%
\pgfpathlineto{\pgfqpoint{10.845784in}{2.864570in}}%
\pgfpathlineto{\pgfqpoint{10.851223in}{1.906694in}}%
\pgfpathlineto{\pgfqpoint{10.856245in}{1.196421in}}%
\pgfpathlineto{\pgfqpoint{10.858755in}{1.088010in}}%
\pgfpathlineto{\pgfqpoint{10.859174in}{1.089765in}}%
\pgfpathlineto{\pgfqpoint{10.860429in}{1.129338in}}%
\pgfpathlineto{\pgfqpoint{10.862521in}{1.303815in}}%
\pgfpathlineto{\pgfqpoint{10.866287in}{1.881087in}}%
\pgfpathlineto{\pgfqpoint{10.874237in}{3.144605in}}%
\pgfpathlineto{\pgfqpoint{10.876748in}{3.252104in}}%
\pgfpathlineto{\pgfqpoint{10.877585in}{3.241895in}}%
\pgfpathlineto{\pgfqpoint{10.879259in}{3.152736in}}%
\pgfpathlineto{\pgfqpoint{10.881769in}{2.864241in}}%
\pgfpathlineto{\pgfqpoint{10.887209in}{1.880700in}}%
\pgfpathlineto{\pgfqpoint{10.892230in}{1.146613in}}%
\pgfpathlineto{\pgfqpoint{10.894741in}{1.031758in}}%
\pgfpathlineto{\pgfqpoint{10.895159in}{1.032911in}}%
\pgfpathlineto{\pgfqpoint{10.896415in}{1.071518in}}%
\pgfpathlineto{\pgfqpoint{10.898507in}{1.247160in}}%
\pgfpathlineto{\pgfqpoint{10.901854in}{1.757316in}}%
\pgfpathlineto{\pgfqpoint{10.910223in}{3.134082in}}%
\pgfpathlineto{\pgfqpoint{10.912734in}{3.251734in}}%
\pgfpathlineto{\pgfqpoint{10.913152in}{3.250994in}}%
\pgfpathlineto{\pgfqpoint{10.914407in}{3.213478in}}%
\pgfpathlineto{\pgfqpoint{10.916500in}{3.039033in}}%
\pgfpathlineto{\pgfqpoint{10.919847in}{2.528712in}}%
\pgfpathlineto{\pgfqpoint{10.928634in}{1.105898in}}%
\pgfpathlineto{\pgfqpoint{10.930727in}{1.017784in}}%
\pgfpathlineto{\pgfqpoint{10.931145in}{1.017716in}}%
\pgfpathlineto{\pgfqpoint{10.931982in}{1.035327in}}%
\pgfpathlineto{\pgfqpoint{10.933656in}{1.139583in}}%
\pgfpathlineto{\pgfqpoint{10.936585in}{1.514655in}}%
\pgfpathlineto{\pgfqpoint{10.947883in}{3.231762in}}%
\pgfpathlineto{\pgfqpoint{10.949138in}{3.251733in}}%
\pgfpathlineto{\pgfqpoint{10.949975in}{3.235450in}}%
\pgfpathlineto{\pgfqpoint{10.951649in}{3.133674in}}%
\pgfpathlineto{\pgfqpoint{10.954578in}{2.761838in}}%
\pgfpathlineto{\pgfqpoint{10.965875in}{1.036774in}}%
\pgfpathlineto{\pgfqpoint{10.967131in}{1.014554in}}%
\pgfpathlineto{\pgfqpoint{10.967968in}{1.029347in}}%
\pgfpathlineto{\pgfqpoint{10.969641in}{1.128305in}}%
\pgfpathlineto{\pgfqpoint{10.972570in}{1.496290in}}%
\pgfpathlineto{\pgfqpoint{10.984287in}{3.241728in}}%
\pgfpathlineto{\pgfqpoint{10.985124in}{3.252104in}}%
\pgfpathlineto{\pgfqpoint{10.985542in}{3.248394in}}%
\pgfpathlineto{\pgfqpoint{10.986797in}{3.202006in}}%
\pgfpathlineto{\pgfqpoint{10.988890in}{3.013885in}}%
\pgfpathlineto{\pgfqpoint{10.992656in}{2.410432in}}%
\pgfpathlineto{\pgfqpoint{11.000187in}{1.156462in}}%
\pgfpathlineto{\pgfqpoint{11.002698in}{1.017057in}}%
\pgfpathlineto{\pgfqpoint{11.003117in}{1.014088in}}%
\pgfpathlineto{\pgfqpoint{11.003535in}{1.017056in}}%
\pgfpathlineto{\pgfqpoint{11.004790in}{1.061260in}}%
\pgfpathlineto{\pgfqpoint{11.006882in}{1.246105in}}%
\pgfpathlineto{\pgfqpoint{11.010648in}{1.845950in}}%
\pgfpathlineto{\pgfqpoint{11.018180in}{3.104765in}}%
\pgfpathlineto{\pgfqpoint{11.020691in}{3.248394in}}%
\pgfpathlineto{\pgfqpoint{11.021109in}{3.252104in}}%
\pgfpathlineto{\pgfqpoint{11.021528in}{3.249877in}}%
\pgfpathlineto{\pgfqpoint{11.022783in}{3.207859in}}%
\pgfpathlineto{\pgfqpoint{11.024875in}{3.026305in}}%
\pgfpathlineto{\pgfqpoint{11.028641in}{2.430117in}}%
\pgfpathlineto{\pgfqpoint{11.036592in}{1.128076in}}%
\pgfpathlineto{\pgfqpoint{11.039102in}{1.014262in}}%
\pgfpathlineto{\pgfqpoint{11.039521in}{1.015747in}}%
\pgfpathlineto{\pgfqpoint{11.040776in}{1.055574in}}%
\pgfpathlineto{\pgfqpoint{11.042868in}{1.233821in}}%
\pgfpathlineto{\pgfqpoint{11.046634in}{1.826298in}}%
\pgfpathlineto{\pgfqpoint{11.054584in}{3.133629in}}%
\pgfpathlineto{\pgfqpoint{11.057095in}{3.251732in}}%
\pgfpathlineto{\pgfqpoint{11.057514in}{3.250990in}}%
\pgfpathlineto{\pgfqpoint{11.058769in}{3.213358in}}%
\pgfpathlineto{\pgfqpoint{11.060861in}{3.038433in}}%
\pgfpathlineto{\pgfqpoint{11.064209in}{2.526996in}}%
\pgfpathlineto{\pgfqpoint{11.072996in}{1.102703in}}%
\pgfpathlineto{\pgfqpoint{11.075088in}{1.014819in}}%
\pgfpathlineto{\pgfqpoint{11.075506in}{1.014819in}}%
\pgfpathlineto{\pgfqpoint{11.076343in}{1.032584in}}%
\pgfpathlineto{\pgfqpoint{11.078017in}{1.137204in}}%
\pgfpathlineto{\pgfqpoint{11.080946in}{1.513000in}}%
\pgfpathlineto{\pgfqpoint{11.092244in}{3.231752in}}%
\pgfpathlineto{\pgfqpoint{11.093499in}{3.251732in}}%
\pgfpathlineto{\pgfqpoint{11.094336in}{3.235443in}}%
\pgfpathlineto{\pgfqpoint{11.096010in}{3.133629in}}%
\pgfpathlineto{\pgfqpoint{11.098939in}{2.761684in}}%
\pgfpathlineto{\pgfqpoint{11.110237in}{1.036457in}}%
\pgfpathlineto{\pgfqpoint{11.111492in}{1.014262in}}%
\pgfpathlineto{\pgfqpoint{11.112329in}{1.029075in}}%
\pgfpathlineto{\pgfqpoint{11.114003in}{1.128076in}}%
\pgfpathlineto{\pgfqpoint{11.116932in}{1.496139in}}%
\pgfpathlineto{\pgfqpoint{11.128648in}{3.241727in}}%
\pgfpathlineto{\pgfqpoint{11.129485in}{3.252104in}}%
\pgfpathlineto{\pgfqpoint{11.129903in}{3.248394in}}%
\pgfpathlineto{\pgfqpoint{11.131159in}{3.202004in}}%
\pgfpathlineto{\pgfqpoint{11.133251in}{3.013880in}}%
\pgfpathlineto{\pgfqpoint{11.137017in}{2.410421in}}%
\pgfpathlineto{\pgfqpoint{11.144549in}{1.156455in}}%
\pgfpathlineto{\pgfqpoint{11.147059in}{1.017056in}}%
\pgfpathlineto{\pgfqpoint{11.147478in}{1.014088in}}%
\pgfpathlineto{\pgfqpoint{11.147896in}{1.017057in}}%
\pgfpathlineto{\pgfqpoint{11.149152in}{1.061264in}}%
\pgfpathlineto{\pgfqpoint{11.151244in}{1.246113in}}%
\pgfpathlineto{\pgfqpoint{11.155010in}{1.845963in}}%
\pgfpathlineto{\pgfqpoint{11.162542in}{3.104768in}}%
\pgfpathlineto{\pgfqpoint{11.165052in}{3.248394in}}%
\pgfpathlineto{\pgfqpoint{11.165471in}{3.252104in}}%
\pgfpathlineto{\pgfqpoint{11.165889in}{3.249877in}}%
\pgfpathlineto{\pgfqpoint{11.167144in}{3.207861in}}%
\pgfpathlineto{\pgfqpoint{11.169237in}{3.026314in}}%
\pgfpathlineto{\pgfqpoint{11.173003in}{2.430164in}}%
\pgfpathlineto{\pgfqpoint{11.180953in}{1.128305in}}%
\pgfpathlineto{\pgfqpoint{11.183464in}{1.014554in}}%
\pgfpathlineto{\pgfqpoint{11.183882in}{1.016047in}}%
\pgfpathlineto{\pgfqpoint{11.185137in}{1.055897in}}%
\pgfpathlineto{\pgfqpoint{11.187230in}{1.234166in}}%
\pgfpathlineto{\pgfqpoint{11.190995in}{1.826616in}}%
\pgfpathlineto{\pgfqpoint{11.198946in}{3.133674in}}%
\pgfpathlineto{\pgfqpoint{11.201456in}{3.251733in}}%
\pgfpathlineto{\pgfqpoint{11.201875in}{3.250991in}}%
\pgfpathlineto{\pgfqpoint{11.203130in}{3.213377in}}%
\pgfpathlineto{\pgfqpoint{11.205222in}{3.038554in}}%
\pgfpathlineto{\pgfqpoint{11.208570in}{2.527497in}}%
\pgfpathlineto{\pgfqpoint{11.217357in}{1.105179in}}%
\pgfpathlineto{\pgfqpoint{11.219449in}{1.017716in}}%
\pgfpathlineto{\pgfqpoint{11.219868in}{1.017784in}}%
\pgfpathlineto{\pgfqpoint{11.220705in}{1.035664in}}%
\pgfpathlineto{\pgfqpoint{11.222378in}{1.140420in}}%
\pgfpathlineto{\pgfqpoint{11.225307in}{1.516096in}}%
\pgfpathlineto{\pgfqpoint{11.236605in}{3.231817in}}%
\pgfpathlineto{\pgfqpoint{11.237861in}{3.251734in}}%
\pgfpathlineto{\pgfqpoint{11.238698in}{3.235502in}}%
\pgfpathlineto{\pgfqpoint{11.240371in}{3.134082in}}%
\pgfpathlineto{\pgfqpoint{11.243300in}{2.763842in}}%
\pgfpathlineto{\pgfqpoint{11.254598in}{1.052843in}}%
\pgfpathlineto{\pgfqpoint{11.255853in}{1.031758in}}%
\pgfpathlineto{\pgfqpoint{11.256690in}{1.047108in}}%
\pgfpathlineto{\pgfqpoint{11.258364in}{1.146613in}}%
\pgfpathlineto{\pgfqpoint{11.261293in}{1.513529in}}%
\pgfpathlineto{\pgfqpoint{11.272591in}{3.228083in}}%
\pgfpathlineto{\pgfqpoint{11.273846in}{3.252104in}}%
\pgfpathlineto{\pgfqpoint{11.274265in}{3.248458in}}%
\pgfpathlineto{\pgfqpoint{11.275520in}{3.202896in}}%
\pgfpathlineto{\pgfqpoint{11.277612in}{3.018482in}}%
\pgfpathlineto{\pgfqpoint{11.281378in}{2.429233in}}%
\pgfpathlineto{\pgfqpoint{11.288910in}{1.218638in}}%
\pgfpathlineto{\pgfqpoint{11.291421in}{1.089765in}}%
\pgfpathlineto{\pgfqpoint{11.291839in}{1.088010in}}%
\pgfpathlineto{\pgfqpoint{11.292676in}{1.101740in}}%
\pgfpathlineto{\pgfqpoint{11.294350in}{1.196421in}}%
\pgfpathlineto{\pgfqpoint{11.297279in}{1.550643in}}%
\pgfpathlineto{\pgfqpoint{11.308995in}{3.239571in}}%
\pgfpathlineto{\pgfqpoint{11.309832in}{3.252106in}}%
\pgfpathlineto{\pgfqpoint{11.310251in}{3.250019in}}%
\pgfpathlineto{\pgfqpoint{11.311506in}{3.210744in}}%
\pgfpathlineto{\pgfqpoint{11.313598in}{3.042049in}}%
\pgfpathlineto{\pgfqpoint{11.317364in}{2.494772in}}%
\pgfpathlineto{\pgfqpoint{11.324896in}{1.373828in}}%
\pgfpathlineto{\pgfqpoint{11.327407in}{1.258152in}}%
\pgfpathlineto{\pgfqpoint{11.327825in}{1.257203in}}%
\pgfpathlineto{\pgfqpoint{11.328662in}{1.271247in}}%
\pgfpathlineto{\pgfqpoint{11.330336in}{1.361261in}}%
\pgfpathlineto{\pgfqpoint{11.333265in}{1.690985in}}%
\pgfpathlineto{\pgfqpoint{11.344563in}{3.224377in}}%
\pgfpathlineto{\pgfqpoint{11.345818in}{3.251808in}}%
\pgfpathlineto{\pgfqpoint{11.346236in}{3.251202in}}%
\pgfpathlineto{\pgfqpoint{11.347492in}{3.220686in}}%
\pgfpathlineto{\pgfqpoint{11.349584in}{3.081085in}}%
\pgfpathlineto{\pgfqpoint{11.353350in}{2.626129in}}%
\pgfpathlineto{\pgfqpoint{11.360463in}{1.753866in}}%
\pgfpathlineto{\pgfqpoint{11.362974in}{1.649961in}}%
\pgfpathlineto{\pgfqpoint{11.363392in}{1.647548in}}%
\pgfpathlineto{\pgfqpoint{11.363811in}{1.649466in}}%
\pgfpathlineto{\pgfqpoint{11.365066in}{1.680828in}}%
\pgfpathlineto{\pgfqpoint{11.367158in}{1.812812in}}%
\pgfpathlineto{\pgfqpoint{11.370924in}{2.237096in}}%
\pgfpathlineto{\pgfqpoint{11.378874in}{3.150232in}}%
\pgfpathlineto{\pgfqpoint{11.381385in}{3.247565in}}%
\pgfpathlineto{\pgfqpoint{11.382222in}{3.251908in}}%
\pgfpathlineto{\pgfqpoint{11.382640in}{3.248924in}}%
\pgfpathlineto{\pgfqpoint{11.383896in}{3.220329in}}%
\pgfpathlineto{\pgfqpoint{11.386406in}{3.086567in}}%
\pgfpathlineto{\pgfqpoint{11.393101in}{2.522626in}}%
\pgfpathlineto{\pgfqpoint{11.396867in}{2.319457in}}%
\pgfpathlineto{\pgfqpoint{11.398960in}{2.292580in}}%
\pgfpathlineto{\pgfqpoint{11.399796in}{2.300381in}}%
\pgfpathlineto{\pgfqpoint{11.401470in}{2.346070in}}%
\pgfpathlineto{\pgfqpoint{11.404399in}{2.506377in}}%
\pgfpathlineto{\pgfqpoint{11.415697in}{3.223308in}}%
\pgfpathlineto{\pgfqpoint{11.417789in}{3.251540in}}%
\pgfpathlineto{\pgfqpoint{11.418208in}{3.252138in}}%
\pgfpathlineto{\pgfqpoint{11.418626in}{3.251168in}}%
\pgfpathlineto{\pgfqpoint{11.419881in}{3.239550in}}%
\pgfpathlineto{\pgfqpoint{11.422392in}{3.184862in}}%
\pgfpathlineto{\pgfqpoint{11.432016in}{2.936129in}}%
\pgfpathlineto{\pgfqpoint{11.433690in}{2.930374in}}%
\pgfpathlineto{\pgfqpoint{11.434945in}{2.935947in}}%
\pgfpathlineto{\pgfqpoint{11.437037in}{2.961925in}}%
\pgfpathlineto{\pgfqpoint{11.440803in}{3.044689in}}%
\pgfpathlineto{\pgfqpoint{11.448335in}{3.212126in}}%
\pgfpathlineto{\pgfqpoint{11.451683in}{3.245479in}}%
\pgfpathlineto{\pgfqpoint{11.453775in}{3.251949in}}%
\pgfpathlineto{\pgfqpoint{11.455030in}{3.251633in}}%
\pgfpathlineto{\pgfqpoint{11.456704in}{3.247656in}}%
\pgfpathlineto{\pgfqpoint{11.461725in}{3.226185in}}%
\pgfpathlineto{\pgfqpoint{11.465491in}{3.215307in}}%
\pgfpathlineto{\pgfqpoint{11.467583in}{3.213922in}}%
\pgfpathlineto{\pgfqpoint{11.469257in}{3.215154in}}%
\pgfpathlineto{\pgfqpoint{11.472186in}{3.221114in}}%
\pgfpathlineto{\pgfqpoint{11.483484in}{3.248814in}}%
\pgfpathlineto{\pgfqpoint{11.487250in}{3.251655in}}%
\pgfpathlineto{\pgfqpoint{11.491853in}{3.252079in}}%
\pgfpathlineto{\pgfqpoint{11.507754in}{3.251696in}}%
\pgfpathlineto{\pgfqpoint{11.531186in}{3.252149in}}%
\pgfpathlineto{\pgfqpoint{12.293582in}{3.252150in}}%
\pgfpathlineto{\pgfqpoint{12.293582in}{3.252150in}}%
\pgfusepath{stroke}%
\end{pgfscope}%
\begin{pgfscope}%
\pgfsetrectcap%
\pgfsetmiterjoin%
\pgfsetlinewidth{0.803000pt}%
\definecolor{currentstroke}{rgb}{0.000000,0.000000,0.000000}%
\pgfsetstrokecolor{currentstroke}%
\pgfsetdash{}{0pt}%
\pgfpathmoveto{\pgfqpoint{7.481955in}{0.722700in}}%
\pgfpathlineto{\pgfqpoint{7.481955in}{5.781600in}}%
\pgfusepath{stroke}%
\end{pgfscope}%
\begin{pgfscope}%
\pgfsetrectcap%
\pgfsetmiterjoin%
\pgfsetlinewidth{0.803000pt}%
\definecolor{currentstroke}{rgb}{0.000000,0.000000,0.000000}%
\pgfsetstrokecolor{currentstroke}%
\pgfsetdash{}{0pt}%
\pgfpathmoveto{\pgfqpoint{12.294000in}{0.722700in}}%
\pgfpathlineto{\pgfqpoint{12.294000in}{5.781600in}}%
\pgfusepath{stroke}%
\end{pgfscope}%
\begin{pgfscope}%
\pgfsetrectcap%
\pgfsetmiterjoin%
\pgfsetlinewidth{0.803000pt}%
\definecolor{currentstroke}{rgb}{0.000000,0.000000,0.000000}%
\pgfsetstrokecolor{currentstroke}%
\pgfsetdash{}{0pt}%
\pgfpathmoveto{\pgfqpoint{7.481955in}{0.722700in}}%
\pgfpathlineto{\pgfqpoint{12.294000in}{0.722700in}}%
\pgfusepath{stroke}%
\end{pgfscope}%
\begin{pgfscope}%
\pgfsetrectcap%
\pgfsetmiterjoin%
\pgfsetlinewidth{0.803000pt}%
\definecolor{currentstroke}{rgb}{0.000000,0.000000,0.000000}%
\pgfsetstrokecolor{currentstroke}%
\pgfsetdash{}{0pt}%
\pgfpathmoveto{\pgfqpoint{7.481955in}{5.781600in}}%
\pgfpathlineto{\pgfqpoint{12.294000in}{5.781600in}}%
\pgfusepath{stroke}%
\end{pgfscope}%
\end{pgfpicture}%
\makeatother%
\endgroup%
}}
        \subcaption{Simulation after the wave hit the right side of the numerical window.}
        \label{fig:task2_2}
    \end{subfigure}
  \caption{Simulation results from task 2.}
  \label{fig:task2}
\end{figure}

\section{Task 3}\label{sec:3}
The result from the simulation can be seen in Figure~\ref{fig:task3} and it is in accordance with the Fresnel equations.

The energy is conserved since there are no components that absorbs energy. This can be verified by adding the reflected and transmitted energy in Figure~\ref{fig:task3_1} and Figure~\ref{fig:task3_2}
\begin{figure}
  \centering
    \begin{subfigure}[b]{0.7\textwidth}
        \noindent\makebox[\textwidth]{\scalebox{0.7}{%% Creator: Matplotlib, PGF backend
%%
%% To include the figure in your LaTeX document, write
%%   \input{<filename>.pgf}
%%
%% Make sure the required packages are loaded in your preamble
%%   \usepackage{pgf}
%%
%% Figures using additional raster images can only be included by \input if
%% they are in the same directory as the main LaTeX file. For loading figures
%% from other directories you can use the `import` package
%%   \usepackage{import}
%% and then include the figures with
%%   \import{<path to file>}{<filename>.pgf}
%%
%% Matplotlib used the following preamble
%%   \usepackage{fontspec}
%%   \setmainfont{DejaVu Serif}
%%   \setsansfont{DejaVu Sans}
%%   \setmonofont{DejaVu Sans Mono}
%%
\begingroup%
\makeatletter%
\begin{pgfpicture}%
\pgfpathrectangle{\pgfpointorigin}{\pgfqpoint{7.730000in}{4.760000in}}%
\pgfusepath{use as bounding box, clip}%
\begin{pgfscope}%
\pgfsetbuttcap%
\pgfsetmiterjoin%
\definecolor{currentfill}{rgb}{1.000000,1.000000,1.000000}%
\pgfsetfillcolor{currentfill}%
\pgfsetlinewidth{0.000000pt}%
\definecolor{currentstroke}{rgb}{1.000000,1.000000,1.000000}%
\pgfsetstrokecolor{currentstroke}%
\pgfsetdash{}{0pt}%
\pgfpathmoveto{\pgfqpoint{0.000000in}{0.000000in}}%
\pgfpathlineto{\pgfqpoint{7.730000in}{0.000000in}}%
\pgfpathlineto{\pgfqpoint{7.730000in}{4.760000in}}%
\pgfpathlineto{\pgfqpoint{0.000000in}{4.760000in}}%
\pgfpathclose%
\pgfusepath{fill}%
\end{pgfscope}%
\begin{pgfscope}%
\pgfsetbuttcap%
\pgfsetmiterjoin%
\definecolor{currentfill}{rgb}{1.000000,1.000000,1.000000}%
\pgfsetfillcolor{currentfill}%
\pgfsetlinewidth{0.000000pt}%
\definecolor{currentstroke}{rgb}{0.000000,0.000000,0.000000}%
\pgfsetstrokecolor{currentstroke}%
\pgfsetstrokeopacity{0.000000}%
\pgfsetdash{}{0pt}%
\pgfpathmoveto{\pgfqpoint{0.966250in}{0.523600in}}%
\pgfpathlineto{\pgfqpoint{6.957000in}{0.523600in}}%
\pgfpathlineto{\pgfqpoint{6.957000in}{4.188800in}}%
\pgfpathlineto{\pgfqpoint{0.966250in}{4.188800in}}%
\pgfpathclose%
\pgfusepath{fill}%
\end{pgfscope}%
\begin{pgfscope}%
\pgfsetbuttcap%
\pgfsetroundjoin%
\definecolor{currentfill}{rgb}{0.000000,0.000000,0.000000}%
\pgfsetfillcolor{currentfill}%
\pgfsetlinewidth{0.803000pt}%
\definecolor{currentstroke}{rgb}{0.000000,0.000000,0.000000}%
\pgfsetstrokecolor{currentstroke}%
\pgfsetdash{}{0pt}%
\pgfsys@defobject{currentmarker}{\pgfqpoint{0.000000in}{-0.048611in}}{\pgfqpoint{0.000000in}{0.000000in}}{%
\pgfpathmoveto{\pgfqpoint{0.000000in}{0.000000in}}%
\pgfpathlineto{\pgfqpoint{0.000000in}{-0.048611in}}%
\pgfusepath{stroke,fill}%
}%
\begin{pgfscope}%
\pgfsys@transformshift{0.966250in}{0.523600in}%
\pgfsys@useobject{currentmarker}{}%
\end{pgfscope}%
\end{pgfscope}%
\begin{pgfscope}%
\pgftext[x=0.966250in,y=0.426378in,,top]{\sffamily\fontsize{10.000000}{12.000000}\selectfont 0}%
\end{pgfscope}%
\begin{pgfscope}%
\pgfsetbuttcap%
\pgfsetroundjoin%
\definecolor{currentfill}{rgb}{0.000000,0.000000,0.000000}%
\pgfsetfillcolor{currentfill}%
\pgfsetlinewidth{0.803000pt}%
\definecolor{currentstroke}{rgb}{0.000000,0.000000,0.000000}%
\pgfsetstrokecolor{currentstroke}%
\pgfsetdash{}{0pt}%
\pgfsys@defobject{currentmarker}{\pgfqpoint{0.000000in}{-0.048611in}}{\pgfqpoint{0.000000in}{0.000000in}}{%
\pgfpathmoveto{\pgfqpoint{0.000000in}{0.000000in}}%
\pgfpathlineto{\pgfqpoint{0.000000in}{-0.048611in}}%
\pgfusepath{stroke,fill}%
}%
\begin{pgfscope}%
\pgfsys@transformshift{1.676055in}{0.523600in}%
\pgfsys@useobject{currentmarker}{}%
\end{pgfscope}%
\end{pgfscope}%
\begin{pgfscope}%
\pgftext[x=1.676055in,y=0.426378in,,top]{\sffamily\fontsize{10.000000}{12.000000}\selectfont 5}%
\end{pgfscope}%
\begin{pgfscope}%
\pgfsetbuttcap%
\pgfsetroundjoin%
\definecolor{currentfill}{rgb}{0.000000,0.000000,0.000000}%
\pgfsetfillcolor{currentfill}%
\pgfsetlinewidth{0.803000pt}%
\definecolor{currentstroke}{rgb}{0.000000,0.000000,0.000000}%
\pgfsetstrokecolor{currentstroke}%
\pgfsetdash{}{0pt}%
\pgfsys@defobject{currentmarker}{\pgfqpoint{0.000000in}{-0.048611in}}{\pgfqpoint{0.000000in}{0.000000in}}{%
\pgfpathmoveto{\pgfqpoint{0.000000in}{0.000000in}}%
\pgfpathlineto{\pgfqpoint{0.000000in}{-0.048611in}}%
\pgfusepath{stroke,fill}%
}%
\begin{pgfscope}%
\pgfsys@transformshift{2.385859in}{0.523600in}%
\pgfsys@useobject{currentmarker}{}%
\end{pgfscope}%
\end{pgfscope}%
\begin{pgfscope}%
\pgftext[x=2.385859in,y=0.426378in,,top]{\sffamily\fontsize{10.000000}{12.000000}\selectfont 10}%
\end{pgfscope}%
\begin{pgfscope}%
\pgfsetbuttcap%
\pgfsetroundjoin%
\definecolor{currentfill}{rgb}{0.000000,0.000000,0.000000}%
\pgfsetfillcolor{currentfill}%
\pgfsetlinewidth{0.803000pt}%
\definecolor{currentstroke}{rgb}{0.000000,0.000000,0.000000}%
\pgfsetstrokecolor{currentstroke}%
\pgfsetdash{}{0pt}%
\pgfsys@defobject{currentmarker}{\pgfqpoint{0.000000in}{-0.048611in}}{\pgfqpoint{0.000000in}{0.000000in}}{%
\pgfpathmoveto{\pgfqpoint{0.000000in}{0.000000in}}%
\pgfpathlineto{\pgfqpoint{0.000000in}{-0.048611in}}%
\pgfusepath{stroke,fill}%
}%
\begin{pgfscope}%
\pgfsys@transformshift{3.095664in}{0.523600in}%
\pgfsys@useobject{currentmarker}{}%
\end{pgfscope}%
\end{pgfscope}%
\begin{pgfscope}%
\pgftext[x=3.095664in,y=0.426378in,,top]{\sffamily\fontsize{10.000000}{12.000000}\selectfont 15}%
\end{pgfscope}%
\begin{pgfscope}%
\pgfsetbuttcap%
\pgfsetroundjoin%
\definecolor{currentfill}{rgb}{0.000000,0.000000,0.000000}%
\pgfsetfillcolor{currentfill}%
\pgfsetlinewidth{0.803000pt}%
\definecolor{currentstroke}{rgb}{0.000000,0.000000,0.000000}%
\pgfsetstrokecolor{currentstroke}%
\pgfsetdash{}{0pt}%
\pgfsys@defobject{currentmarker}{\pgfqpoint{0.000000in}{-0.048611in}}{\pgfqpoint{0.000000in}{0.000000in}}{%
\pgfpathmoveto{\pgfqpoint{0.000000in}{0.000000in}}%
\pgfpathlineto{\pgfqpoint{0.000000in}{-0.048611in}}%
\pgfusepath{stroke,fill}%
}%
\begin{pgfscope}%
\pgfsys@transformshift{3.805468in}{0.523600in}%
\pgfsys@useobject{currentmarker}{}%
\end{pgfscope}%
\end{pgfscope}%
\begin{pgfscope}%
\pgftext[x=3.805468in,y=0.426378in,,top]{\sffamily\fontsize{10.000000}{12.000000}\selectfont 20}%
\end{pgfscope}%
\begin{pgfscope}%
\pgfsetbuttcap%
\pgfsetroundjoin%
\definecolor{currentfill}{rgb}{0.000000,0.000000,0.000000}%
\pgfsetfillcolor{currentfill}%
\pgfsetlinewidth{0.803000pt}%
\definecolor{currentstroke}{rgb}{0.000000,0.000000,0.000000}%
\pgfsetstrokecolor{currentstroke}%
\pgfsetdash{}{0pt}%
\pgfsys@defobject{currentmarker}{\pgfqpoint{0.000000in}{-0.048611in}}{\pgfqpoint{0.000000in}{0.000000in}}{%
\pgfpathmoveto{\pgfqpoint{0.000000in}{0.000000in}}%
\pgfpathlineto{\pgfqpoint{0.000000in}{-0.048611in}}%
\pgfusepath{stroke,fill}%
}%
\begin{pgfscope}%
\pgfsys@transformshift{4.515273in}{0.523600in}%
\pgfsys@useobject{currentmarker}{}%
\end{pgfscope}%
\end{pgfscope}%
\begin{pgfscope}%
\pgftext[x=4.515273in,y=0.426378in,,top]{\sffamily\fontsize{10.000000}{12.000000}\selectfont 25}%
\end{pgfscope}%
\begin{pgfscope}%
\pgfsetbuttcap%
\pgfsetroundjoin%
\definecolor{currentfill}{rgb}{0.000000,0.000000,0.000000}%
\pgfsetfillcolor{currentfill}%
\pgfsetlinewidth{0.803000pt}%
\definecolor{currentstroke}{rgb}{0.000000,0.000000,0.000000}%
\pgfsetstrokecolor{currentstroke}%
\pgfsetdash{}{0pt}%
\pgfsys@defobject{currentmarker}{\pgfqpoint{0.000000in}{-0.048611in}}{\pgfqpoint{0.000000in}{0.000000in}}{%
\pgfpathmoveto{\pgfqpoint{0.000000in}{0.000000in}}%
\pgfpathlineto{\pgfqpoint{0.000000in}{-0.048611in}}%
\pgfusepath{stroke,fill}%
}%
\begin{pgfscope}%
\pgfsys@transformshift{5.225077in}{0.523600in}%
\pgfsys@useobject{currentmarker}{}%
\end{pgfscope}%
\end{pgfscope}%
\begin{pgfscope}%
\pgftext[x=5.225077in,y=0.426378in,,top]{\sffamily\fontsize{10.000000}{12.000000}\selectfont 30}%
\end{pgfscope}%
\begin{pgfscope}%
\pgfsetbuttcap%
\pgfsetroundjoin%
\definecolor{currentfill}{rgb}{0.000000,0.000000,0.000000}%
\pgfsetfillcolor{currentfill}%
\pgfsetlinewidth{0.803000pt}%
\definecolor{currentstroke}{rgb}{0.000000,0.000000,0.000000}%
\pgfsetstrokecolor{currentstroke}%
\pgfsetdash{}{0pt}%
\pgfsys@defobject{currentmarker}{\pgfqpoint{0.000000in}{-0.048611in}}{\pgfqpoint{0.000000in}{0.000000in}}{%
\pgfpathmoveto{\pgfqpoint{0.000000in}{0.000000in}}%
\pgfpathlineto{\pgfqpoint{0.000000in}{-0.048611in}}%
\pgfusepath{stroke,fill}%
}%
\begin{pgfscope}%
\pgfsys@transformshift{5.934882in}{0.523600in}%
\pgfsys@useobject{currentmarker}{}%
\end{pgfscope}%
\end{pgfscope}%
\begin{pgfscope}%
\pgftext[x=5.934882in,y=0.426378in,,top]{\sffamily\fontsize{10.000000}{12.000000}\selectfont 35}%
\end{pgfscope}%
\begin{pgfscope}%
\pgfsetbuttcap%
\pgfsetroundjoin%
\definecolor{currentfill}{rgb}{0.000000,0.000000,0.000000}%
\pgfsetfillcolor{currentfill}%
\pgfsetlinewidth{0.803000pt}%
\definecolor{currentstroke}{rgb}{0.000000,0.000000,0.000000}%
\pgfsetstrokecolor{currentstroke}%
\pgfsetdash{}{0pt}%
\pgfsys@defobject{currentmarker}{\pgfqpoint{0.000000in}{-0.048611in}}{\pgfqpoint{0.000000in}{0.000000in}}{%
\pgfpathmoveto{\pgfqpoint{0.000000in}{0.000000in}}%
\pgfpathlineto{\pgfqpoint{0.000000in}{-0.048611in}}%
\pgfusepath{stroke,fill}%
}%
\begin{pgfscope}%
\pgfsys@transformshift{6.644686in}{0.523600in}%
\pgfsys@useobject{currentmarker}{}%
\end{pgfscope}%
\end{pgfscope}%
\begin{pgfscope}%
\pgftext[x=6.644686in,y=0.426378in,,top]{\sffamily\fontsize{10.000000}{12.000000}\selectfont 40}%
\end{pgfscope}%
\begin{pgfscope}%
\pgftext[x=3.961625in,y=0.236409in,,top]{\sffamily\fontsize{16.000000}{19.200000}\selectfont \(\displaystyle z-position [\mu m]\)}%
\end{pgfscope}%
\begin{pgfscope}%
\pgfsetbuttcap%
\pgfsetroundjoin%
\definecolor{currentfill}{rgb}{0.000000,0.000000,0.000000}%
\pgfsetfillcolor{currentfill}%
\pgfsetlinewidth{0.803000pt}%
\definecolor{currentstroke}{rgb}{0.000000,0.000000,0.000000}%
\pgfsetstrokecolor{currentstroke}%
\pgfsetdash{}{0pt}%
\pgfsys@defobject{currentmarker}{\pgfqpoint{-0.048611in}{0.000000in}}{\pgfqpoint{0.000000in}{0.000000in}}{%
\pgfpathmoveto{\pgfqpoint{0.000000in}{0.000000in}}%
\pgfpathlineto{\pgfqpoint{-0.048611in}{0.000000in}}%
\pgfusepath{stroke,fill}%
}%
\begin{pgfscope}%
\pgfsys@transformshift{0.966250in}{0.523600in}%
\pgfsys@useobject{currentmarker}{}%
\end{pgfscope}%
\end{pgfscope}%
\begin{pgfscope}%
\pgftext[x=0.355044in,y=0.470838in,left,base]{\sffamily\fontsize{10.000000}{12.000000}\selectfont -0.003}%
\end{pgfscope}%
\begin{pgfscope}%
\pgfsetbuttcap%
\pgfsetroundjoin%
\definecolor{currentfill}{rgb}{0.000000,0.000000,0.000000}%
\pgfsetfillcolor{currentfill}%
\pgfsetlinewidth{0.803000pt}%
\definecolor{currentstroke}{rgb}{0.000000,0.000000,0.000000}%
\pgfsetstrokecolor{currentstroke}%
\pgfsetdash{}{0pt}%
\pgfsys@defobject{currentmarker}{\pgfqpoint{-0.048611in}{0.000000in}}{\pgfqpoint{0.000000in}{0.000000in}}{%
\pgfpathmoveto{\pgfqpoint{0.000000in}{0.000000in}}%
\pgfpathlineto{\pgfqpoint{-0.048611in}{0.000000in}}%
\pgfusepath{stroke,fill}%
}%
\begin{pgfscope}%
\pgfsys@transformshift{0.966250in}{1.134467in}%
\pgfsys@useobject{currentmarker}{}%
\end{pgfscope}%
\end{pgfscope}%
\begin{pgfscope}%
\pgftext[x=0.355044in,y=1.081705in,left,base]{\sffamily\fontsize{10.000000}{12.000000}\selectfont -0.002}%
\end{pgfscope}%
\begin{pgfscope}%
\pgfsetbuttcap%
\pgfsetroundjoin%
\definecolor{currentfill}{rgb}{0.000000,0.000000,0.000000}%
\pgfsetfillcolor{currentfill}%
\pgfsetlinewidth{0.803000pt}%
\definecolor{currentstroke}{rgb}{0.000000,0.000000,0.000000}%
\pgfsetstrokecolor{currentstroke}%
\pgfsetdash{}{0pt}%
\pgfsys@defobject{currentmarker}{\pgfqpoint{-0.048611in}{0.000000in}}{\pgfqpoint{0.000000in}{0.000000in}}{%
\pgfpathmoveto{\pgfqpoint{0.000000in}{0.000000in}}%
\pgfpathlineto{\pgfqpoint{-0.048611in}{0.000000in}}%
\pgfusepath{stroke,fill}%
}%
\begin{pgfscope}%
\pgfsys@transformshift{0.966250in}{1.745333in}%
\pgfsys@useobject{currentmarker}{}%
\end{pgfscope}%
\end{pgfscope}%
\begin{pgfscope}%
\pgftext[x=0.355044in,y=1.692572in,left,base]{\sffamily\fontsize{10.000000}{12.000000}\selectfont -0.001}%
\end{pgfscope}%
\begin{pgfscope}%
\pgfsetbuttcap%
\pgfsetroundjoin%
\definecolor{currentfill}{rgb}{0.000000,0.000000,0.000000}%
\pgfsetfillcolor{currentfill}%
\pgfsetlinewidth{0.803000pt}%
\definecolor{currentstroke}{rgb}{0.000000,0.000000,0.000000}%
\pgfsetstrokecolor{currentstroke}%
\pgfsetdash{}{0pt}%
\pgfsys@defobject{currentmarker}{\pgfqpoint{-0.048611in}{0.000000in}}{\pgfqpoint{0.000000in}{0.000000in}}{%
\pgfpathmoveto{\pgfqpoint{0.000000in}{0.000000in}}%
\pgfpathlineto{\pgfqpoint{-0.048611in}{0.000000in}}%
\pgfusepath{stroke,fill}%
}%
\begin{pgfscope}%
\pgfsys@transformshift{0.966250in}{2.356200in}%
\pgfsys@useobject{currentmarker}{}%
\end{pgfscope}%
\end{pgfscope}%
\begin{pgfscope}%
\pgftext[x=0.471418in,y=2.303438in,left,base]{\sffamily\fontsize{10.000000}{12.000000}\selectfont 0.000}%
\end{pgfscope}%
\begin{pgfscope}%
\pgfsetbuttcap%
\pgfsetroundjoin%
\definecolor{currentfill}{rgb}{0.000000,0.000000,0.000000}%
\pgfsetfillcolor{currentfill}%
\pgfsetlinewidth{0.803000pt}%
\definecolor{currentstroke}{rgb}{0.000000,0.000000,0.000000}%
\pgfsetstrokecolor{currentstroke}%
\pgfsetdash{}{0pt}%
\pgfsys@defobject{currentmarker}{\pgfqpoint{-0.048611in}{0.000000in}}{\pgfqpoint{0.000000in}{0.000000in}}{%
\pgfpathmoveto{\pgfqpoint{0.000000in}{0.000000in}}%
\pgfpathlineto{\pgfqpoint{-0.048611in}{0.000000in}}%
\pgfusepath{stroke,fill}%
}%
\begin{pgfscope}%
\pgfsys@transformshift{0.966250in}{2.967067in}%
\pgfsys@useobject{currentmarker}{}%
\end{pgfscope}%
\end{pgfscope}%
\begin{pgfscope}%
\pgftext[x=0.471418in,y=2.914305in,left,base]{\sffamily\fontsize{10.000000}{12.000000}\selectfont 0.001}%
\end{pgfscope}%
\begin{pgfscope}%
\pgfsetbuttcap%
\pgfsetroundjoin%
\definecolor{currentfill}{rgb}{0.000000,0.000000,0.000000}%
\pgfsetfillcolor{currentfill}%
\pgfsetlinewidth{0.803000pt}%
\definecolor{currentstroke}{rgb}{0.000000,0.000000,0.000000}%
\pgfsetstrokecolor{currentstroke}%
\pgfsetdash{}{0pt}%
\pgfsys@defobject{currentmarker}{\pgfqpoint{-0.048611in}{0.000000in}}{\pgfqpoint{0.000000in}{0.000000in}}{%
\pgfpathmoveto{\pgfqpoint{0.000000in}{0.000000in}}%
\pgfpathlineto{\pgfqpoint{-0.048611in}{0.000000in}}%
\pgfusepath{stroke,fill}%
}%
\begin{pgfscope}%
\pgfsys@transformshift{0.966250in}{3.577933in}%
\pgfsys@useobject{currentmarker}{}%
\end{pgfscope}%
\end{pgfscope}%
\begin{pgfscope}%
\pgftext[x=0.471418in,y=3.525172in,left,base]{\sffamily\fontsize{10.000000}{12.000000}\selectfont 0.002}%
\end{pgfscope}%
\begin{pgfscope}%
\pgfsetbuttcap%
\pgfsetroundjoin%
\definecolor{currentfill}{rgb}{0.000000,0.000000,0.000000}%
\pgfsetfillcolor{currentfill}%
\pgfsetlinewidth{0.803000pt}%
\definecolor{currentstroke}{rgb}{0.000000,0.000000,0.000000}%
\pgfsetstrokecolor{currentstroke}%
\pgfsetdash{}{0pt}%
\pgfsys@defobject{currentmarker}{\pgfqpoint{-0.048611in}{0.000000in}}{\pgfqpoint{0.000000in}{0.000000in}}{%
\pgfpathmoveto{\pgfqpoint{0.000000in}{0.000000in}}%
\pgfpathlineto{\pgfqpoint{-0.048611in}{0.000000in}}%
\pgfusepath{stroke,fill}%
}%
\begin{pgfscope}%
\pgfsys@transformshift{0.966250in}{4.188800in}%
\pgfsys@useobject{currentmarker}{}%
\end{pgfscope}%
\end{pgfscope}%
\begin{pgfscope}%
\pgftext[x=0.471418in,y=4.136038in,left,base]{\sffamily\fontsize{10.000000}{12.000000}\selectfont 0.003}%
\end{pgfscope}%
\begin{pgfscope}%
\pgftext[x=0.299488in,y=2.356200in,,bottom,rotate=90.000000]{\sffamily\fontsize{16.000000}{19.200000}\selectfont \(\displaystyle Poynting\) \(\displaystyle vector\)}%
\end{pgfscope}%
\begin{pgfscope}%
\pgfpathrectangle{\pgfqpoint{0.966250in}{0.523600in}}{\pgfqpoint{5.990750in}{3.665200in}} %
\pgfusepath{clip}%
\pgfsetrectcap%
\pgfsetroundjoin%
\pgfsetlinewidth{1.505625pt}%
\definecolor{currentstroke}{rgb}{0.000000,0.000000,0.000000}%
\pgfsetstrokecolor{currentstroke}%
\pgfsetdash{}{0pt}%
\pgfpathmoveto{\pgfqpoint{0.966250in}{2.356200in}}%
\pgfpathlineto{\pgfqpoint{1.051319in}{2.355064in}}%
\pgfpathlineto{\pgfqpoint{1.058907in}{2.351296in}}%
\pgfpathlineto{\pgfqpoint{1.068093in}{2.347025in}}%
\pgfpathlineto{\pgfqpoint{1.072885in}{2.347329in}}%
\pgfpathlineto{\pgfqpoint{1.078077in}{2.350108in}}%
\pgfpathlineto{\pgfqpoint{1.088062in}{2.356133in}}%
\pgfpathlineto{\pgfqpoint{1.091656in}{2.355555in}}%
\pgfpathlineto{\pgfqpoint{1.095251in}{2.352580in}}%
\pgfpathlineto{\pgfqpoint{1.100043in}{2.345359in}}%
\pgfpathlineto{\pgfqpoint{1.110028in}{2.329672in}}%
\pgfpathlineto{\pgfqpoint{1.113223in}{2.328367in}}%
\pgfpathlineto{\pgfqpoint{1.116019in}{2.329567in}}%
\pgfpathlineto{\pgfqpoint{1.119613in}{2.334122in}}%
\pgfpathlineto{\pgfqpoint{1.127601in}{2.349876in}}%
\pgfpathlineto{\pgfqpoint{1.131994in}{2.355556in}}%
\pgfpathlineto{\pgfqpoint{1.134790in}{2.356050in}}%
\pgfpathlineto{\pgfqpoint{1.137186in}{2.354166in}}%
\pgfpathlineto{\pgfqpoint{1.140381in}{2.348447in}}%
\pgfpathlineto{\pgfqpoint{1.145573in}{2.333657in}}%
\pgfpathlineto{\pgfqpoint{1.153161in}{2.313152in}}%
\pgfpathlineto{\pgfqpoint{1.156356in}{2.309801in}}%
\pgfpathlineto{\pgfqpoint{1.158753in}{2.310295in}}%
\pgfpathlineto{\pgfqpoint{1.161149in}{2.313380in}}%
\pgfpathlineto{\pgfqpoint{1.164744in}{2.322095in}}%
\pgfpathlineto{\pgfqpoint{1.176725in}{2.355078in}}%
\pgfpathlineto{\pgfqpoint{1.179121in}{2.356177in}}%
\pgfpathlineto{\pgfqpoint{1.181118in}{2.354828in}}%
\pgfpathlineto{\pgfqpoint{1.183914in}{2.349596in}}%
\pgfpathlineto{\pgfqpoint{1.188307in}{2.335225in}}%
\pgfpathlineto{\pgfqpoint{1.197892in}{2.302129in}}%
\pgfpathlineto{\pgfqpoint{1.201087in}{2.298414in}}%
\pgfpathlineto{\pgfqpoint{1.203084in}{2.298983in}}%
\pgfpathlineto{\pgfqpoint{1.205481in}{2.302574in}}%
\pgfpathlineto{\pgfqpoint{1.209075in}{2.312992in}}%
\pgfpathlineto{\pgfqpoint{1.221855in}{2.355138in}}%
\pgfpathlineto{\pgfqpoint{1.224252in}{2.356126in}}%
\pgfpathlineto{\pgfqpoint{1.226249in}{2.354356in}}%
\pgfpathlineto{\pgfqpoint{1.229044in}{2.348168in}}%
\pgfpathlineto{\pgfqpoint{1.233437in}{2.331930in}}%
\pgfpathlineto{\pgfqpoint{1.242224in}{2.298457in}}%
\pgfpathlineto{\pgfqpoint{1.245419in}{2.293692in}}%
\pgfpathlineto{\pgfqpoint{1.247416in}{2.293812in}}%
\pgfpathlineto{\pgfqpoint{1.249812in}{2.297125in}}%
\pgfpathlineto{\pgfqpoint{1.253007in}{2.306249in}}%
\pgfpathlineto{\pgfqpoint{1.260196in}{2.336347in}}%
\pgfpathlineto{\pgfqpoint{1.265388in}{2.352878in}}%
\pgfpathlineto{\pgfqpoint{1.268184in}{2.356122in}}%
\pgfpathlineto{\pgfqpoint{1.270181in}{2.355502in}}%
\pgfpathlineto{\pgfqpoint{1.272577in}{2.351576in}}%
\pgfpathlineto{\pgfqpoint{1.276171in}{2.340187in}}%
\pgfpathlineto{\pgfqpoint{1.288952in}{2.293354in}}%
\pgfpathlineto{\pgfqpoint{1.291348in}{2.291816in}}%
\pgfpathlineto{\pgfqpoint{1.293345in}{2.293277in}}%
\pgfpathlineto{\pgfqpoint{1.296141in}{2.299267in}}%
\pgfpathlineto{\pgfqpoint{1.300534in}{2.315688in}}%
\pgfpathlineto{\pgfqpoint{1.309720in}{2.351525in}}%
\pgfpathlineto{\pgfqpoint{1.312915in}{2.356046in}}%
\pgfpathlineto{\pgfqpoint{1.314912in}{2.355669in}}%
\pgfpathlineto{\pgfqpoint{1.317308in}{2.351978in}}%
\pgfpathlineto{\pgfqpoint{1.320902in}{2.340793in}}%
\pgfpathlineto{\pgfqpoint{1.334082in}{2.292650in}}%
\pgfpathlineto{\pgfqpoint{1.336478in}{2.291436in}}%
\pgfpathlineto{\pgfqpoint{1.338475in}{2.293181in}}%
\pgfpathlineto{\pgfqpoint{1.341271in}{2.299545in}}%
\pgfpathlineto{\pgfqpoint{1.345664in}{2.316375in}}%
\pgfpathlineto{\pgfqpoint{1.354451in}{2.351029in}}%
\pgfpathlineto{\pgfqpoint{1.357646in}{2.355944in}}%
\pgfpathlineto{\pgfqpoint{1.359643in}{2.355818in}}%
\pgfpathlineto{\pgfqpoint{1.361640in}{2.353209in}}%
\pgfpathlineto{\pgfqpoint{1.364835in}{2.344439in}}%
\pgfpathlineto{\pgfqpoint{1.370825in}{2.319020in}}%
\pgfpathlineto{\pgfqpoint{1.376816in}{2.296734in}}%
\pgfpathlineto{\pgfqpoint{1.380011in}{2.291627in}}%
\pgfpathlineto{\pgfqpoint{1.382008in}{2.291626in}}%
\pgfpathlineto{\pgfqpoint{1.384005in}{2.294114in}}%
\pgfpathlineto{\pgfqpoint{1.387200in}{2.302725in}}%
\pgfpathlineto{\pgfqpoint{1.392791in}{2.326242in}}%
\pgfpathlineto{\pgfqpoint{1.399182in}{2.350525in}}%
\pgfpathlineto{\pgfqpoint{1.402377in}{2.355818in}}%
\pgfpathlineto{\pgfqpoint{1.404374in}{2.355944in}}%
\pgfpathlineto{\pgfqpoint{1.406370in}{2.353576in}}%
\pgfpathlineto{\pgfqpoint{1.409566in}{2.345125in}}%
\pgfpathlineto{\pgfqpoint{1.415157in}{2.321714in}}%
\pgfpathlineto{\pgfqpoint{1.421547in}{2.297236in}}%
\pgfpathlineto{\pgfqpoint{1.424742in}{2.291757in}}%
\pgfpathlineto{\pgfqpoint{1.426739in}{2.291505in}}%
\pgfpathlineto{\pgfqpoint{1.428736in}{2.293751in}}%
\pgfpathlineto{\pgfqpoint{1.431532in}{2.300726in}}%
\pgfpathlineto{\pgfqpoint{1.436324in}{2.319909in}}%
\pgfpathlineto{\pgfqpoint{1.444312in}{2.351027in}}%
\pgfpathlineto{\pgfqpoint{1.447507in}{2.355944in}}%
\pgfpathlineto{\pgfqpoint{1.449504in}{2.355818in}}%
\pgfpathlineto{\pgfqpoint{1.451501in}{2.353207in}}%
\pgfpathlineto{\pgfqpoint{1.454696in}{2.344435in}}%
\pgfpathlineto{\pgfqpoint{1.460687in}{2.319011in}}%
\pgfpathlineto{\pgfqpoint{1.466677in}{2.296724in}}%
\pgfpathlineto{\pgfqpoint{1.469872in}{2.291618in}}%
\pgfpathlineto{\pgfqpoint{1.471869in}{2.291618in}}%
\pgfpathlineto{\pgfqpoint{1.473866in}{2.294108in}}%
\pgfpathlineto{\pgfqpoint{1.477061in}{2.302720in}}%
\pgfpathlineto{\pgfqpoint{1.482653in}{2.326241in}}%
\pgfpathlineto{\pgfqpoint{1.489043in}{2.350525in}}%
\pgfpathlineto{\pgfqpoint{1.492238in}{2.355818in}}%
\pgfpathlineto{\pgfqpoint{1.494235in}{2.355944in}}%
\pgfpathlineto{\pgfqpoint{1.496232in}{2.353576in}}%
\pgfpathlineto{\pgfqpoint{1.499427in}{2.345125in}}%
\pgfpathlineto{\pgfqpoint{1.505018in}{2.321714in}}%
\pgfpathlineto{\pgfqpoint{1.511408in}{2.297236in}}%
\pgfpathlineto{\pgfqpoint{1.514603in}{2.291757in}}%
\pgfpathlineto{\pgfqpoint{1.516600in}{2.291504in}}%
\pgfpathlineto{\pgfqpoint{1.518597in}{2.293751in}}%
\pgfpathlineto{\pgfqpoint{1.521393in}{2.300726in}}%
\pgfpathlineto{\pgfqpoint{1.526185in}{2.319909in}}%
\pgfpathlineto{\pgfqpoint{1.534173in}{2.351027in}}%
\pgfpathlineto{\pgfqpoint{1.537368in}{2.355944in}}%
\pgfpathlineto{\pgfqpoint{1.539365in}{2.355818in}}%
\pgfpathlineto{\pgfqpoint{1.541362in}{2.353207in}}%
\pgfpathlineto{\pgfqpoint{1.544557in}{2.344435in}}%
\pgfpathlineto{\pgfqpoint{1.550548in}{2.319011in}}%
\pgfpathlineto{\pgfqpoint{1.556539in}{2.296724in}}%
\pgfpathlineto{\pgfqpoint{1.559734in}{2.291618in}}%
\pgfpathlineto{\pgfqpoint{1.561731in}{2.291618in}}%
\pgfpathlineto{\pgfqpoint{1.563727in}{2.294108in}}%
\pgfpathlineto{\pgfqpoint{1.566923in}{2.302720in}}%
\pgfpathlineto{\pgfqpoint{1.572514in}{2.326241in}}%
\pgfpathlineto{\pgfqpoint{1.578904in}{2.350525in}}%
\pgfpathlineto{\pgfqpoint{1.582099in}{2.355818in}}%
\pgfpathlineto{\pgfqpoint{1.584096in}{2.355944in}}%
\pgfpathlineto{\pgfqpoint{1.586093in}{2.353576in}}%
\pgfpathlineto{\pgfqpoint{1.589288in}{2.345125in}}%
\pgfpathlineto{\pgfqpoint{1.594879in}{2.321714in}}%
\pgfpathlineto{\pgfqpoint{1.601270in}{2.297236in}}%
\pgfpathlineto{\pgfqpoint{1.604465in}{2.291757in}}%
\pgfpathlineto{\pgfqpoint{1.606461in}{2.291505in}}%
\pgfpathlineto{\pgfqpoint{1.608458in}{2.293751in}}%
\pgfpathlineto{\pgfqpoint{1.611254in}{2.300727in}}%
\pgfpathlineto{\pgfqpoint{1.616047in}{2.319909in}}%
\pgfpathlineto{\pgfqpoint{1.624034in}{2.351027in}}%
\pgfpathlineto{\pgfqpoint{1.627229in}{2.355944in}}%
\pgfpathlineto{\pgfqpoint{1.629226in}{2.355818in}}%
\pgfpathlineto{\pgfqpoint{1.631223in}{2.353208in}}%
\pgfpathlineto{\pgfqpoint{1.634418in}{2.344435in}}%
\pgfpathlineto{\pgfqpoint{1.640409in}{2.319013in}}%
\pgfpathlineto{\pgfqpoint{1.646400in}{2.296730in}}%
\pgfpathlineto{\pgfqpoint{1.649595in}{2.291626in}}%
\pgfpathlineto{\pgfqpoint{1.651592in}{2.291627in}}%
\pgfpathlineto{\pgfqpoint{1.653589in}{2.294117in}}%
\pgfpathlineto{\pgfqpoint{1.656784in}{2.302730in}}%
\pgfpathlineto{\pgfqpoint{1.662375in}{2.326248in}}%
\pgfpathlineto{\pgfqpoint{1.668765in}{2.350527in}}%
\pgfpathlineto{\pgfqpoint{1.671960in}{2.355818in}}%
\pgfpathlineto{\pgfqpoint{1.673957in}{2.355944in}}%
\pgfpathlineto{\pgfqpoint{1.675954in}{2.353577in}}%
\pgfpathlineto{\pgfqpoint{1.679149in}{2.345132in}}%
\pgfpathlineto{\pgfqpoint{1.684741in}{2.321741in}}%
\pgfpathlineto{\pgfqpoint{1.691131in}{2.297299in}}%
\pgfpathlineto{\pgfqpoint{1.694326in}{2.291837in}}%
\pgfpathlineto{\pgfqpoint{1.696323in}{2.291592in}}%
\pgfpathlineto{\pgfqpoint{1.698320in}{2.293843in}}%
\pgfpathlineto{\pgfqpoint{1.701115in}{2.300819in}}%
\pgfpathlineto{\pgfqpoint{1.705908in}{2.319984in}}%
\pgfpathlineto{\pgfqpoint{1.713896in}{2.351041in}}%
\pgfpathlineto{\pgfqpoint{1.717091in}{2.355945in}}%
\pgfpathlineto{\pgfqpoint{1.719088in}{2.355819in}}%
\pgfpathlineto{\pgfqpoint{1.721085in}{2.353219in}}%
\pgfpathlineto{\pgfqpoint{1.724280in}{2.344485in}}%
\pgfpathlineto{\pgfqpoint{1.730270in}{2.319211in}}%
\pgfpathlineto{\pgfqpoint{1.736261in}{2.297122in}}%
\pgfpathlineto{\pgfqpoint{1.739456in}{2.292103in}}%
\pgfpathlineto{\pgfqpoint{1.741453in}{2.292139in}}%
\pgfpathlineto{\pgfqpoint{1.743450in}{2.294645in}}%
\pgfpathlineto{\pgfqpoint{1.746645in}{2.303237in}}%
\pgfpathlineto{\pgfqpoint{1.752636in}{2.328376in}}%
\pgfpathlineto{\pgfqpoint{1.758627in}{2.350607in}}%
\pgfpathlineto{\pgfqpoint{1.761822in}{2.355824in}}%
\pgfpathlineto{\pgfqpoint{1.763819in}{2.355948in}}%
\pgfpathlineto{\pgfqpoint{1.765815in}{2.353624in}}%
\pgfpathlineto{\pgfqpoint{1.769011in}{2.345348in}}%
\pgfpathlineto{\pgfqpoint{1.774602in}{2.322539in}}%
\pgfpathlineto{\pgfqpoint{1.780992in}{2.298947in}}%
\pgfpathlineto{\pgfqpoint{1.784187in}{2.293812in}}%
\pgfpathlineto{\pgfqpoint{1.786184in}{2.293692in}}%
\pgfpathlineto{\pgfqpoint{1.788181in}{2.295988in}}%
\pgfpathlineto{\pgfqpoint{1.791376in}{2.304165in}}%
\pgfpathlineto{\pgfqpoint{1.796967in}{2.326754in}}%
\pgfpathlineto{\pgfqpoint{1.803357in}{2.350342in}}%
\pgfpathlineto{\pgfqpoint{1.806553in}{2.355699in}}%
\pgfpathlineto{\pgfqpoint{1.808549in}{2.356055in}}%
\pgfpathlineto{\pgfqpoint{1.810546in}{2.354072in}}%
\pgfpathlineto{\pgfqpoint{1.813342in}{2.347742in}}%
\pgfpathlineto{\pgfqpoint{1.818135in}{2.330312in}}%
\pgfpathlineto{\pgfqpoint{1.825723in}{2.303460in}}%
\pgfpathlineto{\pgfqpoint{1.828918in}{2.298688in}}%
\pgfpathlineto{\pgfqpoint{1.830915in}{2.298573in}}%
\pgfpathlineto{\pgfqpoint{1.833311in}{2.301370in}}%
\pgfpathlineto{\pgfqpoint{1.836506in}{2.309463in}}%
\pgfpathlineto{\pgfqpoint{1.843296in}{2.335225in}}%
\pgfpathlineto{\pgfqpoint{1.848887in}{2.352290in}}%
\pgfpathlineto{\pgfqpoint{1.852082in}{2.356073in}}%
\pgfpathlineto{\pgfqpoint{1.854079in}{2.355766in}}%
\pgfpathlineto{\pgfqpoint{1.856475in}{2.352796in}}%
\pgfpathlineto{\pgfqpoint{1.860070in}{2.344020in}}%
\pgfpathlineto{\pgfqpoint{1.872051in}{2.311045in}}%
\pgfpathlineto{\pgfqpoint{1.874847in}{2.309699in}}%
\pgfpathlineto{\pgfqpoint{1.877243in}{2.311387in}}%
\pgfpathlineto{\pgfqpoint{1.880438in}{2.317274in}}%
\pgfpathlineto{\pgfqpoint{1.886030in}{2.333657in}}%
\pgfpathlineto{\pgfqpoint{1.892819in}{2.351734in}}%
\pgfpathlineto{\pgfqpoint{1.896414in}{2.355887in}}%
\pgfpathlineto{\pgfqpoint{1.898810in}{2.355981in}}%
\pgfpathlineto{\pgfqpoint{1.901606in}{2.353600in}}%
\pgfpathlineto{\pgfqpoint{1.905999in}{2.346011in}}%
\pgfpathlineto{\pgfqpoint{1.914785in}{2.330307in}}%
\pgfpathlineto{\pgfqpoint{1.917980in}{2.328403in}}%
\pgfpathlineto{\pgfqpoint{1.920776in}{2.329091in}}%
\pgfpathlineto{\pgfqpoint{1.924371in}{2.332851in}}%
\pgfpathlineto{\pgfqpoint{1.930761in}{2.343924in}}%
\pgfpathlineto{\pgfqpoint{1.937151in}{2.353448in}}%
\pgfpathlineto{\pgfqpoint{1.941145in}{2.356005in}}%
\pgfpathlineto{\pgfqpoint{1.944739in}{2.355851in}}%
\pgfpathlineto{\pgfqpoint{1.949532in}{2.353116in}}%
\pgfpathlineto{\pgfqpoint{1.959117in}{2.347213in}}%
\pgfpathlineto{\pgfqpoint{1.963910in}{2.347102in}}%
\pgfpathlineto{\pgfqpoint{1.969501in}{2.349414in}}%
\pgfpathlineto{\pgfqpoint{1.982681in}{2.355719in}}%
\pgfpathlineto{\pgfqpoint{1.989869in}{2.356123in}}%
\pgfpathlineto{\pgfqpoint{2.014232in}{2.355594in}}%
\pgfpathlineto{\pgfqpoint{2.038594in}{2.356187in}}%
\pgfpathlineto{\pgfqpoint{2.445166in}{2.356200in}}%
\pgfpathlineto{\pgfqpoint{3.587802in}{2.357208in}}%
\pgfpathlineto{\pgfqpoint{3.590997in}{2.360446in}}%
\pgfpathlineto{\pgfqpoint{3.594991in}{2.368446in}}%
\pgfpathlineto{\pgfqpoint{3.601781in}{2.382565in}}%
\pgfpathlineto{\pgfqpoint{3.603778in}{2.382573in}}%
\pgfpathlineto{\pgfqpoint{3.605774in}{2.379417in}}%
\pgfpathlineto{\pgfqpoint{3.608970in}{2.368911in}}%
\pgfpathlineto{\pgfqpoint{3.613363in}{2.356279in}}%
\pgfpathlineto{\pgfqpoint{3.614561in}{2.356868in}}%
\pgfpathlineto{\pgfqpoint{3.616158in}{2.362474in}}%
\pgfpathlineto{\pgfqpoint{3.618555in}{2.383432in}}%
\pgfpathlineto{\pgfqpoint{3.621750in}{2.435155in}}%
\pgfpathlineto{\pgfqpoint{3.629737in}{2.576829in}}%
\pgfpathlineto{\pgfqpoint{3.630936in}{2.580674in}}%
\pgfpathlineto{\pgfqpoint{3.631335in}{2.580232in}}%
\pgfpathlineto{\pgfqpoint{3.632533in}{2.573554in}}%
\pgfpathlineto{\pgfqpoint{3.634530in}{2.545164in}}%
\pgfpathlineto{\pgfqpoint{3.638524in}{2.444795in}}%
\pgfpathlineto{\pgfqpoint{3.642518in}{2.361688in}}%
\pgfpathlineto{\pgfqpoint{3.643716in}{2.356091in}}%
\pgfpathlineto{\pgfqpoint{3.644115in}{2.357055in}}%
\pgfpathlineto{\pgfqpoint{3.645313in}{2.369250in}}%
\pgfpathlineto{\pgfqpoint{3.647310in}{2.422058in}}%
\pgfpathlineto{\pgfqpoint{3.650505in}{2.582889in}}%
\pgfpathlineto{\pgfqpoint{3.658493in}{3.012655in}}%
\pgfpathlineto{\pgfqpoint{3.659691in}{3.024574in}}%
\pgfpathlineto{\pgfqpoint{3.660091in}{3.023713in}}%
\pgfpathlineto{\pgfqpoint{3.661289in}{3.006400in}}%
\pgfpathlineto{\pgfqpoint{3.663286in}{2.930840in}}%
\pgfpathlineto{\pgfqpoint{3.667279in}{2.660585in}}%
\pgfpathlineto{\pgfqpoint{3.672072in}{2.378514in}}%
\pgfpathlineto{\pgfqpoint{3.673670in}{2.355964in}}%
\pgfpathlineto{\pgfqpoint{3.674468in}{2.363396in}}%
\pgfpathlineto{\pgfqpoint{3.676066in}{2.417167in}}%
\pgfpathlineto{\pgfqpoint{3.678862in}{2.625345in}}%
\pgfpathlineto{\pgfqpoint{3.688447in}{3.467673in}}%
\pgfpathlineto{\pgfqpoint{3.689246in}{3.472384in}}%
\pgfpathlineto{\pgfqpoint{3.689645in}{3.468765in}}%
\pgfpathlineto{\pgfqpoint{3.690843in}{3.434209in}}%
\pgfpathlineto{\pgfqpoint{3.693239in}{3.267685in}}%
\pgfpathlineto{\pgfqpoint{3.703623in}{2.355869in}}%
\pgfpathlineto{\pgfqpoint{3.704422in}{2.366286in}}%
\pgfpathlineto{\pgfqpoint{3.706020in}{2.440245in}}%
\pgfpathlineto{\pgfqpoint{3.708815in}{2.717813in}}%
\pgfpathlineto{\pgfqpoint{3.717602in}{3.724408in}}%
\pgfpathlineto{\pgfqpoint{3.718800in}{3.743927in}}%
\pgfpathlineto{\pgfqpoint{3.719199in}{3.740664in}}%
\pgfpathlineto{\pgfqpoint{3.720397in}{3.701772in}}%
\pgfpathlineto{\pgfqpoint{3.722394in}{3.547024in}}%
\pgfpathlineto{\pgfqpoint{3.726388in}{3.016310in}}%
\pgfpathlineto{\pgfqpoint{3.731580in}{2.414617in}}%
\pgfpathlineto{\pgfqpoint{3.733577in}{2.355823in}}%
\pgfpathlineto{\pgfqpoint{3.734376in}{2.367737in}}%
\pgfpathlineto{\pgfqpoint{3.735973in}{2.451551in}}%
\pgfpathlineto{\pgfqpoint{3.738769in}{2.761701in}}%
\pgfpathlineto{\pgfqpoint{3.747156in}{3.829016in}}%
\pgfpathlineto{\pgfqpoint{3.748754in}{3.860664in}}%
\pgfpathlineto{\pgfqpoint{3.749552in}{3.844881in}}%
\pgfpathlineto{\pgfqpoint{3.751150in}{3.752197in}}%
\pgfpathlineto{\pgfqpoint{3.753946in}{3.427409in}}%
\pgfpathlineto{\pgfqpoint{3.761933in}{2.394854in}}%
\pgfpathlineto{\pgfqpoint{3.763531in}{2.355807in}}%
\pgfpathlineto{\pgfqpoint{3.764330in}{2.368301in}}%
\pgfpathlineto{\pgfqpoint{3.765927in}{2.455782in}}%
\pgfpathlineto{\pgfqpoint{3.768723in}{2.777632in}}%
\pgfpathlineto{\pgfqpoint{3.777110in}{3.869650in}}%
\pgfpathlineto{\pgfqpoint{3.778308in}{3.900054in}}%
\pgfpathlineto{\pgfqpoint{3.778707in}{3.899407in}}%
\pgfpathlineto{\pgfqpoint{3.779906in}{3.865177in}}%
\pgfpathlineto{\pgfqpoint{3.781903in}{3.707441in}}%
\pgfpathlineto{\pgfqpoint{3.785896in}{3.135478in}}%
\pgfpathlineto{\pgfqpoint{3.791088in}{2.446461in}}%
\pgfpathlineto{\pgfqpoint{3.793485in}{2.355804in}}%
\pgfpathlineto{\pgfqpoint{3.794283in}{2.368490in}}%
\pgfpathlineto{\pgfqpoint{3.795881in}{2.457065in}}%
\pgfpathlineto{\pgfqpoint{3.798677in}{2.782221in}}%
\pgfpathlineto{\pgfqpoint{3.807064in}{3.880231in}}%
\pgfpathlineto{\pgfqpoint{3.808262in}{3.910167in}}%
\pgfpathlineto{\pgfqpoint{3.808661in}{3.909297in}}%
\pgfpathlineto{\pgfqpoint{3.809859in}{3.874225in}}%
\pgfpathlineto{\pgfqpoint{3.811856in}{3.714667in}}%
\pgfpathlineto{\pgfqpoint{3.815850in}{3.138719in}}%
\pgfpathlineto{\pgfqpoint{3.821042in}{2.446670in}}%
\pgfpathlineto{\pgfqpoint{3.823438in}{2.355806in}}%
\pgfpathlineto{\pgfqpoint{3.824237in}{2.368559in}}%
\pgfpathlineto{\pgfqpoint{3.825835in}{2.457415in}}%
\pgfpathlineto{\pgfqpoint{3.828630in}{2.783307in}}%
\pgfpathlineto{\pgfqpoint{3.837017in}{3.882236in}}%
\pgfpathlineto{\pgfqpoint{3.838216in}{3.912015in}}%
\pgfpathlineto{\pgfqpoint{3.838615in}{3.911081in}}%
\pgfpathlineto{\pgfqpoint{3.839813in}{3.875788in}}%
\pgfpathlineto{\pgfqpoint{3.841810in}{3.715810in}}%
\pgfpathlineto{\pgfqpoint{3.845804in}{3.139083in}}%
\pgfpathlineto{\pgfqpoint{3.850996in}{2.446622in}}%
\pgfpathlineto{\pgfqpoint{3.853392in}{2.355808in}}%
\pgfpathlineto{\pgfqpoint{3.854191in}{2.368599in}}%
\pgfpathlineto{\pgfqpoint{3.855788in}{2.457548in}}%
\pgfpathlineto{\pgfqpoint{3.858584in}{2.783603in}}%
\pgfpathlineto{\pgfqpoint{3.866971in}{3.882505in}}%
\pgfpathlineto{\pgfqpoint{3.868169in}{3.912218in}}%
\pgfpathlineto{\pgfqpoint{3.868569in}{3.911260in}}%
\pgfpathlineto{\pgfqpoint{3.869767in}{3.875896in}}%
\pgfpathlineto{\pgfqpoint{3.871764in}{3.715804in}}%
\pgfpathlineto{\pgfqpoint{3.875758in}{3.138934in}}%
\pgfpathlineto{\pgfqpoint{3.880950in}{2.446532in}}%
\pgfpathlineto{\pgfqpoint{3.883346in}{2.355811in}}%
\pgfpathlineto{\pgfqpoint{3.884145in}{2.368636in}}%
\pgfpathlineto{\pgfqpoint{3.885742in}{2.457648in}}%
\pgfpathlineto{\pgfqpoint{3.888538in}{2.783787in}}%
\pgfpathlineto{\pgfqpoint{3.896925in}{3.882569in}}%
\pgfpathlineto{\pgfqpoint{3.898123in}{3.912231in}}%
\pgfpathlineto{\pgfqpoint{3.898522in}{3.911257in}}%
\pgfpathlineto{\pgfqpoint{3.899721in}{3.875842in}}%
\pgfpathlineto{\pgfqpoint{3.901718in}{3.715676in}}%
\pgfpathlineto{\pgfqpoint{3.905711in}{3.138737in}}%
\pgfpathlineto{\pgfqpoint{3.910903in}{2.446439in}}%
\pgfpathlineto{\pgfqpoint{3.913300in}{2.355814in}}%
\pgfpathlineto{\pgfqpoint{3.914098in}{2.368672in}}%
\pgfpathlineto{\pgfqpoint{3.915696in}{2.457747in}}%
\pgfpathlineto{\pgfqpoint{3.918492in}{2.783964in}}%
\pgfpathlineto{\pgfqpoint{3.926879in}{3.882623in}}%
\pgfpathlineto{\pgfqpoint{3.928077in}{3.912237in}}%
\pgfpathlineto{\pgfqpoint{3.928476in}{3.911245in}}%
\pgfpathlineto{\pgfqpoint{3.929674in}{3.875782in}}%
\pgfpathlineto{\pgfqpoint{3.931671in}{3.715544in}}%
\pgfpathlineto{\pgfqpoint{3.935665in}{3.138538in}}%
\pgfpathlineto{\pgfqpoint{3.940857in}{2.446346in}}%
\pgfpathlineto{\pgfqpoint{3.943253in}{2.355817in}}%
\pgfpathlineto{\pgfqpoint{3.944052in}{2.368708in}}%
\pgfpathlineto{\pgfqpoint{3.945650in}{2.457845in}}%
\pgfpathlineto{\pgfqpoint{3.948445in}{2.784142in}}%
\pgfpathlineto{\pgfqpoint{3.956832in}{3.882678in}}%
\pgfpathlineto{\pgfqpoint{3.958031in}{3.912242in}}%
\pgfpathlineto{\pgfqpoint{3.958430in}{3.911234in}}%
\pgfpathlineto{\pgfqpoint{3.959628in}{3.875722in}}%
\pgfpathlineto{\pgfqpoint{3.961625in}{3.715412in}}%
\pgfpathlineto{\pgfqpoint{3.965619in}{3.138340in}}%
\pgfpathlineto{\pgfqpoint{3.970811in}{2.446253in}}%
\pgfpathlineto{\pgfqpoint{3.973207in}{2.355820in}}%
\pgfpathlineto{\pgfqpoint{3.974006in}{2.368744in}}%
\pgfpathlineto{\pgfqpoint{3.975603in}{2.457943in}}%
\pgfpathlineto{\pgfqpoint{3.978399in}{2.784319in}}%
\pgfpathlineto{\pgfqpoint{3.986786in}{3.882726in}}%
\pgfpathlineto{\pgfqpoint{3.987984in}{3.912239in}}%
\pgfpathlineto{\pgfqpoint{3.988384in}{3.911214in}}%
\pgfpathlineto{\pgfqpoint{3.989582in}{3.875652in}}%
\pgfpathlineto{\pgfqpoint{3.991579in}{3.715269in}}%
\pgfpathlineto{\pgfqpoint{3.995573in}{3.138131in}}%
\pgfpathlineto{\pgfqpoint{4.000765in}{2.446158in}}%
\pgfpathlineto{\pgfqpoint{4.003161in}{2.355823in}}%
\pgfpathlineto{\pgfqpoint{4.003960in}{2.368780in}}%
\pgfpathlineto{\pgfqpoint{4.005557in}{2.458038in}}%
\pgfpathlineto{\pgfqpoint{4.008353in}{2.784475in}}%
\pgfpathlineto{\pgfqpoint{4.016740in}{3.882608in}}%
\pgfpathlineto{\pgfqpoint{4.017938in}{3.912049in}}%
\pgfpathlineto{\pgfqpoint{4.018337in}{3.911001in}}%
\pgfpathlineto{\pgfqpoint{4.019536in}{3.875372in}}%
\pgfpathlineto{\pgfqpoint{4.021533in}{3.714904in}}%
\pgfpathlineto{\pgfqpoint{4.025526in}{3.137746in}}%
\pgfpathlineto{\pgfqpoint{4.030718in}{2.446032in}}%
\pgfpathlineto{\pgfqpoint{4.032715in}{2.357507in}}%
\pgfpathlineto{\pgfqpoint{4.033115in}{2.355827in}}%
\pgfpathlineto{\pgfqpoint{4.033913in}{2.368811in}}%
\pgfpathlineto{\pgfqpoint{4.035511in}{2.458086in}}%
\pgfpathlineto{\pgfqpoint{4.038307in}{2.784390in}}%
\pgfpathlineto{\pgfqpoint{4.046694in}{3.881003in}}%
\pgfpathlineto{\pgfqpoint{4.047892in}{3.910225in}}%
\pgfpathlineto{\pgfqpoint{4.048291in}{3.909114in}}%
\pgfpathlineto{\pgfqpoint{4.049489in}{3.873333in}}%
\pgfpathlineto{\pgfqpoint{4.051486in}{3.712764in}}%
\pgfpathlineto{\pgfqpoint{4.055480in}{3.136068in}}%
\pgfpathlineto{\pgfqpoint{4.060672in}{2.445706in}}%
\pgfpathlineto{\pgfqpoint{4.062669in}{2.357490in}}%
\pgfpathlineto{\pgfqpoint{4.063068in}{2.355831in}}%
\pgfpathlineto{\pgfqpoint{4.063867in}{2.368810in}}%
\pgfpathlineto{\pgfqpoint{4.065465in}{2.457847in}}%
\pgfpathlineto{\pgfqpoint{4.068260in}{2.782900in}}%
\pgfpathlineto{\pgfqpoint{4.076647in}{3.871779in}}%
\pgfpathlineto{\pgfqpoint{4.077846in}{3.900178in}}%
\pgfpathlineto{\pgfqpoint{4.078245in}{3.898851in}}%
\pgfpathlineto{\pgfqpoint{4.079443in}{3.862633in}}%
\pgfpathlineto{\pgfqpoint{4.081440in}{3.702120in}}%
\pgfpathlineto{\pgfqpoint{4.085434in}{3.128523in}}%
\pgfpathlineto{\pgfqpoint{4.090626in}{2.444530in}}%
\pgfpathlineto{\pgfqpoint{4.092623in}{2.357458in}}%
\pgfpathlineto{\pgfqpoint{4.093022in}{2.355840in}}%
\pgfpathlineto{\pgfqpoint{4.093821in}{2.368675in}}%
\pgfpathlineto{\pgfqpoint{4.095418in}{2.456440in}}%
\pgfpathlineto{\pgfqpoint{4.098214in}{2.775890in}}%
\pgfpathlineto{\pgfqpoint{4.106601in}{3.835183in}}%
\pgfpathlineto{\pgfqpoint{4.107799in}{3.860936in}}%
\pgfpathlineto{\pgfqpoint{4.108199in}{3.858973in}}%
\pgfpathlineto{\pgfqpoint{4.109397in}{3.821671in}}%
\pgfpathlineto{\pgfqpoint{4.111394in}{3.662345in}}%
\pgfpathlineto{\pgfqpoint{4.115388in}{3.101628in}}%
\pgfpathlineto{\pgfqpoint{4.120580in}{2.440690in}}%
\pgfpathlineto{\pgfqpoint{4.122576in}{2.357384in}}%
\pgfpathlineto{\pgfqpoint{4.122976in}{2.355861in}}%
\pgfpathlineto{\pgfqpoint{4.123775in}{2.368126in}}%
\pgfpathlineto{\pgfqpoint{4.125372in}{2.451499in}}%
\pgfpathlineto{\pgfqpoint{4.128168in}{2.752632in}}%
\pgfpathlineto{\pgfqpoint{4.136156in}{3.707979in}}%
\pgfpathlineto{\pgfqpoint{4.137753in}{3.742983in}}%
\pgfpathlineto{\pgfqpoint{4.138552in}{3.731226in}}%
\pgfpathlineto{\pgfqpoint{4.140149in}{3.651169in}}%
\pgfpathlineto{\pgfqpoint{4.142945in}{3.360597in}}%
\pgfpathlineto{\pgfqpoint{4.151332in}{2.388017in}}%
\pgfpathlineto{\pgfqpoint{4.152930in}{2.355909in}}%
\pgfpathlineto{\pgfqpoint{4.153728in}{2.366606in}}%
\pgfpathlineto{\pgfqpoint{4.155326in}{2.438473in}}%
\pgfpathlineto{\pgfqpoint{4.158122in}{2.693350in}}%
\pgfpathlineto{\pgfqpoint{4.165710in}{3.438622in}}%
\pgfpathlineto{\pgfqpoint{4.167307in}{3.470839in}}%
\pgfpathlineto{\pgfqpoint{4.168106in}{3.463164in}}%
\pgfpathlineto{\pgfqpoint{4.169704in}{3.401995in}}%
\pgfpathlineto{\pgfqpoint{4.172499in}{3.173598in}}%
\pgfpathlineto{\pgfqpoint{4.180886in}{2.392834in}}%
\pgfpathlineto{\pgfqpoint{4.182883in}{2.355998in}}%
\pgfpathlineto{\pgfqpoint{4.183682in}{2.363535in}}%
\pgfpathlineto{\pgfqpoint{4.185280in}{2.413044in}}%
\pgfpathlineto{\pgfqpoint{4.188475in}{2.611047in}}%
\pgfpathlineto{\pgfqpoint{4.194465in}{2.985641in}}%
\pgfpathlineto{\pgfqpoint{4.196462in}{3.021769in}}%
\pgfpathlineto{\pgfqpoint{4.196862in}{3.021691in}}%
\pgfpathlineto{\pgfqpoint{4.197661in}{3.014307in}}%
\pgfpathlineto{\pgfqpoint{4.199258in}{2.972371in}}%
\pgfpathlineto{\pgfqpoint{4.202453in}{2.805620in}}%
\pgfpathlineto{\pgfqpoint{4.209642in}{2.403287in}}%
\pgfpathlineto{\pgfqpoint{4.212038in}{2.358429in}}%
\pgfpathlineto{\pgfqpoint{4.212837in}{2.356108in}}%
\pgfpathlineto{\pgfqpoint{4.213237in}{2.357122in}}%
\pgfpathlineto{\pgfqpoint{4.214435in}{2.367982in}}%
\pgfpathlineto{\pgfqpoint{4.216831in}{2.416423in}}%
\pgfpathlineto{\pgfqpoint{4.224020in}{2.573058in}}%
\pgfpathlineto{\pgfqpoint{4.225617in}{2.578768in}}%
\pgfpathlineto{\pgfqpoint{4.226416in}{2.576412in}}%
\pgfpathlineto{\pgfqpoint{4.228014in}{2.562272in}}%
\pgfpathlineto{\pgfqpoint{4.231209in}{2.506732in}}%
\pgfpathlineto{\pgfqpoint{4.238398in}{2.377109in}}%
\pgfpathlineto{\pgfqpoint{4.241193in}{2.358438in}}%
\pgfpathlineto{\pgfqpoint{4.242791in}{2.356181in}}%
\pgfpathlineto{\pgfqpoint{4.243989in}{2.357473in}}%
\pgfpathlineto{\pgfqpoint{4.246385in}{2.364673in}}%
\pgfpathlineto{\pgfqpoint{4.251977in}{2.381429in}}%
\pgfpathlineto{\pgfqpoint{4.253974in}{2.382429in}}%
\pgfpathlineto{\pgfqpoint{4.255971in}{2.380584in}}%
\pgfpathlineto{\pgfqpoint{4.259565in}{2.372863in}}%
\pgfpathlineto{\pgfqpoint{4.265955in}{2.359675in}}%
\pgfpathlineto{\pgfqpoint{4.269550in}{2.356745in}}%
\pgfpathlineto{\pgfqpoint{4.273943in}{2.356246in}}%
\pgfpathlineto{\pgfqpoint{4.289918in}{2.356371in}}%
\pgfpathlineto{\pgfqpoint{4.320671in}{2.356200in}}%
\pgfpathlineto{\pgfqpoint{6.956601in}{2.356200in}}%
\pgfpathlineto{\pgfqpoint{6.956601in}{2.356200in}}%
\pgfusepath{stroke}%
\end{pgfscope}%
\begin{pgfscope}%
\pgfsetrectcap%
\pgfsetmiterjoin%
\pgfsetlinewidth{0.803000pt}%
\definecolor{currentstroke}{rgb}{0.000000,0.000000,0.000000}%
\pgfsetstrokecolor{currentstroke}%
\pgfsetdash{}{0pt}%
\pgfpathmoveto{\pgfqpoint{0.966250in}{0.523600in}}%
\pgfpathlineto{\pgfqpoint{0.966250in}{4.188800in}}%
\pgfusepath{stroke}%
\end{pgfscope}%
\begin{pgfscope}%
\pgfsetrectcap%
\pgfsetmiterjoin%
\pgfsetlinewidth{0.803000pt}%
\definecolor{currentstroke}{rgb}{0.000000,0.000000,0.000000}%
\pgfsetstrokecolor{currentstroke}%
\pgfsetdash{}{0pt}%
\pgfpathmoveto{\pgfqpoint{6.957000in}{0.523600in}}%
\pgfpathlineto{\pgfqpoint{6.957000in}{4.188800in}}%
\pgfusepath{stroke}%
\end{pgfscope}%
\begin{pgfscope}%
\pgfsetrectcap%
\pgfsetmiterjoin%
\pgfsetlinewidth{0.803000pt}%
\definecolor{currentstroke}{rgb}{0.000000,0.000000,0.000000}%
\pgfsetstrokecolor{currentstroke}%
\pgfsetdash{}{0pt}%
\pgfpathmoveto{\pgfqpoint{0.966250in}{0.523600in}}%
\pgfpathlineto{\pgfqpoint{6.957000in}{0.523600in}}%
\pgfusepath{stroke}%
\end{pgfscope}%
\begin{pgfscope}%
\pgfsetrectcap%
\pgfsetmiterjoin%
\pgfsetlinewidth{0.803000pt}%
\definecolor{currentstroke}{rgb}{0.000000,0.000000,0.000000}%
\pgfsetstrokecolor{currentstroke}%
\pgfsetdash{}{0pt}%
\pgfpathmoveto{\pgfqpoint{0.966250in}{4.188800in}}%
\pgfpathlineto{\pgfqpoint{6.957000in}{4.188800in}}%
\pgfusepath{stroke}%
\end{pgfscope}%
\begin{pgfscope}%
\pgftext[x=0.966250in,y=4.372060in,left,base]{\sffamily\fontsize{10.000000}{12.000000}\selectfont Iterations: 11940, Time: 0.112 ps, RXPWR: 4.0 percent, TXPWR: 96.0 percent}%
\end{pgfscope}%
\end{pgfpicture}%
\makeatother%
\endgroup%
}}
        \subcaption{Simulation using a refractive index of 1.5.}
        \label{fig:task3_1}
    \end{subfigure}\\
    \begin{subfigure}[b]{0.7\textwidth}
        \noindent\makebox[\textwidth]{\scalebox{0.7}{%% Creator: Matplotlib, PGF backend
%%
%% To include the figure in your LaTeX document, write
%%   \input{<filename>.pgf}
%%
%% Make sure the required packages are loaded in your preamble
%%   \usepackage{pgf}
%%
%% Figures using additional raster images can only be included by \input if
%% they are in the same directory as the main LaTeX file. For loading figures
%% from other directories you can use the `import` package
%%   \usepackage{import}
%% and then include the figures with
%%   \import{<path to file>}{<filename>.pgf}
%%
%% Matplotlib used the following preamble
%%   \usepackage{fontspec}
%%   \setmainfont{DejaVu Serif}
%%   \setsansfont{DejaVu Sans}
%%   \setmonofont{DejaVu Sans Mono}
%%
\begingroup%
\makeatletter%
\begin{pgfpicture}%
\pgfpathrectangle{\pgfpointorigin}{\pgfqpoint{7.710000in}{4.680000in}}%
\pgfusepath{use as bounding box, clip}%
\begin{pgfscope}%
\pgfsetbuttcap%
\pgfsetmiterjoin%
\definecolor{currentfill}{rgb}{1.000000,1.000000,1.000000}%
\pgfsetfillcolor{currentfill}%
\pgfsetlinewidth{0.000000pt}%
\definecolor{currentstroke}{rgb}{1.000000,1.000000,1.000000}%
\pgfsetstrokecolor{currentstroke}%
\pgfsetdash{}{0pt}%
\pgfpathmoveto{\pgfqpoint{0.000000in}{0.000000in}}%
\pgfpathlineto{\pgfqpoint{7.710000in}{0.000000in}}%
\pgfpathlineto{\pgfqpoint{7.710000in}{4.680000in}}%
\pgfpathlineto{\pgfqpoint{0.000000in}{4.680000in}}%
\pgfpathclose%
\pgfusepath{fill}%
\end{pgfscope}%
\begin{pgfscope}%
\pgfsetbuttcap%
\pgfsetmiterjoin%
\definecolor{currentfill}{rgb}{1.000000,1.000000,1.000000}%
\pgfsetfillcolor{currentfill}%
\pgfsetlinewidth{0.000000pt}%
\definecolor{currentstroke}{rgb}{0.000000,0.000000,0.000000}%
\pgfsetstrokecolor{currentstroke}%
\pgfsetstrokeopacity{0.000000}%
\pgfsetdash{}{0pt}%
\pgfpathmoveto{\pgfqpoint{0.963750in}{0.514800in}}%
\pgfpathlineto{\pgfqpoint{6.939000in}{0.514800in}}%
\pgfpathlineto{\pgfqpoint{6.939000in}{4.118400in}}%
\pgfpathlineto{\pgfqpoint{0.963750in}{4.118400in}}%
\pgfpathclose%
\pgfusepath{fill}%
\end{pgfscope}%
\begin{pgfscope}%
\pgfsetbuttcap%
\pgfsetroundjoin%
\definecolor{currentfill}{rgb}{0.000000,0.000000,0.000000}%
\pgfsetfillcolor{currentfill}%
\pgfsetlinewidth{0.803000pt}%
\definecolor{currentstroke}{rgb}{0.000000,0.000000,0.000000}%
\pgfsetstrokecolor{currentstroke}%
\pgfsetdash{}{0pt}%
\pgfsys@defobject{currentmarker}{\pgfqpoint{0.000000in}{-0.048611in}}{\pgfqpoint{0.000000in}{0.000000in}}{%
\pgfpathmoveto{\pgfqpoint{0.000000in}{0.000000in}}%
\pgfpathlineto{\pgfqpoint{0.000000in}{-0.048611in}}%
\pgfusepath{stroke,fill}%
}%
\begin{pgfscope}%
\pgfsys@transformshift{0.963750in}{0.514800in}%
\pgfsys@useobject{currentmarker}{}%
\end{pgfscope}%
\end{pgfscope}%
\begin{pgfscope}%
\pgftext[x=0.963750in,y=0.417578in,,top]{\sffamily\fontsize{10.000000}{12.000000}\selectfont 0}%
\end{pgfscope}%
\begin{pgfscope}%
\pgfsetbuttcap%
\pgfsetroundjoin%
\definecolor{currentfill}{rgb}{0.000000,0.000000,0.000000}%
\pgfsetfillcolor{currentfill}%
\pgfsetlinewidth{0.803000pt}%
\definecolor{currentstroke}{rgb}{0.000000,0.000000,0.000000}%
\pgfsetstrokecolor{currentstroke}%
\pgfsetdash{}{0pt}%
\pgfsys@defobject{currentmarker}{\pgfqpoint{0.000000in}{-0.048611in}}{\pgfqpoint{0.000000in}{0.000000in}}{%
\pgfpathmoveto{\pgfqpoint{0.000000in}{0.000000in}}%
\pgfpathlineto{\pgfqpoint{0.000000in}{-0.048611in}}%
\pgfusepath{stroke,fill}%
}%
\begin{pgfscope}%
\pgfsys@transformshift{1.671718in}{0.514800in}%
\pgfsys@useobject{currentmarker}{}%
\end{pgfscope}%
\end{pgfscope}%
\begin{pgfscope}%
\pgftext[x=1.671718in,y=0.417578in,,top]{\sffamily\fontsize{10.000000}{12.000000}\selectfont 5}%
\end{pgfscope}%
\begin{pgfscope}%
\pgfsetbuttcap%
\pgfsetroundjoin%
\definecolor{currentfill}{rgb}{0.000000,0.000000,0.000000}%
\pgfsetfillcolor{currentfill}%
\pgfsetlinewidth{0.803000pt}%
\definecolor{currentstroke}{rgb}{0.000000,0.000000,0.000000}%
\pgfsetstrokecolor{currentstroke}%
\pgfsetdash{}{0pt}%
\pgfsys@defobject{currentmarker}{\pgfqpoint{0.000000in}{-0.048611in}}{\pgfqpoint{0.000000in}{0.000000in}}{%
\pgfpathmoveto{\pgfqpoint{0.000000in}{0.000000in}}%
\pgfpathlineto{\pgfqpoint{0.000000in}{-0.048611in}}%
\pgfusepath{stroke,fill}%
}%
\begin{pgfscope}%
\pgfsys@transformshift{2.379686in}{0.514800in}%
\pgfsys@useobject{currentmarker}{}%
\end{pgfscope}%
\end{pgfscope}%
\begin{pgfscope}%
\pgftext[x=2.379686in,y=0.417578in,,top]{\sffamily\fontsize{10.000000}{12.000000}\selectfont 10}%
\end{pgfscope}%
\begin{pgfscope}%
\pgfsetbuttcap%
\pgfsetroundjoin%
\definecolor{currentfill}{rgb}{0.000000,0.000000,0.000000}%
\pgfsetfillcolor{currentfill}%
\pgfsetlinewidth{0.803000pt}%
\definecolor{currentstroke}{rgb}{0.000000,0.000000,0.000000}%
\pgfsetstrokecolor{currentstroke}%
\pgfsetdash{}{0pt}%
\pgfsys@defobject{currentmarker}{\pgfqpoint{0.000000in}{-0.048611in}}{\pgfqpoint{0.000000in}{0.000000in}}{%
\pgfpathmoveto{\pgfqpoint{0.000000in}{0.000000in}}%
\pgfpathlineto{\pgfqpoint{0.000000in}{-0.048611in}}%
\pgfusepath{stroke,fill}%
}%
\begin{pgfscope}%
\pgfsys@transformshift{3.087654in}{0.514800in}%
\pgfsys@useobject{currentmarker}{}%
\end{pgfscope}%
\end{pgfscope}%
\begin{pgfscope}%
\pgftext[x=3.087654in,y=0.417578in,,top]{\sffamily\fontsize{10.000000}{12.000000}\selectfont 15}%
\end{pgfscope}%
\begin{pgfscope}%
\pgfsetbuttcap%
\pgfsetroundjoin%
\definecolor{currentfill}{rgb}{0.000000,0.000000,0.000000}%
\pgfsetfillcolor{currentfill}%
\pgfsetlinewidth{0.803000pt}%
\definecolor{currentstroke}{rgb}{0.000000,0.000000,0.000000}%
\pgfsetstrokecolor{currentstroke}%
\pgfsetdash{}{0pt}%
\pgfsys@defobject{currentmarker}{\pgfqpoint{0.000000in}{-0.048611in}}{\pgfqpoint{0.000000in}{0.000000in}}{%
\pgfpathmoveto{\pgfqpoint{0.000000in}{0.000000in}}%
\pgfpathlineto{\pgfqpoint{0.000000in}{-0.048611in}}%
\pgfusepath{stroke,fill}%
}%
\begin{pgfscope}%
\pgfsys@transformshift{3.795622in}{0.514800in}%
\pgfsys@useobject{currentmarker}{}%
\end{pgfscope}%
\end{pgfscope}%
\begin{pgfscope}%
\pgftext[x=3.795622in,y=0.417578in,,top]{\sffamily\fontsize{10.000000}{12.000000}\selectfont 20}%
\end{pgfscope}%
\begin{pgfscope}%
\pgfsetbuttcap%
\pgfsetroundjoin%
\definecolor{currentfill}{rgb}{0.000000,0.000000,0.000000}%
\pgfsetfillcolor{currentfill}%
\pgfsetlinewidth{0.803000pt}%
\definecolor{currentstroke}{rgb}{0.000000,0.000000,0.000000}%
\pgfsetstrokecolor{currentstroke}%
\pgfsetdash{}{0pt}%
\pgfsys@defobject{currentmarker}{\pgfqpoint{0.000000in}{-0.048611in}}{\pgfqpoint{0.000000in}{0.000000in}}{%
\pgfpathmoveto{\pgfqpoint{0.000000in}{0.000000in}}%
\pgfpathlineto{\pgfqpoint{0.000000in}{-0.048611in}}%
\pgfusepath{stroke,fill}%
}%
\begin{pgfscope}%
\pgfsys@transformshift{4.503590in}{0.514800in}%
\pgfsys@useobject{currentmarker}{}%
\end{pgfscope}%
\end{pgfscope}%
\begin{pgfscope}%
\pgftext[x=4.503590in,y=0.417578in,,top]{\sffamily\fontsize{10.000000}{12.000000}\selectfont 25}%
\end{pgfscope}%
\begin{pgfscope}%
\pgfsetbuttcap%
\pgfsetroundjoin%
\definecolor{currentfill}{rgb}{0.000000,0.000000,0.000000}%
\pgfsetfillcolor{currentfill}%
\pgfsetlinewidth{0.803000pt}%
\definecolor{currentstroke}{rgb}{0.000000,0.000000,0.000000}%
\pgfsetstrokecolor{currentstroke}%
\pgfsetdash{}{0pt}%
\pgfsys@defobject{currentmarker}{\pgfqpoint{0.000000in}{-0.048611in}}{\pgfqpoint{0.000000in}{0.000000in}}{%
\pgfpathmoveto{\pgfqpoint{0.000000in}{0.000000in}}%
\pgfpathlineto{\pgfqpoint{0.000000in}{-0.048611in}}%
\pgfusepath{stroke,fill}%
}%
\begin{pgfscope}%
\pgfsys@transformshift{5.211558in}{0.514800in}%
\pgfsys@useobject{currentmarker}{}%
\end{pgfscope}%
\end{pgfscope}%
\begin{pgfscope}%
\pgftext[x=5.211558in,y=0.417578in,,top]{\sffamily\fontsize{10.000000}{12.000000}\selectfont 30}%
\end{pgfscope}%
\begin{pgfscope}%
\pgfsetbuttcap%
\pgfsetroundjoin%
\definecolor{currentfill}{rgb}{0.000000,0.000000,0.000000}%
\pgfsetfillcolor{currentfill}%
\pgfsetlinewidth{0.803000pt}%
\definecolor{currentstroke}{rgb}{0.000000,0.000000,0.000000}%
\pgfsetstrokecolor{currentstroke}%
\pgfsetdash{}{0pt}%
\pgfsys@defobject{currentmarker}{\pgfqpoint{0.000000in}{-0.048611in}}{\pgfqpoint{0.000000in}{0.000000in}}{%
\pgfpathmoveto{\pgfqpoint{0.000000in}{0.000000in}}%
\pgfpathlineto{\pgfqpoint{0.000000in}{-0.048611in}}%
\pgfusepath{stroke,fill}%
}%
\begin{pgfscope}%
\pgfsys@transformshift{5.919526in}{0.514800in}%
\pgfsys@useobject{currentmarker}{}%
\end{pgfscope}%
\end{pgfscope}%
\begin{pgfscope}%
\pgftext[x=5.919526in,y=0.417578in,,top]{\sffamily\fontsize{10.000000}{12.000000}\selectfont 35}%
\end{pgfscope}%
\begin{pgfscope}%
\pgfsetbuttcap%
\pgfsetroundjoin%
\definecolor{currentfill}{rgb}{0.000000,0.000000,0.000000}%
\pgfsetfillcolor{currentfill}%
\pgfsetlinewidth{0.803000pt}%
\definecolor{currentstroke}{rgb}{0.000000,0.000000,0.000000}%
\pgfsetstrokecolor{currentstroke}%
\pgfsetdash{}{0pt}%
\pgfsys@defobject{currentmarker}{\pgfqpoint{0.000000in}{-0.048611in}}{\pgfqpoint{0.000000in}{0.000000in}}{%
\pgfpathmoveto{\pgfqpoint{0.000000in}{0.000000in}}%
\pgfpathlineto{\pgfqpoint{0.000000in}{-0.048611in}}%
\pgfusepath{stroke,fill}%
}%
\begin{pgfscope}%
\pgfsys@transformshift{6.627494in}{0.514800in}%
\pgfsys@useobject{currentmarker}{}%
\end{pgfscope}%
\end{pgfscope}%
\begin{pgfscope}%
\pgftext[x=6.627494in,y=0.417578in,,top]{\sffamily\fontsize{10.000000}{12.000000}\selectfont 40}%
\end{pgfscope}%
\begin{pgfscope}%
\pgftext[x=3.951375in,y=0.227609in,,top]{\sffamily\fontsize{16.000000}{19.200000}\selectfont \(\displaystyle z-position [\mu m]\)}%
\end{pgfscope}%
\begin{pgfscope}%
\pgfsetbuttcap%
\pgfsetroundjoin%
\definecolor{currentfill}{rgb}{0.000000,0.000000,0.000000}%
\pgfsetfillcolor{currentfill}%
\pgfsetlinewidth{0.803000pt}%
\definecolor{currentstroke}{rgb}{0.000000,0.000000,0.000000}%
\pgfsetstrokecolor{currentstroke}%
\pgfsetdash{}{0pt}%
\pgfsys@defobject{currentmarker}{\pgfqpoint{-0.048611in}{0.000000in}}{\pgfqpoint{0.000000in}{0.000000in}}{%
\pgfpathmoveto{\pgfqpoint{0.000000in}{0.000000in}}%
\pgfpathlineto{\pgfqpoint{-0.048611in}{0.000000in}}%
\pgfusepath{stroke,fill}%
}%
\begin{pgfscope}%
\pgfsys@transformshift{0.963750in}{0.514800in}%
\pgfsys@useobject{currentmarker}{}%
\end{pgfscope}%
\end{pgfscope}%
\begin{pgfscope}%
\pgftext[x=0.352544in,y=0.462038in,left,base]{\sffamily\fontsize{10.000000}{12.000000}\selectfont -0.003}%
\end{pgfscope}%
\begin{pgfscope}%
\pgfsetbuttcap%
\pgfsetroundjoin%
\definecolor{currentfill}{rgb}{0.000000,0.000000,0.000000}%
\pgfsetfillcolor{currentfill}%
\pgfsetlinewidth{0.803000pt}%
\definecolor{currentstroke}{rgb}{0.000000,0.000000,0.000000}%
\pgfsetstrokecolor{currentstroke}%
\pgfsetdash{}{0pt}%
\pgfsys@defobject{currentmarker}{\pgfqpoint{-0.048611in}{0.000000in}}{\pgfqpoint{0.000000in}{0.000000in}}{%
\pgfpathmoveto{\pgfqpoint{0.000000in}{0.000000in}}%
\pgfpathlineto{\pgfqpoint{-0.048611in}{0.000000in}}%
\pgfusepath{stroke,fill}%
}%
\begin{pgfscope}%
\pgfsys@transformshift{0.963750in}{1.115400in}%
\pgfsys@useobject{currentmarker}{}%
\end{pgfscope}%
\end{pgfscope}%
\begin{pgfscope}%
\pgftext[x=0.352544in,y=1.062638in,left,base]{\sffamily\fontsize{10.000000}{12.000000}\selectfont -0.002}%
\end{pgfscope}%
\begin{pgfscope}%
\pgfsetbuttcap%
\pgfsetroundjoin%
\definecolor{currentfill}{rgb}{0.000000,0.000000,0.000000}%
\pgfsetfillcolor{currentfill}%
\pgfsetlinewidth{0.803000pt}%
\definecolor{currentstroke}{rgb}{0.000000,0.000000,0.000000}%
\pgfsetstrokecolor{currentstroke}%
\pgfsetdash{}{0pt}%
\pgfsys@defobject{currentmarker}{\pgfqpoint{-0.048611in}{0.000000in}}{\pgfqpoint{0.000000in}{0.000000in}}{%
\pgfpathmoveto{\pgfqpoint{0.000000in}{0.000000in}}%
\pgfpathlineto{\pgfqpoint{-0.048611in}{0.000000in}}%
\pgfusepath{stroke,fill}%
}%
\begin{pgfscope}%
\pgfsys@transformshift{0.963750in}{1.716000in}%
\pgfsys@useobject{currentmarker}{}%
\end{pgfscope}%
\end{pgfscope}%
\begin{pgfscope}%
\pgftext[x=0.352544in,y=1.663238in,left,base]{\sffamily\fontsize{10.000000}{12.000000}\selectfont -0.001}%
\end{pgfscope}%
\begin{pgfscope}%
\pgfsetbuttcap%
\pgfsetroundjoin%
\definecolor{currentfill}{rgb}{0.000000,0.000000,0.000000}%
\pgfsetfillcolor{currentfill}%
\pgfsetlinewidth{0.803000pt}%
\definecolor{currentstroke}{rgb}{0.000000,0.000000,0.000000}%
\pgfsetstrokecolor{currentstroke}%
\pgfsetdash{}{0pt}%
\pgfsys@defobject{currentmarker}{\pgfqpoint{-0.048611in}{0.000000in}}{\pgfqpoint{0.000000in}{0.000000in}}{%
\pgfpathmoveto{\pgfqpoint{0.000000in}{0.000000in}}%
\pgfpathlineto{\pgfqpoint{-0.048611in}{0.000000in}}%
\pgfusepath{stroke,fill}%
}%
\begin{pgfscope}%
\pgfsys@transformshift{0.963750in}{2.316600in}%
\pgfsys@useobject{currentmarker}{}%
\end{pgfscope}%
\end{pgfscope}%
\begin{pgfscope}%
\pgftext[x=0.468918in,y=2.263838in,left,base]{\sffamily\fontsize{10.000000}{12.000000}\selectfont 0.000}%
\end{pgfscope}%
\begin{pgfscope}%
\pgfsetbuttcap%
\pgfsetroundjoin%
\definecolor{currentfill}{rgb}{0.000000,0.000000,0.000000}%
\pgfsetfillcolor{currentfill}%
\pgfsetlinewidth{0.803000pt}%
\definecolor{currentstroke}{rgb}{0.000000,0.000000,0.000000}%
\pgfsetstrokecolor{currentstroke}%
\pgfsetdash{}{0pt}%
\pgfsys@defobject{currentmarker}{\pgfqpoint{-0.048611in}{0.000000in}}{\pgfqpoint{0.000000in}{0.000000in}}{%
\pgfpathmoveto{\pgfqpoint{0.000000in}{0.000000in}}%
\pgfpathlineto{\pgfqpoint{-0.048611in}{0.000000in}}%
\pgfusepath{stroke,fill}%
}%
\begin{pgfscope}%
\pgfsys@transformshift{0.963750in}{2.917200in}%
\pgfsys@useobject{currentmarker}{}%
\end{pgfscope}%
\end{pgfscope}%
\begin{pgfscope}%
\pgftext[x=0.468918in,y=2.864438in,left,base]{\sffamily\fontsize{10.000000}{12.000000}\selectfont 0.001}%
\end{pgfscope}%
\begin{pgfscope}%
\pgfsetbuttcap%
\pgfsetroundjoin%
\definecolor{currentfill}{rgb}{0.000000,0.000000,0.000000}%
\pgfsetfillcolor{currentfill}%
\pgfsetlinewidth{0.803000pt}%
\definecolor{currentstroke}{rgb}{0.000000,0.000000,0.000000}%
\pgfsetstrokecolor{currentstroke}%
\pgfsetdash{}{0pt}%
\pgfsys@defobject{currentmarker}{\pgfqpoint{-0.048611in}{0.000000in}}{\pgfqpoint{0.000000in}{0.000000in}}{%
\pgfpathmoveto{\pgfqpoint{0.000000in}{0.000000in}}%
\pgfpathlineto{\pgfqpoint{-0.048611in}{0.000000in}}%
\pgfusepath{stroke,fill}%
}%
\begin{pgfscope}%
\pgfsys@transformshift{0.963750in}{3.517800in}%
\pgfsys@useobject{currentmarker}{}%
\end{pgfscope}%
\end{pgfscope}%
\begin{pgfscope}%
\pgftext[x=0.468918in,y=3.465038in,left,base]{\sffamily\fontsize{10.000000}{12.000000}\selectfont 0.002}%
\end{pgfscope}%
\begin{pgfscope}%
\pgfsetbuttcap%
\pgfsetroundjoin%
\definecolor{currentfill}{rgb}{0.000000,0.000000,0.000000}%
\pgfsetfillcolor{currentfill}%
\pgfsetlinewidth{0.803000pt}%
\definecolor{currentstroke}{rgb}{0.000000,0.000000,0.000000}%
\pgfsetstrokecolor{currentstroke}%
\pgfsetdash{}{0pt}%
\pgfsys@defobject{currentmarker}{\pgfqpoint{-0.048611in}{0.000000in}}{\pgfqpoint{0.000000in}{0.000000in}}{%
\pgfpathmoveto{\pgfqpoint{0.000000in}{0.000000in}}%
\pgfpathlineto{\pgfqpoint{-0.048611in}{0.000000in}}%
\pgfusepath{stroke,fill}%
}%
\begin{pgfscope}%
\pgfsys@transformshift{0.963750in}{4.118400in}%
\pgfsys@useobject{currentmarker}{}%
\end{pgfscope}%
\end{pgfscope}%
\begin{pgfscope}%
\pgftext[x=0.468918in,y=4.065638in,left,base]{\sffamily\fontsize{10.000000}{12.000000}\selectfont 0.003}%
\end{pgfscope}%
\begin{pgfscope}%
\pgftext[x=0.296988in,y=2.316600in,,bottom,rotate=90.000000]{\sffamily\fontsize{16.000000}{19.200000}\selectfont \(\displaystyle Poynting\) \(\displaystyle vector\)}%
\end{pgfscope}%
\begin{pgfscope}%
\pgfpathrectangle{\pgfqpoint{0.963750in}{0.514800in}}{\pgfqpoint{5.975250in}{3.603600in}} %
\pgfusepath{clip}%
\pgfsetrectcap%
\pgfsetroundjoin%
\pgfsetlinewidth{1.505625pt}%
\definecolor{currentstroke}{rgb}{0.000000,0.000000,0.000000}%
\pgfsetstrokecolor{currentstroke}%
\pgfsetdash{}{0pt}%
\pgfpathmoveto{\pgfqpoint{0.963750in}{2.316600in}}%
\pgfpathlineto{\pgfqpoint{1.007170in}{2.315318in}}%
\pgfpathlineto{\pgfqpoint{1.015137in}{2.311791in}}%
\pgfpathlineto{\pgfqpoint{1.023901in}{2.308243in}}%
\pgfpathlineto{\pgfqpoint{1.028283in}{2.308774in}}%
\pgfpathlineto{\pgfqpoint{1.033461in}{2.311959in}}%
\pgfpathlineto{\pgfqpoint{1.040632in}{2.316540in}}%
\pgfpathlineto{\pgfqpoint{1.043420in}{2.315997in}}%
\pgfpathlineto{\pgfqpoint{1.046208in}{2.312952in}}%
\pgfpathlineto{\pgfqpoint{1.049395in}{2.305822in}}%
\pgfpathlineto{\pgfqpoint{1.053777in}{2.289978in}}%
\pgfpathlineto{\pgfqpoint{1.064533in}{2.248217in}}%
\pgfpathlineto{\pgfqpoint{1.066923in}{2.245737in}}%
\pgfpathlineto{\pgfqpoint{1.068516in}{2.246485in}}%
\pgfpathlineto{\pgfqpoint{1.070508in}{2.250229in}}%
\pgfpathlineto{\pgfqpoint{1.073695in}{2.262192in}}%
\pgfpathlineto{\pgfqpoint{1.086043in}{2.316593in}}%
\pgfpathlineto{\pgfqpoint{1.087238in}{2.315814in}}%
\pgfpathlineto{\pgfqpoint{1.088832in}{2.311711in}}%
\pgfpathlineto{\pgfqpoint{1.091222in}{2.298686in}}%
\pgfpathlineto{\pgfqpoint{1.094807in}{2.264559in}}%
\pgfpathlineto{\pgfqpoint{1.109148in}{2.106529in}}%
\pgfpathlineto{\pgfqpoint{1.110343in}{2.105279in}}%
\pgfpathlineto{\pgfqpoint{1.110741in}{2.105547in}}%
\pgfpathlineto{\pgfqpoint{1.111936in}{2.108425in}}%
\pgfpathlineto{\pgfqpoint{1.113928in}{2.119996in}}%
\pgfpathlineto{\pgfqpoint{1.117115in}{2.154114in}}%
\pgfpathlineto{\pgfqpoint{1.129862in}{2.314935in}}%
\pgfpathlineto{\pgfqpoint{1.131057in}{2.316587in}}%
\pgfpathlineto{\pgfqpoint{1.131455in}{2.316252in}}%
\pgfpathlineto{\pgfqpoint{1.132650in}{2.312525in}}%
\pgfpathlineto{\pgfqpoint{1.134642in}{2.297196in}}%
\pgfpathlineto{\pgfqpoint{1.137829in}{2.250598in}}%
\pgfpathlineto{\pgfqpoint{1.143406in}{2.126222in}}%
\pgfpathlineto{\pgfqpoint{1.150178in}{1.989771in}}%
\pgfpathlineto{\pgfqpoint{1.152966in}{1.965651in}}%
\pgfpathlineto{\pgfqpoint{1.154161in}{1.963333in}}%
\pgfpathlineto{\pgfqpoint{1.154560in}{1.963678in}}%
\pgfpathlineto{\pgfqpoint{1.155755in}{1.968071in}}%
\pgfpathlineto{\pgfqpoint{1.157746in}{1.986294in}}%
\pgfpathlineto{\pgfqpoint{1.160933in}{2.040341in}}%
\pgfpathlineto{\pgfqpoint{1.174875in}{2.315014in}}%
\pgfpathlineto{\pgfqpoint{1.175672in}{2.316580in}}%
\pgfpathlineto{\pgfqpoint{1.176071in}{2.316424in}}%
\pgfpathlineto{\pgfqpoint{1.177266in}{2.312166in}}%
\pgfpathlineto{\pgfqpoint{1.179257in}{2.292622in}}%
\pgfpathlineto{\pgfqpoint{1.182444in}{2.232153in}}%
\pgfpathlineto{\pgfqpoint{1.188419in}{2.061246in}}%
\pgfpathlineto{\pgfqpoint{1.194395in}{1.912438in}}%
\pgfpathlineto{\pgfqpoint{1.197581in}{1.879075in}}%
\pgfpathlineto{\pgfqpoint{1.198378in}{1.877358in}}%
\pgfpathlineto{\pgfqpoint{1.198777in}{1.877533in}}%
\pgfpathlineto{\pgfqpoint{1.199972in}{1.882185in}}%
\pgfpathlineto{\pgfqpoint{1.201963in}{1.903322in}}%
\pgfpathlineto{\pgfqpoint{1.205150in}{1.967738in}}%
\pgfpathlineto{\pgfqpoint{1.211922in}{2.168526in}}%
\pgfpathlineto{\pgfqpoint{1.217101in}{2.289711in}}%
\pgfpathlineto{\pgfqpoint{1.219889in}{2.315497in}}%
\pgfpathlineto{\pgfqpoint{1.220686in}{2.316578in}}%
\pgfpathlineto{\pgfqpoint{1.221084in}{2.316037in}}%
\pgfpathlineto{\pgfqpoint{1.222279in}{2.310102in}}%
\pgfpathlineto{\pgfqpoint{1.224271in}{2.286250in}}%
\pgfpathlineto{\pgfqpoint{1.227458in}{2.216270in}}%
\pgfpathlineto{\pgfqpoint{1.235026in}{1.976966in}}%
\pgfpathlineto{\pgfqpoint{1.239807in}{1.865042in}}%
\pgfpathlineto{\pgfqpoint{1.242595in}{1.840714in}}%
\pgfpathlineto{\pgfqpoint{1.243392in}{1.840335in}}%
\pgfpathlineto{\pgfqpoint{1.244587in}{1.845337in}}%
\pgfpathlineto{\pgfqpoint{1.246579in}{1.868086in}}%
\pgfpathlineto{\pgfqpoint{1.249765in}{1.937353in}}%
\pgfpathlineto{\pgfqpoint{1.256537in}{2.153297in}}%
\pgfpathlineto{\pgfqpoint{1.261716in}{2.285048in}}%
\pgfpathlineto{\pgfqpoint{1.264504in}{2.314690in}}%
\pgfpathlineto{\pgfqpoint{1.265301in}{2.316576in}}%
\pgfpathlineto{\pgfqpoint{1.265699in}{2.316388in}}%
\pgfpathlineto{\pgfqpoint{1.266894in}{2.311302in}}%
\pgfpathlineto{\pgfqpoint{1.268886in}{2.288192in}}%
\pgfpathlineto{\pgfqpoint{1.272073in}{2.217773in}}%
\pgfpathlineto{\pgfqpoint{1.278845in}{1.997540in}}%
\pgfpathlineto{\pgfqpoint{1.284023in}{1.861829in}}%
\pgfpathlineto{\pgfqpoint{1.286812in}{1.830227in}}%
\pgfpathlineto{\pgfqpoint{1.288007in}{1.827779in}}%
\pgfpathlineto{\pgfqpoint{1.288804in}{1.829957in}}%
\pgfpathlineto{\pgfqpoint{1.290397in}{1.843314in}}%
\pgfpathlineto{\pgfqpoint{1.293185in}{1.893379in}}%
\pgfpathlineto{\pgfqpoint{1.297567in}{2.022314in}}%
\pgfpathlineto{\pgfqpoint{1.306331in}{2.281106in}}%
\pgfpathlineto{\pgfqpoint{1.309119in}{2.313710in}}%
\pgfpathlineto{\pgfqpoint{1.310315in}{2.316576in}}%
\pgfpathlineto{\pgfqpoint{1.310713in}{2.316002in}}%
\pgfpathlineto{\pgfqpoint{1.311908in}{2.309714in}}%
\pgfpathlineto{\pgfqpoint{1.313900in}{2.284542in}}%
\pgfpathlineto{\pgfqpoint{1.317086in}{2.211110in}}%
\pgfpathlineto{\pgfqpoint{1.331427in}{1.828003in}}%
\pgfpathlineto{\pgfqpoint{1.332622in}{1.824522in}}%
\pgfpathlineto{\pgfqpoint{1.333020in}{1.824895in}}%
\pgfpathlineto{\pgfqpoint{1.334215in}{1.830593in}}%
\pgfpathlineto{\pgfqpoint{1.336207in}{1.854865in}}%
\pgfpathlineto{\pgfqpoint{1.339394in}{1.927201in}}%
\pgfpathlineto{\pgfqpoint{1.346963in}{2.175229in}}%
\pgfpathlineto{\pgfqpoint{1.351743in}{2.290921in}}%
\pgfpathlineto{\pgfqpoint{1.354531in}{2.316000in}}%
\pgfpathlineto{\pgfqpoint{1.355328in}{2.316384in}}%
\pgfpathlineto{\pgfqpoint{1.356523in}{2.311219in}}%
\pgfpathlineto{\pgfqpoint{1.358515in}{2.287774in}}%
\pgfpathlineto{\pgfqpoint{1.361702in}{2.216466in}}%
\pgfpathlineto{\pgfqpoint{1.368474in}{1.994180in}}%
\pgfpathlineto{\pgfqpoint{1.373652in}{1.857824in}}%
\pgfpathlineto{\pgfqpoint{1.376441in}{1.826343in}}%
\pgfpathlineto{\pgfqpoint{1.377636in}{1.824037in}}%
\pgfpathlineto{\pgfqpoint{1.378831in}{1.828628in}}%
\pgfpathlineto{\pgfqpoint{1.380822in}{1.851170in}}%
\pgfpathlineto{\pgfqpoint{1.384009in}{1.921323in}}%
\pgfpathlineto{\pgfqpoint{1.390383in}{2.129850in}}%
\pgfpathlineto{\pgfqpoint{1.395960in}{2.280969in}}%
\pgfpathlineto{\pgfqpoint{1.398748in}{2.313700in}}%
\pgfpathlineto{\pgfqpoint{1.399943in}{2.316576in}}%
\pgfpathlineto{\pgfqpoint{1.400342in}{2.316000in}}%
\pgfpathlineto{\pgfqpoint{1.401537in}{2.309692in}}%
\pgfpathlineto{\pgfqpoint{1.403528in}{2.284449in}}%
\pgfpathlineto{\pgfqpoint{1.406715in}{2.210842in}}%
\pgfpathlineto{\pgfqpoint{1.421056in}{1.827330in}}%
\pgfpathlineto{\pgfqpoint{1.422251in}{1.823880in}}%
\pgfpathlineto{\pgfqpoint{1.422649in}{1.824264in}}%
\pgfpathlineto{\pgfqpoint{1.423844in}{1.830001in}}%
\pgfpathlineto{\pgfqpoint{1.425836in}{1.854352in}}%
\pgfpathlineto{\pgfqpoint{1.429023in}{1.926827in}}%
\pgfpathlineto{\pgfqpoint{1.436591in}{2.175134in}}%
\pgfpathlineto{\pgfqpoint{1.441372in}{2.290907in}}%
\pgfpathlineto{\pgfqpoint{1.444160in}{2.316000in}}%
\pgfpathlineto{\pgfqpoint{1.444957in}{2.316384in}}%
\pgfpathlineto{\pgfqpoint{1.446152in}{2.311216in}}%
\pgfpathlineto{\pgfqpoint{1.448144in}{2.287763in}}%
\pgfpathlineto{\pgfqpoint{1.451330in}{2.216434in}}%
\pgfpathlineto{\pgfqpoint{1.458102in}{1.994109in}}%
\pgfpathlineto{\pgfqpoint{1.463281in}{1.857748in}}%
\pgfpathlineto{\pgfqpoint{1.466069in}{1.826275in}}%
\pgfpathlineto{\pgfqpoint{1.467264in}{1.823973in}}%
\pgfpathlineto{\pgfqpoint{1.468459in}{1.828569in}}%
\pgfpathlineto{\pgfqpoint{1.470451in}{1.851121in}}%
\pgfpathlineto{\pgfqpoint{1.473638in}{1.921289in}}%
\pgfpathlineto{\pgfqpoint{1.480012in}{2.129839in}}%
\pgfpathlineto{\pgfqpoint{1.485589in}{2.280967in}}%
\pgfpathlineto{\pgfqpoint{1.488377in}{2.313700in}}%
\pgfpathlineto{\pgfqpoint{1.489572in}{2.316576in}}%
\pgfpathlineto{\pgfqpoint{1.489970in}{2.316000in}}%
\pgfpathlineto{\pgfqpoint{1.491165in}{2.309692in}}%
\pgfpathlineto{\pgfqpoint{1.493157in}{2.284449in}}%
\pgfpathlineto{\pgfqpoint{1.496344in}{2.210840in}}%
\pgfpathlineto{\pgfqpoint{1.510685in}{1.827327in}}%
\pgfpathlineto{\pgfqpoint{1.511880in}{1.823877in}}%
\pgfpathlineto{\pgfqpoint{1.512278in}{1.824261in}}%
\pgfpathlineto{\pgfqpoint{1.513473in}{1.829999in}}%
\pgfpathlineto{\pgfqpoint{1.515465in}{1.854350in}}%
\pgfpathlineto{\pgfqpoint{1.518652in}{1.926826in}}%
\pgfpathlineto{\pgfqpoint{1.526220in}{2.175134in}}%
\pgfpathlineto{\pgfqpoint{1.531000in}{2.290907in}}%
\pgfpathlineto{\pgfqpoint{1.533789in}{2.316000in}}%
\pgfpathlineto{\pgfqpoint{1.534586in}{2.316384in}}%
\pgfpathlineto{\pgfqpoint{1.535781in}{2.311216in}}%
\pgfpathlineto{\pgfqpoint{1.537772in}{2.287763in}}%
\pgfpathlineto{\pgfqpoint{1.540959in}{2.216434in}}%
\pgfpathlineto{\pgfqpoint{1.547731in}{1.994109in}}%
\pgfpathlineto{\pgfqpoint{1.552910in}{1.857748in}}%
\pgfpathlineto{\pgfqpoint{1.555698in}{1.826275in}}%
\pgfpathlineto{\pgfqpoint{1.556893in}{1.823973in}}%
\pgfpathlineto{\pgfqpoint{1.558088in}{1.828569in}}%
\pgfpathlineto{\pgfqpoint{1.560080in}{1.851121in}}%
\pgfpathlineto{\pgfqpoint{1.563267in}{1.921289in}}%
\pgfpathlineto{\pgfqpoint{1.569640in}{2.129839in}}%
\pgfpathlineto{\pgfqpoint{1.575217in}{2.280967in}}%
\pgfpathlineto{\pgfqpoint{1.578006in}{2.313700in}}%
\pgfpathlineto{\pgfqpoint{1.579201in}{2.316576in}}%
\pgfpathlineto{\pgfqpoint{1.579599in}{2.316000in}}%
\pgfpathlineto{\pgfqpoint{1.580794in}{2.309692in}}%
\pgfpathlineto{\pgfqpoint{1.582786in}{2.284449in}}%
\pgfpathlineto{\pgfqpoint{1.585973in}{2.210840in}}%
\pgfpathlineto{\pgfqpoint{1.600313in}{1.827329in}}%
\pgfpathlineto{\pgfqpoint{1.601508in}{1.823880in}}%
\pgfpathlineto{\pgfqpoint{1.601907in}{1.824264in}}%
\pgfpathlineto{\pgfqpoint{1.603102in}{1.830002in}}%
\pgfpathlineto{\pgfqpoint{1.605093in}{1.854354in}}%
\pgfpathlineto{\pgfqpoint{1.608280in}{1.926829in}}%
\pgfpathlineto{\pgfqpoint{1.615849in}{2.175137in}}%
\pgfpathlineto{\pgfqpoint{1.620629in}{2.290908in}}%
\pgfpathlineto{\pgfqpoint{1.623418in}{2.316000in}}%
\pgfpathlineto{\pgfqpoint{1.624214in}{2.316384in}}%
\pgfpathlineto{\pgfqpoint{1.625409in}{2.311217in}}%
\pgfpathlineto{\pgfqpoint{1.627401in}{2.287764in}}%
\pgfpathlineto{\pgfqpoint{1.630588in}{2.216439in}}%
\pgfpathlineto{\pgfqpoint{1.637360in}{1.994133in}}%
\pgfpathlineto{\pgfqpoint{1.642538in}{1.857796in}}%
\pgfpathlineto{\pgfqpoint{1.645327in}{1.826335in}}%
\pgfpathlineto{\pgfqpoint{1.646522in}{1.824039in}}%
\pgfpathlineto{\pgfqpoint{1.647717in}{1.828639in}}%
\pgfpathlineto{\pgfqpoint{1.649709in}{1.851195in}}%
\pgfpathlineto{\pgfqpoint{1.652896in}{1.921365in}}%
\pgfpathlineto{\pgfqpoint{1.659269in}{2.129890in}}%
\pgfpathlineto{\pgfqpoint{1.664846in}{2.280980in}}%
\pgfpathlineto{\pgfqpoint{1.667634in}{2.313701in}}%
\pgfpathlineto{\pgfqpoint{1.668829in}{2.316576in}}%
\pgfpathlineto{\pgfqpoint{1.669228in}{2.316000in}}%
\pgfpathlineto{\pgfqpoint{1.670423in}{2.309696in}}%
\pgfpathlineto{\pgfqpoint{1.672415in}{2.284466in}}%
\pgfpathlineto{\pgfqpoint{1.675601in}{2.210907in}}%
\pgfpathlineto{\pgfqpoint{1.689942in}{1.827934in}}%
\pgfpathlineto{\pgfqpoint{1.691137in}{1.824522in}}%
\pgfpathlineto{\pgfqpoint{1.691535in}{1.824918in}}%
\pgfpathlineto{\pgfqpoint{1.692731in}{1.830683in}}%
\pgfpathlineto{\pgfqpoint{1.694722in}{1.855060in}}%
\pgfpathlineto{\pgfqpoint{1.697909in}{1.927513in}}%
\pgfpathlineto{\pgfqpoint{1.705478in}{2.175477in}}%
\pgfpathlineto{\pgfqpoint{1.710258in}{2.290983in}}%
\pgfpathlineto{\pgfqpoint{1.713046in}{2.316002in}}%
\pgfpathlineto{\pgfqpoint{1.713843in}{2.316385in}}%
\pgfpathlineto{\pgfqpoint{1.715038in}{2.311235in}}%
\pgfpathlineto{\pgfqpoint{1.717030in}{2.287873in}}%
\pgfpathlineto{\pgfqpoint{1.720217in}{2.216869in}}%
\pgfpathlineto{\pgfqpoint{1.726989in}{1.995920in}}%
\pgfpathlineto{\pgfqpoint{1.732167in}{1.860865in}}%
\pgfpathlineto{\pgfqpoint{1.734956in}{1.829957in}}%
\pgfpathlineto{\pgfqpoint{1.736151in}{1.827833in}}%
\pgfpathlineto{\pgfqpoint{1.737346in}{1.832559in}}%
\pgfpathlineto{\pgfqpoint{1.739337in}{1.855203in}}%
\pgfpathlineto{\pgfqpoint{1.742524in}{1.925162in}}%
\pgfpathlineto{\pgfqpoint{1.749296in}{2.145231in}}%
\pgfpathlineto{\pgfqpoint{1.754475in}{2.281490in}}%
\pgfpathlineto{\pgfqpoint{1.757263in}{2.313747in}}%
\pgfpathlineto{\pgfqpoint{1.758458in}{2.316576in}}%
\pgfpathlineto{\pgfqpoint{1.758857in}{2.316010in}}%
\pgfpathlineto{\pgfqpoint{1.760052in}{2.309813in}}%
\pgfpathlineto{\pgfqpoint{1.762043in}{2.285048in}}%
\pgfpathlineto{\pgfqpoint{1.765230in}{2.213022in}}%
\pgfpathlineto{\pgfqpoint{1.779172in}{1.845337in}}%
\pgfpathlineto{\pgfqpoint{1.780766in}{1.840152in}}%
\pgfpathlineto{\pgfqpoint{1.781563in}{1.842019in}}%
\pgfpathlineto{\pgfqpoint{1.783156in}{1.854545in}}%
\pgfpathlineto{\pgfqpoint{1.785944in}{1.902536in}}%
\pgfpathlineto{\pgfqpoint{1.790326in}{2.027136in}}%
\pgfpathlineto{\pgfqpoint{1.799090in}{2.279499in}}%
\pgfpathlineto{\pgfqpoint{1.802277in}{2.314775in}}%
\pgfpathlineto{\pgfqpoint{1.803073in}{2.316578in}}%
\pgfpathlineto{\pgfqpoint{1.803472in}{2.316397in}}%
\pgfpathlineto{\pgfqpoint{1.804667in}{2.311561in}}%
\pgfpathlineto{\pgfqpoint{1.806659in}{2.289711in}}%
\pgfpathlineto{\pgfqpoint{1.809845in}{2.223793in}}%
\pgfpathlineto{\pgfqpoint{1.823788in}{1.882185in}}%
\pgfpathlineto{\pgfqpoint{1.825381in}{1.877358in}}%
\pgfpathlineto{\pgfqpoint{1.826178in}{1.879075in}}%
\pgfpathlineto{\pgfqpoint{1.827771in}{1.890631in}}%
\pgfpathlineto{\pgfqpoint{1.830560in}{1.934855in}}%
\pgfpathlineto{\pgfqpoint{1.834941in}{2.049300in}}%
\pgfpathlineto{\pgfqpoint{1.843705in}{2.280647in}}%
\pgfpathlineto{\pgfqpoint{1.846892in}{2.314217in}}%
\pgfpathlineto{\pgfqpoint{1.848087in}{2.316580in}}%
\pgfpathlineto{\pgfqpoint{1.848485in}{2.316109in}}%
\pgfpathlineto{\pgfqpoint{1.849680in}{2.310987in}}%
\pgfpathlineto{\pgfqpoint{1.851672in}{2.290748in}}%
\pgfpathlineto{\pgfqpoint{1.855257in}{2.224111in}}%
\pgfpathlineto{\pgfqpoint{1.867208in}{1.973764in}}%
\pgfpathlineto{\pgfqpoint{1.869598in}{1.963333in}}%
\pgfpathlineto{\pgfqpoint{1.870395in}{1.964323in}}%
\pgfpathlineto{\pgfqpoint{1.871988in}{1.972896in}}%
\pgfpathlineto{\pgfqpoint{1.874378in}{2.001041in}}%
\pgfpathlineto{\pgfqpoint{1.878362in}{2.079098in}}%
\pgfpathlineto{\pgfqpoint{1.888719in}{2.292792in}}%
\pgfpathlineto{\pgfqpoint{1.891507in}{2.314224in}}%
\pgfpathlineto{\pgfqpoint{1.893101in}{2.316474in}}%
\pgfpathlineto{\pgfqpoint{1.894296in}{2.313525in}}%
\pgfpathlineto{\pgfqpoint{1.896287in}{2.300538in}}%
\pgfpathlineto{\pgfqpoint{1.899873in}{2.257288in}}%
\pgfpathlineto{\pgfqpoint{1.910628in}{2.114377in}}%
\pgfpathlineto{\pgfqpoint{1.913018in}{2.105547in}}%
\pgfpathlineto{\pgfqpoint{1.914213in}{2.105773in}}%
\pgfpathlineto{\pgfqpoint{1.915806in}{2.110779in}}%
\pgfpathlineto{\pgfqpoint{1.918197in}{2.127475in}}%
\pgfpathlineto{\pgfqpoint{1.922180in}{2.173527in}}%
\pgfpathlineto{\pgfqpoint{1.932537in}{2.298686in}}%
\pgfpathlineto{\pgfqpoint{1.935724in}{2.314213in}}%
\pgfpathlineto{\pgfqpoint{1.937716in}{2.316593in}}%
\pgfpathlineto{\pgfqpoint{1.939309in}{2.314837in}}%
\pgfpathlineto{\pgfqpoint{1.941699in}{2.307204in}}%
\pgfpathlineto{\pgfqpoint{1.946479in}{2.281648in}}%
\pgfpathlineto{\pgfqpoint{1.952455in}{2.252575in}}%
\pgfpathlineto{\pgfqpoint{1.955641in}{2.246111in}}%
\pgfpathlineto{\pgfqpoint{1.957633in}{2.246101in}}%
\pgfpathlineto{\pgfqpoint{1.959625in}{2.249017in}}%
\pgfpathlineto{\pgfqpoint{1.962812in}{2.258650in}}%
\pgfpathlineto{\pgfqpoint{1.978347in}{2.314112in}}%
\pgfpathlineto{\pgfqpoint{1.981534in}{2.316496in}}%
\pgfpathlineto{\pgfqpoint{1.984721in}{2.316013in}}%
\pgfpathlineto{\pgfqpoint{1.990298in}{2.311959in}}%
\pgfpathlineto{\pgfqpoint{1.996273in}{2.308516in}}%
\pgfpathlineto{\pgfqpoint{2.000655in}{2.308366in}}%
\pgfpathlineto{\pgfqpoint{2.006630in}{2.310728in}}%
\pgfpathlineto{\pgfqpoint{2.018581in}{2.315834in}}%
\pgfpathlineto{\pgfqpoint{2.027344in}{2.316600in}}%
\pgfpathlineto{\pgfqpoint{2.378291in}{2.316600in}}%
\pgfpathlineto{\pgfqpoint{3.222793in}{2.317660in}}%
\pgfpathlineto{\pgfqpoint{3.224785in}{2.322555in}}%
\pgfpathlineto{\pgfqpoint{3.229166in}{2.336326in}}%
\pgfpathlineto{\pgfqpoint{3.230361in}{2.332964in}}%
\pgfpathlineto{\pgfqpoint{3.233548in}{2.316542in}}%
\pgfpathlineto{\pgfqpoint{3.233947in}{2.317107in}}%
\pgfpathlineto{\pgfqpoint{3.235142in}{2.327696in}}%
\pgfpathlineto{\pgfqpoint{3.237133in}{2.378891in}}%
\pgfpathlineto{\pgfqpoint{3.240719in}{2.478770in}}%
\pgfpathlineto{\pgfqpoint{3.241117in}{2.478616in}}%
\pgfpathlineto{\pgfqpoint{3.241914in}{2.467796in}}%
\pgfpathlineto{\pgfqpoint{3.243905in}{2.391887in}}%
\pgfpathlineto{\pgfqpoint{3.246296in}{2.316346in}}%
\pgfpathlineto{\pgfqpoint{3.247092in}{2.326924in}}%
\pgfpathlineto{\pgfqpoint{3.248686in}{2.422837in}}%
\pgfpathlineto{\pgfqpoint{3.253067in}{2.794323in}}%
\pgfpathlineto{\pgfqpoint{3.253466in}{2.793049in}}%
\pgfpathlineto{\pgfqpoint{3.254661in}{2.733203in}}%
\pgfpathlineto{\pgfqpoint{3.259043in}{2.316334in}}%
\pgfpathlineto{\pgfqpoint{3.259839in}{2.335263in}}%
\pgfpathlineto{\pgfqpoint{3.261433in}{2.515542in}}%
\pgfpathlineto{\pgfqpoint{3.265815in}{3.110703in}}%
\pgfpathlineto{\pgfqpoint{3.266611in}{3.075208in}}%
\pgfpathlineto{\pgfqpoint{3.268603in}{2.770330in}}%
\pgfpathlineto{\pgfqpoint{3.271790in}{2.316957in}}%
\pgfpathlineto{\pgfqpoint{3.272188in}{2.319035in}}%
\pgfpathlineto{\pgfqpoint{3.273383in}{2.426938in}}%
\pgfpathlineto{\pgfqpoint{3.278562in}{3.299844in}}%
\pgfpathlineto{\pgfqpoint{3.279359in}{3.249682in}}%
\pgfpathlineto{\pgfqpoint{3.281350in}{2.869177in}}%
\pgfpathlineto{\pgfqpoint{3.284537in}{2.318204in}}%
\pgfpathlineto{\pgfqpoint{3.284935in}{2.317902in}}%
\pgfpathlineto{\pgfqpoint{3.285732in}{2.375853in}}%
\pgfpathlineto{\pgfqpoint{3.287724in}{2.785486in}}%
\pgfpathlineto{\pgfqpoint{3.290911in}{3.378395in}}%
\pgfpathlineto{\pgfqpoint{3.291309in}{3.381527in}}%
\pgfpathlineto{\pgfqpoint{3.292106in}{3.327369in}}%
\pgfpathlineto{\pgfqpoint{3.294098in}{2.918504in}}%
\pgfpathlineto{\pgfqpoint{3.297284in}{2.319886in}}%
\pgfpathlineto{\pgfqpoint{3.297683in}{2.316808in}}%
\pgfpathlineto{\pgfqpoint{3.298479in}{2.371950in}}%
\pgfpathlineto{\pgfqpoint{3.300471in}{2.788081in}}%
\pgfpathlineto{\pgfqpoint{3.303658in}{3.404278in}}%
\pgfpathlineto{\pgfqpoint{3.304056in}{3.409183in}}%
\pgfpathlineto{\pgfqpoint{3.304853in}{3.357049in}}%
\pgfpathlineto{\pgfqpoint{3.306845in}{2.943936in}}%
\pgfpathlineto{\pgfqpoint{3.310032in}{2.321908in}}%
\pgfpathlineto{\pgfqpoint{3.310430in}{2.316020in}}%
\pgfpathlineto{\pgfqpoint{3.311227in}{2.366472in}}%
\pgfpathlineto{\pgfqpoint{3.313218in}{2.778515in}}%
\pgfpathlineto{\pgfqpoint{3.316405in}{3.409247in}}%
\pgfpathlineto{\pgfqpoint{3.316803in}{3.416616in}}%
\pgfpathlineto{\pgfqpoint{3.317600in}{3.368915in}}%
\pgfpathlineto{\pgfqpoint{3.319592in}{2.960715in}}%
\pgfpathlineto{\pgfqpoint{3.322779in}{2.324258in}}%
\pgfpathlineto{\pgfqpoint{3.323177in}{2.315593in}}%
\pgfpathlineto{\pgfqpoint{3.323177in}{2.315593in}}%
\pgfpathlineto{\pgfqpoint{3.323177in}{2.315593in}}%
\pgfpathlineto{\pgfqpoint{3.323974in}{2.360808in}}%
\pgfpathlineto{\pgfqpoint{3.325965in}{2.765444in}}%
\pgfpathlineto{\pgfqpoint{3.329152in}{3.408044in}}%
\pgfpathlineto{\pgfqpoint{3.329551in}{3.418099in}}%
\pgfpathlineto{\pgfqpoint{3.329551in}{3.418099in}}%
\pgfpathlineto{\pgfqpoint{3.329551in}{3.418099in}}%
\pgfpathlineto{\pgfqpoint{3.330347in}{3.375584in}}%
\pgfpathlineto{\pgfqpoint{3.331941in}{3.078065in}}%
\pgfpathlineto{\pgfqpoint{3.335924in}{2.315531in}}%
\pgfpathlineto{\pgfqpoint{3.336721in}{2.355376in}}%
\pgfpathlineto{\pgfqpoint{3.338314in}{2.649114in}}%
\pgfpathlineto{\pgfqpoint{3.342298in}{3.418112in}}%
\pgfpathlineto{\pgfqpoint{3.343095in}{3.380970in}}%
\pgfpathlineto{\pgfqpoint{3.344688in}{3.091108in}}%
\pgfpathlineto{\pgfqpoint{3.348671in}{2.315833in}}%
\pgfpathlineto{\pgfqpoint{3.349070in}{2.322678in}}%
\pgfpathlineto{\pgfqpoint{3.350265in}{2.462719in}}%
\pgfpathlineto{\pgfqpoint{3.355045in}{3.417635in}}%
\pgfpathlineto{\pgfqpoint{3.355842in}{3.385913in}}%
\pgfpathlineto{\pgfqpoint{3.357435in}{3.103948in}}%
\pgfpathlineto{\pgfqpoint{3.361419in}{2.316496in}}%
\pgfpathlineto{\pgfqpoint{3.361817in}{2.320586in}}%
\pgfpathlineto{\pgfqpoint{3.363012in}{2.453238in}}%
\pgfpathlineto{\pgfqpoint{3.367792in}{3.416790in}}%
\pgfpathlineto{\pgfqpoint{3.368589in}{3.390511in}}%
\pgfpathlineto{\pgfqpoint{3.370182in}{3.116630in}}%
\pgfpathlineto{\pgfqpoint{3.374166in}{2.317521in}}%
\pgfpathlineto{\pgfqpoint{3.374564in}{2.318852in}}%
\pgfpathlineto{\pgfqpoint{3.375759in}{2.444029in}}%
\pgfpathlineto{\pgfqpoint{3.380938in}{3.415633in}}%
\pgfpathlineto{\pgfqpoint{3.381336in}{3.394765in}}%
\pgfpathlineto{\pgfqpoint{3.382930in}{3.129149in}}%
\pgfpathlineto{\pgfqpoint{3.386913in}{2.318907in}}%
\pgfpathlineto{\pgfqpoint{3.387311in}{2.317479in}}%
\pgfpathlineto{\pgfqpoint{3.388108in}{2.376703in}}%
\pgfpathlineto{\pgfqpoint{3.390100in}{2.802265in}}%
\pgfpathlineto{\pgfqpoint{3.393287in}{3.414013in}}%
\pgfpathlineto{\pgfqpoint{3.393685in}{3.416820in}}%
\pgfpathlineto{\pgfqpoint{3.394482in}{3.360239in}}%
\pgfpathlineto{\pgfqpoint{3.396473in}{2.938493in}}%
\pgfpathlineto{\pgfqpoint{3.399660in}{2.320653in}}%
\pgfpathlineto{\pgfqpoint{3.400059in}{2.316467in}}%
\pgfpathlineto{\pgfqpoint{3.400855in}{2.370392in}}%
\pgfpathlineto{\pgfqpoint{3.402847in}{2.788223in}}%
\pgfpathlineto{\pgfqpoint{3.406034in}{3.411951in}}%
\pgfpathlineto{\pgfqpoint{3.406432in}{3.417503in}}%
\pgfpathlineto{\pgfqpoint{3.407229in}{3.366217in}}%
\pgfpathlineto{\pgfqpoint{3.409221in}{2.952326in}}%
\pgfpathlineto{\pgfqpoint{3.412407in}{2.322754in}}%
\pgfpathlineto{\pgfqpoint{3.412806in}{2.315817in}}%
\pgfpathlineto{\pgfqpoint{3.413603in}{2.364389in}}%
\pgfpathlineto{\pgfqpoint{3.415594in}{2.773955in}}%
\pgfpathlineto{\pgfqpoint{3.418781in}{3.408396in}}%
\pgfpathlineto{\pgfqpoint{3.419179in}{3.416607in}}%
\pgfpathlineto{\pgfqpoint{3.419179in}{3.416607in}}%
\pgfpathlineto{\pgfqpoint{3.419179in}{3.416607in}}%
\pgfpathlineto{\pgfqpoint{3.419976in}{3.370552in}}%
\pgfpathlineto{\pgfqpoint{3.421968in}{2.965013in}}%
\pgfpathlineto{\pgfqpoint{3.425155in}{2.325181in}}%
\pgfpathlineto{\pgfqpoint{3.425553in}{2.315532in}}%
\pgfpathlineto{\pgfqpoint{3.425553in}{2.315532in}}%
\pgfpathlineto{\pgfqpoint{3.425553in}{2.315532in}}%
\pgfpathlineto{\pgfqpoint{3.426350in}{2.358619in}}%
\pgfpathlineto{\pgfqpoint{3.427943in}{2.655727in}}%
\pgfpathlineto{\pgfqpoint{3.431927in}{3.409328in}}%
\pgfpathlineto{\pgfqpoint{3.432723in}{3.368183in}}%
\pgfpathlineto{\pgfqpoint{3.434317in}{3.074536in}}%
\pgfpathlineto{\pgfqpoint{3.438300in}{2.315614in}}%
\pgfpathlineto{\pgfqpoint{3.439097in}{2.352833in}}%
\pgfpathlineto{\pgfqpoint{3.440690in}{2.637228in}}%
\pgfpathlineto{\pgfqpoint{3.444674in}{3.380570in}}%
\pgfpathlineto{\pgfqpoint{3.445471in}{3.343431in}}%
\pgfpathlineto{\pgfqpoint{3.447064in}{3.061841in}}%
\pgfpathlineto{\pgfqpoint{3.451047in}{2.316047in}}%
\pgfpathlineto{\pgfqpoint{3.451446in}{2.321604in}}%
\pgfpathlineto{\pgfqpoint{3.452641in}{2.449161in}}%
\pgfpathlineto{\pgfqpoint{3.457421in}{3.295095in}}%
\pgfpathlineto{\pgfqpoint{3.457819in}{3.286845in}}%
\pgfpathlineto{\pgfqpoint{3.459014in}{3.156620in}}%
\pgfpathlineto{\pgfqpoint{3.463795in}{2.316714in}}%
\pgfpathlineto{\pgfqpoint{3.464591in}{2.338853in}}%
\pgfpathlineto{\pgfqpoint{3.466185in}{2.553405in}}%
\pgfpathlineto{\pgfqpoint{3.469770in}{3.093381in}}%
\pgfpathlineto{\pgfqpoint{3.470168in}{3.098741in}}%
\pgfpathlineto{\pgfqpoint{3.470965in}{3.064484in}}%
\pgfpathlineto{\pgfqpoint{3.472957in}{2.777131in}}%
\pgfpathlineto{\pgfqpoint{3.476143in}{2.329085in}}%
\pgfpathlineto{\pgfqpoint{3.476542in}{2.317277in}}%
\pgfpathlineto{\pgfqpoint{3.476940in}{2.317699in}}%
\pgfpathlineto{\pgfqpoint{3.478135in}{2.384123in}}%
\pgfpathlineto{\pgfqpoint{3.482517in}{2.778177in}}%
\pgfpathlineto{\pgfqpoint{3.482915in}{2.776309in}}%
\pgfpathlineto{\pgfqpoint{3.484110in}{2.719727in}}%
\pgfpathlineto{\pgfqpoint{3.489687in}{2.316747in}}%
\pgfpathlineto{\pgfqpoint{3.490882in}{2.342801in}}%
\pgfpathlineto{\pgfqpoint{3.494866in}{2.467352in}}%
\pgfpathlineto{\pgfqpoint{3.495264in}{2.466436in}}%
\pgfpathlineto{\pgfqpoint{3.496459in}{2.446726in}}%
\pgfpathlineto{\pgfqpoint{3.502036in}{2.316912in}}%
\pgfpathlineto{\pgfqpoint{3.502435in}{2.316563in}}%
\pgfpathlineto{\pgfqpoint{3.502833in}{2.317177in}}%
\pgfpathlineto{\pgfqpoint{3.505223in}{2.329098in}}%
\pgfpathlineto{\pgfqpoint{3.507215in}{2.333437in}}%
\pgfpathlineto{\pgfqpoint{3.508410in}{2.331292in}}%
\pgfpathlineto{\pgfqpoint{3.513987in}{2.316917in}}%
\pgfpathlineto{\pgfqpoint{3.516775in}{2.316734in}}%
\pgfpathlineto{\pgfqpoint{3.524344in}{2.316633in}}%
\pgfpathlineto{\pgfqpoint{3.572943in}{2.316600in}}%
\pgfpathlineto{\pgfqpoint{6.938602in}{2.316600in}}%
\pgfpathlineto{\pgfqpoint{6.938602in}{2.316600in}}%
\pgfusepath{stroke}%
\end{pgfscope}%
\begin{pgfscope}%
\pgfsetrectcap%
\pgfsetmiterjoin%
\pgfsetlinewidth{0.803000pt}%
\definecolor{currentstroke}{rgb}{0.000000,0.000000,0.000000}%
\pgfsetstrokecolor{currentstroke}%
\pgfsetdash{}{0pt}%
\pgfpathmoveto{\pgfqpoint{0.963750in}{0.514800in}}%
\pgfpathlineto{\pgfqpoint{0.963750in}{4.118400in}}%
\pgfusepath{stroke}%
\end{pgfscope}%
\begin{pgfscope}%
\pgfsetrectcap%
\pgfsetmiterjoin%
\pgfsetlinewidth{0.803000pt}%
\definecolor{currentstroke}{rgb}{0.000000,0.000000,0.000000}%
\pgfsetstrokecolor{currentstroke}%
\pgfsetdash{}{0pt}%
\pgfpathmoveto{\pgfqpoint{6.939000in}{0.514800in}}%
\pgfpathlineto{\pgfqpoint{6.939000in}{4.118400in}}%
\pgfusepath{stroke}%
\end{pgfscope}%
\begin{pgfscope}%
\pgfsetrectcap%
\pgfsetmiterjoin%
\pgfsetlinewidth{0.803000pt}%
\definecolor{currentstroke}{rgb}{0.000000,0.000000,0.000000}%
\pgfsetstrokecolor{currentstroke}%
\pgfsetdash{}{0pt}%
\pgfpathmoveto{\pgfqpoint{0.963750in}{0.514800in}}%
\pgfpathlineto{\pgfqpoint{6.939000in}{0.514800in}}%
\pgfusepath{stroke}%
\end{pgfscope}%
\begin{pgfscope}%
\pgfsetrectcap%
\pgfsetmiterjoin%
\pgfsetlinewidth{0.803000pt}%
\definecolor{currentstroke}{rgb}{0.000000,0.000000,0.000000}%
\pgfsetstrokecolor{currentstroke}%
\pgfsetdash{}{0pt}%
\pgfpathmoveto{\pgfqpoint{0.963750in}{4.118400in}}%
\pgfpathlineto{\pgfqpoint{6.939000in}{4.118400in}}%
\pgfusepath{stroke}%
\end{pgfscope}%
\begin{pgfscope}%
\pgftext[x=0.963750in,y=4.298580in,left,base]{\sffamily\fontsize{10.000000}{12.000000}\selectfont Iterations: 11940, Time: 0.112 ps, RXPWR: 30.9 percent, TXPWR: 69.1 percent}%
\end{pgfscope}%
\end{pgfpicture}%
\makeatother%
\endgroup%
}}
        \subcaption{Simulation using a refractive index of 3.5.}
        \label{fig:task3_2}
    \end{subfigure}
  \caption{Simulation results from task 3.}
  \label{fig:task3}
\end{figure}

\section{Task 4}\label{sec:4}
Since we are using a single wavelength the AR-coating can completely remove the reflection (granted the sampling distance aligns with a quarter wavelength, which it can be set to). See Figure~\ref{fig:task4}
\begin{figure}
  \centering
  \noindent\makebox[\textwidth]{\scalebox{0.90}{%% Creator: Matplotlib, PGF backend
%%
%% To include the figure in your LaTeX document, write
%%   \input{<filename>.pgf}
%%
%% Make sure the required packages are loaded in your preamble
%%   \usepackage{pgf}
%%
%% Figures using additional raster images can only be included by \input if
%% they are in the same directory as the main LaTeX file. For loading figures
%% from other directories you can use the `import` package
%%   \usepackage{import}
%% and then include the figures with
%%   \import{<path to file>}{<filename>.pgf}
%%
%% Matplotlib used the following preamble
%%   \usepackage{fontspec}
%%   \setmainfont{DejaVu Serif}
%%   \setsansfont{DejaVu Sans}
%%   \setmonofont{DejaVu Sans Mono}
%%
\begingroup%
\makeatletter%
\begin{pgfpicture}%
\pgfpathrectangle{\pgfpointorigin}{\pgfqpoint{7.730000in}{4.860000in}}%
\pgfusepath{use as bounding box, clip}%
\begin{pgfscope}%
\pgfsetbuttcap%
\pgfsetmiterjoin%
\definecolor{currentfill}{rgb}{1.000000,1.000000,1.000000}%
\pgfsetfillcolor{currentfill}%
\pgfsetlinewidth{0.000000pt}%
\definecolor{currentstroke}{rgb}{1.000000,1.000000,1.000000}%
\pgfsetstrokecolor{currentstroke}%
\pgfsetdash{}{0pt}%
\pgfpathmoveto{\pgfqpoint{0.000000in}{0.000000in}}%
\pgfpathlineto{\pgfqpoint{7.730000in}{0.000000in}}%
\pgfpathlineto{\pgfqpoint{7.730000in}{4.860000in}}%
\pgfpathlineto{\pgfqpoint{0.000000in}{4.860000in}}%
\pgfpathclose%
\pgfusepath{fill}%
\end{pgfscope}%
\begin{pgfscope}%
\pgfsetbuttcap%
\pgfsetmiterjoin%
\definecolor{currentfill}{rgb}{1.000000,1.000000,1.000000}%
\pgfsetfillcolor{currentfill}%
\pgfsetlinewidth{0.000000pt}%
\definecolor{currentstroke}{rgb}{0.000000,0.000000,0.000000}%
\pgfsetstrokecolor{currentstroke}%
\pgfsetstrokeopacity{0.000000}%
\pgfsetdash{}{0pt}%
\pgfpathmoveto{\pgfqpoint{0.966250in}{0.534600in}}%
\pgfpathlineto{\pgfqpoint{6.957000in}{0.534600in}}%
\pgfpathlineto{\pgfqpoint{6.957000in}{4.276800in}}%
\pgfpathlineto{\pgfqpoint{0.966250in}{4.276800in}}%
\pgfpathclose%
\pgfusepath{fill}%
\end{pgfscope}%
\begin{pgfscope}%
\pgfsetbuttcap%
\pgfsetroundjoin%
\definecolor{currentfill}{rgb}{0.000000,0.000000,0.000000}%
\pgfsetfillcolor{currentfill}%
\pgfsetlinewidth{0.803000pt}%
\definecolor{currentstroke}{rgb}{0.000000,0.000000,0.000000}%
\pgfsetstrokecolor{currentstroke}%
\pgfsetdash{}{0pt}%
\pgfsys@defobject{currentmarker}{\pgfqpoint{0.000000in}{-0.048611in}}{\pgfqpoint{0.000000in}{0.000000in}}{%
\pgfpathmoveto{\pgfqpoint{0.000000in}{0.000000in}}%
\pgfpathlineto{\pgfqpoint{0.000000in}{-0.048611in}}%
\pgfusepath{stroke,fill}%
}%
\begin{pgfscope}%
\pgfsys@transformshift{0.966250in}{0.534600in}%
\pgfsys@useobject{currentmarker}{}%
\end{pgfscope}%
\end{pgfscope}%
\begin{pgfscope}%
\pgftext[x=0.966250in,y=0.437378in,,top]{\sffamily\fontsize{10.000000}{12.000000}\selectfont 0}%
\end{pgfscope}%
\begin{pgfscope}%
\pgfsetbuttcap%
\pgfsetroundjoin%
\definecolor{currentfill}{rgb}{0.000000,0.000000,0.000000}%
\pgfsetfillcolor{currentfill}%
\pgfsetlinewidth{0.803000pt}%
\definecolor{currentstroke}{rgb}{0.000000,0.000000,0.000000}%
\pgfsetstrokecolor{currentstroke}%
\pgfsetdash{}{0pt}%
\pgfsys@defobject{currentmarker}{\pgfqpoint{0.000000in}{-0.048611in}}{\pgfqpoint{0.000000in}{0.000000in}}{%
\pgfpathmoveto{\pgfqpoint{0.000000in}{0.000000in}}%
\pgfpathlineto{\pgfqpoint{0.000000in}{-0.048611in}}%
\pgfusepath{stroke,fill}%
}%
\begin{pgfscope}%
\pgfsys@transformshift{1.676055in}{0.534600in}%
\pgfsys@useobject{currentmarker}{}%
\end{pgfscope}%
\end{pgfscope}%
\begin{pgfscope}%
\pgftext[x=1.676055in,y=0.437378in,,top]{\sffamily\fontsize{10.000000}{12.000000}\selectfont 5}%
\end{pgfscope}%
\begin{pgfscope}%
\pgfsetbuttcap%
\pgfsetroundjoin%
\definecolor{currentfill}{rgb}{0.000000,0.000000,0.000000}%
\pgfsetfillcolor{currentfill}%
\pgfsetlinewidth{0.803000pt}%
\definecolor{currentstroke}{rgb}{0.000000,0.000000,0.000000}%
\pgfsetstrokecolor{currentstroke}%
\pgfsetdash{}{0pt}%
\pgfsys@defobject{currentmarker}{\pgfqpoint{0.000000in}{-0.048611in}}{\pgfqpoint{0.000000in}{0.000000in}}{%
\pgfpathmoveto{\pgfqpoint{0.000000in}{0.000000in}}%
\pgfpathlineto{\pgfqpoint{0.000000in}{-0.048611in}}%
\pgfusepath{stroke,fill}%
}%
\begin{pgfscope}%
\pgfsys@transformshift{2.385859in}{0.534600in}%
\pgfsys@useobject{currentmarker}{}%
\end{pgfscope}%
\end{pgfscope}%
\begin{pgfscope}%
\pgftext[x=2.385859in,y=0.437378in,,top]{\sffamily\fontsize{10.000000}{12.000000}\selectfont 10}%
\end{pgfscope}%
\begin{pgfscope}%
\pgfsetbuttcap%
\pgfsetroundjoin%
\definecolor{currentfill}{rgb}{0.000000,0.000000,0.000000}%
\pgfsetfillcolor{currentfill}%
\pgfsetlinewidth{0.803000pt}%
\definecolor{currentstroke}{rgb}{0.000000,0.000000,0.000000}%
\pgfsetstrokecolor{currentstroke}%
\pgfsetdash{}{0pt}%
\pgfsys@defobject{currentmarker}{\pgfqpoint{0.000000in}{-0.048611in}}{\pgfqpoint{0.000000in}{0.000000in}}{%
\pgfpathmoveto{\pgfqpoint{0.000000in}{0.000000in}}%
\pgfpathlineto{\pgfqpoint{0.000000in}{-0.048611in}}%
\pgfusepath{stroke,fill}%
}%
\begin{pgfscope}%
\pgfsys@transformshift{3.095664in}{0.534600in}%
\pgfsys@useobject{currentmarker}{}%
\end{pgfscope}%
\end{pgfscope}%
\begin{pgfscope}%
\pgftext[x=3.095664in,y=0.437378in,,top]{\sffamily\fontsize{10.000000}{12.000000}\selectfont 15}%
\end{pgfscope}%
\begin{pgfscope}%
\pgfsetbuttcap%
\pgfsetroundjoin%
\definecolor{currentfill}{rgb}{0.000000,0.000000,0.000000}%
\pgfsetfillcolor{currentfill}%
\pgfsetlinewidth{0.803000pt}%
\definecolor{currentstroke}{rgb}{0.000000,0.000000,0.000000}%
\pgfsetstrokecolor{currentstroke}%
\pgfsetdash{}{0pt}%
\pgfsys@defobject{currentmarker}{\pgfqpoint{0.000000in}{-0.048611in}}{\pgfqpoint{0.000000in}{0.000000in}}{%
\pgfpathmoveto{\pgfqpoint{0.000000in}{0.000000in}}%
\pgfpathlineto{\pgfqpoint{0.000000in}{-0.048611in}}%
\pgfusepath{stroke,fill}%
}%
\begin{pgfscope}%
\pgfsys@transformshift{3.805468in}{0.534600in}%
\pgfsys@useobject{currentmarker}{}%
\end{pgfscope}%
\end{pgfscope}%
\begin{pgfscope}%
\pgftext[x=3.805468in,y=0.437378in,,top]{\sffamily\fontsize{10.000000}{12.000000}\selectfont 20}%
\end{pgfscope}%
\begin{pgfscope}%
\pgfsetbuttcap%
\pgfsetroundjoin%
\definecolor{currentfill}{rgb}{0.000000,0.000000,0.000000}%
\pgfsetfillcolor{currentfill}%
\pgfsetlinewidth{0.803000pt}%
\definecolor{currentstroke}{rgb}{0.000000,0.000000,0.000000}%
\pgfsetstrokecolor{currentstroke}%
\pgfsetdash{}{0pt}%
\pgfsys@defobject{currentmarker}{\pgfqpoint{0.000000in}{-0.048611in}}{\pgfqpoint{0.000000in}{0.000000in}}{%
\pgfpathmoveto{\pgfqpoint{0.000000in}{0.000000in}}%
\pgfpathlineto{\pgfqpoint{0.000000in}{-0.048611in}}%
\pgfusepath{stroke,fill}%
}%
\begin{pgfscope}%
\pgfsys@transformshift{4.515273in}{0.534600in}%
\pgfsys@useobject{currentmarker}{}%
\end{pgfscope}%
\end{pgfscope}%
\begin{pgfscope}%
\pgftext[x=4.515273in,y=0.437378in,,top]{\sffamily\fontsize{10.000000}{12.000000}\selectfont 25}%
\end{pgfscope}%
\begin{pgfscope}%
\pgfsetbuttcap%
\pgfsetroundjoin%
\definecolor{currentfill}{rgb}{0.000000,0.000000,0.000000}%
\pgfsetfillcolor{currentfill}%
\pgfsetlinewidth{0.803000pt}%
\definecolor{currentstroke}{rgb}{0.000000,0.000000,0.000000}%
\pgfsetstrokecolor{currentstroke}%
\pgfsetdash{}{0pt}%
\pgfsys@defobject{currentmarker}{\pgfqpoint{0.000000in}{-0.048611in}}{\pgfqpoint{0.000000in}{0.000000in}}{%
\pgfpathmoveto{\pgfqpoint{0.000000in}{0.000000in}}%
\pgfpathlineto{\pgfqpoint{0.000000in}{-0.048611in}}%
\pgfusepath{stroke,fill}%
}%
\begin{pgfscope}%
\pgfsys@transformshift{5.225077in}{0.534600in}%
\pgfsys@useobject{currentmarker}{}%
\end{pgfscope}%
\end{pgfscope}%
\begin{pgfscope}%
\pgftext[x=5.225077in,y=0.437378in,,top]{\sffamily\fontsize{10.000000}{12.000000}\selectfont 30}%
\end{pgfscope}%
\begin{pgfscope}%
\pgfsetbuttcap%
\pgfsetroundjoin%
\definecolor{currentfill}{rgb}{0.000000,0.000000,0.000000}%
\pgfsetfillcolor{currentfill}%
\pgfsetlinewidth{0.803000pt}%
\definecolor{currentstroke}{rgb}{0.000000,0.000000,0.000000}%
\pgfsetstrokecolor{currentstroke}%
\pgfsetdash{}{0pt}%
\pgfsys@defobject{currentmarker}{\pgfqpoint{0.000000in}{-0.048611in}}{\pgfqpoint{0.000000in}{0.000000in}}{%
\pgfpathmoveto{\pgfqpoint{0.000000in}{0.000000in}}%
\pgfpathlineto{\pgfqpoint{0.000000in}{-0.048611in}}%
\pgfusepath{stroke,fill}%
}%
\begin{pgfscope}%
\pgfsys@transformshift{5.934882in}{0.534600in}%
\pgfsys@useobject{currentmarker}{}%
\end{pgfscope}%
\end{pgfscope}%
\begin{pgfscope}%
\pgftext[x=5.934882in,y=0.437378in,,top]{\sffamily\fontsize{10.000000}{12.000000}\selectfont 35}%
\end{pgfscope}%
\begin{pgfscope}%
\pgfsetbuttcap%
\pgfsetroundjoin%
\definecolor{currentfill}{rgb}{0.000000,0.000000,0.000000}%
\pgfsetfillcolor{currentfill}%
\pgfsetlinewidth{0.803000pt}%
\definecolor{currentstroke}{rgb}{0.000000,0.000000,0.000000}%
\pgfsetstrokecolor{currentstroke}%
\pgfsetdash{}{0pt}%
\pgfsys@defobject{currentmarker}{\pgfqpoint{0.000000in}{-0.048611in}}{\pgfqpoint{0.000000in}{0.000000in}}{%
\pgfpathmoveto{\pgfqpoint{0.000000in}{0.000000in}}%
\pgfpathlineto{\pgfqpoint{0.000000in}{-0.048611in}}%
\pgfusepath{stroke,fill}%
}%
\begin{pgfscope}%
\pgfsys@transformshift{6.644686in}{0.534600in}%
\pgfsys@useobject{currentmarker}{}%
\end{pgfscope}%
\end{pgfscope}%
\begin{pgfscope}%
\pgftext[x=6.644686in,y=0.437378in,,top]{\sffamily\fontsize{10.000000}{12.000000}\selectfont 40}%
\end{pgfscope}%
\begin{pgfscope}%
\pgftext[x=3.961625in,y=0.247409in,,top]{\sffamily\fontsize{16.000000}{19.200000}\selectfont \(\displaystyle z-position [\mu m]\)}%
\end{pgfscope}%
\begin{pgfscope}%
\pgfsetbuttcap%
\pgfsetroundjoin%
\definecolor{currentfill}{rgb}{0.000000,0.000000,0.000000}%
\pgfsetfillcolor{currentfill}%
\pgfsetlinewidth{0.803000pt}%
\definecolor{currentstroke}{rgb}{0.000000,0.000000,0.000000}%
\pgfsetstrokecolor{currentstroke}%
\pgfsetdash{}{0pt}%
\pgfsys@defobject{currentmarker}{\pgfqpoint{-0.048611in}{0.000000in}}{\pgfqpoint{0.000000in}{0.000000in}}{%
\pgfpathmoveto{\pgfqpoint{0.000000in}{0.000000in}}%
\pgfpathlineto{\pgfqpoint{-0.048611in}{0.000000in}}%
\pgfusepath{stroke,fill}%
}%
\begin{pgfscope}%
\pgfsys@transformshift{0.966250in}{0.534600in}%
\pgfsys@useobject{currentmarker}{}%
\end{pgfscope}%
\end{pgfscope}%
\begin{pgfscope}%
\pgftext[x=0.355044in,y=0.481838in,left,base]{\sffamily\fontsize{10.000000}{12.000000}\selectfont -0.003}%
\end{pgfscope}%
\begin{pgfscope}%
\pgfsetbuttcap%
\pgfsetroundjoin%
\definecolor{currentfill}{rgb}{0.000000,0.000000,0.000000}%
\pgfsetfillcolor{currentfill}%
\pgfsetlinewidth{0.803000pt}%
\definecolor{currentstroke}{rgb}{0.000000,0.000000,0.000000}%
\pgfsetstrokecolor{currentstroke}%
\pgfsetdash{}{0pt}%
\pgfsys@defobject{currentmarker}{\pgfqpoint{-0.048611in}{0.000000in}}{\pgfqpoint{0.000000in}{0.000000in}}{%
\pgfpathmoveto{\pgfqpoint{0.000000in}{0.000000in}}%
\pgfpathlineto{\pgfqpoint{-0.048611in}{0.000000in}}%
\pgfusepath{stroke,fill}%
}%
\begin{pgfscope}%
\pgfsys@transformshift{0.966250in}{1.158300in}%
\pgfsys@useobject{currentmarker}{}%
\end{pgfscope}%
\end{pgfscope}%
\begin{pgfscope}%
\pgftext[x=0.355044in,y=1.105538in,left,base]{\sffamily\fontsize{10.000000}{12.000000}\selectfont -0.002}%
\end{pgfscope}%
\begin{pgfscope}%
\pgfsetbuttcap%
\pgfsetroundjoin%
\definecolor{currentfill}{rgb}{0.000000,0.000000,0.000000}%
\pgfsetfillcolor{currentfill}%
\pgfsetlinewidth{0.803000pt}%
\definecolor{currentstroke}{rgb}{0.000000,0.000000,0.000000}%
\pgfsetstrokecolor{currentstroke}%
\pgfsetdash{}{0pt}%
\pgfsys@defobject{currentmarker}{\pgfqpoint{-0.048611in}{0.000000in}}{\pgfqpoint{0.000000in}{0.000000in}}{%
\pgfpathmoveto{\pgfqpoint{0.000000in}{0.000000in}}%
\pgfpathlineto{\pgfqpoint{-0.048611in}{0.000000in}}%
\pgfusepath{stroke,fill}%
}%
\begin{pgfscope}%
\pgfsys@transformshift{0.966250in}{1.782000in}%
\pgfsys@useobject{currentmarker}{}%
\end{pgfscope}%
\end{pgfscope}%
\begin{pgfscope}%
\pgftext[x=0.355044in,y=1.729238in,left,base]{\sffamily\fontsize{10.000000}{12.000000}\selectfont -0.001}%
\end{pgfscope}%
\begin{pgfscope}%
\pgfsetbuttcap%
\pgfsetroundjoin%
\definecolor{currentfill}{rgb}{0.000000,0.000000,0.000000}%
\pgfsetfillcolor{currentfill}%
\pgfsetlinewidth{0.803000pt}%
\definecolor{currentstroke}{rgb}{0.000000,0.000000,0.000000}%
\pgfsetstrokecolor{currentstroke}%
\pgfsetdash{}{0pt}%
\pgfsys@defobject{currentmarker}{\pgfqpoint{-0.048611in}{0.000000in}}{\pgfqpoint{0.000000in}{0.000000in}}{%
\pgfpathmoveto{\pgfqpoint{0.000000in}{0.000000in}}%
\pgfpathlineto{\pgfqpoint{-0.048611in}{0.000000in}}%
\pgfusepath{stroke,fill}%
}%
\begin{pgfscope}%
\pgfsys@transformshift{0.966250in}{2.405700in}%
\pgfsys@useobject{currentmarker}{}%
\end{pgfscope}%
\end{pgfscope}%
\begin{pgfscope}%
\pgftext[x=0.471418in,y=2.352938in,left,base]{\sffamily\fontsize{10.000000}{12.000000}\selectfont 0.000}%
\end{pgfscope}%
\begin{pgfscope}%
\pgfsetbuttcap%
\pgfsetroundjoin%
\definecolor{currentfill}{rgb}{0.000000,0.000000,0.000000}%
\pgfsetfillcolor{currentfill}%
\pgfsetlinewidth{0.803000pt}%
\definecolor{currentstroke}{rgb}{0.000000,0.000000,0.000000}%
\pgfsetstrokecolor{currentstroke}%
\pgfsetdash{}{0pt}%
\pgfsys@defobject{currentmarker}{\pgfqpoint{-0.048611in}{0.000000in}}{\pgfqpoint{0.000000in}{0.000000in}}{%
\pgfpathmoveto{\pgfqpoint{0.000000in}{0.000000in}}%
\pgfpathlineto{\pgfqpoint{-0.048611in}{0.000000in}}%
\pgfusepath{stroke,fill}%
}%
\begin{pgfscope}%
\pgfsys@transformshift{0.966250in}{3.029400in}%
\pgfsys@useobject{currentmarker}{}%
\end{pgfscope}%
\end{pgfscope}%
\begin{pgfscope}%
\pgftext[x=0.471418in,y=2.976638in,left,base]{\sffamily\fontsize{10.000000}{12.000000}\selectfont 0.001}%
\end{pgfscope}%
\begin{pgfscope}%
\pgfsetbuttcap%
\pgfsetroundjoin%
\definecolor{currentfill}{rgb}{0.000000,0.000000,0.000000}%
\pgfsetfillcolor{currentfill}%
\pgfsetlinewidth{0.803000pt}%
\definecolor{currentstroke}{rgb}{0.000000,0.000000,0.000000}%
\pgfsetstrokecolor{currentstroke}%
\pgfsetdash{}{0pt}%
\pgfsys@defobject{currentmarker}{\pgfqpoint{-0.048611in}{0.000000in}}{\pgfqpoint{0.000000in}{0.000000in}}{%
\pgfpathmoveto{\pgfqpoint{0.000000in}{0.000000in}}%
\pgfpathlineto{\pgfqpoint{-0.048611in}{0.000000in}}%
\pgfusepath{stroke,fill}%
}%
\begin{pgfscope}%
\pgfsys@transformshift{0.966250in}{3.653100in}%
\pgfsys@useobject{currentmarker}{}%
\end{pgfscope}%
\end{pgfscope}%
\begin{pgfscope}%
\pgftext[x=0.471418in,y=3.600338in,left,base]{\sffamily\fontsize{10.000000}{12.000000}\selectfont 0.002}%
\end{pgfscope}%
\begin{pgfscope}%
\pgfsetbuttcap%
\pgfsetroundjoin%
\definecolor{currentfill}{rgb}{0.000000,0.000000,0.000000}%
\pgfsetfillcolor{currentfill}%
\pgfsetlinewidth{0.803000pt}%
\definecolor{currentstroke}{rgb}{0.000000,0.000000,0.000000}%
\pgfsetstrokecolor{currentstroke}%
\pgfsetdash{}{0pt}%
\pgfsys@defobject{currentmarker}{\pgfqpoint{-0.048611in}{0.000000in}}{\pgfqpoint{0.000000in}{0.000000in}}{%
\pgfpathmoveto{\pgfqpoint{0.000000in}{0.000000in}}%
\pgfpathlineto{\pgfqpoint{-0.048611in}{0.000000in}}%
\pgfusepath{stroke,fill}%
}%
\begin{pgfscope}%
\pgfsys@transformshift{0.966250in}{4.276800in}%
\pgfsys@useobject{currentmarker}{}%
\end{pgfscope}%
\end{pgfscope}%
\begin{pgfscope}%
\pgftext[x=0.471418in,y=4.224038in,left,base]{\sffamily\fontsize{10.000000}{12.000000}\selectfont 0.003}%
\end{pgfscope}%
\begin{pgfscope}%
\pgftext[x=0.299488in,y=2.405700in,,bottom,rotate=90.000000]{\sffamily\fontsize{16.000000}{19.200000}\selectfont \(\displaystyle Poynting\) \(\displaystyle vector\)}%
\end{pgfscope}%
\begin{pgfscope}%
\pgfpathrectangle{\pgfqpoint{0.966250in}{0.534600in}}{\pgfqpoint{5.990750in}{3.742200in}} %
\pgfusepath{clip}%
\pgfsetrectcap%
\pgfsetroundjoin%
\pgfsetlinewidth{1.505625pt}%
\definecolor{currentstroke}{rgb}{0.000000,0.000000,0.000000}%
\pgfsetstrokecolor{currentstroke}%
\pgfsetdash{}{0pt}%
\pgfpathmoveto{\pgfqpoint{0.966250in}{2.405700in}}%
\pgfpathlineto{\pgfqpoint{1.106434in}{2.404563in}}%
\pgfpathlineto{\pgfqpoint{1.120412in}{2.404780in}}%
\pgfpathlineto{\pgfqpoint{1.142777in}{2.405525in}}%
\pgfpathlineto{\pgfqpoint{1.170335in}{2.405419in}}%
\pgfpathlineto{\pgfqpoint{1.202685in}{2.405526in}}%
\pgfpathlineto{\pgfqpoint{1.413160in}{2.405660in}}%
\pgfpathlineto{\pgfqpoint{1.651991in}{2.405699in}}%
\pgfpathlineto{\pgfqpoint{2.474321in}{2.405700in}}%
\pgfpathlineto{\pgfqpoint{3.590997in}{2.406762in}}%
\pgfpathlineto{\pgfqpoint{3.594192in}{2.410289in}}%
\pgfpathlineto{\pgfqpoint{3.598186in}{2.419071in}}%
\pgfpathlineto{\pgfqpoint{3.604976in}{2.434775in}}%
\pgfpathlineto{\pgfqpoint{3.606973in}{2.434903in}}%
\pgfpathlineto{\pgfqpoint{3.608970in}{2.431564in}}%
\pgfpathlineto{\pgfqpoint{3.612165in}{2.420142in}}%
\pgfpathlineto{\pgfqpoint{3.616558in}{2.405875in}}%
\pgfpathlineto{\pgfqpoint{3.617756in}{2.406232in}}%
\pgfpathlineto{\pgfqpoint{3.619353in}{2.411860in}}%
\pgfpathlineto{\pgfqpoint{3.621350in}{2.428947in}}%
\pgfpathlineto{\pgfqpoint{3.624545in}{2.480648in}}%
\pgfpathlineto{\pgfqpoint{3.633332in}{2.645679in}}%
\pgfpathlineto{\pgfqpoint{3.634131in}{2.647973in}}%
\pgfpathlineto{\pgfqpoint{3.634530in}{2.647722in}}%
\pgfpathlineto{\pgfqpoint{3.635728in}{2.641208in}}%
\pgfpathlineto{\pgfqpoint{3.637725in}{2.611663in}}%
\pgfpathlineto{\pgfqpoint{3.641719in}{2.504372in}}%
\pgfpathlineto{\pgfqpoint{3.645713in}{2.412812in}}%
\pgfpathlineto{\pgfqpoint{3.646911in}{2.405577in}}%
\pgfpathlineto{\pgfqpoint{3.647310in}{2.406164in}}%
\pgfpathlineto{\pgfqpoint{3.648508in}{2.417827in}}%
\pgfpathlineto{\pgfqpoint{3.650505in}{2.472034in}}%
\pgfpathlineto{\pgfqpoint{3.653700in}{2.641391in}}%
\pgfpathlineto{\pgfqpoint{3.661688in}{3.105925in}}%
\pgfpathlineto{\pgfqpoint{3.662886in}{3.120641in}}%
\pgfpathlineto{\pgfqpoint{3.663286in}{3.120398in}}%
\pgfpathlineto{\pgfqpoint{3.664484in}{3.103908in}}%
\pgfpathlineto{\pgfqpoint{3.666481in}{3.026154in}}%
\pgfpathlineto{\pgfqpoint{3.670075in}{2.772231in}}%
\pgfpathlineto{\pgfqpoint{3.675267in}{2.433047in}}%
\pgfpathlineto{\pgfqpoint{3.676865in}{2.405430in}}%
\pgfpathlineto{\pgfqpoint{3.677264in}{2.406727in}}%
\pgfpathlineto{\pgfqpoint{3.678462in}{2.431340in}}%
\pgfpathlineto{\pgfqpoint{3.680459in}{2.540407in}}%
\pgfpathlineto{\pgfqpoint{3.684054in}{2.904727in}}%
\pgfpathlineto{\pgfqpoint{3.690444in}{3.546804in}}%
\pgfpathlineto{\pgfqpoint{3.692441in}{3.595933in}}%
\pgfpathlineto{\pgfqpoint{3.693239in}{3.586243in}}%
\pgfpathlineto{\pgfqpoint{3.694837in}{3.517004in}}%
\pgfpathlineto{\pgfqpoint{3.697633in}{3.260040in}}%
\pgfpathlineto{\pgfqpoint{3.705620in}{2.428069in}}%
\pgfpathlineto{\pgfqpoint{3.706818in}{2.405318in}}%
\pgfpathlineto{\pgfqpoint{3.707218in}{2.407156in}}%
\pgfpathlineto{\pgfqpoint{3.708416in}{2.441315in}}%
\pgfpathlineto{\pgfqpoint{3.710413in}{2.589164in}}%
\pgfpathlineto{\pgfqpoint{3.714007in}{3.064198in}}%
\pgfpathlineto{\pgfqpoint{3.719998in}{3.816092in}}%
\pgfpathlineto{\pgfqpoint{3.721995in}{3.882905in}}%
\pgfpathlineto{\pgfqpoint{3.722794in}{3.873577in}}%
\pgfpathlineto{\pgfqpoint{3.724391in}{3.793793in}}%
\pgfpathlineto{\pgfqpoint{3.727187in}{3.487351in}}%
\pgfpathlineto{\pgfqpoint{3.735574in}{2.431407in}}%
\pgfpathlineto{\pgfqpoint{3.736772in}{2.405261in}}%
\pgfpathlineto{\pgfqpoint{3.737172in}{2.407378in}}%
\pgfpathlineto{\pgfqpoint{3.738370in}{2.446256in}}%
\pgfpathlineto{\pgfqpoint{3.740367in}{2.612714in}}%
\pgfpathlineto{\pgfqpoint{3.744360in}{3.205060in}}%
\pgfpathlineto{\pgfqpoint{3.749552in}{3.914699in}}%
\pgfpathlineto{\pgfqpoint{3.751549in}{4.005158in}}%
\pgfpathlineto{\pgfqpoint{3.751949in}{4.006727in}}%
\pgfpathlineto{\pgfqpoint{3.752748in}{3.992994in}}%
\pgfpathlineto{\pgfqpoint{3.754345in}{3.900132in}}%
\pgfpathlineto{\pgfqpoint{3.757141in}{3.561850in}}%
\pgfpathlineto{\pgfqpoint{3.765528in}{2.432645in}}%
\pgfpathlineto{\pgfqpoint{3.766726in}{2.405238in}}%
\pgfpathlineto{\pgfqpoint{3.767125in}{2.407471in}}%
\pgfpathlineto{\pgfqpoint{3.768323in}{2.448133in}}%
\pgfpathlineto{\pgfqpoint{3.770320in}{2.621405in}}%
\pgfpathlineto{\pgfqpoint{3.774314in}{3.233389in}}%
\pgfpathlineto{\pgfqpoint{3.779506in}{3.957644in}}%
\pgfpathlineto{\pgfqpoint{3.781503in}{4.046931in}}%
\pgfpathlineto{\pgfqpoint{3.781903in}{4.047823in}}%
\pgfpathlineto{\pgfqpoint{3.782701in}{4.032345in}}%
\pgfpathlineto{\pgfqpoint{3.784299in}{3.934667in}}%
\pgfpathlineto{\pgfqpoint{3.787094in}{3.585411in}}%
\pgfpathlineto{\pgfqpoint{3.795482in}{2.432974in}}%
\pgfpathlineto{\pgfqpoint{3.796680in}{2.405230in}}%
\pgfpathlineto{\pgfqpoint{3.797079in}{2.407509in}}%
\pgfpathlineto{\pgfqpoint{3.798277in}{2.448723in}}%
\pgfpathlineto{\pgfqpoint{3.800274in}{2.623976in}}%
\pgfpathlineto{\pgfqpoint{3.804268in}{3.241309in}}%
\pgfpathlineto{\pgfqpoint{3.809460in}{3.968940in}}%
\pgfpathlineto{\pgfqpoint{3.811457in}{4.057654in}}%
\pgfpathlineto{\pgfqpoint{3.811856in}{4.058319in}}%
\pgfpathlineto{\pgfqpoint{3.812655in}{4.042291in}}%
\pgfpathlineto{\pgfqpoint{3.814253in}{3.943208in}}%
\pgfpathlineto{\pgfqpoint{3.817048in}{3.590989in}}%
\pgfpathlineto{\pgfqpoint{3.825036in}{2.453609in}}%
\pgfpathlineto{\pgfqpoint{3.826633in}{2.405227in}}%
\pgfpathlineto{\pgfqpoint{3.827033in}{2.407530in}}%
\pgfpathlineto{\pgfqpoint{3.828231in}{2.448903in}}%
\pgfpathlineto{\pgfqpoint{3.830228in}{2.624629in}}%
\pgfpathlineto{\pgfqpoint{3.834222in}{3.243063in}}%
\pgfpathlineto{\pgfqpoint{3.839414in}{3.971135in}}%
\pgfpathlineto{\pgfqpoint{3.841411in}{4.059619in}}%
\pgfpathlineto{\pgfqpoint{3.841810in}{4.060218in}}%
\pgfpathlineto{\pgfqpoint{3.842609in}{4.044041in}}%
\pgfpathlineto{\pgfqpoint{3.844206in}{3.944618in}}%
\pgfpathlineto{\pgfqpoint{3.847002in}{3.591770in}}%
\pgfpathlineto{\pgfqpoint{3.854990in}{2.453561in}}%
\pgfpathlineto{\pgfqpoint{3.856587in}{2.405225in}}%
\pgfpathlineto{\pgfqpoint{3.856987in}{2.407546in}}%
\pgfpathlineto{\pgfqpoint{3.858185in}{2.448986in}}%
\pgfpathlineto{\pgfqpoint{3.860182in}{2.624841in}}%
\pgfpathlineto{\pgfqpoint{3.864175in}{3.243467in}}%
\pgfpathlineto{\pgfqpoint{3.869367in}{3.971466in}}%
\pgfpathlineto{\pgfqpoint{3.871364in}{4.059839in}}%
\pgfpathlineto{\pgfqpoint{3.871764in}{4.060413in}}%
\pgfpathlineto{\pgfqpoint{3.872563in}{4.044186in}}%
\pgfpathlineto{\pgfqpoint{3.874160in}{3.944662in}}%
\pgfpathlineto{\pgfqpoint{3.876956in}{3.591671in}}%
\pgfpathlineto{\pgfqpoint{3.884943in}{2.453491in}}%
\pgfpathlineto{\pgfqpoint{3.886541in}{2.405223in}}%
\pgfpathlineto{\pgfqpoint{3.886940in}{2.407562in}}%
\pgfpathlineto{\pgfqpoint{3.888138in}{2.449055in}}%
\pgfpathlineto{\pgfqpoint{3.890135in}{2.624988in}}%
\pgfpathlineto{\pgfqpoint{3.894129in}{3.243689in}}%
\pgfpathlineto{\pgfqpoint{3.899321in}{3.971572in}}%
\pgfpathlineto{\pgfqpoint{3.901318in}{4.059858in}}%
\pgfpathlineto{\pgfqpoint{3.901718in}{4.060414in}}%
\pgfpathlineto{\pgfqpoint{3.902516in}{4.044151in}}%
\pgfpathlineto{\pgfqpoint{3.904114in}{3.944560in}}%
\pgfpathlineto{\pgfqpoint{3.906909in}{3.591484in}}%
\pgfpathlineto{\pgfqpoint{3.914897in}{2.453420in}}%
\pgfpathlineto{\pgfqpoint{3.916495in}{2.405221in}}%
\pgfpathlineto{\pgfqpoint{3.916894in}{2.407578in}}%
\pgfpathlineto{\pgfqpoint{3.918092in}{2.449123in}}%
\pgfpathlineto{\pgfqpoint{3.920089in}{2.625132in}}%
\pgfpathlineto{\pgfqpoint{3.924083in}{3.243901in}}%
\pgfpathlineto{\pgfqpoint{3.929275in}{3.971667in}}%
\pgfpathlineto{\pgfqpoint{3.931272in}{4.059869in}}%
\pgfpathlineto{\pgfqpoint{3.931671in}{4.060407in}}%
\pgfpathlineto{\pgfqpoint{3.932470in}{4.044109in}}%
\pgfpathlineto{\pgfqpoint{3.934068in}{3.944452in}}%
\pgfpathlineto{\pgfqpoint{3.936863in}{3.591293in}}%
\pgfpathlineto{\pgfqpoint{3.944851in}{2.453349in}}%
\pgfpathlineto{\pgfqpoint{3.946448in}{2.405220in}}%
\pgfpathlineto{\pgfqpoint{3.946848in}{2.407594in}}%
\pgfpathlineto{\pgfqpoint{3.948046in}{2.449191in}}%
\pgfpathlineto{\pgfqpoint{3.950043in}{2.625275in}}%
\pgfpathlineto{\pgfqpoint{3.954037in}{3.244112in}}%
\pgfpathlineto{\pgfqpoint{3.959229in}{3.971762in}}%
\pgfpathlineto{\pgfqpoint{3.961226in}{4.059879in}}%
\pgfpathlineto{\pgfqpoint{3.961625in}{4.060400in}}%
\pgfpathlineto{\pgfqpoint{3.962424in}{4.044067in}}%
\pgfpathlineto{\pgfqpoint{3.964021in}{3.944344in}}%
\pgfpathlineto{\pgfqpoint{3.966817in}{3.591103in}}%
\pgfpathlineto{\pgfqpoint{3.974805in}{2.453278in}}%
\pgfpathlineto{\pgfqpoint{3.976402in}{2.405218in}}%
\pgfpathlineto{\pgfqpoint{3.976802in}{2.407610in}}%
\pgfpathlineto{\pgfqpoint{3.978000in}{2.449259in}}%
\pgfpathlineto{\pgfqpoint{3.979997in}{2.625418in}}%
\pgfpathlineto{\pgfqpoint{3.983990in}{3.244321in}}%
\pgfpathlineto{\pgfqpoint{3.989182in}{3.971850in}}%
\pgfpathlineto{\pgfqpoint{3.991179in}{4.059879in}}%
\pgfpathlineto{\pgfqpoint{3.991579in}{4.060382in}}%
\pgfpathlineto{\pgfqpoint{3.992378in}{4.044012in}}%
\pgfpathlineto{\pgfqpoint{3.993975in}{3.944222in}}%
\pgfpathlineto{\pgfqpoint{3.996771in}{3.590897in}}%
\pgfpathlineto{\pgfqpoint{4.004758in}{2.453206in}}%
\pgfpathlineto{\pgfqpoint{4.006356in}{2.405217in}}%
\pgfpathlineto{\pgfqpoint{4.006755in}{2.407627in}}%
\pgfpathlineto{\pgfqpoint{4.007953in}{2.449325in}}%
\pgfpathlineto{\pgfqpoint{4.009950in}{2.625552in}}%
\pgfpathlineto{\pgfqpoint{4.013944in}{3.244474in}}%
\pgfpathlineto{\pgfqpoint{4.019136in}{3.971768in}}%
\pgfpathlineto{\pgfqpoint{4.021133in}{4.059666in}}%
\pgfpathlineto{\pgfqpoint{4.021533in}{4.060143in}}%
\pgfpathlineto{\pgfqpoint{4.022331in}{4.043725in}}%
\pgfpathlineto{\pgfqpoint{4.023929in}{3.943850in}}%
\pgfpathlineto{\pgfqpoint{4.026724in}{3.590451in}}%
\pgfpathlineto{\pgfqpoint{4.034712in}{2.453115in}}%
\pgfpathlineto{\pgfqpoint{4.036310in}{2.405215in}}%
\pgfpathlineto{\pgfqpoint{4.036709in}{2.407642in}}%
\pgfpathlineto{\pgfqpoint{4.037907in}{2.449373in}}%
\pgfpathlineto{\pgfqpoint{4.039904in}{2.625571in}}%
\pgfpathlineto{\pgfqpoint{4.043898in}{3.244051in}}%
\pgfpathlineto{\pgfqpoint{4.049090in}{3.970184in}}%
\pgfpathlineto{\pgfqpoint{4.051087in}{4.057657in}}%
\pgfpathlineto{\pgfqpoint{4.051486in}{4.058063in}}%
\pgfpathlineto{\pgfqpoint{4.052285in}{4.041523in}}%
\pgfpathlineto{\pgfqpoint{4.053883in}{3.941498in}}%
\pgfpathlineto{\pgfqpoint{4.056678in}{3.588204in}}%
\pgfpathlineto{\pgfqpoint{4.064666in}{2.452909in}}%
\pgfpathlineto{\pgfqpoint{4.066263in}{2.405215in}}%
\pgfpathlineto{\pgfqpoint{4.067062in}{2.415860in}}%
\pgfpathlineto{\pgfqpoint{4.068660in}{2.504590in}}%
\pgfpathlineto{\pgfqpoint{4.071455in}{2.842638in}}%
\pgfpathlineto{\pgfqpoint{4.079842in}{4.011876in}}%
\pgfpathlineto{\pgfqpoint{4.081440in}{4.046883in}}%
\pgfpathlineto{\pgfqpoint{4.082239in}{4.029965in}}%
\pgfpathlineto{\pgfqpoint{4.083836in}{3.929680in}}%
\pgfpathlineto{\pgfqpoint{4.086632in}{3.577665in}}%
\pgfpathlineto{\pgfqpoint{4.094620in}{2.452225in}}%
\pgfpathlineto{\pgfqpoint{4.096217in}{2.405220in}}%
\pgfpathlineto{\pgfqpoint{4.097016in}{2.415757in}}%
\pgfpathlineto{\pgfqpoint{4.098613in}{2.503207in}}%
\pgfpathlineto{\pgfqpoint{4.101409in}{2.835345in}}%
\pgfpathlineto{\pgfqpoint{4.109796in}{3.972447in}}%
\pgfpathlineto{\pgfqpoint{4.110994in}{4.004323in}}%
\pgfpathlineto{\pgfqpoint{4.111394in}{4.003757in}}%
\pgfpathlineto{\pgfqpoint{4.112592in}{3.968583in}}%
\pgfpathlineto{\pgfqpoint{4.114589in}{3.805786in}}%
\pgfpathlineto{\pgfqpoint{4.118583in}{3.215898in}}%
\pgfpathlineto{\pgfqpoint{4.123775in}{2.503986in}}%
\pgfpathlineto{\pgfqpoint{4.126171in}{2.405241in}}%
\pgfpathlineto{\pgfqpoint{4.126970in}{2.415320in}}%
\pgfpathlineto{\pgfqpoint{4.128567in}{2.498363in}}%
\pgfpathlineto{\pgfqpoint{4.131363in}{2.811270in}}%
\pgfpathlineto{\pgfqpoint{4.139750in}{3.853268in}}%
\pgfpathlineto{\pgfqpoint{4.140948in}{3.877559in}}%
\pgfpathlineto{\pgfqpoint{4.141348in}{3.875281in}}%
\pgfpathlineto{\pgfqpoint{4.142546in}{3.837666in}}%
\pgfpathlineto{\pgfqpoint{4.144543in}{3.680287in}}%
\pgfpathlineto{\pgfqpoint{4.148536in}{3.132925in}}%
\pgfpathlineto{\pgfqpoint{4.153728in}{2.491993in}}%
\pgfpathlineto{\pgfqpoint{4.156125in}{2.405298in}}%
\pgfpathlineto{\pgfqpoint{4.156923in}{2.414101in}}%
\pgfpathlineto{\pgfqpoint{4.158521in}{2.485626in}}%
\pgfpathlineto{\pgfqpoint{4.161317in}{2.750171in}}%
\pgfpathlineto{\pgfqpoint{4.168905in}{3.547055in}}%
\pgfpathlineto{\pgfqpoint{4.170502in}{3.585466in}}%
\pgfpathlineto{\pgfqpoint{4.170902in}{3.584580in}}%
\pgfpathlineto{\pgfqpoint{4.172100in}{3.557009in}}%
\pgfpathlineto{\pgfqpoint{4.174097in}{3.434697in}}%
\pgfpathlineto{\pgfqpoint{4.178091in}{3.003480in}}%
\pgfpathlineto{\pgfqpoint{4.183283in}{2.490081in}}%
\pgfpathlineto{\pgfqpoint{4.185679in}{2.407284in}}%
\pgfpathlineto{\pgfqpoint{4.186078in}{2.405415in}}%
\pgfpathlineto{\pgfqpoint{4.186478in}{2.406890in}}%
\pgfpathlineto{\pgfqpoint{4.187676in}{2.430466in}}%
\pgfpathlineto{\pgfqpoint{4.190072in}{2.550404in}}%
\pgfpathlineto{\pgfqpoint{4.199258in}{3.103915in}}%
\pgfpathlineto{\pgfqpoint{4.200057in}{3.107622in}}%
\pgfpathlineto{\pgfqpoint{4.200456in}{3.105620in}}%
\pgfpathlineto{\pgfqpoint{4.201654in}{3.084716in}}%
\pgfpathlineto{\pgfqpoint{4.204051in}{2.984327in}}%
\pgfpathlineto{\pgfqpoint{4.215233in}{2.408848in}}%
\pgfpathlineto{\pgfqpoint{4.216032in}{2.405570in}}%
\pgfpathlineto{\pgfqpoint{4.216432in}{2.406244in}}%
\pgfpathlineto{\pgfqpoint{4.217630in}{2.416619in}}%
\pgfpathlineto{\pgfqpoint{4.220026in}{2.466191in}}%
\pgfpathlineto{\pgfqpoint{4.227215in}{2.632382in}}%
\pgfpathlineto{\pgfqpoint{4.228812in}{2.639292in}}%
\pgfpathlineto{\pgfqpoint{4.229611in}{2.637256in}}%
\pgfpathlineto{\pgfqpoint{4.231209in}{2.623183in}}%
\pgfpathlineto{\pgfqpoint{4.234404in}{2.565788in}}%
\pgfpathlineto{\pgfqpoint{4.241593in}{2.428759in}}%
\pgfpathlineto{\pgfqpoint{4.244388in}{2.408397in}}%
\pgfpathlineto{\pgfqpoint{4.246385in}{2.405800in}}%
\pgfpathlineto{\pgfqpoint{4.247983in}{2.408714in}}%
\pgfpathlineto{\pgfqpoint{4.257169in}{2.433134in}}%
\pgfpathlineto{\pgfqpoint{4.259166in}{2.431337in}}%
\pgfpathlineto{\pgfqpoint{4.262760in}{2.423373in}}%
\pgfpathlineto{\pgfqpoint{4.269150in}{2.409467in}}%
\pgfpathlineto{\pgfqpoint{4.272745in}{2.406314in}}%
\pgfpathlineto{\pgfqpoint{4.276738in}{2.405717in}}%
\pgfpathlineto{\pgfqpoint{4.296708in}{2.405763in}}%
\pgfpathlineto{\pgfqpoint{4.352222in}{2.405700in}}%
\pgfpathlineto{\pgfqpoint{6.956601in}{2.405700in}}%
\pgfpathlineto{\pgfqpoint{6.956601in}{2.405700in}}%
\pgfusepath{stroke}%
\end{pgfscope}%
\begin{pgfscope}%
\pgfsetrectcap%
\pgfsetmiterjoin%
\pgfsetlinewidth{0.803000pt}%
\definecolor{currentstroke}{rgb}{0.000000,0.000000,0.000000}%
\pgfsetstrokecolor{currentstroke}%
\pgfsetdash{}{0pt}%
\pgfpathmoveto{\pgfqpoint{0.966250in}{0.534600in}}%
\pgfpathlineto{\pgfqpoint{0.966250in}{4.276800in}}%
\pgfusepath{stroke}%
\end{pgfscope}%
\begin{pgfscope}%
\pgfsetrectcap%
\pgfsetmiterjoin%
\pgfsetlinewidth{0.803000pt}%
\definecolor{currentstroke}{rgb}{0.000000,0.000000,0.000000}%
\pgfsetstrokecolor{currentstroke}%
\pgfsetdash{}{0pt}%
\pgfpathmoveto{\pgfqpoint{6.957000in}{0.534600in}}%
\pgfpathlineto{\pgfqpoint{6.957000in}{4.276800in}}%
\pgfusepath{stroke}%
\end{pgfscope}%
\begin{pgfscope}%
\pgfsetrectcap%
\pgfsetmiterjoin%
\pgfsetlinewidth{0.803000pt}%
\definecolor{currentstroke}{rgb}{0.000000,0.000000,0.000000}%
\pgfsetstrokecolor{currentstroke}%
\pgfsetdash{}{0pt}%
\pgfpathmoveto{\pgfqpoint{0.966250in}{0.534600in}}%
\pgfpathlineto{\pgfqpoint{6.957000in}{0.534600in}}%
\pgfusepath{stroke}%
\end{pgfscope}%
\begin{pgfscope}%
\pgfsetrectcap%
\pgfsetmiterjoin%
\pgfsetlinewidth{0.803000pt}%
\definecolor{currentstroke}{rgb}{0.000000,0.000000,0.000000}%
\pgfsetstrokecolor{currentstroke}%
\pgfsetdash{}{0pt}%
\pgfpathmoveto{\pgfqpoint{0.966250in}{4.276800in}}%
\pgfpathlineto{\pgfqpoint{6.957000in}{4.276800in}}%
\pgfusepath{stroke}%
\end{pgfscope}%
\begin{pgfscope}%
\pgftext[x=0.966250in,y=4.463910in,left,base]{\sffamily\fontsize{10.000000}{12.000000}\selectfont Iterations: 11940, Time: 0.112 ps, RXPWR: 0.0 percent, TXPWR: 100.0 percent}%
\end{pgfscope}%
\end{pgfpicture}%
\makeatother%
\endgroup%
}}
  \caption{Simulation results from task 4 using 20 pairs of high and low refractiv indices.}
  \label{fig:task4}
\end{figure}

\section{Task 5}\label{sec:5}
There is a difference between long and short pulses. There are two numerical components that affect the resolution of the simulation: sampling distance and pulse length.

The sampling distance can improves the resolution of them pulse and lower sampling distance (higher resolution) decreases the difference between the discrete function used in a computer and the continous function found in reality

The pulse length decreases the oscilating effect (from the cosine part of the wave) and makes the pule look more like a pure super-Gaussian function.

Another explanation could be that shord pulses typically contain more high frequency components with higher amplitudes that long pulses. The high frequency components may not be as strongly affected by the thickness of the layers.
\\
\noindent The simulation results are closer to the theoretical result if the pulse length increases. This can be seen when comparing the reflected and transmitted power in Figure~\ref{fig:task5_1} and Figure~\ref{fig:task5_2}.
\begin{figure}
  \centering
    \begin{subfigure}[b]{0.7\textwidth}
        \noindent\makebox[\textwidth]{\scalebox{0.7}{%% Creator: Matplotlib, PGF backend
%%
%% To include the figure in your LaTeX document, write
%%   \input{<filename>.pgf}
%%
%% Make sure the required packages are loaded in your preamble
%%   \usepackage{pgf}
%%
%% Figures using additional raster images can only be included by \input if
%% they are in the same directory as the main LaTeX file. For loading figures
%% from other directories you can use the `import` package
%%   \usepackage{import}
%% and then include the figures with
%%   \import{<path to file>}{<filename>.pgf}
%%
%% Matplotlib used the following preamble
%%   \usepackage{fontspec}
%%   \setmainfont{DejaVu Serif}
%%   \setsansfont{DejaVu Sans}
%%   \setmonofont{DejaVu Sans Mono}
%%
\begingroup%
\makeatletter%
\begin{pgfpicture}%
\pgfpathrectangle{\pgfpointorigin}{\pgfqpoint{7.620000in}{4.720000in}}%
\pgfusepath{use as bounding box, clip}%
\begin{pgfscope}%
\pgfsetbuttcap%
\pgfsetmiterjoin%
\definecolor{currentfill}{rgb}{1.000000,1.000000,1.000000}%
\pgfsetfillcolor{currentfill}%
\pgfsetlinewidth{0.000000pt}%
\definecolor{currentstroke}{rgb}{1.000000,1.000000,1.000000}%
\pgfsetstrokecolor{currentstroke}%
\pgfsetdash{}{0pt}%
\pgfpathmoveto{\pgfqpoint{0.000000in}{0.000000in}}%
\pgfpathlineto{\pgfqpoint{7.620000in}{0.000000in}}%
\pgfpathlineto{\pgfqpoint{7.620000in}{4.720000in}}%
\pgfpathlineto{\pgfqpoint{0.000000in}{4.720000in}}%
\pgfpathclose%
\pgfusepath{fill}%
\end{pgfscope}%
\begin{pgfscope}%
\pgfsetbuttcap%
\pgfsetmiterjoin%
\definecolor{currentfill}{rgb}{1.000000,1.000000,1.000000}%
\pgfsetfillcolor{currentfill}%
\pgfsetlinewidth{0.000000pt}%
\definecolor{currentstroke}{rgb}{0.000000,0.000000,0.000000}%
\pgfsetstrokecolor{currentstroke}%
\pgfsetstrokeopacity{0.000000}%
\pgfsetdash{}{0pt}%
\pgfpathmoveto{\pgfqpoint{0.952500in}{0.519200in}}%
\pgfpathlineto{\pgfqpoint{6.858000in}{0.519200in}}%
\pgfpathlineto{\pgfqpoint{6.858000in}{4.153600in}}%
\pgfpathlineto{\pgfqpoint{0.952500in}{4.153600in}}%
\pgfpathclose%
\pgfusepath{fill}%
\end{pgfscope}%
\begin{pgfscope}%
\pgfsetbuttcap%
\pgfsetroundjoin%
\definecolor{currentfill}{rgb}{0.000000,0.000000,0.000000}%
\pgfsetfillcolor{currentfill}%
\pgfsetlinewidth{0.803000pt}%
\definecolor{currentstroke}{rgb}{0.000000,0.000000,0.000000}%
\pgfsetstrokecolor{currentstroke}%
\pgfsetdash{}{0pt}%
\pgfsys@defobject{currentmarker}{\pgfqpoint{0.000000in}{-0.048611in}}{\pgfqpoint{0.000000in}{0.000000in}}{%
\pgfpathmoveto{\pgfqpoint{0.000000in}{0.000000in}}%
\pgfpathlineto{\pgfqpoint{0.000000in}{-0.048611in}}%
\pgfusepath{stroke,fill}%
}%
\begin{pgfscope}%
\pgfsys@transformshift{0.952500in}{0.519200in}%
\pgfsys@useobject{currentmarker}{}%
\end{pgfscope}%
\end{pgfscope}%
\begin{pgfscope}%
\pgftext[x=0.952500in,y=0.421978in,,top]{\sffamily\fontsize{10.000000}{12.000000}\selectfont 0}%
\end{pgfscope}%
\begin{pgfscope}%
\pgfsetbuttcap%
\pgfsetroundjoin%
\definecolor{currentfill}{rgb}{0.000000,0.000000,0.000000}%
\pgfsetfillcolor{currentfill}%
\pgfsetlinewidth{0.803000pt}%
\definecolor{currentstroke}{rgb}{0.000000,0.000000,0.000000}%
\pgfsetstrokecolor{currentstroke}%
\pgfsetdash{}{0pt}%
\pgfsys@defobject{currentmarker}{\pgfqpoint{0.000000in}{-0.048611in}}{\pgfqpoint{0.000000in}{0.000000in}}{%
\pgfpathmoveto{\pgfqpoint{0.000000in}{0.000000in}}%
\pgfpathlineto{\pgfqpoint{0.000000in}{-0.048611in}}%
\pgfusepath{stroke,fill}%
}%
\begin{pgfscope}%
\pgfsys@transformshift{1.885438in}{0.519200in}%
\pgfsys@useobject{currentmarker}{}%
\end{pgfscope}%
\end{pgfscope}%
\begin{pgfscope}%
\pgftext[x=1.885438in,y=0.421978in,,top]{\sffamily\fontsize{10.000000}{12.000000}\selectfont 20}%
\end{pgfscope}%
\begin{pgfscope}%
\pgfsetbuttcap%
\pgfsetroundjoin%
\definecolor{currentfill}{rgb}{0.000000,0.000000,0.000000}%
\pgfsetfillcolor{currentfill}%
\pgfsetlinewidth{0.803000pt}%
\definecolor{currentstroke}{rgb}{0.000000,0.000000,0.000000}%
\pgfsetstrokecolor{currentstroke}%
\pgfsetdash{}{0pt}%
\pgfsys@defobject{currentmarker}{\pgfqpoint{0.000000in}{-0.048611in}}{\pgfqpoint{0.000000in}{0.000000in}}{%
\pgfpathmoveto{\pgfqpoint{0.000000in}{0.000000in}}%
\pgfpathlineto{\pgfqpoint{0.000000in}{-0.048611in}}%
\pgfusepath{stroke,fill}%
}%
\begin{pgfscope}%
\pgfsys@transformshift{2.818377in}{0.519200in}%
\pgfsys@useobject{currentmarker}{}%
\end{pgfscope}%
\end{pgfscope}%
\begin{pgfscope}%
\pgftext[x=2.818377in,y=0.421978in,,top]{\sffamily\fontsize{10.000000}{12.000000}\selectfont 40}%
\end{pgfscope}%
\begin{pgfscope}%
\pgfsetbuttcap%
\pgfsetroundjoin%
\definecolor{currentfill}{rgb}{0.000000,0.000000,0.000000}%
\pgfsetfillcolor{currentfill}%
\pgfsetlinewidth{0.803000pt}%
\definecolor{currentstroke}{rgb}{0.000000,0.000000,0.000000}%
\pgfsetstrokecolor{currentstroke}%
\pgfsetdash{}{0pt}%
\pgfsys@defobject{currentmarker}{\pgfqpoint{0.000000in}{-0.048611in}}{\pgfqpoint{0.000000in}{0.000000in}}{%
\pgfpathmoveto{\pgfqpoint{0.000000in}{0.000000in}}%
\pgfpathlineto{\pgfqpoint{0.000000in}{-0.048611in}}%
\pgfusepath{stroke,fill}%
}%
\begin{pgfscope}%
\pgfsys@transformshift{3.751315in}{0.519200in}%
\pgfsys@useobject{currentmarker}{}%
\end{pgfscope}%
\end{pgfscope}%
\begin{pgfscope}%
\pgftext[x=3.751315in,y=0.421978in,,top]{\sffamily\fontsize{10.000000}{12.000000}\selectfont 60}%
\end{pgfscope}%
\begin{pgfscope}%
\pgfsetbuttcap%
\pgfsetroundjoin%
\definecolor{currentfill}{rgb}{0.000000,0.000000,0.000000}%
\pgfsetfillcolor{currentfill}%
\pgfsetlinewidth{0.803000pt}%
\definecolor{currentstroke}{rgb}{0.000000,0.000000,0.000000}%
\pgfsetstrokecolor{currentstroke}%
\pgfsetdash{}{0pt}%
\pgfsys@defobject{currentmarker}{\pgfqpoint{0.000000in}{-0.048611in}}{\pgfqpoint{0.000000in}{0.000000in}}{%
\pgfpathmoveto{\pgfqpoint{0.000000in}{0.000000in}}%
\pgfpathlineto{\pgfqpoint{0.000000in}{-0.048611in}}%
\pgfusepath{stroke,fill}%
}%
\begin{pgfscope}%
\pgfsys@transformshift{4.684254in}{0.519200in}%
\pgfsys@useobject{currentmarker}{}%
\end{pgfscope}%
\end{pgfscope}%
\begin{pgfscope}%
\pgftext[x=4.684254in,y=0.421978in,,top]{\sffamily\fontsize{10.000000}{12.000000}\selectfont 80}%
\end{pgfscope}%
\begin{pgfscope}%
\pgfsetbuttcap%
\pgfsetroundjoin%
\definecolor{currentfill}{rgb}{0.000000,0.000000,0.000000}%
\pgfsetfillcolor{currentfill}%
\pgfsetlinewidth{0.803000pt}%
\definecolor{currentstroke}{rgb}{0.000000,0.000000,0.000000}%
\pgfsetstrokecolor{currentstroke}%
\pgfsetdash{}{0pt}%
\pgfsys@defobject{currentmarker}{\pgfqpoint{0.000000in}{-0.048611in}}{\pgfqpoint{0.000000in}{0.000000in}}{%
\pgfpathmoveto{\pgfqpoint{0.000000in}{0.000000in}}%
\pgfpathlineto{\pgfqpoint{0.000000in}{-0.048611in}}%
\pgfusepath{stroke,fill}%
}%
\begin{pgfscope}%
\pgfsys@transformshift{5.617192in}{0.519200in}%
\pgfsys@useobject{currentmarker}{}%
\end{pgfscope}%
\end{pgfscope}%
\begin{pgfscope}%
\pgftext[x=5.617192in,y=0.421978in,,top]{\sffamily\fontsize{10.000000}{12.000000}\selectfont 100}%
\end{pgfscope}%
\begin{pgfscope}%
\pgfsetbuttcap%
\pgfsetroundjoin%
\definecolor{currentfill}{rgb}{0.000000,0.000000,0.000000}%
\pgfsetfillcolor{currentfill}%
\pgfsetlinewidth{0.803000pt}%
\definecolor{currentstroke}{rgb}{0.000000,0.000000,0.000000}%
\pgfsetstrokecolor{currentstroke}%
\pgfsetdash{}{0pt}%
\pgfsys@defobject{currentmarker}{\pgfqpoint{0.000000in}{-0.048611in}}{\pgfqpoint{0.000000in}{0.000000in}}{%
\pgfpathmoveto{\pgfqpoint{0.000000in}{0.000000in}}%
\pgfpathlineto{\pgfqpoint{0.000000in}{-0.048611in}}%
\pgfusepath{stroke,fill}%
}%
\begin{pgfscope}%
\pgfsys@transformshift{6.550130in}{0.519200in}%
\pgfsys@useobject{currentmarker}{}%
\end{pgfscope}%
\end{pgfscope}%
\begin{pgfscope}%
\pgftext[x=6.550130in,y=0.421978in,,top]{\sffamily\fontsize{10.000000}{12.000000}\selectfont 120}%
\end{pgfscope}%
\begin{pgfscope}%
\pgftext[x=3.905250in,y=0.232009in,,top]{\sffamily\fontsize{16.000000}{19.200000}\selectfont \(\displaystyle z-position [\mu m]\)}%
\end{pgfscope}%
\begin{pgfscope}%
\pgfsetbuttcap%
\pgfsetroundjoin%
\definecolor{currentfill}{rgb}{0.000000,0.000000,0.000000}%
\pgfsetfillcolor{currentfill}%
\pgfsetlinewidth{0.803000pt}%
\definecolor{currentstroke}{rgb}{0.000000,0.000000,0.000000}%
\pgfsetstrokecolor{currentstroke}%
\pgfsetdash{}{0pt}%
\pgfsys@defobject{currentmarker}{\pgfqpoint{-0.048611in}{0.000000in}}{\pgfqpoint{0.000000in}{0.000000in}}{%
\pgfpathmoveto{\pgfqpoint{0.000000in}{0.000000in}}%
\pgfpathlineto{\pgfqpoint{-0.048611in}{0.000000in}}%
\pgfusepath{stroke,fill}%
}%
\begin{pgfscope}%
\pgfsys@transformshift{0.952500in}{0.519200in}%
\pgfsys@useobject{currentmarker}{}%
\end{pgfscope}%
\end{pgfscope}%
\begin{pgfscope}%
\pgftext[x=0.341294in,y=0.466438in,left,base]{\sffamily\fontsize{10.000000}{12.000000}\selectfont -0.003}%
\end{pgfscope}%
\begin{pgfscope}%
\pgfsetbuttcap%
\pgfsetroundjoin%
\definecolor{currentfill}{rgb}{0.000000,0.000000,0.000000}%
\pgfsetfillcolor{currentfill}%
\pgfsetlinewidth{0.803000pt}%
\definecolor{currentstroke}{rgb}{0.000000,0.000000,0.000000}%
\pgfsetstrokecolor{currentstroke}%
\pgfsetdash{}{0pt}%
\pgfsys@defobject{currentmarker}{\pgfqpoint{-0.048611in}{0.000000in}}{\pgfqpoint{0.000000in}{0.000000in}}{%
\pgfpathmoveto{\pgfqpoint{0.000000in}{0.000000in}}%
\pgfpathlineto{\pgfqpoint{-0.048611in}{0.000000in}}%
\pgfusepath{stroke,fill}%
}%
\begin{pgfscope}%
\pgfsys@transformshift{0.952500in}{1.124933in}%
\pgfsys@useobject{currentmarker}{}%
\end{pgfscope}%
\end{pgfscope}%
\begin{pgfscope}%
\pgftext[x=0.341294in,y=1.072172in,left,base]{\sffamily\fontsize{10.000000}{12.000000}\selectfont -0.002}%
\end{pgfscope}%
\begin{pgfscope}%
\pgfsetbuttcap%
\pgfsetroundjoin%
\definecolor{currentfill}{rgb}{0.000000,0.000000,0.000000}%
\pgfsetfillcolor{currentfill}%
\pgfsetlinewidth{0.803000pt}%
\definecolor{currentstroke}{rgb}{0.000000,0.000000,0.000000}%
\pgfsetstrokecolor{currentstroke}%
\pgfsetdash{}{0pt}%
\pgfsys@defobject{currentmarker}{\pgfqpoint{-0.048611in}{0.000000in}}{\pgfqpoint{0.000000in}{0.000000in}}{%
\pgfpathmoveto{\pgfqpoint{0.000000in}{0.000000in}}%
\pgfpathlineto{\pgfqpoint{-0.048611in}{0.000000in}}%
\pgfusepath{stroke,fill}%
}%
\begin{pgfscope}%
\pgfsys@transformshift{0.952500in}{1.730667in}%
\pgfsys@useobject{currentmarker}{}%
\end{pgfscope}%
\end{pgfscope}%
\begin{pgfscope}%
\pgftext[x=0.341294in,y=1.677905in,left,base]{\sffamily\fontsize{10.000000}{12.000000}\selectfont -0.001}%
\end{pgfscope}%
\begin{pgfscope}%
\pgfsetbuttcap%
\pgfsetroundjoin%
\definecolor{currentfill}{rgb}{0.000000,0.000000,0.000000}%
\pgfsetfillcolor{currentfill}%
\pgfsetlinewidth{0.803000pt}%
\definecolor{currentstroke}{rgb}{0.000000,0.000000,0.000000}%
\pgfsetstrokecolor{currentstroke}%
\pgfsetdash{}{0pt}%
\pgfsys@defobject{currentmarker}{\pgfqpoint{-0.048611in}{0.000000in}}{\pgfqpoint{0.000000in}{0.000000in}}{%
\pgfpathmoveto{\pgfqpoint{0.000000in}{0.000000in}}%
\pgfpathlineto{\pgfqpoint{-0.048611in}{0.000000in}}%
\pgfusepath{stroke,fill}%
}%
\begin{pgfscope}%
\pgfsys@transformshift{0.952500in}{2.336400in}%
\pgfsys@useobject{currentmarker}{}%
\end{pgfscope}%
\end{pgfscope}%
\begin{pgfscope}%
\pgftext[x=0.457668in,y=2.283638in,left,base]{\sffamily\fontsize{10.000000}{12.000000}\selectfont 0.000}%
\end{pgfscope}%
\begin{pgfscope}%
\pgfsetbuttcap%
\pgfsetroundjoin%
\definecolor{currentfill}{rgb}{0.000000,0.000000,0.000000}%
\pgfsetfillcolor{currentfill}%
\pgfsetlinewidth{0.803000pt}%
\definecolor{currentstroke}{rgb}{0.000000,0.000000,0.000000}%
\pgfsetstrokecolor{currentstroke}%
\pgfsetdash{}{0pt}%
\pgfsys@defobject{currentmarker}{\pgfqpoint{-0.048611in}{0.000000in}}{\pgfqpoint{0.000000in}{0.000000in}}{%
\pgfpathmoveto{\pgfqpoint{0.000000in}{0.000000in}}%
\pgfpathlineto{\pgfqpoint{-0.048611in}{0.000000in}}%
\pgfusepath{stroke,fill}%
}%
\begin{pgfscope}%
\pgfsys@transformshift{0.952500in}{2.942133in}%
\pgfsys@useobject{currentmarker}{}%
\end{pgfscope}%
\end{pgfscope}%
\begin{pgfscope}%
\pgftext[x=0.457668in,y=2.889372in,left,base]{\sffamily\fontsize{10.000000}{12.000000}\selectfont 0.001}%
\end{pgfscope}%
\begin{pgfscope}%
\pgfsetbuttcap%
\pgfsetroundjoin%
\definecolor{currentfill}{rgb}{0.000000,0.000000,0.000000}%
\pgfsetfillcolor{currentfill}%
\pgfsetlinewidth{0.803000pt}%
\definecolor{currentstroke}{rgb}{0.000000,0.000000,0.000000}%
\pgfsetstrokecolor{currentstroke}%
\pgfsetdash{}{0pt}%
\pgfsys@defobject{currentmarker}{\pgfqpoint{-0.048611in}{0.000000in}}{\pgfqpoint{0.000000in}{0.000000in}}{%
\pgfpathmoveto{\pgfqpoint{0.000000in}{0.000000in}}%
\pgfpathlineto{\pgfqpoint{-0.048611in}{0.000000in}}%
\pgfusepath{stroke,fill}%
}%
\begin{pgfscope}%
\pgfsys@transformshift{0.952500in}{3.547867in}%
\pgfsys@useobject{currentmarker}{}%
\end{pgfscope}%
\end{pgfscope}%
\begin{pgfscope}%
\pgftext[x=0.457668in,y=3.495105in,left,base]{\sffamily\fontsize{10.000000}{12.000000}\selectfont 0.002}%
\end{pgfscope}%
\begin{pgfscope}%
\pgfsetbuttcap%
\pgfsetroundjoin%
\definecolor{currentfill}{rgb}{0.000000,0.000000,0.000000}%
\pgfsetfillcolor{currentfill}%
\pgfsetlinewidth{0.803000pt}%
\definecolor{currentstroke}{rgb}{0.000000,0.000000,0.000000}%
\pgfsetstrokecolor{currentstroke}%
\pgfsetdash{}{0pt}%
\pgfsys@defobject{currentmarker}{\pgfqpoint{-0.048611in}{0.000000in}}{\pgfqpoint{0.000000in}{0.000000in}}{%
\pgfpathmoveto{\pgfqpoint{0.000000in}{0.000000in}}%
\pgfpathlineto{\pgfqpoint{-0.048611in}{0.000000in}}%
\pgfusepath{stroke,fill}%
}%
\begin{pgfscope}%
\pgfsys@transformshift{0.952500in}{4.153600in}%
\pgfsys@useobject{currentmarker}{}%
\end{pgfscope}%
\end{pgfscope}%
\begin{pgfscope}%
\pgftext[x=0.457668in,y=4.100838in,left,base]{\sffamily\fontsize{10.000000}{12.000000}\selectfont 0.003}%
\end{pgfscope}%
\begin{pgfscope}%
\pgftext[x=0.285738in,y=2.336400in,,bottom,rotate=90.000000]{\sffamily\fontsize{16.000000}{19.200000}\selectfont \(\displaystyle Poynting\) \(\displaystyle vector\)}%
\end{pgfscope}%
\begin{pgfscope}%
\pgfpathrectangle{\pgfqpoint{0.952500in}{0.519200in}}{\pgfqpoint{5.905500in}{3.634400in}} %
\pgfusepath{clip}%
\pgfsetrectcap%
\pgfsetroundjoin%
\pgfsetlinewidth{1.505625pt}%
\definecolor{currentstroke}{rgb}{0.000000,0.000000,0.000000}%
\pgfsetstrokecolor{currentstroke}%
\pgfsetdash{}{0pt}%
\pgfpathmoveto{\pgfqpoint{0.952500in}{2.336400in}}%
\pgfpathlineto{\pgfqpoint{1.428680in}{2.335306in}}%
\pgfpathlineto{\pgfqpoint{1.433405in}{2.334669in}}%
\pgfpathlineto{\pgfqpoint{1.438916in}{2.335544in}}%
\pgfpathlineto{\pgfqpoint{1.440885in}{2.331486in}}%
\pgfpathlineto{\pgfqpoint{1.446101in}{2.318092in}}%
\pgfpathlineto{\pgfqpoint{1.446495in}{2.318368in}}%
\pgfpathlineto{\pgfqpoint{1.447873in}{2.321580in}}%
\pgfpathlineto{\pgfqpoint{1.452204in}{2.336383in}}%
\pgfpathlineto{\pgfqpoint{1.452893in}{2.335316in}}%
\pgfpathlineto{\pgfqpoint{1.454271in}{2.327834in}}%
\pgfpathlineto{\pgfqpoint{1.457027in}{2.296511in}}%
\pgfpathlineto{\pgfqpoint{1.459979in}{2.272327in}}%
\pgfpathlineto{\pgfqpoint{1.460373in}{2.272337in}}%
\pgfpathlineto{\pgfqpoint{1.460668in}{2.272968in}}%
\pgfpathlineto{\pgfqpoint{1.461849in}{2.280679in}}%
\pgfpathlineto{\pgfqpoint{1.464901in}{2.322239in}}%
\pgfpathlineto{\pgfqpoint{1.466869in}{2.336400in}}%
\pgfpathlineto{\pgfqpoint{1.467066in}{2.336218in}}%
\pgfpathlineto{\pgfqpoint{1.467853in}{2.332014in}}%
\pgfpathlineto{\pgfqpoint{1.469428in}{2.307406in}}%
\pgfpathlineto{\pgfqpoint{1.474645in}{2.204236in}}%
\pgfpathlineto{\pgfqpoint{1.475137in}{2.205651in}}%
\pgfpathlineto{\pgfqpoint{1.476219in}{2.218887in}}%
\pgfpathlineto{\pgfqpoint{1.478778in}{2.286548in}}%
\pgfpathlineto{\pgfqpoint{1.481534in}{2.336344in}}%
\pgfpathlineto{\pgfqpoint{1.481731in}{2.336306in}}%
\pgfpathlineto{\pgfqpoint{1.481928in}{2.335671in}}%
\pgfpathlineto{\pgfqpoint{1.482814in}{2.325351in}}%
\pgfpathlineto{\pgfqpoint{1.484684in}{2.268399in}}%
\pgfpathlineto{\pgfqpoint{1.489212in}{2.128526in}}%
\pgfpathlineto{\pgfqpoint{1.489408in}{2.128651in}}%
\pgfpathlineto{\pgfqpoint{1.489605in}{2.129513in}}%
\pgfpathlineto{\pgfqpoint{1.490589in}{2.144624in}}%
\pgfpathlineto{\pgfqpoint{1.492755in}{2.226105in}}%
\pgfpathlineto{\pgfqpoint{1.496200in}{2.336046in}}%
\pgfpathlineto{\pgfqpoint{1.496397in}{2.336396in}}%
\pgfpathlineto{\pgfqpoint{1.496495in}{2.336248in}}%
\pgfpathlineto{\pgfqpoint{1.496495in}{2.336248in}}%
\pgfpathlineto{\pgfqpoint{1.497086in}{2.330778in}}%
\pgfpathlineto{\pgfqpoint{1.498463in}{2.288947in}}%
\pgfpathlineto{\pgfqpoint{1.503975in}{2.050092in}}%
\pgfpathlineto{\pgfqpoint{1.504763in}{2.058350in}}%
\pgfpathlineto{\pgfqpoint{1.506239in}{2.113232in}}%
\pgfpathlineto{\pgfqpoint{1.511160in}{2.336391in}}%
\pgfpathlineto{\pgfqpoint{1.511554in}{2.333798in}}%
\pgfpathlineto{\pgfqpoint{1.512637in}{2.303383in}}%
\pgfpathlineto{\pgfqpoint{1.514999in}{2.149694in}}%
\pgfpathlineto{\pgfqpoint{1.518444in}{1.969417in}}%
\pgfpathlineto{\pgfqpoint{1.518641in}{1.968487in}}%
\pgfpathlineto{\pgfqpoint{1.518837in}{1.968851in}}%
\pgfpathlineto{\pgfqpoint{1.518837in}{1.968851in}}%
\pgfpathlineto{\pgfqpoint{1.519526in}{1.980260in}}%
\pgfpathlineto{\pgfqpoint{1.521101in}{2.058898in}}%
\pgfpathlineto{\pgfqpoint{1.525924in}{2.336381in}}%
\pgfpathlineto{\pgfqpoint{1.526121in}{2.335424in}}%
\pgfpathlineto{\pgfqpoint{1.526908in}{2.317233in}}%
\pgfpathlineto{\pgfqpoint{1.528582in}{2.211433in}}%
\pgfpathlineto{\pgfqpoint{1.533404in}{1.882418in}}%
\pgfpathlineto{\pgfqpoint{1.533503in}{1.882572in}}%
\pgfpathlineto{\pgfqpoint{1.534093in}{1.891849in}}%
\pgfpathlineto{\pgfqpoint{1.535471in}{1.964828in}}%
\pgfpathlineto{\pgfqpoint{1.540688in}{2.336365in}}%
\pgfpathlineto{\pgfqpoint{1.541278in}{2.327419in}}%
\pgfpathlineto{\pgfqpoint{1.542558in}{2.256637in}}%
\pgfpathlineto{\pgfqpoint{1.548168in}{1.791807in}}%
\pgfpathlineto{\pgfqpoint{1.549054in}{1.811665in}}%
\pgfpathlineto{\pgfqpoint{1.550629in}{1.930957in}}%
\pgfpathlineto{\pgfqpoint{1.555452in}{2.336341in}}%
\pgfpathlineto{\pgfqpoint{1.555550in}{2.335833in}}%
\pgfpathlineto{\pgfqpoint{1.556239in}{2.317779in}}%
\pgfpathlineto{\pgfqpoint{1.557715in}{2.201218in}}%
\pgfpathlineto{\pgfqpoint{1.562932in}{1.697341in}}%
\pgfpathlineto{\pgfqpoint{1.563326in}{1.702670in}}%
\pgfpathlineto{\pgfqpoint{1.564408in}{1.761773in}}%
\pgfpathlineto{\pgfqpoint{1.566869in}{2.059659in}}%
\pgfpathlineto{\pgfqpoint{1.569920in}{2.334599in}}%
\pgfpathlineto{\pgfqpoint{1.570117in}{2.336340in}}%
\pgfpathlineto{\pgfqpoint{1.570314in}{2.335672in}}%
\pgfpathlineto{\pgfqpoint{1.570314in}{2.335672in}}%
\pgfpathlineto{\pgfqpoint{1.571003in}{2.314388in}}%
\pgfpathlineto{\pgfqpoint{1.572479in}{2.178748in}}%
\pgfpathlineto{\pgfqpoint{1.577597in}{1.600092in}}%
\pgfpathlineto{\pgfqpoint{1.577991in}{1.604175in}}%
\pgfpathlineto{\pgfqpoint{1.578975in}{1.658348in}}%
\pgfpathlineto{\pgfqpoint{1.581140in}{1.942190in}}%
\pgfpathlineto{\pgfqpoint{1.584684in}{2.334466in}}%
\pgfpathlineto{\pgfqpoint{1.584881in}{2.336352in}}%
\pgfpathlineto{\pgfqpoint{1.585077in}{2.335485in}}%
\pgfpathlineto{\pgfqpoint{1.585077in}{2.335485in}}%
\pgfpathlineto{\pgfqpoint{1.585766in}{2.310823in}}%
\pgfpathlineto{\pgfqpoint{1.587243in}{2.155590in}}%
\pgfpathlineto{\pgfqpoint{1.592361in}{1.501258in}}%
\pgfpathlineto{\pgfqpoint{1.592755in}{1.506432in}}%
\pgfpathlineto{\pgfqpoint{1.593739in}{1.569110in}}%
\pgfpathlineto{\pgfqpoint{1.595904in}{1.892146in}}%
\pgfpathlineto{\pgfqpoint{1.599448in}{2.334354in}}%
\pgfpathlineto{\pgfqpoint{1.599644in}{2.336366in}}%
\pgfpathlineto{\pgfqpoint{1.599841in}{2.335274in}}%
\pgfpathlineto{\pgfqpoint{1.599841in}{2.335274in}}%
\pgfpathlineto{\pgfqpoint{1.600530in}{2.307127in}}%
\pgfpathlineto{\pgfqpoint{1.602105in}{2.115855in}}%
\pgfpathlineto{\pgfqpoint{1.607125in}{1.402155in}}%
\pgfpathlineto{\pgfqpoint{1.607420in}{1.405688in}}%
\pgfpathlineto{\pgfqpoint{1.608306in}{1.459584in}}%
\pgfpathlineto{\pgfqpoint{1.610274in}{1.764273in}}%
\pgfpathlineto{\pgfqpoint{1.614211in}{2.334266in}}%
\pgfpathlineto{\pgfqpoint{1.614408in}{2.336378in}}%
\pgfpathlineto{\pgfqpoint{1.614605in}{2.335040in}}%
\pgfpathlineto{\pgfqpoint{1.614605in}{2.335040in}}%
\pgfpathlineto{\pgfqpoint{1.615294in}{2.303344in}}%
\pgfpathlineto{\pgfqpoint{1.616869in}{2.090399in}}%
\pgfpathlineto{\pgfqpoint{1.621888in}{1.303981in}}%
\pgfpathlineto{\pgfqpoint{1.622085in}{1.305964in}}%
\pgfpathlineto{\pgfqpoint{1.622873in}{1.349602in}}%
\pgfpathlineto{\pgfqpoint{1.624546in}{1.604211in}}%
\pgfpathlineto{\pgfqpoint{1.629172in}{2.336388in}}%
\pgfpathlineto{\pgfqpoint{1.629467in}{2.332560in}}%
\pgfpathlineto{\pgfqpoint{1.630353in}{2.270741in}}%
\pgfpathlineto{\pgfqpoint{1.632223in}{1.937333in}}%
\pgfpathlineto{\pgfqpoint{1.636455in}{1.209418in}}%
\pgfpathlineto{\pgfqpoint{1.636652in}{1.207915in}}%
\pgfpathlineto{\pgfqpoint{1.636751in}{1.208650in}}%
\pgfpathlineto{\pgfqpoint{1.636751in}{1.208650in}}%
\pgfpathlineto{\pgfqpoint{1.637341in}{1.233720in}}%
\pgfpathlineto{\pgfqpoint{1.638719in}{1.417975in}}%
\pgfpathlineto{\pgfqpoint{1.643936in}{2.336396in}}%
\pgfpathlineto{\pgfqpoint{1.644526in}{2.318405in}}%
\pgfpathlineto{\pgfqpoint{1.645806in}{2.159591in}}%
\pgfpathlineto{\pgfqpoint{1.651416in}{1.115362in}}%
\pgfpathlineto{\pgfqpoint{1.652302in}{1.161985in}}%
\pgfpathlineto{\pgfqpoint{1.653975in}{1.453486in}}%
\pgfpathlineto{\pgfqpoint{1.658699in}{2.336400in}}%
\pgfpathlineto{\pgfqpoint{1.658798in}{2.335870in}}%
\pgfpathlineto{\pgfqpoint{1.659388in}{2.309469in}}%
\pgfpathlineto{\pgfqpoint{1.660766in}{2.105138in}}%
\pgfpathlineto{\pgfqpoint{1.666081in}{1.028975in}}%
\pgfpathlineto{\pgfqpoint{1.666770in}{1.053247in}}%
\pgfpathlineto{\pgfqpoint{1.668050in}{1.236616in}}%
\pgfpathlineto{\pgfqpoint{1.673463in}{2.336399in}}%
\pgfpathlineto{\pgfqpoint{1.674251in}{2.298433in}}%
\pgfpathlineto{\pgfqpoint{1.675727in}{2.043808in}}%
\pgfpathlineto{\pgfqpoint{1.680845in}{0.955394in}}%
\pgfpathlineto{\pgfqpoint{1.681239in}{0.963769in}}%
\pgfpathlineto{\pgfqpoint{1.682223in}{1.066765in}}%
\pgfpathlineto{\pgfqpoint{1.684388in}{1.597555in}}%
\pgfpathlineto{\pgfqpoint{1.687932in}{2.331230in}}%
\pgfpathlineto{\pgfqpoint{1.688227in}{2.336393in}}%
\pgfpathlineto{\pgfqpoint{1.688424in}{2.333659in}}%
\pgfpathlineto{\pgfqpoint{1.688424in}{2.333659in}}%
\pgfpathlineto{\pgfqpoint{1.689211in}{2.274056in}}%
\pgfpathlineto{\pgfqpoint{1.690884in}{1.926691in}}%
\pgfpathlineto{\pgfqpoint{1.695609in}{0.911070in}}%
\pgfpathlineto{\pgfqpoint{1.696002in}{0.921244in}}%
\pgfpathlineto{\pgfqpoint{1.696987in}{1.031091in}}%
\pgfpathlineto{\pgfqpoint{1.699250in}{1.612014in}}%
\pgfpathlineto{\pgfqpoint{1.702695in}{2.331428in}}%
\pgfpathlineto{\pgfqpoint{1.702991in}{2.336379in}}%
\pgfpathlineto{\pgfqpoint{1.703187in}{2.333418in}}%
\pgfpathlineto{\pgfqpoint{1.703187in}{2.333418in}}%
\pgfpathlineto{\pgfqpoint{1.703975in}{2.272447in}}%
\pgfpathlineto{\pgfqpoint{1.705747in}{1.895106in}}%
\pgfpathlineto{\pgfqpoint{1.710274in}{0.927361in}}%
\pgfpathlineto{\pgfqpoint{1.710471in}{0.928732in}}%
\pgfpathlineto{\pgfqpoint{1.711160in}{0.972156in}}%
\pgfpathlineto{\pgfqpoint{1.712735in}{1.268946in}}%
\pgfpathlineto{\pgfqpoint{1.717754in}{2.336350in}}%
\pgfpathlineto{\pgfqpoint{1.717853in}{2.335406in}}%
\pgfpathlineto{\pgfqpoint{1.718443in}{2.304904in}}%
\pgfpathlineto{\pgfqpoint{1.719821in}{2.085216in}}%
\pgfpathlineto{\pgfqpoint{1.725038in}{1.040757in}}%
\pgfpathlineto{\pgfqpoint{1.725432in}{1.051436in}}%
\pgfpathlineto{\pgfqpoint{1.726514in}{1.170265in}}%
\pgfpathlineto{\pgfqpoint{1.729073in}{1.788892in}}%
\pgfpathlineto{\pgfqpoint{1.732124in}{2.329729in}}%
\pgfpathlineto{\pgfqpoint{1.732518in}{2.336294in}}%
\pgfpathlineto{\pgfqpoint{1.732813in}{2.330165in}}%
\pgfpathlineto{\pgfqpoint{1.732813in}{2.330165in}}%
\pgfpathlineto{\pgfqpoint{1.733798in}{2.244989in}}%
\pgfpathlineto{\pgfqpoint{1.736061in}{1.787296in}}%
\pgfpathlineto{\pgfqpoint{1.739408in}{1.251217in}}%
\pgfpathlineto{\pgfqpoint{1.739703in}{1.247332in}}%
\pgfpathlineto{\pgfqpoint{1.739900in}{1.249549in}}%
\pgfpathlineto{\pgfqpoint{1.739900in}{1.249549in}}%
\pgfpathlineto{\pgfqpoint{1.740687in}{1.295930in}}%
\pgfpathlineto{\pgfqpoint{1.742459in}{1.581959in}}%
\pgfpathlineto{\pgfqpoint{1.747085in}{2.335642in}}%
\pgfpathlineto{\pgfqpoint{1.747183in}{2.336354in}}%
\pgfpathlineto{\pgfqpoint{1.747282in}{2.336207in}}%
\pgfpathlineto{\pgfqpoint{1.747282in}{2.336207in}}%
\pgfpathlineto{\pgfqpoint{1.747774in}{2.322787in}}%
\pgfpathlineto{\pgfqpoint{1.749054in}{2.198302in}}%
\pgfpathlineto{\pgfqpoint{1.754467in}{1.462283in}}%
\pgfpathlineto{\pgfqpoint{1.755156in}{1.482973in}}%
\pgfpathlineto{\pgfqpoint{1.756534in}{1.624986in}}%
\pgfpathlineto{\pgfqpoint{1.761947in}{2.336397in}}%
\pgfpathlineto{\pgfqpoint{1.762439in}{2.328108in}}%
\pgfpathlineto{\pgfqpoint{1.763620in}{2.242904in}}%
\pgfpathlineto{\pgfqpoint{1.769231in}{1.607706in}}%
\pgfpathlineto{\pgfqpoint{1.770215in}{1.640233in}}%
\pgfpathlineto{\pgfqpoint{1.771987in}{1.831673in}}%
\pgfpathlineto{\pgfqpoint{1.776613in}{2.336151in}}%
\pgfpathlineto{\pgfqpoint{1.776711in}{2.336398in}}%
\pgfpathlineto{\pgfqpoint{1.776711in}{2.336398in}}%
\pgfpathlineto{\pgfqpoint{1.776711in}{2.336398in}}%
\pgfpathlineto{\pgfqpoint{1.777105in}{2.331472in}}%
\pgfpathlineto{\pgfqpoint{1.778187in}{2.271927in}}%
\pgfpathlineto{\pgfqpoint{1.780746in}{1.962136in}}%
\pgfpathlineto{\pgfqpoint{1.783699in}{1.708862in}}%
\pgfpathlineto{\pgfqpoint{1.783994in}{1.706807in}}%
\pgfpathlineto{\pgfqpoint{1.784191in}{1.708205in}}%
\pgfpathlineto{\pgfqpoint{1.784191in}{1.708205in}}%
\pgfpathlineto{\pgfqpoint{1.784979in}{1.735426in}}%
\pgfpathlineto{\pgfqpoint{1.786750in}{1.901725in}}%
\pgfpathlineto{\pgfqpoint{1.791278in}{2.335603in}}%
\pgfpathlineto{\pgfqpoint{1.791475in}{2.336385in}}%
\pgfpathlineto{\pgfqpoint{1.791573in}{2.336005in}}%
\pgfpathlineto{\pgfqpoint{1.791573in}{2.336005in}}%
\pgfpathlineto{\pgfqpoint{1.792164in}{2.323131in}}%
\pgfpathlineto{\pgfqpoint{1.793542in}{2.230250in}}%
\pgfpathlineto{\pgfqpoint{1.798660in}{1.794810in}}%
\pgfpathlineto{\pgfqpoint{1.799152in}{1.799571in}}%
\pgfpathlineto{\pgfqpoint{1.800235in}{1.849788in}}%
\pgfpathlineto{\pgfqpoint{1.802794in}{2.108022in}}%
\pgfpathlineto{\pgfqpoint{1.805845in}{2.333535in}}%
\pgfpathlineto{\pgfqpoint{1.806238in}{2.336366in}}%
\pgfpathlineto{\pgfqpoint{1.806534in}{2.333888in}}%
\pgfpathlineto{\pgfqpoint{1.806534in}{2.333888in}}%
\pgfpathlineto{\pgfqpoint{1.807518in}{2.298606in}}%
\pgfpathlineto{\pgfqpoint{1.809782in}{2.107221in}}%
\pgfpathlineto{\pgfqpoint{1.813128in}{1.878696in}}%
\pgfpathlineto{\pgfqpoint{1.813423in}{1.876587in}}%
\pgfpathlineto{\pgfqpoint{1.813719in}{1.878117in}}%
\pgfpathlineto{\pgfqpoint{1.813719in}{1.878117in}}%
\pgfpathlineto{\pgfqpoint{1.814605in}{1.903894in}}%
\pgfpathlineto{\pgfqpoint{1.816573in}{2.050420in}}%
\pgfpathlineto{\pgfqpoint{1.820609in}{2.334177in}}%
\pgfpathlineto{\pgfqpoint{1.820904in}{2.336360in}}%
\pgfpathlineto{\pgfqpoint{1.821199in}{2.335213in}}%
\pgfpathlineto{\pgfqpoint{1.821199in}{2.335213in}}%
\pgfpathlineto{\pgfqpoint{1.822085in}{2.312575in}}%
\pgfpathlineto{\pgfqpoint{1.824053in}{2.183569in}}%
\pgfpathlineto{\pgfqpoint{1.827892in}{1.953274in}}%
\pgfpathlineto{\pgfqpoint{1.828187in}{1.951775in}}%
\pgfpathlineto{\pgfqpoint{1.828483in}{1.953319in}}%
\pgfpathlineto{\pgfqpoint{1.828483in}{1.953319in}}%
\pgfpathlineto{\pgfqpoint{1.829368in}{1.975624in}}%
\pgfpathlineto{\pgfqpoint{1.831337in}{2.099091in}}%
\pgfpathlineto{\pgfqpoint{1.835372in}{2.334702in}}%
\pgfpathlineto{\pgfqpoint{1.835668in}{2.336384in}}%
\pgfpathlineto{\pgfqpoint{1.835963in}{2.335303in}}%
\pgfpathlineto{\pgfqpoint{1.835963in}{2.335303in}}%
\pgfpathlineto{\pgfqpoint{1.836849in}{2.316187in}}%
\pgfpathlineto{\pgfqpoint{1.838817in}{2.209111in}}%
\pgfpathlineto{\pgfqpoint{1.842656in}{2.020937in}}%
\pgfpathlineto{\pgfqpoint{1.842951in}{2.019929in}}%
\pgfpathlineto{\pgfqpoint{1.843148in}{2.020649in}}%
\pgfpathlineto{\pgfqpoint{1.843148in}{2.020649in}}%
\pgfpathlineto{\pgfqpoint{1.843935in}{2.034390in}}%
\pgfpathlineto{\pgfqpoint{1.845707in}{2.117827in}}%
\pgfpathlineto{\pgfqpoint{1.850333in}{2.336224in}}%
\pgfpathlineto{\pgfqpoint{1.850431in}{2.336396in}}%
\pgfpathlineto{\pgfqpoint{1.850530in}{2.336317in}}%
\pgfpathlineto{\pgfqpoint{1.850530in}{2.336317in}}%
\pgfpathlineto{\pgfqpoint{1.851120in}{2.330685in}}%
\pgfpathlineto{\pgfqpoint{1.852498in}{2.287033in}}%
\pgfpathlineto{\pgfqpoint{1.857616in}{2.080769in}}%
\pgfpathlineto{\pgfqpoint{1.858108in}{2.083092in}}%
\pgfpathlineto{\pgfqpoint{1.859191in}{2.106903in}}%
\pgfpathlineto{\pgfqpoint{1.861849in}{2.233484in}}%
\pgfpathlineto{\pgfqpoint{1.864900in}{2.335460in}}%
\pgfpathlineto{\pgfqpoint{1.865195in}{2.336400in}}%
\pgfpathlineto{\pgfqpoint{1.865490in}{2.335540in}}%
\pgfpathlineto{\pgfqpoint{1.865490in}{2.335540in}}%
\pgfpathlineto{\pgfqpoint{1.866376in}{2.322693in}}%
\pgfpathlineto{\pgfqpoint{1.868443in}{2.248735in}}%
\pgfpathlineto{\pgfqpoint{1.872085in}{2.135066in}}%
\pgfpathlineto{\pgfqpoint{1.872380in}{2.134193in}}%
\pgfpathlineto{\pgfqpoint{1.872675in}{2.134922in}}%
\pgfpathlineto{\pgfqpoint{1.872675in}{2.134922in}}%
\pgfpathlineto{\pgfqpoint{1.873561in}{2.146401in}}%
\pgfpathlineto{\pgfqpoint{1.875530in}{2.210762in}}%
\pgfpathlineto{\pgfqpoint{1.879664in}{2.335720in}}%
\pgfpathlineto{\pgfqpoint{1.879959in}{2.336400in}}%
\pgfpathlineto{\pgfqpoint{1.880254in}{2.335675in}}%
\pgfpathlineto{\pgfqpoint{1.880254in}{2.335675in}}%
\pgfpathlineto{\pgfqpoint{1.881140in}{2.325510in}}%
\pgfpathlineto{\pgfqpoint{1.883207in}{2.267890in}}%
\pgfpathlineto{\pgfqpoint{1.886849in}{2.180873in}}%
\pgfpathlineto{\pgfqpoint{1.887144in}{2.180327in}}%
\pgfpathlineto{\pgfqpoint{1.887439in}{2.181016in}}%
\pgfpathlineto{\pgfqpoint{1.887439in}{2.181016in}}%
\pgfpathlineto{\pgfqpoint{1.888325in}{2.190230in}}%
\pgfpathlineto{\pgfqpoint{1.890293in}{2.240311in}}%
\pgfpathlineto{\pgfqpoint{1.894329in}{2.335517in}}%
\pgfpathlineto{\pgfqpoint{1.894723in}{2.336399in}}%
\pgfpathlineto{\pgfqpoint{1.895116in}{2.335384in}}%
\pgfpathlineto{\pgfqpoint{1.895116in}{2.335384in}}%
\pgfpathlineto{\pgfqpoint{1.896199in}{2.323575in}}%
\pgfpathlineto{\pgfqpoint{1.899053in}{2.257140in}}%
\pgfpathlineto{\pgfqpoint{1.901711in}{2.219489in}}%
\pgfpathlineto{\pgfqpoint{1.901908in}{2.219388in}}%
\pgfpathlineto{\pgfqpoint{1.902104in}{2.219698in}}%
\pgfpathlineto{\pgfqpoint{1.902104in}{2.219698in}}%
\pgfpathlineto{\pgfqpoint{1.902990in}{2.226026in}}%
\pgfpathlineto{\pgfqpoint{1.904959in}{2.262728in}}%
\pgfpathlineto{\pgfqpoint{1.909093in}{2.335776in}}%
\pgfpathlineto{\pgfqpoint{1.909486in}{2.336398in}}%
\pgfpathlineto{\pgfqpoint{1.909880in}{2.335623in}}%
\pgfpathlineto{\pgfqpoint{1.909880in}{2.335623in}}%
\pgfpathlineto{\pgfqpoint{1.910963in}{2.326891in}}%
\pgfpathlineto{\pgfqpoint{1.913915in}{2.276813in}}%
\pgfpathlineto{\pgfqpoint{1.916474in}{2.251744in}}%
\pgfpathlineto{\pgfqpoint{1.916868in}{2.252004in}}%
\pgfpathlineto{\pgfqpoint{1.917065in}{2.252577in}}%
\pgfpathlineto{\pgfqpoint{1.918246in}{2.261772in}}%
\pgfpathlineto{\pgfqpoint{1.921396in}{2.312134in}}%
\pgfpathlineto{\pgfqpoint{1.924053in}{2.336309in}}%
\pgfpathlineto{\pgfqpoint{1.924447in}{2.336239in}}%
\pgfpathlineto{\pgfqpoint{1.924742in}{2.335548in}}%
\pgfpathlineto{\pgfqpoint{1.926022in}{2.326962in}}%
\pgfpathlineto{\pgfqpoint{1.931337in}{2.277754in}}%
\pgfpathlineto{\pgfqpoint{1.931927in}{2.278764in}}%
\pgfpathlineto{\pgfqpoint{1.933207in}{2.286724in}}%
\pgfpathlineto{\pgfqpoint{1.939014in}{2.336398in}}%
\pgfpathlineto{\pgfqpoint{1.939703in}{2.335322in}}%
\pgfpathlineto{\pgfqpoint{1.941179in}{2.327362in}}%
\pgfpathlineto{\pgfqpoint{1.946100in}{2.298100in}}%
\pgfpathlineto{\pgfqpoint{1.946199in}{2.298143in}}%
\pgfpathlineto{\pgfqpoint{1.947085in}{2.300002in}}%
\pgfpathlineto{\pgfqpoint{1.948955in}{2.310894in}}%
\pgfpathlineto{\pgfqpoint{1.953384in}{2.336211in}}%
\pgfpathlineto{\pgfqpoint{1.954171in}{2.336170in}}%
\pgfpathlineto{\pgfqpoint{1.954467in}{2.335738in}}%
\pgfpathlineto{\pgfqpoint{1.956041in}{2.330376in}}%
\pgfpathlineto{\pgfqpoint{1.960766in}{2.313322in}}%
\pgfpathlineto{\pgfqpoint{1.961848in}{2.314604in}}%
\pgfpathlineto{\pgfqpoint{1.963915in}{2.322176in}}%
\pgfpathlineto{\pgfqpoint{1.968049in}{2.336213in}}%
\pgfpathlineto{\pgfqpoint{1.969427in}{2.335825in}}%
\pgfpathlineto{\pgfqpoint{1.971396in}{2.331522in}}%
\pgfpathlineto{\pgfqpoint{1.975234in}{2.324111in}}%
\pgfpathlineto{\pgfqpoint{1.976711in}{2.325012in}}%
\pgfpathlineto{\pgfqpoint{1.979270in}{2.330451in}}%
\pgfpathlineto{\pgfqpoint{1.982813in}{2.336290in}}%
\pgfpathlineto{\pgfqpoint{1.984683in}{2.335809in}}%
\pgfpathlineto{\pgfqpoint{1.991376in}{2.331519in}}%
\pgfpathlineto{\pgfqpoint{2.000529in}{2.335934in}}%
\pgfpathlineto{\pgfqpoint{2.005844in}{2.335356in}}%
\pgfpathlineto{\pgfqpoint{2.016474in}{2.336307in}}%
\pgfpathlineto{\pgfqpoint{2.029663in}{2.334240in}}%
\pgfpathlineto{\pgfqpoint{2.032616in}{2.326902in}}%
\pgfpathlineto{\pgfqpoint{2.035569in}{2.321795in}}%
\pgfpathlineto{\pgfqpoint{2.036750in}{2.322812in}}%
\pgfpathlineto{\pgfqpoint{2.038817in}{2.328822in}}%
\pgfpathlineto{\pgfqpoint{2.041966in}{2.336375in}}%
\pgfpathlineto{\pgfqpoint{2.042951in}{2.334732in}}%
\pgfpathlineto{\pgfqpoint{2.044624in}{2.325926in}}%
\pgfpathlineto{\pgfqpoint{2.049840in}{2.292140in}}%
\pgfpathlineto{\pgfqpoint{2.050234in}{2.292581in}}%
\pgfpathlineto{\pgfqpoint{2.051415in}{2.297666in}}%
\pgfpathlineto{\pgfqpoint{2.054368in}{2.324750in}}%
\pgfpathlineto{\pgfqpoint{2.056632in}{2.336400in}}%
\pgfpathlineto{\pgfqpoint{2.056730in}{2.336373in}}%
\pgfpathlineto{\pgfqpoint{2.057419in}{2.334718in}}%
\pgfpathlineto{\pgfqpoint{2.058895in}{2.322890in}}%
\pgfpathlineto{\pgfqpoint{2.064210in}{2.265289in}}%
\pgfpathlineto{\pgfqpoint{2.064801in}{2.266352in}}%
\pgfpathlineto{\pgfqpoint{2.066080in}{2.276017in}}%
\pgfpathlineto{\pgfqpoint{2.071395in}{2.336399in}}%
\pgfpathlineto{\pgfqpoint{2.072281in}{2.333714in}}%
\pgfpathlineto{\pgfqpoint{2.073856in}{2.316537in}}%
\pgfpathlineto{\pgfqpoint{2.078876in}{2.253690in}}%
\pgfpathlineto{\pgfqpoint{2.079073in}{2.253893in}}%
\pgfpathlineto{\pgfqpoint{2.079958in}{2.258339in}}%
\pgfpathlineto{\pgfqpoint{2.081828in}{2.283127in}}%
\pgfpathlineto{\pgfqpoint{2.086061in}{2.336338in}}%
\pgfpathlineto{\pgfqpoint{2.086454in}{2.336135in}}%
\pgfpathlineto{\pgfqpoint{2.086651in}{2.335598in}}%
\pgfpathlineto{\pgfqpoint{2.087734in}{2.327761in}}%
\pgfpathlineto{\pgfqpoint{2.090391in}{2.286397in}}%
\pgfpathlineto{\pgfqpoint{2.093344in}{2.255287in}}%
\pgfpathlineto{\pgfqpoint{2.093639in}{2.255221in}}%
\pgfpathlineto{\pgfqpoint{2.093935in}{2.255796in}}%
\pgfpathlineto{\pgfqpoint{2.093935in}{2.255796in}}%
\pgfpathlineto{\pgfqpoint{2.095017in}{2.263121in}}%
\pgfpathlineto{\pgfqpoint{2.097576in}{2.301851in}}%
\pgfpathlineto{\pgfqpoint{2.100628in}{2.335999in}}%
\pgfpathlineto{\pgfqpoint{2.101021in}{2.336389in}}%
\pgfpathlineto{\pgfqpoint{2.101415in}{2.335698in}}%
\pgfpathlineto{\pgfqpoint{2.101415in}{2.335698in}}%
\pgfpathlineto{\pgfqpoint{2.102498in}{2.328565in}}%
\pgfpathlineto{\pgfqpoint{2.105254in}{2.289694in}}%
\pgfpathlineto{\pgfqpoint{2.108010in}{2.263941in}}%
\pgfpathlineto{\pgfqpoint{2.108403in}{2.263836in}}%
\pgfpathlineto{\pgfqpoint{2.108698in}{2.264425in}}%
\pgfpathlineto{\pgfqpoint{2.109781in}{2.271202in}}%
\pgfpathlineto{\pgfqpoint{2.112439in}{2.307323in}}%
\pgfpathlineto{\pgfqpoint{2.115490in}{2.336202in}}%
\pgfpathlineto{\pgfqpoint{2.115884in}{2.336342in}}%
\pgfpathlineto{\pgfqpoint{2.116179in}{2.335835in}}%
\pgfpathlineto{\pgfqpoint{2.116179in}{2.335835in}}%
\pgfpathlineto{\pgfqpoint{2.117261in}{2.329791in}}%
\pgfpathlineto{\pgfqpoint{2.120017in}{2.296824in}}%
\pgfpathlineto{\pgfqpoint{2.122773in}{2.275291in}}%
\pgfpathlineto{\pgfqpoint{2.123167in}{2.275247in}}%
\pgfpathlineto{\pgfqpoint{2.123561in}{2.276056in}}%
\pgfpathlineto{\pgfqpoint{2.124840in}{2.284105in}}%
\pgfpathlineto{\pgfqpoint{2.130549in}{2.336399in}}%
\pgfpathlineto{\pgfqpoint{2.131336in}{2.334808in}}%
\pgfpathlineto{\pgfqpoint{2.132911in}{2.323873in}}%
\pgfpathlineto{\pgfqpoint{2.137734in}{2.287075in}}%
\pgfpathlineto{\pgfqpoint{2.137832in}{2.287093in}}%
\pgfpathlineto{\pgfqpoint{2.138620in}{2.288762in}}%
\pgfpathlineto{\pgfqpoint{2.140194in}{2.299171in}}%
\pgfpathlineto{\pgfqpoint{2.145313in}{2.336400in}}%
\pgfpathlineto{\pgfqpoint{2.145411in}{2.336382in}}%
\pgfpathlineto{\pgfqpoint{2.146198in}{2.334915in}}%
\pgfpathlineto{\pgfqpoint{2.147872in}{2.325315in}}%
\pgfpathlineto{\pgfqpoint{2.152498in}{2.298319in}}%
\pgfpathlineto{\pgfqpoint{2.153285in}{2.299358in}}%
\pgfpathlineto{\pgfqpoint{2.154860in}{2.306994in}}%
\pgfpathlineto{\pgfqpoint{2.160175in}{2.336394in}}%
\pgfpathlineto{\pgfqpoint{2.160273in}{2.336362in}}%
\pgfpathlineto{\pgfqpoint{2.161257in}{2.334543in}}%
\pgfpathlineto{\pgfqpoint{2.163324in}{2.324207in}}%
\pgfpathlineto{\pgfqpoint{2.166966in}{2.308392in}}%
\pgfpathlineto{\pgfqpoint{2.167852in}{2.308701in}}%
\pgfpathlineto{\pgfqpoint{2.167950in}{2.308857in}}%
\pgfpathlineto{\pgfqpoint{2.169427in}{2.313708in}}%
\pgfpathlineto{\pgfqpoint{2.175037in}{2.336386in}}%
\pgfpathlineto{\pgfqpoint{2.175332in}{2.336222in}}%
\pgfpathlineto{\pgfqpoint{2.176612in}{2.333600in}}%
\pgfpathlineto{\pgfqpoint{2.182222in}{2.316661in}}%
\pgfpathlineto{\pgfqpoint{2.182911in}{2.317291in}}%
\pgfpathlineto{\pgfqpoint{2.184683in}{2.322111in}}%
\pgfpathlineto{\pgfqpoint{2.189604in}{2.336386in}}%
\pgfpathlineto{\pgfqpoint{2.190883in}{2.335507in}}%
\pgfpathlineto{\pgfqpoint{2.193147in}{2.330015in}}%
\pgfpathlineto{\pgfqpoint{2.196592in}{2.323334in}}%
\pgfpathlineto{\pgfqpoint{2.197970in}{2.324043in}}%
\pgfpathlineto{\pgfqpoint{2.200234in}{2.328867in}}%
\pgfpathlineto{\pgfqpoint{2.204171in}{2.336316in}}%
\pgfpathlineto{\pgfqpoint{2.205844in}{2.335726in}}%
\pgfpathlineto{\pgfqpoint{2.208797in}{2.331008in}}%
\pgfpathlineto{\pgfqpoint{2.211651in}{2.328337in}}%
\pgfpathlineto{\pgfqpoint{2.213521in}{2.329532in}}%
\pgfpathlineto{\pgfqpoint{2.220411in}{2.336184in}}%
\pgfpathlineto{\pgfqpoint{2.223560in}{2.333474in}}%
\pgfpathlineto{\pgfqpoint{2.226808in}{2.331935in}}%
\pgfpathlineto{\pgfqpoint{2.229466in}{2.333422in}}%
\pgfpathlineto{\pgfqpoint{2.234584in}{2.336398in}}%
\pgfpathlineto{\pgfqpoint{2.238127in}{2.335165in}}%
\pgfpathlineto{\pgfqpoint{2.242360in}{2.334354in}}%
\pgfpathlineto{\pgfqpoint{2.255647in}{2.335685in}}%
\pgfpathlineto{\pgfqpoint{2.264604in}{2.336383in}}%
\pgfpathlineto{\pgfqpoint{2.294131in}{2.336078in}}%
\pgfpathlineto{\pgfqpoint{2.303580in}{2.334654in}}%
\pgfpathlineto{\pgfqpoint{2.308009in}{2.336166in}}%
\pgfpathlineto{\pgfqpoint{2.310076in}{2.333778in}}%
\pgfpathlineto{\pgfqpoint{2.316080in}{2.324510in}}%
\pgfpathlineto{\pgfqpoint{2.316375in}{2.324734in}}%
\pgfpathlineto{\pgfqpoint{2.318147in}{2.327966in}}%
\pgfpathlineto{\pgfqpoint{2.322281in}{2.336372in}}%
\pgfpathlineto{\pgfqpoint{2.322478in}{2.336271in}}%
\pgfpathlineto{\pgfqpoint{2.323659in}{2.334027in}}%
\pgfpathlineto{\pgfqpoint{2.325726in}{2.323850in}}%
\pgfpathlineto{\pgfqpoint{2.330056in}{2.303057in}}%
\pgfpathlineto{\pgfqpoint{2.330942in}{2.304062in}}%
\pgfpathlineto{\pgfqpoint{2.332419in}{2.310781in}}%
\pgfpathlineto{\pgfqpoint{2.337044in}{2.336389in}}%
\pgfpathlineto{\pgfqpoint{2.337438in}{2.335947in}}%
\pgfpathlineto{\pgfqpoint{2.338521in}{2.331374in}}%
\pgfpathlineto{\pgfqpoint{2.340785in}{2.309356in}}%
\pgfpathlineto{\pgfqpoint{2.344328in}{2.280366in}}%
\pgfpathlineto{\pgfqpoint{2.344722in}{2.280261in}}%
\pgfpathlineto{\pgfqpoint{2.345115in}{2.280962in}}%
\pgfpathlineto{\pgfqpoint{2.346296in}{2.287629in}}%
\pgfpathlineto{\pgfqpoint{2.351808in}{2.336396in}}%
\pgfpathlineto{\pgfqpoint{2.352792in}{2.333575in}}%
\pgfpathlineto{\pgfqpoint{2.354564in}{2.317395in}}%
\pgfpathlineto{\pgfqpoint{2.358993in}{2.277209in}}%
\pgfpathlineto{\pgfqpoint{2.359682in}{2.278223in}}%
\pgfpathlineto{\pgfqpoint{2.360962in}{2.286389in}}%
\pgfpathlineto{\pgfqpoint{2.366572in}{2.336397in}}%
\pgfpathlineto{\pgfqpoint{2.367261in}{2.335144in}}%
\pgfpathlineto{\pgfqpoint{2.368737in}{2.325760in}}%
\pgfpathlineto{\pgfqpoint{2.373757in}{2.288560in}}%
\pgfpathlineto{\pgfqpoint{2.373954in}{2.288681in}}%
\pgfpathlineto{\pgfqpoint{2.374840in}{2.291244in}}%
\pgfpathlineto{\pgfqpoint{2.376808in}{2.306246in}}%
\pgfpathlineto{\pgfqpoint{2.380942in}{2.336156in}}%
\pgfpathlineto{\pgfqpoint{2.381533in}{2.336302in}}%
\pgfpathlineto{\pgfqpoint{2.381926in}{2.335690in}}%
\pgfpathlineto{\pgfqpoint{2.383304in}{2.329678in}}%
\pgfpathlineto{\pgfqpoint{2.388422in}{2.301501in}}%
\pgfpathlineto{\pgfqpoint{2.388816in}{2.301748in}}%
\pgfpathlineto{\pgfqpoint{2.389997in}{2.305259in}}%
\pgfpathlineto{\pgfqpoint{2.392851in}{2.323855in}}%
\pgfpathlineto{\pgfqpoint{2.395706in}{2.336224in}}%
\pgfpathlineto{\pgfqpoint{2.396592in}{2.336054in}}%
\pgfpathlineto{\pgfqpoint{2.396788in}{2.335749in}}%
\pgfpathlineto{\pgfqpoint{2.398363in}{2.330475in}}%
\pgfpathlineto{\pgfqpoint{2.403186in}{2.313017in}}%
\pgfpathlineto{\pgfqpoint{2.404269in}{2.314372in}}%
\pgfpathlineto{\pgfqpoint{2.406336in}{2.322147in}}%
\pgfpathlineto{\pgfqpoint{2.410371in}{2.336209in}}%
\pgfpathlineto{\pgfqpoint{2.411651in}{2.335888in}}%
\pgfpathlineto{\pgfqpoint{2.413422in}{2.331880in}}%
\pgfpathlineto{\pgfqpoint{2.417753in}{2.322350in}}%
\pgfpathlineto{\pgfqpoint{2.419131in}{2.323424in}}%
\pgfpathlineto{\pgfqpoint{2.421592in}{2.329422in}}%
\pgfpathlineto{\pgfqpoint{2.425135in}{2.336280in}}%
\pgfpathlineto{\pgfqpoint{2.426808in}{2.335820in}}%
\pgfpathlineto{\pgfqpoint{2.429958in}{2.331180in}}%
\pgfpathlineto{\pgfqpoint{2.432615in}{2.329215in}}%
\pgfpathlineto{\pgfqpoint{2.434584in}{2.330550in}}%
\pgfpathlineto{\pgfqpoint{2.440883in}{2.336366in}}%
\pgfpathlineto{\pgfqpoint{2.444131in}{2.334657in}}%
\pgfpathlineto{\pgfqpoint{2.447773in}{2.333691in}}%
\pgfpathlineto{\pgfqpoint{2.452202in}{2.335686in}}%
\pgfpathlineto{\pgfqpoint{2.456434in}{2.336350in}}%
\pgfpathlineto{\pgfqpoint{2.468048in}{2.336347in}}%
\pgfpathlineto{\pgfqpoint{2.488422in}{2.336030in}}%
\pgfpathlineto{\pgfqpoint{2.495902in}{2.336017in}}%
\pgfpathlineto{\pgfqpoint{2.501316in}{2.335973in}}%
\pgfpathlineto{\pgfqpoint{2.508895in}{2.334520in}}%
\pgfpathlineto{\pgfqpoint{2.515391in}{2.335993in}}%
\pgfpathlineto{\pgfqpoint{2.523658in}{2.333135in}}%
\pgfpathlineto{\pgfqpoint{2.529564in}{2.336158in}}%
\pgfpathlineto{\pgfqpoint{2.532221in}{2.333690in}}%
\pgfpathlineto{\pgfqpoint{2.536257in}{2.330531in}}%
\pgfpathlineto{\pgfqpoint{2.538422in}{2.331835in}}%
\pgfpathlineto{\pgfqpoint{2.544032in}{2.336248in}}%
\pgfpathlineto{\pgfqpoint{2.546198in}{2.334126in}}%
\pgfpathlineto{\pgfqpoint{2.551020in}{2.329125in}}%
\pgfpathlineto{\pgfqpoint{2.552989in}{2.330562in}}%
\pgfpathlineto{\pgfqpoint{2.558796in}{2.336226in}}%
\pgfpathlineto{\pgfqpoint{2.560863in}{2.333941in}}%
\pgfpathlineto{\pgfqpoint{2.565686in}{2.328206in}}%
\pgfpathlineto{\pgfqpoint{2.567457in}{2.329501in}}%
\pgfpathlineto{\pgfqpoint{2.573757in}{2.336097in}}%
\pgfpathlineto{\pgfqpoint{2.575922in}{2.333277in}}%
\pgfpathlineto{\pgfqpoint{2.580253in}{2.327849in}}%
\pgfpathlineto{\pgfqpoint{2.582024in}{2.329013in}}%
\pgfpathlineto{\pgfqpoint{2.588717in}{2.335951in}}%
\pgfpathlineto{\pgfqpoint{2.591079in}{2.332597in}}%
\pgfpathlineto{\pgfqpoint{2.594820in}{2.328031in}}%
\pgfpathlineto{\pgfqpoint{2.596591in}{2.328974in}}%
\pgfpathlineto{\pgfqpoint{2.603579in}{2.335890in}}%
\pgfpathlineto{\pgfqpoint{2.603776in}{2.335713in}}%
\pgfpathlineto{\pgfqpoint{2.606827in}{2.331139in}}%
\pgfpathlineto{\pgfqpoint{2.609780in}{2.328494in}}%
\pgfpathlineto{\pgfqpoint{2.611650in}{2.329756in}}%
\pgfpathlineto{\pgfqpoint{2.618048in}{2.336125in}}%
\pgfpathlineto{\pgfqpoint{2.620410in}{2.333342in}}%
\pgfpathlineto{\pgfqpoint{2.624347in}{2.329152in}}%
\pgfpathlineto{\pgfqpoint{2.626217in}{2.330138in}}%
\pgfpathlineto{\pgfqpoint{2.633107in}{2.335973in}}%
\pgfpathlineto{\pgfqpoint{2.636158in}{2.332424in}}%
\pgfpathlineto{\pgfqpoint{2.639209in}{2.330247in}}%
\pgfpathlineto{\pgfqpoint{2.641276in}{2.331448in}}%
\pgfpathlineto{\pgfqpoint{2.647379in}{2.336280in}}%
\pgfpathlineto{\pgfqpoint{2.650134in}{2.334135in}}%
\pgfpathlineto{\pgfqpoint{2.653973in}{2.331818in}}%
\pgfpathlineto{\pgfqpoint{2.656434in}{2.333106in}}%
\pgfpathlineto{\pgfqpoint{2.661847in}{2.336374in}}%
\pgfpathlineto{\pgfqpoint{2.665095in}{2.334787in}}%
\pgfpathlineto{\pgfqpoint{2.669032in}{2.333536in}}%
\pgfpathlineto{\pgfqpoint{2.672772in}{2.335277in}}%
\pgfpathlineto{\pgfqpoint{2.676906in}{2.336361in}}%
\pgfpathlineto{\pgfqpoint{2.688225in}{2.336031in}}%
\pgfpathlineto{\pgfqpoint{2.693638in}{2.336203in}}%
\pgfpathlineto{\pgfqpoint{2.702300in}{2.336216in}}%
\pgfpathlineto{\pgfqpoint{2.716079in}{2.336373in}}%
\pgfpathlineto{\pgfqpoint{2.755646in}{2.335707in}}%
\pgfpathlineto{\pgfqpoint{2.761748in}{2.336144in}}%
\pgfpathlineto{\pgfqpoint{2.767359in}{2.335975in}}%
\pgfpathlineto{\pgfqpoint{2.774248in}{2.335345in}}%
\pgfpathlineto{\pgfqpoint{2.781532in}{2.335999in}}%
\pgfpathlineto{\pgfqpoint{2.788717in}{2.334871in}}%
\pgfpathlineto{\pgfqpoint{2.796099in}{2.335979in}}%
\pgfpathlineto{\pgfqpoint{2.803284in}{2.334488in}}%
\pgfpathlineto{\pgfqpoint{2.810764in}{2.335966in}}%
\pgfpathlineto{\pgfqpoint{2.817851in}{2.334222in}}%
\pgfpathlineto{\pgfqpoint{2.825626in}{2.335906in}}%
\pgfpathlineto{\pgfqpoint{2.832418in}{2.334102in}}%
\pgfpathlineto{\pgfqpoint{2.840685in}{2.335788in}}%
\pgfpathlineto{\pgfqpoint{2.846886in}{2.334103in}}%
\pgfpathlineto{\pgfqpoint{2.856040in}{2.335573in}}%
\pgfpathlineto{\pgfqpoint{2.861354in}{2.334244in}}%
\pgfpathlineto{\pgfqpoint{2.871689in}{2.335306in}}%
\pgfpathlineto{\pgfqpoint{2.876315in}{2.334567in}}%
\pgfpathlineto{\pgfqpoint{2.886158in}{2.335589in}}%
\pgfpathlineto{\pgfqpoint{2.891276in}{2.334901in}}%
\pgfpathlineto{\pgfqpoint{2.900921in}{2.335742in}}%
\pgfpathlineto{\pgfqpoint{2.906532in}{2.335267in}}%
\pgfpathlineto{\pgfqpoint{2.915291in}{2.335979in}}%
\pgfpathlineto{\pgfqpoint{2.921591in}{2.335678in}}%
\pgfpathlineto{\pgfqpoint{2.926118in}{2.336177in}}%
\pgfpathlineto{\pgfqpoint{2.931138in}{2.336127in}}%
\pgfpathlineto{\pgfqpoint{2.935764in}{2.336242in}}%
\pgfpathlineto{\pgfqpoint{2.944720in}{2.336369in}}%
\pgfpathlineto{\pgfqpoint{2.957516in}{2.336949in}}%
\pgfpathlineto{\pgfqpoint{2.961650in}{2.336643in}}%
\pgfpathlineto{\pgfqpoint{2.965587in}{2.336729in}}%
\pgfpathlineto{\pgfqpoint{2.970213in}{2.336977in}}%
\pgfpathlineto{\pgfqpoint{2.974445in}{2.336564in}}%
\pgfpathlineto{\pgfqpoint{2.979661in}{2.337034in}}%
\pgfpathlineto{\pgfqpoint{2.983500in}{2.336412in}}%
\pgfpathlineto{\pgfqpoint{2.989110in}{2.337055in}}%
\pgfpathlineto{\pgfqpoint{2.992653in}{2.336290in}}%
\pgfpathlineto{\pgfqpoint{2.998559in}{2.337032in}}%
\pgfpathlineto{\pgfqpoint{3.001709in}{2.336166in}}%
\pgfpathlineto{\pgfqpoint{3.008106in}{2.336890in}}%
\pgfpathlineto{\pgfqpoint{3.010862in}{2.336108in}}%
\pgfpathlineto{\pgfqpoint{3.017457in}{2.336861in}}%
\pgfpathlineto{\pgfqpoint{3.020114in}{2.336118in}}%
\pgfpathlineto{\pgfqpoint{3.026610in}{2.336981in}}%
\pgfpathlineto{\pgfqpoint{3.029268in}{2.336135in}}%
\pgfpathlineto{\pgfqpoint{3.035862in}{2.337003in}}%
\pgfpathlineto{\pgfqpoint{3.038421in}{2.336177in}}%
\pgfpathlineto{\pgfqpoint{3.041669in}{2.338241in}}%
\pgfpathlineto{\pgfqpoint{3.043736in}{2.338107in}}%
\pgfpathlineto{\pgfqpoint{3.048165in}{2.336446in}}%
\pgfpathlineto{\pgfqpoint{3.053283in}{2.337834in}}%
\pgfpathlineto{\pgfqpoint{3.057023in}{2.336368in}}%
\pgfpathlineto{\pgfqpoint{3.063027in}{2.337351in}}%
\pgfpathlineto{\pgfqpoint{3.066177in}{2.336405in}}%
\pgfpathlineto{\pgfqpoint{3.072476in}{2.337081in}}%
\pgfpathlineto{\pgfqpoint{3.075527in}{2.336473in}}%
\pgfpathlineto{\pgfqpoint{3.081531in}{2.337025in}}%
\pgfpathlineto{\pgfqpoint{3.084976in}{2.336526in}}%
\pgfpathlineto{\pgfqpoint{3.090783in}{2.336838in}}%
\pgfpathlineto{\pgfqpoint{3.094819in}{2.336598in}}%
\pgfpathlineto{\pgfqpoint{3.100429in}{2.336590in}}%
\pgfpathlineto{\pgfqpoint{3.105941in}{2.336666in}}%
\pgfpathlineto{\pgfqpoint{3.121492in}{2.336481in}}%
\pgfpathlineto{\pgfqpoint{3.193244in}{2.337670in}}%
\pgfpathlineto{\pgfqpoint{3.197673in}{2.338272in}}%
\pgfpathlineto{\pgfqpoint{3.206236in}{2.337275in}}%
\pgfpathlineto{\pgfqpoint{3.212043in}{2.339348in}}%
\pgfpathlineto{\pgfqpoint{3.215980in}{2.337106in}}%
\pgfpathlineto{\pgfqpoint{3.219129in}{2.336547in}}%
\pgfpathlineto{\pgfqpoint{3.222279in}{2.338707in}}%
\pgfpathlineto{\pgfqpoint{3.226118in}{2.340577in}}%
\pgfpathlineto{\pgfqpoint{3.228775in}{2.339072in}}%
\pgfpathlineto{\pgfqpoint{3.233401in}{2.336442in}}%
\pgfpathlineto{\pgfqpoint{3.235862in}{2.338056in}}%
\pgfpathlineto{\pgfqpoint{3.240783in}{2.341612in}}%
\pgfpathlineto{\pgfqpoint{3.243145in}{2.340099in}}%
\pgfpathlineto{\pgfqpoint{3.248165in}{2.336449in}}%
\pgfpathlineto{\pgfqpoint{3.250428in}{2.338116in}}%
\pgfpathlineto{\pgfqpoint{3.255547in}{2.342380in}}%
\pgfpathlineto{\pgfqpoint{3.257712in}{2.340824in}}%
\pgfpathlineto{\pgfqpoint{3.262928in}{2.336453in}}%
\pgfpathlineto{\pgfqpoint{3.265094in}{2.338142in}}%
\pgfpathlineto{\pgfqpoint{3.270212in}{2.342750in}}%
\pgfpathlineto{\pgfqpoint{3.272279in}{2.341264in}}%
\pgfpathlineto{\pgfqpoint{3.277791in}{2.336479in}}%
\pgfpathlineto{\pgfqpoint{3.280054in}{2.338374in}}%
\pgfpathlineto{\pgfqpoint{3.284779in}{2.342586in}}%
\pgfpathlineto{\pgfqpoint{3.286846in}{2.341278in}}%
\pgfpathlineto{\pgfqpoint{3.292653in}{2.336498in}}%
\pgfpathlineto{\pgfqpoint{3.295113in}{2.338505in}}%
\pgfpathlineto{\pgfqpoint{3.299444in}{2.341806in}}%
\pgfpathlineto{\pgfqpoint{3.301708in}{2.340502in}}%
\pgfpathlineto{\pgfqpoint{3.307318in}{2.336451in}}%
\pgfpathlineto{\pgfqpoint{3.310074in}{2.338221in}}%
\pgfpathlineto{\pgfqpoint{3.314208in}{2.340482in}}%
\pgfpathlineto{\pgfqpoint{3.316964in}{2.339050in}}%
\pgfpathlineto{\pgfqpoint{3.321885in}{2.336409in}}%
\pgfpathlineto{\pgfqpoint{3.325330in}{2.337820in}}%
\pgfpathlineto{\pgfqpoint{3.329365in}{2.338792in}}%
\pgfpathlineto{\pgfqpoint{3.334582in}{2.336684in}}%
\pgfpathlineto{\pgfqpoint{3.338617in}{2.336654in}}%
\pgfpathlineto{\pgfqpoint{3.345802in}{2.337033in}}%
\pgfpathlineto{\pgfqpoint{3.355743in}{2.336450in}}%
\pgfpathlineto{\pgfqpoint{3.384385in}{2.337591in}}%
\pgfpathlineto{\pgfqpoint{3.389109in}{2.338574in}}%
\pgfpathlineto{\pgfqpoint{3.397869in}{2.337610in}}%
\pgfpathlineto{\pgfqpoint{3.403184in}{2.340154in}}%
\pgfpathlineto{\pgfqpoint{3.406235in}{2.338244in}}%
\pgfpathlineto{\pgfqpoint{3.410074in}{2.336426in}}%
\pgfpathlineto{\pgfqpoint{3.412633in}{2.337906in}}%
\pgfpathlineto{\pgfqpoint{3.417653in}{2.341041in}}%
\pgfpathlineto{\pgfqpoint{3.420212in}{2.339427in}}%
\pgfpathlineto{\pgfqpoint{3.424838in}{2.336423in}}%
\pgfpathlineto{\pgfqpoint{3.427200in}{2.337916in}}%
\pgfpathlineto{\pgfqpoint{3.432515in}{2.341865in}}%
\pgfpathlineto{\pgfqpoint{3.434877in}{2.340145in}}%
\pgfpathlineto{\pgfqpoint{3.439601in}{2.336421in}}%
\pgfpathlineto{\pgfqpoint{3.441767in}{2.337944in}}%
\pgfpathlineto{\pgfqpoint{3.447475in}{2.342984in}}%
\pgfpathlineto{\pgfqpoint{3.449641in}{2.341001in}}%
\pgfpathlineto{\pgfqpoint{3.454365in}{2.336421in}}%
\pgfpathlineto{\pgfqpoint{3.456235in}{2.337842in}}%
\pgfpathlineto{\pgfqpoint{3.462436in}{2.344293in}}%
\pgfpathlineto{\pgfqpoint{3.464601in}{2.341677in}}%
\pgfpathlineto{\pgfqpoint{3.468932in}{2.336399in}}%
\pgfpathlineto{\pgfqpoint{3.470605in}{2.337534in}}%
\pgfpathlineto{\pgfqpoint{3.474444in}{2.344413in}}%
\pgfpathlineto{\pgfqpoint{3.476708in}{2.345921in}}%
\pgfpathlineto{\pgfqpoint{3.478381in}{2.344346in}}%
\pgfpathlineto{\pgfqpoint{3.484385in}{2.336627in}}%
\pgfpathlineto{\pgfqpoint{3.484483in}{2.336695in}}%
\pgfpathlineto{\pgfqpoint{3.486452in}{2.339615in}}%
\pgfpathlineto{\pgfqpoint{3.491274in}{2.347149in}}%
\pgfpathlineto{\pgfqpoint{3.492849in}{2.345790in}}%
\pgfpathlineto{\pgfqpoint{3.499050in}{2.336574in}}%
\pgfpathlineto{\pgfqpoint{3.499641in}{2.337092in}}%
\pgfpathlineto{\pgfqpoint{3.501904in}{2.341469in}}%
\pgfpathlineto{\pgfqpoint{3.505743in}{2.347873in}}%
\pgfpathlineto{\pgfqpoint{3.507219in}{2.346969in}}%
\pgfpathlineto{\pgfqpoint{3.509778in}{2.341609in}}%
\pgfpathlineto{\pgfqpoint{3.513026in}{2.336422in}}%
\pgfpathlineto{\pgfqpoint{3.514503in}{2.337218in}}%
\pgfpathlineto{\pgfqpoint{3.516865in}{2.342034in}}%
\pgfpathlineto{\pgfqpoint{3.520408in}{2.347935in}}%
\pgfpathlineto{\pgfqpoint{3.521885in}{2.347125in}}%
\pgfpathlineto{\pgfqpoint{3.524345in}{2.342074in}}%
\pgfpathlineto{\pgfqpoint{3.527790in}{2.336425in}}%
\pgfpathlineto{\pgfqpoint{3.529267in}{2.337172in}}%
\pgfpathlineto{\pgfqpoint{3.531727in}{2.341988in}}%
\pgfpathlineto{\pgfqpoint{3.535172in}{2.347254in}}%
\pgfpathlineto{\pgfqpoint{3.536747in}{2.346316in}}%
\pgfpathlineto{\pgfqpoint{3.539503in}{2.340779in}}%
\pgfpathlineto{\pgfqpoint{3.542652in}{2.336409in}}%
\pgfpathlineto{\pgfqpoint{3.544227in}{2.337296in}}%
\pgfpathlineto{\pgfqpoint{3.547278in}{2.342852in}}%
\pgfpathlineto{\pgfqpoint{3.550034in}{2.345794in}}%
\pgfpathlineto{\pgfqpoint{3.551707in}{2.344677in}}%
\pgfpathlineto{\pgfqpoint{3.558696in}{2.336841in}}%
\pgfpathlineto{\pgfqpoint{3.558892in}{2.337007in}}%
\pgfpathlineto{\pgfqpoint{3.561944in}{2.341310in}}%
\pgfpathlineto{\pgfqpoint{3.564798in}{2.343645in}}%
\pgfpathlineto{\pgfqpoint{3.566668in}{2.342509in}}%
\pgfpathlineto{\pgfqpoint{3.573262in}{2.336591in}}%
\pgfpathlineto{\pgfqpoint{3.576117in}{2.339065in}}%
\pgfpathlineto{\pgfqpoint{3.579562in}{2.341090in}}%
\pgfpathlineto{\pgfqpoint{3.582022in}{2.339808in}}%
\pgfpathlineto{\pgfqpoint{3.587534in}{2.336411in}}%
\pgfpathlineto{\pgfqpoint{3.591176in}{2.337857in}}%
\pgfpathlineto{\pgfqpoint{3.595014in}{2.338490in}}%
\pgfpathlineto{\pgfqpoint{3.608203in}{2.336819in}}%
\pgfpathlineto{\pgfqpoint{3.635566in}{2.337617in}}%
\pgfpathlineto{\pgfqpoint{3.640388in}{2.338579in}}%
\pgfpathlineto{\pgfqpoint{3.648065in}{2.337657in}}%
\pgfpathlineto{\pgfqpoint{3.654857in}{2.343304in}}%
\pgfpathlineto{\pgfqpoint{3.657317in}{2.339943in}}%
\pgfpathlineto{\pgfqpoint{3.660664in}{2.336413in}}%
\pgfpathlineto{\pgfqpoint{3.662140in}{2.337747in}}%
\pgfpathlineto{\pgfqpoint{3.664699in}{2.344181in}}%
\pgfpathlineto{\pgfqpoint{3.668144in}{2.351027in}}%
\pgfpathlineto{\pgfqpoint{3.669424in}{2.350192in}}%
\pgfpathlineto{\pgfqpoint{3.671491in}{2.344988in}}%
\pgfpathlineto{\pgfqpoint{3.675231in}{2.336401in}}%
\pgfpathlineto{\pgfqpoint{3.676412in}{2.337453in}}%
\pgfpathlineto{\pgfqpoint{3.678282in}{2.343414in}}%
\pgfpathlineto{\pgfqpoint{3.682908in}{2.358372in}}%
\pgfpathlineto{\pgfqpoint{3.683991in}{2.357175in}}%
\pgfpathlineto{\pgfqpoint{3.685959in}{2.350058in}}%
\pgfpathlineto{\pgfqpoint{3.689995in}{2.336402in}}%
\pgfpathlineto{\pgfqpoint{3.690979in}{2.337347in}}%
\pgfpathlineto{\pgfqpoint{3.692652in}{2.343431in}}%
\pgfpathlineto{\pgfqpoint{3.697573in}{2.363214in}}%
\pgfpathlineto{\pgfqpoint{3.697672in}{2.363192in}}%
\pgfpathlineto{\pgfqpoint{3.698656in}{2.361691in}}%
\pgfpathlineto{\pgfqpoint{3.700624in}{2.353040in}}%
\pgfpathlineto{\pgfqpoint{3.704660in}{2.336425in}}%
\pgfpathlineto{\pgfqpoint{3.705644in}{2.337292in}}%
\pgfpathlineto{\pgfqpoint{3.707219in}{2.343348in}}%
\pgfpathlineto{\pgfqpoint{3.712337in}{2.366136in}}%
\pgfpathlineto{\pgfqpoint{3.712534in}{2.366041in}}%
\pgfpathlineto{\pgfqpoint{3.713617in}{2.363712in}}%
\pgfpathlineto{\pgfqpoint{3.715880in}{2.351584in}}%
\pgfpathlineto{\pgfqpoint{3.719325in}{2.336472in}}%
\pgfpathlineto{\pgfqpoint{3.720211in}{2.337006in}}%
\pgfpathlineto{\pgfqpoint{3.720309in}{2.337203in}}%
\pgfpathlineto{\pgfqpoint{3.721786in}{2.343080in}}%
\pgfpathlineto{\pgfqpoint{3.727101in}{2.369803in}}%
\pgfpathlineto{\pgfqpoint{3.727494in}{2.369482in}}%
\pgfpathlineto{\pgfqpoint{3.728676in}{2.365845in}}%
\pgfpathlineto{\pgfqpoint{3.731628in}{2.346561in}}%
\pgfpathlineto{\pgfqpoint{3.734286in}{2.336397in}}%
\pgfpathlineto{\pgfqpoint{3.735073in}{2.337324in}}%
\pgfpathlineto{\pgfqpoint{3.736550in}{2.343986in}}%
\pgfpathlineto{\pgfqpoint{3.741766in}{2.373872in}}%
\pgfpathlineto{\pgfqpoint{3.742258in}{2.373458in}}%
\pgfpathlineto{\pgfqpoint{3.743439in}{2.369281in}}%
\pgfpathlineto{\pgfqpoint{3.746490in}{2.346917in}}%
\pgfpathlineto{\pgfqpoint{3.749050in}{2.336397in}}%
\pgfpathlineto{\pgfqpoint{3.749837in}{2.337423in}}%
\pgfpathlineto{\pgfqpoint{3.751313in}{2.344710in}}%
\pgfpathlineto{\pgfqpoint{3.756530in}{2.376609in}}%
\pgfpathlineto{\pgfqpoint{3.756924in}{2.376264in}}%
\pgfpathlineto{\pgfqpoint{3.758105in}{2.372008in}}%
\pgfpathlineto{\pgfqpoint{3.760959in}{2.349810in}}%
\pgfpathlineto{\pgfqpoint{3.763715in}{2.336422in}}%
\pgfpathlineto{\pgfqpoint{3.764502in}{2.337217in}}%
\pgfpathlineto{\pgfqpoint{3.765880in}{2.343674in}}%
\pgfpathlineto{\pgfqpoint{3.771294in}{2.377247in}}%
\pgfpathlineto{\pgfqpoint{3.771786in}{2.376639in}}%
\pgfpathlineto{\pgfqpoint{3.773065in}{2.371172in}}%
\pgfpathlineto{\pgfqpoint{3.778675in}{2.336407in}}%
\pgfpathlineto{\pgfqpoint{3.779463in}{2.337743in}}%
\pgfpathlineto{\pgfqpoint{3.781038in}{2.346229in}}%
\pgfpathlineto{\pgfqpoint{3.785959in}{2.375406in}}%
\pgfpathlineto{\pgfqpoint{3.786057in}{2.375375in}}%
\pgfpathlineto{\pgfqpoint{3.786943in}{2.373581in}}%
\pgfpathlineto{\pgfqpoint{3.788715in}{2.363176in}}%
\pgfpathlineto{\pgfqpoint{3.793242in}{2.336421in}}%
\pgfpathlineto{\pgfqpoint{3.794030in}{2.337130in}}%
\pgfpathlineto{\pgfqpoint{3.795408in}{2.342854in}}%
\pgfpathlineto{\pgfqpoint{3.800723in}{2.370871in}}%
\pgfpathlineto{\pgfqpoint{3.801116in}{2.370508in}}%
\pgfpathlineto{\pgfqpoint{3.802396in}{2.366220in}}%
\pgfpathlineto{\pgfqpoint{3.808203in}{2.336402in}}%
\pgfpathlineto{\pgfqpoint{3.809089in}{2.337636in}}%
\pgfpathlineto{\pgfqpoint{3.810860in}{2.345336in}}%
\pgfpathlineto{\pgfqpoint{3.815191in}{2.363879in}}%
\pgfpathlineto{\pgfqpoint{3.816175in}{2.362977in}}%
\pgfpathlineto{\pgfqpoint{3.817849in}{2.356703in}}%
\pgfpathlineto{\pgfqpoint{3.822868in}{2.336401in}}%
\pgfpathlineto{\pgfqpoint{3.823951in}{2.337446in}}%
\pgfpathlineto{\pgfqpoint{3.825919in}{2.343919in}}%
\pgfpathlineto{\pgfqpoint{3.829758in}{2.355116in}}%
\pgfpathlineto{\pgfqpoint{3.830939in}{2.354356in}}%
\pgfpathlineto{\pgfqpoint{3.832809in}{2.349191in}}%
\pgfpathlineto{\pgfqpoint{3.837435in}{2.336462in}}%
\pgfpathlineto{\pgfqpoint{3.838813in}{2.337058in}}%
\pgfpathlineto{\pgfqpoint{3.841274in}{2.341810in}}%
\pgfpathlineto{\pgfqpoint{3.844423in}{2.346040in}}%
\pgfpathlineto{\pgfqpoint{3.846097in}{2.345006in}}%
\pgfpathlineto{\pgfqpoint{3.853971in}{2.336724in}}%
\pgfpathlineto{\pgfqpoint{3.860270in}{2.338536in}}%
\pgfpathlineto{\pgfqpoint{3.870014in}{2.336407in}}%
\pgfpathlineto{\pgfqpoint{3.883892in}{2.337562in}}%
\pgfpathlineto{\pgfqpoint{3.887337in}{2.342146in}}%
\pgfpathlineto{\pgfqpoint{3.890191in}{2.344383in}}%
\pgfpathlineto{\pgfqpoint{3.891864in}{2.342979in}}%
\pgfpathlineto{\pgfqpoint{3.896589in}{2.336537in}}%
\pgfpathlineto{\pgfqpoint{3.897081in}{2.337061in}}%
\pgfpathlineto{\pgfqpoint{3.898656in}{2.341560in}}%
\pgfpathlineto{\pgfqpoint{3.902494in}{2.362812in}}%
\pgfpathlineto{\pgfqpoint{3.904463in}{2.367673in}}%
\pgfpathlineto{\pgfqpoint{3.904659in}{2.367585in}}%
\pgfpathlineto{\pgfqpoint{3.905644in}{2.365430in}}%
\pgfpathlineto{\pgfqpoint{3.907612in}{2.354044in}}%
\pgfpathlineto{\pgfqpoint{3.911057in}{2.336396in}}%
\pgfpathlineto{\pgfqpoint{3.911746in}{2.337578in}}%
\pgfpathlineto{\pgfqpoint{3.913026in}{2.345873in}}%
\pgfpathlineto{\pgfqpoint{3.915683in}{2.382121in}}%
\pgfpathlineto{\pgfqpoint{3.918833in}{2.413994in}}%
\pgfpathlineto{\pgfqpoint{3.919128in}{2.414088in}}%
\pgfpathlineto{\pgfqpoint{3.919423in}{2.413554in}}%
\pgfpathlineto{\pgfqpoint{3.919423in}{2.413554in}}%
\pgfpathlineto{\pgfqpoint{3.920506in}{2.406316in}}%
\pgfpathlineto{\pgfqpoint{3.922868in}{2.369826in}}%
\pgfpathlineto{\pgfqpoint{3.925821in}{2.336389in}}%
\pgfpathlineto{\pgfqpoint{3.926313in}{2.337545in}}%
\pgfpathlineto{\pgfqpoint{3.927396in}{2.349021in}}%
\pgfpathlineto{\pgfqpoint{3.929561in}{2.403152in}}%
\pgfpathlineto{\pgfqpoint{3.933498in}{2.492761in}}%
\pgfpathlineto{\pgfqpoint{3.933695in}{2.493051in}}%
\pgfpathlineto{\pgfqpoint{3.933892in}{2.492783in}}%
\pgfpathlineto{\pgfqpoint{3.933892in}{2.492783in}}%
\pgfpathlineto{\pgfqpoint{3.934679in}{2.486149in}}%
\pgfpathlineto{\pgfqpoint{3.936352in}{2.446072in}}%
\pgfpathlineto{\pgfqpoint{3.940585in}{2.336387in}}%
\pgfpathlineto{\pgfqpoint{3.940781in}{2.336636in}}%
\pgfpathlineto{\pgfqpoint{3.941470in}{2.343452in}}%
\pgfpathlineto{\pgfqpoint{3.942947in}{2.388317in}}%
\pgfpathlineto{\pgfqpoint{3.948360in}{2.615170in}}%
\pgfpathlineto{\pgfqpoint{3.949049in}{2.609681in}}%
\pgfpathlineto{\pgfqpoint{3.950329in}{2.569006in}}%
\pgfpathlineto{\pgfqpoint{3.955348in}{2.336383in}}%
\pgfpathlineto{\pgfqpoint{3.956136in}{2.345539in}}%
\pgfpathlineto{\pgfqpoint{3.957514in}{2.408722in}}%
\pgfpathlineto{\pgfqpoint{3.963025in}{2.769626in}}%
\pgfpathlineto{\pgfqpoint{3.963813in}{2.756579in}}%
\pgfpathlineto{\pgfqpoint{3.965289in}{2.671796in}}%
\pgfpathlineto{\pgfqpoint{3.970112in}{2.336385in}}%
\pgfpathlineto{\pgfqpoint{3.970506in}{2.339043in}}%
\pgfpathlineto{\pgfqpoint{3.971490in}{2.374374in}}%
\pgfpathlineto{\pgfqpoint{3.973655in}{2.558463in}}%
\pgfpathlineto{\pgfqpoint{3.977100in}{2.802675in}}%
\pgfpathlineto{\pgfqpoint{3.977396in}{2.804483in}}%
\pgfpathlineto{\pgfqpoint{3.977592in}{2.803572in}}%
\pgfpathlineto{\pgfqpoint{3.977592in}{2.803572in}}%
\pgfpathlineto{\pgfqpoint{3.978380in}{2.783408in}}%
\pgfpathlineto{\pgfqpoint{3.980151in}{2.658621in}}%
\pgfpathlineto{\pgfqpoint{3.984679in}{2.337603in}}%
\pgfpathlineto{\pgfqpoint{3.984974in}{2.336372in}}%
\pgfpathlineto{\pgfqpoint{3.985171in}{2.337225in}}%
\pgfpathlineto{\pgfqpoint{3.985171in}{2.337225in}}%
\pgfpathlineto{\pgfqpoint{3.985959in}{2.353106in}}%
\pgfpathlineto{\pgfqpoint{3.987927in}{2.454167in}}%
\pgfpathlineto{\pgfqpoint{3.991372in}{2.604449in}}%
\pgfpathlineto{\pgfqpoint{3.991667in}{2.605710in}}%
\pgfpathlineto{\pgfqpoint{3.991962in}{2.604748in}}%
\pgfpathlineto{\pgfqpoint{3.991962in}{2.604748in}}%
\pgfpathlineto{\pgfqpoint{3.992848in}{2.589242in}}%
\pgfpathlineto{\pgfqpoint{3.994915in}{2.499735in}}%
\pgfpathlineto{\pgfqpoint{3.998951in}{2.341583in}}%
\pgfpathlineto{\pgfqpoint{3.999836in}{2.336389in}}%
\pgfpathlineto{\pgfqpoint{4.000329in}{2.338097in}}%
\pgfpathlineto{\pgfqpoint{4.001707in}{2.355418in}}%
\pgfpathlineto{\pgfqpoint{4.006037in}{2.411559in}}%
\pgfpathlineto{\pgfqpoint{4.006628in}{2.410412in}}%
\pgfpathlineto{\pgfqpoint{4.007907in}{2.400505in}}%
\pgfpathlineto{\pgfqpoint{4.014699in}{2.336455in}}%
\pgfpathlineto{\pgfqpoint{4.015781in}{2.337057in}}%
\pgfpathlineto{\pgfqpoint{4.021490in}{2.343976in}}%
\pgfpathlineto{\pgfqpoint{4.021588in}{2.343904in}}%
\pgfpathlineto{\pgfqpoint{4.024738in}{2.340058in}}%
\pgfpathlineto{\pgfqpoint{4.028577in}{2.336973in}}%
\pgfpathlineto{\pgfqpoint{4.033596in}{2.336464in}}%
\pgfpathlineto{\pgfqpoint{4.045801in}{2.337533in}}%
\pgfpathlineto{\pgfqpoint{4.051116in}{2.338651in}}%
\pgfpathlineto{\pgfqpoint{4.059777in}{2.337555in}}%
\pgfpathlineto{\pgfqpoint{4.065486in}{2.340604in}}%
\pgfpathlineto{\pgfqpoint{4.068340in}{2.338671in}}%
\pgfpathlineto{\pgfqpoint{4.072376in}{2.336427in}}%
\pgfpathlineto{\pgfqpoint{4.074738in}{2.337945in}}%
\pgfpathlineto{\pgfqpoint{4.080250in}{2.342270in}}%
\pgfpathlineto{\pgfqpoint{4.082513in}{2.340551in}}%
\pgfpathlineto{\pgfqpoint{4.087435in}{2.336449in}}%
\pgfpathlineto{\pgfqpoint{4.089403in}{2.338075in}}%
\pgfpathlineto{\pgfqpoint{4.095604in}{2.344641in}}%
\pgfpathlineto{\pgfqpoint{4.097671in}{2.342156in}}%
\pgfpathlineto{\pgfqpoint{4.102297in}{2.336428in}}%
\pgfpathlineto{\pgfqpoint{4.103872in}{2.337922in}}%
\pgfpathlineto{\pgfqpoint{4.106923in}{2.345094in}}%
\pgfpathlineto{\pgfqpoint{4.109777in}{2.349394in}}%
\pgfpathlineto{\pgfqpoint{4.111155in}{2.348495in}}%
\pgfpathlineto{\pgfqpoint{4.113419in}{2.343293in}}%
\pgfpathlineto{\pgfqpoint{4.116962in}{2.336402in}}%
\pgfpathlineto{\pgfqpoint{4.118242in}{2.337469in}}%
\pgfpathlineto{\pgfqpoint{4.120210in}{2.343327in}}%
\pgfpathlineto{\pgfqpoint{4.124836in}{2.357352in}}%
\pgfpathlineto{\pgfqpoint{4.125919in}{2.356360in}}%
\pgfpathlineto{\pgfqpoint{4.127789in}{2.350225in}}%
\pgfpathlineto{\pgfqpoint{4.132021in}{2.336399in}}%
\pgfpathlineto{\pgfqpoint{4.133005in}{2.337656in}}%
\pgfpathlineto{\pgfqpoint{4.134679in}{2.344816in}}%
\pgfpathlineto{\pgfqpoint{4.139895in}{2.370686in}}%
\pgfpathlineto{\pgfqpoint{4.140092in}{2.370575in}}%
\pgfpathlineto{\pgfqpoint{4.141076in}{2.368266in}}%
\pgfpathlineto{\pgfqpoint{4.143143in}{2.355835in}}%
\pgfpathlineto{\pgfqpoint{4.146785in}{2.336416in}}%
\pgfpathlineto{\pgfqpoint{4.147572in}{2.337319in}}%
\pgfpathlineto{\pgfqpoint{4.148950in}{2.344687in}}%
\pgfpathlineto{\pgfqpoint{4.154757in}{2.392224in}}%
\pgfpathlineto{\pgfqpoint{4.155643in}{2.389948in}}%
\pgfpathlineto{\pgfqpoint{4.157316in}{2.376189in}}%
\pgfpathlineto{\pgfqpoint{4.161746in}{2.336393in}}%
\pgfpathlineto{\pgfqpoint{4.161942in}{2.336503in}}%
\pgfpathlineto{\pgfqpoint{4.162730in}{2.339433in}}%
\pgfpathlineto{\pgfqpoint{4.164305in}{2.356526in}}%
\pgfpathlineto{\pgfqpoint{4.169521in}{2.425824in}}%
\pgfpathlineto{\pgfqpoint{4.169915in}{2.425207in}}%
\pgfpathlineto{\pgfqpoint{4.170899in}{2.418304in}}%
\pgfpathlineto{\pgfqpoint{4.173064in}{2.382779in}}%
\pgfpathlineto{\pgfqpoint{4.176509in}{2.336424in}}%
\pgfpathlineto{\pgfqpoint{4.176903in}{2.336868in}}%
\pgfpathlineto{\pgfqpoint{4.177001in}{2.337226in}}%
\pgfpathlineto{\pgfqpoint{4.177986in}{2.346220in}}%
\pgfpathlineto{\pgfqpoint{4.180053in}{2.391873in}}%
\pgfpathlineto{\pgfqpoint{4.184186in}{2.476394in}}%
\pgfpathlineto{\pgfqpoint{4.184383in}{2.476514in}}%
\pgfpathlineto{\pgfqpoint{4.184580in}{2.476142in}}%
\pgfpathlineto{\pgfqpoint{4.184580in}{2.476142in}}%
\pgfpathlineto{\pgfqpoint{4.185368in}{2.469786in}}%
\pgfpathlineto{\pgfqpoint{4.187139in}{2.431141in}}%
\pgfpathlineto{\pgfqpoint{4.191371in}{2.336394in}}%
\pgfpathlineto{\pgfqpoint{4.191470in}{2.336407in}}%
\pgfpathlineto{\pgfqpoint{4.191962in}{2.338766in}}%
\pgfpathlineto{\pgfqpoint{4.193143in}{2.359825in}}%
\pgfpathlineto{\pgfqpoint{4.195702in}{2.458243in}}%
\pgfpathlineto{\pgfqpoint{4.198852in}{2.549673in}}%
\pgfpathlineto{\pgfqpoint{4.199147in}{2.550359in}}%
\pgfpathlineto{\pgfqpoint{4.199344in}{2.549877in}}%
\pgfpathlineto{\pgfqpoint{4.199344in}{2.549877in}}%
\pgfpathlineto{\pgfqpoint{4.200131in}{2.540498in}}%
\pgfpathlineto{\pgfqpoint{4.201805in}{2.486309in}}%
\pgfpathlineto{\pgfqpoint{4.206234in}{2.336377in}}%
\pgfpathlineto{\pgfqpoint{4.206332in}{2.336540in}}%
\pgfpathlineto{\pgfqpoint{4.206923in}{2.342354in}}%
\pgfpathlineto{\pgfqpoint{4.208301in}{2.386965in}}%
\pgfpathlineto{\pgfqpoint{4.213911in}{2.651986in}}%
\pgfpathlineto{\pgfqpoint{4.214698in}{2.642482in}}%
\pgfpathlineto{\pgfqpoint{4.216175in}{2.581237in}}%
\pgfpathlineto{\pgfqpoint{4.220997in}{2.336359in}}%
\pgfpathlineto{\pgfqpoint{4.221391in}{2.338948in}}%
\pgfpathlineto{\pgfqpoint{4.222375in}{2.368410in}}%
\pgfpathlineto{\pgfqpoint{4.224442in}{2.516839in}}%
\pgfpathlineto{\pgfqpoint{4.228379in}{2.775386in}}%
\pgfpathlineto{\pgfqpoint{4.228576in}{2.776430in}}%
\pgfpathlineto{\pgfqpoint{4.228773in}{2.775924in}}%
\pgfpathlineto{\pgfqpoint{4.228773in}{2.775924in}}%
\pgfpathlineto{\pgfqpoint{4.229462in}{2.762009in}}%
\pgfpathlineto{\pgfqpoint{4.231037in}{2.667124in}}%
\pgfpathlineto{\pgfqpoint{4.235761in}{2.336354in}}%
\pgfpathlineto{\pgfqpoint{4.236056in}{2.338067in}}%
\pgfpathlineto{\pgfqpoint{4.236942in}{2.366827in}}%
\pgfpathlineto{\pgfqpoint{4.238812in}{2.525681in}}%
\pgfpathlineto{\pgfqpoint{4.243143in}{2.889345in}}%
\pgfpathlineto{\pgfqpoint{4.243241in}{2.889794in}}%
\pgfpathlineto{\pgfqpoint{4.243340in}{2.889753in}}%
\pgfpathlineto{\pgfqpoint{4.243340in}{2.889753in}}%
\pgfpathlineto{\pgfqpoint{4.243832in}{2.882223in}}%
\pgfpathlineto{\pgfqpoint{4.245013in}{2.817100in}}%
\pgfpathlineto{\pgfqpoint{4.248556in}{2.434458in}}%
\pgfpathlineto{\pgfqpoint{4.250525in}{2.336360in}}%
\pgfpathlineto{\pgfqpoint{4.250722in}{2.337070in}}%
\pgfpathlineto{\pgfqpoint{4.251411in}{2.355445in}}%
\pgfpathlineto{\pgfqpoint{4.252986in}{2.478935in}}%
\pgfpathlineto{\pgfqpoint{4.257907in}{2.909374in}}%
\pgfpathlineto{\pgfqpoint{4.258005in}{2.908865in}}%
\pgfpathlineto{\pgfqpoint{4.258694in}{2.891203in}}%
\pgfpathlineto{\pgfqpoint{4.260269in}{2.770036in}}%
\pgfpathlineto{\pgfqpoint{4.265289in}{2.336372in}}%
\pgfpathlineto{\pgfqpoint{4.265387in}{2.336437in}}%
\pgfpathlineto{\pgfqpoint{4.265879in}{2.343418in}}%
\pgfpathlineto{\pgfqpoint{4.267159in}{2.407804in}}%
\pgfpathlineto{\pgfqpoint{4.272277in}{2.750556in}}%
\pgfpathlineto{\pgfqpoint{4.272769in}{2.745753in}}%
\pgfpathlineto{\pgfqpoint{4.273950in}{2.698978in}}%
\pgfpathlineto{\pgfqpoint{4.280052in}{2.336388in}}%
\pgfpathlineto{\pgfqpoint{4.280938in}{2.344774in}}%
\pgfpathlineto{\pgfqpoint{4.282808in}{2.404254in}}%
\pgfpathlineto{\pgfqpoint{4.286155in}{2.493494in}}%
\pgfpathlineto{\pgfqpoint{4.286450in}{2.494093in}}%
\pgfpathlineto{\pgfqpoint{4.286647in}{2.493726in}}%
\pgfpathlineto{\pgfqpoint{4.286647in}{2.493726in}}%
\pgfpathlineto{\pgfqpoint{4.287434in}{2.486479in}}%
\pgfpathlineto{\pgfqpoint{4.289304in}{2.441637in}}%
\pgfpathlineto{\pgfqpoint{4.293733in}{2.340350in}}%
\pgfpathlineto{\pgfqpoint{4.294915in}{2.336402in}}%
\pgfpathlineto{\pgfqpoint{4.295308in}{2.336908in}}%
\pgfpathlineto{\pgfqpoint{4.296686in}{2.342908in}}%
\pgfpathlineto{\pgfqpoint{4.300131in}{2.356519in}}%
\pgfpathlineto{\pgfqpoint{4.301115in}{2.355739in}}%
\pgfpathlineto{\pgfqpoint{4.303084in}{2.349609in}}%
\pgfpathlineto{\pgfqpoint{4.307316in}{2.337514in}}%
\pgfpathlineto{\pgfqpoint{4.309678in}{2.336400in}}%
\pgfpathlineto{\pgfqpoint{4.358596in}{2.336400in}}%
\pgfpathlineto{\pgfqpoint{6.857902in}{2.336400in}}%
\pgfpathlineto{\pgfqpoint{6.857902in}{2.336400in}}%
\pgfusepath{stroke}%
\end{pgfscope}%
\begin{pgfscope}%
\pgfsetrectcap%
\pgfsetmiterjoin%
\pgfsetlinewidth{0.803000pt}%
\definecolor{currentstroke}{rgb}{0.000000,0.000000,0.000000}%
\pgfsetstrokecolor{currentstroke}%
\pgfsetdash{}{0pt}%
\pgfpathmoveto{\pgfqpoint{0.952500in}{0.519200in}}%
\pgfpathlineto{\pgfqpoint{0.952500in}{4.153600in}}%
\pgfusepath{stroke}%
\end{pgfscope}%
\begin{pgfscope}%
\pgfsetrectcap%
\pgfsetmiterjoin%
\pgfsetlinewidth{0.803000pt}%
\definecolor{currentstroke}{rgb}{0.000000,0.000000,0.000000}%
\pgfsetstrokecolor{currentstroke}%
\pgfsetdash{}{0pt}%
\pgfpathmoveto{\pgfqpoint{6.858000in}{0.519200in}}%
\pgfpathlineto{\pgfqpoint{6.858000in}{4.153600in}}%
\pgfusepath{stroke}%
\end{pgfscope}%
\begin{pgfscope}%
\pgfsetrectcap%
\pgfsetmiterjoin%
\pgfsetlinewidth{0.803000pt}%
\definecolor{currentstroke}{rgb}{0.000000,0.000000,0.000000}%
\pgfsetstrokecolor{currentstroke}%
\pgfsetdash{}{0pt}%
\pgfpathmoveto{\pgfqpoint{0.952500in}{0.519200in}}%
\pgfpathlineto{\pgfqpoint{6.858000in}{0.519200in}}%
\pgfusepath{stroke}%
\end{pgfscope}%
\begin{pgfscope}%
\pgfsetrectcap%
\pgfsetmiterjoin%
\pgfsetlinewidth{0.803000pt}%
\definecolor{currentstroke}{rgb}{0.000000,0.000000,0.000000}%
\pgfsetstrokecolor{currentstroke}%
\pgfsetdash{}{0pt}%
\pgfpathmoveto{\pgfqpoint{0.952500in}{4.153600in}}%
\pgfpathlineto{\pgfqpoint{6.858000in}{4.153600in}}%
\pgfusepath{stroke}%
\end{pgfscope}%
\begin{pgfscope}%
\pgftext[x=0.952500in,y=4.335320in,left,base]{\sffamily\fontsize{10.000000}{12.000000}\selectfont Iterations: 37810, Time: 0.266 ps, RXPWR: 80.6 percent, TXPWR: 19.4 percent}%
\end{pgfscope}%
\end{pgfpicture}%
\makeatother%
\endgroup%
}}
        \subcaption{Simulation results using a wave with 10 peroids.}
        \label{fig:task5_1}
    \end{subfigure}\\
    \begin{subfigure}[b]{0.49\textwidth}
        \noindent\makebox[\textwidth]{\scalebox{0.7}{%% Creator: Matplotlib, PGF backend
%%
%% To include the figure in your LaTeX document, write
%%   \input{<filename>.pgf}
%%
%% Make sure the required packages are loaded in your preamble
%%   \usepackage{pgf}
%%
%% Figures using additional raster images can only be included by \input if
%% they are in the same directory as the main LaTeX file. For loading figures
%% from other directories you can use the `import` package
%%   \usepackage{import}
%% and then include the figures with
%%   \import{<path to file>}{<filename>.pgf}
%%
%% Matplotlib used the following preamble
%%   \usepackage{fontspec}
%%   \setmainfont{DejaVu Serif}
%%   \setsansfont{DejaVu Sans}
%%   \setmonofont{DejaVu Sans Mono}
%%
\begingroup%
\makeatletter%
\begin{pgfpicture}%
\pgfpathrectangle{\pgfpointorigin}{\pgfqpoint{7.590000in}{4.670000in}}%
\pgfusepath{use as bounding box, clip}%
\begin{pgfscope}%
\pgfsetbuttcap%
\pgfsetmiterjoin%
\definecolor{currentfill}{rgb}{1.000000,1.000000,1.000000}%
\pgfsetfillcolor{currentfill}%
\pgfsetlinewidth{0.000000pt}%
\definecolor{currentstroke}{rgb}{1.000000,1.000000,1.000000}%
\pgfsetstrokecolor{currentstroke}%
\pgfsetdash{}{0pt}%
\pgfpathmoveto{\pgfqpoint{0.000000in}{0.000000in}}%
\pgfpathlineto{\pgfqpoint{7.590000in}{0.000000in}}%
\pgfpathlineto{\pgfqpoint{7.590000in}{4.670000in}}%
\pgfpathlineto{\pgfqpoint{0.000000in}{4.670000in}}%
\pgfpathclose%
\pgfusepath{fill}%
\end{pgfscope}%
\begin{pgfscope}%
\pgfsetbuttcap%
\pgfsetmiterjoin%
\definecolor{currentfill}{rgb}{1.000000,1.000000,1.000000}%
\pgfsetfillcolor{currentfill}%
\pgfsetlinewidth{0.000000pt}%
\definecolor{currentstroke}{rgb}{0.000000,0.000000,0.000000}%
\pgfsetstrokecolor{currentstroke}%
\pgfsetstrokeopacity{0.000000}%
\pgfsetdash{}{0pt}%
\pgfpathmoveto{\pgfqpoint{0.948750in}{0.513700in}}%
\pgfpathlineto{\pgfqpoint{6.831000in}{0.513700in}}%
\pgfpathlineto{\pgfqpoint{6.831000in}{4.109600in}}%
\pgfpathlineto{\pgfqpoint{0.948750in}{4.109600in}}%
\pgfpathclose%
\pgfusepath{fill}%
\end{pgfscope}%
\begin{pgfscope}%
\pgfsetbuttcap%
\pgfsetroundjoin%
\definecolor{currentfill}{rgb}{0.000000,0.000000,0.000000}%
\pgfsetfillcolor{currentfill}%
\pgfsetlinewidth{0.803000pt}%
\definecolor{currentstroke}{rgb}{0.000000,0.000000,0.000000}%
\pgfsetstrokecolor{currentstroke}%
\pgfsetdash{}{0pt}%
\pgfsys@defobject{currentmarker}{\pgfqpoint{0.000000in}{-0.048611in}}{\pgfqpoint{0.000000in}{0.000000in}}{%
\pgfpathmoveto{\pgfqpoint{0.000000in}{0.000000in}}%
\pgfpathlineto{\pgfqpoint{0.000000in}{-0.048611in}}%
\pgfusepath{stroke,fill}%
}%
\begin{pgfscope}%
\pgfsys@transformshift{0.948750in}{0.513700in}%
\pgfsys@useobject{currentmarker}{}%
\end{pgfscope}%
\end{pgfscope}%
\begin{pgfscope}%
\pgftext[x=0.948750in,y=0.416478in,,top]{\sffamily\fontsize{10.000000}{12.000000}\selectfont 0}%
\end{pgfscope}%
\begin{pgfscope}%
\pgfsetbuttcap%
\pgfsetroundjoin%
\definecolor{currentfill}{rgb}{0.000000,0.000000,0.000000}%
\pgfsetfillcolor{currentfill}%
\pgfsetlinewidth{0.803000pt}%
\definecolor{currentstroke}{rgb}{0.000000,0.000000,0.000000}%
\pgfsetstrokecolor{currentstroke}%
\pgfsetdash{}{0pt}%
\pgfsys@defobject{currentmarker}{\pgfqpoint{0.000000in}{-0.048611in}}{\pgfqpoint{0.000000in}{0.000000in}}{%
\pgfpathmoveto{\pgfqpoint{0.000000in}{0.000000in}}%
\pgfpathlineto{\pgfqpoint{0.000000in}{-0.048611in}}%
\pgfusepath{stroke,fill}%
}%
\begin{pgfscope}%
\pgfsys@transformshift{1.878015in}{0.513700in}%
\pgfsys@useobject{currentmarker}{}%
\end{pgfscope}%
\end{pgfscope}%
\begin{pgfscope}%
\pgftext[x=1.878015in,y=0.416478in,,top]{\sffamily\fontsize{10.000000}{12.000000}\selectfont 20}%
\end{pgfscope}%
\begin{pgfscope}%
\pgfsetbuttcap%
\pgfsetroundjoin%
\definecolor{currentfill}{rgb}{0.000000,0.000000,0.000000}%
\pgfsetfillcolor{currentfill}%
\pgfsetlinewidth{0.803000pt}%
\definecolor{currentstroke}{rgb}{0.000000,0.000000,0.000000}%
\pgfsetstrokecolor{currentstroke}%
\pgfsetdash{}{0pt}%
\pgfsys@defobject{currentmarker}{\pgfqpoint{0.000000in}{-0.048611in}}{\pgfqpoint{0.000000in}{0.000000in}}{%
\pgfpathmoveto{\pgfqpoint{0.000000in}{0.000000in}}%
\pgfpathlineto{\pgfqpoint{0.000000in}{-0.048611in}}%
\pgfusepath{stroke,fill}%
}%
\begin{pgfscope}%
\pgfsys@transformshift{2.807281in}{0.513700in}%
\pgfsys@useobject{currentmarker}{}%
\end{pgfscope}%
\end{pgfscope}%
\begin{pgfscope}%
\pgftext[x=2.807281in,y=0.416478in,,top]{\sffamily\fontsize{10.000000}{12.000000}\selectfont 40}%
\end{pgfscope}%
\begin{pgfscope}%
\pgfsetbuttcap%
\pgfsetroundjoin%
\definecolor{currentfill}{rgb}{0.000000,0.000000,0.000000}%
\pgfsetfillcolor{currentfill}%
\pgfsetlinewidth{0.803000pt}%
\definecolor{currentstroke}{rgb}{0.000000,0.000000,0.000000}%
\pgfsetstrokecolor{currentstroke}%
\pgfsetdash{}{0pt}%
\pgfsys@defobject{currentmarker}{\pgfqpoint{0.000000in}{-0.048611in}}{\pgfqpoint{0.000000in}{0.000000in}}{%
\pgfpathmoveto{\pgfqpoint{0.000000in}{0.000000in}}%
\pgfpathlineto{\pgfqpoint{0.000000in}{-0.048611in}}%
\pgfusepath{stroke,fill}%
}%
\begin{pgfscope}%
\pgfsys@transformshift{3.736546in}{0.513700in}%
\pgfsys@useobject{currentmarker}{}%
\end{pgfscope}%
\end{pgfscope}%
\begin{pgfscope}%
\pgftext[x=3.736546in,y=0.416478in,,top]{\sffamily\fontsize{10.000000}{12.000000}\selectfont 60}%
\end{pgfscope}%
\begin{pgfscope}%
\pgfsetbuttcap%
\pgfsetroundjoin%
\definecolor{currentfill}{rgb}{0.000000,0.000000,0.000000}%
\pgfsetfillcolor{currentfill}%
\pgfsetlinewidth{0.803000pt}%
\definecolor{currentstroke}{rgb}{0.000000,0.000000,0.000000}%
\pgfsetstrokecolor{currentstroke}%
\pgfsetdash{}{0pt}%
\pgfsys@defobject{currentmarker}{\pgfqpoint{0.000000in}{-0.048611in}}{\pgfqpoint{0.000000in}{0.000000in}}{%
\pgfpathmoveto{\pgfqpoint{0.000000in}{0.000000in}}%
\pgfpathlineto{\pgfqpoint{0.000000in}{-0.048611in}}%
\pgfusepath{stroke,fill}%
}%
\begin{pgfscope}%
\pgfsys@transformshift{4.665812in}{0.513700in}%
\pgfsys@useobject{currentmarker}{}%
\end{pgfscope}%
\end{pgfscope}%
\begin{pgfscope}%
\pgftext[x=4.665812in,y=0.416478in,,top]{\sffamily\fontsize{10.000000}{12.000000}\selectfont 80}%
\end{pgfscope}%
\begin{pgfscope}%
\pgfsetbuttcap%
\pgfsetroundjoin%
\definecolor{currentfill}{rgb}{0.000000,0.000000,0.000000}%
\pgfsetfillcolor{currentfill}%
\pgfsetlinewidth{0.803000pt}%
\definecolor{currentstroke}{rgb}{0.000000,0.000000,0.000000}%
\pgfsetstrokecolor{currentstroke}%
\pgfsetdash{}{0pt}%
\pgfsys@defobject{currentmarker}{\pgfqpoint{0.000000in}{-0.048611in}}{\pgfqpoint{0.000000in}{0.000000in}}{%
\pgfpathmoveto{\pgfqpoint{0.000000in}{0.000000in}}%
\pgfpathlineto{\pgfqpoint{0.000000in}{-0.048611in}}%
\pgfusepath{stroke,fill}%
}%
\begin{pgfscope}%
\pgfsys@transformshift{5.595077in}{0.513700in}%
\pgfsys@useobject{currentmarker}{}%
\end{pgfscope}%
\end{pgfscope}%
\begin{pgfscope}%
\pgftext[x=5.595077in,y=0.416478in,,top]{\sffamily\fontsize{10.000000}{12.000000}\selectfont 100}%
\end{pgfscope}%
\begin{pgfscope}%
\pgfsetbuttcap%
\pgfsetroundjoin%
\definecolor{currentfill}{rgb}{0.000000,0.000000,0.000000}%
\pgfsetfillcolor{currentfill}%
\pgfsetlinewidth{0.803000pt}%
\definecolor{currentstroke}{rgb}{0.000000,0.000000,0.000000}%
\pgfsetstrokecolor{currentstroke}%
\pgfsetdash{}{0pt}%
\pgfsys@defobject{currentmarker}{\pgfqpoint{0.000000in}{-0.048611in}}{\pgfqpoint{0.000000in}{0.000000in}}{%
\pgfpathmoveto{\pgfqpoint{0.000000in}{0.000000in}}%
\pgfpathlineto{\pgfqpoint{0.000000in}{-0.048611in}}%
\pgfusepath{stroke,fill}%
}%
\begin{pgfscope}%
\pgfsys@transformshift{6.524342in}{0.513700in}%
\pgfsys@useobject{currentmarker}{}%
\end{pgfscope}%
\end{pgfscope}%
\begin{pgfscope}%
\pgftext[x=6.524342in,y=0.416478in,,top]{\sffamily\fontsize{10.000000}{12.000000}\selectfont 120}%
\end{pgfscope}%
\begin{pgfscope}%
\pgftext[x=3.889875in,y=0.226509in,,top]{\sffamily\fontsize{16.000000}{19.200000}\selectfont \(\displaystyle z-position [\mu m]\)}%
\end{pgfscope}%
\begin{pgfscope}%
\pgfsetbuttcap%
\pgfsetroundjoin%
\definecolor{currentfill}{rgb}{0.000000,0.000000,0.000000}%
\pgfsetfillcolor{currentfill}%
\pgfsetlinewidth{0.803000pt}%
\definecolor{currentstroke}{rgb}{0.000000,0.000000,0.000000}%
\pgfsetstrokecolor{currentstroke}%
\pgfsetdash{}{0pt}%
\pgfsys@defobject{currentmarker}{\pgfqpoint{-0.048611in}{0.000000in}}{\pgfqpoint{0.000000in}{0.000000in}}{%
\pgfpathmoveto{\pgfqpoint{0.000000in}{0.000000in}}%
\pgfpathlineto{\pgfqpoint{-0.048611in}{0.000000in}}%
\pgfusepath{stroke,fill}%
}%
\begin{pgfscope}%
\pgfsys@transformshift{0.948750in}{0.513700in}%
\pgfsys@useobject{currentmarker}{}%
\end{pgfscope}%
\end{pgfscope}%
\begin{pgfscope}%
\pgftext[x=0.337544in,y=0.460938in,left,base]{\sffamily\fontsize{10.000000}{12.000000}\selectfont -0.003}%
\end{pgfscope}%
\begin{pgfscope}%
\pgfsetbuttcap%
\pgfsetroundjoin%
\definecolor{currentfill}{rgb}{0.000000,0.000000,0.000000}%
\pgfsetfillcolor{currentfill}%
\pgfsetlinewidth{0.803000pt}%
\definecolor{currentstroke}{rgb}{0.000000,0.000000,0.000000}%
\pgfsetstrokecolor{currentstroke}%
\pgfsetdash{}{0pt}%
\pgfsys@defobject{currentmarker}{\pgfqpoint{-0.048611in}{0.000000in}}{\pgfqpoint{0.000000in}{0.000000in}}{%
\pgfpathmoveto{\pgfqpoint{0.000000in}{0.000000in}}%
\pgfpathlineto{\pgfqpoint{-0.048611in}{0.000000in}}%
\pgfusepath{stroke,fill}%
}%
\begin{pgfscope}%
\pgfsys@transformshift{0.948750in}{1.113017in}%
\pgfsys@useobject{currentmarker}{}%
\end{pgfscope}%
\end{pgfscope}%
\begin{pgfscope}%
\pgftext[x=0.337544in,y=1.060255in,left,base]{\sffamily\fontsize{10.000000}{12.000000}\selectfont -0.002}%
\end{pgfscope}%
\begin{pgfscope}%
\pgfsetbuttcap%
\pgfsetroundjoin%
\definecolor{currentfill}{rgb}{0.000000,0.000000,0.000000}%
\pgfsetfillcolor{currentfill}%
\pgfsetlinewidth{0.803000pt}%
\definecolor{currentstroke}{rgb}{0.000000,0.000000,0.000000}%
\pgfsetstrokecolor{currentstroke}%
\pgfsetdash{}{0pt}%
\pgfsys@defobject{currentmarker}{\pgfqpoint{-0.048611in}{0.000000in}}{\pgfqpoint{0.000000in}{0.000000in}}{%
\pgfpathmoveto{\pgfqpoint{0.000000in}{0.000000in}}%
\pgfpathlineto{\pgfqpoint{-0.048611in}{0.000000in}}%
\pgfusepath{stroke,fill}%
}%
\begin{pgfscope}%
\pgfsys@transformshift{0.948750in}{1.712333in}%
\pgfsys@useobject{currentmarker}{}%
\end{pgfscope}%
\end{pgfscope}%
\begin{pgfscope}%
\pgftext[x=0.337544in,y=1.659572in,left,base]{\sffamily\fontsize{10.000000}{12.000000}\selectfont -0.001}%
\end{pgfscope}%
\begin{pgfscope}%
\pgfsetbuttcap%
\pgfsetroundjoin%
\definecolor{currentfill}{rgb}{0.000000,0.000000,0.000000}%
\pgfsetfillcolor{currentfill}%
\pgfsetlinewidth{0.803000pt}%
\definecolor{currentstroke}{rgb}{0.000000,0.000000,0.000000}%
\pgfsetstrokecolor{currentstroke}%
\pgfsetdash{}{0pt}%
\pgfsys@defobject{currentmarker}{\pgfqpoint{-0.048611in}{0.000000in}}{\pgfqpoint{0.000000in}{0.000000in}}{%
\pgfpathmoveto{\pgfqpoint{0.000000in}{0.000000in}}%
\pgfpathlineto{\pgfqpoint{-0.048611in}{0.000000in}}%
\pgfusepath{stroke,fill}%
}%
\begin{pgfscope}%
\pgfsys@transformshift{0.948750in}{2.311650in}%
\pgfsys@useobject{currentmarker}{}%
\end{pgfscope}%
\end{pgfscope}%
\begin{pgfscope}%
\pgftext[x=0.453918in,y=2.258888in,left,base]{\sffamily\fontsize{10.000000}{12.000000}\selectfont 0.000}%
\end{pgfscope}%
\begin{pgfscope}%
\pgfsetbuttcap%
\pgfsetroundjoin%
\definecolor{currentfill}{rgb}{0.000000,0.000000,0.000000}%
\pgfsetfillcolor{currentfill}%
\pgfsetlinewidth{0.803000pt}%
\definecolor{currentstroke}{rgb}{0.000000,0.000000,0.000000}%
\pgfsetstrokecolor{currentstroke}%
\pgfsetdash{}{0pt}%
\pgfsys@defobject{currentmarker}{\pgfqpoint{-0.048611in}{0.000000in}}{\pgfqpoint{0.000000in}{0.000000in}}{%
\pgfpathmoveto{\pgfqpoint{0.000000in}{0.000000in}}%
\pgfpathlineto{\pgfqpoint{-0.048611in}{0.000000in}}%
\pgfusepath{stroke,fill}%
}%
\begin{pgfscope}%
\pgfsys@transformshift{0.948750in}{2.910967in}%
\pgfsys@useobject{currentmarker}{}%
\end{pgfscope}%
\end{pgfscope}%
\begin{pgfscope}%
\pgftext[x=0.453918in,y=2.858205in,left,base]{\sffamily\fontsize{10.000000}{12.000000}\selectfont 0.001}%
\end{pgfscope}%
\begin{pgfscope}%
\pgfsetbuttcap%
\pgfsetroundjoin%
\definecolor{currentfill}{rgb}{0.000000,0.000000,0.000000}%
\pgfsetfillcolor{currentfill}%
\pgfsetlinewidth{0.803000pt}%
\definecolor{currentstroke}{rgb}{0.000000,0.000000,0.000000}%
\pgfsetstrokecolor{currentstroke}%
\pgfsetdash{}{0pt}%
\pgfsys@defobject{currentmarker}{\pgfqpoint{-0.048611in}{0.000000in}}{\pgfqpoint{0.000000in}{0.000000in}}{%
\pgfpathmoveto{\pgfqpoint{0.000000in}{0.000000in}}%
\pgfpathlineto{\pgfqpoint{-0.048611in}{0.000000in}}%
\pgfusepath{stroke,fill}%
}%
\begin{pgfscope}%
\pgfsys@transformshift{0.948750in}{3.510283in}%
\pgfsys@useobject{currentmarker}{}%
\end{pgfscope}%
\end{pgfscope}%
\begin{pgfscope}%
\pgftext[x=0.453918in,y=3.457522in,left,base]{\sffamily\fontsize{10.000000}{12.000000}\selectfont 0.002}%
\end{pgfscope}%
\begin{pgfscope}%
\pgfsetbuttcap%
\pgfsetroundjoin%
\definecolor{currentfill}{rgb}{0.000000,0.000000,0.000000}%
\pgfsetfillcolor{currentfill}%
\pgfsetlinewidth{0.803000pt}%
\definecolor{currentstroke}{rgb}{0.000000,0.000000,0.000000}%
\pgfsetstrokecolor{currentstroke}%
\pgfsetdash{}{0pt}%
\pgfsys@defobject{currentmarker}{\pgfqpoint{-0.048611in}{0.000000in}}{\pgfqpoint{0.000000in}{0.000000in}}{%
\pgfpathmoveto{\pgfqpoint{0.000000in}{0.000000in}}%
\pgfpathlineto{\pgfqpoint{-0.048611in}{0.000000in}}%
\pgfusepath{stroke,fill}%
}%
\begin{pgfscope}%
\pgfsys@transformshift{0.948750in}{4.109600in}%
\pgfsys@useobject{currentmarker}{}%
\end{pgfscope}%
\end{pgfscope}%
\begin{pgfscope}%
\pgftext[x=0.453918in,y=4.056838in,left,base]{\sffamily\fontsize{10.000000}{12.000000}\selectfont 0.003}%
\end{pgfscope}%
\begin{pgfscope}%
\pgftext[x=0.281988in,y=2.311650in,,bottom,rotate=90.000000]{\sffamily\fontsize{16.000000}{19.200000}\selectfont \(\displaystyle Poynting\) \(\displaystyle vector\)}%
\end{pgfscope}%
\begin{pgfscope}%
\pgfpathrectangle{\pgfqpoint{0.948750in}{0.513700in}}{\pgfqpoint{5.882250in}{3.595900in}} %
\pgfusepath{clip}%
\pgfsetrectcap%
\pgfsetroundjoin%
\pgfsetlinewidth{1.505625pt}%
\definecolor{currentstroke}{rgb}{0.000000,0.000000,0.000000}%
\pgfsetstrokecolor{currentstroke}%
\pgfsetdash{}{0pt}%
\pgfpathmoveto{\pgfqpoint{0.948750in}{2.311650in}}%
\pgfpathlineto{\pgfqpoint{1.327567in}{2.310580in}}%
\pgfpathlineto{\pgfqpoint{1.332763in}{2.309131in}}%
\pgfpathlineto{\pgfqpoint{1.337175in}{2.311353in}}%
\pgfpathlineto{\pgfqpoint{1.339920in}{2.311292in}}%
\pgfpathlineto{\pgfqpoint{1.342959in}{2.308402in}}%
\pgfpathlineto{\pgfqpoint{1.346586in}{2.305834in}}%
\pgfpathlineto{\pgfqpoint{1.348645in}{2.307097in}}%
\pgfpathlineto{\pgfqpoint{1.354037in}{2.311403in}}%
\pgfpathlineto{\pgfqpoint{1.355900in}{2.308946in}}%
\pgfpathlineto{\pgfqpoint{1.361586in}{2.300160in}}%
\pgfpathlineto{\pgfqpoint{1.363253in}{2.302476in}}%
\pgfpathlineto{\pgfqpoint{1.368252in}{2.311579in}}%
\pgfpathlineto{\pgfqpoint{1.368547in}{2.311380in}}%
\pgfpathlineto{\pgfqpoint{1.370017in}{2.308487in}}%
\pgfpathlineto{\pgfqpoint{1.375899in}{2.290395in}}%
\pgfpathlineto{\pgfqpoint{1.376782in}{2.291484in}}%
\pgfpathlineto{\pgfqpoint{1.378644in}{2.298011in}}%
\pgfpathlineto{\pgfqpoint{1.382664in}{2.311649in}}%
\pgfpathlineto{\pgfqpoint{1.383644in}{2.310351in}}%
\pgfpathlineto{\pgfqpoint{1.385311in}{2.302752in}}%
\pgfpathlineto{\pgfqpoint{1.390409in}{2.275556in}}%
\pgfpathlineto{\pgfqpoint{1.390703in}{2.275748in}}%
\pgfpathlineto{\pgfqpoint{1.391781in}{2.278860in}}%
\pgfpathlineto{\pgfqpoint{1.394134in}{2.295143in}}%
\pgfpathlineto{\pgfqpoint{1.397272in}{2.311639in}}%
\pgfpathlineto{\pgfqpoint{1.397958in}{2.310857in}}%
\pgfpathlineto{\pgfqpoint{1.399232in}{2.304203in}}%
\pgfpathlineto{\pgfqpoint{1.402271in}{2.271607in}}%
\pgfpathlineto{\pgfqpoint{1.404918in}{2.254375in}}%
\pgfpathlineto{\pgfqpoint{1.405605in}{2.255259in}}%
\pgfpathlineto{\pgfqpoint{1.406879in}{2.263163in}}%
\pgfpathlineto{\pgfqpoint{1.412075in}{2.311647in}}%
\pgfpathlineto{\pgfqpoint{1.412958in}{2.308953in}}%
\pgfpathlineto{\pgfqpoint{1.414526in}{2.292429in}}%
\pgfpathlineto{\pgfqpoint{1.419722in}{2.225723in}}%
\pgfpathlineto{\pgfqpoint{1.420016in}{2.226163in}}%
\pgfpathlineto{\pgfqpoint{1.420997in}{2.232452in}}%
\pgfpathlineto{\pgfqpoint{1.423153in}{2.266158in}}%
\pgfpathlineto{\pgfqpoint{1.426585in}{2.311524in}}%
\pgfpathlineto{\pgfqpoint{1.426879in}{2.311570in}}%
\pgfpathlineto{\pgfqpoint{1.427075in}{2.311147in}}%
\pgfpathlineto{\pgfqpoint{1.427075in}{2.311147in}}%
\pgfpathlineto{\pgfqpoint{1.428055in}{2.303581in}}%
\pgfpathlineto{\pgfqpoint{1.430016in}{2.265651in}}%
\pgfpathlineto{\pgfqpoint{1.434232in}{2.189017in}}%
\pgfpathlineto{\pgfqpoint{1.434428in}{2.188976in}}%
\pgfpathlineto{\pgfqpoint{1.434624in}{2.189369in}}%
\pgfpathlineto{\pgfqpoint{1.434624in}{2.189369in}}%
\pgfpathlineto{\pgfqpoint{1.435506in}{2.196403in}}%
\pgfpathlineto{\pgfqpoint{1.437369in}{2.234441in}}%
\pgfpathlineto{\pgfqpoint{1.441388in}{2.311625in}}%
\pgfpathlineto{\pgfqpoint{1.441781in}{2.310929in}}%
\pgfpathlineto{\pgfqpoint{1.442761in}{2.300341in}}%
\pgfpathlineto{\pgfqpoint{1.444820in}{2.244468in}}%
\pgfpathlineto{\pgfqpoint{1.448839in}{2.143938in}}%
\pgfpathlineto{\pgfqpoint{1.449035in}{2.143688in}}%
\pgfpathlineto{\pgfqpoint{1.449231in}{2.144030in}}%
\pgfpathlineto{\pgfqpoint{1.449231in}{2.144030in}}%
\pgfpathlineto{\pgfqpoint{1.450016in}{2.151252in}}%
\pgfpathlineto{\pgfqpoint{1.451682in}{2.193392in}}%
\pgfpathlineto{\pgfqpoint{1.456192in}{2.311635in}}%
\pgfpathlineto{\pgfqpoint{1.456290in}{2.311480in}}%
\pgfpathlineto{\pgfqpoint{1.456976in}{2.305609in}}%
\pgfpathlineto{\pgfqpoint{1.458447in}{2.266847in}}%
\pgfpathlineto{\pgfqpoint{1.463741in}{2.090020in}}%
\pgfpathlineto{\pgfqpoint{1.464133in}{2.091938in}}%
\pgfpathlineto{\pgfqpoint{1.465212in}{2.112711in}}%
\pgfpathlineto{\pgfqpoint{1.467761in}{2.221877in}}%
\pgfpathlineto{\pgfqpoint{1.470702in}{2.311390in}}%
\pgfpathlineto{\pgfqpoint{1.470898in}{2.311627in}}%
\pgfpathlineto{\pgfqpoint{1.470996in}{2.311414in}}%
\pgfpathlineto{\pgfqpoint{1.470996in}{2.311414in}}%
\pgfpathlineto{\pgfqpoint{1.471682in}{2.303731in}}%
\pgfpathlineto{\pgfqpoint{1.473153in}{2.253529in}}%
\pgfpathlineto{\pgfqpoint{1.478349in}{2.028282in}}%
\pgfpathlineto{\pgfqpoint{1.478839in}{2.031060in}}%
\pgfpathlineto{\pgfqpoint{1.479917in}{2.058259in}}%
\pgfpathlineto{\pgfqpoint{1.482466in}{2.198084in}}%
\pgfpathlineto{\pgfqpoint{1.485407in}{2.311342in}}%
\pgfpathlineto{\pgfqpoint{1.485603in}{2.311614in}}%
\pgfpathlineto{\pgfqpoint{1.485701in}{2.311333in}}%
\pgfpathlineto{\pgfqpoint{1.485701in}{2.311333in}}%
\pgfpathlineto{\pgfqpoint{1.486388in}{2.301566in}}%
\pgfpathlineto{\pgfqpoint{1.487858in}{2.238396in}}%
\pgfpathlineto{\pgfqpoint{1.493054in}{1.959190in}}%
\pgfpathlineto{\pgfqpoint{1.493446in}{1.961637in}}%
\pgfpathlineto{\pgfqpoint{1.494427in}{1.988794in}}%
\pgfpathlineto{\pgfqpoint{1.496682in}{2.134329in}}%
\pgfpathlineto{\pgfqpoint{1.500015in}{2.310638in}}%
\pgfpathlineto{\pgfqpoint{1.500211in}{2.311617in}}%
\pgfpathlineto{\pgfqpoint{1.500407in}{2.311233in}}%
\pgfpathlineto{\pgfqpoint{1.500407in}{2.311233in}}%
\pgfpathlineto{\pgfqpoint{1.501093in}{2.299129in}}%
\pgfpathlineto{\pgfqpoint{1.502564in}{2.221611in}}%
\pgfpathlineto{\pgfqpoint{1.507760in}{1.883665in}}%
\pgfpathlineto{\pgfqpoint{1.508152in}{1.886967in}}%
\pgfpathlineto{\pgfqpoint{1.509230in}{1.925918in}}%
\pgfpathlineto{\pgfqpoint{1.511681in}{2.125281in}}%
\pgfpathlineto{\pgfqpoint{1.514720in}{2.310484in}}%
\pgfpathlineto{\pgfqpoint{1.514917in}{2.311619in}}%
\pgfpathlineto{\pgfqpoint{1.515113in}{2.311116in}}%
\pgfpathlineto{\pgfqpoint{1.515113in}{2.311116in}}%
\pgfpathlineto{\pgfqpoint{1.515799in}{2.296443in}}%
\pgfpathlineto{\pgfqpoint{1.517269in}{2.203383in}}%
\pgfpathlineto{\pgfqpoint{1.522465in}{1.802774in}}%
\pgfpathlineto{\pgfqpoint{1.522858in}{1.807058in}}%
\pgfpathlineto{\pgfqpoint{1.523936in}{1.854240in}}%
\pgfpathlineto{\pgfqpoint{1.526485in}{2.102469in}}%
\pgfpathlineto{\pgfqpoint{1.529426in}{2.310336in}}%
\pgfpathlineto{\pgfqpoint{1.529622in}{2.311623in}}%
\pgfpathlineto{\pgfqpoint{1.529818in}{2.310979in}}%
\pgfpathlineto{\pgfqpoint{1.529818in}{2.310979in}}%
\pgfpathlineto{\pgfqpoint{1.530505in}{2.293537in}}%
\pgfpathlineto{\pgfqpoint{1.532073in}{2.173786in}}%
\pgfpathlineto{\pgfqpoint{1.537073in}{1.717637in}}%
\pgfpathlineto{\pgfqpoint{1.537367in}{1.719331in}}%
\pgfpathlineto{\pgfqpoint{1.538151in}{1.746448in}}%
\pgfpathlineto{\pgfqpoint{1.539916in}{1.908130in}}%
\pgfpathlineto{\pgfqpoint{1.544328in}{2.311628in}}%
\pgfpathlineto{\pgfqpoint{1.544622in}{2.309579in}}%
\pgfpathlineto{\pgfqpoint{1.545504in}{2.273578in}}%
\pgfpathlineto{\pgfqpoint{1.547367in}{2.075553in}}%
\pgfpathlineto{\pgfqpoint{1.551681in}{1.629888in}}%
\pgfpathlineto{\pgfqpoint{1.551779in}{1.629439in}}%
\pgfpathlineto{\pgfqpoint{1.551877in}{1.629590in}}%
\pgfpathlineto{\pgfqpoint{1.551877in}{1.629590in}}%
\pgfpathlineto{\pgfqpoint{1.552367in}{1.639306in}}%
\pgfpathlineto{\pgfqpoint{1.553543in}{1.720062in}}%
\pgfpathlineto{\pgfqpoint{1.559033in}{2.311633in}}%
\pgfpathlineto{\pgfqpoint{1.560014in}{2.281366in}}%
\pgfpathlineto{\pgfqpoint{1.561680in}{2.101431in}}%
\pgfpathlineto{\pgfqpoint{1.566484in}{1.539439in}}%
\pgfpathlineto{\pgfqpoint{1.566582in}{1.539716in}}%
\pgfpathlineto{\pgfqpoint{1.567171in}{1.555546in}}%
\pgfpathlineto{\pgfqpoint{1.568543in}{1.679226in}}%
\pgfpathlineto{\pgfqpoint{1.573739in}{2.311639in}}%
\pgfpathlineto{\pgfqpoint{1.574327in}{2.299486in}}%
\pgfpathlineto{\pgfqpoint{1.575504in}{2.201433in}}%
\pgfpathlineto{\pgfqpoint{1.578641in}{1.678530in}}%
\pgfpathlineto{\pgfqpoint{1.581092in}{1.449157in}}%
\pgfpathlineto{\pgfqpoint{1.581190in}{1.448819in}}%
\pgfpathlineto{\pgfqpoint{1.581190in}{1.448819in}}%
\pgfpathlineto{\pgfqpoint{1.581190in}{1.448819in}}%
\pgfpathlineto{\pgfqpoint{1.581582in}{1.455039in}}%
\pgfpathlineto{\pgfqpoint{1.582562in}{1.521784in}}%
\pgfpathlineto{\pgfqpoint{1.584817in}{1.875691in}}%
\pgfpathlineto{\pgfqpoint{1.588249in}{2.309858in}}%
\pgfpathlineto{\pgfqpoint{1.588445in}{2.311644in}}%
\pgfpathlineto{\pgfqpoint{1.588641in}{2.310244in}}%
\pgfpathlineto{\pgfqpoint{1.588641in}{2.310244in}}%
\pgfpathlineto{\pgfqpoint{1.589327in}{2.280424in}}%
\pgfpathlineto{\pgfqpoint{1.590896in}{2.082867in}}%
\pgfpathlineto{\pgfqpoint{1.595896in}{1.358709in}}%
\pgfpathlineto{\pgfqpoint{1.596092in}{1.360700in}}%
\pgfpathlineto{\pgfqpoint{1.596876in}{1.401592in}}%
\pgfpathlineto{\pgfqpoint{1.598641in}{1.655959in}}%
\pgfpathlineto{\pgfqpoint{1.603150in}{2.311648in}}%
\pgfpathlineto{\pgfqpoint{1.603444in}{2.307893in}}%
\pgfpathlineto{\pgfqpoint{1.604327in}{2.250212in}}%
\pgfpathlineto{\pgfqpoint{1.606288in}{1.920852in}}%
\pgfpathlineto{\pgfqpoint{1.610405in}{1.271413in}}%
\pgfpathlineto{\pgfqpoint{1.610601in}{1.270162in}}%
\pgfpathlineto{\pgfqpoint{1.610699in}{1.270909in}}%
\pgfpathlineto{\pgfqpoint{1.610699in}{1.270909in}}%
\pgfpathlineto{\pgfqpoint{1.611287in}{1.294445in}}%
\pgfpathlineto{\pgfqpoint{1.612660in}{1.465243in}}%
\pgfpathlineto{\pgfqpoint{1.617856in}{2.311650in}}%
\pgfpathlineto{\pgfqpoint{1.618444in}{2.294664in}}%
\pgfpathlineto{\pgfqpoint{1.619719in}{2.147363in}}%
\pgfpathlineto{\pgfqpoint{1.625307in}{1.184146in}}%
\pgfpathlineto{\pgfqpoint{1.626189in}{1.227624in}}%
\pgfpathlineto{\pgfqpoint{1.627856in}{1.497359in}}%
\pgfpathlineto{\pgfqpoint{1.632562in}{2.311650in}}%
\pgfpathlineto{\pgfqpoint{1.632660in}{2.311108in}}%
\pgfpathlineto{\pgfqpoint{1.633248in}{2.286406in}}%
\pgfpathlineto{\pgfqpoint{1.634620in}{2.096936in}}%
\pgfpathlineto{\pgfqpoint{1.639914in}{1.101481in}}%
\pgfpathlineto{\pgfqpoint{1.640601in}{1.124009in}}%
\pgfpathlineto{\pgfqpoint{1.641875in}{1.293805in}}%
\pgfpathlineto{\pgfqpoint{1.647267in}{2.311647in}}%
\pgfpathlineto{\pgfqpoint{1.648051in}{2.276111in}}%
\pgfpathlineto{\pgfqpoint{1.649522in}{2.039021in}}%
\pgfpathlineto{\pgfqpoint{1.654620in}{1.022901in}}%
\pgfpathlineto{\pgfqpoint{1.655012in}{1.030343in}}%
\pgfpathlineto{\pgfqpoint{1.655993in}{1.125553in}}%
\pgfpathlineto{\pgfqpoint{1.658149in}{1.620054in}}%
\pgfpathlineto{\pgfqpoint{1.661679in}{2.306823in}}%
\pgfpathlineto{\pgfqpoint{1.661973in}{2.311642in}}%
\pgfpathlineto{\pgfqpoint{1.662169in}{2.309041in}}%
\pgfpathlineto{\pgfqpoint{1.662169in}{2.309041in}}%
\pgfpathlineto{\pgfqpoint{1.662953in}{2.252738in}}%
\pgfpathlineto{\pgfqpoint{1.664620in}{1.923900in}}%
\pgfpathlineto{\pgfqpoint{1.669326in}{0.949122in}}%
\pgfpathlineto{\pgfqpoint{1.669424in}{0.949410in}}%
\pgfpathlineto{\pgfqpoint{1.669914in}{0.968719in}}%
\pgfpathlineto{\pgfqpoint{1.671090in}{1.129221in}}%
\pgfpathlineto{\pgfqpoint{1.676678in}{2.311635in}}%
\pgfpathlineto{\pgfqpoint{1.677659in}{2.249110in}}%
\pgfpathlineto{\pgfqpoint{1.679325in}{1.902490in}}%
\pgfpathlineto{\pgfqpoint{1.684031in}{0.880679in}}%
\pgfpathlineto{\pgfqpoint{1.684325in}{0.885686in}}%
\pgfpathlineto{\pgfqpoint{1.685208in}{0.966889in}}%
\pgfpathlineto{\pgfqpoint{1.687168in}{1.429727in}}%
\pgfpathlineto{\pgfqpoint{1.691188in}{2.309563in}}%
\pgfpathlineto{\pgfqpoint{1.691384in}{2.311625in}}%
\pgfpathlineto{\pgfqpoint{1.691482in}{2.310731in}}%
\pgfpathlineto{\pgfqpoint{1.691482in}{2.310731in}}%
\pgfpathlineto{\pgfqpoint{1.692070in}{2.278585in}}%
\pgfpathlineto{\pgfqpoint{1.693443in}{2.040033in}}%
\pgfpathlineto{\pgfqpoint{1.698737in}{0.817996in}}%
\pgfpathlineto{\pgfqpoint{1.699325in}{0.840792in}}%
\pgfpathlineto{\pgfqpoint{1.700600in}{1.040814in}}%
\pgfpathlineto{\pgfqpoint{1.706090in}{2.311614in}}%
\pgfpathlineto{\pgfqpoint{1.706874in}{2.266735in}}%
\pgfpathlineto{\pgfqpoint{1.708345in}{1.976711in}}%
\pgfpathlineto{\pgfqpoint{1.713443in}{0.761382in}}%
\pgfpathlineto{\pgfqpoint{1.713737in}{0.767438in}}%
\pgfpathlineto{\pgfqpoint{1.714619in}{0.857176in}}%
\pgfpathlineto{\pgfqpoint{1.716580in}{1.360652in}}%
\pgfpathlineto{\pgfqpoint{1.720599in}{2.309563in}}%
\pgfpathlineto{\pgfqpoint{1.720795in}{2.311602in}}%
\pgfpathlineto{\pgfqpoint{1.720893in}{2.310548in}}%
\pgfpathlineto{\pgfqpoint{1.720893in}{2.310548in}}%
\pgfpathlineto{\pgfqpoint{1.721482in}{2.275396in}}%
\pgfpathlineto{\pgfqpoint{1.722854in}{2.017785in}}%
\pgfpathlineto{\pgfqpoint{1.728148in}{0.711035in}}%
\pgfpathlineto{\pgfqpoint{1.728736in}{0.736642in}}%
\pgfpathlineto{\pgfqpoint{1.730011in}{0.953085in}}%
\pgfpathlineto{\pgfqpoint{1.735501in}{2.311590in}}%
\pgfpathlineto{\pgfqpoint{1.736285in}{2.263115in}}%
\pgfpathlineto{\pgfqpoint{1.737756in}{1.953483in}}%
\pgfpathlineto{\pgfqpoint{1.742854in}{0.667051in}}%
\pgfpathlineto{\pgfqpoint{1.743148in}{0.674030in}}%
\pgfpathlineto{\pgfqpoint{1.744030in}{0.770772in}}%
\pgfpathlineto{\pgfqpoint{1.745991in}{1.306648in}}%
\pgfpathlineto{\pgfqpoint{1.750010in}{2.309574in}}%
\pgfpathlineto{\pgfqpoint{1.750207in}{2.311578in}}%
\pgfpathlineto{\pgfqpoint{1.750305in}{2.310390in}}%
\pgfpathlineto{\pgfqpoint{1.750305in}{2.310390in}}%
\pgfpathlineto{\pgfqpoint{1.750893in}{2.272856in}}%
\pgfpathlineto{\pgfqpoint{1.752265in}{2.000421in}}%
\pgfpathlineto{\pgfqpoint{1.757559in}{0.629428in}}%
\pgfpathlineto{\pgfqpoint{1.758050in}{0.649053in}}%
\pgfpathlineto{\pgfqpoint{1.759226in}{0.837938in}}%
\pgfpathlineto{\pgfqpoint{1.762363in}{1.867600in}}%
\pgfpathlineto{\pgfqpoint{1.764814in}{2.311318in}}%
\pgfpathlineto{\pgfqpoint{1.764912in}{2.311567in}}%
\pgfpathlineto{\pgfqpoint{1.764912in}{2.311567in}}%
\pgfpathlineto{\pgfqpoint{1.764912in}{2.311567in}}%
\pgfpathlineto{\pgfqpoint{1.765304in}{2.297688in}}%
\pgfpathlineto{\pgfqpoint{1.766383in}{2.142056in}}%
\pgfpathlineto{\pgfqpoint{1.768834in}{1.357002in}}%
\pgfpathlineto{\pgfqpoint{1.771971in}{0.603866in}}%
\pgfpathlineto{\pgfqpoint{1.772265in}{0.598079in}}%
\pgfpathlineto{\pgfqpoint{1.772461in}{0.601733in}}%
\pgfpathlineto{\pgfqpoint{1.772461in}{0.601733in}}%
\pgfpathlineto{\pgfqpoint{1.773245in}{0.675359in}}%
\pgfpathlineto{\pgfqpoint{1.775010in}{1.130648in}}%
\pgfpathlineto{\pgfqpoint{1.779520in}{2.311328in}}%
\pgfpathlineto{\pgfqpoint{1.779618in}{2.311558in}}%
\pgfpathlineto{\pgfqpoint{1.779618in}{2.311558in}}%
\pgfpathlineto{\pgfqpoint{1.779618in}{2.311558in}}%
\pgfpathlineto{\pgfqpoint{1.780010in}{2.297358in}}%
\pgfpathlineto{\pgfqpoint{1.781088in}{2.138978in}}%
\pgfpathlineto{\pgfqpoint{1.783637in}{1.305548in}}%
\pgfpathlineto{\pgfqpoint{1.786677in}{0.578535in}}%
\pgfpathlineto{\pgfqpoint{1.786971in}{0.572851in}}%
\pgfpathlineto{\pgfqpoint{1.787167in}{0.576685in}}%
\pgfpathlineto{\pgfqpoint{1.787167in}{0.576685in}}%
\pgfpathlineto{\pgfqpoint{1.787951in}{0.651868in}}%
\pgfpathlineto{\pgfqpoint{1.789716in}{1.114495in}}%
\pgfpathlineto{\pgfqpoint{1.794225in}{2.311335in}}%
\pgfpathlineto{\pgfqpoint{1.794323in}{2.311551in}}%
\pgfpathlineto{\pgfqpoint{1.794323in}{2.311551in}}%
\pgfpathlineto{\pgfqpoint{1.794323in}{2.311551in}}%
\pgfpathlineto{\pgfqpoint{1.794716in}{2.297105in}}%
\pgfpathlineto{\pgfqpoint{1.795794in}{2.136589in}}%
\pgfpathlineto{\pgfqpoint{1.798343in}{1.293237in}}%
\pgfpathlineto{\pgfqpoint{1.801382in}{0.559158in}}%
\pgfpathlineto{\pgfqpoint{1.801676in}{0.553573in}}%
\pgfpathlineto{\pgfqpoint{1.801872in}{0.557557in}}%
\pgfpathlineto{\pgfqpoint{1.801872in}{0.557557in}}%
\pgfpathlineto{\pgfqpoint{1.802657in}{0.633976in}}%
\pgfpathlineto{\pgfqpoint{1.804421in}{1.102260in}}%
\pgfpathlineto{\pgfqpoint{1.808931in}{2.311338in}}%
\pgfpathlineto{\pgfqpoint{1.809029in}{2.311547in}}%
\pgfpathlineto{\pgfqpoint{1.809029in}{2.311547in}}%
\pgfpathlineto{\pgfqpoint{1.809029in}{2.311547in}}%
\pgfpathlineto{\pgfqpoint{1.809421in}{2.296930in}}%
\pgfpathlineto{\pgfqpoint{1.810500in}{2.134889in}}%
\pgfpathlineto{\pgfqpoint{1.813049in}{1.284492in}}%
\pgfpathlineto{\pgfqpoint{1.816088in}{0.545615in}}%
\pgfpathlineto{\pgfqpoint{1.816382in}{0.540122in}}%
\pgfpathlineto{\pgfqpoint{1.816578in}{0.544226in}}%
\pgfpathlineto{\pgfqpoint{1.816578in}{0.544226in}}%
\pgfpathlineto{\pgfqpoint{1.817362in}{0.621565in}}%
\pgfpathlineto{\pgfqpoint{1.819127in}{1.093852in}}%
\pgfpathlineto{\pgfqpoint{1.823637in}{2.311337in}}%
\pgfpathlineto{\pgfqpoint{1.823735in}{2.311546in}}%
\pgfpathlineto{\pgfqpoint{1.823735in}{2.311546in}}%
\pgfpathlineto{\pgfqpoint{1.823735in}{2.311546in}}%
\pgfpathlineto{\pgfqpoint{1.824127in}{2.296840in}}%
\pgfpathlineto{\pgfqpoint{1.825205in}{2.133890in}}%
\pgfpathlineto{\pgfqpoint{1.827754in}{1.279335in}}%
\pgfpathlineto{\pgfqpoint{1.830793in}{0.537934in}}%
\pgfpathlineto{\pgfqpoint{1.831087in}{0.532527in}}%
\pgfpathlineto{\pgfqpoint{1.831284in}{0.536720in}}%
\pgfpathlineto{\pgfqpoint{1.831284in}{0.536720in}}%
\pgfpathlineto{\pgfqpoint{1.832068in}{0.614659in}}%
\pgfpathlineto{\pgfqpoint{1.833833in}{1.089283in}}%
\pgfpathlineto{\pgfqpoint{1.838342in}{2.311330in}}%
\pgfpathlineto{\pgfqpoint{1.838440in}{2.311549in}}%
\pgfpathlineto{\pgfqpoint{1.838440in}{2.311549in}}%
\pgfpathlineto{\pgfqpoint{1.838440in}{2.311549in}}%
\pgfpathlineto{\pgfqpoint{1.838832in}{2.296842in}}%
\pgfpathlineto{\pgfqpoint{1.839911in}{2.133632in}}%
\pgfpathlineto{\pgfqpoint{1.842460in}{1.277929in}}%
\pgfpathlineto{\pgfqpoint{1.845499in}{0.536376in}}%
\pgfpathlineto{\pgfqpoint{1.845793in}{0.531049in}}%
\pgfpathlineto{\pgfqpoint{1.845989in}{0.535301in}}%
\pgfpathlineto{\pgfqpoint{1.845989in}{0.535301in}}%
\pgfpathlineto{\pgfqpoint{1.846774in}{0.613506in}}%
\pgfpathlineto{\pgfqpoint{1.848538in}{1.088722in}}%
\pgfpathlineto{\pgfqpoint{1.853048in}{2.311317in}}%
\pgfpathlineto{\pgfqpoint{1.853146in}{2.311556in}}%
\pgfpathlineto{\pgfqpoint{1.853146in}{2.311556in}}%
\pgfpathlineto{\pgfqpoint{1.853146in}{2.311556in}}%
\pgfpathlineto{\pgfqpoint{1.853538in}{2.296949in}}%
\pgfpathlineto{\pgfqpoint{1.854617in}{2.134192in}}%
\pgfpathlineto{\pgfqpoint{1.857165in}{1.280598in}}%
\pgfpathlineto{\pgfqpoint{1.860205in}{0.541454in}}%
\pgfpathlineto{\pgfqpoint{1.860499in}{0.536198in}}%
\pgfpathlineto{\pgfqpoint{1.860695in}{0.540475in}}%
\pgfpathlineto{\pgfqpoint{1.860695in}{0.540475in}}%
\pgfpathlineto{\pgfqpoint{1.861479in}{0.618581in}}%
\pgfpathlineto{\pgfqpoint{1.863244in}{1.092486in}}%
\pgfpathlineto{\pgfqpoint{1.867754in}{2.311295in}}%
\pgfpathlineto{\pgfqpoint{1.867852in}{2.311568in}}%
\pgfpathlineto{\pgfqpoint{1.867852in}{2.311568in}}%
\pgfpathlineto{\pgfqpoint{1.867852in}{2.311568in}}%
\pgfpathlineto{\pgfqpoint{1.868244in}{2.297178in}}%
\pgfpathlineto{\pgfqpoint{1.869322in}{2.135668in}}%
\pgfpathlineto{\pgfqpoint{1.871871in}{1.287768in}}%
\pgfpathlineto{\pgfqpoint{1.874910in}{0.553809in}}%
\pgfpathlineto{\pgfqpoint{1.875204in}{0.548610in}}%
\pgfpathlineto{\pgfqpoint{1.875400in}{0.552869in}}%
\pgfpathlineto{\pgfqpoint{1.875400in}{0.552869in}}%
\pgfpathlineto{\pgfqpoint{1.876185in}{0.630463in}}%
\pgfpathlineto{\pgfqpoint{1.877949in}{1.100951in}}%
\pgfpathlineto{\pgfqpoint{1.882459in}{2.311262in}}%
\pgfpathlineto{\pgfqpoint{1.882557in}{2.311586in}}%
\pgfpathlineto{\pgfqpoint{1.882557in}{2.311586in}}%
\pgfpathlineto{\pgfqpoint{1.882557in}{2.311586in}}%
\pgfpathlineto{\pgfqpoint{1.882949in}{2.297544in}}%
\pgfpathlineto{\pgfqpoint{1.884028in}{2.138154in}}%
\pgfpathlineto{\pgfqpoint{1.886577in}{1.299821in}}%
\pgfpathlineto{\pgfqpoint{1.889616in}{0.573952in}}%
\pgfpathlineto{\pgfqpoint{1.889910in}{0.568786in}}%
\pgfpathlineto{\pgfqpoint{1.890106in}{0.572978in}}%
\pgfpathlineto{\pgfqpoint{1.890106in}{0.572978in}}%
\pgfpathlineto{\pgfqpoint{1.890890in}{0.649591in}}%
\pgfpathlineto{\pgfqpoint{1.892655in}{1.114379in}}%
\pgfpathlineto{\pgfqpoint{1.897165in}{2.311213in}}%
\pgfpathlineto{\pgfqpoint{1.897263in}{2.311606in}}%
\pgfpathlineto{\pgfqpoint{1.897263in}{2.311606in}}%
\pgfpathlineto{\pgfqpoint{1.897263in}{2.311606in}}%
\pgfpathlineto{\pgfqpoint{1.897655in}{2.298052in}}%
\pgfpathlineto{\pgfqpoint{1.898733in}{2.141692in}}%
\pgfpathlineto{\pgfqpoint{1.901282in}{1.316879in}}%
\pgfpathlineto{\pgfqpoint{1.904322in}{0.601945in}}%
\pgfpathlineto{\pgfqpoint{1.904616in}{0.596773in}}%
\pgfpathlineto{\pgfqpoint{1.904812in}{0.600837in}}%
\pgfpathlineto{\pgfqpoint{1.904812in}{0.600837in}}%
\pgfpathlineto{\pgfqpoint{1.905596in}{0.675968in}}%
\pgfpathlineto{\pgfqpoint{1.907361in}{1.132719in}}%
\pgfpathlineto{\pgfqpoint{1.911870in}{2.311148in}}%
\pgfpathlineto{\pgfqpoint{1.911968in}{2.311626in}}%
\pgfpathlineto{\pgfqpoint{1.911968in}{2.311626in}}%
\pgfpathlineto{\pgfqpoint{1.911968in}{2.311626in}}%
\pgfpathlineto{\pgfqpoint{1.912361in}{2.298695in}}%
\pgfpathlineto{\pgfqpoint{1.913439in}{2.146231in}}%
\pgfpathlineto{\pgfqpoint{1.915890in}{1.373517in}}%
\pgfpathlineto{\pgfqpoint{1.919027in}{0.637122in}}%
\pgfpathlineto{\pgfqpoint{1.919321in}{0.631894in}}%
\pgfpathlineto{\pgfqpoint{1.919517in}{0.635765in}}%
\pgfpathlineto{\pgfqpoint{1.919517in}{0.635765in}}%
\pgfpathlineto{\pgfqpoint{1.920302in}{0.708916in}}%
\pgfpathlineto{\pgfqpoint{1.922066in}{1.155449in}}%
\pgfpathlineto{\pgfqpoint{1.926576in}{2.311064in}}%
\pgfpathlineto{\pgfqpoint{1.926674in}{2.311642in}}%
\pgfpathlineto{\pgfqpoint{1.926674in}{2.311642in}}%
\pgfpathlineto{\pgfqpoint{1.926674in}{2.311642in}}%
\pgfpathlineto{\pgfqpoint{1.927066in}{2.299446in}}%
\pgfpathlineto{\pgfqpoint{1.928145in}{2.151603in}}%
\pgfpathlineto{\pgfqpoint{1.930596in}{1.398261in}}%
\pgfpathlineto{\pgfqpoint{1.933733in}{0.677992in}}%
\pgfpathlineto{\pgfqpoint{1.934027in}{0.672650in}}%
\pgfpathlineto{\pgfqpoint{1.934223in}{0.676264in}}%
\pgfpathlineto{\pgfqpoint{1.934223in}{0.676264in}}%
\pgfpathlineto{\pgfqpoint{1.935007in}{0.746997in}}%
\pgfpathlineto{\pgfqpoint{1.936772in}{1.181544in}}%
\pgfpathlineto{\pgfqpoint{1.941282in}{2.310963in}}%
\pgfpathlineto{\pgfqpoint{1.941380in}{2.311650in}}%
\pgfpathlineto{\pgfqpoint{1.941380in}{2.311650in}}%
\pgfpathlineto{\pgfqpoint{1.941380in}{2.311650in}}%
\pgfpathlineto{\pgfqpoint{1.941674in}{2.305221in}}%
\pgfpathlineto{\pgfqpoint{1.942556in}{2.211726in}}%
\pgfpathlineto{\pgfqpoint{1.944517in}{1.691515in}}%
\pgfpathlineto{\pgfqpoint{1.948438in}{0.722344in}}%
\pgfpathlineto{\pgfqpoint{1.948733in}{0.716829in}}%
\pgfpathlineto{\pgfqpoint{1.948929in}{0.720133in}}%
\pgfpathlineto{\pgfqpoint{1.948929in}{0.720133in}}%
\pgfpathlineto{\pgfqpoint{1.949713in}{0.788129in}}%
\pgfpathlineto{\pgfqpoint{1.951478in}{1.209559in}}%
\pgfpathlineto{\pgfqpoint{1.955987in}{2.310847in}}%
\pgfpathlineto{\pgfqpoint{1.956085in}{2.311646in}}%
\pgfpathlineto{\pgfqpoint{1.956183in}{2.311066in}}%
\pgfpathlineto{\pgfqpoint{1.956183in}{2.311066in}}%
\pgfpathlineto{\pgfqpoint{1.956772in}{2.278935in}}%
\pgfpathlineto{\pgfqpoint{1.958144in}{2.030103in}}%
\pgfpathlineto{\pgfqpoint{1.963438in}{0.761795in}}%
\pgfpathlineto{\pgfqpoint{1.964026in}{0.786828in}}%
\pgfpathlineto{\pgfqpoint{1.965301in}{0.996071in}}%
\pgfpathlineto{\pgfqpoint{1.970791in}{2.311629in}}%
\pgfpathlineto{\pgfqpoint{1.971575in}{2.271096in}}%
\pgfpathlineto{\pgfqpoint{1.973046in}{1.990056in}}%
\pgfpathlineto{\pgfqpoint{1.978144in}{0.804845in}}%
\pgfpathlineto{\pgfqpoint{1.978438in}{0.810721in}}%
\pgfpathlineto{\pgfqpoint{1.979320in}{0.897633in}}%
\pgfpathlineto{\pgfqpoint{1.981281in}{1.384151in}}%
\pgfpathlineto{\pgfqpoint{1.985300in}{2.308303in}}%
\pgfpathlineto{\pgfqpoint{1.985497in}{2.311603in}}%
\pgfpathlineto{\pgfqpoint{1.985693in}{2.309695in}}%
\pgfpathlineto{\pgfqpoint{1.985693in}{2.309695in}}%
\pgfpathlineto{\pgfqpoint{1.986379in}{2.262597in}}%
\pgfpathlineto{\pgfqpoint{1.987948in}{1.949069in}}%
\pgfpathlineto{\pgfqpoint{1.992849in}{0.843547in}}%
\pgfpathlineto{\pgfqpoint{1.993045in}{0.845801in}}%
\pgfpathlineto{\pgfqpoint{1.993732in}{0.893610in}}%
\pgfpathlineto{\pgfqpoint{1.995300in}{1.206424in}}%
\pgfpathlineto{\pgfqpoint{2.000202in}{2.311573in}}%
\pgfpathlineto{\pgfqpoint{2.000398in}{2.309915in}}%
\pgfpathlineto{\pgfqpoint{2.001085in}{2.264624in}}%
\pgfpathlineto{\pgfqpoint{2.002653in}{1.959586in}}%
\pgfpathlineto{\pgfqpoint{2.007555in}{0.875976in}}%
\pgfpathlineto{\pgfqpoint{2.007751in}{0.877914in}}%
\pgfpathlineto{\pgfqpoint{2.008437in}{0.923775in}}%
\pgfpathlineto{\pgfqpoint{2.010006in}{1.228218in}}%
\pgfpathlineto{\pgfqpoint{2.014908in}{2.311545in}}%
\pgfpathlineto{\pgfqpoint{2.015104in}{2.310077in}}%
\pgfpathlineto{\pgfqpoint{2.015790in}{2.266198in}}%
\pgfpathlineto{\pgfqpoint{2.017359in}{1.967810in}}%
\pgfpathlineto{\pgfqpoint{2.022261in}{0.900829in}}%
\pgfpathlineto{\pgfqpoint{2.022457in}{0.902502in}}%
\pgfpathlineto{\pgfqpoint{2.023143in}{0.946795in}}%
\pgfpathlineto{\pgfqpoint{2.024712in}{1.244736in}}%
\pgfpathlineto{\pgfqpoint{2.029613in}{2.311524in}}%
\pgfpathlineto{\pgfqpoint{2.029810in}{2.310181in}}%
\pgfpathlineto{\pgfqpoint{2.030496in}{2.267271in}}%
\pgfpathlineto{\pgfqpoint{2.032064in}{1.973483in}}%
\pgfpathlineto{\pgfqpoint{2.036966in}{0.917447in}}%
\pgfpathlineto{\pgfqpoint{2.037162in}{0.918917in}}%
\pgfpathlineto{\pgfqpoint{2.037849in}{0.962079in}}%
\pgfpathlineto{\pgfqpoint{2.039417in}{1.255585in}}%
\pgfpathlineto{\pgfqpoint{2.044319in}{2.311514in}}%
\pgfpathlineto{\pgfqpoint{2.044515in}{2.310230in}}%
\pgfpathlineto{\pgfqpoint{2.045201in}{2.267831in}}%
\pgfpathlineto{\pgfqpoint{2.046770in}{1.976533in}}%
\pgfpathlineto{\pgfqpoint{2.051672in}{0.925745in}}%
\pgfpathlineto{\pgfqpoint{2.051868in}{0.927081in}}%
\pgfpathlineto{\pgfqpoint{2.052554in}{0.969577in}}%
\pgfpathlineto{\pgfqpoint{2.054123in}{1.260761in}}%
\pgfpathlineto{\pgfqpoint{2.059025in}{2.311517in}}%
\pgfpathlineto{\pgfqpoint{2.059221in}{2.310225in}}%
\pgfpathlineto{\pgfqpoint{2.059907in}{2.267892in}}%
\pgfpathlineto{\pgfqpoint{2.061476in}{1.977041in}}%
\pgfpathlineto{\pgfqpoint{2.066378in}{0.926118in}}%
\pgfpathlineto{\pgfqpoint{2.066574in}{0.927391in}}%
\pgfpathlineto{\pgfqpoint{2.067260in}{0.969681in}}%
\pgfpathlineto{\pgfqpoint{2.068828in}{1.260584in}}%
\pgfpathlineto{\pgfqpoint{2.073730in}{2.311531in}}%
\pgfpathlineto{\pgfqpoint{2.073926in}{2.310171in}}%
\pgfpathlineto{\pgfqpoint{2.074613in}{2.267490in}}%
\pgfpathlineto{\pgfqpoint{2.076181in}{1.975209in}}%
\pgfpathlineto{\pgfqpoint{2.081083in}{0.919331in}}%
\pgfpathlineto{\pgfqpoint{2.081279in}{0.920612in}}%
\pgfpathlineto{\pgfqpoint{2.081965in}{0.963129in}}%
\pgfpathlineto{\pgfqpoint{2.083534in}{1.255615in}}%
\pgfpathlineto{\pgfqpoint{2.088436in}{2.311554in}}%
\pgfpathlineto{\pgfqpoint{2.088632in}{2.310071in}}%
\pgfpathlineto{\pgfqpoint{2.089318in}{2.266679in}}%
\pgfpathlineto{\pgfqpoint{2.090887in}{1.971327in}}%
\pgfpathlineto{\pgfqpoint{2.095789in}{0.906410in}}%
\pgfpathlineto{\pgfqpoint{2.095985in}{0.907763in}}%
\pgfpathlineto{\pgfqpoint{2.096671in}{0.950895in}}%
\pgfpathlineto{\pgfqpoint{2.098240in}{1.246587in}}%
\pgfpathlineto{\pgfqpoint{2.103142in}{2.311580in}}%
\pgfpathlineto{\pgfqpoint{2.103338in}{2.309929in}}%
\pgfpathlineto{\pgfqpoint{2.104024in}{2.265521in}}%
\pgfpathlineto{\pgfqpoint{2.105593in}{1.965748in}}%
\pgfpathlineto{\pgfqpoint{2.110494in}{0.888547in}}%
\pgfpathlineto{\pgfqpoint{2.110690in}{0.890031in}}%
\pgfpathlineto{\pgfqpoint{2.111377in}{0.934106in}}%
\pgfpathlineto{\pgfqpoint{2.112945in}{1.234334in}}%
\pgfpathlineto{\pgfqpoint{2.117847in}{2.311607in}}%
\pgfpathlineto{\pgfqpoint{2.118043in}{2.309751in}}%
\pgfpathlineto{\pgfqpoint{2.118730in}{2.264090in}}%
\pgfpathlineto{\pgfqpoint{2.120298in}{1.958860in}}%
\pgfpathlineto{\pgfqpoint{2.125200in}{0.867017in}}%
\pgfpathlineto{\pgfqpoint{2.125396in}{0.868680in}}%
\pgfpathlineto{\pgfqpoint{2.126082in}{0.913962in}}%
\pgfpathlineto{\pgfqpoint{2.127651in}{1.219735in}}%
\pgfpathlineto{\pgfqpoint{2.132553in}{2.311628in}}%
\pgfpathlineto{\pgfqpoint{2.132749in}{2.309542in}}%
\pgfpathlineto{\pgfqpoint{2.133435in}{2.262461in}}%
\pgfpathlineto{\pgfqpoint{2.135004in}{1.951067in}}%
\pgfpathlineto{\pgfqpoint{2.139906in}{0.843112in}}%
\pgfpathlineto{\pgfqpoint{2.140102in}{0.844991in}}%
\pgfpathlineto{\pgfqpoint{2.140788in}{0.891668in}}%
\pgfpathlineto{\pgfqpoint{2.142357in}{1.203666in}}%
\pgfpathlineto{\pgfqpoint{2.147258in}{2.311643in}}%
\pgfpathlineto{\pgfqpoint{2.147455in}{2.309311in}}%
\pgfpathlineto{\pgfqpoint{2.148141in}{2.260716in}}%
\pgfpathlineto{\pgfqpoint{2.149709in}{1.942772in}}%
\pgfpathlineto{\pgfqpoint{2.154611in}{0.818076in}}%
\pgfpathlineto{\pgfqpoint{2.154807in}{0.820197in}}%
\pgfpathlineto{\pgfqpoint{2.155494in}{0.868387in}}%
\pgfpathlineto{\pgfqpoint{2.157062in}{1.186967in}}%
\pgfpathlineto{\pgfqpoint{2.161964in}{2.311650in}}%
\pgfpathlineto{\pgfqpoint{2.162160in}{2.309067in}}%
\pgfpathlineto{\pgfqpoint{2.162944in}{2.246706in}}%
\pgfpathlineto{\pgfqpoint{2.164611in}{1.878199in}}%
\pgfpathlineto{\pgfqpoint{2.169317in}{0.793066in}}%
\pgfpathlineto{\pgfqpoint{2.169611in}{0.798628in}}%
\pgfpathlineto{\pgfqpoint{2.170493in}{0.885462in}}%
\pgfpathlineto{\pgfqpoint{2.172454in}{1.376723in}}%
\pgfpathlineto{\pgfqpoint{2.176474in}{2.309102in}}%
\pgfpathlineto{\pgfqpoint{2.176670in}{2.311648in}}%
\pgfpathlineto{\pgfqpoint{2.176768in}{2.310906in}}%
\pgfpathlineto{\pgfqpoint{2.176768in}{2.310906in}}%
\pgfpathlineto{\pgfqpoint{2.177356in}{2.278443in}}%
\pgfpathlineto{\pgfqpoint{2.178728in}{2.031996in}}%
\pgfpathlineto{\pgfqpoint{2.184023in}{0.769115in}}%
\pgfpathlineto{\pgfqpoint{2.184611in}{0.793163in}}%
\pgfpathlineto{\pgfqpoint{2.185885in}{1.000503in}}%
\pgfpathlineto{\pgfqpoint{2.191375in}{2.311640in}}%
\pgfpathlineto{\pgfqpoint{2.192160in}{2.267060in}}%
\pgfpathlineto{\pgfqpoint{2.193630in}{1.973899in}}%
\pgfpathlineto{\pgfqpoint{2.198728in}{0.747108in}}%
\pgfpathlineto{\pgfqpoint{2.199022in}{0.753493in}}%
\pgfpathlineto{\pgfqpoint{2.199905in}{0.844800in}}%
\pgfpathlineto{\pgfqpoint{2.201865in}{1.353333in}}%
\pgfpathlineto{\pgfqpoint{2.205885in}{2.309365in}}%
\pgfpathlineto{\pgfqpoint{2.206081in}{2.311627in}}%
\pgfpathlineto{\pgfqpoint{2.206179in}{2.310686in}}%
\pgfpathlineto{\pgfqpoint{2.206179in}{2.310686in}}%
\pgfpathlineto{\pgfqpoint{2.206767in}{2.276252in}}%
\pgfpathlineto{\pgfqpoint{2.208140in}{2.020878in}}%
\pgfpathlineto{\pgfqpoint{2.213434in}{0.727768in}}%
\pgfpathlineto{\pgfqpoint{2.213924in}{0.745935in}}%
\pgfpathlineto{\pgfqpoint{2.215100in}{0.923092in}}%
\pgfpathlineto{\pgfqpoint{2.218140in}{1.862024in}}%
\pgfpathlineto{\pgfqpoint{2.220689in}{2.311238in}}%
\pgfpathlineto{\pgfqpoint{2.220787in}{2.311612in}}%
\pgfpathlineto{\pgfqpoint{2.220787in}{2.311612in}}%
\pgfpathlineto{\pgfqpoint{2.220787in}{2.311612in}}%
\pgfpathlineto{\pgfqpoint{2.221179in}{2.299162in}}%
\pgfpathlineto{\pgfqpoint{2.222257in}{2.154764in}}%
\pgfpathlineto{\pgfqpoint{2.224708in}{1.421811in}}%
\pgfpathlineto{\pgfqpoint{2.227845in}{0.717191in}}%
\pgfpathlineto{\pgfqpoint{2.228139in}{0.711657in}}%
\pgfpathlineto{\pgfqpoint{2.228335in}{0.714984in}}%
\pgfpathlineto{\pgfqpoint{2.228335in}{0.714984in}}%
\pgfpathlineto{\pgfqpoint{2.229120in}{0.783402in}}%
\pgfpathlineto{\pgfqpoint{2.230884in}{1.207879in}}%
\pgfpathlineto{\pgfqpoint{2.235394in}{2.311276in}}%
\pgfpathlineto{\pgfqpoint{2.235492in}{2.311598in}}%
\pgfpathlineto{\pgfqpoint{2.235492in}{2.311598in}}%
\pgfpathlineto{\pgfqpoint{2.235492in}{2.311598in}}%
\pgfpathlineto{\pgfqpoint{2.235884in}{2.298813in}}%
\pgfpathlineto{\pgfqpoint{2.236963in}{2.152605in}}%
\pgfpathlineto{\pgfqpoint{2.239414in}{1.413116in}}%
\pgfpathlineto{\pgfqpoint{2.242551in}{0.704532in}}%
\pgfpathlineto{\pgfqpoint{2.242845in}{0.699180in}}%
\pgfpathlineto{\pgfqpoint{2.243041in}{0.702682in}}%
\pgfpathlineto{\pgfqpoint{2.243041in}{0.702682in}}%
\pgfpathlineto{\pgfqpoint{2.243825in}{0.772198in}}%
\pgfpathlineto{\pgfqpoint{2.245590in}{1.200787in}}%
\pgfpathlineto{\pgfqpoint{2.250100in}{2.311303in}}%
\pgfpathlineto{\pgfqpoint{2.250198in}{2.311586in}}%
\pgfpathlineto{\pgfqpoint{2.250198in}{2.311586in}}%
\pgfpathlineto{\pgfqpoint{2.250198in}{2.311586in}}%
\pgfpathlineto{\pgfqpoint{2.250590in}{2.298555in}}%
\pgfpathlineto{\pgfqpoint{2.251668in}{2.151031in}}%
\pgfpathlineto{\pgfqpoint{2.254217in}{1.373298in}}%
\pgfpathlineto{\pgfqpoint{2.257257in}{0.695801in}}%
\pgfpathlineto{\pgfqpoint{2.257551in}{0.690602in}}%
\pgfpathlineto{\pgfqpoint{2.257747in}{0.694244in}}%
\pgfpathlineto{\pgfqpoint{2.257747in}{0.694244in}}%
\pgfpathlineto{\pgfqpoint{2.258531in}{0.764583in}}%
\pgfpathlineto{\pgfqpoint{2.260296in}{1.196073in}}%
\pgfpathlineto{\pgfqpoint{2.264805in}{2.311320in}}%
\pgfpathlineto{\pgfqpoint{2.264903in}{2.311577in}}%
\pgfpathlineto{\pgfqpoint{2.264903in}{2.311577in}}%
\pgfpathlineto{\pgfqpoint{2.264903in}{2.311577in}}%
\pgfpathlineto{\pgfqpoint{2.265296in}{2.298393in}}%
\pgfpathlineto{\pgfqpoint{2.266374in}{2.150070in}}%
\pgfpathlineto{\pgfqpoint{2.268923in}{1.369621in}}%
\pgfpathlineto{\pgfqpoint{2.271962in}{0.691148in}}%
\pgfpathlineto{\pgfqpoint{2.272256in}{0.686073in}}%
\pgfpathlineto{\pgfqpoint{2.272452in}{0.689815in}}%
\pgfpathlineto{\pgfqpoint{2.272452in}{0.689815in}}%
\pgfpathlineto{\pgfqpoint{2.273237in}{0.760690in}}%
\pgfpathlineto{\pgfqpoint{2.275001in}{1.193824in}}%
\pgfpathlineto{\pgfqpoint{2.279511in}{2.311327in}}%
\pgfpathlineto{\pgfqpoint{2.279609in}{2.311574in}}%
\pgfpathlineto{\pgfqpoint{2.279609in}{2.311574in}}%
\pgfpathlineto{\pgfqpoint{2.279609in}{2.311574in}}%
\pgfpathlineto{\pgfqpoint{2.280001in}{2.298329in}}%
\pgfpathlineto{\pgfqpoint{2.281080in}{2.149730in}}%
\pgfpathlineto{\pgfqpoint{2.283629in}{1.368673in}}%
\pgfpathlineto{\pgfqpoint{2.286668in}{0.690633in}}%
\pgfpathlineto{\pgfqpoint{2.286962in}{0.685651in}}%
\pgfpathlineto{\pgfqpoint{2.287158in}{0.689455in}}%
\pgfpathlineto{\pgfqpoint{2.287158in}{0.689455in}}%
\pgfpathlineto{\pgfqpoint{2.287942in}{0.760574in}}%
\pgfpathlineto{\pgfqpoint{2.289707in}{1.194081in}}%
\pgfpathlineto{\pgfqpoint{2.294217in}{2.311325in}}%
\pgfpathlineto{\pgfqpoint{2.294315in}{2.311575in}}%
\pgfpathlineto{\pgfqpoint{2.294315in}{2.311575in}}%
\pgfpathlineto{\pgfqpoint{2.294315in}{2.311575in}}%
\pgfpathlineto{\pgfqpoint{2.294707in}{2.298360in}}%
\pgfpathlineto{\pgfqpoint{2.295785in}{2.150003in}}%
\pgfpathlineto{\pgfqpoint{2.298334in}{1.370441in}}%
\pgfpathlineto{\pgfqpoint{2.301373in}{0.694257in}}%
\pgfpathlineto{\pgfqpoint{2.301668in}{0.689337in}}%
\pgfpathlineto{\pgfqpoint{2.301864in}{0.693168in}}%
\pgfpathlineto{\pgfqpoint{2.301864in}{0.693168in}}%
\pgfpathlineto{\pgfqpoint{2.302648in}{0.764245in}}%
\pgfpathlineto{\pgfqpoint{2.304413in}{1.196867in}}%
\pgfpathlineto{\pgfqpoint{2.308922in}{2.311315in}}%
\pgfpathlineto{\pgfqpoint{2.309020in}{2.311580in}}%
\pgfpathlineto{\pgfqpoint{2.309020in}{2.311580in}}%
\pgfpathlineto{\pgfqpoint{2.309020in}{2.311580in}}%
\pgfpathlineto{\pgfqpoint{2.309412in}{2.298480in}}%
\pgfpathlineto{\pgfqpoint{2.310491in}{2.150868in}}%
\pgfpathlineto{\pgfqpoint{2.313040in}{1.374882in}}%
\pgfpathlineto{\pgfqpoint{2.316079in}{0.701992in}}%
\pgfpathlineto{\pgfqpoint{2.316373in}{0.697112in}}%
\pgfpathlineto{\pgfqpoint{2.316569in}{0.700933in}}%
\pgfpathlineto{\pgfqpoint{2.316569in}{0.700933in}}%
\pgfpathlineto{\pgfqpoint{2.317354in}{0.771697in}}%
\pgfpathlineto{\pgfqpoint{2.319118in}{1.202198in}}%
\pgfpathlineto{\pgfqpoint{2.323628in}{2.311297in}}%
\pgfpathlineto{\pgfqpoint{2.323726in}{2.311589in}}%
\pgfpathlineto{\pgfqpoint{2.323726in}{2.311589in}}%
\pgfpathlineto{\pgfqpoint{2.323726in}{2.311589in}}%
\pgfpathlineto{\pgfqpoint{2.324118in}{2.298681in}}%
\pgfpathlineto{\pgfqpoint{2.325197in}{2.152298in}}%
\pgfpathlineto{\pgfqpoint{2.327745in}{1.381944in}}%
\pgfpathlineto{\pgfqpoint{2.330785in}{0.713824in}}%
\pgfpathlineto{\pgfqpoint{2.331079in}{0.708964in}}%
\pgfpathlineto{\pgfqpoint{2.331275in}{0.712747in}}%
\pgfpathlineto{\pgfqpoint{2.331275in}{0.712747in}}%
\pgfpathlineto{\pgfqpoint{2.332059in}{0.782940in}}%
\pgfpathlineto{\pgfqpoint{2.333824in}{1.210116in}}%
\pgfpathlineto{\pgfqpoint{2.338334in}{2.311274in}}%
\pgfpathlineto{\pgfqpoint{2.338432in}{2.311600in}}%
\pgfpathlineto{\pgfqpoint{2.338432in}{2.311600in}}%
\pgfpathlineto{\pgfqpoint{2.338432in}{2.311600in}}%
\pgfpathlineto{\pgfqpoint{2.338824in}{2.298955in}}%
\pgfpathlineto{\pgfqpoint{2.339902in}{2.154264in}}%
\pgfpathlineto{\pgfqpoint{2.342451in}{1.391588in}}%
\pgfpathlineto{\pgfqpoint{2.345490in}{0.729778in}}%
\pgfpathlineto{\pgfqpoint{2.345784in}{0.724929in}}%
\pgfpathlineto{\pgfqpoint{2.345980in}{0.728647in}}%
\pgfpathlineto{\pgfqpoint{2.345980in}{0.728647in}}%
\pgfpathlineto{\pgfqpoint{2.346765in}{0.798033in}}%
\pgfpathlineto{\pgfqpoint{2.348529in}{1.220701in}}%
\pgfpathlineto{\pgfqpoint{2.353039in}{2.311244in}}%
\pgfpathlineto{\pgfqpoint{2.353137in}{2.311612in}}%
\pgfpathlineto{\pgfqpoint{2.353137in}{2.311612in}}%
\pgfpathlineto{\pgfqpoint{2.353137in}{2.311612in}}%
\pgfpathlineto{\pgfqpoint{2.353529in}{2.299291in}}%
\pgfpathlineto{\pgfqpoint{2.354608in}{2.156740in}}%
\pgfpathlineto{\pgfqpoint{2.357157in}{1.403809in}}%
\pgfpathlineto{\pgfqpoint{2.360196in}{0.749954in}}%
\pgfpathlineto{\pgfqpoint{2.360490in}{0.745113in}}%
\pgfpathlineto{\pgfqpoint{2.360686in}{0.748749in}}%
\pgfpathlineto{\pgfqpoint{2.360686in}{0.748749in}}%
\pgfpathlineto{\pgfqpoint{2.361470in}{0.817109in}}%
\pgfpathlineto{\pgfqpoint{2.363235in}{1.234093in}}%
\pgfpathlineto{\pgfqpoint{2.367745in}{2.311210in}}%
\pgfpathlineto{\pgfqpoint{2.367843in}{2.311624in}}%
\pgfpathlineto{\pgfqpoint{2.367843in}{2.311624in}}%
\pgfpathlineto{\pgfqpoint{2.367843in}{2.311624in}}%
\pgfpathlineto{\pgfqpoint{2.368235in}{2.299678in}}%
\pgfpathlineto{\pgfqpoint{2.369313in}{2.159704in}}%
\pgfpathlineto{\pgfqpoint{2.371764in}{1.450674in}}%
\pgfpathlineto{\pgfqpoint{2.374902in}{0.774547in}}%
\pgfpathlineto{\pgfqpoint{2.375196in}{0.769725in}}%
\pgfpathlineto{\pgfqpoint{2.375392in}{0.773265in}}%
\pgfpathlineto{\pgfqpoint{2.375392in}{0.773265in}}%
\pgfpathlineto{\pgfqpoint{2.376176in}{0.840396in}}%
\pgfpathlineto{\pgfqpoint{2.377941in}{1.250506in}}%
\pgfpathlineto{\pgfqpoint{2.382450in}{2.311174in}}%
\pgfpathlineto{\pgfqpoint{2.382548in}{2.311633in}}%
\pgfpathlineto{\pgfqpoint{2.382548in}{2.311633in}}%
\pgfpathlineto{\pgfqpoint{2.382548in}{2.311633in}}%
\pgfpathlineto{\pgfqpoint{2.382941in}{2.300108in}}%
\pgfpathlineto{\pgfqpoint{2.384019in}{2.163146in}}%
\pgfpathlineto{\pgfqpoint{2.386470in}{1.467647in}}%
\pgfpathlineto{\pgfqpoint{2.389607in}{0.803868in}}%
\pgfpathlineto{\pgfqpoint{2.389901in}{0.799085in}}%
\pgfpathlineto{\pgfqpoint{2.390097in}{0.802523in}}%
\pgfpathlineto{\pgfqpoint{2.390097in}{0.802523in}}%
\pgfpathlineto{\pgfqpoint{2.390882in}{0.868236in}}%
\pgfpathlineto{\pgfqpoint{2.392646in}{1.270235in}}%
\pgfpathlineto{\pgfqpoint{2.397156in}{2.311138in}}%
\pgfpathlineto{\pgfqpoint{2.397254in}{2.311641in}}%
\pgfpathlineto{\pgfqpoint{2.397254in}{2.311641in}}%
\pgfpathlineto{\pgfqpoint{2.397254in}{2.311641in}}%
\pgfpathlineto{\pgfqpoint{2.397646in}{2.300573in}}%
\pgfpathlineto{\pgfqpoint{2.398725in}{2.167067in}}%
\pgfpathlineto{\pgfqpoint{2.401176in}{1.487418in}}%
\pgfpathlineto{\pgfqpoint{2.404313in}{0.838348in}}%
\pgfpathlineto{\pgfqpoint{2.404607in}{0.833636in}}%
\pgfpathlineto{\pgfqpoint{2.404803in}{0.836970in}}%
\pgfpathlineto{\pgfqpoint{2.404803in}{0.836970in}}%
\pgfpathlineto{\pgfqpoint{2.405587in}{0.901083in}}%
\pgfpathlineto{\pgfqpoint{2.407352in}{1.293656in}}%
\pgfpathlineto{\pgfqpoint{2.411862in}{2.311103in}}%
\pgfpathlineto{\pgfqpoint{2.411960in}{2.311646in}}%
\pgfpathlineto{\pgfqpoint{2.411960in}{2.311646in}}%
\pgfpathlineto{\pgfqpoint{2.411960in}{2.311646in}}%
\pgfpathlineto{\pgfqpoint{2.412352in}{2.301066in}}%
\pgfpathlineto{\pgfqpoint{2.413430in}{2.171482in}}%
\pgfpathlineto{\pgfqpoint{2.415881in}{1.510225in}}%
\pgfpathlineto{\pgfqpoint{2.419018in}{0.878526in}}%
\pgfpathlineto{\pgfqpoint{2.419313in}{0.873929in}}%
\pgfpathlineto{\pgfqpoint{2.419509in}{0.877165in}}%
\pgfpathlineto{\pgfqpoint{2.419509in}{0.877165in}}%
\pgfpathlineto{\pgfqpoint{2.420293in}{0.939493in}}%
\pgfpathlineto{\pgfqpoint{2.422058in}{1.321215in}}%
\pgfpathlineto{\pgfqpoint{2.426567in}{2.311072in}}%
\pgfpathlineto{\pgfqpoint{2.426665in}{2.311649in}}%
\pgfpathlineto{\pgfqpoint{2.426665in}{2.311649in}}%
\pgfpathlineto{\pgfqpoint{2.426665in}{2.311649in}}%
\pgfpathlineto{\pgfqpoint{2.427057in}{2.301582in}}%
\pgfpathlineto{\pgfqpoint{2.428038in}{2.193277in}}%
\pgfpathlineto{\pgfqpoint{2.430293in}{1.623842in}}%
\pgfpathlineto{\pgfqpoint{2.433724in}{0.925022in}}%
\pgfpathlineto{\pgfqpoint{2.434018in}{0.920594in}}%
\pgfpathlineto{\pgfqpoint{2.434214in}{0.923738in}}%
\pgfpathlineto{\pgfqpoint{2.434214in}{0.923738in}}%
\pgfpathlineto{\pgfqpoint{2.434999in}{0.984091in}}%
\pgfpathlineto{\pgfqpoint{2.436763in}{1.353400in}}%
\pgfpathlineto{\pgfqpoint{2.441273in}{2.311048in}}%
\pgfpathlineto{\pgfqpoint{2.441371in}{2.311650in}}%
\pgfpathlineto{\pgfqpoint{2.441469in}{2.311055in}}%
\pgfpathlineto{\pgfqpoint{2.441469in}{2.311055in}}%
\pgfpathlineto{\pgfqpoint{2.442057in}{2.282606in}}%
\pgfpathlineto{\pgfqpoint{2.443430in}{2.065706in}}%
\pgfpathlineto{\pgfqpoint{2.448724in}{0.974273in}}%
\pgfpathlineto{\pgfqpoint{2.449214in}{0.990660in}}%
\pgfpathlineto{\pgfqpoint{2.450390in}{1.142094in}}%
\pgfpathlineto{\pgfqpoint{2.453626in}{1.981457in}}%
\pgfpathlineto{\pgfqpoint{2.456077in}{2.311649in}}%
\pgfpathlineto{\pgfqpoint{2.456371in}{2.306624in}}%
\pgfpathlineto{\pgfqpoint{2.457253in}{2.231583in}}%
\pgfpathlineto{\pgfqpoint{2.459214in}{1.813335in}}%
\pgfpathlineto{\pgfqpoint{2.463135in}{1.039438in}}%
\pgfpathlineto{\pgfqpoint{2.463429in}{1.035534in}}%
\pgfpathlineto{\pgfqpoint{2.463625in}{1.038525in}}%
\pgfpathlineto{\pgfqpoint{2.463625in}{1.038525in}}%
\pgfpathlineto{\pgfqpoint{2.464410in}{1.094297in}}%
\pgfpathlineto{\pgfqpoint{2.466174in}{1.433495in}}%
\pgfpathlineto{\pgfqpoint{2.470684in}{2.311032in}}%
\pgfpathlineto{\pgfqpoint{2.470782in}{2.311648in}}%
\pgfpathlineto{\pgfqpoint{2.470880in}{2.311174in}}%
\pgfpathlineto{\pgfqpoint{2.470880in}{2.311174in}}%
\pgfpathlineto{\pgfqpoint{2.471468in}{2.285697in}}%
\pgfpathlineto{\pgfqpoint{2.472841in}{2.089529in}}%
\pgfpathlineto{\pgfqpoint{2.478135in}{1.104745in}}%
\pgfpathlineto{\pgfqpoint{2.478625in}{1.119942in}}%
\pgfpathlineto{\pgfqpoint{2.479802in}{1.257416in}}%
\pgfpathlineto{\pgfqpoint{2.483037in}{2.014325in}}%
\pgfpathlineto{\pgfqpoint{2.485488in}{2.311646in}}%
\pgfpathlineto{\pgfqpoint{2.485782in}{2.307312in}}%
\pgfpathlineto{\pgfqpoint{2.486664in}{2.240761in}}%
\pgfpathlineto{\pgfqpoint{2.488625in}{1.869087in}}%
\pgfpathlineto{\pgfqpoint{2.492547in}{1.185068in}}%
\pgfpathlineto{\pgfqpoint{2.492841in}{1.181942in}}%
\pgfpathlineto{\pgfqpoint{2.493037in}{1.184812in}}%
\pgfpathlineto{\pgfqpoint{2.493037in}{1.184812in}}%
\pgfpathlineto{\pgfqpoint{2.493821in}{1.235030in}}%
\pgfpathlineto{\pgfqpoint{2.495586in}{1.536360in}}%
\pgfpathlineto{\pgfqpoint{2.500095in}{2.311069in}}%
\pgfpathlineto{\pgfqpoint{2.500193in}{2.311645in}}%
\pgfpathlineto{\pgfqpoint{2.500291in}{2.311267in}}%
\pgfpathlineto{\pgfqpoint{2.500291in}{2.311267in}}%
\pgfpathlineto{\pgfqpoint{2.500880in}{2.289188in}}%
\pgfpathlineto{\pgfqpoint{2.502252in}{2.118350in}}%
\pgfpathlineto{\pgfqpoint{2.507448in}{1.266663in}}%
\pgfpathlineto{\pgfqpoint{2.508036in}{1.280533in}}%
\pgfpathlineto{\pgfqpoint{2.509213in}{1.400940in}}%
\pgfpathlineto{\pgfqpoint{2.512644in}{2.092086in}}%
\pgfpathlineto{\pgfqpoint{2.514899in}{2.311645in}}%
\pgfpathlineto{\pgfqpoint{2.515193in}{2.307998in}}%
\pgfpathlineto{\pgfqpoint{2.516076in}{2.251356in}}%
\pgfpathlineto{\pgfqpoint{2.518036in}{1.935589in}}%
\pgfpathlineto{\pgfqpoint{2.521958in}{1.360161in}}%
\pgfpathlineto{\pgfqpoint{2.522154in}{1.357881in}}%
\pgfpathlineto{\pgfqpoint{2.522350in}{1.358951in}}%
\pgfpathlineto{\pgfqpoint{2.522350in}{1.358951in}}%
\pgfpathlineto{\pgfqpoint{2.523036in}{1.388688in}}%
\pgfpathlineto{\pgfqpoint{2.524605in}{1.589119in}}%
\pgfpathlineto{\pgfqpoint{2.529605in}{2.311645in}}%
\pgfpathlineto{\pgfqpoint{2.529801in}{2.310229in}}%
\pgfpathlineto{\pgfqpoint{2.530487in}{2.280776in}}%
\pgfpathlineto{\pgfqpoint{2.532056in}{2.090871in}}%
\pgfpathlineto{\pgfqpoint{2.536859in}{1.454039in}}%
\pgfpathlineto{\pgfqpoint{2.536958in}{1.454239in}}%
\pgfpathlineto{\pgfqpoint{2.537448in}{1.466466in}}%
\pgfpathlineto{\pgfqpoint{2.538722in}{1.579413in}}%
\pgfpathlineto{\pgfqpoint{2.544310in}{2.311647in}}%
\pgfpathlineto{\pgfqpoint{2.545095in}{2.289822in}}%
\pgfpathlineto{\pgfqpoint{2.546565in}{2.142668in}}%
\pgfpathlineto{\pgfqpoint{2.551565in}{1.552950in}}%
\pgfpathlineto{\pgfqpoint{2.551859in}{1.555777in}}%
\pgfpathlineto{\pgfqpoint{2.552742in}{1.599150in}}%
\pgfpathlineto{\pgfqpoint{2.554702in}{1.842414in}}%
\pgfpathlineto{\pgfqpoint{2.558722in}{2.308727in}}%
\pgfpathlineto{\pgfqpoint{2.559016in}{2.311649in}}%
\pgfpathlineto{\pgfqpoint{2.559310in}{2.308979in}}%
\pgfpathlineto{\pgfqpoint{2.559310in}{2.308979in}}%
\pgfpathlineto{\pgfqpoint{2.560192in}{2.268738in}}%
\pgfpathlineto{\pgfqpoint{2.562153in}{2.047148in}}%
\pgfpathlineto{\pgfqpoint{2.565977in}{1.654457in}}%
\pgfpathlineto{\pgfqpoint{2.566271in}{1.651999in}}%
\pgfpathlineto{\pgfqpoint{2.566467in}{1.653257in}}%
\pgfpathlineto{\pgfqpoint{2.566467in}{1.653257in}}%
\pgfpathlineto{\pgfqpoint{2.567251in}{1.680963in}}%
\pgfpathlineto{\pgfqpoint{2.569016in}{1.853771in}}%
\pgfpathlineto{\pgfqpoint{2.573624in}{2.311371in}}%
\pgfpathlineto{\pgfqpoint{2.573722in}{2.311650in}}%
\pgfpathlineto{\pgfqpoint{2.573820in}{2.311393in}}%
\pgfpathlineto{\pgfqpoint{2.573820in}{2.311393in}}%
\pgfpathlineto{\pgfqpoint{2.574408in}{2.298781in}}%
\pgfpathlineto{\pgfqpoint{2.575780in}{2.203523in}}%
\pgfpathlineto{\pgfqpoint{2.580976in}{1.748373in}}%
\pgfpathlineto{\pgfqpoint{2.581368in}{1.752864in}}%
\pgfpathlineto{\pgfqpoint{2.582447in}{1.803897in}}%
\pgfpathlineto{\pgfqpoint{2.584996in}{2.071044in}}%
\pgfpathlineto{\pgfqpoint{2.588133in}{2.309658in}}%
\pgfpathlineto{\pgfqpoint{2.588427in}{2.311650in}}%
\pgfpathlineto{\pgfqpoint{2.588623in}{2.310709in}}%
\pgfpathlineto{\pgfqpoint{2.588623in}{2.310709in}}%
\pgfpathlineto{\pgfqpoint{2.589408in}{2.289390in}}%
\pgfpathlineto{\pgfqpoint{2.591172in}{2.158125in}}%
\pgfpathlineto{\pgfqpoint{2.595486in}{1.839810in}}%
\pgfpathlineto{\pgfqpoint{2.595682in}{1.839379in}}%
\pgfpathlineto{\pgfqpoint{2.595780in}{1.839787in}}%
\pgfpathlineto{\pgfqpoint{2.595780in}{1.839787in}}%
\pgfpathlineto{\pgfqpoint{2.596466in}{1.854071in}}%
\pgfpathlineto{\pgfqpoint{2.598035in}{1.952231in}}%
\pgfpathlineto{\pgfqpoint{2.603133in}{2.311647in}}%
\pgfpathlineto{\pgfqpoint{2.603231in}{2.311415in}}%
\pgfpathlineto{\pgfqpoint{2.603819in}{2.302263in}}%
\pgfpathlineto{\pgfqpoint{2.605192in}{2.235093in}}%
\pgfpathlineto{\pgfqpoint{2.610290in}{1.922692in}}%
\pgfpathlineto{\pgfqpoint{2.610682in}{1.925014in}}%
\pgfpathlineto{\pgfqpoint{2.611662in}{1.953749in}}%
\pgfpathlineto{\pgfqpoint{2.613917in}{2.108559in}}%
\pgfpathlineto{\pgfqpoint{2.617446in}{2.309433in}}%
\pgfpathlineto{\pgfqpoint{2.617838in}{2.311643in}}%
\pgfpathlineto{\pgfqpoint{2.618133in}{2.310074in}}%
\pgfpathlineto{\pgfqpoint{2.618133in}{2.310074in}}%
\pgfpathlineto{\pgfqpoint{2.619113in}{2.285963in}}%
\pgfpathlineto{\pgfqpoint{2.621368in}{2.154129in}}%
\pgfpathlineto{\pgfqpoint{2.624701in}{1.998138in}}%
\pgfpathlineto{\pgfqpoint{2.624995in}{1.996835in}}%
\pgfpathlineto{\pgfqpoint{2.625289in}{1.998023in}}%
\pgfpathlineto{\pgfqpoint{2.625289in}{1.998023in}}%
\pgfpathlineto{\pgfqpoint{2.626172in}{2.016049in}}%
\pgfpathlineto{\pgfqpoint{2.628132in}{2.116584in}}%
\pgfpathlineto{\pgfqpoint{2.632152in}{2.309972in}}%
\pgfpathlineto{\pgfqpoint{2.632544in}{2.311637in}}%
\pgfpathlineto{\pgfqpoint{2.632838in}{2.310297in}}%
\pgfpathlineto{\pgfqpoint{2.632838in}{2.310297in}}%
\pgfpathlineto{\pgfqpoint{2.633819in}{2.290732in}}%
\pgfpathlineto{\pgfqpoint{2.636073in}{2.185144in}}%
\pgfpathlineto{\pgfqpoint{2.639407in}{2.061942in}}%
\pgfpathlineto{\pgfqpoint{2.639701in}{2.061064in}}%
\pgfpathlineto{\pgfqpoint{2.639897in}{2.061582in}}%
\pgfpathlineto{\pgfqpoint{2.639897in}{2.061582in}}%
\pgfpathlineto{\pgfqpoint{2.640681in}{2.072253in}}%
\pgfpathlineto{\pgfqpoint{2.642446in}{2.137881in}}%
\pgfpathlineto{\pgfqpoint{2.647054in}{2.311414in}}%
\pgfpathlineto{\pgfqpoint{2.647250in}{2.311631in}}%
\pgfpathlineto{\pgfqpoint{2.647348in}{2.311447in}}%
\pgfpathlineto{\pgfqpoint{2.647348in}{2.311447in}}%
\pgfpathlineto{\pgfqpoint{2.648034in}{2.304863in}}%
\pgfpathlineto{\pgfqpoint{2.649603in}{2.260409in}}%
\pgfpathlineto{\pgfqpoint{2.654406in}{2.115411in}}%
\pgfpathlineto{\pgfqpoint{2.654897in}{2.117924in}}%
\pgfpathlineto{\pgfqpoint{2.656073in}{2.140270in}}%
\pgfpathlineto{\pgfqpoint{2.659504in}{2.268518in}}%
\pgfpathlineto{\pgfqpoint{2.661857in}{2.311637in}}%
\pgfpathlineto{\pgfqpoint{2.662249in}{2.310686in}}%
\pgfpathlineto{\pgfqpoint{2.663230in}{2.298373in}}%
\pgfpathlineto{\pgfqpoint{2.665583in}{2.230602in}}%
\pgfpathlineto{\pgfqpoint{2.668720in}{2.161206in}}%
\pgfpathlineto{\pgfqpoint{2.669014in}{2.160486in}}%
\pgfpathlineto{\pgfqpoint{2.669308in}{2.160965in}}%
\pgfpathlineto{\pgfqpoint{2.669308in}{2.160965in}}%
\pgfpathlineto{\pgfqpoint{2.670190in}{2.169356in}}%
\pgfpathlineto{\pgfqpoint{2.672151in}{2.217157in}}%
\pgfpathlineto{\pgfqpoint{2.676269in}{2.311027in}}%
\pgfpathlineto{\pgfqpoint{2.676661in}{2.311620in}}%
\pgfpathlineto{\pgfqpoint{2.676955in}{2.310855in}}%
\pgfpathlineto{\pgfqpoint{2.676955in}{2.310855in}}%
\pgfpathlineto{\pgfqpoint{2.677935in}{2.301326in}}%
\pgfpathlineto{\pgfqpoint{2.680386in}{2.247277in}}%
\pgfpathlineto{\pgfqpoint{2.683426in}{2.197781in}}%
\pgfpathlineto{\pgfqpoint{2.683720in}{2.197326in}}%
\pgfpathlineto{\pgfqpoint{2.684014in}{2.197777in}}%
\pgfpathlineto{\pgfqpoint{2.684014in}{2.197777in}}%
\pgfpathlineto{\pgfqpoint{2.684896in}{2.204373in}}%
\pgfpathlineto{\pgfqpoint{2.686857in}{2.240826in}}%
\pgfpathlineto{\pgfqpoint{2.690974in}{2.311225in}}%
\pgfpathlineto{\pgfqpoint{2.691367in}{2.311617in}}%
\pgfpathlineto{\pgfqpoint{2.691661in}{2.311008in}}%
\pgfpathlineto{\pgfqpoint{2.691661in}{2.311008in}}%
\pgfpathlineto{\pgfqpoint{2.692739in}{2.302664in}}%
\pgfpathlineto{\pgfqpoint{2.695484in}{2.256417in}}%
\pgfpathlineto{\pgfqpoint{2.698229in}{2.227166in}}%
\pgfpathlineto{\pgfqpoint{2.698523in}{2.227126in}}%
\pgfpathlineto{\pgfqpoint{2.698817in}{2.227755in}}%
\pgfpathlineto{\pgfqpoint{2.699896in}{2.235439in}}%
\pgfpathlineto{\pgfqpoint{2.702445in}{2.275214in}}%
\pgfpathlineto{\pgfqpoint{2.705582in}{2.311145in}}%
\pgfpathlineto{\pgfqpoint{2.706072in}{2.311618in}}%
\pgfpathlineto{\pgfqpoint{2.706464in}{2.310868in}}%
\pgfpathlineto{\pgfqpoint{2.707641in}{2.303213in}}%
\pgfpathlineto{\pgfqpoint{2.713033in}{2.250766in}}%
\pgfpathlineto{\pgfqpoint{2.713915in}{2.252763in}}%
\pgfpathlineto{\pgfqpoint{2.715484in}{2.265484in}}%
\pgfpathlineto{\pgfqpoint{2.720680in}{2.311650in}}%
\pgfpathlineto{\pgfqpoint{2.721366in}{2.310538in}}%
\pgfpathlineto{\pgfqpoint{2.722837in}{2.302001in}}%
\pgfpathlineto{\pgfqpoint{2.727738in}{2.269374in}}%
\pgfpathlineto{\pgfqpoint{2.727935in}{2.269459in}}%
\pgfpathlineto{\pgfqpoint{2.728817in}{2.271628in}}%
\pgfpathlineto{\pgfqpoint{2.730680in}{2.283855in}}%
\pgfpathlineto{\pgfqpoint{2.734993in}{2.311436in}}%
\pgfpathlineto{\pgfqpoint{2.735680in}{2.311491in}}%
\pgfpathlineto{\pgfqpoint{2.735974in}{2.311071in}}%
\pgfpathlineto{\pgfqpoint{2.737346in}{2.306121in}}%
\pgfpathlineto{\pgfqpoint{2.742444in}{2.283685in}}%
\pgfpathlineto{\pgfqpoint{2.742738in}{2.283844in}}%
\pgfpathlineto{\pgfqpoint{2.743915in}{2.286563in}}%
\pgfpathlineto{\pgfqpoint{2.746660in}{2.300781in}}%
\pgfpathlineto{\pgfqpoint{2.749699in}{2.311518in}}%
\pgfpathlineto{\pgfqpoint{2.750875in}{2.311002in}}%
\pgfpathlineto{\pgfqpoint{2.752640in}{2.306082in}}%
\pgfpathlineto{\pgfqpoint{2.756954in}{2.294349in}}%
\pgfpathlineto{\pgfqpoint{2.758130in}{2.295296in}}%
\pgfpathlineto{\pgfqpoint{2.760189in}{2.300976in}}%
\pgfpathlineto{\pgfqpoint{2.764306in}{2.311523in}}%
\pgfpathlineto{\pgfqpoint{2.765777in}{2.311073in}}%
\pgfpathlineto{\pgfqpoint{2.768130in}{2.306717in}}%
\pgfpathlineto{\pgfqpoint{2.771463in}{2.301966in}}%
\pgfpathlineto{\pgfqpoint{2.773032in}{2.302778in}}%
\pgfpathlineto{\pgfqpoint{2.775973in}{2.307918in}}%
\pgfpathlineto{\pgfqpoint{2.779110in}{2.311606in}}%
\pgfpathlineto{\pgfqpoint{2.781071in}{2.310961in}}%
\pgfpathlineto{\pgfqpoint{2.787247in}{2.307240in}}%
\pgfpathlineto{\pgfqpoint{2.790581in}{2.309871in}}%
\pgfpathlineto{\pgfqpoint{2.794208in}{2.311650in}}%
\pgfpathlineto{\pgfqpoint{2.798031in}{2.310551in}}%
\pgfpathlineto{\pgfqpoint{2.802247in}{2.310203in}}%
\pgfpathlineto{\pgfqpoint{2.813129in}{2.311443in}}%
\pgfpathlineto{\pgfqpoint{2.835580in}{2.311618in}}%
\pgfpathlineto{\pgfqpoint{2.855677in}{2.311017in}}%
\pgfpathlineto{\pgfqpoint{2.861952in}{2.309828in}}%
\pgfpathlineto{\pgfqpoint{2.869599in}{2.310967in}}%
\pgfpathlineto{\pgfqpoint{2.876265in}{2.308349in}}%
\pgfpathlineto{\pgfqpoint{2.884304in}{2.310624in}}%
\pgfpathlineto{\pgfqpoint{2.890481in}{2.306691in}}%
\pgfpathlineto{\pgfqpoint{2.893226in}{2.309105in}}%
\pgfpathlineto{\pgfqpoint{2.896853in}{2.311645in}}%
\pgfpathlineto{\pgfqpoint{2.898912in}{2.310402in}}%
\pgfpathlineto{\pgfqpoint{2.905088in}{2.305290in}}%
\pgfpathlineto{\pgfqpoint{2.907441in}{2.307767in}}%
\pgfpathlineto{\pgfqpoint{2.911069in}{2.311803in}}%
\pgfpathlineto{\pgfqpoint{2.911363in}{2.311506in}}%
\pgfpathlineto{\pgfqpoint{2.913029in}{2.307441in}}%
\pgfpathlineto{\pgfqpoint{2.915480in}{2.303448in}}%
\pgfpathlineto{\pgfqpoint{2.916657in}{2.304966in}}%
\pgfpathlineto{\pgfqpoint{2.920186in}{2.312156in}}%
\pgfpathlineto{\pgfqpoint{2.920774in}{2.311446in}}%
\pgfpathlineto{\pgfqpoint{2.922833in}{2.304941in}}%
\pgfpathlineto{\pgfqpoint{2.924696in}{2.302013in}}%
\pgfpathlineto{\pgfqpoint{2.924892in}{2.302110in}}%
\pgfpathlineto{\pgfqpoint{2.926166in}{2.304534in}}%
\pgfpathlineto{\pgfqpoint{2.929402in}{2.312526in}}%
\pgfpathlineto{\pgfqpoint{2.929794in}{2.312138in}}%
\pgfpathlineto{\pgfqpoint{2.931264in}{2.307492in}}%
\pgfpathlineto{\pgfqpoint{2.933911in}{2.300708in}}%
\pgfpathlineto{\pgfqpoint{2.934009in}{2.300742in}}%
\pgfpathlineto{\pgfqpoint{2.935088in}{2.302594in}}%
\pgfpathlineto{\pgfqpoint{2.938617in}{2.312941in}}%
\pgfpathlineto{\pgfqpoint{2.939303in}{2.311951in}}%
\pgfpathlineto{\pgfqpoint{2.941362in}{2.303389in}}%
\pgfpathlineto{\pgfqpoint{2.943127in}{2.299558in}}%
\pgfpathlineto{\pgfqpoint{2.943323in}{2.299637in}}%
\pgfpathlineto{\pgfqpoint{2.944401in}{2.301874in}}%
\pgfpathlineto{\pgfqpoint{2.947833in}{2.313361in}}%
\pgfpathlineto{\pgfqpoint{2.948519in}{2.312363in}}%
\pgfpathlineto{\pgfqpoint{2.950382in}{2.303925in}}%
\pgfpathlineto{\pgfqpoint{2.952440in}{2.298604in}}%
\pgfpathlineto{\pgfqpoint{2.952538in}{2.298651in}}%
\pgfpathlineto{\pgfqpoint{2.953519in}{2.300670in}}%
\pgfpathlineto{\pgfqpoint{2.957146in}{2.313713in}}%
\pgfpathlineto{\pgfqpoint{2.957931in}{2.312130in}}%
\pgfpathlineto{\pgfqpoint{2.961656in}{2.297829in}}%
\pgfpathlineto{\pgfqpoint{2.962538in}{2.299321in}}%
\pgfpathlineto{\pgfqpoint{2.964989in}{2.310575in}}%
\pgfpathlineto{\pgfqpoint{2.966362in}{2.314000in}}%
\pgfpathlineto{\pgfqpoint{2.966656in}{2.313696in}}%
\pgfpathlineto{\pgfqpoint{2.967832in}{2.309231in}}%
\pgfpathlineto{\pgfqpoint{2.970871in}{2.297294in}}%
\pgfpathlineto{\pgfqpoint{2.971068in}{2.297406in}}%
\pgfpathlineto{\pgfqpoint{2.972048in}{2.299845in}}%
\pgfpathlineto{\pgfqpoint{2.975577in}{2.314127in}}%
\pgfpathlineto{\pgfqpoint{2.976362in}{2.312496in}}%
\pgfpathlineto{\pgfqpoint{2.980087in}{2.297036in}}%
\pgfpathlineto{\pgfqpoint{2.980969in}{2.298588in}}%
\pgfpathlineto{\pgfqpoint{2.983420in}{2.310449in}}%
\pgfpathlineto{\pgfqpoint{2.984793in}{2.314043in}}%
\pgfpathlineto{\pgfqpoint{2.985087in}{2.313730in}}%
\pgfpathlineto{\pgfqpoint{2.986263in}{2.309133in}}%
\pgfpathlineto{\pgfqpoint{2.989303in}{2.297101in}}%
\pgfpathlineto{\pgfqpoint{2.989499in}{2.297229in}}%
\pgfpathlineto{\pgfqpoint{2.990577in}{2.300140in}}%
\pgfpathlineto{\pgfqpoint{2.993910in}{2.313742in}}%
\pgfpathlineto{\pgfqpoint{2.994597in}{2.312677in}}%
\pgfpathlineto{\pgfqpoint{2.996459in}{2.303270in}}%
\pgfpathlineto{\pgfqpoint{2.998420in}{2.297506in}}%
\pgfpathlineto{\pgfqpoint{2.998616in}{2.297592in}}%
\pgfpathlineto{\pgfqpoint{2.999596in}{2.299836in}}%
\pgfpathlineto{\pgfqpoint{3.003126in}{2.313198in}}%
\pgfpathlineto{\pgfqpoint{3.003910in}{2.311776in}}%
\pgfpathlineto{\pgfqpoint{3.007636in}{2.298334in}}%
\pgfpathlineto{\pgfqpoint{3.008518in}{2.299770in}}%
\pgfpathlineto{\pgfqpoint{3.012341in}{2.312508in}}%
\pgfpathlineto{\pgfqpoint{3.013224in}{2.310755in}}%
\pgfpathlineto{\pgfqpoint{3.016753in}{2.299584in}}%
\pgfpathlineto{\pgfqpoint{3.017341in}{2.300213in}}%
\pgfpathlineto{\pgfqpoint{3.019106in}{2.305962in}}%
\pgfpathlineto{\pgfqpoint{3.021459in}{2.311909in}}%
\pgfpathlineto{\pgfqpoint{3.021655in}{2.311788in}}%
\pgfpathlineto{\pgfqpoint{3.022831in}{2.308946in}}%
\pgfpathlineto{\pgfqpoint{3.025871in}{2.301340in}}%
\pgfpathlineto{\pgfqpoint{3.025969in}{2.301378in}}%
\pgfpathlineto{\pgfqpoint{3.027145in}{2.303315in}}%
\pgfpathlineto{\pgfqpoint{3.030674in}{2.311631in}}%
\pgfpathlineto{\pgfqpoint{3.031165in}{2.311206in}}%
\pgfpathlineto{\pgfqpoint{3.033027in}{2.306472in}}%
\pgfpathlineto{\pgfqpoint{3.035086in}{2.303725in}}%
\pgfpathlineto{\pgfqpoint{3.035184in}{2.303761in}}%
\pgfpathlineto{\pgfqpoint{3.036557in}{2.305699in}}%
\pgfpathlineto{\pgfqpoint{3.040282in}{2.312049in}}%
\pgfpathlineto{\pgfqpoint{3.040478in}{2.311927in}}%
\pgfpathlineto{\pgfqpoint{3.042243in}{2.309086in}}%
\pgfpathlineto{\pgfqpoint{3.044596in}{2.306768in}}%
\pgfpathlineto{\pgfqpoint{3.046066in}{2.308096in}}%
\pgfpathlineto{\pgfqpoint{3.050674in}{2.314288in}}%
\pgfpathlineto{\pgfqpoint{3.050968in}{2.314051in}}%
\pgfpathlineto{\pgfqpoint{3.055478in}{2.309856in}}%
\pgfpathlineto{\pgfqpoint{3.055968in}{2.310466in}}%
\pgfpathlineto{\pgfqpoint{3.058027in}{2.315902in}}%
\pgfpathlineto{\pgfqpoint{3.060184in}{2.320018in}}%
\pgfpathlineto{\pgfqpoint{3.060380in}{2.319928in}}%
\pgfpathlineto{\pgfqpoint{3.061654in}{2.317491in}}%
\pgfpathlineto{\pgfqpoint{3.064791in}{2.311008in}}%
\pgfpathlineto{\pgfqpoint{3.065086in}{2.311259in}}%
\pgfpathlineto{\pgfqpoint{3.066262in}{2.314448in}}%
\pgfpathlineto{\pgfqpoint{3.069497in}{2.327485in}}%
\pgfpathlineto{\pgfqpoint{3.070085in}{2.326600in}}%
\pgfpathlineto{\pgfqpoint{3.072046in}{2.317687in}}%
\pgfpathlineto{\pgfqpoint{3.074007in}{2.311581in}}%
\pgfpathlineto{\pgfqpoint{3.074301in}{2.311789in}}%
\pgfpathlineto{\pgfqpoint{3.075281in}{2.314982in}}%
\pgfpathlineto{\pgfqpoint{3.078615in}{2.335364in}}%
\pgfpathlineto{\pgfqpoint{3.079595in}{2.332499in}}%
\pgfpathlineto{\pgfqpoint{3.083222in}{2.311566in}}%
\pgfpathlineto{\pgfqpoint{3.083909in}{2.312885in}}%
\pgfpathlineto{\pgfqpoint{3.085183in}{2.322266in}}%
\pgfpathlineto{\pgfqpoint{3.087830in}{2.342175in}}%
\pgfpathlineto{\pgfqpoint{3.088124in}{2.341687in}}%
\pgfpathlineto{\pgfqpoint{3.089203in}{2.334745in}}%
\pgfpathlineto{\pgfqpoint{3.092438in}{2.310964in}}%
\pgfpathlineto{\pgfqpoint{3.092830in}{2.311555in}}%
\pgfpathlineto{\pgfqpoint{3.094203in}{2.319566in}}%
\pgfpathlineto{\pgfqpoint{3.099203in}{2.345118in}}%
\pgfpathlineto{\pgfqpoint{3.099399in}{2.345049in}}%
\pgfpathlineto{\pgfqpoint{3.100379in}{2.342982in}}%
\pgfpathlineto{\pgfqpoint{3.102438in}{2.331342in}}%
\pgfpathlineto{\pgfqpoint{3.106261in}{2.311749in}}%
\pgfpathlineto{\pgfqpoint{3.107046in}{2.312067in}}%
\pgfpathlineto{\pgfqpoint{3.107242in}{2.312435in}}%
\pgfpathlineto{\pgfqpoint{3.108712in}{2.318446in}}%
\pgfpathlineto{\pgfqpoint{3.113908in}{2.344395in}}%
\pgfpathlineto{\pgfqpoint{3.114202in}{2.344191in}}%
\pgfpathlineto{\pgfqpoint{3.115379in}{2.340917in}}%
\pgfpathlineto{\pgfqpoint{3.118124in}{2.323808in}}%
\pgfpathlineto{\pgfqpoint{3.121065in}{2.311694in}}%
\pgfpathlineto{\pgfqpoint{3.121947in}{2.312325in}}%
\pgfpathlineto{\pgfqpoint{3.123418in}{2.317680in}}%
\pgfpathlineto{\pgfqpoint{3.128516in}{2.340058in}}%
\pgfpathlineto{\pgfqpoint{3.128810in}{2.339905in}}%
\pgfpathlineto{\pgfqpoint{3.129986in}{2.337163in}}%
\pgfpathlineto{\pgfqpoint{3.132633in}{2.323100in}}%
\pgfpathlineto{\pgfqpoint{3.135673in}{2.311758in}}%
\pgfpathlineto{\pgfqpoint{3.136751in}{2.312306in}}%
\pgfpathlineto{\pgfqpoint{3.138418in}{2.317554in}}%
\pgfpathlineto{\pgfqpoint{3.143123in}{2.333288in}}%
\pgfpathlineto{\pgfqpoint{3.144202in}{2.332113in}}%
\pgfpathlineto{\pgfqpoint{3.146163in}{2.325407in}}%
\pgfpathlineto{\pgfqpoint{3.150378in}{2.311750in}}%
\pgfpathlineto{\pgfqpoint{3.151653in}{2.312303in}}%
\pgfpathlineto{\pgfqpoint{3.153613in}{2.317038in}}%
\pgfpathlineto{\pgfqpoint{3.157535in}{2.325710in}}%
\pgfpathlineto{\pgfqpoint{3.158907in}{2.324859in}}%
\pgfpathlineto{\pgfqpoint{3.161162in}{2.319589in}}%
\pgfpathlineto{\pgfqpoint{3.164986in}{2.311781in}}%
\pgfpathlineto{\pgfqpoint{3.166652in}{2.312217in}}%
\pgfpathlineto{\pgfqpoint{3.169593in}{2.316630in}}%
\pgfpathlineto{\pgfqpoint{3.172437in}{2.318973in}}%
\pgfpathlineto{\pgfqpoint{3.174397in}{2.317654in}}%
\pgfpathlineto{\pgfqpoint{3.180868in}{2.311697in}}%
\pgfpathlineto{\pgfqpoint{3.184593in}{2.313588in}}%
\pgfpathlineto{\pgfqpoint{3.187927in}{2.314026in}}%
\pgfpathlineto{\pgfqpoint{3.200867in}{2.311867in}}%
\pgfpathlineto{\pgfqpoint{3.227338in}{2.312758in}}%
\pgfpathlineto{\pgfqpoint{3.232730in}{2.314379in}}%
\pgfpathlineto{\pgfqpoint{3.236749in}{2.312114in}}%
\pgfpathlineto{\pgfqpoint{3.239494in}{2.311925in}}%
\pgfpathlineto{\pgfqpoint{3.242043in}{2.314426in}}%
\pgfpathlineto{\pgfqpoint{3.246357in}{2.318490in}}%
\pgfpathlineto{\pgfqpoint{3.248318in}{2.317236in}}%
\pgfpathlineto{\pgfqpoint{3.254102in}{2.311999in}}%
\pgfpathlineto{\pgfqpoint{3.256063in}{2.314978in}}%
\pgfpathlineto{\pgfqpoint{3.261161in}{2.323384in}}%
\pgfpathlineto{\pgfqpoint{3.262631in}{2.321862in}}%
\pgfpathlineto{\pgfqpoint{3.268415in}{2.311786in}}%
\pgfpathlineto{\pgfqpoint{3.269102in}{2.312509in}}%
\pgfpathlineto{\pgfqpoint{3.271062in}{2.317487in}}%
\pgfpathlineto{\pgfqpoint{3.275474in}{2.328695in}}%
\pgfpathlineto{\pgfqpoint{3.276650in}{2.327782in}}%
\pgfpathlineto{\pgfqpoint{3.278611in}{2.322292in}}%
\pgfpathlineto{\pgfqpoint{3.282631in}{2.311652in}}%
\pgfpathlineto{\pgfqpoint{3.283709in}{2.312567in}}%
\pgfpathlineto{\pgfqpoint{3.285474in}{2.318075in}}%
\pgfpathlineto{\pgfqpoint{3.290278in}{2.333859in}}%
\pgfpathlineto{\pgfqpoint{3.291356in}{2.332583in}}%
\pgfpathlineto{\pgfqpoint{3.293317in}{2.325378in}}%
\pgfpathlineto{\pgfqpoint{3.297336in}{2.311655in}}%
\pgfpathlineto{\pgfqpoint{3.298317in}{2.312548in}}%
\pgfpathlineto{\pgfqpoint{3.299983in}{2.318527in}}%
\pgfpathlineto{\pgfqpoint{3.304983in}{2.338546in}}%
\pgfpathlineto{\pgfqpoint{3.305081in}{2.338514in}}%
\pgfpathlineto{\pgfqpoint{3.306160in}{2.336640in}}%
\pgfpathlineto{\pgfqpoint{3.308317in}{2.326447in}}%
\pgfpathlineto{\pgfqpoint{3.311944in}{2.311695in}}%
\pgfpathlineto{\pgfqpoint{3.312924in}{2.312468in}}%
\pgfpathlineto{\pgfqpoint{3.314493in}{2.318543in}}%
\pgfpathlineto{\pgfqpoint{3.319689in}{2.342542in}}%
\pgfpathlineto{\pgfqpoint{3.319885in}{2.342425in}}%
\pgfpathlineto{\pgfqpoint{3.320963in}{2.339912in}}%
\pgfpathlineto{\pgfqpoint{3.323316in}{2.326541in}}%
\pgfpathlineto{\pgfqpoint{3.326650in}{2.311702in}}%
\pgfpathlineto{\pgfqpoint{3.327532in}{2.312350in}}%
\pgfpathlineto{\pgfqpoint{3.329002in}{2.318198in}}%
\pgfpathlineto{\pgfqpoint{3.334296in}{2.345760in}}%
\pgfpathlineto{\pgfqpoint{3.334787in}{2.345325in}}%
\pgfpathlineto{\pgfqpoint{3.336061in}{2.340911in}}%
\pgfpathlineto{\pgfqpoint{3.341649in}{2.311663in}}%
\pgfpathlineto{\pgfqpoint{3.342434in}{2.312910in}}%
\pgfpathlineto{\pgfqpoint{3.344100in}{2.321297in}}%
\pgfpathlineto{\pgfqpoint{3.349002in}{2.348128in}}%
\pgfpathlineto{\pgfqpoint{3.349100in}{2.348090in}}%
\pgfpathlineto{\pgfqpoint{3.349982in}{2.346333in}}%
\pgfpathlineto{\pgfqpoint{3.351845in}{2.335696in}}%
\pgfpathlineto{\pgfqpoint{3.356159in}{2.311658in}}%
\pgfpathlineto{\pgfqpoint{3.356943in}{2.312477in}}%
\pgfpathlineto{\pgfqpoint{3.358414in}{2.319174in}}%
\pgfpathlineto{\pgfqpoint{3.363610in}{2.349710in}}%
\pgfpathlineto{\pgfqpoint{3.364100in}{2.349301in}}%
\pgfpathlineto{\pgfqpoint{3.365276in}{2.345092in}}%
\pgfpathlineto{\pgfqpoint{3.368316in}{2.322503in}}%
\pgfpathlineto{\pgfqpoint{3.370864in}{2.311655in}}%
\pgfpathlineto{\pgfqpoint{3.371649in}{2.312534in}}%
\pgfpathlineto{\pgfqpoint{3.373119in}{2.319480in}}%
\pgfpathlineto{\pgfqpoint{3.378315in}{2.350562in}}%
\pgfpathlineto{\pgfqpoint{3.378707in}{2.350256in}}%
\pgfpathlineto{\pgfqpoint{3.379884in}{2.346218in}}%
\pgfpathlineto{\pgfqpoint{3.382629in}{2.325610in}}%
\pgfpathlineto{\pgfqpoint{3.385472in}{2.311688in}}%
\pgfpathlineto{\pgfqpoint{3.386256in}{2.312352in}}%
\pgfpathlineto{\pgfqpoint{3.387629in}{2.318399in}}%
\pgfpathlineto{\pgfqpoint{3.393021in}{2.350790in}}%
\pgfpathlineto{\pgfqpoint{3.393609in}{2.350063in}}%
\pgfpathlineto{\pgfqpoint{3.394982in}{2.343967in}}%
\pgfpathlineto{\pgfqpoint{3.400374in}{2.311653in}}%
\pgfpathlineto{\pgfqpoint{3.400962in}{2.312397in}}%
\pgfpathlineto{\pgfqpoint{3.402335in}{2.318511in}}%
\pgfpathlineto{\pgfqpoint{3.407727in}{2.350508in}}%
\pgfpathlineto{\pgfqpoint{3.408217in}{2.349939in}}%
\pgfpathlineto{\pgfqpoint{3.409491in}{2.344755in}}%
\pgfpathlineto{\pgfqpoint{3.415079in}{2.311659in}}%
\pgfpathlineto{\pgfqpoint{3.415864in}{2.312966in}}%
\pgfpathlineto{\pgfqpoint{3.417530in}{2.321909in}}%
\pgfpathlineto{\pgfqpoint{3.422334in}{2.349861in}}%
\pgfpathlineto{\pgfqpoint{3.422432in}{2.349838in}}%
\pgfpathlineto{\pgfqpoint{3.423315in}{2.348150in}}%
\pgfpathlineto{\pgfqpoint{3.425079in}{2.338010in}}%
\pgfpathlineto{\pgfqpoint{3.429589in}{2.311658in}}%
\pgfpathlineto{\pgfqpoint{3.430373in}{2.312486in}}%
\pgfpathlineto{\pgfqpoint{3.431844in}{2.319200in}}%
\pgfpathlineto{\pgfqpoint{3.437040in}{2.348933in}}%
\pgfpathlineto{\pgfqpoint{3.437432in}{2.348608in}}%
\pgfpathlineto{\pgfqpoint{3.438608in}{2.344652in}}%
\pgfpathlineto{\pgfqpoint{3.441451in}{2.324152in}}%
\pgfpathlineto{\pgfqpoint{3.444197in}{2.311686in}}%
\pgfpathlineto{\pgfqpoint{3.445079in}{2.312530in}}%
\pgfpathlineto{\pgfqpoint{3.446549in}{2.319170in}}%
\pgfpathlineto{\pgfqpoint{3.451745in}{2.347853in}}%
\pgfpathlineto{\pgfqpoint{3.452040in}{2.347633in}}%
\pgfpathlineto{\pgfqpoint{3.453118in}{2.344481in}}%
\pgfpathlineto{\pgfqpoint{3.455569in}{2.327992in}}%
\pgfpathlineto{\pgfqpoint{3.458804in}{2.311728in}}%
\pgfpathlineto{\pgfqpoint{3.459588in}{2.312157in}}%
\pgfpathlineto{\pgfqpoint{3.459686in}{2.312349in}}%
\pgfpathlineto{\pgfqpoint{3.461059in}{2.317945in}}%
\pgfpathlineto{\pgfqpoint{3.466353in}{2.346745in}}%
\pgfpathlineto{\pgfqpoint{3.466941in}{2.346173in}}%
\pgfpathlineto{\pgfqpoint{3.468314in}{2.340868in}}%
\pgfpathlineto{\pgfqpoint{3.473804in}{2.311662in}}%
\pgfpathlineto{\pgfqpoint{3.474392in}{2.312387in}}%
\pgfpathlineto{\pgfqpoint{3.475863in}{2.318476in}}%
\pgfpathlineto{\pgfqpoint{3.481059in}{2.345679in}}%
\pgfpathlineto{\pgfqpoint{3.481451in}{2.345396in}}%
\pgfpathlineto{\pgfqpoint{3.482627in}{2.341824in}}%
\pgfpathlineto{\pgfqpoint{3.485470in}{2.323148in}}%
\pgfpathlineto{\pgfqpoint{3.488215in}{2.311690in}}%
\pgfpathlineto{\pgfqpoint{3.489098in}{2.312422in}}%
\pgfpathlineto{\pgfqpoint{3.490568in}{2.318431in}}%
\pgfpathlineto{\pgfqpoint{3.495764in}{2.344752in}}%
\pgfpathlineto{\pgfqpoint{3.496156in}{2.344450in}}%
\pgfpathlineto{\pgfqpoint{3.497333in}{2.340909in}}%
\pgfpathlineto{\pgfqpoint{3.500274in}{2.322045in}}%
\pgfpathlineto{\pgfqpoint{3.502921in}{2.311678in}}%
\pgfpathlineto{\pgfqpoint{3.503803in}{2.312452in}}%
\pgfpathlineto{\pgfqpoint{3.505274in}{2.318396in}}%
\pgfpathlineto{\pgfqpoint{3.510470in}{2.344039in}}%
\pgfpathlineto{\pgfqpoint{3.510764in}{2.343847in}}%
\pgfpathlineto{\pgfqpoint{3.511842in}{2.341048in}}%
\pgfpathlineto{\pgfqpoint{3.514293in}{2.326317in}}%
\pgfpathlineto{\pgfqpoint{3.517529in}{2.311721in}}%
\pgfpathlineto{\pgfqpoint{3.518411in}{2.312276in}}%
\pgfpathlineto{\pgfqpoint{3.519783in}{2.317318in}}%
\pgfpathlineto{\pgfqpoint{3.525176in}{2.343609in}}%
\pgfpathlineto{\pgfqpoint{3.525666in}{2.343144in}}%
\pgfpathlineto{\pgfqpoint{3.527038in}{2.338413in}}%
\pgfpathlineto{\pgfqpoint{3.532528in}{2.311656in}}%
\pgfpathlineto{\pgfqpoint{3.533215in}{2.312492in}}%
\pgfpathlineto{\pgfqpoint{3.534783in}{2.318940in}}%
\pgfpathlineto{\pgfqpoint{3.539881in}{2.343516in}}%
\pgfpathlineto{\pgfqpoint{3.540077in}{2.343415in}}%
\pgfpathlineto{\pgfqpoint{3.541156in}{2.340934in}}%
\pgfpathlineto{\pgfqpoint{3.543410in}{2.328049in}}%
\pgfpathlineto{\pgfqpoint{3.546842in}{2.311787in}}%
\pgfpathlineto{\pgfqpoint{3.547626in}{2.311973in}}%
\pgfpathlineto{\pgfqpoint{3.547822in}{2.312296in}}%
\pgfpathlineto{\pgfqpoint{3.549195in}{2.317362in}}%
\pgfpathlineto{\pgfqpoint{3.554587in}{2.343810in}}%
\pgfpathlineto{\pgfqpoint{3.555077in}{2.343363in}}%
\pgfpathlineto{\pgfqpoint{3.556351in}{2.339138in}}%
\pgfpathlineto{\pgfqpoint{3.561940in}{2.311655in}}%
\pgfpathlineto{\pgfqpoint{3.562724in}{2.312728in}}%
\pgfpathlineto{\pgfqpoint{3.564292in}{2.319643in}}%
\pgfpathlineto{\pgfqpoint{3.569292in}{2.344528in}}%
\pgfpathlineto{\pgfqpoint{3.569488in}{2.344443in}}%
\pgfpathlineto{\pgfqpoint{3.570469in}{2.342341in}}%
\pgfpathlineto{\pgfqpoint{3.572528in}{2.330854in}}%
\pgfpathlineto{\pgfqpoint{3.576351in}{2.311730in}}%
\pgfpathlineto{\pgfqpoint{3.577135in}{2.312103in}}%
\pgfpathlineto{\pgfqpoint{3.577233in}{2.312280in}}%
\pgfpathlineto{\pgfqpoint{3.578606in}{2.317502in}}%
\pgfpathlineto{\pgfqpoint{3.583998in}{2.345695in}}%
\pgfpathlineto{\pgfqpoint{3.584586in}{2.345131in}}%
\pgfpathlineto{\pgfqpoint{3.585959in}{2.339980in}}%
\pgfpathlineto{\pgfqpoint{3.591351in}{2.311648in}}%
\pgfpathlineto{\pgfqpoint{3.592037in}{2.312464in}}%
\pgfpathlineto{\pgfqpoint{3.593508in}{2.318758in}}%
\pgfpathlineto{\pgfqpoint{3.598802in}{2.347313in}}%
\pgfpathlineto{\pgfqpoint{3.599194in}{2.346964in}}%
\pgfpathlineto{\pgfqpoint{3.600370in}{2.343093in}}%
\pgfpathlineto{\pgfqpoint{3.603311in}{2.322686in}}%
\pgfpathlineto{\pgfqpoint{3.605958in}{2.311661in}}%
\pgfpathlineto{\pgfqpoint{3.606743in}{2.312435in}}%
\pgfpathlineto{\pgfqpoint{3.608115in}{2.318339in}}%
\pgfpathlineto{\pgfqpoint{3.613507in}{2.349405in}}%
\pgfpathlineto{\pgfqpoint{3.614096in}{2.348737in}}%
\pgfpathlineto{\pgfqpoint{3.615468in}{2.342942in}}%
\pgfpathlineto{\pgfqpoint{3.620860in}{2.311653in}}%
\pgfpathlineto{\pgfqpoint{3.621448in}{2.312399in}}%
\pgfpathlineto{\pgfqpoint{3.622821in}{2.318531in}}%
\pgfpathlineto{\pgfqpoint{3.628213in}{2.351916in}}%
\pgfpathlineto{\pgfqpoint{3.628899in}{2.351069in}}%
\pgfpathlineto{\pgfqpoint{3.630272in}{2.344654in}}%
\pgfpathlineto{\pgfqpoint{3.635566in}{2.311647in}}%
\pgfpathlineto{\pgfqpoint{3.636154in}{2.312359in}}%
\pgfpathlineto{\pgfqpoint{3.637526in}{2.318747in}}%
\pgfpathlineto{\pgfqpoint{3.643017in}{2.354778in}}%
\pgfpathlineto{\pgfqpoint{3.643703in}{2.353720in}}%
\pgfpathlineto{\pgfqpoint{3.645173in}{2.345875in}}%
\pgfpathlineto{\pgfqpoint{3.650271in}{2.311645in}}%
\pgfpathlineto{\pgfqpoint{3.650664in}{2.311938in}}%
\pgfpathlineto{\pgfqpoint{3.651742in}{2.315850in}}%
\pgfpathlineto{\pgfqpoint{3.654095in}{2.335600in}}%
\pgfpathlineto{\pgfqpoint{3.657428in}{2.357741in}}%
\pgfpathlineto{\pgfqpoint{3.657918in}{2.357811in}}%
\pgfpathlineto{\pgfqpoint{3.658310in}{2.357144in}}%
\pgfpathlineto{\pgfqpoint{3.659585in}{2.350865in}}%
\pgfpathlineto{\pgfqpoint{3.665075in}{2.311650in}}%
\pgfpathlineto{\pgfqpoint{3.665859in}{2.313141in}}%
\pgfpathlineto{\pgfqpoint{3.667428in}{2.323247in}}%
\pgfpathlineto{\pgfqpoint{3.672428in}{2.361139in}}%
\pgfpathlineto{\pgfqpoint{3.672722in}{2.360969in}}%
\pgfpathlineto{\pgfqpoint{3.673702in}{2.357669in}}%
\pgfpathlineto{\pgfqpoint{3.675761in}{2.340187in}}%
\pgfpathlineto{\pgfqpoint{3.679585in}{2.311722in}}%
\pgfpathlineto{\pgfqpoint{3.680075in}{2.311864in}}%
\pgfpathlineto{\pgfqpoint{3.680369in}{2.312482in}}%
\pgfpathlineto{\pgfqpoint{3.681643in}{2.319452in}}%
\pgfpathlineto{\pgfqpoint{3.687231in}{2.364307in}}%
\pgfpathlineto{\pgfqpoint{3.688016in}{2.362653in}}%
\pgfpathlineto{\pgfqpoint{3.689584in}{2.351595in}}%
\pgfpathlineto{\pgfqpoint{3.694486in}{2.311645in}}%
\pgfpathlineto{\pgfqpoint{3.694682in}{2.311723in}}%
\pgfpathlineto{\pgfqpoint{3.695565in}{2.314383in}}%
\pgfpathlineto{\pgfqpoint{3.697329in}{2.329227in}}%
\pgfpathlineto{\pgfqpoint{3.701839in}{2.367221in}}%
\pgfpathlineto{\pgfqpoint{3.702525in}{2.366266in}}%
\pgfpathlineto{\pgfqpoint{3.703800in}{2.358623in}}%
\pgfpathlineto{\pgfqpoint{3.709290in}{2.311652in}}%
\pgfpathlineto{\pgfqpoint{3.709976in}{2.313061in}}%
\pgfpathlineto{\pgfqpoint{3.711447in}{2.323481in}}%
\pgfpathlineto{\pgfqpoint{3.716643in}{2.369703in}}%
\pgfpathlineto{\pgfqpoint{3.717035in}{2.369262in}}%
\pgfpathlineto{\pgfqpoint{3.718113in}{2.364037in}}%
\pgfpathlineto{\pgfqpoint{3.720564in}{2.337390in}}%
\pgfpathlineto{\pgfqpoint{3.723701in}{2.311822in}}%
\pgfpathlineto{\pgfqpoint{3.724094in}{2.311692in}}%
\pgfpathlineto{\pgfqpoint{3.724486in}{2.312387in}}%
\pgfpathlineto{\pgfqpoint{3.725662in}{2.319134in}}%
\pgfpathlineto{\pgfqpoint{3.728799in}{2.355635in}}%
\pgfpathlineto{\pgfqpoint{3.731250in}{2.371540in}}%
\pgfpathlineto{\pgfqpoint{3.731937in}{2.370589in}}%
\pgfpathlineto{\pgfqpoint{3.733211in}{2.362474in}}%
\pgfpathlineto{\pgfqpoint{3.738701in}{2.311645in}}%
\pgfpathlineto{\pgfqpoint{3.739485in}{2.313424in}}%
\pgfpathlineto{\pgfqpoint{3.740956in}{2.324902in}}%
\pgfpathlineto{\pgfqpoint{3.746054in}{2.372680in}}%
\pgfpathlineto{\pgfqpoint{3.746348in}{2.372425in}}%
\pgfpathlineto{\pgfqpoint{3.747328in}{2.368208in}}%
\pgfpathlineto{\pgfqpoint{3.749485in}{2.345229in}}%
\pgfpathlineto{\pgfqpoint{3.753113in}{2.311862in}}%
\pgfpathlineto{\pgfqpoint{3.753505in}{2.311678in}}%
\pgfpathlineto{\pgfqpoint{3.753897in}{2.312351in}}%
\pgfpathlineto{\pgfqpoint{3.755073in}{2.319209in}}%
\pgfpathlineto{\pgfqpoint{3.758211in}{2.356679in}}%
\pgfpathlineto{\pgfqpoint{3.760662in}{2.372978in}}%
\pgfpathlineto{\pgfqpoint{3.761348in}{2.371992in}}%
\pgfpathlineto{\pgfqpoint{3.762622in}{2.363662in}}%
\pgfpathlineto{\pgfqpoint{3.768112in}{2.311644in}}%
\pgfpathlineto{\pgfqpoint{3.768897in}{2.313392in}}%
\pgfpathlineto{\pgfqpoint{3.770367in}{2.324881in}}%
\pgfpathlineto{\pgfqpoint{3.775465in}{2.372515in}}%
\pgfpathlineto{\pgfqpoint{3.775759in}{2.372239in}}%
\pgfpathlineto{\pgfqpoint{3.776740in}{2.367964in}}%
\pgfpathlineto{\pgfqpoint{3.778897in}{2.344986in}}%
\pgfpathlineto{\pgfqpoint{3.782524in}{2.311861in}}%
\pgfpathlineto{\pgfqpoint{3.782916in}{2.311677in}}%
\pgfpathlineto{\pgfqpoint{3.783308in}{2.312337in}}%
\pgfpathlineto{\pgfqpoint{3.784485in}{2.319070in}}%
\pgfpathlineto{\pgfqpoint{3.787720in}{2.356725in}}%
\pgfpathlineto{\pgfqpoint{3.790171in}{2.371285in}}%
\pgfpathlineto{\pgfqpoint{3.790857in}{2.369882in}}%
\pgfpathlineto{\pgfqpoint{3.792328in}{2.359138in}}%
\pgfpathlineto{\pgfqpoint{3.797524in}{2.311645in}}%
\pgfpathlineto{\pgfqpoint{3.797916in}{2.312096in}}%
\pgfpathlineto{\pgfqpoint{3.798994in}{2.317386in}}%
\pgfpathlineto{\pgfqpoint{3.801543in}{2.345281in}}%
\pgfpathlineto{\pgfqpoint{3.804582in}{2.369252in}}%
\pgfpathlineto{\pgfqpoint{3.804975in}{2.369347in}}%
\pgfpathlineto{\pgfqpoint{3.805367in}{2.368634in}}%
\pgfpathlineto{\pgfqpoint{3.806641in}{2.361132in}}%
\pgfpathlineto{\pgfqpoint{3.812229in}{2.311646in}}%
\pgfpathlineto{\pgfqpoint{3.813112in}{2.313763in}}%
\pgfpathlineto{\pgfqpoint{3.814778in}{2.327057in}}%
\pgfpathlineto{\pgfqpoint{3.819484in}{2.366974in}}%
\pgfpathlineto{\pgfqpoint{3.819582in}{2.366962in}}%
\pgfpathlineto{\pgfqpoint{3.820268in}{2.365533in}}%
\pgfpathlineto{\pgfqpoint{3.821739in}{2.355348in}}%
\pgfpathlineto{\pgfqpoint{3.826935in}{2.311649in}}%
\pgfpathlineto{\pgfqpoint{3.827229in}{2.311925in}}%
\pgfpathlineto{\pgfqpoint{3.828209in}{2.315800in}}%
\pgfpathlineto{\pgfqpoint{3.830464in}{2.337131in}}%
\pgfpathlineto{\pgfqpoint{3.833896in}{2.363891in}}%
\pgfpathlineto{\pgfqpoint{3.834386in}{2.364011in}}%
\pgfpathlineto{\pgfqpoint{3.834778in}{2.363281in}}%
\pgfpathlineto{\pgfqpoint{3.836052in}{2.356237in}}%
\pgfpathlineto{\pgfqpoint{3.841641in}{2.311653in}}%
\pgfpathlineto{\pgfqpoint{3.842425in}{2.313288in}}%
\pgfpathlineto{\pgfqpoint{3.843993in}{2.323927in}}%
\pgfpathlineto{\pgfqpoint{3.848895in}{2.360880in}}%
\pgfpathlineto{\pgfqpoint{3.848993in}{2.360847in}}%
\pgfpathlineto{\pgfqpoint{3.849778in}{2.359050in}}%
\pgfpathlineto{\pgfqpoint{3.851444in}{2.347439in}}%
\pgfpathlineto{\pgfqpoint{3.856248in}{2.311645in}}%
\pgfpathlineto{\pgfqpoint{3.856346in}{2.311660in}}%
\pgfpathlineto{\pgfqpoint{3.857130in}{2.313258in}}%
\pgfpathlineto{\pgfqpoint{3.858699in}{2.323290in}}%
\pgfpathlineto{\pgfqpoint{3.863601in}{2.357398in}}%
\pgfpathlineto{\pgfqpoint{3.863699in}{2.357355in}}%
\pgfpathlineto{\pgfqpoint{3.864581in}{2.355192in}}%
\pgfpathlineto{\pgfqpoint{3.866346in}{2.342894in}}%
\pgfpathlineto{\pgfqpoint{3.870856in}{2.311660in}}%
\pgfpathlineto{\pgfqpoint{3.871640in}{2.312614in}}%
\pgfpathlineto{\pgfqpoint{3.873111in}{2.320343in}}%
\pgfpathlineto{\pgfqpoint{3.878209in}{2.353761in}}%
\pgfpathlineto{\pgfqpoint{3.878601in}{2.353475in}}%
\pgfpathlineto{\pgfqpoint{3.879679in}{2.349768in}}%
\pgfpathlineto{\pgfqpoint{3.882130in}{2.330623in}}%
\pgfpathlineto{\pgfqpoint{3.885365in}{2.311763in}}%
\pgfpathlineto{\pgfqpoint{3.885953in}{2.311853in}}%
\pgfpathlineto{\pgfqpoint{3.886248in}{2.312370in}}%
\pgfpathlineto{\pgfqpoint{3.887620in}{2.318515in}}%
\pgfpathlineto{\pgfqpoint{3.892914in}{2.350043in}}%
\pgfpathlineto{\pgfqpoint{3.893404in}{2.349566in}}%
\pgfpathlineto{\pgfqpoint{3.894679in}{2.344650in}}%
\pgfpathlineto{\pgfqpoint{3.900365in}{2.311656in}}%
\pgfpathlineto{\pgfqpoint{3.901149in}{2.312866in}}%
\pgfpathlineto{\pgfqpoint{3.902816in}{2.321177in}}%
\pgfpathlineto{\pgfqpoint{3.907620in}{2.346346in}}%
\pgfpathlineto{\pgfqpoint{3.908502in}{2.344916in}}%
\pgfpathlineto{\pgfqpoint{3.910267in}{2.335884in}}%
\pgfpathlineto{\pgfqpoint{3.914875in}{2.311656in}}%
\pgfpathlineto{\pgfqpoint{3.915757in}{2.312603in}}%
\pgfpathlineto{\pgfqpoint{3.917326in}{2.319294in}}%
\pgfpathlineto{\pgfqpoint{3.922227in}{2.342794in}}%
\pgfpathlineto{\pgfqpoint{3.922423in}{2.342725in}}%
\pgfpathlineto{\pgfqpoint{3.923404in}{2.340783in}}%
\pgfpathlineto{\pgfqpoint{3.925463in}{2.330000in}}%
\pgfpathlineto{\pgfqpoint{3.929384in}{2.311719in}}%
\pgfpathlineto{\pgfqpoint{3.930364in}{2.312394in}}%
\pgfpathlineto{\pgfqpoint{3.931933in}{2.318212in}}%
\pgfpathlineto{\pgfqpoint{3.936933in}{2.339457in}}%
\pgfpathlineto{\pgfqpoint{3.937031in}{2.339427in}}%
\pgfpathlineto{\pgfqpoint{3.938011in}{2.337807in}}%
\pgfpathlineto{\pgfqpoint{3.939972in}{2.328886in}}%
\pgfpathlineto{\pgfqpoint{3.943992in}{2.311742in}}%
\pgfpathlineto{\pgfqpoint{3.945070in}{2.312403in}}%
\pgfpathlineto{\pgfqpoint{3.946639in}{2.317730in}}%
\pgfpathlineto{\pgfqpoint{3.951639in}{2.336438in}}%
\pgfpathlineto{\pgfqpoint{3.951737in}{2.336400in}}%
\pgfpathlineto{\pgfqpoint{3.952815in}{2.334584in}}%
\pgfpathlineto{\pgfqpoint{3.955070in}{2.324717in}}%
\pgfpathlineto{\pgfqpoint{3.958599in}{2.311763in}}%
\pgfpathlineto{\pgfqpoint{3.959678in}{2.312247in}}%
\pgfpathlineto{\pgfqpoint{3.961246in}{2.316892in}}%
\pgfpathlineto{\pgfqpoint{3.966246in}{2.333840in}}%
\pgfpathlineto{\pgfqpoint{3.966442in}{2.333772in}}%
\pgfpathlineto{\pgfqpoint{3.967619in}{2.331786in}}%
\pgfpathlineto{\pgfqpoint{3.970168in}{2.321268in}}%
\pgfpathlineto{\pgfqpoint{3.973305in}{2.311720in}}%
\pgfpathlineto{\pgfqpoint{3.974383in}{2.312265in}}%
\pgfpathlineto{\pgfqpoint{3.976050in}{2.316937in}}%
\pgfpathlineto{\pgfqpoint{3.980952in}{2.331687in}}%
\pgfpathlineto{\pgfqpoint{3.982128in}{2.330202in}}%
\pgfpathlineto{\pgfqpoint{3.984383in}{2.322233in}}%
\pgfpathlineto{\pgfqpoint{3.987913in}{2.311738in}}%
\pgfpathlineto{\pgfqpoint{3.989089in}{2.312283in}}%
\pgfpathlineto{\pgfqpoint{3.990854in}{2.317006in}}%
\pgfpathlineto{\pgfqpoint{3.995559in}{2.330066in}}%
\pgfpathlineto{\pgfqpoint{3.996736in}{2.328824in}}%
\pgfpathlineto{\pgfqpoint{3.998893in}{2.322004in}}%
\pgfpathlineto{\pgfqpoint{4.002618in}{2.311710in}}%
\pgfpathlineto{\pgfqpoint{4.003795in}{2.312302in}}%
\pgfpathlineto{\pgfqpoint{4.005559in}{2.316799in}}%
\pgfpathlineto{\pgfqpoint{4.010167in}{2.329000in}}%
\pgfpathlineto{\pgfqpoint{4.011343in}{2.327974in}}%
\pgfpathlineto{\pgfqpoint{4.013402in}{2.322050in}}%
\pgfpathlineto{\pgfqpoint{4.017324in}{2.311693in}}%
\pgfpathlineto{\pgfqpoint{4.018598in}{2.312468in}}%
\pgfpathlineto{\pgfqpoint{4.020559in}{2.317713in}}%
\pgfpathlineto{\pgfqpoint{4.024775in}{2.328538in}}%
\pgfpathlineto{\pgfqpoint{4.025951in}{2.327713in}}%
\pgfpathlineto{\pgfqpoint{4.027912in}{2.322468in}}%
\pgfpathlineto{\pgfqpoint{4.032127in}{2.311660in}}%
\pgfpathlineto{\pgfqpoint{4.033304in}{2.312491in}}%
\pgfpathlineto{\pgfqpoint{4.035265in}{2.317773in}}%
\pgfpathlineto{\pgfqpoint{4.039480in}{2.328769in}}%
\pgfpathlineto{\pgfqpoint{4.040657in}{2.327975in}}%
\pgfpathlineto{\pgfqpoint{4.042617in}{2.322700in}}%
\pgfpathlineto{\pgfqpoint{4.046833in}{2.311661in}}%
\pgfpathlineto{\pgfqpoint{4.048009in}{2.312520in}}%
\pgfpathlineto{\pgfqpoint{4.049970in}{2.318034in}}%
\pgfpathlineto{\pgfqpoint{4.054284in}{2.329806in}}%
\pgfpathlineto{\pgfqpoint{4.055460in}{2.328865in}}%
\pgfpathlineto{\pgfqpoint{4.057421in}{2.323159in}}%
\pgfpathlineto{\pgfqpoint{4.061539in}{2.311666in}}%
\pgfpathlineto{\pgfqpoint{4.062715in}{2.312557in}}%
\pgfpathlineto{\pgfqpoint{4.064578in}{2.318142in}}%
\pgfpathlineto{\pgfqpoint{4.069088in}{2.331772in}}%
\pgfpathlineto{\pgfqpoint{4.070166in}{2.330841in}}%
\pgfpathlineto{\pgfqpoint{4.072029in}{2.325057in}}%
\pgfpathlineto{\pgfqpoint{4.076342in}{2.311655in}}%
\pgfpathlineto{\pgfqpoint{4.077421in}{2.312602in}}%
\pgfpathlineto{\pgfqpoint{4.079185in}{2.318428in}}%
\pgfpathlineto{\pgfqpoint{4.083989in}{2.334862in}}%
\pgfpathlineto{\pgfqpoint{4.085068in}{2.333518in}}%
\pgfpathlineto{\pgfqpoint{4.087028in}{2.326048in}}%
\pgfpathlineto{\pgfqpoint{4.091048in}{2.311669in}}%
\pgfpathlineto{\pgfqpoint{4.092028in}{2.312458in}}%
\pgfpathlineto{\pgfqpoint{4.093597in}{2.317910in}}%
\pgfpathlineto{\pgfqpoint{4.098793in}{2.339310in}}%
\pgfpathlineto{\pgfqpoint{4.099087in}{2.339139in}}%
\pgfpathlineto{\pgfqpoint{4.100263in}{2.336378in}}%
\pgfpathlineto{\pgfqpoint{4.102910in}{2.322262in}}%
\pgfpathlineto{\pgfqpoint{4.105852in}{2.311660in}}%
\pgfpathlineto{\pgfqpoint{4.106734in}{2.312497in}}%
\pgfpathlineto{\pgfqpoint{4.108205in}{2.318340in}}%
\pgfpathlineto{\pgfqpoint{4.113499in}{2.345387in}}%
\pgfpathlineto{\pgfqpoint{4.113989in}{2.345002in}}%
\pgfpathlineto{\pgfqpoint{4.115263in}{2.340748in}}%
\pgfpathlineto{\pgfqpoint{4.120753in}{2.311649in}}%
\pgfpathlineto{\pgfqpoint{4.121734in}{2.313406in}}%
\pgfpathlineto{\pgfqpoint{4.123498in}{2.323790in}}%
\pgfpathlineto{\pgfqpoint{4.128302in}{2.353298in}}%
\pgfpathlineto{\pgfqpoint{4.129087in}{2.352005in}}%
\pgfpathlineto{\pgfqpoint{4.130655in}{2.343229in}}%
\pgfpathlineto{\pgfqpoint{4.135459in}{2.311646in}}%
\pgfpathlineto{\pgfqpoint{4.135753in}{2.311806in}}%
\pgfpathlineto{\pgfqpoint{4.136733in}{2.314943in}}%
\pgfpathlineto{\pgfqpoint{4.138694in}{2.331062in}}%
\pgfpathlineto{\pgfqpoint{4.142812in}{2.363109in}}%
\pgfpathlineto{\pgfqpoint{4.143302in}{2.363008in}}%
\pgfpathlineto{\pgfqpoint{4.143596in}{2.362410in}}%
\pgfpathlineto{\pgfqpoint{4.144871in}{2.355478in}}%
\pgfpathlineto{\pgfqpoint{4.150263in}{2.311646in}}%
\pgfpathlineto{\pgfqpoint{4.151145in}{2.313801in}}%
\pgfpathlineto{\pgfqpoint{4.152714in}{2.326683in}}%
\pgfpathlineto{\pgfqpoint{4.157811in}{2.375055in}}%
\pgfpathlineto{\pgfqpoint{4.158008in}{2.374886in}}%
\pgfpathlineto{\pgfqpoint{4.158890in}{2.371435in}}%
\pgfpathlineto{\pgfqpoint{4.160753in}{2.352295in}}%
\pgfpathlineto{\pgfqpoint{4.164870in}{2.311683in}}%
\pgfpathlineto{\pgfqpoint{4.165360in}{2.312093in}}%
\pgfpathlineto{\pgfqpoint{4.165458in}{2.312359in}}%
\pgfpathlineto{\pgfqpoint{4.166635in}{2.320123in}}%
\pgfpathlineto{\pgfqpoint{4.169282in}{2.358135in}}%
\pgfpathlineto{\pgfqpoint{4.172321in}{2.388483in}}%
\pgfpathlineto{\pgfqpoint{4.172615in}{2.388538in}}%
\pgfpathlineto{\pgfqpoint{4.172909in}{2.387987in}}%
\pgfpathlineto{\pgfqpoint{4.172909in}{2.387987in}}%
\pgfpathlineto{\pgfqpoint{4.173988in}{2.381001in}}%
\pgfpathlineto{\pgfqpoint{4.176439in}{2.345281in}}%
\pgfpathlineto{\pgfqpoint{4.179576in}{2.311730in}}%
\pgfpathlineto{\pgfqpoint{4.179870in}{2.311717in}}%
\pgfpathlineto{\pgfqpoint{4.180164in}{2.312367in}}%
\pgfpathlineto{\pgfqpoint{4.181242in}{2.320245in}}%
\pgfpathlineto{\pgfqpoint{4.183595in}{2.359007in}}%
\pgfpathlineto{\pgfqpoint{4.186929in}{2.402790in}}%
\pgfpathlineto{\pgfqpoint{4.187321in}{2.403098in}}%
\pgfpathlineto{\pgfqpoint{4.187615in}{2.402488in}}%
\pgfpathlineto{\pgfqpoint{4.187615in}{2.402488in}}%
\pgfpathlineto{\pgfqpoint{4.188693in}{2.394335in}}%
\pgfpathlineto{\pgfqpoint{4.191144in}{2.352062in}}%
\pgfpathlineto{\pgfqpoint{4.194281in}{2.311794in}}%
\pgfpathlineto{\pgfqpoint{4.194576in}{2.311692in}}%
\pgfpathlineto{\pgfqpoint{4.194772in}{2.312056in}}%
\pgfpathlineto{\pgfqpoint{4.194772in}{2.312056in}}%
\pgfpathlineto{\pgfqpoint{4.195654in}{2.317904in}}%
\pgfpathlineto{\pgfqpoint{4.197615in}{2.351043in}}%
\pgfpathlineto{\pgfqpoint{4.201732in}{2.417495in}}%
\pgfpathlineto{\pgfqpoint{4.202026in}{2.417649in}}%
\pgfpathlineto{\pgfqpoint{4.202222in}{2.417287in}}%
\pgfpathlineto{\pgfqpoint{4.202222in}{2.417287in}}%
\pgfpathlineto{\pgfqpoint{4.203105in}{2.411177in}}%
\pgfpathlineto{\pgfqpoint{4.205066in}{2.376618in}}%
\pgfpathlineto{\pgfqpoint{4.208987in}{2.311871in}}%
\pgfpathlineto{\pgfqpoint{4.209281in}{2.311671in}}%
\pgfpathlineto{\pgfqpoint{4.209477in}{2.312032in}}%
\pgfpathlineto{\pgfqpoint{4.209477in}{2.312032in}}%
\pgfpathlineto{\pgfqpoint{4.210360in}{2.318459in}}%
\pgfpathlineto{\pgfqpoint{4.212222in}{2.353341in}}%
\pgfpathlineto{\pgfqpoint{4.216536in}{2.430781in}}%
\pgfpathlineto{\pgfqpoint{4.216830in}{2.430643in}}%
\pgfpathlineto{\pgfqpoint{4.217026in}{2.430029in}}%
\pgfpathlineto{\pgfqpoint{4.218007in}{2.420941in}}%
\pgfpathlineto{\pgfqpoint{4.220261in}{2.372296in}}%
\pgfpathlineto{\pgfqpoint{4.223693in}{2.311952in}}%
\pgfpathlineto{\pgfqpoint{4.223987in}{2.311656in}}%
\pgfpathlineto{\pgfqpoint{4.224183in}{2.312003in}}%
\pgfpathlineto{\pgfqpoint{4.224183in}{2.312003in}}%
\pgfpathlineto{\pgfqpoint{4.225065in}{2.318870in}}%
\pgfpathlineto{\pgfqpoint{4.226928in}{2.356810in}}%
\pgfpathlineto{\pgfqpoint{4.231144in}{2.440725in}}%
\pgfpathlineto{\pgfqpoint{4.231340in}{2.440948in}}%
\pgfpathlineto{\pgfqpoint{4.231536in}{2.440716in}}%
\pgfpathlineto{\pgfqpoint{4.231536in}{2.440716in}}%
\pgfpathlineto{\pgfqpoint{4.232320in}{2.435330in}}%
\pgfpathlineto{\pgfqpoint{4.233987in}{2.403665in}}%
\pgfpathlineto{\pgfqpoint{4.238692in}{2.311645in}}%
\pgfpathlineto{\pgfqpoint{4.239183in}{2.313325in}}%
\pgfpathlineto{\pgfqpoint{4.240359in}{2.328533in}}%
\pgfpathlineto{\pgfqpoint{4.243496in}{2.410679in}}%
\pgfpathlineto{\pgfqpoint{4.245947in}{2.446325in}}%
\pgfpathlineto{\pgfqpoint{4.246241in}{2.446087in}}%
\pgfpathlineto{\pgfqpoint{4.246437in}{2.445339in}}%
\pgfpathlineto{\pgfqpoint{4.247516in}{2.433202in}}%
\pgfpathlineto{\pgfqpoint{4.249967in}{2.371389in}}%
\pgfpathlineto{\pgfqpoint{4.253104in}{2.312084in}}%
\pgfpathlineto{\pgfqpoint{4.253398in}{2.311639in}}%
\pgfpathlineto{\pgfqpoint{4.253692in}{2.312257in}}%
\pgfpathlineto{\pgfqpoint{4.253692in}{2.312257in}}%
\pgfpathlineto{\pgfqpoint{4.254673in}{2.321753in}}%
\pgfpathlineto{\pgfqpoint{4.256829in}{2.372644in}}%
\pgfpathlineto{\pgfqpoint{4.260457in}{2.445374in}}%
\pgfpathlineto{\pgfqpoint{4.260751in}{2.445763in}}%
\pgfpathlineto{\pgfqpoint{4.260947in}{2.445434in}}%
\pgfpathlineto{\pgfqpoint{4.260947in}{2.445434in}}%
\pgfpathlineto{\pgfqpoint{4.261731in}{2.439510in}}%
\pgfpathlineto{\pgfqpoint{4.263496in}{2.403707in}}%
\pgfpathlineto{\pgfqpoint{4.268006in}{2.311681in}}%
\pgfpathlineto{\pgfqpoint{4.268300in}{2.311893in}}%
\pgfpathlineto{\pgfqpoint{4.268496in}{2.312606in}}%
\pgfpathlineto{\pgfqpoint{4.269574in}{2.324274in}}%
\pgfpathlineto{\pgfqpoint{4.272123in}{2.385845in}}%
\pgfpathlineto{\pgfqpoint{4.275162in}{2.438240in}}%
\pgfpathlineto{\pgfqpoint{4.275456in}{2.438510in}}%
\pgfpathlineto{\pgfqpoint{4.275653in}{2.438133in}}%
\pgfpathlineto{\pgfqpoint{4.275653in}{2.438133in}}%
\pgfpathlineto{\pgfqpoint{4.276535in}{2.431081in}}%
\pgfpathlineto{\pgfqpoint{4.278496in}{2.390561in}}%
\pgfpathlineto{\pgfqpoint{4.282515in}{2.312110in}}%
\pgfpathlineto{\pgfqpoint{4.282809in}{2.311637in}}%
\pgfpathlineto{\pgfqpoint{4.283103in}{2.312115in}}%
\pgfpathlineto{\pgfqpoint{4.283103in}{2.312115in}}%
\pgfpathlineto{\pgfqpoint{4.283986in}{2.319046in}}%
\pgfpathlineto{\pgfqpoint{4.285947in}{2.357063in}}%
\pgfpathlineto{\pgfqpoint{4.289770in}{2.424580in}}%
\pgfpathlineto{\pgfqpoint{4.290064in}{2.425011in}}%
\pgfpathlineto{\pgfqpoint{4.290358in}{2.424546in}}%
\pgfpathlineto{\pgfqpoint{4.290358in}{2.424546in}}%
\pgfpathlineto{\pgfqpoint{4.291241in}{2.417940in}}%
\pgfpathlineto{\pgfqpoint{4.293201in}{2.381439in}}%
\pgfpathlineto{\pgfqpoint{4.297221in}{2.312072in}}%
\pgfpathlineto{\pgfqpoint{4.297613in}{2.311678in}}%
\pgfpathlineto{\pgfqpoint{4.297907in}{2.312341in}}%
\pgfpathlineto{\pgfqpoint{4.297907in}{2.312341in}}%
\pgfpathlineto{\pgfqpoint{4.298985in}{2.321346in}}%
\pgfpathlineto{\pgfqpoint{4.301633in}{2.370376in}}%
\pgfpathlineto{\pgfqpoint{4.304476in}{2.406363in}}%
\pgfpathlineto{\pgfqpoint{4.304770in}{2.406614in}}%
\pgfpathlineto{\pgfqpoint{4.305064in}{2.406114in}}%
\pgfpathlineto{\pgfqpoint{4.305064in}{2.406114in}}%
\pgfpathlineto{\pgfqpoint{4.306044in}{2.399249in}}%
\pgfpathlineto{\pgfqpoint{4.308299in}{2.361646in}}%
\pgfpathlineto{\pgfqpoint{4.311828in}{2.312279in}}%
\pgfpathlineto{\pgfqpoint{4.312319in}{2.311668in}}%
\pgfpathlineto{\pgfqpoint{4.312711in}{2.312505in}}%
\pgfpathlineto{\pgfqpoint{4.313887in}{2.321457in}}%
\pgfpathlineto{\pgfqpoint{4.319377in}{2.385676in}}%
\pgfpathlineto{\pgfqpoint{4.320260in}{2.383067in}}%
\pgfpathlineto{\pgfqpoint{4.321926in}{2.365993in}}%
\pgfpathlineto{\pgfqpoint{4.326926in}{2.311644in}}%
\pgfpathlineto{\pgfqpoint{4.327514in}{2.312561in}}%
\pgfpathlineto{\pgfqpoint{4.328789in}{2.320568in}}%
\pgfpathlineto{\pgfqpoint{4.334083in}{2.364838in}}%
\pgfpathlineto{\pgfqpoint{4.334573in}{2.364110in}}%
\pgfpathlineto{\pgfqpoint{4.335848in}{2.357215in}}%
\pgfpathlineto{\pgfqpoint{4.341730in}{2.311656in}}%
\pgfpathlineto{\pgfqpoint{4.342416in}{2.312756in}}%
\pgfpathlineto{\pgfqpoint{4.343985in}{2.320879in}}%
\pgfpathlineto{\pgfqpoint{4.348691in}{2.346533in}}%
\pgfpathlineto{\pgfqpoint{4.349573in}{2.345223in}}%
\pgfpathlineto{\pgfqpoint{4.351239in}{2.337120in}}%
\pgfpathlineto{\pgfqpoint{4.356239in}{2.311666in}}%
\pgfpathlineto{\pgfqpoint{4.357220in}{2.312511in}}%
\pgfpathlineto{\pgfqpoint{4.358984in}{2.318563in}}%
\pgfpathlineto{\pgfqpoint{4.363102in}{2.332251in}}%
\pgfpathlineto{\pgfqpoint{4.364180in}{2.331540in}}%
\pgfpathlineto{\pgfqpoint{4.365945in}{2.326431in}}%
\pgfpathlineto{\pgfqpoint{4.370847in}{2.311684in}}%
\pgfpathlineto{\pgfqpoint{4.372220in}{2.312489in}}%
\pgfpathlineto{\pgfqpoint{4.374768in}{2.318072in}}%
\pgfpathlineto{\pgfqpoint{4.377808in}{2.322509in}}%
\pgfpathlineto{\pgfqpoint{4.379376in}{2.321479in}}%
\pgfpathlineto{\pgfqpoint{4.382513in}{2.315269in}}%
\pgfpathlineto{\pgfqpoint{4.385455in}{2.311686in}}%
\pgfpathlineto{\pgfqpoint{4.387317in}{2.312357in}}%
\pgfpathlineto{\pgfqpoint{4.393592in}{2.316402in}}%
\pgfpathlineto{\pgfqpoint{4.396925in}{2.313523in}}%
\pgfpathlineto{\pgfqpoint{4.400454in}{2.311650in}}%
\pgfpathlineto{\pgfqpoint{4.403886in}{2.312768in}}%
\pgfpathlineto{\pgfqpoint{4.408101in}{2.313557in}}%
\pgfpathlineto{\pgfqpoint{4.420944in}{2.312305in}}%
\pgfpathlineto{\pgfqpoint{4.435062in}{2.311825in}}%
\pgfpathlineto{\pgfqpoint{4.479963in}{2.311651in}}%
\pgfpathlineto{\pgfqpoint{6.830902in}{2.311650in}}%
\pgfpathlineto{\pgfqpoint{6.830902in}{2.311650in}}%
\pgfusepath{stroke}%
\end{pgfscope}%
\begin{pgfscope}%
\pgfsetrectcap%
\pgfsetmiterjoin%
\pgfsetlinewidth{0.803000pt}%
\definecolor{currentstroke}{rgb}{0.000000,0.000000,0.000000}%
\pgfsetstrokecolor{currentstroke}%
\pgfsetdash{}{0pt}%
\pgfpathmoveto{\pgfqpoint{0.948750in}{0.513700in}}%
\pgfpathlineto{\pgfqpoint{0.948750in}{4.109600in}}%
\pgfusepath{stroke}%
\end{pgfscope}%
\begin{pgfscope}%
\pgfsetrectcap%
\pgfsetmiterjoin%
\pgfsetlinewidth{0.803000pt}%
\definecolor{currentstroke}{rgb}{0.000000,0.000000,0.000000}%
\pgfsetstrokecolor{currentstroke}%
\pgfsetdash{}{0pt}%
\pgfpathmoveto{\pgfqpoint{6.831000in}{0.513700in}}%
\pgfpathlineto{\pgfqpoint{6.831000in}{4.109600in}}%
\pgfusepath{stroke}%
\end{pgfscope}%
\begin{pgfscope}%
\pgfsetrectcap%
\pgfsetmiterjoin%
\pgfsetlinewidth{0.803000pt}%
\definecolor{currentstroke}{rgb}{0.000000,0.000000,0.000000}%
\pgfsetstrokecolor{currentstroke}%
\pgfsetdash{}{0pt}%
\pgfpathmoveto{\pgfqpoint{0.948750in}{0.513700in}}%
\pgfpathlineto{\pgfqpoint{6.831000in}{0.513700in}}%
\pgfusepath{stroke}%
\end{pgfscope}%
\begin{pgfscope}%
\pgfsetrectcap%
\pgfsetmiterjoin%
\pgfsetlinewidth{0.803000pt}%
\definecolor{currentstroke}{rgb}{0.000000,0.000000,0.000000}%
\pgfsetstrokecolor{currentstroke}%
\pgfsetdash{}{0pt}%
\pgfpathmoveto{\pgfqpoint{0.948750in}{4.109600in}}%
\pgfpathlineto{\pgfqpoint{6.831000in}{4.109600in}}%
\pgfusepath{stroke}%
\end{pgfscope}%
\begin{pgfscope}%
\pgftext[x=0.948750in,y=4.289395in,left,base]{\sffamily\fontsize{10.000000}{12.000000}\selectfont Iterations: 42785, Time: 0.301 ps, RXPWR: 96.4 percent, TXPWR: 3.5 percent}%
\end{pgfscope}%
\end{pgfpicture}%
\makeatother%
\endgroup%
}}
        \subcaption{Simulation results using a wave with 40 periods.}
        \label{fig:task5_2}
    \end{subfigure}
  \caption{Simulation results from task 5.}
  \label{fig:task5}
\end{figure}

\section{Task 6}\label{sec:6}
The simulation results can be seen in Figure~\ref{fig:task6}.
\subsection{a}
The $\vec{J}$ component should be added to the part where them $E$-field is updated. Since the differentiation process is done forward \Big($\frac{f(x_{i+1})-f(x_{i})}{x_{i+1}-x_{i}}$ rather than $\frac{f(x_{i})-f(x_{i-1})}{x_{i}-x_{i-1}}$ \Big) and the forward value is sought. This means the $\vec{J}=\sigma E_{i}$ part is not sought and the expression will look similar to what it did without $\vec{J}$.
\subsection{b}
The wave rapidly decays as it propagates into the conducting material
\subsection{c}
It can be seen in Figure~\ref{fig:task6_1} that the energy is not conserved. The wave in the conduction material gives rise to currents in the conducting materials so it looses its energy.
\subsection{d}
The wave gets amplified as it propagates into the material with negative conductivity. This means that energy is added to the wave and in order for this to comply with the law of energy conservation this energy must come from the outside world.
\subsection{e}
This would be and optical amplifier. if the borders allowed some of the wave to escape it could be a laser.
\begin{figure}
  \centering
    \begin{subfigure}[b]{0.7\textwidth}
        \noindent\makebox[\textwidth]{\scalebox{0.7}{%% Creator: Matplotlib, PGF backend
%%
%% To include the figure in your LaTeX document, write
%%   \input{<filename>.pgf}
%%
%% Make sure the required packages are loaded in your preamble
%%   \usepackage{pgf}
%%
%% Figures using additional raster images can only be included by \input if
%% they are in the same directory as the main LaTeX file. For loading figures
%% from other directories you can use the `import` package
%%   \usepackage{import}
%% and then include the figures with
%%   \import{<path to file>}{<filename>.pgf}
%%
%% Matplotlib used the following preamble
%%   \usepackage{fontspec}
%%   \setmainfont{DejaVu Serif}
%%   \setsansfont{DejaVu Sans}
%%   \setmonofont{DejaVu Sans Mono}
%%
\begingroup%
\makeatletter%
\begin{pgfpicture}%
\pgfpathrectangle{\pgfpointorigin}{\pgfqpoint{7.620000in}{4.780000in}}%
\pgfusepath{use as bounding box, clip}%
\begin{pgfscope}%
\pgfsetbuttcap%
\pgfsetmiterjoin%
\definecolor{currentfill}{rgb}{1.000000,1.000000,1.000000}%
\pgfsetfillcolor{currentfill}%
\pgfsetlinewidth{0.000000pt}%
\definecolor{currentstroke}{rgb}{1.000000,1.000000,1.000000}%
\pgfsetstrokecolor{currentstroke}%
\pgfsetdash{}{0pt}%
\pgfpathmoveto{\pgfqpoint{0.000000in}{0.000000in}}%
\pgfpathlineto{\pgfqpoint{7.620000in}{0.000000in}}%
\pgfpathlineto{\pgfqpoint{7.620000in}{4.780000in}}%
\pgfpathlineto{\pgfqpoint{0.000000in}{4.780000in}}%
\pgfpathclose%
\pgfusepath{fill}%
\end{pgfscope}%
\begin{pgfscope}%
\pgfsetbuttcap%
\pgfsetmiterjoin%
\definecolor{currentfill}{rgb}{1.000000,1.000000,1.000000}%
\pgfsetfillcolor{currentfill}%
\pgfsetlinewidth{0.000000pt}%
\definecolor{currentstroke}{rgb}{0.000000,0.000000,0.000000}%
\pgfsetstrokecolor{currentstroke}%
\pgfsetstrokeopacity{0.000000}%
\pgfsetdash{}{0pt}%
\pgfpathmoveto{\pgfqpoint{0.952500in}{0.525800in}}%
\pgfpathlineto{\pgfqpoint{6.858000in}{0.525800in}}%
\pgfpathlineto{\pgfqpoint{6.858000in}{4.206400in}}%
\pgfpathlineto{\pgfqpoint{0.952500in}{4.206400in}}%
\pgfpathclose%
\pgfusepath{fill}%
\end{pgfscope}%
\begin{pgfscope}%
\pgfsetbuttcap%
\pgfsetroundjoin%
\definecolor{currentfill}{rgb}{0.000000,0.000000,0.000000}%
\pgfsetfillcolor{currentfill}%
\pgfsetlinewidth{0.803000pt}%
\definecolor{currentstroke}{rgb}{0.000000,0.000000,0.000000}%
\pgfsetstrokecolor{currentstroke}%
\pgfsetdash{}{0pt}%
\pgfsys@defobject{currentmarker}{\pgfqpoint{0.000000in}{-0.048611in}}{\pgfqpoint{0.000000in}{0.000000in}}{%
\pgfpathmoveto{\pgfqpoint{0.000000in}{0.000000in}}%
\pgfpathlineto{\pgfqpoint{0.000000in}{-0.048611in}}%
\pgfusepath{stroke,fill}%
}%
\begin{pgfscope}%
\pgfsys@transformshift{0.952500in}{0.525800in}%
\pgfsys@useobject{currentmarker}{}%
\end{pgfscope}%
\end{pgfscope}%
\begin{pgfscope}%
\pgftext[x=0.952500in,y=0.428578in,,top]{\sffamily\fontsize{10.000000}{12.000000}\selectfont 0}%
\end{pgfscope}%
\begin{pgfscope}%
\pgfsetbuttcap%
\pgfsetroundjoin%
\definecolor{currentfill}{rgb}{0.000000,0.000000,0.000000}%
\pgfsetfillcolor{currentfill}%
\pgfsetlinewidth{0.803000pt}%
\definecolor{currentstroke}{rgb}{0.000000,0.000000,0.000000}%
\pgfsetstrokecolor{currentstroke}%
\pgfsetdash{}{0pt}%
\pgfsys@defobject{currentmarker}{\pgfqpoint{0.000000in}{-0.048611in}}{\pgfqpoint{0.000000in}{0.000000in}}{%
\pgfpathmoveto{\pgfqpoint{0.000000in}{0.000000in}}%
\pgfpathlineto{\pgfqpoint{0.000000in}{-0.048611in}}%
\pgfusepath{stroke,fill}%
}%
\begin{pgfscope}%
\pgfsys@transformshift{1.652204in}{0.525800in}%
\pgfsys@useobject{currentmarker}{}%
\end{pgfscope}%
\end{pgfscope}%
\begin{pgfscope}%
\pgftext[x=1.652204in,y=0.428578in,,top]{\sffamily\fontsize{10.000000}{12.000000}\selectfont 10}%
\end{pgfscope}%
\begin{pgfscope}%
\pgfsetbuttcap%
\pgfsetroundjoin%
\definecolor{currentfill}{rgb}{0.000000,0.000000,0.000000}%
\pgfsetfillcolor{currentfill}%
\pgfsetlinewidth{0.803000pt}%
\definecolor{currentstroke}{rgb}{0.000000,0.000000,0.000000}%
\pgfsetstrokecolor{currentstroke}%
\pgfsetdash{}{0pt}%
\pgfsys@defobject{currentmarker}{\pgfqpoint{0.000000in}{-0.048611in}}{\pgfqpoint{0.000000in}{0.000000in}}{%
\pgfpathmoveto{\pgfqpoint{0.000000in}{0.000000in}}%
\pgfpathlineto{\pgfqpoint{0.000000in}{-0.048611in}}%
\pgfusepath{stroke,fill}%
}%
\begin{pgfscope}%
\pgfsys@transformshift{2.351908in}{0.525800in}%
\pgfsys@useobject{currentmarker}{}%
\end{pgfscope}%
\end{pgfscope}%
\begin{pgfscope}%
\pgftext[x=2.351908in,y=0.428578in,,top]{\sffamily\fontsize{10.000000}{12.000000}\selectfont 20}%
\end{pgfscope}%
\begin{pgfscope}%
\pgfsetbuttcap%
\pgfsetroundjoin%
\definecolor{currentfill}{rgb}{0.000000,0.000000,0.000000}%
\pgfsetfillcolor{currentfill}%
\pgfsetlinewidth{0.803000pt}%
\definecolor{currentstroke}{rgb}{0.000000,0.000000,0.000000}%
\pgfsetstrokecolor{currentstroke}%
\pgfsetdash{}{0pt}%
\pgfsys@defobject{currentmarker}{\pgfqpoint{0.000000in}{-0.048611in}}{\pgfqpoint{0.000000in}{0.000000in}}{%
\pgfpathmoveto{\pgfqpoint{0.000000in}{0.000000in}}%
\pgfpathlineto{\pgfqpoint{0.000000in}{-0.048611in}}%
\pgfusepath{stroke,fill}%
}%
\begin{pgfscope}%
\pgfsys@transformshift{3.051611in}{0.525800in}%
\pgfsys@useobject{currentmarker}{}%
\end{pgfscope}%
\end{pgfscope}%
\begin{pgfscope}%
\pgftext[x=3.051611in,y=0.428578in,,top]{\sffamily\fontsize{10.000000}{12.000000}\selectfont 30}%
\end{pgfscope}%
\begin{pgfscope}%
\pgfsetbuttcap%
\pgfsetroundjoin%
\definecolor{currentfill}{rgb}{0.000000,0.000000,0.000000}%
\pgfsetfillcolor{currentfill}%
\pgfsetlinewidth{0.803000pt}%
\definecolor{currentstroke}{rgb}{0.000000,0.000000,0.000000}%
\pgfsetstrokecolor{currentstroke}%
\pgfsetdash{}{0pt}%
\pgfsys@defobject{currentmarker}{\pgfqpoint{0.000000in}{-0.048611in}}{\pgfqpoint{0.000000in}{0.000000in}}{%
\pgfpathmoveto{\pgfqpoint{0.000000in}{0.000000in}}%
\pgfpathlineto{\pgfqpoint{0.000000in}{-0.048611in}}%
\pgfusepath{stroke,fill}%
}%
\begin{pgfscope}%
\pgfsys@transformshift{3.751315in}{0.525800in}%
\pgfsys@useobject{currentmarker}{}%
\end{pgfscope}%
\end{pgfscope}%
\begin{pgfscope}%
\pgftext[x=3.751315in,y=0.428578in,,top]{\sffamily\fontsize{10.000000}{12.000000}\selectfont 40}%
\end{pgfscope}%
\begin{pgfscope}%
\pgfsetbuttcap%
\pgfsetroundjoin%
\definecolor{currentfill}{rgb}{0.000000,0.000000,0.000000}%
\pgfsetfillcolor{currentfill}%
\pgfsetlinewidth{0.803000pt}%
\definecolor{currentstroke}{rgb}{0.000000,0.000000,0.000000}%
\pgfsetstrokecolor{currentstroke}%
\pgfsetdash{}{0pt}%
\pgfsys@defobject{currentmarker}{\pgfqpoint{0.000000in}{-0.048611in}}{\pgfqpoint{0.000000in}{0.000000in}}{%
\pgfpathmoveto{\pgfqpoint{0.000000in}{0.000000in}}%
\pgfpathlineto{\pgfqpoint{0.000000in}{-0.048611in}}%
\pgfusepath{stroke,fill}%
}%
\begin{pgfscope}%
\pgfsys@transformshift{4.451019in}{0.525800in}%
\pgfsys@useobject{currentmarker}{}%
\end{pgfscope}%
\end{pgfscope}%
\begin{pgfscope}%
\pgftext[x=4.451019in,y=0.428578in,,top]{\sffamily\fontsize{10.000000}{12.000000}\selectfont 50}%
\end{pgfscope}%
\begin{pgfscope}%
\pgfsetbuttcap%
\pgfsetroundjoin%
\definecolor{currentfill}{rgb}{0.000000,0.000000,0.000000}%
\pgfsetfillcolor{currentfill}%
\pgfsetlinewidth{0.803000pt}%
\definecolor{currentstroke}{rgb}{0.000000,0.000000,0.000000}%
\pgfsetstrokecolor{currentstroke}%
\pgfsetdash{}{0pt}%
\pgfsys@defobject{currentmarker}{\pgfqpoint{0.000000in}{-0.048611in}}{\pgfqpoint{0.000000in}{0.000000in}}{%
\pgfpathmoveto{\pgfqpoint{0.000000in}{0.000000in}}%
\pgfpathlineto{\pgfqpoint{0.000000in}{-0.048611in}}%
\pgfusepath{stroke,fill}%
}%
\begin{pgfscope}%
\pgfsys@transformshift{5.150723in}{0.525800in}%
\pgfsys@useobject{currentmarker}{}%
\end{pgfscope}%
\end{pgfscope}%
\begin{pgfscope}%
\pgftext[x=5.150723in,y=0.428578in,,top]{\sffamily\fontsize{10.000000}{12.000000}\selectfont 60}%
\end{pgfscope}%
\begin{pgfscope}%
\pgfsetbuttcap%
\pgfsetroundjoin%
\definecolor{currentfill}{rgb}{0.000000,0.000000,0.000000}%
\pgfsetfillcolor{currentfill}%
\pgfsetlinewidth{0.803000pt}%
\definecolor{currentstroke}{rgb}{0.000000,0.000000,0.000000}%
\pgfsetstrokecolor{currentstroke}%
\pgfsetdash{}{0pt}%
\pgfsys@defobject{currentmarker}{\pgfqpoint{0.000000in}{-0.048611in}}{\pgfqpoint{0.000000in}{0.000000in}}{%
\pgfpathmoveto{\pgfqpoint{0.000000in}{0.000000in}}%
\pgfpathlineto{\pgfqpoint{0.000000in}{-0.048611in}}%
\pgfusepath{stroke,fill}%
}%
\begin{pgfscope}%
\pgfsys@transformshift{5.850427in}{0.525800in}%
\pgfsys@useobject{currentmarker}{}%
\end{pgfscope}%
\end{pgfscope}%
\begin{pgfscope}%
\pgftext[x=5.850427in,y=0.428578in,,top]{\sffamily\fontsize{10.000000}{12.000000}\selectfont 70}%
\end{pgfscope}%
\begin{pgfscope}%
\pgfsetbuttcap%
\pgfsetroundjoin%
\definecolor{currentfill}{rgb}{0.000000,0.000000,0.000000}%
\pgfsetfillcolor{currentfill}%
\pgfsetlinewidth{0.803000pt}%
\definecolor{currentstroke}{rgb}{0.000000,0.000000,0.000000}%
\pgfsetstrokecolor{currentstroke}%
\pgfsetdash{}{0pt}%
\pgfsys@defobject{currentmarker}{\pgfqpoint{0.000000in}{-0.048611in}}{\pgfqpoint{0.000000in}{0.000000in}}{%
\pgfpathmoveto{\pgfqpoint{0.000000in}{0.000000in}}%
\pgfpathlineto{\pgfqpoint{0.000000in}{-0.048611in}}%
\pgfusepath{stroke,fill}%
}%
\begin{pgfscope}%
\pgfsys@transformshift{6.550130in}{0.525800in}%
\pgfsys@useobject{currentmarker}{}%
\end{pgfscope}%
\end{pgfscope}%
\begin{pgfscope}%
\pgftext[x=6.550130in,y=0.428578in,,top]{\sffamily\fontsize{10.000000}{12.000000}\selectfont 80}%
\end{pgfscope}%
\begin{pgfscope}%
\pgftext[x=3.905250in,y=0.238609in,,top]{\sffamily\fontsize{16.000000}{19.200000}\selectfont \(\displaystyle z-position [\mu m]\)}%
\end{pgfscope}%
\begin{pgfscope}%
\pgfsetbuttcap%
\pgfsetroundjoin%
\definecolor{currentfill}{rgb}{0.000000,0.000000,0.000000}%
\pgfsetfillcolor{currentfill}%
\pgfsetlinewidth{0.803000pt}%
\definecolor{currentstroke}{rgb}{0.000000,0.000000,0.000000}%
\pgfsetstrokecolor{currentstroke}%
\pgfsetdash{}{0pt}%
\pgfsys@defobject{currentmarker}{\pgfqpoint{-0.048611in}{0.000000in}}{\pgfqpoint{0.000000in}{0.000000in}}{%
\pgfpathmoveto{\pgfqpoint{0.000000in}{0.000000in}}%
\pgfpathlineto{\pgfqpoint{-0.048611in}{0.000000in}}%
\pgfusepath{stroke,fill}%
}%
\begin{pgfscope}%
\pgfsys@transformshift{0.952500in}{0.525800in}%
\pgfsys@useobject{currentmarker}{}%
\end{pgfscope}%
\end{pgfscope}%
\begin{pgfscope}%
\pgftext[x=0.341294in,y=0.473038in,left,base]{\sffamily\fontsize{10.000000}{12.000000}\selectfont -0.003}%
\end{pgfscope}%
\begin{pgfscope}%
\pgfsetbuttcap%
\pgfsetroundjoin%
\definecolor{currentfill}{rgb}{0.000000,0.000000,0.000000}%
\pgfsetfillcolor{currentfill}%
\pgfsetlinewidth{0.803000pt}%
\definecolor{currentstroke}{rgb}{0.000000,0.000000,0.000000}%
\pgfsetstrokecolor{currentstroke}%
\pgfsetdash{}{0pt}%
\pgfsys@defobject{currentmarker}{\pgfqpoint{-0.048611in}{0.000000in}}{\pgfqpoint{0.000000in}{0.000000in}}{%
\pgfpathmoveto{\pgfqpoint{0.000000in}{0.000000in}}%
\pgfpathlineto{\pgfqpoint{-0.048611in}{0.000000in}}%
\pgfusepath{stroke,fill}%
}%
\begin{pgfscope}%
\pgfsys@transformshift{0.952500in}{1.139233in}%
\pgfsys@useobject{currentmarker}{}%
\end{pgfscope}%
\end{pgfscope}%
\begin{pgfscope}%
\pgftext[x=0.341294in,y=1.086472in,left,base]{\sffamily\fontsize{10.000000}{12.000000}\selectfont -0.002}%
\end{pgfscope}%
\begin{pgfscope}%
\pgfsetbuttcap%
\pgfsetroundjoin%
\definecolor{currentfill}{rgb}{0.000000,0.000000,0.000000}%
\pgfsetfillcolor{currentfill}%
\pgfsetlinewidth{0.803000pt}%
\definecolor{currentstroke}{rgb}{0.000000,0.000000,0.000000}%
\pgfsetstrokecolor{currentstroke}%
\pgfsetdash{}{0pt}%
\pgfsys@defobject{currentmarker}{\pgfqpoint{-0.048611in}{0.000000in}}{\pgfqpoint{0.000000in}{0.000000in}}{%
\pgfpathmoveto{\pgfqpoint{0.000000in}{0.000000in}}%
\pgfpathlineto{\pgfqpoint{-0.048611in}{0.000000in}}%
\pgfusepath{stroke,fill}%
}%
\begin{pgfscope}%
\pgfsys@transformshift{0.952500in}{1.752667in}%
\pgfsys@useobject{currentmarker}{}%
\end{pgfscope}%
\end{pgfscope}%
\begin{pgfscope}%
\pgftext[x=0.341294in,y=1.699905in,left,base]{\sffamily\fontsize{10.000000}{12.000000}\selectfont -0.001}%
\end{pgfscope}%
\begin{pgfscope}%
\pgfsetbuttcap%
\pgfsetroundjoin%
\definecolor{currentfill}{rgb}{0.000000,0.000000,0.000000}%
\pgfsetfillcolor{currentfill}%
\pgfsetlinewidth{0.803000pt}%
\definecolor{currentstroke}{rgb}{0.000000,0.000000,0.000000}%
\pgfsetstrokecolor{currentstroke}%
\pgfsetdash{}{0pt}%
\pgfsys@defobject{currentmarker}{\pgfqpoint{-0.048611in}{0.000000in}}{\pgfqpoint{0.000000in}{0.000000in}}{%
\pgfpathmoveto{\pgfqpoint{0.000000in}{0.000000in}}%
\pgfpathlineto{\pgfqpoint{-0.048611in}{0.000000in}}%
\pgfusepath{stroke,fill}%
}%
\begin{pgfscope}%
\pgfsys@transformshift{0.952500in}{2.366100in}%
\pgfsys@useobject{currentmarker}{}%
\end{pgfscope}%
\end{pgfscope}%
\begin{pgfscope}%
\pgftext[x=0.457668in,y=2.313338in,left,base]{\sffamily\fontsize{10.000000}{12.000000}\selectfont 0.000}%
\end{pgfscope}%
\begin{pgfscope}%
\pgfsetbuttcap%
\pgfsetroundjoin%
\definecolor{currentfill}{rgb}{0.000000,0.000000,0.000000}%
\pgfsetfillcolor{currentfill}%
\pgfsetlinewidth{0.803000pt}%
\definecolor{currentstroke}{rgb}{0.000000,0.000000,0.000000}%
\pgfsetstrokecolor{currentstroke}%
\pgfsetdash{}{0pt}%
\pgfsys@defobject{currentmarker}{\pgfqpoint{-0.048611in}{0.000000in}}{\pgfqpoint{0.000000in}{0.000000in}}{%
\pgfpathmoveto{\pgfqpoint{0.000000in}{0.000000in}}%
\pgfpathlineto{\pgfqpoint{-0.048611in}{0.000000in}}%
\pgfusepath{stroke,fill}%
}%
\begin{pgfscope}%
\pgfsys@transformshift{0.952500in}{2.979533in}%
\pgfsys@useobject{currentmarker}{}%
\end{pgfscope}%
\end{pgfscope}%
\begin{pgfscope}%
\pgftext[x=0.457668in,y=2.926772in,left,base]{\sffamily\fontsize{10.000000}{12.000000}\selectfont 0.001}%
\end{pgfscope}%
\begin{pgfscope}%
\pgfsetbuttcap%
\pgfsetroundjoin%
\definecolor{currentfill}{rgb}{0.000000,0.000000,0.000000}%
\pgfsetfillcolor{currentfill}%
\pgfsetlinewidth{0.803000pt}%
\definecolor{currentstroke}{rgb}{0.000000,0.000000,0.000000}%
\pgfsetstrokecolor{currentstroke}%
\pgfsetdash{}{0pt}%
\pgfsys@defobject{currentmarker}{\pgfqpoint{-0.048611in}{0.000000in}}{\pgfqpoint{0.000000in}{0.000000in}}{%
\pgfpathmoveto{\pgfqpoint{0.000000in}{0.000000in}}%
\pgfpathlineto{\pgfqpoint{-0.048611in}{0.000000in}}%
\pgfusepath{stroke,fill}%
}%
\begin{pgfscope}%
\pgfsys@transformshift{0.952500in}{3.592967in}%
\pgfsys@useobject{currentmarker}{}%
\end{pgfscope}%
\end{pgfscope}%
\begin{pgfscope}%
\pgftext[x=0.457668in,y=3.540205in,left,base]{\sffamily\fontsize{10.000000}{12.000000}\selectfont 0.002}%
\end{pgfscope}%
\begin{pgfscope}%
\pgfsetbuttcap%
\pgfsetroundjoin%
\definecolor{currentfill}{rgb}{0.000000,0.000000,0.000000}%
\pgfsetfillcolor{currentfill}%
\pgfsetlinewidth{0.803000pt}%
\definecolor{currentstroke}{rgb}{0.000000,0.000000,0.000000}%
\pgfsetstrokecolor{currentstroke}%
\pgfsetdash{}{0pt}%
\pgfsys@defobject{currentmarker}{\pgfqpoint{-0.048611in}{0.000000in}}{\pgfqpoint{0.000000in}{0.000000in}}{%
\pgfpathmoveto{\pgfqpoint{0.000000in}{0.000000in}}%
\pgfpathlineto{\pgfqpoint{-0.048611in}{0.000000in}}%
\pgfusepath{stroke,fill}%
}%
\begin{pgfscope}%
\pgfsys@transformshift{0.952500in}{4.206400in}%
\pgfsys@useobject{currentmarker}{}%
\end{pgfscope}%
\end{pgfscope}%
\begin{pgfscope}%
\pgftext[x=0.457668in,y=4.153638in,left,base]{\sffamily\fontsize{10.000000}{12.000000}\selectfont 0.003}%
\end{pgfscope}%
\begin{pgfscope}%
\pgftext[x=0.285738in,y=2.366100in,,bottom,rotate=90.000000]{\sffamily\fontsize{16.000000}{19.200000}\selectfont \(\displaystyle Poynting\) \(\displaystyle vector\)}%
\end{pgfscope}%
\begin{pgfscope}%
\pgfpathrectangle{\pgfqpoint{0.952500in}{0.525800in}}{\pgfqpoint{5.905500in}{3.680600in}} %
\pgfusepath{clip}%
\pgfsetrectcap%
\pgfsetroundjoin%
\pgfsetlinewidth{1.505625pt}%
\definecolor{currentstroke}{rgb}{0.000000,0.000000,0.000000}%
\pgfsetstrokecolor{currentstroke}%
\pgfsetdash{}{0pt}%
\pgfpathmoveto{\pgfqpoint{0.952500in}{2.366100in}}%
\pgfpathlineto{\pgfqpoint{1.149965in}{2.365081in}}%
\pgfpathlineto{\pgfqpoint{1.157716in}{2.363440in}}%
\pgfpathlineto{\pgfqpoint{1.163253in}{2.365291in}}%
\pgfpathlineto{\pgfqpoint{1.168051in}{2.365969in}}%
\pgfpathlineto{\pgfqpoint{1.172111in}{2.363897in}}%
\pgfpathlineto{\pgfqpoint{1.178385in}{2.360454in}}%
\pgfpathlineto{\pgfqpoint{1.181707in}{2.361528in}}%
\pgfpathlineto{\pgfqpoint{1.189827in}{2.365979in}}%
\pgfpathlineto{\pgfqpoint{1.192780in}{2.363852in}}%
\pgfpathlineto{\pgfqpoint{1.200900in}{2.356017in}}%
\pgfpathlineto{\pgfqpoint{1.203484in}{2.357645in}}%
\pgfpathlineto{\pgfqpoint{1.211973in}{2.365903in}}%
\pgfpathlineto{\pgfqpoint{1.214187in}{2.363653in}}%
\pgfpathlineto{\pgfqpoint{1.223415in}{2.350298in}}%
\pgfpathlineto{\pgfqpoint{1.225629in}{2.352882in}}%
\pgfpathlineto{\pgfqpoint{1.233749in}{2.366006in}}%
\pgfpathlineto{\pgfqpoint{1.235595in}{2.363979in}}%
\pgfpathlineto{\pgfqpoint{1.238548in}{2.356342in}}%
\pgfpathlineto{\pgfqpoint{1.243715in}{2.343585in}}%
\pgfpathlineto{\pgfqpoint{1.245560in}{2.343495in}}%
\pgfpathlineto{\pgfqpoint{1.247406in}{2.346440in}}%
\pgfpathlineto{\pgfqpoint{1.251835in}{2.360106in}}%
\pgfpathlineto{\pgfqpoint{1.254788in}{2.365894in}}%
\pgfpathlineto{\pgfqpoint{1.256264in}{2.365709in}}%
\pgfpathlineto{\pgfqpoint{1.258110in}{2.362265in}}%
\pgfpathlineto{\pgfqpoint{1.261801in}{2.348278in}}%
\pgfpathlineto{\pgfqpoint{1.265861in}{2.336192in}}%
\pgfpathlineto{\pgfqpoint{1.267337in}{2.335896in}}%
\pgfpathlineto{\pgfqpoint{1.268813in}{2.338222in}}%
\pgfpathlineto{\pgfqpoint{1.271766in}{2.348848in}}%
\pgfpathlineto{\pgfqpoint{1.276564in}{2.365428in}}%
\pgfpathlineto{\pgfqpoint{1.278041in}{2.365943in}}%
\pgfpathlineto{\pgfqpoint{1.279517in}{2.363498in}}%
\pgfpathlineto{\pgfqpoint{1.282101in}{2.353281in}}%
\pgfpathlineto{\pgfqpoint{1.288006in}{2.328873in}}%
\pgfpathlineto{\pgfqpoint{1.289483in}{2.328639in}}%
\pgfpathlineto{\pgfqpoint{1.290959in}{2.331644in}}%
\pgfpathlineto{\pgfqpoint{1.293912in}{2.344920in}}%
\pgfpathlineto{\pgfqpoint{1.298710in}{2.365278in}}%
\pgfpathlineto{\pgfqpoint{1.300186in}{2.365913in}}%
\pgfpathlineto{\pgfqpoint{1.301663in}{2.362980in}}%
\pgfpathlineto{\pgfqpoint{1.304246in}{2.350777in}}%
\pgfpathlineto{\pgfqpoint{1.309783in}{2.322743in}}%
\pgfpathlineto{\pgfqpoint{1.311259in}{2.321624in}}%
\pgfpathlineto{\pgfqpoint{1.312736in}{2.324359in}}%
\pgfpathlineto{\pgfqpoint{1.315319in}{2.336740in}}%
\pgfpathlineto{\pgfqpoint{1.320856in}{2.365143in}}%
\pgfpathlineto{\pgfqpoint{1.322332in}{2.365887in}}%
\pgfpathlineto{\pgfqpoint{1.323808in}{2.362519in}}%
\pgfpathlineto{\pgfqpoint{1.326392in}{2.348560in}}%
\pgfpathlineto{\pgfqpoint{1.331928in}{2.316886in}}%
\pgfpathlineto{\pgfqpoint{1.333405in}{2.315729in}}%
\pgfpathlineto{\pgfqpoint{1.334881in}{2.318931in}}%
\pgfpathlineto{\pgfqpoint{1.337465in}{2.333045in}}%
\pgfpathlineto{\pgfqpoint{1.343001in}{2.365028in}}%
\pgfpathlineto{\pgfqpoint{1.344108in}{2.366089in}}%
\pgfpathlineto{\pgfqpoint{1.344478in}{2.365865in}}%
\pgfpathlineto{\pgfqpoint{1.345954in}{2.362129in}}%
\pgfpathlineto{\pgfqpoint{1.348538in}{2.346693in}}%
\pgfpathlineto{\pgfqpoint{1.354074in}{2.312004in}}%
\pgfpathlineto{\pgfqpoint{1.355550in}{2.310829in}}%
\pgfpathlineto{\pgfqpoint{1.357027in}{2.314431in}}%
\pgfpathlineto{\pgfqpoint{1.359610in}{2.329997in}}%
\pgfpathlineto{\pgfqpoint{1.365147in}{2.364933in}}%
\pgfpathlineto{\pgfqpoint{1.366254in}{2.366089in}}%
\pgfpathlineto{\pgfqpoint{1.366623in}{2.365848in}}%
\pgfpathlineto{\pgfqpoint{1.368100in}{2.361813in}}%
\pgfpathlineto{\pgfqpoint{1.370683in}{2.345186in}}%
\pgfpathlineto{\pgfqpoint{1.376220in}{2.308099in}}%
\pgfpathlineto{\pgfqpoint{1.377327in}{2.306728in}}%
\pgfpathlineto{\pgfqpoint{1.377696in}{2.306919in}}%
\pgfpathlineto{\pgfqpoint{1.379172in}{2.310849in}}%
\pgfpathlineto{\pgfqpoint{1.381756in}{2.327581in}}%
\pgfpathlineto{\pgfqpoint{1.387292in}{2.364858in}}%
\pgfpathlineto{\pgfqpoint{1.388400in}{2.366089in}}%
\pgfpathlineto{\pgfqpoint{1.388769in}{2.365834in}}%
\pgfpathlineto{\pgfqpoint{1.390245in}{2.361567in}}%
\pgfpathlineto{\pgfqpoint{1.392829in}{2.344013in}}%
\pgfpathlineto{\pgfqpoint{1.398365in}{2.305085in}}%
\pgfpathlineto{\pgfqpoint{1.399473in}{2.303691in}}%
\pgfpathlineto{\pgfqpoint{1.399842in}{2.303908in}}%
\pgfpathlineto{\pgfqpoint{1.401318in}{2.308096in}}%
\pgfpathlineto{\pgfqpoint{1.403902in}{2.325730in}}%
\pgfpathlineto{\pgfqpoint{1.409438in}{2.364801in}}%
\pgfpathlineto{\pgfqpoint{1.410545in}{2.366088in}}%
\pgfpathlineto{\pgfqpoint{1.410914in}{2.365823in}}%
\pgfpathlineto{\pgfqpoint{1.412391in}{2.361381in}}%
\pgfpathlineto{\pgfqpoint{1.414974in}{2.343131in}}%
\pgfpathlineto{\pgfqpoint{1.420511in}{2.302831in}}%
\pgfpathlineto{\pgfqpoint{1.421618in}{2.301424in}}%
\pgfpathlineto{\pgfqpoint{1.421987in}{2.301661in}}%
\pgfpathlineto{\pgfqpoint{1.423464in}{2.306046in}}%
\pgfpathlineto{\pgfqpoint{1.426047in}{2.324357in}}%
\pgfpathlineto{\pgfqpoint{1.431584in}{2.364759in}}%
\pgfpathlineto{\pgfqpoint{1.432691in}{2.366088in}}%
\pgfpathlineto{\pgfqpoint{1.433060in}{2.365816in}}%
\pgfpathlineto{\pgfqpoint{1.434536in}{2.361245in}}%
\pgfpathlineto{\pgfqpoint{1.437120in}{2.342485in}}%
\pgfpathlineto{\pgfqpoint{1.442657in}{2.301195in}}%
\pgfpathlineto{\pgfqpoint{1.443764in}{2.299780in}}%
\pgfpathlineto{\pgfqpoint{1.444133in}{2.300033in}}%
\pgfpathlineto{\pgfqpoint{1.445609in}{2.304563in}}%
\pgfpathlineto{\pgfqpoint{1.448193in}{2.323366in}}%
\pgfpathlineto{\pgfqpoint{1.453729in}{2.364729in}}%
\pgfpathlineto{\pgfqpoint{1.454837in}{2.366088in}}%
\pgfpathlineto{\pgfqpoint{1.455206in}{2.365811in}}%
\pgfpathlineto{\pgfqpoint{1.456682in}{2.361147in}}%
\pgfpathlineto{\pgfqpoint{1.459266in}{2.342027in}}%
\pgfpathlineto{\pgfqpoint{1.464802in}{2.300040in}}%
\pgfpathlineto{\pgfqpoint{1.465909in}{2.298621in}}%
\pgfpathlineto{\pgfqpoint{1.466279in}{2.298885in}}%
\pgfpathlineto{\pgfqpoint{1.467755in}{2.303520in}}%
\pgfpathlineto{\pgfqpoint{1.470339in}{2.322672in}}%
\pgfpathlineto{\pgfqpoint{1.475875in}{2.364708in}}%
\pgfpathlineto{\pgfqpoint{1.476982in}{2.366088in}}%
\pgfpathlineto{\pgfqpoint{1.477351in}{2.365807in}}%
\pgfpathlineto{\pgfqpoint{1.478828in}{2.361080in}}%
\pgfpathlineto{\pgfqpoint{1.481411in}{2.341710in}}%
\pgfpathlineto{\pgfqpoint{1.486948in}{2.299247in}}%
\pgfpathlineto{\pgfqpoint{1.488055in}{2.297826in}}%
\pgfpathlineto{\pgfqpoint{1.488424in}{2.298098in}}%
\pgfpathlineto{\pgfqpoint{1.489901in}{2.302806in}}%
\pgfpathlineto{\pgfqpoint{1.492484in}{2.322197in}}%
\pgfpathlineto{\pgfqpoint{1.498021in}{2.364694in}}%
\pgfpathlineto{\pgfqpoint{1.499128in}{2.366088in}}%
\pgfpathlineto{\pgfqpoint{1.499497in}{2.365804in}}%
\pgfpathlineto{\pgfqpoint{1.500973in}{2.361035in}}%
\pgfpathlineto{\pgfqpoint{1.503557in}{2.341496in}}%
\pgfpathlineto{\pgfqpoint{1.509093in}{2.298716in}}%
\pgfpathlineto{\pgfqpoint{1.510201in}{2.297295in}}%
\pgfpathlineto{\pgfqpoint{1.510570in}{2.297573in}}%
\pgfpathlineto{\pgfqpoint{1.512046in}{2.302330in}}%
\pgfpathlineto{\pgfqpoint{1.514630in}{2.321883in}}%
\pgfpathlineto{\pgfqpoint{1.520166in}{2.364684in}}%
\pgfpathlineto{\pgfqpoint{1.521273in}{2.366088in}}%
\pgfpathlineto{\pgfqpoint{1.521643in}{2.365803in}}%
\pgfpathlineto{\pgfqpoint{1.523119in}{2.361005in}}%
\pgfpathlineto{\pgfqpoint{1.525703in}{2.341357in}}%
\pgfpathlineto{\pgfqpoint{1.531239in}{2.298372in}}%
\pgfpathlineto{\pgfqpoint{1.532346in}{2.296952in}}%
\pgfpathlineto{\pgfqpoint{1.532715in}{2.297233in}}%
\pgfpathlineto{\pgfqpoint{1.534192in}{2.302023in}}%
\pgfpathlineto{\pgfqpoint{1.536775in}{2.321680in}}%
\pgfpathlineto{\pgfqpoint{1.542312in}{2.364678in}}%
\pgfpathlineto{\pgfqpoint{1.543419in}{2.366088in}}%
\pgfpathlineto{\pgfqpoint{1.543788in}{2.365802in}}%
\pgfpathlineto{\pgfqpoint{1.545265in}{2.360986in}}%
\pgfpathlineto{\pgfqpoint{1.547848in}{2.341268in}}%
\pgfpathlineto{\pgfqpoint{1.553385in}{2.298156in}}%
\pgfpathlineto{\pgfqpoint{1.554492in}{2.296736in}}%
\pgfpathlineto{\pgfqpoint{1.554861in}{2.297020in}}%
\pgfpathlineto{\pgfqpoint{1.556337in}{2.301831in}}%
\pgfpathlineto{\pgfqpoint{1.558921in}{2.321554in}}%
\pgfpathlineto{\pgfqpoint{1.564457in}{2.364674in}}%
\pgfpathlineto{\pgfqpoint{1.565565in}{2.366088in}}%
\pgfpathlineto{\pgfqpoint{1.565934in}{2.365801in}}%
\pgfpathlineto{\pgfqpoint{1.567410in}{2.360974in}}%
\pgfpathlineto{\pgfqpoint{1.569994in}{2.341214in}}%
\pgfpathlineto{\pgfqpoint{1.575530in}{2.298025in}}%
\pgfpathlineto{\pgfqpoint{1.576638in}{2.296605in}}%
\pgfpathlineto{\pgfqpoint{1.577007in}{2.296891in}}%
\pgfpathlineto{\pgfqpoint{1.578483in}{2.301714in}}%
\pgfpathlineto{\pgfqpoint{1.581067in}{2.321478in}}%
\pgfpathlineto{\pgfqpoint{1.586603in}{2.364672in}}%
\pgfpathlineto{\pgfqpoint{1.587710in}{2.366088in}}%
\pgfpathlineto{\pgfqpoint{1.588079in}{2.365801in}}%
\pgfpathlineto{\pgfqpoint{1.589556in}{2.360967in}}%
\pgfpathlineto{\pgfqpoint{1.592139in}{2.341182in}}%
\pgfpathlineto{\pgfqpoint{1.597676in}{2.297948in}}%
\pgfpathlineto{\pgfqpoint{1.598783in}{2.296529in}}%
\pgfpathlineto{\pgfqpoint{1.599152in}{2.296815in}}%
\pgfpathlineto{\pgfqpoint{1.600629in}{2.301647in}}%
\pgfpathlineto{\pgfqpoint{1.603212in}{2.321434in}}%
\pgfpathlineto{\pgfqpoint{1.608749in}{2.364671in}}%
\pgfpathlineto{\pgfqpoint{1.609856in}{2.366088in}}%
\pgfpathlineto{\pgfqpoint{1.610225in}{2.365800in}}%
\pgfpathlineto{\pgfqpoint{1.611701in}{2.360964in}}%
\pgfpathlineto{\pgfqpoint{1.614285in}{2.341165in}}%
\pgfpathlineto{\pgfqpoint{1.619822in}{2.297906in}}%
\pgfpathlineto{\pgfqpoint{1.620929in}{2.296487in}}%
\pgfpathlineto{\pgfqpoint{1.621298in}{2.296774in}}%
\pgfpathlineto{\pgfqpoint{1.622774in}{2.301610in}}%
\pgfpathlineto{\pgfqpoint{1.625358in}{2.321410in}}%
\pgfpathlineto{\pgfqpoint{1.630894in}{2.364670in}}%
\pgfpathlineto{\pgfqpoint{1.632002in}{2.366088in}}%
\pgfpathlineto{\pgfqpoint{1.632371in}{2.365800in}}%
\pgfpathlineto{\pgfqpoint{1.633847in}{2.360961in}}%
\pgfpathlineto{\pgfqpoint{1.636431in}{2.341155in}}%
\pgfpathlineto{\pgfqpoint{1.641967in}{2.297883in}}%
\pgfpathlineto{\pgfqpoint{1.643074in}{2.296465in}}%
\pgfpathlineto{\pgfqpoint{1.643444in}{2.296752in}}%
\pgfpathlineto{\pgfqpoint{1.644920in}{2.301590in}}%
\pgfpathlineto{\pgfqpoint{1.647504in}{2.321397in}}%
\pgfpathlineto{\pgfqpoint{1.653040in}{2.364670in}}%
\pgfpathlineto{\pgfqpoint{1.654147in}{2.366088in}}%
\pgfpathlineto{\pgfqpoint{1.654516in}{2.365800in}}%
\pgfpathlineto{\pgfqpoint{1.655993in}{2.360960in}}%
\pgfpathlineto{\pgfqpoint{1.658576in}{2.341150in}}%
\pgfpathlineto{\pgfqpoint{1.664113in}{2.297872in}}%
\pgfpathlineto{\pgfqpoint{1.665220in}{2.296454in}}%
\pgfpathlineto{\pgfqpoint{1.665589in}{2.296741in}}%
\pgfpathlineto{\pgfqpoint{1.667065in}{2.301581in}}%
\pgfpathlineto{\pgfqpoint{1.669649in}{2.321391in}}%
\pgfpathlineto{\pgfqpoint{1.675186in}{2.364670in}}%
\pgfpathlineto{\pgfqpoint{1.676293in}{2.366088in}}%
\pgfpathlineto{\pgfqpoint{1.676662in}{2.365800in}}%
\pgfpathlineto{\pgfqpoint{1.678138in}{2.360960in}}%
\pgfpathlineto{\pgfqpoint{1.680722in}{2.341148in}}%
\pgfpathlineto{\pgfqpoint{1.686258in}{2.297867in}}%
\pgfpathlineto{\pgfqpoint{1.687366in}{2.296449in}}%
\pgfpathlineto{\pgfqpoint{1.687735in}{2.296736in}}%
\pgfpathlineto{\pgfqpoint{1.689211in}{2.301576in}}%
\pgfpathlineto{\pgfqpoint{1.691795in}{2.321388in}}%
\pgfpathlineto{\pgfqpoint{1.697331in}{2.364670in}}%
\pgfpathlineto{\pgfqpoint{1.698438in}{2.366088in}}%
\pgfpathlineto{\pgfqpoint{1.698808in}{2.365800in}}%
\pgfpathlineto{\pgfqpoint{1.700284in}{2.360960in}}%
\pgfpathlineto{\pgfqpoint{1.702868in}{2.341147in}}%
\pgfpathlineto{\pgfqpoint{1.708404in}{2.297865in}}%
\pgfpathlineto{\pgfqpoint{1.709511in}{2.296446in}}%
\pgfpathlineto{\pgfqpoint{1.709880in}{2.296734in}}%
\pgfpathlineto{\pgfqpoint{1.711357in}{2.301574in}}%
\pgfpathlineto{\pgfqpoint{1.713940in}{2.321387in}}%
\pgfpathlineto{\pgfqpoint{1.719477in}{2.364670in}}%
\pgfpathlineto{\pgfqpoint{1.720584in}{2.366088in}}%
\pgfpathlineto{\pgfqpoint{1.720953in}{2.365800in}}%
\pgfpathlineto{\pgfqpoint{1.722430in}{2.360960in}}%
\pgfpathlineto{\pgfqpoint{1.725013in}{2.341146in}}%
\pgfpathlineto{\pgfqpoint{1.730550in}{2.297864in}}%
\pgfpathlineto{\pgfqpoint{1.731657in}{2.296446in}}%
\pgfpathlineto{\pgfqpoint{1.732026in}{2.296733in}}%
\pgfpathlineto{\pgfqpoint{1.733502in}{2.301574in}}%
\pgfpathlineto{\pgfqpoint{1.736086in}{2.321387in}}%
\pgfpathlineto{\pgfqpoint{1.741622in}{2.364670in}}%
\pgfpathlineto{\pgfqpoint{1.742730in}{2.366088in}}%
\pgfpathlineto{\pgfqpoint{1.743099in}{2.365800in}}%
\pgfpathlineto{\pgfqpoint{1.744575in}{2.360960in}}%
\pgfpathlineto{\pgfqpoint{1.747159in}{2.341146in}}%
\pgfpathlineto{\pgfqpoint{1.752695in}{2.297864in}}%
\pgfpathlineto{\pgfqpoint{1.753803in}{2.296445in}}%
\pgfpathlineto{\pgfqpoint{1.754172in}{2.296733in}}%
\pgfpathlineto{\pgfqpoint{1.755648in}{2.301574in}}%
\pgfpathlineto{\pgfqpoint{1.758232in}{2.321387in}}%
\pgfpathlineto{\pgfqpoint{1.763768in}{2.364670in}}%
\pgfpathlineto{\pgfqpoint{1.764875in}{2.366088in}}%
\pgfpathlineto{\pgfqpoint{1.765244in}{2.365800in}}%
\pgfpathlineto{\pgfqpoint{1.766721in}{2.360960in}}%
\pgfpathlineto{\pgfqpoint{1.769304in}{2.341146in}}%
\pgfpathlineto{\pgfqpoint{1.774841in}{2.297864in}}%
\pgfpathlineto{\pgfqpoint{1.775948in}{2.296445in}}%
\pgfpathlineto{\pgfqpoint{1.776317in}{2.296733in}}%
\pgfpathlineto{\pgfqpoint{1.777794in}{2.301574in}}%
\pgfpathlineto{\pgfqpoint{1.780377in}{2.321387in}}%
\pgfpathlineto{\pgfqpoint{1.785914in}{2.364670in}}%
\pgfpathlineto{\pgfqpoint{1.787021in}{2.366088in}}%
\pgfpathlineto{\pgfqpoint{1.787390in}{2.365800in}}%
\pgfpathlineto{\pgfqpoint{1.788866in}{2.360960in}}%
\pgfpathlineto{\pgfqpoint{1.791450in}{2.341146in}}%
\pgfpathlineto{\pgfqpoint{1.796986in}{2.297864in}}%
\pgfpathlineto{\pgfqpoint{1.798094in}{2.296445in}}%
\pgfpathlineto{\pgfqpoint{1.798463in}{2.296733in}}%
\pgfpathlineto{\pgfqpoint{1.799939in}{2.301574in}}%
\pgfpathlineto{\pgfqpoint{1.802523in}{2.321387in}}%
\pgfpathlineto{\pgfqpoint{1.808059in}{2.364670in}}%
\pgfpathlineto{\pgfqpoint{1.809167in}{2.366088in}}%
\pgfpathlineto{\pgfqpoint{1.809536in}{2.365800in}}%
\pgfpathlineto{\pgfqpoint{1.811012in}{2.360960in}}%
\pgfpathlineto{\pgfqpoint{1.813596in}{2.341146in}}%
\pgfpathlineto{\pgfqpoint{1.819132in}{2.297864in}}%
\pgfpathlineto{\pgfqpoint{1.820239in}{2.296445in}}%
\pgfpathlineto{\pgfqpoint{1.820608in}{2.296733in}}%
\pgfpathlineto{\pgfqpoint{1.822085in}{2.301574in}}%
\pgfpathlineto{\pgfqpoint{1.824669in}{2.321387in}}%
\pgfpathlineto{\pgfqpoint{1.830205in}{2.364670in}}%
\pgfpathlineto{\pgfqpoint{1.831312in}{2.366088in}}%
\pgfpathlineto{\pgfqpoint{1.831681in}{2.365800in}}%
\pgfpathlineto{\pgfqpoint{1.833158in}{2.360960in}}%
\pgfpathlineto{\pgfqpoint{1.835741in}{2.341146in}}%
\pgfpathlineto{\pgfqpoint{1.841278in}{2.297864in}}%
\pgfpathlineto{\pgfqpoint{1.842385in}{2.296445in}}%
\pgfpathlineto{\pgfqpoint{1.842754in}{2.296733in}}%
\pgfpathlineto{\pgfqpoint{1.844230in}{2.301574in}}%
\pgfpathlineto{\pgfqpoint{1.846814in}{2.321387in}}%
\pgfpathlineto{\pgfqpoint{1.852351in}{2.364670in}}%
\pgfpathlineto{\pgfqpoint{1.853458in}{2.366088in}}%
\pgfpathlineto{\pgfqpoint{1.853827in}{2.365800in}}%
\pgfpathlineto{\pgfqpoint{1.855303in}{2.360960in}}%
\pgfpathlineto{\pgfqpoint{1.857887in}{2.341146in}}%
\pgfpathlineto{\pgfqpoint{1.863423in}{2.297864in}}%
\pgfpathlineto{\pgfqpoint{1.864531in}{2.296445in}}%
\pgfpathlineto{\pgfqpoint{1.864900in}{2.296733in}}%
\pgfpathlineto{\pgfqpoint{1.866376in}{2.301574in}}%
\pgfpathlineto{\pgfqpoint{1.868960in}{2.321387in}}%
\pgfpathlineto{\pgfqpoint{1.874496in}{2.364670in}}%
\pgfpathlineto{\pgfqpoint{1.875603in}{2.366088in}}%
\pgfpathlineto{\pgfqpoint{1.875973in}{2.365800in}}%
\pgfpathlineto{\pgfqpoint{1.877449in}{2.360960in}}%
\pgfpathlineto{\pgfqpoint{1.880033in}{2.341146in}}%
\pgfpathlineto{\pgfqpoint{1.885569in}{2.297864in}}%
\pgfpathlineto{\pgfqpoint{1.886676in}{2.296445in}}%
\pgfpathlineto{\pgfqpoint{1.887045in}{2.296733in}}%
\pgfpathlineto{\pgfqpoint{1.888522in}{2.301574in}}%
\pgfpathlineto{\pgfqpoint{1.891105in}{2.321387in}}%
\pgfpathlineto{\pgfqpoint{1.896642in}{2.364670in}}%
\pgfpathlineto{\pgfqpoint{1.897749in}{2.366088in}}%
\pgfpathlineto{\pgfqpoint{1.898118in}{2.365800in}}%
\pgfpathlineto{\pgfqpoint{1.899595in}{2.360960in}}%
\pgfpathlineto{\pgfqpoint{1.902178in}{2.341146in}}%
\pgfpathlineto{\pgfqpoint{1.907715in}{2.297864in}}%
\pgfpathlineto{\pgfqpoint{1.908822in}{2.296445in}}%
\pgfpathlineto{\pgfqpoint{1.909191in}{2.296733in}}%
\pgfpathlineto{\pgfqpoint{1.910667in}{2.301574in}}%
\pgfpathlineto{\pgfqpoint{1.913251in}{2.321387in}}%
\pgfpathlineto{\pgfqpoint{1.918787in}{2.364670in}}%
\pgfpathlineto{\pgfqpoint{1.919895in}{2.366088in}}%
\pgfpathlineto{\pgfqpoint{1.920264in}{2.365800in}}%
\pgfpathlineto{\pgfqpoint{1.921740in}{2.360960in}}%
\pgfpathlineto{\pgfqpoint{1.924324in}{2.341146in}}%
\pgfpathlineto{\pgfqpoint{1.929860in}{2.297864in}}%
\pgfpathlineto{\pgfqpoint{1.930968in}{2.296445in}}%
\pgfpathlineto{\pgfqpoint{1.931337in}{2.296733in}}%
\pgfpathlineto{\pgfqpoint{1.932813in}{2.301574in}}%
\pgfpathlineto{\pgfqpoint{1.935397in}{2.321387in}}%
\pgfpathlineto{\pgfqpoint{1.940933in}{2.364670in}}%
\pgfpathlineto{\pgfqpoint{1.942040in}{2.366088in}}%
\pgfpathlineto{\pgfqpoint{1.942409in}{2.365800in}}%
\pgfpathlineto{\pgfqpoint{1.943886in}{2.360960in}}%
\pgfpathlineto{\pgfqpoint{1.946469in}{2.341146in}}%
\pgfpathlineto{\pgfqpoint{1.952006in}{2.297864in}}%
\pgfpathlineto{\pgfqpoint{1.953113in}{2.296445in}}%
\pgfpathlineto{\pgfqpoint{1.953482in}{2.296733in}}%
\pgfpathlineto{\pgfqpoint{1.954959in}{2.301574in}}%
\pgfpathlineto{\pgfqpoint{1.957542in}{2.321387in}}%
\pgfpathlineto{\pgfqpoint{1.963079in}{2.364670in}}%
\pgfpathlineto{\pgfqpoint{1.964186in}{2.366088in}}%
\pgfpathlineto{\pgfqpoint{1.964555in}{2.365800in}}%
\pgfpathlineto{\pgfqpoint{1.966031in}{2.360960in}}%
\pgfpathlineto{\pgfqpoint{1.968615in}{2.341146in}}%
\pgfpathlineto{\pgfqpoint{1.974152in}{2.297864in}}%
\pgfpathlineto{\pgfqpoint{1.975259in}{2.296445in}}%
\pgfpathlineto{\pgfqpoint{1.975628in}{2.296733in}}%
\pgfpathlineto{\pgfqpoint{1.977104in}{2.301574in}}%
\pgfpathlineto{\pgfqpoint{1.979688in}{2.321387in}}%
\pgfpathlineto{\pgfqpoint{1.985224in}{2.364670in}}%
\pgfpathlineto{\pgfqpoint{1.986332in}{2.366088in}}%
\pgfpathlineto{\pgfqpoint{1.986701in}{2.365800in}}%
\pgfpathlineto{\pgfqpoint{1.988177in}{2.360960in}}%
\pgfpathlineto{\pgfqpoint{1.990761in}{2.341146in}}%
\pgfpathlineto{\pgfqpoint{1.996297in}{2.297864in}}%
\pgfpathlineto{\pgfqpoint{1.997404in}{2.296446in}}%
\pgfpathlineto{\pgfqpoint{1.997774in}{2.296733in}}%
\pgfpathlineto{\pgfqpoint{1.999250in}{2.301574in}}%
\pgfpathlineto{\pgfqpoint{2.001834in}{2.321387in}}%
\pgfpathlineto{\pgfqpoint{2.007370in}{2.364670in}}%
\pgfpathlineto{\pgfqpoint{2.008477in}{2.366088in}}%
\pgfpathlineto{\pgfqpoint{2.008846in}{2.365800in}}%
\pgfpathlineto{\pgfqpoint{2.010323in}{2.360960in}}%
\pgfpathlineto{\pgfqpoint{2.012906in}{2.341147in}}%
\pgfpathlineto{\pgfqpoint{2.018443in}{2.297865in}}%
\pgfpathlineto{\pgfqpoint{2.019550in}{2.296447in}}%
\pgfpathlineto{\pgfqpoint{2.019919in}{2.296734in}}%
\pgfpathlineto{\pgfqpoint{2.021396in}{2.301575in}}%
\pgfpathlineto{\pgfqpoint{2.023979in}{2.321388in}}%
\pgfpathlineto{\pgfqpoint{2.029516in}{2.364670in}}%
\pgfpathlineto{\pgfqpoint{2.030623in}{2.366088in}}%
\pgfpathlineto{\pgfqpoint{2.030992in}{2.365800in}}%
\pgfpathlineto{\pgfqpoint{2.032468in}{2.360960in}}%
\pgfpathlineto{\pgfqpoint{2.035052in}{2.341147in}}%
\pgfpathlineto{\pgfqpoint{2.040588in}{2.297867in}}%
\pgfpathlineto{\pgfqpoint{2.041696in}{2.296449in}}%
\pgfpathlineto{\pgfqpoint{2.042065in}{2.296737in}}%
\pgfpathlineto{\pgfqpoint{2.043541in}{2.301577in}}%
\pgfpathlineto{\pgfqpoint{2.046125in}{2.321390in}}%
\pgfpathlineto{\pgfqpoint{2.051661in}{2.364670in}}%
\pgfpathlineto{\pgfqpoint{2.052768in}{2.366088in}}%
\pgfpathlineto{\pgfqpoint{2.053138in}{2.365800in}}%
\pgfpathlineto{\pgfqpoint{2.054614in}{2.360960in}}%
\pgfpathlineto{\pgfqpoint{2.057198in}{2.341149in}}%
\pgfpathlineto{\pgfqpoint{2.062734in}{2.297873in}}%
\pgfpathlineto{\pgfqpoint{2.063841in}{2.296455in}}%
\pgfpathlineto{\pgfqpoint{2.064210in}{2.296743in}}%
\pgfpathlineto{\pgfqpoint{2.065687in}{2.301583in}}%
\pgfpathlineto{\pgfqpoint{2.068270in}{2.321394in}}%
\pgfpathlineto{\pgfqpoint{2.073807in}{2.364670in}}%
\pgfpathlineto{\pgfqpoint{2.074914in}{2.366088in}}%
\pgfpathlineto{\pgfqpoint{2.075283in}{2.365800in}}%
\pgfpathlineto{\pgfqpoint{2.076760in}{2.360961in}}%
\pgfpathlineto{\pgfqpoint{2.079343in}{2.341153in}}%
\pgfpathlineto{\pgfqpoint{2.084880in}{2.297886in}}%
\pgfpathlineto{\pgfqpoint{2.085987in}{2.296469in}}%
\pgfpathlineto{\pgfqpoint{2.086356in}{2.296756in}}%
\pgfpathlineto{\pgfqpoint{2.087832in}{2.301596in}}%
\pgfpathlineto{\pgfqpoint{2.090416in}{2.321404in}}%
\pgfpathlineto{\pgfqpoint{2.095952in}{2.364670in}}%
\pgfpathlineto{\pgfqpoint{2.097060in}{2.366088in}}%
\pgfpathlineto{\pgfqpoint{2.097429in}{2.365800in}}%
\pgfpathlineto{\pgfqpoint{2.098905in}{2.360962in}}%
\pgfpathlineto{\pgfqpoint{2.101489in}{2.341161in}}%
\pgfpathlineto{\pgfqpoint{2.107025in}{2.297910in}}%
\pgfpathlineto{\pgfqpoint{2.108133in}{2.296495in}}%
\pgfpathlineto{\pgfqpoint{2.108502in}{2.296783in}}%
\pgfpathlineto{\pgfqpoint{2.109978in}{2.301622in}}%
\pgfpathlineto{\pgfqpoint{2.112562in}{2.321423in}}%
\pgfpathlineto{\pgfqpoint{2.118098in}{2.364671in}}%
\pgfpathlineto{\pgfqpoint{2.119205in}{2.366088in}}%
\pgfpathlineto{\pgfqpoint{2.119574in}{2.365801in}}%
\pgfpathlineto{\pgfqpoint{2.121051in}{2.360965in}}%
\pgfpathlineto{\pgfqpoint{2.123634in}{2.341176in}}%
\pgfpathlineto{\pgfqpoint{2.129171in}{2.297957in}}%
\pgfpathlineto{\pgfqpoint{2.130278in}{2.296544in}}%
\pgfpathlineto{\pgfqpoint{2.130647in}{2.296832in}}%
\pgfpathlineto{\pgfqpoint{2.132124in}{2.301669in}}%
\pgfpathlineto{\pgfqpoint{2.134707in}{2.321458in}}%
\pgfpathlineto{\pgfqpoint{2.140244in}{2.364672in}}%
\pgfpathlineto{\pgfqpoint{2.141351in}{2.366088in}}%
\pgfpathlineto{\pgfqpoint{2.141720in}{2.365801in}}%
\pgfpathlineto{\pgfqpoint{2.143196in}{2.360971in}}%
\pgfpathlineto{\pgfqpoint{2.145780in}{2.341203in}}%
\pgfpathlineto{\pgfqpoint{2.151317in}{2.298041in}}%
\pgfpathlineto{\pgfqpoint{2.152424in}{2.296631in}}%
\pgfpathlineto{\pgfqpoint{2.152793in}{2.296920in}}%
\pgfpathlineto{\pgfqpoint{2.154269in}{2.301754in}}%
\pgfpathlineto{\pgfqpoint{2.156853in}{2.321520in}}%
\pgfpathlineto{\pgfqpoint{2.162389in}{2.364674in}}%
\pgfpathlineto{\pgfqpoint{2.163497in}{2.366088in}}%
\pgfpathlineto{\pgfqpoint{2.163866in}{2.365802in}}%
\pgfpathlineto{\pgfqpoint{2.165342in}{2.360980in}}%
\pgfpathlineto{\pgfqpoint{2.167926in}{2.341249in}}%
\pgfpathlineto{\pgfqpoint{2.173462in}{2.298183in}}%
\pgfpathlineto{\pgfqpoint{2.174569in}{2.296780in}}%
\pgfpathlineto{\pgfqpoint{2.174939in}{2.297069in}}%
\pgfpathlineto{\pgfqpoint{2.176415in}{2.301897in}}%
\pgfpathlineto{\pgfqpoint{2.178999in}{2.321624in}}%
\pgfpathlineto{\pgfqpoint{2.184535in}{2.364677in}}%
\pgfpathlineto{\pgfqpoint{2.185642in}{2.366088in}}%
\pgfpathlineto{\pgfqpoint{2.186011in}{2.365803in}}%
\pgfpathlineto{\pgfqpoint{2.187488in}{2.360996in}}%
\pgfpathlineto{\pgfqpoint{2.190071in}{2.341326in}}%
\pgfpathlineto{\pgfqpoint{2.195608in}{2.298415in}}%
\pgfpathlineto{\pgfqpoint{2.196715in}{2.297022in}}%
\pgfpathlineto{\pgfqpoint{2.197084in}{2.297312in}}%
\pgfpathlineto{\pgfqpoint{2.198560in}{2.302130in}}%
\pgfpathlineto{\pgfqpoint{2.201144in}{2.321794in}}%
\pgfpathlineto{\pgfqpoint{2.206681in}{2.364683in}}%
\pgfpathlineto{\pgfqpoint{2.207788in}{2.366088in}}%
\pgfpathlineto{\pgfqpoint{2.208157in}{2.365804in}}%
\pgfpathlineto{\pgfqpoint{2.209633in}{2.361020in}}%
\pgfpathlineto{\pgfqpoint{2.212217in}{2.341448in}}%
\pgfpathlineto{\pgfqpoint{2.217753in}{2.298784in}}%
\pgfpathlineto{\pgfqpoint{2.218861in}{2.297406in}}%
\pgfpathlineto{\pgfqpoint{2.219230in}{2.297697in}}%
\pgfpathlineto{\pgfqpoint{2.220706in}{2.302497in}}%
\pgfpathlineto{\pgfqpoint{2.223290in}{2.322060in}}%
\pgfpathlineto{\pgfqpoint{2.228826in}{2.364692in}}%
\pgfpathlineto{\pgfqpoint{2.229933in}{2.366088in}}%
\pgfpathlineto{\pgfqpoint{2.230303in}{2.365807in}}%
\pgfpathlineto{\pgfqpoint{2.231779in}{2.361058in}}%
\pgfpathlineto{\pgfqpoint{2.234363in}{2.341637in}}%
\pgfpathlineto{\pgfqpoint{2.239899in}{2.299349in}}%
\pgfpathlineto{\pgfqpoint{2.241006in}{2.297994in}}%
\pgfpathlineto{\pgfqpoint{2.241375in}{2.298286in}}%
\pgfpathlineto{\pgfqpoint{2.242852in}{2.303059in}}%
\pgfpathlineto{\pgfqpoint{2.245435in}{2.322466in}}%
\pgfpathlineto{\pgfqpoint{2.250972in}{2.364705in}}%
\pgfpathlineto{\pgfqpoint{2.252079in}{2.366088in}}%
\pgfpathlineto{\pgfqpoint{2.252448in}{2.365810in}}%
\pgfpathlineto{\pgfqpoint{2.253925in}{2.361115in}}%
\pgfpathlineto{\pgfqpoint{2.256508in}{2.341920in}}%
\pgfpathlineto{\pgfqpoint{2.262045in}{2.300191in}}%
\pgfpathlineto{\pgfqpoint{2.263152in}{2.298869in}}%
\pgfpathlineto{\pgfqpoint{2.263521in}{2.299162in}}%
\pgfpathlineto{\pgfqpoint{2.264997in}{2.303893in}}%
\pgfpathlineto{\pgfqpoint{2.267581in}{2.323068in}}%
\pgfpathlineto{\pgfqpoint{2.273117in}{2.364725in}}%
\pgfpathlineto{\pgfqpoint{2.274225in}{2.366089in}}%
\pgfpathlineto{\pgfqpoint{2.274594in}{2.365815in}}%
\pgfpathlineto{\pgfqpoint{2.276070in}{2.361198in}}%
\pgfpathlineto{\pgfqpoint{2.278654in}{2.342333in}}%
\pgfpathlineto{\pgfqpoint{2.284190in}{2.301412in}}%
\pgfpathlineto{\pgfqpoint{2.285298in}{2.300136in}}%
\pgfpathlineto{\pgfqpoint{2.285667in}{2.300431in}}%
\pgfpathlineto{\pgfqpoint{2.287143in}{2.305099in}}%
\pgfpathlineto{\pgfqpoint{2.289727in}{2.323934in}}%
\pgfpathlineto{\pgfqpoint{2.295263in}{2.364754in}}%
\pgfpathlineto{\pgfqpoint{2.296370in}{2.366089in}}%
\pgfpathlineto{\pgfqpoint{2.296739in}{2.365823in}}%
\pgfpathlineto{\pgfqpoint{2.298216in}{2.361317in}}%
\pgfpathlineto{\pgfqpoint{2.300799in}{2.342918in}}%
\pgfpathlineto{\pgfqpoint{2.306336in}{2.303134in}}%
\pgfpathlineto{\pgfqpoint{2.307443in}{2.301920in}}%
\pgfpathlineto{\pgfqpoint{2.307812in}{2.302217in}}%
\pgfpathlineto{\pgfqpoint{2.309289in}{2.306794in}}%
\pgfpathlineto{\pgfqpoint{2.311872in}{2.325149in}}%
\pgfpathlineto{\pgfqpoint{2.317409in}{2.364793in}}%
\pgfpathlineto{\pgfqpoint{2.318516in}{2.366090in}}%
\pgfpathlineto{\pgfqpoint{2.318885in}{2.365833in}}%
\pgfpathlineto{\pgfqpoint{2.320361in}{2.361480in}}%
\pgfpathlineto{\pgfqpoint{2.322945in}{2.343726in}}%
\pgfpathlineto{\pgfqpoint{2.328482in}{2.305495in}}%
\pgfpathlineto{\pgfqpoint{2.329589in}{2.304364in}}%
\pgfpathlineto{\pgfqpoint{2.329958in}{2.304662in}}%
\pgfpathlineto{\pgfqpoint{2.331434in}{2.309110in}}%
\pgfpathlineto{\pgfqpoint{2.334018in}{2.326805in}}%
\pgfpathlineto{\pgfqpoint{2.339554in}{2.364848in}}%
\pgfpathlineto{\pgfqpoint{2.340662in}{2.366090in}}%
\pgfpathlineto{\pgfqpoint{2.341031in}{2.365847in}}%
\pgfpathlineto{\pgfqpoint{2.342507in}{2.361699in}}%
\pgfpathlineto{\pgfqpoint{2.345091in}{2.344809in}}%
\pgfpathlineto{\pgfqpoint{2.350627in}{2.308639in}}%
\pgfpathlineto{\pgfqpoint{2.351734in}{2.307613in}}%
\pgfpathlineto{\pgfqpoint{2.352104in}{2.307911in}}%
\pgfpathlineto{\pgfqpoint{2.353580in}{2.312182in}}%
\pgfpathlineto{\pgfqpoint{2.356164in}{2.328994in}}%
\pgfpathlineto{\pgfqpoint{2.361700in}{2.364919in}}%
\pgfpathlineto{\pgfqpoint{2.362807in}{2.366091in}}%
\pgfpathlineto{\pgfqpoint{2.363176in}{2.365864in}}%
\pgfpathlineto{\pgfqpoint{2.364653in}{2.361984in}}%
\pgfpathlineto{\pgfqpoint{2.367236in}{2.346215in}}%
\pgfpathlineto{\pgfqpoint{2.372773in}{2.312691in}}%
\pgfpathlineto{\pgfqpoint{2.373880in}{2.311795in}}%
\pgfpathlineto{\pgfqpoint{2.374249in}{2.312090in}}%
\pgfpathlineto{\pgfqpoint{2.375726in}{2.316126in}}%
\pgfpathlineto{\pgfqpoint{2.378678in}{2.334555in}}%
\pgfpathlineto{\pgfqpoint{2.383476in}{2.364092in}}%
\pgfpathlineto{\pgfqpoint{2.384953in}{2.366093in}}%
\pgfpathlineto{\pgfqpoint{2.385322in}{2.365886in}}%
\pgfpathlineto{\pgfqpoint{2.386798in}{2.362343in}}%
\pgfpathlineto{\pgfqpoint{2.389382in}{2.347976in}}%
\pgfpathlineto{\pgfqpoint{2.394549in}{2.318506in}}%
\pgfpathlineto{\pgfqpoint{2.396026in}{2.316982in}}%
\pgfpathlineto{\pgfqpoint{2.397133in}{2.318641in}}%
\pgfpathlineto{\pgfqpoint{2.399348in}{2.328249in}}%
\pgfpathlineto{\pgfqpoint{2.406729in}{2.366023in}}%
\pgfpathlineto{\pgfqpoint{2.407837in}{2.365482in}}%
\pgfpathlineto{\pgfqpoint{2.409682in}{2.359911in}}%
\pgfpathlineto{\pgfqpoint{2.414480in}{2.332633in}}%
\pgfpathlineto{\pgfqpoint{2.417433in}{2.323300in}}%
\pgfpathlineto{\pgfqpoint{2.418540in}{2.323429in}}%
\pgfpathlineto{\pgfqpoint{2.420017in}{2.326784in}}%
\pgfpathlineto{\pgfqpoint{2.422970in}{2.341491in}}%
\pgfpathlineto{\pgfqpoint{2.427768in}{2.364546in}}%
\pgfpathlineto{\pgfqpoint{2.429613in}{2.365942in}}%
\pgfpathlineto{\pgfqpoint{2.431090in}{2.363271in}}%
\pgfpathlineto{\pgfqpoint{2.434042in}{2.350636in}}%
\pgfpathlineto{\pgfqpoint{2.438841in}{2.331061in}}%
\pgfpathlineto{\pgfqpoint{2.440317in}{2.330130in}}%
\pgfpathlineto{\pgfqpoint{2.441793in}{2.332304in}}%
\pgfpathlineto{\pgfqpoint{2.444377in}{2.342079in}}%
\pgfpathlineto{\pgfqpoint{2.450282in}{2.365403in}}%
\pgfpathlineto{\pgfqpoint{2.451759in}{2.365973in}}%
\pgfpathlineto{\pgfqpoint{2.453235in}{2.363812in}}%
\pgfpathlineto{\pgfqpoint{2.456188in}{2.353649in}}%
\pgfpathlineto{\pgfqpoint{2.460986in}{2.338206in}}%
\pgfpathlineto{\pgfqpoint{2.462832in}{2.337805in}}%
\pgfpathlineto{\pgfqpoint{2.464677in}{2.341139in}}%
\pgfpathlineto{\pgfqpoint{2.468737in}{2.355896in}}%
\pgfpathlineto{\pgfqpoint{2.472428in}{2.365558in}}%
\pgfpathlineto{\pgfqpoint{2.474274in}{2.365780in}}%
\pgfpathlineto{\pgfqpoint{2.476119in}{2.362872in}}%
\pgfpathlineto{\pgfqpoint{2.484608in}{2.344968in}}%
\pgfpathlineto{\pgfqpoint{2.486454in}{2.347034in}}%
\pgfpathlineto{\pgfqpoint{2.489775in}{2.355518in}}%
\pgfpathlineto{\pgfqpoint{2.494205in}{2.365382in}}%
\pgfpathlineto{\pgfqpoint{2.496419in}{2.365877in}}%
\pgfpathlineto{\pgfqpoint{2.498634in}{2.363221in}}%
\pgfpathlineto{\pgfqpoint{2.506016in}{2.351646in}}%
\pgfpathlineto{\pgfqpoint{2.507861in}{2.352411in}}%
\pgfpathlineto{\pgfqpoint{2.510814in}{2.356874in}}%
\pgfpathlineto{\pgfqpoint{2.516350in}{2.365627in}}%
\pgfpathlineto{\pgfqpoint{2.518565in}{2.365958in}}%
\pgfpathlineto{\pgfqpoint{2.521148in}{2.363849in}}%
\pgfpathlineto{\pgfqpoint{2.527792in}{2.357245in}}%
\pgfpathlineto{\pgfqpoint{2.530376in}{2.357999in}}%
\pgfpathlineto{\pgfqpoint{2.534805in}{2.362760in}}%
\pgfpathlineto{\pgfqpoint{2.538865in}{2.365948in}}%
\pgfpathlineto{\pgfqpoint{2.541818in}{2.365660in}}%
\pgfpathlineto{\pgfqpoint{2.552521in}{2.361772in}}%
\pgfpathlineto{\pgfqpoint{2.564332in}{2.365799in}}%
\pgfpathlineto{\pgfqpoint{2.574667in}{2.364125in}}%
\pgfpathlineto{\pgfqpoint{2.587954in}{2.365800in}}%
\pgfpathlineto{\pgfqpoint{2.598289in}{2.365498in}}%
\pgfpathlineto{\pgfqpoint{2.612315in}{2.365909in}}%
\pgfpathlineto{\pgfqpoint{2.639258in}{2.366039in}}%
\pgfpathlineto{\pgfqpoint{2.933795in}{2.366100in}}%
\pgfpathlineto{\pgfqpoint{6.857631in}{2.366100in}}%
\pgfpathlineto{\pgfqpoint{6.857631in}{2.366100in}}%
\pgfusepath{stroke}%
\end{pgfscope}%
\begin{pgfscope}%
\pgfsetrectcap%
\pgfsetmiterjoin%
\pgfsetlinewidth{0.803000pt}%
\definecolor{currentstroke}{rgb}{0.000000,0.000000,0.000000}%
\pgfsetstrokecolor{currentstroke}%
\pgfsetdash{}{0pt}%
\pgfpathmoveto{\pgfqpoint{0.952500in}{0.525800in}}%
\pgfpathlineto{\pgfqpoint{0.952500in}{4.206400in}}%
\pgfusepath{stroke}%
\end{pgfscope}%
\begin{pgfscope}%
\pgfsetrectcap%
\pgfsetmiterjoin%
\pgfsetlinewidth{0.803000pt}%
\definecolor{currentstroke}{rgb}{0.000000,0.000000,0.000000}%
\pgfsetstrokecolor{currentstroke}%
\pgfsetdash{}{0pt}%
\pgfpathmoveto{\pgfqpoint{6.858000in}{0.525800in}}%
\pgfpathlineto{\pgfqpoint{6.858000in}{4.206400in}}%
\pgfusepath{stroke}%
\end{pgfscope}%
\begin{pgfscope}%
\pgfsetrectcap%
\pgfsetmiterjoin%
\pgfsetlinewidth{0.803000pt}%
\definecolor{currentstroke}{rgb}{0.000000,0.000000,0.000000}%
\pgfsetstrokecolor{currentstroke}%
\pgfsetdash{}{0pt}%
\pgfpathmoveto{\pgfqpoint{0.952500in}{0.525800in}}%
\pgfpathlineto{\pgfqpoint{6.858000in}{0.525800in}}%
\pgfusepath{stroke}%
\end{pgfscope}%
\begin{pgfscope}%
\pgfsetrectcap%
\pgfsetmiterjoin%
\pgfsetlinewidth{0.803000pt}%
\definecolor{currentstroke}{rgb}{0.000000,0.000000,0.000000}%
\pgfsetstrokecolor{currentstroke}%
\pgfsetdash{}{0pt}%
\pgfpathmoveto{\pgfqpoint{0.952500in}{4.206400in}}%
\pgfpathlineto{\pgfqpoint{6.858000in}{4.206400in}}%
\pgfusepath{stroke}%
\end{pgfscope}%
\begin{pgfscope}%
\pgftext[x=0.952500in,y=4.390430in,left,base]{\sffamily\fontsize{10.000000}{12.000000}\selectfont Iterations: 12935, Time: 0.228 ps, RXPWR: 100.0 percent}%
\end{pgfscope}%
\end{pgfpicture}%
\makeatother%
\endgroup%
}}
        \subcaption{Simulation using a material with positive conductivity.}
        \label{fig:task6_1}
    \end{subfigure}\\
    \begin{subfigure}[b]{0.49\textwidth}
        \noindent\makebox[\textwidth]{\scalebox{0.7}{%% Creator: Matplotlib, PGF backend
%%
%% To include the figure in your LaTeX document, write
%%   \input{<filename>.pgf}
%%
%% Make sure the required packages are loaded in your preamble
%%   \usepackage{pgf}
%%
%% Figures using additional raster images can only be included by \input if
%% they are in the same directory as the main LaTeX file. For loading figures
%% from other directories you can use the `import` package
%%   \usepackage{import}
%% and then include the figures with
%%   \import{<path to file>}{<filename>.pgf}
%%
%% Matplotlib used the following preamble
%%   \usepackage{fontspec}
%%   \setmainfont{DejaVu Serif}
%%   \setsansfont{DejaVu Sans}
%%   \setmonofont{DejaVu Sans Mono}
%%
\begingroup%
\makeatletter%
\begin{pgfpicture}%
\pgfpathrectangle{\pgfpointorigin}{\pgfqpoint{7.590000in}{4.840000in}}%
\pgfusepath{use as bounding box, clip}%
\begin{pgfscope}%
\pgfsetbuttcap%
\pgfsetmiterjoin%
\definecolor{currentfill}{rgb}{1.000000,1.000000,1.000000}%
\pgfsetfillcolor{currentfill}%
\pgfsetlinewidth{0.000000pt}%
\definecolor{currentstroke}{rgb}{1.000000,1.000000,1.000000}%
\pgfsetstrokecolor{currentstroke}%
\pgfsetdash{}{0pt}%
\pgfpathmoveto{\pgfqpoint{0.000000in}{0.000000in}}%
\pgfpathlineto{\pgfqpoint{7.590000in}{0.000000in}}%
\pgfpathlineto{\pgfqpoint{7.590000in}{4.840000in}}%
\pgfpathlineto{\pgfqpoint{0.000000in}{4.840000in}}%
\pgfpathclose%
\pgfusepath{fill}%
\end{pgfscope}%
\begin{pgfscope}%
\pgfsetbuttcap%
\pgfsetmiterjoin%
\definecolor{currentfill}{rgb}{1.000000,1.000000,1.000000}%
\pgfsetfillcolor{currentfill}%
\pgfsetlinewidth{0.000000pt}%
\definecolor{currentstroke}{rgb}{0.000000,0.000000,0.000000}%
\pgfsetstrokecolor{currentstroke}%
\pgfsetstrokeopacity{0.000000}%
\pgfsetdash{}{0pt}%
\pgfpathmoveto{\pgfqpoint{0.948750in}{0.532400in}}%
\pgfpathlineto{\pgfqpoint{6.831000in}{0.532400in}}%
\pgfpathlineto{\pgfqpoint{6.831000in}{4.259200in}}%
\pgfpathlineto{\pgfqpoint{0.948750in}{4.259200in}}%
\pgfpathclose%
\pgfusepath{fill}%
\end{pgfscope}%
\begin{pgfscope}%
\pgfsetbuttcap%
\pgfsetroundjoin%
\definecolor{currentfill}{rgb}{0.000000,0.000000,0.000000}%
\pgfsetfillcolor{currentfill}%
\pgfsetlinewidth{0.803000pt}%
\definecolor{currentstroke}{rgb}{0.000000,0.000000,0.000000}%
\pgfsetstrokecolor{currentstroke}%
\pgfsetdash{}{0pt}%
\pgfsys@defobject{currentmarker}{\pgfqpoint{0.000000in}{-0.048611in}}{\pgfqpoint{0.000000in}{0.000000in}}{%
\pgfpathmoveto{\pgfqpoint{0.000000in}{0.000000in}}%
\pgfpathlineto{\pgfqpoint{0.000000in}{-0.048611in}}%
\pgfusepath{stroke,fill}%
}%
\begin{pgfscope}%
\pgfsys@transformshift{0.948750in}{0.532400in}%
\pgfsys@useobject{currentmarker}{}%
\end{pgfscope}%
\end{pgfscope}%
\begin{pgfscope}%
\pgftext[x=0.948750in,y=0.435178in,,top]{\sffamily\fontsize{10.000000}{12.000000}\selectfont 0}%
\end{pgfscope}%
\begin{pgfscope}%
\pgfsetbuttcap%
\pgfsetroundjoin%
\definecolor{currentfill}{rgb}{0.000000,0.000000,0.000000}%
\pgfsetfillcolor{currentfill}%
\pgfsetlinewidth{0.803000pt}%
\definecolor{currentstroke}{rgb}{0.000000,0.000000,0.000000}%
\pgfsetstrokecolor{currentstroke}%
\pgfsetdash{}{0pt}%
\pgfsys@defobject{currentmarker}{\pgfqpoint{0.000000in}{-0.048611in}}{\pgfqpoint{0.000000in}{0.000000in}}{%
\pgfpathmoveto{\pgfqpoint{0.000000in}{0.000000in}}%
\pgfpathlineto{\pgfqpoint{0.000000in}{-0.048611in}}%
\pgfusepath{stroke,fill}%
}%
\begin{pgfscope}%
\pgfsys@transformshift{1.645699in}{0.532400in}%
\pgfsys@useobject{currentmarker}{}%
\end{pgfscope}%
\end{pgfscope}%
\begin{pgfscope}%
\pgftext[x=1.645699in,y=0.435178in,,top]{\sffamily\fontsize{10.000000}{12.000000}\selectfont 10}%
\end{pgfscope}%
\begin{pgfscope}%
\pgfsetbuttcap%
\pgfsetroundjoin%
\definecolor{currentfill}{rgb}{0.000000,0.000000,0.000000}%
\pgfsetfillcolor{currentfill}%
\pgfsetlinewidth{0.803000pt}%
\definecolor{currentstroke}{rgb}{0.000000,0.000000,0.000000}%
\pgfsetstrokecolor{currentstroke}%
\pgfsetdash{}{0pt}%
\pgfsys@defobject{currentmarker}{\pgfqpoint{0.000000in}{-0.048611in}}{\pgfqpoint{0.000000in}{0.000000in}}{%
\pgfpathmoveto{\pgfqpoint{0.000000in}{0.000000in}}%
\pgfpathlineto{\pgfqpoint{0.000000in}{-0.048611in}}%
\pgfusepath{stroke,fill}%
}%
\begin{pgfscope}%
\pgfsys@transformshift{2.342648in}{0.532400in}%
\pgfsys@useobject{currentmarker}{}%
\end{pgfscope}%
\end{pgfscope}%
\begin{pgfscope}%
\pgftext[x=2.342648in,y=0.435178in,,top]{\sffamily\fontsize{10.000000}{12.000000}\selectfont 20}%
\end{pgfscope}%
\begin{pgfscope}%
\pgfsetbuttcap%
\pgfsetroundjoin%
\definecolor{currentfill}{rgb}{0.000000,0.000000,0.000000}%
\pgfsetfillcolor{currentfill}%
\pgfsetlinewidth{0.803000pt}%
\definecolor{currentstroke}{rgb}{0.000000,0.000000,0.000000}%
\pgfsetstrokecolor{currentstroke}%
\pgfsetdash{}{0pt}%
\pgfsys@defobject{currentmarker}{\pgfqpoint{0.000000in}{-0.048611in}}{\pgfqpoint{0.000000in}{0.000000in}}{%
\pgfpathmoveto{\pgfqpoint{0.000000in}{0.000000in}}%
\pgfpathlineto{\pgfqpoint{0.000000in}{-0.048611in}}%
\pgfusepath{stroke,fill}%
}%
\begin{pgfscope}%
\pgfsys@transformshift{3.039597in}{0.532400in}%
\pgfsys@useobject{currentmarker}{}%
\end{pgfscope}%
\end{pgfscope}%
\begin{pgfscope}%
\pgftext[x=3.039597in,y=0.435178in,,top]{\sffamily\fontsize{10.000000}{12.000000}\selectfont 30}%
\end{pgfscope}%
\begin{pgfscope}%
\pgfsetbuttcap%
\pgfsetroundjoin%
\definecolor{currentfill}{rgb}{0.000000,0.000000,0.000000}%
\pgfsetfillcolor{currentfill}%
\pgfsetlinewidth{0.803000pt}%
\definecolor{currentstroke}{rgb}{0.000000,0.000000,0.000000}%
\pgfsetstrokecolor{currentstroke}%
\pgfsetdash{}{0pt}%
\pgfsys@defobject{currentmarker}{\pgfqpoint{0.000000in}{-0.048611in}}{\pgfqpoint{0.000000in}{0.000000in}}{%
\pgfpathmoveto{\pgfqpoint{0.000000in}{0.000000in}}%
\pgfpathlineto{\pgfqpoint{0.000000in}{-0.048611in}}%
\pgfusepath{stroke,fill}%
}%
\begin{pgfscope}%
\pgfsys@transformshift{3.736546in}{0.532400in}%
\pgfsys@useobject{currentmarker}{}%
\end{pgfscope}%
\end{pgfscope}%
\begin{pgfscope}%
\pgftext[x=3.736546in,y=0.435178in,,top]{\sffamily\fontsize{10.000000}{12.000000}\selectfont 40}%
\end{pgfscope}%
\begin{pgfscope}%
\pgfsetbuttcap%
\pgfsetroundjoin%
\definecolor{currentfill}{rgb}{0.000000,0.000000,0.000000}%
\pgfsetfillcolor{currentfill}%
\pgfsetlinewidth{0.803000pt}%
\definecolor{currentstroke}{rgb}{0.000000,0.000000,0.000000}%
\pgfsetstrokecolor{currentstroke}%
\pgfsetdash{}{0pt}%
\pgfsys@defobject{currentmarker}{\pgfqpoint{0.000000in}{-0.048611in}}{\pgfqpoint{0.000000in}{0.000000in}}{%
\pgfpathmoveto{\pgfqpoint{0.000000in}{0.000000in}}%
\pgfpathlineto{\pgfqpoint{0.000000in}{-0.048611in}}%
\pgfusepath{stroke,fill}%
}%
\begin{pgfscope}%
\pgfsys@transformshift{4.433495in}{0.532400in}%
\pgfsys@useobject{currentmarker}{}%
\end{pgfscope}%
\end{pgfscope}%
\begin{pgfscope}%
\pgftext[x=4.433495in,y=0.435178in,,top]{\sffamily\fontsize{10.000000}{12.000000}\selectfont 50}%
\end{pgfscope}%
\begin{pgfscope}%
\pgfsetbuttcap%
\pgfsetroundjoin%
\definecolor{currentfill}{rgb}{0.000000,0.000000,0.000000}%
\pgfsetfillcolor{currentfill}%
\pgfsetlinewidth{0.803000pt}%
\definecolor{currentstroke}{rgb}{0.000000,0.000000,0.000000}%
\pgfsetstrokecolor{currentstroke}%
\pgfsetdash{}{0pt}%
\pgfsys@defobject{currentmarker}{\pgfqpoint{0.000000in}{-0.048611in}}{\pgfqpoint{0.000000in}{0.000000in}}{%
\pgfpathmoveto{\pgfqpoint{0.000000in}{0.000000in}}%
\pgfpathlineto{\pgfqpoint{0.000000in}{-0.048611in}}%
\pgfusepath{stroke,fill}%
}%
\begin{pgfscope}%
\pgfsys@transformshift{5.130444in}{0.532400in}%
\pgfsys@useobject{currentmarker}{}%
\end{pgfscope}%
\end{pgfscope}%
\begin{pgfscope}%
\pgftext[x=5.130444in,y=0.435178in,,top]{\sffamily\fontsize{10.000000}{12.000000}\selectfont 60}%
\end{pgfscope}%
\begin{pgfscope}%
\pgfsetbuttcap%
\pgfsetroundjoin%
\definecolor{currentfill}{rgb}{0.000000,0.000000,0.000000}%
\pgfsetfillcolor{currentfill}%
\pgfsetlinewidth{0.803000pt}%
\definecolor{currentstroke}{rgb}{0.000000,0.000000,0.000000}%
\pgfsetstrokecolor{currentstroke}%
\pgfsetdash{}{0pt}%
\pgfsys@defobject{currentmarker}{\pgfqpoint{0.000000in}{-0.048611in}}{\pgfqpoint{0.000000in}{0.000000in}}{%
\pgfpathmoveto{\pgfqpoint{0.000000in}{0.000000in}}%
\pgfpathlineto{\pgfqpoint{0.000000in}{-0.048611in}}%
\pgfusepath{stroke,fill}%
}%
\begin{pgfscope}%
\pgfsys@transformshift{5.827393in}{0.532400in}%
\pgfsys@useobject{currentmarker}{}%
\end{pgfscope}%
\end{pgfscope}%
\begin{pgfscope}%
\pgftext[x=5.827393in,y=0.435178in,,top]{\sffamily\fontsize{10.000000}{12.000000}\selectfont 70}%
\end{pgfscope}%
\begin{pgfscope}%
\pgfsetbuttcap%
\pgfsetroundjoin%
\definecolor{currentfill}{rgb}{0.000000,0.000000,0.000000}%
\pgfsetfillcolor{currentfill}%
\pgfsetlinewidth{0.803000pt}%
\definecolor{currentstroke}{rgb}{0.000000,0.000000,0.000000}%
\pgfsetstrokecolor{currentstroke}%
\pgfsetdash{}{0pt}%
\pgfsys@defobject{currentmarker}{\pgfqpoint{0.000000in}{-0.048611in}}{\pgfqpoint{0.000000in}{0.000000in}}{%
\pgfpathmoveto{\pgfqpoint{0.000000in}{0.000000in}}%
\pgfpathlineto{\pgfqpoint{0.000000in}{-0.048611in}}%
\pgfusepath{stroke,fill}%
}%
\begin{pgfscope}%
\pgfsys@transformshift{6.524342in}{0.532400in}%
\pgfsys@useobject{currentmarker}{}%
\end{pgfscope}%
\end{pgfscope}%
\begin{pgfscope}%
\pgftext[x=6.524342in,y=0.435178in,,top]{\sffamily\fontsize{10.000000}{12.000000}\selectfont 80}%
\end{pgfscope}%
\begin{pgfscope}%
\pgftext[x=3.889875in,y=0.245209in,,top]{\sffamily\fontsize{16.000000}{19.200000}\selectfont \(\displaystyle z-position [\mu m]\)}%
\end{pgfscope}%
\begin{pgfscope}%
\pgfsetbuttcap%
\pgfsetroundjoin%
\definecolor{currentfill}{rgb}{0.000000,0.000000,0.000000}%
\pgfsetfillcolor{currentfill}%
\pgfsetlinewidth{0.803000pt}%
\definecolor{currentstroke}{rgb}{0.000000,0.000000,0.000000}%
\pgfsetstrokecolor{currentstroke}%
\pgfsetdash{}{0pt}%
\pgfsys@defobject{currentmarker}{\pgfqpoint{-0.048611in}{0.000000in}}{\pgfqpoint{0.000000in}{0.000000in}}{%
\pgfpathmoveto{\pgfqpoint{0.000000in}{0.000000in}}%
\pgfpathlineto{\pgfqpoint{-0.048611in}{0.000000in}}%
\pgfusepath{stroke,fill}%
}%
\begin{pgfscope}%
\pgfsys@transformshift{0.948750in}{0.532400in}%
\pgfsys@useobject{currentmarker}{}%
\end{pgfscope}%
\end{pgfscope}%
\begin{pgfscope}%
\pgftext[x=0.337544in,y=0.479638in,left,base]{\sffamily\fontsize{10.000000}{12.000000}\selectfont -0.004}%
\end{pgfscope}%
\begin{pgfscope}%
\pgfsetbuttcap%
\pgfsetroundjoin%
\definecolor{currentfill}{rgb}{0.000000,0.000000,0.000000}%
\pgfsetfillcolor{currentfill}%
\pgfsetlinewidth{0.803000pt}%
\definecolor{currentstroke}{rgb}{0.000000,0.000000,0.000000}%
\pgfsetstrokecolor{currentstroke}%
\pgfsetdash{}{0pt}%
\pgfsys@defobject{currentmarker}{\pgfqpoint{-0.048611in}{0.000000in}}{\pgfqpoint{0.000000in}{0.000000in}}{%
\pgfpathmoveto{\pgfqpoint{0.000000in}{0.000000in}}%
\pgfpathlineto{\pgfqpoint{-0.048611in}{0.000000in}}%
\pgfusepath{stroke,fill}%
}%
\begin{pgfscope}%
\pgfsys@transformshift{0.948750in}{0.998250in}%
\pgfsys@useobject{currentmarker}{}%
\end{pgfscope}%
\end{pgfscope}%
\begin{pgfscope}%
\pgftext[x=0.337544in,y=0.945488in,left,base]{\sffamily\fontsize{10.000000}{12.000000}\selectfont -0.003}%
\end{pgfscope}%
\begin{pgfscope}%
\pgfsetbuttcap%
\pgfsetroundjoin%
\definecolor{currentfill}{rgb}{0.000000,0.000000,0.000000}%
\pgfsetfillcolor{currentfill}%
\pgfsetlinewidth{0.803000pt}%
\definecolor{currentstroke}{rgb}{0.000000,0.000000,0.000000}%
\pgfsetstrokecolor{currentstroke}%
\pgfsetdash{}{0pt}%
\pgfsys@defobject{currentmarker}{\pgfqpoint{-0.048611in}{0.000000in}}{\pgfqpoint{0.000000in}{0.000000in}}{%
\pgfpathmoveto{\pgfqpoint{0.000000in}{0.000000in}}%
\pgfpathlineto{\pgfqpoint{-0.048611in}{0.000000in}}%
\pgfusepath{stroke,fill}%
}%
\begin{pgfscope}%
\pgfsys@transformshift{0.948750in}{1.464100in}%
\pgfsys@useobject{currentmarker}{}%
\end{pgfscope}%
\end{pgfscope}%
\begin{pgfscope}%
\pgftext[x=0.337544in,y=1.411338in,left,base]{\sffamily\fontsize{10.000000}{12.000000}\selectfont -0.002}%
\end{pgfscope}%
\begin{pgfscope}%
\pgfsetbuttcap%
\pgfsetroundjoin%
\definecolor{currentfill}{rgb}{0.000000,0.000000,0.000000}%
\pgfsetfillcolor{currentfill}%
\pgfsetlinewidth{0.803000pt}%
\definecolor{currentstroke}{rgb}{0.000000,0.000000,0.000000}%
\pgfsetstrokecolor{currentstroke}%
\pgfsetdash{}{0pt}%
\pgfsys@defobject{currentmarker}{\pgfqpoint{-0.048611in}{0.000000in}}{\pgfqpoint{0.000000in}{0.000000in}}{%
\pgfpathmoveto{\pgfqpoint{0.000000in}{0.000000in}}%
\pgfpathlineto{\pgfqpoint{-0.048611in}{0.000000in}}%
\pgfusepath{stroke,fill}%
}%
\begin{pgfscope}%
\pgfsys@transformshift{0.948750in}{1.929950in}%
\pgfsys@useobject{currentmarker}{}%
\end{pgfscope}%
\end{pgfscope}%
\begin{pgfscope}%
\pgftext[x=0.337544in,y=1.877188in,left,base]{\sffamily\fontsize{10.000000}{12.000000}\selectfont -0.001}%
\end{pgfscope}%
\begin{pgfscope}%
\pgfsetbuttcap%
\pgfsetroundjoin%
\definecolor{currentfill}{rgb}{0.000000,0.000000,0.000000}%
\pgfsetfillcolor{currentfill}%
\pgfsetlinewidth{0.803000pt}%
\definecolor{currentstroke}{rgb}{0.000000,0.000000,0.000000}%
\pgfsetstrokecolor{currentstroke}%
\pgfsetdash{}{0pt}%
\pgfsys@defobject{currentmarker}{\pgfqpoint{-0.048611in}{0.000000in}}{\pgfqpoint{0.000000in}{0.000000in}}{%
\pgfpathmoveto{\pgfqpoint{0.000000in}{0.000000in}}%
\pgfpathlineto{\pgfqpoint{-0.048611in}{0.000000in}}%
\pgfusepath{stroke,fill}%
}%
\begin{pgfscope}%
\pgfsys@transformshift{0.948750in}{2.395800in}%
\pgfsys@useobject{currentmarker}{}%
\end{pgfscope}%
\end{pgfscope}%
\begin{pgfscope}%
\pgftext[x=0.453918in,y=2.343038in,left,base]{\sffamily\fontsize{10.000000}{12.000000}\selectfont 0.000}%
\end{pgfscope}%
\begin{pgfscope}%
\pgfsetbuttcap%
\pgfsetroundjoin%
\definecolor{currentfill}{rgb}{0.000000,0.000000,0.000000}%
\pgfsetfillcolor{currentfill}%
\pgfsetlinewidth{0.803000pt}%
\definecolor{currentstroke}{rgb}{0.000000,0.000000,0.000000}%
\pgfsetstrokecolor{currentstroke}%
\pgfsetdash{}{0pt}%
\pgfsys@defobject{currentmarker}{\pgfqpoint{-0.048611in}{0.000000in}}{\pgfqpoint{0.000000in}{0.000000in}}{%
\pgfpathmoveto{\pgfqpoint{0.000000in}{0.000000in}}%
\pgfpathlineto{\pgfqpoint{-0.048611in}{0.000000in}}%
\pgfusepath{stroke,fill}%
}%
\begin{pgfscope}%
\pgfsys@transformshift{0.948750in}{2.861650in}%
\pgfsys@useobject{currentmarker}{}%
\end{pgfscope}%
\end{pgfscope}%
\begin{pgfscope}%
\pgftext[x=0.453918in,y=2.808888in,left,base]{\sffamily\fontsize{10.000000}{12.000000}\selectfont 0.001}%
\end{pgfscope}%
\begin{pgfscope}%
\pgfsetbuttcap%
\pgfsetroundjoin%
\definecolor{currentfill}{rgb}{0.000000,0.000000,0.000000}%
\pgfsetfillcolor{currentfill}%
\pgfsetlinewidth{0.803000pt}%
\definecolor{currentstroke}{rgb}{0.000000,0.000000,0.000000}%
\pgfsetstrokecolor{currentstroke}%
\pgfsetdash{}{0pt}%
\pgfsys@defobject{currentmarker}{\pgfqpoint{-0.048611in}{0.000000in}}{\pgfqpoint{0.000000in}{0.000000in}}{%
\pgfpathmoveto{\pgfqpoint{0.000000in}{0.000000in}}%
\pgfpathlineto{\pgfqpoint{-0.048611in}{0.000000in}}%
\pgfusepath{stroke,fill}%
}%
\begin{pgfscope}%
\pgfsys@transformshift{0.948750in}{3.327500in}%
\pgfsys@useobject{currentmarker}{}%
\end{pgfscope}%
\end{pgfscope}%
\begin{pgfscope}%
\pgftext[x=0.453918in,y=3.274738in,left,base]{\sffamily\fontsize{10.000000}{12.000000}\selectfont 0.002}%
\end{pgfscope}%
\begin{pgfscope}%
\pgfsetbuttcap%
\pgfsetroundjoin%
\definecolor{currentfill}{rgb}{0.000000,0.000000,0.000000}%
\pgfsetfillcolor{currentfill}%
\pgfsetlinewidth{0.803000pt}%
\definecolor{currentstroke}{rgb}{0.000000,0.000000,0.000000}%
\pgfsetstrokecolor{currentstroke}%
\pgfsetdash{}{0pt}%
\pgfsys@defobject{currentmarker}{\pgfqpoint{-0.048611in}{0.000000in}}{\pgfqpoint{0.000000in}{0.000000in}}{%
\pgfpathmoveto{\pgfqpoint{0.000000in}{0.000000in}}%
\pgfpathlineto{\pgfqpoint{-0.048611in}{0.000000in}}%
\pgfusepath{stroke,fill}%
}%
\begin{pgfscope}%
\pgfsys@transformshift{0.948750in}{3.793350in}%
\pgfsys@useobject{currentmarker}{}%
\end{pgfscope}%
\end{pgfscope}%
\begin{pgfscope}%
\pgftext[x=0.453918in,y=3.740588in,left,base]{\sffamily\fontsize{10.000000}{12.000000}\selectfont 0.003}%
\end{pgfscope}%
\begin{pgfscope}%
\pgfsetbuttcap%
\pgfsetroundjoin%
\definecolor{currentfill}{rgb}{0.000000,0.000000,0.000000}%
\pgfsetfillcolor{currentfill}%
\pgfsetlinewidth{0.803000pt}%
\definecolor{currentstroke}{rgb}{0.000000,0.000000,0.000000}%
\pgfsetstrokecolor{currentstroke}%
\pgfsetdash{}{0pt}%
\pgfsys@defobject{currentmarker}{\pgfqpoint{-0.048611in}{0.000000in}}{\pgfqpoint{0.000000in}{0.000000in}}{%
\pgfpathmoveto{\pgfqpoint{0.000000in}{0.000000in}}%
\pgfpathlineto{\pgfqpoint{-0.048611in}{0.000000in}}%
\pgfusepath{stroke,fill}%
}%
\begin{pgfscope}%
\pgfsys@transformshift{0.948750in}{4.259200in}%
\pgfsys@useobject{currentmarker}{}%
\end{pgfscope}%
\end{pgfscope}%
\begin{pgfscope}%
\pgftext[x=0.453918in,y=4.206438in,left,base]{\sffamily\fontsize{10.000000}{12.000000}\selectfont 0.004}%
\end{pgfscope}%
\begin{pgfscope}%
\pgftext[x=0.281988in,y=2.395800in,,bottom,rotate=90.000000]{\sffamily\fontsize{16.000000}{19.200000}\selectfont \(\displaystyle Poynting\) \(\displaystyle vector\)}%
\end{pgfscope}%
\begin{pgfscope}%
\pgfpathrectangle{\pgfqpoint{0.948750in}{0.532400in}}{\pgfqpoint{5.882250in}{3.726800in}} %
\pgfusepath{clip}%
\pgfsetrectcap%
\pgfsetroundjoin%
\pgfsetlinewidth{1.505625pt}%
\definecolor{currentstroke}{rgb}{0.000000,0.000000,0.000000}%
\pgfsetstrokecolor{currentstroke}%
\pgfsetdash{}{0pt}%
\pgfpathmoveto{\pgfqpoint{0.948750in}{2.395800in}}%
\pgfpathlineto{\pgfqpoint{1.145805in}{2.394754in}}%
\pgfpathlineto{\pgfqpoint{1.152791in}{2.394186in}}%
\pgfpathlineto{\pgfqpoint{1.164555in}{2.395066in}}%
\pgfpathlineto{\pgfqpoint{1.174114in}{2.392247in}}%
\pgfpathlineto{\pgfqpoint{1.179261in}{2.394666in}}%
\pgfpathlineto{\pgfqpoint{1.183672in}{2.395742in}}%
\pgfpathlineto{\pgfqpoint{1.186981in}{2.394121in}}%
\pgfpathlineto{\pgfqpoint{1.194702in}{2.389051in}}%
\pgfpathlineto{\pgfqpoint{1.197643in}{2.390324in}}%
\pgfpathlineto{\pgfqpoint{1.205731in}{2.395701in}}%
\pgfpathlineto{\pgfqpoint{1.208304in}{2.393880in}}%
\pgfpathlineto{\pgfqpoint{1.217128in}{2.385063in}}%
\pgfpathlineto{\pgfqpoint{1.219701in}{2.387154in}}%
\pgfpathlineto{\pgfqpoint{1.227422in}{2.395761in}}%
\pgfpathlineto{\pgfqpoint{1.229627in}{2.394020in}}%
\pgfpathlineto{\pgfqpoint{1.233304in}{2.386979in}}%
\pgfpathlineto{\pgfqpoint{1.237348in}{2.380468in}}%
\pgfpathlineto{\pgfqpoint{1.239186in}{2.380275in}}%
\pgfpathlineto{\pgfqpoint{1.241392in}{2.382771in}}%
\pgfpathlineto{\pgfqpoint{1.249480in}{2.395746in}}%
\pgfpathlineto{\pgfqpoint{1.251318in}{2.393982in}}%
\pgfpathlineto{\pgfqpoint{1.254259in}{2.387019in}}%
\pgfpathlineto{\pgfqpoint{1.259406in}{2.375244in}}%
\pgfpathlineto{\pgfqpoint{1.261245in}{2.375112in}}%
\pgfpathlineto{\pgfqpoint{1.263083in}{2.377750in}}%
\pgfpathlineto{\pgfqpoint{1.267494in}{2.390156in}}%
\pgfpathlineto{\pgfqpoint{1.270436in}{2.395553in}}%
\pgfpathlineto{\pgfqpoint{1.271906in}{2.395530in}}%
\pgfpathlineto{\pgfqpoint{1.273744in}{2.392656in}}%
\pgfpathlineto{\pgfqpoint{1.277421in}{2.380591in}}%
\pgfpathlineto{\pgfqpoint{1.281465in}{2.370021in}}%
\pgfpathlineto{\pgfqpoint{1.282935in}{2.369704in}}%
\pgfpathlineto{\pgfqpoint{1.284774in}{2.372459in}}%
\pgfpathlineto{\pgfqpoint{1.288082in}{2.383505in}}%
\pgfpathlineto{\pgfqpoint{1.292126in}{2.395120in}}%
\pgfpathlineto{\pgfqpoint{1.293965in}{2.395472in}}%
\pgfpathlineto{\pgfqpoint{1.295803in}{2.391997in}}%
\pgfpathlineto{\pgfqpoint{1.299479in}{2.377556in}}%
\pgfpathlineto{\pgfqpoint{1.303156in}{2.365640in}}%
\pgfpathlineto{\pgfqpoint{1.304994in}{2.364871in}}%
\pgfpathlineto{\pgfqpoint{1.306464in}{2.367264in}}%
\pgfpathlineto{\pgfqpoint{1.309405in}{2.378084in}}%
\pgfpathlineto{\pgfqpoint{1.314185in}{2.395009in}}%
\pgfpathlineto{\pgfqpoint{1.315655in}{2.395701in}}%
\pgfpathlineto{\pgfqpoint{1.317126in}{2.393507in}}%
\pgfpathlineto{\pgfqpoint{1.319699in}{2.383914in}}%
\pgfpathlineto{\pgfqpoint{1.325582in}{2.360866in}}%
\pgfpathlineto{\pgfqpoint{1.327052in}{2.360634in}}%
\pgfpathlineto{\pgfqpoint{1.328523in}{2.363433in}}%
\pgfpathlineto{\pgfqpoint{1.331464in}{2.375801in}}%
\pgfpathlineto{\pgfqpoint{1.336243in}{2.394914in}}%
\pgfpathlineto{\pgfqpoint{1.337714in}{2.395690in}}%
\pgfpathlineto{\pgfqpoint{1.339184in}{2.393242in}}%
\pgfpathlineto{\pgfqpoint{1.341758in}{2.382587in}}%
\pgfpathlineto{\pgfqpoint{1.347640in}{2.357277in}}%
\pgfpathlineto{\pgfqpoint{1.349111in}{2.357095in}}%
\pgfpathlineto{\pgfqpoint{1.350581in}{2.360243in}}%
\pgfpathlineto{\pgfqpoint{1.353522in}{2.373911in}}%
\pgfpathlineto{\pgfqpoint{1.358302in}{2.394836in}}%
\pgfpathlineto{\pgfqpoint{1.359772in}{2.395680in}}%
\pgfpathlineto{\pgfqpoint{1.361243in}{2.393025in}}%
\pgfpathlineto{\pgfqpoint{1.363816in}{2.381509in}}%
\pgfpathlineto{\pgfqpoint{1.369331in}{2.354993in}}%
\pgfpathlineto{\pgfqpoint{1.370801in}{2.353949in}}%
\pgfpathlineto{\pgfqpoint{1.372272in}{2.356522in}}%
\pgfpathlineto{\pgfqpoint{1.374845in}{2.368110in}}%
\pgfpathlineto{\pgfqpoint{1.380360in}{2.394774in}}%
\pgfpathlineto{\pgfqpoint{1.381831in}{2.395673in}}%
\pgfpathlineto{\pgfqpoint{1.383301in}{2.392855in}}%
\pgfpathlineto{\pgfqpoint{1.385875in}{2.380666in}}%
\pgfpathlineto{\pgfqpoint{1.391389in}{2.352778in}}%
\pgfpathlineto{\pgfqpoint{1.392860in}{2.351728in}}%
\pgfpathlineto{\pgfqpoint{1.394330in}{2.354484in}}%
\pgfpathlineto{\pgfqpoint{1.396904in}{2.366728in}}%
\pgfpathlineto{\pgfqpoint{1.402419in}{2.394727in}}%
\pgfpathlineto{\pgfqpoint{1.403889in}{2.395667in}}%
\pgfpathlineto{\pgfqpoint{1.405360in}{2.392726in}}%
\pgfpathlineto{\pgfqpoint{1.407933in}{2.380029in}}%
\pgfpathlineto{\pgfqpoint{1.413448in}{2.351116in}}%
\pgfpathlineto{\pgfqpoint{1.414918in}{2.350064in}}%
\pgfpathlineto{\pgfqpoint{1.416389in}{2.352960in}}%
\pgfpathlineto{\pgfqpoint{1.418962in}{2.365699in}}%
\pgfpathlineto{\pgfqpoint{1.424477in}{2.394692in}}%
\pgfpathlineto{\pgfqpoint{1.425948in}{2.395663in}}%
\pgfpathlineto{\pgfqpoint{1.427418in}{2.392631in}}%
\pgfpathlineto{\pgfqpoint{1.429992in}{2.379561in}}%
\pgfpathlineto{\pgfqpoint{1.435506in}{2.349905in}}%
\pgfpathlineto{\pgfqpoint{1.436977in}{2.348854in}}%
\pgfpathlineto{\pgfqpoint{1.438447in}{2.351854in}}%
\pgfpathlineto{\pgfqpoint{1.441021in}{2.364954in}}%
\pgfpathlineto{\pgfqpoint{1.446535in}{2.394667in}}%
\pgfpathlineto{\pgfqpoint{1.448006in}{2.395660in}}%
\pgfpathlineto{\pgfqpoint{1.449477in}{2.392563in}}%
\pgfpathlineto{\pgfqpoint{1.452050in}{2.379227in}}%
\pgfpathlineto{\pgfqpoint{1.457565in}{2.349047in}}%
\pgfpathlineto{\pgfqpoint{1.459035in}{2.347998in}}%
\pgfpathlineto{\pgfqpoint{1.460506in}{2.351073in}}%
\pgfpathlineto{\pgfqpoint{1.463079in}{2.364430in}}%
\pgfpathlineto{\pgfqpoint{1.468594in}{2.394650in}}%
\pgfpathlineto{\pgfqpoint{1.470064in}{2.395657in}}%
\pgfpathlineto{\pgfqpoint{1.471535in}{2.392515in}}%
\pgfpathlineto{\pgfqpoint{1.474108in}{2.378995in}}%
\pgfpathlineto{\pgfqpoint{1.479623in}{2.348455in}}%
\pgfpathlineto{\pgfqpoint{1.481094in}{2.347409in}}%
\pgfpathlineto{\pgfqpoint{1.482564in}{2.350536in}}%
\pgfpathlineto{\pgfqpoint{1.485138in}{2.364071in}}%
\pgfpathlineto{\pgfqpoint{1.490652in}{2.394638in}}%
\pgfpathlineto{\pgfqpoint{1.492123in}{2.395656in}}%
\pgfpathlineto{\pgfqpoint{1.493593in}{2.392483in}}%
\pgfpathlineto{\pgfqpoint{1.496167in}{2.378838in}}%
\pgfpathlineto{\pgfqpoint{1.501682in}{2.348058in}}%
\pgfpathlineto{\pgfqpoint{1.503152in}{2.347014in}}%
\pgfpathlineto{\pgfqpoint{1.504623in}{2.350178in}}%
\pgfpathlineto{\pgfqpoint{1.507196in}{2.363832in}}%
\pgfpathlineto{\pgfqpoint{1.512711in}{2.394630in}}%
\pgfpathlineto{\pgfqpoint{1.514181in}{2.395655in}}%
\pgfpathlineto{\pgfqpoint{1.515652in}{2.392462in}}%
\pgfpathlineto{\pgfqpoint{1.518225in}{2.378736in}}%
\pgfpathlineto{\pgfqpoint{1.523740in}{2.347799in}}%
\pgfpathlineto{\pgfqpoint{1.525211in}{2.346758in}}%
\pgfpathlineto{\pgfqpoint{1.526681in}{2.349945in}}%
\pgfpathlineto{\pgfqpoint{1.529255in}{2.363678in}}%
\pgfpathlineto{\pgfqpoint{1.534769in}{2.394625in}}%
\pgfpathlineto{\pgfqpoint{1.536240in}{2.395654in}}%
\pgfpathlineto{\pgfqpoint{1.537710in}{2.392448in}}%
\pgfpathlineto{\pgfqpoint{1.540284in}{2.378670in}}%
\pgfpathlineto{\pgfqpoint{1.545798in}{2.347636in}}%
\pgfpathlineto{\pgfqpoint{1.547269in}{2.346596in}}%
\pgfpathlineto{\pgfqpoint{1.548740in}{2.349799in}}%
\pgfpathlineto{\pgfqpoint{1.551313in}{2.363581in}}%
\pgfpathlineto{\pgfqpoint{1.556828in}{2.394622in}}%
\pgfpathlineto{\pgfqpoint{1.558298in}{2.395654in}}%
\pgfpathlineto{\pgfqpoint{1.559769in}{2.392440in}}%
\pgfpathlineto{\pgfqpoint{1.562342in}{2.378630in}}%
\pgfpathlineto{\pgfqpoint{1.567857in}{2.347536in}}%
\pgfpathlineto{\pgfqpoint{1.569327in}{2.346498in}}%
\pgfpathlineto{\pgfqpoint{1.570798in}{2.349710in}}%
\pgfpathlineto{\pgfqpoint{1.573371in}{2.363523in}}%
\pgfpathlineto{\pgfqpoint{1.578886in}{2.394620in}}%
\pgfpathlineto{\pgfqpoint{1.580357in}{2.395654in}}%
\pgfpathlineto{\pgfqpoint{1.581827in}{2.392435in}}%
\pgfpathlineto{\pgfqpoint{1.584401in}{2.378606in}}%
\pgfpathlineto{\pgfqpoint{1.589915in}{2.347478in}}%
\pgfpathlineto{\pgfqpoint{1.591386in}{2.346441in}}%
\pgfpathlineto{\pgfqpoint{1.592856in}{2.349658in}}%
\pgfpathlineto{\pgfqpoint{1.595430in}{2.363489in}}%
\pgfpathlineto{\pgfqpoint{1.600944in}{2.394619in}}%
\pgfpathlineto{\pgfqpoint{1.602415in}{2.395654in}}%
\pgfpathlineto{\pgfqpoint{1.603886in}{2.392432in}}%
\pgfpathlineto{\pgfqpoint{1.606459in}{2.378592in}}%
\pgfpathlineto{\pgfqpoint{1.611974in}{2.347445in}}%
\pgfpathlineto{\pgfqpoint{1.613444in}{2.346408in}}%
\pgfpathlineto{\pgfqpoint{1.614915in}{2.349629in}}%
\pgfpathlineto{\pgfqpoint{1.617488in}{2.363470in}}%
\pgfpathlineto{\pgfqpoint{1.623003in}{2.394618in}}%
\pgfpathlineto{\pgfqpoint{1.624473in}{2.395654in}}%
\pgfpathlineto{\pgfqpoint{1.625944in}{2.392431in}}%
\pgfpathlineto{\pgfqpoint{1.628518in}{2.378585in}}%
\pgfpathlineto{\pgfqpoint{1.634032in}{2.347428in}}%
\pgfpathlineto{\pgfqpoint{1.635503in}{2.346391in}}%
\pgfpathlineto{\pgfqpoint{1.636973in}{2.349614in}}%
\pgfpathlineto{\pgfqpoint{1.639547in}{2.363460in}}%
\pgfpathlineto{\pgfqpoint{1.645061in}{2.394618in}}%
\pgfpathlineto{\pgfqpoint{1.646532in}{2.395654in}}%
\pgfpathlineto{\pgfqpoint{1.648002in}{2.392430in}}%
\pgfpathlineto{\pgfqpoint{1.650576in}{2.378581in}}%
\pgfpathlineto{\pgfqpoint{1.656091in}{2.347419in}}%
\pgfpathlineto{\pgfqpoint{1.657561in}{2.346383in}}%
\pgfpathlineto{\pgfqpoint{1.659032in}{2.349607in}}%
\pgfpathlineto{\pgfqpoint{1.661605in}{2.363455in}}%
\pgfpathlineto{\pgfqpoint{1.667120in}{2.394618in}}%
\pgfpathlineto{\pgfqpoint{1.668590in}{2.395654in}}%
\pgfpathlineto{\pgfqpoint{1.670061in}{2.392429in}}%
\pgfpathlineto{\pgfqpoint{1.672634in}{2.378580in}}%
\pgfpathlineto{\pgfqpoint{1.678149in}{2.347415in}}%
\pgfpathlineto{\pgfqpoint{1.679620in}{2.346379in}}%
\pgfpathlineto{\pgfqpoint{1.681090in}{2.349603in}}%
\pgfpathlineto{\pgfqpoint{1.683664in}{2.363453in}}%
\pgfpathlineto{\pgfqpoint{1.689178in}{2.394618in}}%
\pgfpathlineto{\pgfqpoint{1.690649in}{2.395654in}}%
\pgfpathlineto{\pgfqpoint{1.692119in}{2.392429in}}%
\pgfpathlineto{\pgfqpoint{1.694693in}{2.378579in}}%
\pgfpathlineto{\pgfqpoint{1.700207in}{2.347413in}}%
\pgfpathlineto{\pgfqpoint{1.701678in}{2.346377in}}%
\pgfpathlineto{\pgfqpoint{1.703149in}{2.349602in}}%
\pgfpathlineto{\pgfqpoint{1.705722in}{2.363452in}}%
\pgfpathlineto{\pgfqpoint{1.711237in}{2.394618in}}%
\pgfpathlineto{\pgfqpoint{1.712707in}{2.395654in}}%
\pgfpathlineto{\pgfqpoint{1.714178in}{2.392429in}}%
\pgfpathlineto{\pgfqpoint{1.716751in}{2.378579in}}%
\pgfpathlineto{\pgfqpoint{1.722266in}{2.347413in}}%
\pgfpathlineto{\pgfqpoint{1.723736in}{2.346377in}}%
\pgfpathlineto{\pgfqpoint{1.725207in}{2.349601in}}%
\pgfpathlineto{\pgfqpoint{1.727780in}{2.363452in}}%
\pgfpathlineto{\pgfqpoint{1.733295in}{2.394618in}}%
\pgfpathlineto{\pgfqpoint{1.734766in}{2.395654in}}%
\pgfpathlineto{\pgfqpoint{1.736236in}{2.392429in}}%
\pgfpathlineto{\pgfqpoint{1.738810in}{2.378579in}}%
\pgfpathlineto{\pgfqpoint{1.744324in}{2.347412in}}%
\pgfpathlineto{\pgfqpoint{1.745795in}{2.346377in}}%
\pgfpathlineto{\pgfqpoint{1.747265in}{2.349601in}}%
\pgfpathlineto{\pgfqpoint{1.749839in}{2.363452in}}%
\pgfpathlineto{\pgfqpoint{1.755354in}{2.394618in}}%
\pgfpathlineto{\pgfqpoint{1.756824in}{2.395654in}}%
\pgfpathlineto{\pgfqpoint{1.758295in}{2.392429in}}%
\pgfpathlineto{\pgfqpoint{1.760868in}{2.378579in}}%
\pgfpathlineto{\pgfqpoint{1.766383in}{2.347412in}}%
\pgfpathlineto{\pgfqpoint{1.767853in}{2.346377in}}%
\pgfpathlineto{\pgfqpoint{1.769324in}{2.349601in}}%
\pgfpathlineto{\pgfqpoint{1.771897in}{2.363452in}}%
\pgfpathlineto{\pgfqpoint{1.777412in}{2.394618in}}%
\pgfpathlineto{\pgfqpoint{1.778883in}{2.395654in}}%
\pgfpathlineto{\pgfqpoint{1.780353in}{2.392429in}}%
\pgfpathlineto{\pgfqpoint{1.782927in}{2.378579in}}%
\pgfpathlineto{\pgfqpoint{1.788441in}{2.347412in}}%
\pgfpathlineto{\pgfqpoint{1.789912in}{2.346377in}}%
\pgfpathlineto{\pgfqpoint{1.791382in}{2.349601in}}%
\pgfpathlineto{\pgfqpoint{1.793956in}{2.363452in}}%
\pgfpathlineto{\pgfqpoint{1.799470in}{2.394618in}}%
\pgfpathlineto{\pgfqpoint{1.800941in}{2.395654in}}%
\pgfpathlineto{\pgfqpoint{1.802412in}{2.392429in}}%
\pgfpathlineto{\pgfqpoint{1.804985in}{2.378579in}}%
\pgfpathlineto{\pgfqpoint{1.810500in}{2.347412in}}%
\pgfpathlineto{\pgfqpoint{1.811970in}{2.346377in}}%
\pgfpathlineto{\pgfqpoint{1.813441in}{2.349601in}}%
\pgfpathlineto{\pgfqpoint{1.816014in}{2.363452in}}%
\pgfpathlineto{\pgfqpoint{1.821529in}{2.394618in}}%
\pgfpathlineto{\pgfqpoint{1.822999in}{2.395654in}}%
\pgfpathlineto{\pgfqpoint{1.824470in}{2.392429in}}%
\pgfpathlineto{\pgfqpoint{1.827043in}{2.378579in}}%
\pgfpathlineto{\pgfqpoint{1.832558in}{2.347412in}}%
\pgfpathlineto{\pgfqpoint{1.834029in}{2.346377in}}%
\pgfpathlineto{\pgfqpoint{1.835499in}{2.349601in}}%
\pgfpathlineto{\pgfqpoint{1.838073in}{2.363452in}}%
\pgfpathlineto{\pgfqpoint{1.843587in}{2.394618in}}%
\pgfpathlineto{\pgfqpoint{1.845058in}{2.395654in}}%
\pgfpathlineto{\pgfqpoint{1.846528in}{2.392429in}}%
\pgfpathlineto{\pgfqpoint{1.849102in}{2.378579in}}%
\pgfpathlineto{\pgfqpoint{1.854616in}{2.347412in}}%
\pgfpathlineto{\pgfqpoint{1.856087in}{2.346377in}}%
\pgfpathlineto{\pgfqpoint{1.857558in}{2.349601in}}%
\pgfpathlineto{\pgfqpoint{1.860131in}{2.363452in}}%
\pgfpathlineto{\pgfqpoint{1.865646in}{2.394618in}}%
\pgfpathlineto{\pgfqpoint{1.867116in}{2.395654in}}%
\pgfpathlineto{\pgfqpoint{1.868587in}{2.392429in}}%
\pgfpathlineto{\pgfqpoint{1.871160in}{2.378579in}}%
\pgfpathlineto{\pgfqpoint{1.876675in}{2.347412in}}%
\pgfpathlineto{\pgfqpoint{1.878146in}{2.346377in}}%
\pgfpathlineto{\pgfqpoint{1.879616in}{2.349601in}}%
\pgfpathlineto{\pgfqpoint{1.882190in}{2.363452in}}%
\pgfpathlineto{\pgfqpoint{1.887704in}{2.394618in}}%
\pgfpathlineto{\pgfqpoint{1.889175in}{2.395654in}}%
\pgfpathlineto{\pgfqpoint{1.890645in}{2.392429in}}%
\pgfpathlineto{\pgfqpoint{1.893219in}{2.378579in}}%
\pgfpathlineto{\pgfqpoint{1.898733in}{2.347412in}}%
\pgfpathlineto{\pgfqpoint{1.900204in}{2.346377in}}%
\pgfpathlineto{\pgfqpoint{1.901675in}{2.349601in}}%
\pgfpathlineto{\pgfqpoint{1.904248in}{2.363452in}}%
\pgfpathlineto{\pgfqpoint{1.909763in}{2.394618in}}%
\pgfpathlineto{\pgfqpoint{1.911233in}{2.395654in}}%
\pgfpathlineto{\pgfqpoint{1.912704in}{2.392429in}}%
\pgfpathlineto{\pgfqpoint{1.915277in}{2.378579in}}%
\pgfpathlineto{\pgfqpoint{1.920792in}{2.347412in}}%
\pgfpathlineto{\pgfqpoint{1.922262in}{2.346377in}}%
\pgfpathlineto{\pgfqpoint{1.923733in}{2.349601in}}%
\pgfpathlineto{\pgfqpoint{1.926306in}{2.363452in}}%
\pgfpathlineto{\pgfqpoint{1.931821in}{2.394618in}}%
\pgfpathlineto{\pgfqpoint{1.933292in}{2.395654in}}%
\pgfpathlineto{\pgfqpoint{1.934762in}{2.392429in}}%
\pgfpathlineto{\pgfqpoint{1.937336in}{2.378579in}}%
\pgfpathlineto{\pgfqpoint{1.942850in}{2.347412in}}%
\pgfpathlineto{\pgfqpoint{1.944321in}{2.346377in}}%
\pgfpathlineto{\pgfqpoint{1.945791in}{2.349601in}}%
\pgfpathlineto{\pgfqpoint{1.948365in}{2.363452in}}%
\pgfpathlineto{\pgfqpoint{1.953879in}{2.394618in}}%
\pgfpathlineto{\pgfqpoint{1.955350in}{2.395654in}}%
\pgfpathlineto{\pgfqpoint{1.956821in}{2.392429in}}%
\pgfpathlineto{\pgfqpoint{1.959394in}{2.378579in}}%
\pgfpathlineto{\pgfqpoint{1.964909in}{2.347412in}}%
\pgfpathlineto{\pgfqpoint{1.966379in}{2.346377in}}%
\pgfpathlineto{\pgfqpoint{1.967850in}{2.349601in}}%
\pgfpathlineto{\pgfqpoint{1.970423in}{2.363452in}}%
\pgfpathlineto{\pgfqpoint{1.975938in}{2.394618in}}%
\pgfpathlineto{\pgfqpoint{1.977408in}{2.395654in}}%
\pgfpathlineto{\pgfqpoint{1.978879in}{2.392429in}}%
\pgfpathlineto{\pgfqpoint{1.981453in}{2.378579in}}%
\pgfpathlineto{\pgfqpoint{1.986967in}{2.347413in}}%
\pgfpathlineto{\pgfqpoint{1.988438in}{2.346377in}}%
\pgfpathlineto{\pgfqpoint{1.989908in}{2.349601in}}%
\pgfpathlineto{\pgfqpoint{1.992482in}{2.363452in}}%
\pgfpathlineto{\pgfqpoint{1.997996in}{2.394618in}}%
\pgfpathlineto{\pgfqpoint{1.999467in}{2.395654in}}%
\pgfpathlineto{\pgfqpoint{2.000937in}{2.392429in}}%
\pgfpathlineto{\pgfqpoint{2.003511in}{2.378579in}}%
\pgfpathlineto{\pgfqpoint{2.009026in}{2.347413in}}%
\pgfpathlineto{\pgfqpoint{2.010496in}{2.346377in}}%
\pgfpathlineto{\pgfqpoint{2.011967in}{2.349602in}}%
\pgfpathlineto{\pgfqpoint{2.014540in}{2.363452in}}%
\pgfpathlineto{\pgfqpoint{2.020055in}{2.394618in}}%
\pgfpathlineto{\pgfqpoint{2.021525in}{2.395654in}}%
\pgfpathlineto{\pgfqpoint{2.022996in}{2.392429in}}%
\pgfpathlineto{\pgfqpoint{2.025569in}{2.378579in}}%
\pgfpathlineto{\pgfqpoint{2.031084in}{2.347415in}}%
\pgfpathlineto{\pgfqpoint{2.032555in}{2.346379in}}%
\pgfpathlineto{\pgfqpoint{2.034025in}{2.349604in}}%
\pgfpathlineto{\pgfqpoint{2.036599in}{2.363454in}}%
\pgfpathlineto{\pgfqpoint{2.042113in}{2.394618in}}%
\pgfpathlineto{\pgfqpoint{2.043584in}{2.395654in}}%
\pgfpathlineto{\pgfqpoint{2.045054in}{2.392429in}}%
\pgfpathlineto{\pgfqpoint{2.047628in}{2.378580in}}%
\pgfpathlineto{\pgfqpoint{2.053142in}{2.347418in}}%
\pgfpathlineto{\pgfqpoint{2.054613in}{2.346383in}}%
\pgfpathlineto{\pgfqpoint{2.056084in}{2.349608in}}%
\pgfpathlineto{\pgfqpoint{2.058657in}{2.363457in}}%
\pgfpathlineto{\pgfqpoint{2.064172in}{2.394618in}}%
\pgfpathlineto{\pgfqpoint{2.065642in}{2.395654in}}%
\pgfpathlineto{\pgfqpoint{2.067113in}{2.392430in}}%
\pgfpathlineto{\pgfqpoint{2.069686in}{2.378583in}}%
\pgfpathlineto{\pgfqpoint{2.075201in}{2.347426in}}%
\pgfpathlineto{\pgfqpoint{2.076671in}{2.346392in}}%
\pgfpathlineto{\pgfqpoint{2.078142in}{2.349616in}}%
\pgfpathlineto{\pgfqpoint{2.080715in}{2.363463in}}%
\pgfpathlineto{\pgfqpoint{2.086230in}{2.394618in}}%
\pgfpathlineto{\pgfqpoint{2.087701in}{2.395654in}}%
\pgfpathlineto{\pgfqpoint{2.089171in}{2.392431in}}%
\pgfpathlineto{\pgfqpoint{2.091745in}{2.378588in}}%
\pgfpathlineto{\pgfqpoint{2.097259in}{2.347443in}}%
\pgfpathlineto{\pgfqpoint{2.098730in}{2.346409in}}%
\pgfpathlineto{\pgfqpoint{2.100200in}{2.349633in}}%
\pgfpathlineto{\pgfqpoint{2.102774in}{2.363476in}}%
\pgfpathlineto{\pgfqpoint{2.108289in}{2.394619in}}%
\pgfpathlineto{\pgfqpoint{2.109759in}{2.395654in}}%
\pgfpathlineto{\pgfqpoint{2.111230in}{2.392432in}}%
\pgfpathlineto{\pgfqpoint{2.113803in}{2.378597in}}%
\pgfpathlineto{\pgfqpoint{2.119318in}{2.347474in}}%
\pgfpathlineto{\pgfqpoint{2.120788in}{2.346442in}}%
\pgfpathlineto{\pgfqpoint{2.122259in}{2.349665in}}%
\pgfpathlineto{\pgfqpoint{2.124832in}{2.363500in}}%
\pgfpathlineto{\pgfqpoint{2.130347in}{2.394620in}}%
\pgfpathlineto{\pgfqpoint{2.131818in}{2.395654in}}%
\pgfpathlineto{\pgfqpoint{2.133288in}{2.392436in}}%
\pgfpathlineto{\pgfqpoint{2.135862in}{2.378614in}}%
\pgfpathlineto{\pgfqpoint{2.141376in}{2.347529in}}%
\pgfpathlineto{\pgfqpoint{2.142847in}{2.346500in}}%
\pgfpathlineto{\pgfqpoint{2.144317in}{2.349722in}}%
\pgfpathlineto{\pgfqpoint{2.146891in}{2.363542in}}%
\pgfpathlineto{\pgfqpoint{2.152405in}{2.394622in}}%
\pgfpathlineto{\pgfqpoint{2.153876in}{2.395654in}}%
\pgfpathlineto{\pgfqpoint{2.155347in}{2.392441in}}%
\pgfpathlineto{\pgfqpoint{2.157920in}{2.378643in}}%
\pgfpathlineto{\pgfqpoint{2.163435in}{2.347623in}}%
\pgfpathlineto{\pgfqpoint{2.164905in}{2.346600in}}%
\pgfpathlineto{\pgfqpoint{2.166376in}{2.349818in}}%
\pgfpathlineto{\pgfqpoint{2.168949in}{2.363614in}}%
\pgfpathlineto{\pgfqpoint{2.174464in}{2.394625in}}%
\pgfpathlineto{\pgfqpoint{2.175934in}{2.395654in}}%
\pgfpathlineto{\pgfqpoint{2.177405in}{2.392450in}}%
\pgfpathlineto{\pgfqpoint{2.179978in}{2.378692in}}%
\pgfpathlineto{\pgfqpoint{2.185493in}{2.347779in}}%
\pgfpathlineto{\pgfqpoint{2.186964in}{2.346764in}}%
\pgfpathlineto{\pgfqpoint{2.188434in}{2.349977in}}%
\pgfpathlineto{\pgfqpoint{2.191008in}{2.363731in}}%
\pgfpathlineto{\pgfqpoint{2.196522in}{2.394629in}}%
\pgfpathlineto{\pgfqpoint{2.197993in}{2.395655in}}%
\pgfpathlineto{\pgfqpoint{2.199463in}{2.392465in}}%
\pgfpathlineto{\pgfqpoint{2.202037in}{2.378771in}}%
\pgfpathlineto{\pgfqpoint{2.207552in}{2.348026in}}%
\pgfpathlineto{\pgfqpoint{2.209022in}{2.347025in}}%
\pgfpathlineto{\pgfqpoint{2.210493in}{2.350227in}}%
\pgfpathlineto{\pgfqpoint{2.213066in}{2.363916in}}%
\pgfpathlineto{\pgfqpoint{2.218581in}{2.394637in}}%
\pgfpathlineto{\pgfqpoint{2.220051in}{2.395656in}}%
\pgfpathlineto{\pgfqpoint{2.221522in}{2.392487in}}%
\pgfpathlineto{\pgfqpoint{2.224095in}{2.378893in}}%
\pgfpathlineto{\pgfqpoint{2.229610in}{2.348407in}}%
\pgfpathlineto{\pgfqpoint{2.231081in}{2.347424in}}%
\pgfpathlineto{\pgfqpoint{2.232551in}{2.350611in}}%
\pgfpathlineto{\pgfqpoint{2.235125in}{2.364198in}}%
\pgfpathlineto{\pgfqpoint{2.240639in}{2.394648in}}%
\pgfpathlineto{\pgfqpoint{2.242110in}{2.395657in}}%
\pgfpathlineto{\pgfqpoint{2.243580in}{2.392522in}}%
\pgfpathlineto{\pgfqpoint{2.246154in}{2.379077in}}%
\pgfpathlineto{\pgfqpoint{2.251668in}{2.348976in}}%
\pgfpathlineto{\pgfqpoint{2.253139in}{2.348021in}}%
\pgfpathlineto{\pgfqpoint{2.254609in}{2.351183in}}%
\pgfpathlineto{\pgfqpoint{2.257183in}{2.364617in}}%
\pgfpathlineto{\pgfqpoint{2.262698in}{2.394665in}}%
\pgfpathlineto{\pgfqpoint{2.264168in}{2.395660in}}%
\pgfpathlineto{\pgfqpoint{2.265639in}{2.392572in}}%
\pgfpathlineto{\pgfqpoint{2.268212in}{2.379345in}}%
\pgfpathlineto{\pgfqpoint{2.273727in}{2.349803in}}%
\pgfpathlineto{\pgfqpoint{2.275197in}{2.348887in}}%
\pgfpathlineto{\pgfqpoint{2.276668in}{2.352012in}}%
\pgfpathlineto{\pgfqpoint{2.279241in}{2.365222in}}%
\pgfpathlineto{\pgfqpoint{2.284756in}{2.394689in}}%
\pgfpathlineto{\pgfqpoint{2.286227in}{2.395663in}}%
\pgfpathlineto{\pgfqpoint{2.287697in}{2.392644in}}%
\pgfpathlineto{\pgfqpoint{2.290271in}{2.379728in}}%
\pgfpathlineto{\pgfqpoint{2.295785in}{2.350974in}}%
\pgfpathlineto{\pgfqpoint{2.297256in}{2.350110in}}%
\pgfpathlineto{\pgfqpoint{2.298726in}{2.353181in}}%
\pgfpathlineto{\pgfqpoint{2.301300in}{2.366073in}}%
\pgfpathlineto{\pgfqpoint{2.306814in}{2.394723in}}%
\pgfpathlineto{\pgfqpoint{2.308285in}{2.395667in}}%
\pgfpathlineto{\pgfqpoint{2.309756in}{2.392744in}}%
\pgfpathlineto{\pgfqpoint{2.312329in}{2.380259in}}%
\pgfpathlineto{\pgfqpoint{2.317844in}{2.352586in}}%
\pgfpathlineto{\pgfqpoint{2.319314in}{2.351790in}}%
\pgfpathlineto{\pgfqpoint{2.320785in}{2.354784in}}%
\pgfpathlineto{\pgfqpoint{2.323358in}{2.367235in}}%
\pgfpathlineto{\pgfqpoint{2.328873in}{2.394769in}}%
\pgfpathlineto{\pgfqpoint{2.330343in}{2.395673in}}%
\pgfpathlineto{\pgfqpoint{2.331814in}{2.392879in}}%
\pgfpathlineto{\pgfqpoint{2.334388in}{2.380974in}}%
\pgfpathlineto{\pgfqpoint{2.339902in}{2.354740in}}%
\pgfpathlineto{\pgfqpoint{2.341373in}{2.354030in}}%
\pgfpathlineto{\pgfqpoint{2.342843in}{2.356916in}}%
\pgfpathlineto{\pgfqpoint{2.345417in}{2.368776in}}%
\pgfpathlineto{\pgfqpoint{2.350931in}{2.394829in}}%
\pgfpathlineto{\pgfqpoint{2.352402in}{2.395680in}}%
\pgfpathlineto{\pgfqpoint{2.353872in}{2.393057in}}%
\pgfpathlineto{\pgfqpoint{2.356446in}{2.381907in}}%
\pgfpathlineto{\pgfqpoint{2.361961in}{2.357528in}}%
\pgfpathlineto{\pgfqpoint{2.363431in}{2.356924in}}%
\pgfpathlineto{\pgfqpoint{2.364902in}{2.359665in}}%
\pgfpathlineto{\pgfqpoint{2.367475in}{2.370753in}}%
\pgfpathlineto{\pgfqpoint{2.372990in}{2.394906in}}%
\pgfpathlineto{\pgfqpoint{2.374460in}{2.395690in}}%
\pgfpathlineto{\pgfqpoint{2.375931in}{2.393281in}}%
\pgfpathlineto{\pgfqpoint{2.378504in}{2.383083in}}%
\pgfpathlineto{\pgfqpoint{2.384019in}{2.361010in}}%
\pgfpathlineto{\pgfqpoint{2.385490in}{2.360528in}}%
\pgfpathlineto{\pgfqpoint{2.386960in}{2.363078in}}%
\pgfpathlineto{\pgfqpoint{2.389901in}{2.374993in}}%
\pgfpathlineto{\pgfqpoint{2.395048in}{2.394999in}}%
\pgfpathlineto{\pgfqpoint{2.396519in}{2.395701in}}%
\pgfpathlineto{\pgfqpoint{2.397989in}{2.393554in}}%
\pgfpathlineto{\pgfqpoint{2.400563in}{2.384508in}}%
\pgfpathlineto{\pgfqpoint{2.406077in}{2.365181in}}%
\pgfpathlineto{\pgfqpoint{2.407548in}{2.364834in}}%
\pgfpathlineto{\pgfqpoint{2.409019in}{2.367144in}}%
\pgfpathlineto{\pgfqpoint{2.411960in}{2.377672in}}%
\pgfpathlineto{\pgfqpoint{2.416739in}{2.394568in}}%
\pgfpathlineto{\pgfqpoint{2.418577in}{2.395715in}}%
\pgfpathlineto{\pgfqpoint{2.420048in}{2.393872in}}%
\pgfpathlineto{\pgfqpoint{2.422989in}{2.384785in}}%
\pgfpathlineto{\pgfqpoint{2.427768in}{2.370354in}}%
\pgfpathlineto{\pgfqpoint{2.429606in}{2.369732in}}%
\pgfpathlineto{\pgfqpoint{2.431445in}{2.372572in}}%
\pgfpathlineto{\pgfqpoint{2.435121in}{2.384726in}}%
\pgfpathlineto{\pgfqpoint{2.439165in}{2.395231in}}%
\pgfpathlineto{\pgfqpoint{2.441003in}{2.395533in}}%
\pgfpathlineto{\pgfqpoint{2.442841in}{2.392813in}}%
\pgfpathlineto{\pgfqpoint{2.451665in}{2.374991in}}%
\pgfpathlineto{\pgfqpoint{2.453503in}{2.377350in}}%
\pgfpathlineto{\pgfqpoint{2.457547in}{2.388139in}}%
\pgfpathlineto{\pgfqpoint{2.461224in}{2.395358in}}%
\pgfpathlineto{\pgfqpoint{2.463062in}{2.395594in}}%
\pgfpathlineto{\pgfqpoint{2.465268in}{2.392863in}}%
\pgfpathlineto{\pgfqpoint{2.472988in}{2.380081in}}%
\pgfpathlineto{\pgfqpoint{2.474826in}{2.381146in}}%
\pgfpathlineto{\pgfqpoint{2.477767in}{2.386241in}}%
\pgfpathlineto{\pgfqpoint{2.482914in}{2.395229in}}%
\pgfpathlineto{\pgfqpoint{2.485120in}{2.395652in}}%
\pgfpathlineto{\pgfqpoint{2.487694in}{2.393217in}}%
\pgfpathlineto{\pgfqpoint{2.494679in}{2.384972in}}%
\pgfpathlineto{\pgfqpoint{2.496885in}{2.385772in}}%
\pgfpathlineto{\pgfqpoint{2.500561in}{2.390434in}}%
\pgfpathlineto{\pgfqpoint{2.505340in}{2.395590in}}%
\pgfpathlineto{\pgfqpoint{2.507914in}{2.395431in}}%
\pgfpathlineto{\pgfqpoint{2.511590in}{2.392334in}}%
\pgfpathlineto{\pgfqpoint{2.516370in}{2.389073in}}%
\pgfpathlineto{\pgfqpoint{2.519311in}{2.389821in}}%
\pgfpathlineto{\pgfqpoint{2.529972in}{2.395586in}}%
\pgfpathlineto{\pgfqpoint{2.542104in}{2.392839in}}%
\pgfpathlineto{\pgfqpoint{2.551295in}{2.395771in}}%
\pgfpathlineto{\pgfqpoint{2.567839in}{2.395264in}}%
\pgfpathlineto{\pgfqpoint{2.575192in}{2.395670in}}%
\pgfpathlineto{\pgfqpoint{2.588795in}{2.395508in}}%
\pgfpathlineto{\pgfqpoint{2.602030in}{2.395601in}}%
\pgfpathlineto{\pgfqpoint{2.648353in}{2.395790in}}%
\pgfpathlineto{\pgfqpoint{3.096507in}{2.397001in}}%
\pgfpathlineto{\pgfqpoint{3.100918in}{2.396778in}}%
\pgfpathlineto{\pgfqpoint{3.106065in}{2.396106in}}%
\pgfpathlineto{\pgfqpoint{3.109006in}{2.398590in}}%
\pgfpathlineto{\pgfqpoint{3.113050in}{2.401570in}}%
\pgfpathlineto{\pgfqpoint{3.115256in}{2.400049in}}%
\pgfpathlineto{\pgfqpoint{3.119668in}{2.395855in}}%
\pgfpathlineto{\pgfqpoint{3.121138in}{2.397520in}}%
\pgfpathlineto{\pgfqpoint{3.123712in}{2.405172in}}%
\pgfpathlineto{\pgfqpoint{3.127021in}{2.413867in}}%
\pgfpathlineto{\pgfqpoint{3.128124in}{2.413555in}}%
\pgfpathlineto{\pgfqpoint{3.129962in}{2.408653in}}%
\pgfpathlineto{\pgfqpoint{3.134006in}{2.395770in}}%
\pgfpathlineto{\pgfqpoint{3.135109in}{2.397485in}}%
\pgfpathlineto{\pgfqpoint{3.136947in}{2.407889in}}%
\pgfpathlineto{\pgfqpoint{3.142094in}{2.441922in}}%
\pgfpathlineto{\pgfqpoint{3.143197in}{2.438695in}}%
\pgfpathlineto{\pgfqpoint{3.145403in}{2.419631in}}%
\pgfpathlineto{\pgfqpoint{3.148712in}{2.395730in}}%
\pgfpathlineto{\pgfqpoint{3.149447in}{2.397487in}}%
\pgfpathlineto{\pgfqpoint{3.150917in}{2.411433in}}%
\pgfpathlineto{\pgfqpoint{3.156432in}{2.493466in}}%
\pgfpathlineto{\pgfqpoint{3.157167in}{2.491799in}}%
\pgfpathlineto{\pgfqpoint{3.158638in}{2.474467in}}%
\pgfpathlineto{\pgfqpoint{3.163417in}{2.395662in}}%
\pgfpathlineto{\pgfqpoint{3.163785in}{2.396544in}}%
\pgfpathlineto{\pgfqpoint{3.164888in}{2.409360in}}%
\pgfpathlineto{\pgfqpoint{3.167094in}{2.473152in}}%
\pgfpathlineto{\pgfqpoint{3.170770in}{2.572756in}}%
\pgfpathlineto{\pgfqpoint{3.171138in}{2.573964in}}%
\pgfpathlineto{\pgfqpoint{3.171505in}{2.572947in}}%
\pgfpathlineto{\pgfqpoint{3.172608in}{2.556794in}}%
\pgfpathlineto{\pgfqpoint{3.175182in}{2.466546in}}%
\pgfpathlineto{\pgfqpoint{3.177755in}{2.396850in}}%
\pgfpathlineto{\pgfqpoint{3.178123in}{2.395564in}}%
\pgfpathlineto{\pgfqpoint{3.178123in}{2.395564in}}%
\pgfpathlineto{\pgfqpoint{3.178123in}{2.395564in}}%
\pgfpathlineto{\pgfqpoint{3.178858in}{2.401500in}}%
\pgfpathlineto{\pgfqpoint{3.180329in}{2.446669in}}%
\pgfpathlineto{\pgfqpoint{3.185843in}{2.682868in}}%
\pgfpathlineto{\pgfqpoint{3.186211in}{2.680408in}}%
\pgfpathlineto{\pgfqpoint{3.187314in}{2.652244in}}%
\pgfpathlineto{\pgfqpoint{3.190255in}{2.483722in}}%
\pgfpathlineto{\pgfqpoint{3.192828in}{2.395439in}}%
\pgfpathlineto{\pgfqpoint{3.193564in}{2.404549in}}%
\pgfpathlineto{\pgfqpoint{3.195034in}{2.472722in}}%
\pgfpathlineto{\pgfqpoint{3.200549in}{2.813792in}}%
\pgfpathlineto{\pgfqpoint{3.200916in}{2.809233in}}%
\pgfpathlineto{\pgfqpoint{3.202387in}{2.742352in}}%
\pgfpathlineto{\pgfqpoint{3.207534in}{2.395298in}}%
\pgfpathlineto{\pgfqpoint{3.207902in}{2.398642in}}%
\pgfpathlineto{\pgfqpoint{3.209005in}{2.444568in}}%
\pgfpathlineto{\pgfqpoint{3.211578in}{2.703162in}}%
\pgfpathlineto{\pgfqpoint{3.214519in}{2.950009in}}%
\pgfpathlineto{\pgfqpoint{3.214887in}{2.956710in}}%
\pgfpathlineto{\pgfqpoint{3.215254in}{2.956491in}}%
\pgfpathlineto{\pgfqpoint{3.216357in}{2.914792in}}%
\pgfpathlineto{\pgfqpoint{3.218563in}{2.691494in}}%
\pgfpathlineto{\pgfqpoint{3.221872in}{2.398583in}}%
\pgfpathlineto{\pgfqpoint{3.222240in}{2.395152in}}%
\pgfpathlineto{\pgfqpoint{3.222240in}{2.395152in}}%
\pgfpathlineto{\pgfqpoint{3.222240in}{2.395152in}}%
\pgfpathlineto{\pgfqpoint{3.222975in}{2.411768in}}%
\pgfpathlineto{\pgfqpoint{3.224445in}{2.532849in}}%
\pgfpathlineto{\pgfqpoint{3.229592in}{3.101280in}}%
\pgfpathlineto{\pgfqpoint{3.229960in}{3.099907in}}%
\pgfpathlineto{\pgfqpoint{3.231063in}{3.044529in}}%
\pgfpathlineto{\pgfqpoint{3.233636in}{2.705507in}}%
\pgfpathlineto{\pgfqpoint{3.236578in}{2.399149in}}%
\pgfpathlineto{\pgfqpoint{3.236945in}{2.395011in}}%
\pgfpathlineto{\pgfqpoint{3.236945in}{2.395011in}}%
\pgfpathlineto{\pgfqpoint{3.236945in}{2.395011in}}%
\pgfpathlineto{\pgfqpoint{3.237681in}{2.415408in}}%
\pgfpathlineto{\pgfqpoint{3.239151in}{2.562458in}}%
\pgfpathlineto{\pgfqpoint{3.244298in}{3.237408in}}%
\pgfpathlineto{\pgfqpoint{3.244666in}{3.234718in}}%
\pgfpathlineto{\pgfqpoint{3.245769in}{3.165860in}}%
\pgfpathlineto{\pgfqpoint{3.248342in}{2.760228in}}%
\pgfpathlineto{\pgfqpoint{3.251283in}{2.399644in}}%
\pgfpathlineto{\pgfqpoint{3.251651in}{2.394884in}}%
\pgfpathlineto{\pgfqpoint{3.251651in}{2.394884in}}%
\pgfpathlineto{\pgfqpoint{3.251651in}{2.394884in}}%
\pgfpathlineto{\pgfqpoint{3.252386in}{2.418783in}}%
\pgfpathlineto{\pgfqpoint{3.253857in}{2.589460in}}%
\pgfpathlineto{\pgfqpoint{3.259004in}{3.358701in}}%
\pgfpathlineto{\pgfqpoint{3.259371in}{3.354662in}}%
\pgfpathlineto{\pgfqpoint{3.260474in}{3.273336in}}%
\pgfpathlineto{\pgfqpoint{3.263048in}{2.808169in}}%
\pgfpathlineto{\pgfqpoint{3.265989in}{2.400050in}}%
\pgfpathlineto{\pgfqpoint{3.266357in}{2.394777in}}%
\pgfpathlineto{\pgfqpoint{3.266357in}{2.394777in}}%
\pgfpathlineto{\pgfqpoint{3.266357in}{2.394777in}}%
\pgfpathlineto{\pgfqpoint{3.267092in}{2.421769in}}%
\pgfpathlineto{\pgfqpoint{3.268562in}{2.612944in}}%
\pgfpathlineto{\pgfqpoint{3.273709in}{3.462008in}}%
\pgfpathlineto{\pgfqpoint{3.274077in}{3.456688in}}%
\pgfpathlineto{\pgfqpoint{3.275180in}{3.364401in}}%
\pgfpathlineto{\pgfqpoint{3.277753in}{2.848383in}}%
\pgfpathlineto{\pgfqpoint{3.280694in}{2.400364in}}%
\pgfpathlineto{\pgfqpoint{3.281062in}{2.394690in}}%
\pgfpathlineto{\pgfqpoint{3.281062in}{2.394690in}}%
\pgfpathlineto{\pgfqpoint{3.281062in}{2.394690in}}%
\pgfpathlineto{\pgfqpoint{3.281797in}{2.424319in}}%
\pgfpathlineto{\pgfqpoint{3.283268in}{2.632610in}}%
\pgfpathlineto{\pgfqpoint{3.288415in}{3.546842in}}%
\pgfpathlineto{\pgfqpoint{3.289150in}{3.519836in}}%
\pgfpathlineto{\pgfqpoint{3.290621in}{3.311639in}}%
\pgfpathlineto{\pgfqpoint{3.295768in}{2.394622in}}%
\pgfpathlineto{\pgfqpoint{3.296503in}{2.426439in}}%
\pgfpathlineto{\pgfqpoint{3.297974in}{2.648600in}}%
\pgfpathlineto{\pgfqpoint{3.303121in}{3.614523in}}%
\pgfpathlineto{\pgfqpoint{3.303856in}{3.584739in}}%
\pgfpathlineto{\pgfqpoint{3.305326in}{3.362580in}}%
\pgfpathlineto{\pgfqpoint{3.310473in}{2.394572in}}%
\pgfpathlineto{\pgfqpoint{3.311209in}{2.428171in}}%
\pgfpathlineto{\pgfqpoint{3.312679in}{2.661326in}}%
\pgfpathlineto{\pgfqpoint{3.317826in}{3.667360in}}%
\pgfpathlineto{\pgfqpoint{3.318561in}{3.635301in}}%
\pgfpathlineto{\pgfqpoint{3.320032in}{3.402088in}}%
\pgfpathlineto{\pgfqpoint{3.325179in}{2.394537in}}%
\pgfpathlineto{\pgfqpoint{3.325914in}{2.429575in}}%
\pgfpathlineto{\pgfqpoint{3.327385in}{2.671319in}}%
\pgfpathlineto{\pgfqpoint{3.332532in}{3.708014in}}%
\pgfpathlineto{\pgfqpoint{3.333267in}{3.674124in}}%
\pgfpathlineto{\pgfqpoint{3.334738in}{3.432283in}}%
\pgfpathlineto{\pgfqpoint{3.339885in}{2.394514in}}%
\pgfpathlineto{\pgfqpoint{3.340620in}{2.430713in}}%
\pgfpathlineto{\pgfqpoint{3.342090in}{2.679126in}}%
\pgfpathlineto{\pgfqpoint{3.347237in}{3.739087in}}%
\pgfpathlineto{\pgfqpoint{3.347973in}{3.703736in}}%
\pgfpathlineto{\pgfqpoint{3.349443in}{3.455203in}}%
\pgfpathlineto{\pgfqpoint{3.354590in}{2.394500in}}%
\pgfpathlineto{\pgfqpoint{3.355326in}{2.431646in}}%
\pgfpathlineto{\pgfqpoint{3.356796in}{2.685253in}}%
\pgfpathlineto{\pgfqpoint{3.361943in}{3.762896in}}%
\pgfpathlineto{\pgfqpoint{3.362678in}{3.726376in}}%
\pgfpathlineto{\pgfqpoint{3.364149in}{3.472636in}}%
\pgfpathlineto{\pgfqpoint{3.369296in}{2.394494in}}%
\pgfpathlineto{\pgfqpoint{3.370031in}{2.432426in}}%
\pgfpathlineto{\pgfqpoint{3.371502in}{2.690132in}}%
\pgfpathlineto{\pgfqpoint{3.376649in}{3.781374in}}%
\pgfpathlineto{\pgfqpoint{3.377384in}{3.743908in}}%
\pgfpathlineto{\pgfqpoint{3.378855in}{3.486062in}}%
\pgfpathlineto{\pgfqpoint{3.384002in}{2.394493in}}%
\pgfpathlineto{\pgfqpoint{3.384737in}{2.433094in}}%
\pgfpathlineto{\pgfqpoint{3.386207in}{2.694113in}}%
\pgfpathlineto{\pgfqpoint{3.391354in}{3.796060in}}%
\pgfpathlineto{\pgfqpoint{3.392090in}{3.757814in}}%
\pgfpathlineto{\pgfqpoint{3.393560in}{3.496653in}}%
\pgfpathlineto{\pgfqpoint{3.398707in}{2.394496in}}%
\pgfpathlineto{\pgfqpoint{3.399442in}{2.433686in}}%
\pgfpathlineto{\pgfqpoint{3.400913in}{2.697468in}}%
\pgfpathlineto{\pgfqpoint{3.406060in}{3.808133in}}%
\pgfpathlineto{\pgfqpoint{3.406795in}{3.769223in}}%
\pgfpathlineto{\pgfqpoint{3.408266in}{3.505298in}}%
\pgfpathlineto{\pgfqpoint{3.413413in}{2.394502in}}%
\pgfpathlineto{\pgfqpoint{3.414148in}{2.434226in}}%
\pgfpathlineto{\pgfqpoint{3.415619in}{2.700403in}}%
\pgfpathlineto{\pgfqpoint{3.420766in}{3.818466in}}%
\pgfpathlineto{\pgfqpoint{3.421501in}{3.778972in}}%
\pgfpathlineto{\pgfqpoint{3.422971in}{3.512652in}}%
\pgfpathlineto{\pgfqpoint{3.428118in}{2.394510in}}%
\pgfpathlineto{\pgfqpoint{3.428854in}{2.434734in}}%
\pgfpathlineto{\pgfqpoint{3.430324in}{2.703066in}}%
\pgfpathlineto{\pgfqpoint{3.435471in}{3.827681in}}%
\pgfpathlineto{\pgfqpoint{3.436206in}{3.787657in}}%
\pgfpathlineto{\pgfqpoint{3.437677in}{3.519182in}}%
\pgfpathlineto{\pgfqpoint{3.442824in}{2.394520in}}%
\pgfpathlineto{\pgfqpoint{3.443559in}{2.435222in}}%
\pgfpathlineto{\pgfqpoint{3.445030in}{2.705563in}}%
\pgfpathlineto{\pgfqpoint{3.450177in}{3.836212in}}%
\pgfpathlineto{\pgfqpoint{3.450912in}{3.795692in}}%
\pgfpathlineto{\pgfqpoint{3.452383in}{3.525208in}}%
\pgfpathlineto{\pgfqpoint{3.457530in}{2.394532in}}%
\pgfpathlineto{\pgfqpoint{3.458265in}{2.435700in}}%
\pgfpathlineto{\pgfqpoint{3.459735in}{2.707963in}}%
\pgfpathlineto{\pgfqpoint{3.464882in}{3.844350in}}%
\pgfpathlineto{\pgfqpoint{3.465618in}{3.803352in}}%
\pgfpathlineto{\pgfqpoint{3.467088in}{3.530946in}}%
\pgfpathlineto{\pgfqpoint{3.472235in}{2.394544in}}%
\pgfpathlineto{\pgfqpoint{3.472971in}{2.436174in}}%
\pgfpathlineto{\pgfqpoint{3.474809in}{2.809333in}}%
\pgfpathlineto{\pgfqpoint{3.479220in}{3.846227in}}%
\pgfpathlineto{\pgfqpoint{3.479588in}{3.852280in}}%
\pgfpathlineto{\pgfqpoint{3.479588in}{3.852280in}}%
\pgfpathlineto{\pgfqpoint{3.479588in}{3.852280in}}%
\pgfpathlineto{\pgfqpoint{3.480323in}{3.810815in}}%
\pgfpathlineto{\pgfqpoint{3.481794in}{3.536530in}}%
\pgfpathlineto{\pgfqpoint{3.486941in}{2.394558in}}%
\pgfpathlineto{\pgfqpoint{3.487676in}{2.436647in}}%
\pgfpathlineto{\pgfqpoint{3.489514in}{2.812260in}}%
\pgfpathlineto{\pgfqpoint{3.493926in}{3.854152in}}%
\pgfpathlineto{\pgfqpoint{3.494294in}{3.860116in}}%
\pgfpathlineto{\pgfqpoint{3.494294in}{3.860116in}}%
\pgfpathlineto{\pgfqpoint{3.494294in}{3.860116in}}%
\pgfpathlineto{\pgfqpoint{3.495029in}{3.818189in}}%
\pgfpathlineto{\pgfqpoint{3.496867in}{3.442331in}}%
\pgfpathlineto{\pgfqpoint{3.501279in}{2.400297in}}%
\pgfpathlineto{\pgfqpoint{3.501646in}{2.394573in}}%
\pgfpathlineto{\pgfqpoint{3.502382in}{2.437123in}}%
\pgfpathlineto{\pgfqpoint{3.504220in}{2.815183in}}%
\pgfpathlineto{\pgfqpoint{3.508632in}{3.862052in}}%
\pgfpathlineto{\pgfqpoint{3.508999in}{3.867927in}}%
\pgfpathlineto{\pgfqpoint{3.509735in}{3.825537in}}%
\pgfpathlineto{\pgfqpoint{3.511573in}{3.447231in}}%
\pgfpathlineto{\pgfqpoint{3.515984in}{2.400223in}}%
\pgfpathlineto{\pgfqpoint{3.516352in}{2.394589in}}%
\pgfpathlineto{\pgfqpoint{3.517087in}{2.437602in}}%
\pgfpathlineto{\pgfqpoint{3.518926in}{2.818114in}}%
\pgfpathlineto{\pgfqpoint{3.523337in}{3.869962in}}%
\pgfpathlineto{\pgfqpoint{3.523705in}{3.875747in}}%
\pgfpathlineto{\pgfqpoint{3.524440in}{3.832894in}}%
\pgfpathlineto{\pgfqpoint{3.526278in}{3.452134in}}%
\pgfpathlineto{\pgfqpoint{3.530690in}{2.400150in}}%
\pgfpathlineto{\pgfqpoint{3.531058in}{2.394606in}}%
\pgfpathlineto{\pgfqpoint{3.531793in}{2.438085in}}%
\pgfpathlineto{\pgfqpoint{3.533631in}{2.821060in}}%
\pgfpathlineto{\pgfqpoint{3.538043in}{3.877901in}}%
\pgfpathlineto{\pgfqpoint{3.538411in}{3.883595in}}%
\pgfpathlineto{\pgfqpoint{3.539146in}{3.840276in}}%
\pgfpathlineto{\pgfqpoint{3.540984in}{3.457052in}}%
\pgfpathlineto{\pgfqpoint{3.545396in}{2.400075in}}%
\pgfpathlineto{\pgfqpoint{3.545763in}{2.394625in}}%
\pgfpathlineto{\pgfqpoint{3.546499in}{2.438573in}}%
\pgfpathlineto{\pgfqpoint{3.548337in}{2.824024in}}%
\pgfpathlineto{\pgfqpoint{3.552749in}{3.885878in}}%
\pgfpathlineto{\pgfqpoint{3.553116in}{3.891480in}}%
\pgfpathlineto{\pgfqpoint{3.553851in}{3.847692in}}%
\pgfpathlineto{\pgfqpoint{3.555690in}{3.461989in}}%
\pgfpathlineto{\pgfqpoint{3.560101in}{2.400001in}}%
\pgfpathlineto{\pgfqpoint{3.560469in}{2.394644in}}%
\pgfpathlineto{\pgfqpoint{3.561204in}{2.439066in}}%
\pgfpathlineto{\pgfqpoint{3.563042in}{2.827007in}}%
\pgfpathlineto{\pgfqpoint{3.567454in}{3.893896in}}%
\pgfpathlineto{\pgfqpoint{3.567822in}{3.899405in}}%
\pgfpathlineto{\pgfqpoint{3.568557in}{3.855144in}}%
\pgfpathlineto{\pgfqpoint{3.570395in}{3.466948in}}%
\pgfpathlineto{\pgfqpoint{3.574807in}{2.399927in}}%
\pgfpathlineto{\pgfqpoint{3.575175in}{2.394664in}}%
\pgfpathlineto{\pgfqpoint{3.575910in}{2.439563in}}%
\pgfpathlineto{\pgfqpoint{3.577748in}{2.830010in}}%
\pgfpathlineto{\pgfqpoint{3.582160in}{3.901955in}}%
\pgfpathlineto{\pgfqpoint{3.582527in}{3.907371in}}%
\pgfpathlineto{\pgfqpoint{3.583263in}{3.862632in}}%
\pgfpathlineto{\pgfqpoint{3.585101in}{3.471928in}}%
\pgfpathlineto{\pgfqpoint{3.589513in}{2.399853in}}%
\pgfpathlineto{\pgfqpoint{3.589880in}{2.394685in}}%
\pgfpathlineto{\pgfqpoint{3.590616in}{2.440065in}}%
\pgfpathlineto{\pgfqpoint{3.592454in}{2.833033in}}%
\pgfpathlineto{\pgfqpoint{3.596865in}{3.910058in}}%
\pgfpathlineto{\pgfqpoint{3.597233in}{3.915378in}}%
\pgfpathlineto{\pgfqpoint{3.597968in}{3.870159in}}%
\pgfpathlineto{\pgfqpoint{3.599807in}{3.476932in}}%
\pgfpathlineto{\pgfqpoint{3.604218in}{2.399778in}}%
\pgfpathlineto{\pgfqpoint{3.604586in}{2.394707in}}%
\pgfpathlineto{\pgfqpoint{3.605321in}{2.440572in}}%
\pgfpathlineto{\pgfqpoint{3.607159in}{2.836076in}}%
\pgfpathlineto{\pgfqpoint{3.611571in}{3.918203in}}%
\pgfpathlineto{\pgfqpoint{3.611939in}{3.923427in}}%
\pgfpathlineto{\pgfqpoint{3.612674in}{3.877723in}}%
\pgfpathlineto{\pgfqpoint{3.614512in}{3.481957in}}%
\pgfpathlineto{\pgfqpoint{3.618924in}{2.399704in}}%
\pgfpathlineto{\pgfqpoint{3.619292in}{2.394731in}}%
\pgfpathlineto{\pgfqpoint{3.620027in}{2.441084in}}%
\pgfpathlineto{\pgfqpoint{3.621865in}{2.839140in}}%
\pgfpathlineto{\pgfqpoint{3.626277in}{3.926390in}}%
\pgfpathlineto{\pgfqpoint{3.626644in}{3.931517in}}%
\pgfpathlineto{\pgfqpoint{3.627380in}{3.885325in}}%
\pgfpathlineto{\pgfqpoint{3.629218in}{3.487005in}}%
\pgfpathlineto{\pgfqpoint{3.633629in}{2.399629in}}%
\pgfpathlineto{\pgfqpoint{3.633997in}{2.394755in}}%
\pgfpathlineto{\pgfqpoint{3.634732in}{2.441601in}}%
\pgfpathlineto{\pgfqpoint{3.636571in}{2.842224in}}%
\pgfpathlineto{\pgfqpoint{3.640982in}{3.934622in}}%
\pgfpathlineto{\pgfqpoint{3.641350in}{3.939650in}}%
\pgfpathlineto{\pgfqpoint{3.642085in}{3.892966in}}%
\pgfpathlineto{\pgfqpoint{3.643923in}{3.492076in}}%
\pgfpathlineto{\pgfqpoint{3.648335in}{2.399554in}}%
\pgfpathlineto{\pgfqpoint{3.648703in}{2.394781in}}%
\pgfpathlineto{\pgfqpoint{3.649438in}{2.442123in}}%
\pgfpathlineto{\pgfqpoint{3.651276in}{2.845329in}}%
\pgfpathlineto{\pgfqpoint{3.655688in}{3.942896in}}%
\pgfpathlineto{\pgfqpoint{3.656056in}{3.947824in}}%
\pgfpathlineto{\pgfqpoint{3.656791in}{3.900644in}}%
\pgfpathlineto{\pgfqpoint{3.658629in}{3.497169in}}%
\pgfpathlineto{\pgfqpoint{3.663041in}{2.399480in}}%
\pgfpathlineto{\pgfqpoint{3.663408in}{2.394808in}}%
\pgfpathlineto{\pgfqpoint{3.664144in}{2.442650in}}%
\pgfpathlineto{\pgfqpoint{3.665982in}{2.848454in}}%
\pgfpathlineto{\pgfqpoint{3.670394in}{3.951214in}}%
\pgfpathlineto{\pgfqpoint{3.670761in}{3.956041in}}%
\pgfpathlineto{\pgfqpoint{3.671496in}{3.908361in}}%
\pgfpathlineto{\pgfqpoint{3.673335in}{3.502285in}}%
\pgfpathlineto{\pgfqpoint{3.677746in}{2.399405in}}%
\pgfpathlineto{\pgfqpoint{3.678114in}{2.394836in}}%
\pgfpathlineto{\pgfqpoint{3.678849in}{2.443182in}}%
\pgfpathlineto{\pgfqpoint{3.680687in}{2.851601in}}%
\pgfpathlineto{\pgfqpoint{3.685099in}{3.959575in}}%
\pgfpathlineto{\pgfqpoint{3.685467in}{3.964300in}}%
\pgfpathlineto{\pgfqpoint{3.686202in}{3.916116in}}%
\pgfpathlineto{\pgfqpoint{3.688040in}{3.507423in}}%
\pgfpathlineto{\pgfqpoint{3.692452in}{2.399330in}}%
\pgfpathlineto{\pgfqpoint{3.692820in}{2.394865in}}%
\pgfpathlineto{\pgfqpoint{3.693555in}{2.443719in}}%
\pgfpathlineto{\pgfqpoint{3.695393in}{2.854767in}}%
\pgfpathlineto{\pgfqpoint{3.699805in}{3.967975in}}%
\pgfpathlineto{\pgfqpoint{3.700172in}{3.972596in}}%
\pgfpathlineto{\pgfqpoint{3.700908in}{3.923904in}}%
\pgfpathlineto{\pgfqpoint{3.702746in}{3.512580in}}%
\pgfpathlineto{\pgfqpoint{3.707158in}{2.399255in}}%
\pgfpathlineto{\pgfqpoint{3.707525in}{2.394895in}}%
\pgfpathlineto{\pgfqpoint{3.708261in}{2.444261in}}%
\pgfpathlineto{\pgfqpoint{3.710099in}{2.857951in}}%
\pgfpathlineto{\pgfqpoint{3.714510in}{3.976406in}}%
\pgfpathlineto{\pgfqpoint{3.714878in}{3.980921in}}%
\pgfpathlineto{\pgfqpoint{3.715613in}{3.931717in}}%
\pgfpathlineto{\pgfqpoint{3.717452in}{3.517748in}}%
\pgfpathlineto{\pgfqpoint{3.721863in}{2.399180in}}%
\pgfpathlineto{\pgfqpoint{3.722231in}{2.394927in}}%
\pgfpathlineto{\pgfqpoint{3.722966in}{2.444807in}}%
\pgfpathlineto{\pgfqpoint{3.724804in}{2.861149in}}%
\pgfpathlineto{\pgfqpoint{3.729216in}{3.984847in}}%
\pgfpathlineto{\pgfqpoint{3.729584in}{3.989256in}}%
\pgfpathlineto{\pgfqpoint{3.730319in}{3.939535in}}%
\pgfpathlineto{\pgfqpoint{3.732157in}{3.522910in}}%
\pgfpathlineto{\pgfqpoint{3.736569in}{2.399105in}}%
\pgfpathlineto{\pgfqpoint{3.736937in}{2.394960in}}%
\pgfpathlineto{\pgfqpoint{3.737672in}{2.445357in}}%
\pgfpathlineto{\pgfqpoint{3.739510in}{2.864352in}}%
\pgfpathlineto{\pgfqpoint{3.743922in}{3.993259in}}%
\pgfpathlineto{\pgfqpoint{3.744289in}{3.997557in}}%
\pgfpathlineto{\pgfqpoint{3.745025in}{3.947315in}}%
\pgfpathlineto{\pgfqpoint{3.746863in}{3.528035in}}%
\pgfpathlineto{\pgfqpoint{3.751274in}{2.399029in}}%
\pgfpathlineto{\pgfqpoint{3.751642in}{2.394994in}}%
\pgfpathlineto{\pgfqpoint{3.752377in}{2.445909in}}%
\pgfpathlineto{\pgfqpoint{3.754216in}{2.867539in}}%
\pgfpathlineto{\pgfqpoint{3.758627in}{4.001564in}}%
\pgfpathlineto{\pgfqpoint{3.758995in}{4.005748in}}%
\pgfpathlineto{\pgfqpoint{3.759730in}{3.954981in}}%
\pgfpathlineto{\pgfqpoint{3.761568in}{3.533061in}}%
\pgfpathlineto{\pgfqpoint{3.765980in}{2.398953in}}%
\pgfpathlineto{\pgfqpoint{3.766348in}{2.395029in}}%
\pgfpathlineto{\pgfqpoint{3.767083in}{2.446459in}}%
\pgfpathlineto{\pgfqpoint{3.768921in}{2.870678in}}%
\pgfpathlineto{\pgfqpoint{3.773333in}{4.009630in}}%
\pgfpathlineto{\pgfqpoint{3.773701in}{4.013692in}}%
\pgfpathlineto{\pgfqpoint{3.774436in}{3.962396in}}%
\pgfpathlineto{\pgfqpoint{3.776274in}{3.537883in}}%
\pgfpathlineto{\pgfqpoint{3.780686in}{2.398877in}}%
\pgfpathlineto{\pgfqpoint{3.781053in}{2.395066in}}%
\pgfpathlineto{\pgfqpoint{3.781789in}{2.447002in}}%
\pgfpathlineto{\pgfqpoint{3.783627in}{2.873713in}}%
\pgfpathlineto{\pgfqpoint{3.788039in}{4.017234in}}%
\pgfpathlineto{\pgfqpoint{3.788406in}{4.021166in}}%
\pgfpathlineto{\pgfqpoint{3.789141in}{3.969340in}}%
\pgfpathlineto{\pgfqpoint{3.790980in}{3.542329in}}%
\pgfpathlineto{\pgfqpoint{3.795391in}{2.398799in}}%
\pgfpathlineto{\pgfqpoint{3.795759in}{2.395105in}}%
\pgfpathlineto{\pgfqpoint{3.796494in}{2.447529in}}%
\pgfpathlineto{\pgfqpoint{3.798332in}{2.876552in}}%
\pgfpathlineto{\pgfqpoint{3.802744in}{4.024027in}}%
\pgfpathlineto{\pgfqpoint{3.803112in}{4.027815in}}%
\pgfpathlineto{\pgfqpoint{3.803847in}{3.975462in}}%
\pgfpathlineto{\pgfqpoint{3.805685in}{3.546132in}}%
\pgfpathlineto{\pgfqpoint{3.810097in}{2.398719in}}%
\pgfpathlineto{\pgfqpoint{3.810465in}{2.395145in}}%
\pgfpathlineto{\pgfqpoint{3.811200in}{2.448025in}}%
\pgfpathlineto{\pgfqpoint{3.813038in}{2.879056in}}%
\pgfpathlineto{\pgfqpoint{3.817450in}{4.029481in}}%
\pgfpathlineto{\pgfqpoint{3.817817in}{4.033104in}}%
\pgfpathlineto{\pgfqpoint{3.818553in}{3.980234in}}%
\pgfpathlineto{\pgfqpoint{3.820391in}{3.548888in}}%
\pgfpathlineto{\pgfqpoint{3.824803in}{2.398637in}}%
\pgfpathlineto{\pgfqpoint{3.825170in}{2.395187in}}%
\pgfpathlineto{\pgfqpoint{3.825906in}{2.448467in}}%
\pgfpathlineto{\pgfqpoint{3.827744in}{2.881018in}}%
\pgfpathlineto{\pgfqpoint{3.832155in}{4.032817in}}%
\pgfpathlineto{\pgfqpoint{3.832523in}{4.036247in}}%
\pgfpathlineto{\pgfqpoint{3.833258in}{3.982881in}}%
\pgfpathlineto{\pgfqpoint{3.835097in}{3.550010in}}%
\pgfpathlineto{\pgfqpoint{3.839508in}{2.398550in}}%
\pgfpathlineto{\pgfqpoint{3.839876in}{2.395231in}}%
\pgfpathlineto{\pgfqpoint{3.840611in}{2.448825in}}%
\pgfpathlineto{\pgfqpoint{3.842449in}{2.882139in}}%
\pgfpathlineto{\pgfqpoint{3.846861in}{4.032933in}}%
\pgfpathlineto{\pgfqpoint{3.847229in}{4.036130in}}%
\pgfpathlineto{\pgfqpoint{3.847964in}{3.982310in}}%
\pgfpathlineto{\pgfqpoint{3.849802in}{3.548670in}}%
\pgfpathlineto{\pgfqpoint{3.854214in}{2.398457in}}%
\pgfpathlineto{\pgfqpoint{3.854582in}{2.395278in}}%
\pgfpathlineto{\pgfqpoint{3.855317in}{2.449054in}}%
\pgfpathlineto{\pgfqpoint{3.857155in}{2.882005in}}%
\pgfpathlineto{\pgfqpoint{3.861567in}{4.028313in}}%
\pgfpathlineto{\pgfqpoint{3.861934in}{4.031223in}}%
\pgfpathlineto{\pgfqpoint{3.862670in}{3.977021in}}%
\pgfpathlineto{\pgfqpoint{3.864508in}{3.543742in}}%
\pgfpathlineto{\pgfqpoint{3.868919in}{2.398356in}}%
\pgfpathlineto{\pgfqpoint{3.869287in}{2.395327in}}%
\pgfpathlineto{\pgfqpoint{3.870022in}{2.449094in}}%
\pgfpathlineto{\pgfqpoint{3.871861in}{2.880057in}}%
\pgfpathlineto{\pgfqpoint{3.876272in}{4.016943in}}%
\pgfpathlineto{\pgfqpoint{3.876640in}{4.019497in}}%
\pgfpathlineto{\pgfqpoint{3.877375in}{3.965026in}}%
\pgfpathlineto{\pgfqpoint{3.879213in}{3.533738in}}%
\pgfpathlineto{\pgfqpoint{3.883625in}{2.398245in}}%
\pgfpathlineto{\pgfqpoint{3.883993in}{2.395379in}}%
\pgfpathlineto{\pgfqpoint{3.884728in}{2.448863in}}%
\pgfpathlineto{\pgfqpoint{3.886566in}{2.875571in}}%
\pgfpathlineto{\pgfqpoint{3.890978in}{3.996244in}}%
\pgfpathlineto{\pgfqpoint{3.891346in}{3.998355in}}%
\pgfpathlineto{\pgfqpoint{3.892081in}{3.943790in}}%
\pgfpathlineto{\pgfqpoint{3.893919in}{3.516776in}}%
\pgfpathlineto{\pgfqpoint{3.898331in}{2.398119in}}%
\pgfpathlineto{\pgfqpoint{3.898698in}{2.395434in}}%
\pgfpathlineto{\pgfqpoint{3.899434in}{2.448262in}}%
\pgfpathlineto{\pgfqpoint{3.901272in}{2.867641in}}%
\pgfpathlineto{\pgfqpoint{3.905684in}{3.963060in}}%
\pgfpathlineto{\pgfqpoint{3.906051in}{3.964624in}}%
\pgfpathlineto{\pgfqpoint{3.906786in}{3.910224in}}%
\pgfpathlineto{\pgfqpoint{3.908625in}{3.490576in}}%
\pgfpathlineto{\pgfqpoint{3.913036in}{2.397977in}}%
\pgfpathlineto{\pgfqpoint{3.913404in}{2.395491in}}%
\pgfpathlineto{\pgfqpoint{3.914139in}{2.447168in}}%
\pgfpathlineto{\pgfqpoint{3.915977in}{2.855200in}}%
\pgfpathlineto{\pgfqpoint{3.920389in}{3.913751in}}%
\pgfpathlineto{\pgfqpoint{3.920757in}{3.914653in}}%
\pgfpathlineto{\pgfqpoint{3.921492in}{3.860787in}}%
\pgfpathlineto{\pgfqpoint{3.923330in}{3.452556in}}%
\pgfpathlineto{\pgfqpoint{3.927742in}{2.397815in}}%
\pgfpathlineto{\pgfqpoint{3.928110in}{2.395550in}}%
\pgfpathlineto{\pgfqpoint{3.928845in}{2.445446in}}%
\pgfpathlineto{\pgfqpoint{3.930683in}{2.837079in}}%
\pgfpathlineto{\pgfqpoint{3.935095in}{3.844462in}}%
\pgfpathlineto{\pgfqpoint{3.935462in}{3.844583in}}%
\pgfpathlineto{\pgfqpoint{3.936565in}{3.740171in}}%
\pgfpathlineto{\pgfqpoint{3.938771in}{3.180080in}}%
\pgfpathlineto{\pgfqpoint{3.942080in}{2.416975in}}%
\pgfpathlineto{\pgfqpoint{3.942815in}{2.395609in}}%
\pgfpathlineto{\pgfqpoint{3.943183in}{2.410872in}}%
\pgfpathlineto{\pgfqpoint{3.944653in}{2.629467in}}%
\pgfpathlineto{\pgfqpoint{3.949800in}{3.751643in}}%
\pgfpathlineto{\pgfqpoint{3.950168in}{3.750884in}}%
\pgfpathlineto{\pgfqpoint{3.951271in}{3.650762in}}%
\pgfpathlineto{\pgfqpoint{3.953844in}{3.018941in}}%
\pgfpathlineto{\pgfqpoint{3.956786in}{2.415217in}}%
\pgfpathlineto{\pgfqpoint{3.957521in}{2.395667in}}%
\pgfpathlineto{\pgfqpoint{3.957889in}{2.409874in}}%
\pgfpathlineto{\pgfqpoint{3.959359in}{2.611585in}}%
\pgfpathlineto{\pgfqpoint{3.964506in}{3.632889in}}%
\pgfpathlineto{\pgfqpoint{3.964874in}{3.631195in}}%
\pgfpathlineto{\pgfqpoint{3.965977in}{3.537099in}}%
\pgfpathlineto{\pgfqpoint{3.968550in}{2.959019in}}%
\pgfpathlineto{\pgfqpoint{3.971491in}{2.413099in}}%
\pgfpathlineto{\pgfqpoint{3.972227in}{2.395720in}}%
\pgfpathlineto{\pgfqpoint{3.972594in}{2.408562in}}%
\pgfpathlineto{\pgfqpoint{3.974065in}{2.589169in}}%
\pgfpathlineto{\pgfqpoint{3.979212in}{3.488072in}}%
\pgfpathlineto{\pgfqpoint{3.979579in}{3.485469in}}%
\pgfpathlineto{\pgfqpoint{3.980682in}{3.399369in}}%
\pgfpathlineto{\pgfqpoint{3.983256in}{2.887273in}}%
\pgfpathlineto{\pgfqpoint{3.986197in}{2.410653in}}%
\pgfpathlineto{\pgfqpoint{3.986932in}{2.395765in}}%
\pgfpathlineto{\pgfqpoint{3.987300in}{2.406944in}}%
\pgfpathlineto{\pgfqpoint{3.988770in}{2.562527in}}%
\pgfpathlineto{\pgfqpoint{3.993917in}{3.320596in}}%
\pgfpathlineto{\pgfqpoint{3.994285in}{3.317225in}}%
\pgfpathlineto{\pgfqpoint{3.995388in}{3.241158in}}%
\pgfpathlineto{\pgfqpoint{3.997961in}{2.805890in}}%
\pgfpathlineto{\pgfqpoint{4.000902in}{2.407969in}}%
\pgfpathlineto{\pgfqpoint{4.001638in}{2.395800in}}%
\pgfpathlineto{\pgfqpoint{4.002005in}{2.405074in}}%
\pgfpathlineto{\pgfqpoint{4.003476in}{2.532687in}}%
\pgfpathlineto{\pgfqpoint{4.008623in}{3.138340in}}%
\pgfpathlineto{\pgfqpoint{4.008991in}{3.134471in}}%
\pgfpathlineto{\pgfqpoint{4.010093in}{3.070266in}}%
\pgfpathlineto{\pgfqpoint{4.013035in}{2.663520in}}%
\pgfpathlineto{\pgfqpoint{4.015976in}{2.396482in}}%
\pgfpathlineto{\pgfqpoint{4.016343in}{2.395822in}}%
\pgfpathlineto{\pgfqpoint{4.017079in}{2.417862in}}%
\pgfpathlineto{\pgfqpoint{4.018917in}{2.581241in}}%
\pgfpathlineto{\pgfqpoint{4.022961in}{2.950690in}}%
\pgfpathlineto{\pgfqpoint{4.023329in}{2.953630in}}%
\pgfpathlineto{\pgfqpoint{4.023329in}{2.953630in}}%
\pgfpathlineto{\pgfqpoint{4.023329in}{2.953630in}}%
\pgfpathlineto{\pgfqpoint{4.024064in}{2.938904in}}%
\pgfpathlineto{\pgfqpoint{4.025534in}{2.836875in}}%
\pgfpathlineto{\pgfqpoint{4.030681in}{2.396272in}}%
\pgfpathlineto{\pgfqpoint{4.031049in}{2.395830in}}%
\pgfpathlineto{\pgfqpoint{4.032152in}{2.427347in}}%
\pgfpathlineto{\pgfqpoint{4.034725in}{2.621462in}}%
\pgfpathlineto{\pgfqpoint{4.037667in}{2.780467in}}%
\pgfpathlineto{\pgfqpoint{4.038034in}{2.781593in}}%
\pgfpathlineto{\pgfqpoint{4.038769in}{2.769626in}}%
\pgfpathlineto{\pgfqpoint{4.040608in}{2.670621in}}%
\pgfpathlineto{\pgfqpoint{4.045019in}{2.400226in}}%
\pgfpathlineto{\pgfqpoint{4.045755in}{2.395828in}}%
\pgfpathlineto{\pgfqpoint{4.046122in}{2.399285in}}%
\pgfpathlineto{\pgfqpoint{4.047593in}{2.444890in}}%
\pgfpathlineto{\pgfqpoint{4.052740in}{2.636715in}}%
\pgfpathlineto{\pgfqpoint{4.053843in}{2.619356in}}%
\pgfpathlineto{\pgfqpoint{4.056416in}{2.510403in}}%
\pgfpathlineto{\pgfqpoint{4.059725in}{2.398399in}}%
\pgfpathlineto{\pgfqpoint{4.060460in}{2.395820in}}%
\pgfpathlineto{\pgfqpoint{4.060828in}{2.397850in}}%
\pgfpathlineto{\pgfqpoint{4.062298in}{2.424177in}}%
\pgfpathlineto{\pgfqpoint{4.067078in}{2.529007in}}%
\pgfpathlineto{\pgfqpoint{4.067813in}{2.526455in}}%
\pgfpathlineto{\pgfqpoint{4.069284in}{2.503774in}}%
\pgfpathlineto{\pgfqpoint{4.074798in}{2.395887in}}%
\pgfpathlineto{\pgfqpoint{4.075534in}{2.396845in}}%
\pgfpathlineto{\pgfqpoint{4.077004in}{2.410017in}}%
\pgfpathlineto{\pgfqpoint{4.081783in}{2.459096in}}%
\pgfpathlineto{\pgfqpoint{4.082519in}{2.457366in}}%
\pgfpathlineto{\pgfqpoint{4.084357in}{2.441713in}}%
\pgfpathlineto{\pgfqpoint{4.089136in}{2.396382in}}%
\pgfpathlineto{\pgfqpoint{4.089872in}{2.395804in}}%
\pgfpathlineto{\pgfqpoint{4.090239in}{2.396247in}}%
\pgfpathlineto{\pgfqpoint{4.091710in}{2.401787in}}%
\pgfpathlineto{\pgfqpoint{4.096121in}{2.420835in}}%
\pgfpathlineto{\pgfqpoint{4.097224in}{2.419910in}}%
\pgfpathlineto{\pgfqpoint{4.099063in}{2.413331in}}%
\pgfpathlineto{\pgfqpoint{4.103474in}{2.396397in}}%
\pgfpathlineto{\pgfqpoint{4.104945in}{2.395955in}}%
\pgfpathlineto{\pgfqpoint{4.106783in}{2.398520in}}%
\pgfpathlineto{\pgfqpoint{4.110459in}{2.403732in}}%
\pgfpathlineto{\pgfqpoint{4.112298in}{2.403037in}}%
\pgfpathlineto{\pgfqpoint{4.120018in}{2.395923in}}%
\pgfpathlineto{\pgfqpoint{4.127738in}{2.397302in}}%
\pgfpathlineto{\pgfqpoint{4.135091in}{2.395847in}}%
\pgfpathlineto{\pgfqpoint{4.149062in}{2.395801in}}%
\pgfpathlineto{\pgfqpoint{4.327735in}{2.395800in}}%
\pgfpathlineto{\pgfqpoint{6.830632in}{2.395800in}}%
\pgfpathlineto{\pgfqpoint{6.830632in}{2.395800in}}%
\pgfusepath{stroke}%
\end{pgfscope}%
\begin{pgfscope}%
\pgfsetrectcap%
\pgfsetmiterjoin%
\pgfsetlinewidth{0.803000pt}%
\definecolor{currentstroke}{rgb}{0.000000,0.000000,0.000000}%
\pgfsetstrokecolor{currentstroke}%
\pgfsetdash{}{0pt}%
\pgfpathmoveto{\pgfqpoint{0.948750in}{0.532400in}}%
\pgfpathlineto{\pgfqpoint{0.948750in}{4.259200in}}%
\pgfusepath{stroke}%
\end{pgfscope}%
\begin{pgfscope}%
\pgfsetrectcap%
\pgfsetmiterjoin%
\pgfsetlinewidth{0.803000pt}%
\definecolor{currentstroke}{rgb}{0.000000,0.000000,0.000000}%
\pgfsetstrokecolor{currentstroke}%
\pgfsetdash{}{0pt}%
\pgfpathmoveto{\pgfqpoint{6.831000in}{0.532400in}}%
\pgfpathlineto{\pgfqpoint{6.831000in}{4.259200in}}%
\pgfusepath{stroke}%
\end{pgfscope}%
\begin{pgfscope}%
\pgfsetrectcap%
\pgfsetmiterjoin%
\pgfsetlinewidth{0.803000pt}%
\definecolor{currentstroke}{rgb}{0.000000,0.000000,0.000000}%
\pgfsetstrokecolor{currentstroke}%
\pgfsetdash{}{0pt}%
\pgfpathmoveto{\pgfqpoint{0.948750in}{0.532400in}}%
\pgfpathlineto{\pgfqpoint{6.831000in}{0.532400in}}%
\pgfusepath{stroke}%
\end{pgfscope}%
\begin{pgfscope}%
\pgfsetrectcap%
\pgfsetmiterjoin%
\pgfsetlinewidth{0.803000pt}%
\definecolor{currentstroke}{rgb}{0.000000,0.000000,0.000000}%
\pgfsetstrokecolor{currentstroke}%
\pgfsetdash{}{0pt}%
\pgfpathmoveto{\pgfqpoint{0.948750in}{4.259200in}}%
\pgfpathlineto{\pgfqpoint{6.831000in}{4.259200in}}%
\pgfusepath{stroke}%
\end{pgfscope}%
\begin{pgfscope}%
\pgftext[x=0.948750in,y=4.445540in,left,base]{\sffamily\fontsize{10.000000}{12.000000}\selectfont Iterations: 12935, Time: 0.228 ps, RXPWR: 4.6 percent}%
\end{pgfscope}%
\end{pgfpicture}%
\makeatother%
\endgroup%
}}
        \subcaption{Simulation using a material with negative conductivity.}
        \label{fig:task6_2}
    \end{subfigure}
  \caption{Simulation results from task 6.}
  \label{fig:task6}
\end{figure}

\newpage
\appendix
\section{Code}
\begin{changemargin}{-3cm}{0.5cm}
\lstinputlisting[language=Python]{calcs.py}
\end{changemargin}
\end{document}