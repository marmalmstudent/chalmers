\message{ !name(Handin_1.tex)}\documentclass[12pt,a4paper]{article}
\pdfoutput=1

\usepackage[utf8]{inputenc}
\usepackage[T1]{fontenc}
\usepackage[swedish]{babel}
\usepackage{amsmath}
\usepackage{lmodern}
\usepackage{units}
\usepackage{siunitx}
\usepackage{icomma}
\usepackage{color}
\usepackage{graphicx}
\usepackage{bbm}
\newcommand{\N}{\ensuremath{\mathbbm{N}}}
\newcommand{\Z}{\ensuremath{\mathbbm{Z}}}
\newcommand{\Q}{\ensuremath{\mathbbm{Q}}}
\newcommand{\R}{\ensuremath{\mathbbm{R}}}
\newcommand{\C}{\ensuremath{\mathbbm{C}}}
\newcommand{\rd}{\ensuremath{\mathrm{d}}}
\newcommand{\id}{\ensuremath{\,\rd}}
\usepackage{hyperref}

\begin{document}

\message{ !name(Handin_1.tex) !offset(-3) }


\title{SSY036 Handin 1}
\author{Marcus Malmquist}
\date{\today}
\maketitle

\begin{abstract}

\end{abstract}

\newpage
\tableofcontents
\newpage

\section{Task 1}
To achieve a 50 dB attenuation using an enclosure of aluminium (with a conductivity of $\sigma = 3.5\cdot 10^{7}\SI{}{\siemens\metre^{-1}}$) foil of thickness $d=\SI{1}{\milli\meter}$ and $d=\SI{20}{\micro\metre}$, an attenuation of $\alpha = \frac{5\log{10}}{d}$ (from $10^{-50/10}=e^{\alpha d}$) is needed. Since aluminium is a good conductor, equation 2.8.3 in Orfanidis
\subsection{}

% Figurer inkluderade som eps-filer
%% \begin{figure}\centering
%% \includegraphics{filnamn.eps}
%% \caption{\label{figuren} Perioden $T$ som funktion av pendellängden.}
%% \end{figure}

% Figurer inkluderade med xfigs "Combined PDF/LaTeX"
%% \begin{figure}\centering
%% \input{filnamn.pdf_t}
%% \caption{\label{finafiguren} Perioden $T$ som funktion av
%%   pendellängden.}
%% \end{figure}

\end{document}

\message{ !name(Handin_1.tex) !offset(-61) }
