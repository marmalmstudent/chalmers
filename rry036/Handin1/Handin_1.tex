\documentclass[12pt,a4paper]{article}
\pdfoutput=1

\usepackage[utf8]{inputenc}
\usepackage[T1]{fontenc}
\usepackage[english]{babel}
\usepackage{amsmath}
\usepackage{lmodern}
\usepackage{units}
\usepackage{siunitx}
\usepackage{icomma}
\usepackage{color}
\usepackage{graphicx}
\usepackage{bbm}
\newcommand{\N}{\ensuremath{\mathbbm{N}}}
\newcommand{\Z}{\ensuremath{\mathbbm{Z}}}
\newcommand{\Q}{\ensuremath{\mathbbm{Q}}}
\newcommand{\R}{\ensuremath{\mathbbm{R}}}
\newcommand{\C}{\ensuremath{\mathbbm{C}}}
\newcommand{\rd}{\ensuremath{\mathrm{d}}}
\newcommand{\id}{\ensuremath{\,\rd}}
\usepackage{hyperref}

\begin{document}

\title{SSY036 Handin 1}
\author{Marcus Malmquist}
\date{\today}
\maketitle

\begin{abstract}
This text aim to describe how I solved the tasks of Hand-in 1.
\end{abstract}

\newpage
\tableofcontents
\newpage

\section{Task 1}
To achieve a 50 dB attenuation using an enclosure of aluminium (with a conductivity of $\sigma = \SI{35}{\micro\siemens\metre^{-1}}$) foil of thickness $d=\SI{1}{\milli\meter}$ and $d=\SI{20}{\micro\metre}$, an attenuation of $\alpha = \frac{5\log{10}}{d}$ (from $10^{-50/10}=e^{\alpha d}$) is needed. Since aluminium is a good conductor, equation 2.8.3 in \textbf{Orfanidis} can be used to relate $\alpha$ to the frequency of the wave 
\begin{equation}
  \alpha=\sqrt{\pi f \mu \sigma} \Leftrightarrow f=\alpha^2(\pi\mu\sigma)^{-1}=\begin{cases}\SI{960}{\kilo\hertz}, & d=\SI{1}{\milli\meter} \\ \SI{2.4}{\giga\hertz}, & d=\SI{20}{\micro\meter}\end{cases}
\end{equation}
This solution assumes that $\mu=\mu_0$.

\section{Task 2}
Some given constants for this task: \\
\begin{tabular}{|lcl|} \hline
  $f_1$ & = & $\SI{40}{\kilo\hertz}$ \\
  $\sigma$ & = & $\SI{4}{\siemens\metre^{-1}}$ \\
  $\mu$ & = & $\mu_0$ \\
  $\epsilon$ & = & $80\epsilon_0$ \\ \hline
\end{tabular}

\subsection{a}
Since $\frac{\sigma}{\omega\epsilon} >> 1$ the weakly lossy dielectrics approximation can not be used and equation 2.8.27 and 2.8.28 in \textbf{Orfanidis} will be used. Since $\epsilon \in \R \Rightarrow \epsilon = \epsilon_{d}^\prime$, we get
\begin{equation}
  \begin{array}{lcl}
    k_c & = & \omega\sqrt{\mu_0\epsilon_{d}^\prime}\big(1-j\frac{\sigma}{\omega\epsilon_d^\prime}\big)^{\frac{1}{2}} \\
        & = & \big(\frac{\omega^4\epsilon_r^2}{c_0^4}+\omega^2\mu_0^2\sigma^2\big)^{\frac{1}{4}}\exp{\big(j\frac{1}{2}\arctan{\big[-\frac{\sigma}{\omega\epsilon_d^\prime}\big]}\big)} \\
        & = & \alpha-j\beta
  \end{array}
  \label{eq:k_c}
\end{equation}
Using equation~\ref{eq:k_c} we get
\begin{equation}
  \begin{array}{lll}
    \alpha & = & -\Im(k_c) \\
           & = & -\big(\frac{\omega^4\epsilon_r^2}{c_0^4}+\omega^2\mu_0^2\sigma^2\big)^{\frac{1}{4}}\sin{\big(j\frac{1}{2}\arctan{\big[-\frac{\sigma}{\omega\epsilon_d^\prime}\big]}\big)} \\
           & = & \SI{0.79}{\neper} \\
  \end{array}
\end{equation}
$\lambda$ can be acquired from
\begin{equation}
  \begin{array}{lll}
    \lambda & = & \frac{2\pi}{\Re(k_c)} \\
           & = & 2\pi\Big\{\big(\frac{\omega^4\epsilon_r^2}{c_0^4}+\omega^2\mu_0^2\sigma^2\big)^{\frac{1}{4}}\cos{\big(j\frac{1}{2}\arctan{\big[-\frac{\sigma}{\omega\epsilon_d^\prime}\big]}\big)}\Big\} \\
           & = & \SI{7.9}{\metre} \\
  \end{array}
\end{equation}
The conclusion is that since the skin depth, $\delta$, for this frequency is only $\SI{1.26}{\metre}$ water is not a good conductor for this frequency.

\subsection{b}
The propagation distance where the signal strength have decreased by $\SI{10}{\deci\bel}$ can be found by solvig $10^{-1}=e^{-\alpha z}$  for $z$, which will yield the result $z_{\SI{10}{\deci\bel}}=\SI{2.9}{\metre}$

\subsection{c}
finding $\delta(f)=\frac{1}{\alpha(f)}$ in the limit yields
\begin{equation}
  \begin{array}{lcl}
    \lim_{f\to 0}\delta & = & \lim_{f\to 0}\frac{1}{\alpha} \\
                            & = & \infty \\
    \lim_{f\to\infty}\delta & = & \lim_{f\to\infty}\frac{1}{\alpha} \\
                            & = & \frac{2c_0\epsilon_0\sqrt{\epsilon_r}}{\sigma} \\
     & = & \SI{0.012}{\metre}
  \end{array}
\end{equation}
The plot of $\delta(f)$ for $0<f<\SI{100}{\giga\hertz}$ can be seen in Figure~\ref{fig:skindepth}.
\begin{figure}\centering
  \scalebox{0.7}{\input{handin1_2c_edit_2.pdf_t}}
  \caption{\label{fig:skindepth} Skindepth $\delta$ as a function of frequency for a field propagating in saltwater.}
\end{figure}

\section{Task 3}
The solar radiation incident on upper atmosphere of Earth is given as $|\vec{\mathcal{P}}|=\SI{1350}{\watt\metre^{-2}}$.
\subsection{a}
The length of the poynting vector $|\vec{\mathcal{P}}|$ can be written as
\begin{equation}
  |\vec{\mathcal{P}}| = |\vec{E}\times\vec{H}^*|=\frac{1}{\eta^*}|\vec{E}_+|^2
\end{equation}
Assuming $\eta=\eta_0=120\pi$ (since it is practically in space) and that $\vec{E}=\vec{E}_+$ we get
\begin{equation}
  \begin{array}{lcl}
    |\vec{E}| & = & \sqrt{|\vec{\mathcal{P}}|\eta^*} \\
              & = & \SI{3.58}{\volt\metre^{-1}} \\
    |\vec{H}| & = & \sqrt{\frac{|\vec{\mathcal{P}}|}{\eta^*}} \\
              & = & \SI{9.50}{\milli\tesla} \\
  \end{array}
\end{equation}

\subsection{b}
A model of the problem can be seen in Figure~\ref{fig:box}. The box has an area $A=\SI{1}{\metre^2}$ and a length $l=\SI{1}{\metre}$.
The energy contained in this box is $E_{box}=|\vec{\mathcal{P}}|Al=|\vec{\mathcal{P}}|A\frac{l}{c}=\SI{4.5}{\joule}$
\begin{figure}\centering
  \scalebox{1}{\input{handin1_box.pdf_t}}
  \caption{\label{fig:box} solar radiation in a imaginary box with volum $V=Al$.}
\end{figure}

\subsection{c}
A photon has the energy $E_p=\frac{hc}{\lambda}$. A photon with a wavelenght $\lambda$ also has momentum $p=\frac{h}{\lambda}$. Thus a photon with energy $E_p$ has a momentum $p=\frac{E_p}{c}$.

Given that the photons are assumed to hit the solar sail parallel to its normal, and that no photons are absorbed by the sail or transmitted through it, the change in momentum this elastic collision is $\Delta p=2\frac{E_p}{c}$.

The number of photons per unit area is then $N_p=\frac{|\vec{\mathcal{P}}|}{E_p}$ making the preasure per unit area $P=N_p\delta p=2\frac{|\vec{\mathcal{P}}|}{c}$.

It is assumed that the only forces affecting the ship is the gravitational pull of the sun and the push of the solar rays (Figure~\ref{fig:ship}). Equilibrium gives te following expression
\begin{equation}
  G\frac{m_{sun}m_{ship}}{r^2}=2\frac{|\vec{\mathcal{P}}|}{c}A_{ship}
  \label{eq:equilibrium}
\end{equation}
\begin{figure}\centering
  \scalebox{1}{\input{handin1_ship.pdf_t}}
  \caption{\label{fig:ship} The forces excerted on the ship is assumed to come from the gravitational force of the sun and solar rays.}
\end{figure}
Using the values from table~\ref{tab:constants} and solving equation~\ref{eq:equilibrium} for $A_{ship}$ yields (when the ship is at the same distance from the sun as the Earth is) $A_{ship}=3.3\cdot 10^{6}\text{ }\SI{}{\metre^2}$.
\begin{table}
  \centering
  \caption{\label{tab:constants} Some values needed to calculate the gravitational pull on the ship excerted by the sun when the distance between the ship and the sun is the same as the distance between the Earth and the sun.}
  \begin{tabular}{|lll|} \hline
    $G$ & = & $6.7\cdot 10^{-11}$ \\
    $m_{sun}$ & = & $2.0\cdot 10^{30}$ kg \\
    $m_{ship}$ & = & $5.0\cdot 10^{3}$ kg \\
    $r$ & = & $150\cdot 10^{9}$ m \\ \hline
  \end{tabular}
\end{table}

The intensity of the solar rays at a given distance from the sun, $|\vec{\mathcal{P}}|$, can be expressed as a power $S$ evenly distributed over the surface of a sphere
\begin{equation}
  |\vec{\mathcal{P}}|=\frac{S}{4\pi r^2}
  \label{eq:power}
\end{equation}
Combining equation~\ref{eq:power} with \ref{eq:equilibrium} and again solving for $A_{ship}$ yields the expression
\begin{equation}
  A_{ship}=2\pi G\frac{m_{sun}m_{ship}c}{S}
\end{equation}
which does not depend on the distance to the sun.

\section{Task 4}
A complex field $\vec{E}_c=(\sqrt{2}\hat{x}+\hat{y}-\hat{z})\exp(-j2\pi\cdot 10^{6}(y+z))$ is propagating through a medium with $\epsilon=2\epsilon_0$ and $\mu=\mu_0$.
\subsection{a}
$\vec{k}=2\pi\cdot 10^{6}(y\hat{y}+z\hat{z})$
\subsection{b}
The dispersion relation is $\vec{k}\cdot\vec{k}=\omega^2\mu\epsilon$. This means that $f=\frac{1}{2\pi}2\sqrt{2}\pi\cdot 10^{6}\frac{c_0}{\sqrt{2}}=\SI{300}{\tera\hertz}$
\subsection{c}
By introducing a new orthonormal basis ($\hat{\xi}=\hat{x}, \hat{\eta}=\frac{1}{\sqrt{2}}(\hat{y}-\hat{z}), \hat{\zeta}=\frac{1}{\sqrt{2}}(\hat{y}+\hat{z})$) it can be seen that the polarisation is linear as there is no phase shift between the field in $\hat{\xi}$ and $\hat{\eta}$.

\subsection{d}
Using $\vec{k}$ as described earlier, we have
\begin{equation}
  \begin{array}{lcl}
    \vec{E}(\vec{r}, t) & = & \Re(\vec{E}_ce^{j\omega t}) \\
                        & = &(\sqrt{2}\hat{x}+\hat{y}-\hat{z})\cos(wt-\vec{k}\cdot\vec{r}) \\
    \vec{H}(\vec{r}, t) & = & \Re(\vec{H}_ce^{j\omega t}) \\
                        & = & \Re(\sqrt{\frac{\epsilon}{\mu}}\vec{k}\times\vec{E}_ce^{j\omega t}) \\
                        & = & \frac{\sqrt{2}}{120\pi}(-\sqrt{2}\hat{x}+\hat{y}-\hat{z})\cos(wt-\vec{k}\cdot\vec{r})
  \end{array}
\end{equation}
\end{document}
