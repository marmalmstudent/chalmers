\documentclass[12pt,a4paper]{article}
\pdfoutput=1

\usepackage[utf8]{inputenc}
\usepackage[T1]{fontenc}
\usepackage[english]{babel}
\usepackage{amsmath}
\usepackage{lmodern}
\usepackage{units}
\usepackage{siunitx}
\usepackage{icomma}
\usepackage{color}
\usepackage{graphicx}
\usepackage{bbm}
\newcommand{\N}{\ensuremath{\mathbbm{N}}}
\newcommand{\Z}{\ensuremath{\mathbbm{Z}}}
\newcommand{\Q}{\ensuremath{\mathbbm{Q}}}
\newcommand{\R}{\ensuremath{\mathbbm{R}}}
\newcommand{\C}{\ensuremath{\mathbbm{C}}}
\newcommand{\rd}{\ensuremath{\mathrm{d}}}
\newcommand{\id}{\ensuremath{\,\rd}}
\usepackage{hyperref}

\begin{document}

\title{Title}
\author{Marcus Malmquist}
\date{\today}
\maketitle

\begin{abstract}
This is an abstract
\end{abstract}

\newpage
\tableofcontents
\newpage

\section{Section}

%\begin{figure}\centering
%  \scalebox{1}{\input{handin1_ship.pdf_t}}
%  \caption{\label{fig:ship} The forces excerted on the ship is assumed to come from the gravitational force of the sun and solar rays.}
%\end{figure}

%\begin{table}
%  \centering
%  \caption{\label{tab:table}A table.}
%  \begin{tabular}{|l|} \hline
%    _ & _ & _ \\ \hline
%  \end{tabular}
%\end{table}
\newpage
\begin{thebibliography}{1}

\bibitem{ref1} \emph{DuckDuckGo}, Chalmers,\\ Available:\hspace{5pt}\url{https://duckduckgo.com} [2016-01-18]

\end{thebibliography}

\end{document}
