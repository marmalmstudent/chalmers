\documentclass{article}

\pdfoutput=1

\usepackage[utf8]{inputenc}
\usepackage[T1]{fontenc}
\usepackage[english]{babel}
\usepackage{amsmath}
\usepackage{mathtools}
\usepackage{lmodern}
\usepackage{units}
\usepackage{siunitx}
\usepackage{icomma}
\usepackage{graphicx}
\usepackage{caption}
\usepackage{subcaption}
\usepackage{color}
\newcommand{\N}{\ensuremath{\mathbbm{N}}}
\newcommand{\Z}{\ensuremath{\mathbbm{Z}}}
\newcommand{\Q}{\ensuremath{\mathbbm{Q}}}
\newcommand{\R}{\ensuremath{\mathbbm{R}}}
\newcommand{\C}{\ensuremath{\mathbbm{C}}}
\newcommand{\rd}{\ensuremath{\mathrm{d}}}
\newcommand{\id}{\ensuremath{\,\rd}}
\usepackage{hyperref}
%\usepackage{a4wide} % puts the page numbering further down the page.
\usepackage{pdfpages}
\usepackage{epstopdf}
\DeclareMathOperator{\sgn}{sgn}
\DeclareGraphicsExtensions{.eps}

\title{Home Assigment 4}
\author{Marcus Malmquist, marmalm, 941022}
\date{\today}

\begin{document}
\maketitle

\section{Task 1}\label{sec:1}
The scattering matrix $\mathbf{S}$ of a four port is shown in (\ref{eq:scatt}).
\begin{equation}
  \mathbf{S} =
  \begin{pmatrix}
    0.2e^{j\frac{\pi}{2}} & 0.5e^{-j\frac{\pi}{4}} & 0.5e^{j\frac{\pi}{4}} & 0 \\
    0.5e^{-j\frac{\pi}{4}} & 0 & 0 & 0.5e^{j\frac{\pi}{4}} \\
    0.5e^{-j\frac{\pi}{4}} & 0 & 0 & 0.5e^{-j\frac{\pi}{4}} \\
    0 & 0.5e^{-j\frac{\pi}{4}} & 0.5e^{-j\frac{\pi}{4}} & 0
  \end{pmatrix}
  \label{eq:scatt}
\end{equation}

\subsection{a}\label{sec:1a}
The circuit is lossless if $\mathbf{S}$ is unitary. In order for $\mathbf{S}$ to be unitary it has to be equal to its Hermitian conjugate (denoted $\dagger$) (\ref{eq:hermit}), which is not the case for $\mathbf{S}$ as it is defined in (\ref{eq:scatt}).
\begin{equation}
  \mathbf{S}\mathbf{S}^\dagger=\mathbf{I}
  \label{eq:hermit}
\end{equation}

\subsection{b}\label{sec:1b}
The circuit islossless if $\mathbf{S}$ is symmetric (\ref{eq:symm}), which is not the case for $\mathbf{S}$ as it is defined in (\ref{eq:scatt}).
\begin{equation}
  \mathbf{S}=\mathbf{S}^T
  \label{eq:symm}
\end{equation}

\subsection{c}\label{sec:1c}
The incident and reflected voltage waves are related through the scattering matrix as shown in (\ref{eq:volt})
\begin{equation}
  \vec{V}^{-}=\mathbf{S}\vec{V}^{+}
  \label{eq:volt}
\end{equation}

If the system is driven by a voltage wave incident at port 1 while all other ports are matched then
\begin{equation}
  V^{-}_{1}=\mathbf{S}_{11}V^{+}_{1} + \mathbf{S}_{1j}\overbrace{(1-\delta_{i1})\mathbf{\Gamma}\underbrace{\mathbf{S}\vec{V}^{+}}_{\vec{V}^{-}}}^{\vec{V}^{'+}}=\mathbf{S}_{11}V^{+}_{1}
  \label{eq:return}
\end{equation}
where $\mathbf{\Gamma}_{ij}=\delta_{i1}\delta_{j1}$. Equation (\ref{eq:return}) implies that the reflection coefficient and return loss is
\begin{subequations}
  \begin{align}
    \Gamma & = \mathbf{S}_{11}=0.2e^{j\frac{\pi}{2}}, \\
    L_\text{return} & = 10\log_{10}|\mathbf{S}_{11}|^{-2}=\SI{13.98}{\deci\bel}
  \end{align}
  \label{eq:RL}
\end{subequations}

\subsection{c}\label{sec:1c}
If the system is driven by a voltage wave incident at port 3 or 4 and measurements are made at port 4 or 3 while all other ports are matched then the outgoing wave at the port is (\ref{eq:s43}) or (\ref{eq:s34})
\begin{subequations}
  \begin{align}
    V^{-}_{4} & =\mathbf{S}_{43}V^{+}_{3}, \label{eq:s43} \\
    V^{-}_{3} & =\mathbf{S}_{34}V^{+}_{4} \label{eq:s34}
  \end{align}
\end{subequations}
When the system is driven at port $j$ and the output power is measured at port $i$ then the losses between port $i$ and $j$ for this system can be calculated using (\ref{eq:loss}).
\begin{equation}
  L_{ij}=10\log_{10}\frac{P_\text{in}}{P_\text{out}}=10\log_{10}|\mathbf{S}_{ij}|^{-2}
  \label{eq:loss}
\end{equation}
When evaluating the losses between port 3 and 4 it is evident that $\mathbf{S}_{34}=\mathbf{S}_{43}=0.5e^{-j\frac{\pi}{4}}$ so the loss is $\SI{6}{\deci\bel}$ both ways.

\subsection{d}\label{sec:1d}
If the system is driven by a voltage wave incident at port 1 while port 3 is shorted and ports 2 and 4 are matched
\begin{equation}
  V^{-}_{1}=\mathbf{S}_{11}V^{+}_{1} + \mathbf{S}_{1j}\overbrace{(1-\delta_{i1})\mathbf{\Gamma}\underbrace{\mathbf{S}\vec{V}^{+}}_{\vec{V}^{-}}}^{\vec{V}^{'+}}=(\mathbf{S}_{11}-\mathbf{S}_{13}\mathbf{S}_{31})V^{+}_{1}
  \label{eq:return_d}
\end{equation}
where $\mathbf{\Gamma}_{ij}=\delta_{i1}\delta_{j1}-\delta_{i3}\delta_{j3}$. Equation (\ref{eq:return_d}) implies that the reflection coefficient and return loss is
\begin{subequations}
  \begin{align}
    \Gamma & = \mathbf{S}_{11}-\mathbf{S}_{13}\mathbf{S}_{31}=0.2e^{j\frac{\pi}{2}}-0.5^2=-0.25+j0.2, \\
    L_\text{return} & = 10\log_{10}|\mathbf{S}_{11}-\mathbf{S}_{13}\mathbf{S}_{31}|^{-2}=\SI{9.89}{\deci\bel}.
  \end{align}
  \label{eq:RL}
\end{subequations}

\end{document}