\documentclass[12pt,a4paper]{IEEEtran}

%\pdfoutput=1

\usepackage[utf8]{inputenc}
\usepackage[T1]{fontenc}
\usepackage[english]{babel}
\usepackage{amsmath}
\usepackage{mathabx}
\usepackage{lmodern}
\usepackage{units}
\usepackage{siunitx}
\usepackage{icomma}
\usepackage{graphicx}
\usepackage{caption}
\usepackage{subcaption}
\usepackage{color}
\usepackage{pgf}
\DeclareMathOperator{\acosh}{arccosh}
\newcommand{\N}{\ensuremath{\mathbbm{N}}}
\newcommand{\Z}{\ensuremath{\mathbbm{Z}}}
\newcommand{\Q}{\ensuremath{\mathbbm{Q}}}
\newcommand{\R}{\ensuremath{\mathbbm{R}}}
\newcommand{\C}{\ensuremath{\mathbbm{C}}}
\newcommand{\rd}{\ensuremath{\mathrm{d}}}
\newcommand{\id}{\ensuremath{\,\rd}}
\usepackage{hyperref}
%\usepackage{a4wide} % puts the page numbering further down the page.
\usepackage{pdfpages}
\usepackage{epstopdf}
\DeclareGraphicsExtensions{.eps}

\title{Upgift 1}
\author{Marcus Malmquist}
\date{\today}

\begin{document}
\maketitle

\section*{Författarnas argument}
Begreppet teknisk determinism innefattar hurvida man kan förutspå hur teknik kommer att användas och om tekniken formar samhället eller samhället formar tekniken. Det är min uppfattning att ingen av de tre författarna tycker att teknik är deterministisk. Det är en ståndpunkt som jag delar eftersom man inte kan förutspå framtiden samt att teknikens och samhällets utveckling är sammanflätad.\\\\

\noindent Edgerton diskuterar teknikdeterminism utifrån ett militärt perspektiv. Främst diskuterar han myten om att gamla tekniker glöms bort när nya tekniker växer fram, samt att militären är bakåtsträvare vars framsteg inte bidrar till framsteg i det civila samhället. Han säger exempelvis att första världskriget var kemisternas krig på grund av att man bland annat använde mycket gas i sin krigsföring. Andra världskriget beskrivs å andra sidan som fysikernas krig på grund avsnitt atombomben.\\
En annan aspekt författaren tar upp är det faktum att måånga gamla tekniker används i senare krig. Exempel på detta är att utrustning från första världskriget användes i krigen i mellanöstern under 80-talet.\\\\

\noindent Nye hävdar att ny teknik skapas och formas av samhälle och kultur snarare än att det skulla vara tekniken som formar samhälle och kultur. Han säger även att när tekniken blir en del av samhället kan samhället inte tänka sig en vardag utan tekniken, och kan göra att samhällen blir fast med de val som gjorts (bland annat nämns val så som växelström eller likström, 110 V eller 220 V).
Jag uppfattar att 'technological desterminism' är ett centralt begrepp och att författaren framhävar att en viktig del i att anamma ny teknik har med bekvämligheter att göra. Om folk i allmänhet inte set någon ökad bekvämlighet eller fördelar i allmänhet blir tekniken en flopp.\\\\
\noindent Susan Douglas angriper ämnet ur teknikens synpunkt. Hon skriver om var tekniken gör med samhället och pekar både på möjligheter men även problem som har uppstått under förra seklet. Exempel på detta är att tekniken har gjort det möjligt för människor att kunna kommunicera lättare och över längre avstånd. Dett borde ha gjort att hela världen är mer sammankopplad men ur ett amerikanskt perspektiv har denna teknik enligt författaren gjort att länder vänder sig inåt och att amerikanare kan mindre om sin omvärd.
Även i denna text är 'technological desterminism' ett centralt begrepp.

\section*{Egen ``text''}
Den enda 'text' i meningen kulturell yttring jag haft omkring mig denna veckan är några avsnitt av den amerikanska serien Family Guy, som är en tecknad humorserie med tydliga samhällskritiska och stereotypa inslag. Serien spär mer eller mindre på alla stereotyper och myter om teknik och vetenskap.
Exempelvis målas bilden av vetenskapsmän upp som äldre män i vita rockar, oftas i ett labb med en tavla fylld av ekvationer i bakgrunden och en uppsättning glasrör för kemiska experiment i förgrunden. Teknikanvändare å andra sidan framställs som tekniskt okunniga.

Det teknikdeterministiska synsättet är i vissa episoder tydliga. Oftast följer episoden då ett tema, såsom hysterin kring Apple-produkter. I andra episoder är det mindre inslag såsom att tradidionella butiker blir utkonkurrerade av e-handel. I båda fallen upplever jag att de visar en bild av ett samhälle som helt omfamnar ny teknik och en utveckling som man i stort sätt är tvungen att vara med på.

I och med att hela serien är ganska samhällskritisk (framförallt mot det amerikanska samhället) berättas historien (som varierar från episod till episod) på så sätt att författarna kritiserar de delar av samhället eller händelser som är aktuella vid tiden då episoden gjordes, detta har skaparen av serien själv sagt under det episod 100. Även om detta kan göra faktainnehållet mindre trovärdigt tycker jag inte att det påverkar porträttet av samhället i och med att serien är samhällskritisk och därmed vill återspegla samhället med en satirisk twist.

\end{document}