% \documentclass[12pt,a4paper]{article}
\documentclass[12pt,a4paper]{IEEEtran}

%\pdfoutput=1

\usepackage[utf8]{inputenc}
\usepackage[T1]{fontenc}
\usepackage[english]{babel}
\usepackage{amsmath}
\usepackage{mathabx}
\usepackage{lmodern}
\usepackage{units}
\usepackage{siunitx}
\usepackage{icomma}
\usepackage{graphicx}
\usepackage{caption}
\usepackage{subcaption}
\usepackage{color}
\usepackage{pgf}
\usepackage[super]{nth}
\newcommand{\N}{\ensuremath{\mathbbm{N}}}
\newcommand{\Z}{\ensuremath{\mathbbm{Z}}}
\newcommand{\Q}{\ensuremath{\mathbbm{Q}}}
\newcommand{\R}{\ensuremath{\mathbbm{R}}}
\newcommand{\C}{\ensuremath{\mathbbm{C}}}
\newcommand{\rd}{\ensuremath{\mathrm{d}}}
\newcommand{\id}{\ensuremath{\,\rd}}
\usepackage{hyperref}
% \usepackage{a4wide} % puts the page numbering further down the page.
\usepackage{pdfpages}
\usepackage{epstopdf}
\DeclareGraphicsExtensions{.eps}

\title{Handin 7}
\author{Marcus Malmquist}
\date{\today}

\begin{document}
\maketitle
% 109 words
\noindent Cross-country skiing is an ancient Scandinavian activity dating back thousands of years ago.
Prior to the \nth{20} century, cross-country skiing was merely used as a means of transportation and only in the late \nth{19} and early \nth{20} century did people compete in more formal competitions.
One such competition is Vasaloppet which started in the 1920s and is a tribute to the Swedish king Gustaf Vasa, who ruled suding the early \nth{16} century.

Science made its entry into the world of cross-country skiing in the 1940s.
Before that time it had been mostly tradition and experience that played the larger role in exercising the sport.\\

\noindent Cross-country skiing is an excellent example of a sport where science have had a major impact on how it is exercised.
Before science made its entry to the sport (and a few decades after), the tracks man-made which meant that they were rather rough and not very stable, putting a lot of focus on lower-body strength and balance.
Many cross-country skiiers even used a single pole since most of the work were done byt the lower body.
As machine-made tracks was introduced, the tracks became alot more smooth and stable which meant that lower-body strength and balance was not as important as before.
Important skills of a cross-country skiier became upper-body and arm-strength and more and more skiiers made use of double poles.

An interesting (side-)effect of machine made tracks which worked in favor of the media and people watching cross-country skiing competitions, is that since large machines now had to be able to access the tracks, they could no longer go through inaccessible parts of the forest, but had to go through more open parts of the forest.
This made it possible to place cameras as wells as spectators by the tracks, allowing people to see virtually all parts of the tracks.

\noindent Daniel Svensson says in his thesis that the diversity among the cross-country skiiers were very small prior to th 1940's.
The typical cross-country skiier at that time lived in the north and had a background in forestry.
This was primarily due to the physical stress that came with the sport, so it was beleved that only a person who was used to a heavy physical workload in their daily life were fit to exercise cross-country skiing.
As such, elite cross-country skiiers were male.
Despite this, cross-country skiing have been more open to women than many other sports.
One of the reasons given for this is that it was beleved that endurance training for women was a positive thing, even if it should be exercised in smalled doses for women compared to men.\\
While sports in general favors men, it seems that sportification and a shift from natural training to rational training have been favorable to women (since it opens up the sport to women).
Barker-Ruchti \textit{et al.} writes in their paper \textit{Shifting, crossing and transforming gender boundaries in physical cultures} that is can be seen when science makes its entry in a sport that myths (such as heavy physical workload is dangerous to women) can be falsified.
The authors calls this \textit{shifting boundaries}.
They also use the term \textit{crossing boundaries} for when for example women participates in competitions that only allows male participants (such as Vasaloppet for cross-country skiing) and \textit{transforming boundaries} which is a reaction resulting from shifting and crossing meaning that society have accepted this group (e.g. women) performing these activities.

It should also be noted that most elite cross-country skiiers even today are white Europeans and (North-)Americans.
Since this is true for winter sports in general, it is not a problem with cross-country skiing specifically.\\

\noindent While winter sports such as Cross-country skiing heavily relies on nature and weather conditions to provide a good layer of snow, technology have allowed us to defy nature since we can create artificial snow, provided that the outdoor temperature is not too high.
We have even taken this as far as to create indoor facilities that provide both artificial snow and a feasable surrounding temperature to be able to practice Cross-country skiing on snow even if there is no snow and the temperature is too high.
An example of such facilities is Torsby Skidtunnel.
The technological impact of cross-country has even gone so far that (ironicly enough) snowfall which used to be a requirement for the sport and its competitions to even be made possible, has been regarded as something bad that creates unfair conditions.
An example of this is the FIS World Championship in 2015 when there was a heavy snowfall during the Men's 50 km race and Petter Northug made an impossible comback and won.
The snowfall make the machine-made tracks more dificult to follow in, so it is a major disadvantage to be in the lead.

Steven J. Haake writes in his article that the impact of technology can be seen by looking at statistics of world records.
While he does not mention cross-country skiing, the author shows that there is a huge difference in the impact of technology between sports.
For example, 100 m sprint have seen a world record time improvement of 24~\% over 108 years, where only 4~\% came from technology while one-hour bicycling which have seen a world record time improvement 220~\% over 111 years for pre-1972 bicycles and 321~\% for post-1972, indicating that at least 101~\% came from technology.
It would be more difficult to to make these comparations for cross-country skiing since the tack is typically different every time.


\end{document}