% \documentclass[12pt,a4paper]{article}
\documentclass[12pt,a4paper]{IEEEtran}

%\pdfoutput=1

\usepackage[utf8]{inputenc}
\usepackage[T1]{fontenc}
\usepackage[english]{babel}
\usepackage{amsmath}
\usepackage{mathabx}
\usepackage{lmodern}
\usepackage{units}
\usepackage{siunitx}
\usepackage{icomma}
\usepackage{graphicx}
\usepackage{caption}
\usepackage{subcaption}
\usepackage{color}
\usepackage{pgf}
\newcommand{\N}{\ensuremath{\mathbbm{N}}}
\newcommand{\Z}{\ensuremath{\mathbbm{Z}}}
\newcommand{\Q}{\ensuremath{\mathbbm{Q}}}
\newcommand{\R}{\ensuremath{\mathbbm{R}}}
\newcommand{\C}{\ensuremath{\mathbbm{C}}}
\newcommand{\rd}{\ensuremath{\mathrm{d}}}
\newcommand{\id}{\ensuremath{\,\rd}}
\usepackage{hyperref}
% \usepackage{a4wide} % puts the page numbering further down the page.
\usepackage{pdfpages}
\usepackage{epstopdf}
\DeclareGraphicsExtensions{.eps}

\title{Handin 2}
\author{Marcus Malmquist}
\date{\today}

\begin{document}
\maketitle
\section{Introduction}
I chose to study the laboratory environment from the film \textit{Contact} from 1997. The plot of the film is about a group of scientists who uses an array of radio telescopes to look into space in search of (intellegent) life on other planets. The laboratory is therefore set up to control the telescopes and process the data from them.

\section{Laboratory set up}
The laboratory is located in the desert in Central America and is comprised of a large open area where the radio telescopes are placed and a smaller control center for controlling the telescopes and process the data from them.

The laboratory is instantly recognizable as a physics laboratory since it contains many large machines including osciloscopes and power supplies and many computer screens displaying terminals and general ``science-looking'' interfaces.

The people working there are a small group of the typical geeky and out-of-fashion physicist. They are initially working at a slow pace (one of them is fishing in a small inflatable pool during the day another one is watching TV).

Once they discover something interesting they start rushing between monitors, typing in some commands and pretty looking graphics appear on the screens. After processing the signal they start discussing what it might mean and discovers that it is a signal from outer space.

\section{Comparison to Latour and Woolgar's observations}
The observations in the text by Latour and Woolgar indicates that the output of a laboratory is scientific papers, while the work in the laboratory in \textit{Contact} has a more visionary nature. While they have a clear goal of finding life on other planets, the method that they use may not yield any results and they do not even know if there are any results to find. As such (and the fact that it is a Hollywood film) shifts the focus from producing a scientific papers to making a historical discovery. Whether or not a scientific paper will be published in the end is not known.

In many other senses the laboratory works in the way described by Latour and Woolgart. For example they use machines to which they give some input data (such as instructions to the telescopes) and observer the output data (such as observed signal), where the output data is of interest to the scientists work.

However, one thing Latour and Woolgar mentions is that each person in the laboratory has separate responsibilities. That is not something that appear in the film apart from a blind scientist who is exceptionally good at listening to the signal and (obviously) not so good at operating the telescopes or the computers.

\section{Comparison to a real laboratory}
The laboratory they use in the film is a real laboratory, and the data you see on the computer screens is real data although slightly modified (perhaps to make it more visually appealing) according to the text by Kirby.

What differs more from a real laboratory is the scientists themselves and the way them do research. I am mostly referring to the fact that astronomers are depicted as geeks wearing hawaii shirts or t-shirts with the periodic table of elements printed on them. From my own experience this is how scientists are regularly depicted in Hollywood films and that stereotypes make it easier for the viewers to understand the characters and the plot.

One thing that bothers me about the way the astronomers do their work in the movie, having worked in laboratories and done signal processing myself, is that the way they appear to process the data from the radio telescopes is by listening to it with headphones. Even if the signal band is mapped to frequencies humans can hear it is unlikely that what they hear is anything but gibberish. It also requires the scientists to constantly focus on listening to the signal, which clearly is not happening. It is probably easier for the average viewers today understand what is going on this way so it works for a Hollywood film.


\end{document}