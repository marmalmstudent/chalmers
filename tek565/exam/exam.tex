\documentclass[12pt,a4paper]{article}
% \documentclass[12pt,a4paper]{IEEEtran}

%\pdfoutput=1

\usepackage[utf8]{inputenc}
\usepackage[T1]{fontenc}
\usepackage[swedish]{babel}
\usepackage{amsmath}
\usepackage{mathabx}
\usepackage{lmodern}
\usepackage{units}
\usepackage{siunitx}
\usepackage{icomma}
\usepackage{graphicx}
\usepackage{caption}
\usepackage{subcaption}
\usepackage{color}
\usepackage{pgf}
\usepackage[super]{nth}
\newcommand{\N}{\ensuremath{\mathbbm{N}}}
\newcommand{\Z}{\ensuremath{\mathbbm{Z}}}
\newcommand{\Q}{\ensuremath{\mathbbm{Q}}}
\newcommand{\R}{\ensuremath{\mathbbm{R}}}
\newcommand{\C}{\ensuremath{\mathbbm{C}}}
\newcommand{\rd}{\ensuremath{\mathrm{d}}}
\newcommand{\id}{\ensuremath{\,\rd}}
\usepackage{hyperref}
% \usepackage{a4wide} % puts the page numbering further down the page.
\usepackage{pdfpages}
\usepackage{epstopdf}
\DeclareGraphicsExtensions{.eps}

\title{\textbf{TEK565 Hemtenta}\\Uppgitf 1}
\author{Anonymt ID: 116}
\date{}

\begin{document}
\maketitle
% 109 words
\section{Inledning}
Under kursens gång har vi tittat på hur teknik och populärkultur har format och formats av samhället.
Genus har varit en röd tråd genom hela kursen då vi i de flesta områden diskuterat hur vilket utrymme kvinnor får i dessa områden, både historiskt och i nutid.\\

\noindent Till en början studerade och diskuterade vi främst teknik och hurvida tekniken formar eller formas av samhället.
Initialt tittade vi på olika myter inom teknik inom populärkultur såsom Hollywood-produktioner och försökte slå hål på dessa genom att först titta på filmer eller läsa texter som påvisade dessa myter och sedan läste vi texter som pekade på varför de var myter.
Vi tittade även på de som utvecklar tekniken och diskuterade vilka som omfattas av denna definition och även vilka som borde omfattas av samma definition.\\
Syftet med detta var främst att diskutera teknikyrken ur ett genusperspektiv, där det idag är typiska manliga yrken som hamnar i gruppen teknikutvecklare och typiskt kvinliga yrken som hamnar i gruppen som teknikanvändare.
Eftersom användarna i många fall till stor del bidrar till utvecklandet av produkter skulle detta kunna utjämna könsfördelningen bland teknikutvecklare samt höja statusen för teknikanvändare.\\
Exempelvis tittade tittade vi på utvecklingsmiljöer för teknik (i det här fallet främst laboratoriemiljö) ur en antropologs synvinkel och ämnade att beskriva denna samt de som arbetade i dessa miljöer på ett sätt som en antropolog skulle kunna ha gjort.
Vi tog även en mer handfast djupdykning och diskuterade om yrkesgrupper som använder teknik i sitt dagliga arbete borde tillhöra de yrkesgrupper som utvecklar teknik.
Ett exempel på en sådan problemställning var om bibliotekarier (vilka använder datorer i sitt dagliga arbete) borde grupperas tillsammans med mjukvaruutvecklare som teknikutvecklare.\\

\noindent Fram mot slutet av kursen tittade vi mer på teknikens roll i populärkultur och sociala sammanhang.
Här analyserade vi bland annat klimat, sport och genteknik.\\
När vi diskuterade genteknik tittade vi på filmen \textit{Gattica} som handlar om ett framtidssamhälle där gentekniken blivit så avancerad att det var möjligt att på konstgjord väg producera barn med bästa möjliga genetiska egenskaper.
Detta gjorde att människor som inte kommit till världen på detta sätt, utan det naturliga, ansågs tillhöra en underklass av människor med ogynnsamma genetiska förhållanden.
Denna insyn väckte en diskussion om det är rätt eller fel att välja bort människor med genetiska egenskaper som anses vara ogynnsamma. Även om detta inte ledde oss direkt in på diskussionen om sport, hade det varit intressant att diskutera hur sporten hade påverkats av ett sådant samhälle.

Diskussionen om sport handlade mycket om sporters tendens att övergå från naturlig tävlings- och träningsform till rationell tävlings- och träningsform (så kallad sportifiering).
Vad som menas med denna övergång är att sporter oftast initialt har en naturlig träningsform som går ut på att till exempel löpare springer i skogen och simmare simmar i sjöar.
Vid det här stadiet är det erfarenhet och kultur som styr sträningen.
När sporten övergår till mer rationell börjar till exempel löpare springa på förutbestämda spår med standardiserade längder.
Träningen styrs nu av vetenskap och forskning.

En annan viktig del i diskussionen om sport har varig genus och det utrymme kvinnor har fårr i sporten samt hur sportifiering har påverkat kvinnors roll i sporten.

\section{Sammanfattning och Diskussion}
Under kursens gång har vi läst och diskuterat material (främst text och film) utifrån ett genusperspektiv, etikperspektiv och kulturperspektiv.
Syftet med detta har (som jag förstått det) varit att få oss ingenjörssudenter att börja tänka på vårt framtida ansvar när vi som ingenjörer behöver ta beslut som kan påverka eller kränka människor, djur och natur.\\

\noindent Inför första seminariet hade vi tittat på hur myten om den ensamma uppfinnare framträder i Hollywood-produktioner.
Som exempel tittade vi på ``The Imitation Game'' där Allan Turing framställs som en nyckelfigur i de allierades försök att dekryptera tyskarnas meddelanden.
Efter att ha diskuterat och läst artiklar om detta kunde vi slå hål på myten i och med att det var en stor grupp arbetare (varav många kvinnor vilket inte framgår i filmen) som jobbade med att bygga maskinen som dekrypterade meddelanden.
Framförallt krävdes att flera människor arbetade med maskinen för att den skulle fungera.\\

\noindent Senare diskuterade vi myten om att tekniken är revolutionär snarare än evolutionär.
Vi läste texter från författare som var både för och emot detta, men övervägande emot.
Texterna vi läste som var hävdade att tekniken är revolutionär visade exempel på hur tekniken förändrat samhället medan texterna som hävdade att tekniken snarare är evolutionär visade exempel på hur teknik skapats, förändrats och försvunnit i takt med att samhället förändrats.

Vår slutsats som grupp var att tekniken var evolutionär och altså formades altså av samhället men min personliga åsikt är att samhället öppnar upp för teknisk utveckling och att när teknik blivit en tillräckligt stor del av samhället kommer den att ge samhället nya möjligheter att utvecklas.
Utvecklingen av internet och telefoner tycker jag är två tydliga exempel på detta.\\

\noindent När vi tittade på genus i teknikyrken hade vi som exempel hur kvinnor framställs och representeras bland mjukvaruutvecklare.
Vi hade tidigare i kursen sett att när yrken blir mer tekniskt krävande ökar statusen på (de ofta kvinnodominerade) yrket och kvinnor har successivt bytts ut mot män.
Exempel på detta är mejeribranchen där det var ett kvinnodominerat yrke fram tills tekniken fick ett större utrymme och blev ett mannsdominerat yrke.
Mjukvarubranchen genomgick samma förändring då det till en början var kvinnor som tidigare jobbade som telefonoperatörer som jobbade med datorerna, vilket ansågs vara ett lågstatusarbete.
När datorerna krävde mer tekniskt kunnande byttes successiv kvinnorna ut mot män.

Än idag är det ett problem med att andelen kvinnor bland mjukvaruutvecklare är låg.
I texterna vi läste föreslogs bland annat att branchen skulle göra det mer attraktivt för kvinnor att jobba samt att yrken som använder teknik i sitt vardagliga arbete, så som bibliotekarier, skulle räknas till mjukvaruutvecklare.
Vi kom inte riktigt fram till någon slutsats gällande detta men min åsikt är att en teknikanvändare inte är mer mjukvaruutvecklare än yrkesförare är fordonstekniker eller restaurangbesökare är kockar.

Däremot är det en intressant fråga att personer som arbetar med att göra PowerPoint presentationer (vilket i och för sig skulle kunna vara bibliotekatier) skulle klassas som mjukvaruutvecklare.
Anledningen till detta är att det finns så kraftfulla verktyg för att göra hemsidor att det inte är någon större skillnad att designa en PowerPoint presentation som att designa en hemsida.
Eftersom man även kan göra en hemsida ``för hand'' är detta något som faller under kategorin mjukvaruutvecklare.

När det gäller mjukvarubranchens uppgift att göra branchen mer attraktiv för kvinnor håller jag inte med om att det är branchens uppgift utöver att se till att alla medarbetare behandlas lika, vilket borde vara en självklarhet inom alla brancher.
Eftersom mjukvarubranchen har monopol på mjukvara, och hela samhället använder mjukvara förlorar branchen inget på att ha en homogen arbetskår.
Det som skulle kunna förbättras med en heterogen arbetskår är troligtvis att fler grupper i samhället skulle uppskatta mjukvaran bättre.
Därför tycket jag det är de grupper som är underrepresenterade i mjukvarubranchen som antingen borde bojkotta branchen och alla dess produkter eller söka sig till den för att förbättra den.
Då tjejer generellt sätt har högre gymnasiebetyg än killar skulle det inte vara några svårigheter att få en plats på utbildningar inom IT-branchen.
Min åsikt om insatser för att få tjejer mer intresserade av mjukvaruutveckling är att detta är något bra, men man ska heller inte glömma att det finns väldigt mycket att vinna på att visa killar att bara för att man tycker det är kul med datorspel är det ett bra val att utbilda sig till datoringenjör.\\

\noindent Vi tittade även på om dokumentärer och filmer om miljöförstörning faktiskt hade någon större effekt på hur de som tittade på dokumentärerna faktiskt ändrade sin livsstil.
Vi läste vetenskapliga artiklar om detta för att ta reda på vad forskning kunnat visa och fann att den långsiktiga effekten var väldigt liten och att tittarna oftast la om in livsstil under några veckor efter att ha sett filmen.
En annan intressant aspekt som togs upp var hur stort intryck filmen gjorde på människors vilja att förändra sin livsstil om man visade en optimistisk syn på problemet (till exempel visa att med rätt insatser kan vi lösa klimatproblemen) eller en pesimistisk syn (till exempel att det nästan är för sent att göra något).
Artiklarna och texterna vi läste pekade på att ett optimistiskt (men inte alltför optimistiskt) synsätt var att föredra.

Det är såklart svårt att ha en åsikt om vad som är bäst, men jag tror att det viktigaste är att utveckla produkter som inte kräver att vi drastiskt ändrar vår livsstil och att göra det ekonomiskt fördelaktigt att välja miljövänliga alternativ är det bästa sätter att uppnå positiv förändring.\\

\noindent När vi närmade oss slutet av kursen diskuterade vi genetik i samhället.
Filmen \textit{Gattica} användes som underlag för diskussion och frågan som var i fokus är om samhället ska tolerera att man med hjälp av genetik ska bestämma vilka som får komma till livet och villka som ska förkastas innan de ens blivit till.
I samhället som avbildades i filmen fanns det människor som blivit till på det naturliga sättet och behandlades därför som en underklass av samhället.

Vi diskuterade vilka potentiella fördelar, nackdelar och etiska problem som skulle kunna uppstå i ett sådant samhälle.
Som en grupp kom vi inte fram till en enhällig ståndpunkt i frågan, men vi kunde alla enas om att samhället i filmen var problematiskt med en moralisk kompass som inte pekade rakt norr.
Å ena sidan kunde man med hjälp av detta skapa en värld fri från genetiska sjukdommar och minska på mänskligt lidande.
Å andra sidan bestämmer man då vilka egenskaper som är önskvärda och vilka som är mindre önskvärda, snarare än bara se egenskaper som olika (varken positiva eller negativa).
Det framgick inte i filmen om man kunde uppskatta om fostren hade hög intellegensnivå och tog detta i beakande när man valde vilka som skulle få komma till livet, eller vad man gjorde med foster som hade hög intellegensnivå men i övrigt ogynnsamma genetiska förhållanden.
Jag finner det troligt att många framstående tänkare genom historien valde sin karriär delvis på grund av att de inte hade så bra fysiska egenskaper och att i ett samhälle där sådana individer sållas bort redan innan födseln skulle vara ogynnsammt för samhället.
Till exempel skulle personer som Stephen Hawking säkerligen inte ha fått komma till världen i det samhälle som avbildas i \textit{Gattica}.

Till viss del kan det ändå vara bra att ha möjligheten att kolla om personer har medfödda sjukdommar om det är berättigat.
Precis som att man har intellegenstest och psykologitest för yrken som kräver hög intellegens eller att man kan utstå tuffa psykiska förhållanden, tycker jag det kan vara berättigat att yrken där man utsätts för stor fysisk press (till exempel astronaut och stridspilot) kan kräva någon form av inträdesprov som innefattar att man testar delar av ens genom.\\

\noindent Det sista ämnet vi diskuterade var vad som menades med sportifiering och hur sporter påverkades av detta.
Vi läste artiklar och texter som förklarade sportifiering och tittade på statistik för världsrekord.
tabellerna visade inte enhälligt att någon större förändring av hur mycket världsrekord förändrats under 1900-talet (vilket är den tid då de flesta sporter sportifierades), men det kan ju å andra sidan betyda att man inte närmat sig den naturliga träningens begränsningar.
Man har däremot kunnas se tydligare hur tekniken har påverkat statistiken över resultat.
I de artiklar vi läste framgick det att i stavhopp efter Andra Världskriget ökade resultaten kraftigt tack vara nya material i stavarna och att i cykling under 1970-talet skapades en ny typ av cykel som gjorde att resultaten kraftigt ökade så kraftigt att den nya typen av cyklar blev en kategori för sig.
I båda fallen, men framförallt stavhopp, behövdes nya tekniker för att handskas med den tekniska utvecklingen.
Utvecklandet av dessa tekniker gynnades säkerligen av en träningsform som bygger på vetenskap och forskning istället för kultur och erfarenhet.

Något som syntes tydligt i när en sport sportifierades var att kvinnor fick större plats i sporten (i en del fall fick de i och med sportifieringen möjligheten att ens delta).
Detta kan förvisso förklaras till viss del med att samhället hade utvecklats, men även att när vetenskapen gör sitt inträde kan man motbevisa påståenden så som att det är skadligt för kvinnor att utsättas för tunga fysiska påfrästningar så som löpning.

\end{document}