% \documentclass[12pt,a4paper]{article}
\documentclass[12pt,a4paper]{IEEEtran}

%\pdfoutput=1

\usepackage[utf8]{inputenc}
\usepackage[T1]{fontenc}
\usepackage[english]{babel}
\usepackage{amsmath}
\usepackage{mathabx}
\usepackage{lmodern}
\usepackage{units}
\usepackage{siunitx}
\usepackage{icomma}
\usepackage{graphicx}
\usepackage{caption}
\usepackage{subcaption}
\usepackage{color}
\usepackage{pgf}
\newcommand{\N}{\ensuremath{\mathbbm{N}}}
\newcommand{\Z}{\ensuremath{\mathbbm{Z}}}
\newcommand{\Q}{\ensuremath{\mathbbm{Q}}}
\newcommand{\R}{\ensuremath{\mathbbm{R}}}
\newcommand{\C}{\ensuremath{\mathbbm{C}}}
\newcommand{\rd}{\ensuremath{\mathrm{d}}}
\newcommand{\id}{\ensuremath{\,\rd}}
\usepackage{hyperref}
% \usepackage{a4wide} % puts the page numbering further down the page.
\usepackage{pdfpages}
\usepackage{epstopdf}
\DeclareGraphicsExtensions{.eps}

\title{Handin 5}
\author{Marcus Malmquist}
\date{\today}

\begin{document}
\maketitle
\noindent In the text ``Social-Scientist India'' by Sambit Mallick, E. Haribabu and S. G. Kulkarni, the authors discuss why engineering is a popular career choice in India.
This idea originates from before World War II, when India decided that in order to bo successfull as a country, they needed to properly industrialize India.
This would be done by shift towards incorporating machines into their agriculture and build the foundation for industries.

The indian engineer Mokshagundam Visvesvaray had a famous slogan ``Industrialize or perish.''
He was one of those who advocated, as the slogan might reveal, that India should make every attempt to become an industrialized nation.
He claimed, in the 1930's, that this was the age of the machine.
The only way to achieve this industrialization was throught education, and change the focus of universities from theoretical research to a more practical approach to both research and education.\\

\noindent In the text ``Naturvetarna, ingenjörerna och valfrihetens samhälle'' by Löwheim, the author discuss engineering and science in Sweden during the Cold War, and post World War II in general.
In the 1960's OECD estimated that the number of engineers and scientists in Sweden would decrease in the near future, partly due to a lack of posoitive propagande in favor of science and engineering.
As the superpowers USA and Soviet put great efforts into scientific advancement, it became clear during the 50's and 60's that not only Sweden had fallen behind, but all of Europe was concidered scientifically inferior.
This lead to most European countries joining forces and develop a modern school where science had been given more time and attention.

During the 1970's the number of people who chose an education in science or engineering declined.
A rising skepticism in society had many people doubt that the needed for scientists and engineers would be as geat as had previously estimated.
OECD had in the 60's estimated that the number of people choosing those educations would incdease, and that it would become more common that people in other fields of work, such as librarians and journalists, would have an education in science or engineering.
This, however, had not happened.

Alarming reports on the shortage of engineers and scientists had once again entered the debate in the 1990's and the government blamed the government of the 70's and 80's for this.

Löwheim concluded by saying that the reports regarding the shortage of scientists and engineers in the 70's and 90's had very little impact on the political landscape in Sweden.

\noindent The two texts describe the situation in two (at the time) very different countries.
In Mallick's text he talks about a country that have or are near gaining independence and must make choices in order to not fall behind western countries.
As such India needed to encourage people to choose a career in science or engineering in order to escape poverty and turn India into an industrialized country able to compete internationally.
Löwheim on the other hand describes the situation for a country that is already competing internationally hand have done so for some time.
The problem Sweden faced was that if falling behind, would not have much at all to compete with on a global market.

\noindent It is difficult for me to have an opinion the text ``Only 7 per cent engineering graduates employable: What's wrong with India's engineers?'' since i do not know if India today is a trustworthy source, but to me it seems that there is a misunderstanding of what the purpose of education is.
Both among the studends and companies.
The purpose of an education in engineering is to learn a way of thinking and to solve problems.
As such you should not be focusing too much on the contents of the courses as you are likely doing something completely different when you have graduated, and companies expecting to find people to instantly be profitable among newly graduated engineers are mistaken.

\noindent I do not recognize the situation described by Löwheim and Mallick since they were relevant many years ago.
I do however beleve that as when a country goes from being under-developed to developed the generation that grew up when it the counrty was under-developed work hard to educate themselves and contribute to society while the generation that grew up when the country was developed will put much less effort into doing the same.
When it comes to the content by All India Bakshod i do not think it is relevant for engineering students in Sweden since engineers do not have such a high status among the general public.
I would not be surprised if it shares more similatiries with medical school in Sweden, as it has a high status among the general public.
In any case, I do not think we listen as much to our parents in Sweden as it seems you do in India, so the situation is less likely to arise here.

\end{document}