% \documentclass[12pt,a4paper]{article}
\documentclass[12pt,a4paper]{IEEEtran}

%\pdfoutput=1

\usepackage[utf8]{inputenc}
\usepackage[T1]{fontenc}
\usepackage[english]{babel}
\usepackage{amsmath}
\usepackage{mathabx}
\usepackage{lmodern}
\usepackage{units}
\usepackage{siunitx}
\usepackage{icomma}
\usepackage{graphicx}
\usepackage{caption}
\usepackage{subcaption}
\usepackage{color}
\usepackage{pgf}
\newcommand{\N}{\ensuremath{\mathbbm{N}}}
\newcommand{\Z}{\ensuremath{\mathbbm{Z}}}
\newcommand{\Q}{\ensuremath{\mathbbm{Q}}}
\newcommand{\R}{\ensuremath{\mathbbm{R}}}
\newcommand{\C}{\ensuremath{\mathbbm{C}}}
\newcommand{\rd}{\ensuremath{\mathrm{d}}}
\newcommand{\id}{\ensuremath{\,\rd}}
\usepackage{hyperref}
% \usepackage{a4wide} % puts the page numbering further down the page.
\usepackage{pdfpages}
\usepackage{epstopdf}
\DeclareGraphicsExtensions{.eps}

\title{Handin 2}
\author{Marcus Malmquist}
\date{\today}

\begin{document}
\maketitle

\noindent In the article ``Kvinnorna IT-förlorare?'' the author claims that women are only seen as users of information technology and not as developers.
An example that was given was that librarians and teachers, who use digital platforms in their daily work, should not only be seen as users of technology, but also developers.
This needs to changed to include women in the concept of information technology to include both men and women as both users and developers.
In order to solve this, the author argues that women need to be more frequently occuring in top positions in the information society, in addition to being more visible in general.

The author's sources also highlights that there are a lot of work in information technology that are appealing to women. An problem in raising awareness of this is that a large portion of the media more often focus on the absense of women in there IT-world, rather than focusing on the accomplishments of the women who are working in the IT-world.

Finally, the author encourages women to get more involved in the information society.\\

\noindent In the article ``Only the clothes changed: Women Operators in British Computing and Advertising 1950-1970'' the author points out that even if the technology have raidly developed, the advertisements have not changed.
Even if women have played an important role in the transition from pre-electronic to electronic office computing, they have been used to advertise computers and related technology in order to show that there technology is easy to use.
this has damaged the professional position of women in long run acording to the author.
Women are seen as uneducated and low-cost labour. In addition to this, companies have used women and sexualty to further increase their sales.\\

\noindent I chose to analyze the recent American tv-series \textit{Mr. Robot}.
The plot features a small group of hackers which attempts to erase all digital records of dept, paticularly to mega-corporations.
The mega-corporation in focus is called ``E-corp,'' which (perhaps intentionally) has a company logo that looks very much like that of \text{Dell inc}.

The hackers of the group are a multicultural group of people, both male and female, but the leader is a young white male (who, in addition to being a hacking genious, suffers from social anxiety disorder and clinical depression).
The leader of the hackers group, Eliot, is portrayed as perhaps a modern stereotypical hacker.

The image of the stereotypical hacker, a teenage boy living in his mother's basement, who can launch a full-scale nuclear missile attack from an old computer and a dial-up modem using only a few keystrokes, originates from a time when nobody thought about cyber security.
The protagonist of \textit{Mr. Robot} instead portrays a young adult who uses sophisticated tools and realistic methods to cause damage that at least seems proportional to his efforts.
When it comes to his mental health, I find it difficult to believe that this is frequently occuring amongst modern computer enthusiasts and hackers, but seems more likely a characteristic chosen to enhance the image of an odd person.
The same goes with his drug abuse. Although it is probably not specific to hackers, drugs seem more commonly used by people today.

The creators of \textit{Mr. Robot} have attempted to be realistic when it comes to portraying hackers (the description of who is a part of their group is similar to that of \textit{Anonymous}).
In order to do this, they hired multiple experts in the field of cyber security.

Being a software developer and GNU/Linux enthusiast myself I can confirm that they are using proper terminal commands and that most of what is displayed on the screen is relevant.
Sometimes, however, they only use random scrolling text on the screen when doing thing.
For example, at one point in the series the protagonist attempts to hack the FBI, but what is displayed on the sceen is simply a repository update (i.e. checking for software updates).

\end{document}