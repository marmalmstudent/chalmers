\documentclass[12pt,a4paper]{article}

%\pdfoutput=1

\usepackage[utf8]{inputenc}
\usepackage[T1]{fontenc}
\usepackage[english]{babel}
\usepackage{amsmath}
\usepackage{mathabx}
\usepackage{lmodern}
\usepackage{dcolumn}
\usepackage{units}
\usepackage{siunitx}
\usepackage{icomma}
\usepackage{graphicx}
\usepackage{caption}
\usepackage{subcaption}
\usepackage{color}
\usepackage{pgf}
\DeclareMathOperator{\acosh}{arccosh}
\newcommand{\N}{\ensuremath{\mathbbm{N}}}
\newcommand{\Z}{\ensuremath{\mathbbm{Z}}}
\newcommand{\Q}{\ensuremath{\mathbbm{Q}}}
\newcommand{\R}{\ensuremath{\mathbbm{R}}}
\newcommand{\C}{\ensuremath{\mathbbm{C}}}
\newcommand{\rd}{\ensuremath{\mathrm{d}}}
\newcommand{\pout}{\ensuremath{\,\rd}}
\usepackage{hyperref}
%\usepackage{a4wide} % puts the page numbering further down the page.
\usepackage{pdfpages}
\usepackage{epstopdf}
\DeclareGraphicsExtensions{.eps}

\title{Handin 4 and 5: Large signal capacitance modelling}
\author{Marcus Malmquist, marmalm}
\date{\today}

\begin{document}
\maketitle

\section{Matching network}\label{sec:1}
In order to determine the last parameters an input and output matching network had to be constructed. The reflection coefficients that needed to be matched to a $\SI{50}{\ohm}$ load can be seen in (\ref{eq:reflection}).
\\\\
The matching network on both input and output consisted of a transmission line and an open stub. Even though a short stub would would result in a higher bandwidth according to the \textit{ADS} built-in Smith chart tool, the short stub caused the output power to decreas with increase input power. The input (gate-side) and output (drain-side) matching network, along with the setup used in the simulations can be seen in Figure~\ref{fig:scematic}.
\begin{subequations}
  \label{eq:reflection}
  \begin{align}
    \Gamma_S&=0.24728+j0.35739, \label{eq:reflection_s}\\
    \Gamma_L&=0.29251+j0.39982 \label{eq:reflection_l}
  \end{align}
\end{subequations}
\begin{figure}[h]
  \centering
  \noindent\makebox[\textwidth]{\includegraphics[width=\textwidth]{Task4_sc.pdf}}
  \caption{The scematic that was used to produce the results presented in this report.}
  \label{fig:scematic}
\end{figure}
When the matching networks were implemented the final tuning process for Cds, Cgspi, Cgs0, Cgdpi, Cgd0, Cgdpe began. There final values can be seen in Figure~\ref{fig:scematic} or Table~\ref{tab:results}.
\begin{table}[h]
  \centering
  \begin{tabular}{|l|l|} \hline
    Cds & \SI{389}{\femto\farad} \\
    Cgspi & \SI{8.86}{\femto\farad} \\
    Cgs0 & \SI{695}{\atto\farad} \\
    Cgdpi & \SI{39.1}{\femto\farad} \\
    Cgd0 & \SI{1.46}{\femto\farad} \\
    Cgdpe & \SI{91.3}{\femto\farad} \\ \hline
  \end{tabular}
  \caption{The final values for the capacitive components.}
  \label{tab:results}
\end{table}

The $P_\text{out}$ vs $P_\text{in}$ curve using these values (red dots), along with the data from \textit{Load\_pull.midf} (blue curve) can be seen in Figure~\ref{fig:pout_pin}. The simulated data follows the data quite well.
\begin{figure}[h]
  \centering
  \noindent\makebox[\textwidth]{\includegraphics[width=\textwidth]{Task4.pdf}}
  \caption{The simulated curve (red dots) and the data from \textit{Load\_pull.midf} (blue curve).}
  \label{fig:pout_pin}
\end{figure}
\end{document}