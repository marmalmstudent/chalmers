\documentclass[12pt,a4paper]{article}

%\pdfoutput=1

\usepackage[utf8]{inputenc}
\usepackage[T1]{fontenc}
\usepackage[english]{babel}
\usepackage{amsmath}
\usepackage{mathabx}
\usepackage{lmodern}
\usepackage{dcolumn}
\usepackage{units}
\usepackage{siunitx}
\usepackage{icomma}
\usepackage{graphicx}
\usepackage{caption}
\usepackage{subcaption}
\usepackage{color}
\usepackage{pgf}
\DeclareMathOperator{\acosh}{arccosh}
\newcommand{\N}{\ensuremath{\mathbbm{N}}}
\newcommand{\Z}{\ensuremath{\mathbbm{Z}}}
\newcommand{\Q}{\ensuremath{\mathbbm{Q}}}
\newcommand{\R}{\ensuremath{\mathbbm{R}}}
\newcommand{\C}{\ensuremath{\mathbbm{C}}}
\newcommand{\rd}{\ensuremath{\mathrm{d}}}
\newcommand{\id}{\ensuremath{\,\rd}}
\usepackage{hyperref}
%\usepackage{a4wide} % puts the page numbering further down the page.
\usepackage{pdfpages}
\usepackage{epstopdf}
\DeclareGraphicsExtensions{.eps}

\title{Handin 3: DC modelling.}
\author{Marcus Malmquist, marmalm}
\date{\today}

\begin{document}
\maketitle

\section{Task 3}\label{sec:1}
The simulation was set up according to Figure~\ref{fig:scematic}. I could not quite get the lines to fit in the $I_d$ vs. $V_{ds}$ plot for higher $V_{ds}$ and the lower values of $V_{gs}$ in the $I_d$ vs. $V_{gs}$ plot but the tuned values can be seen in Table~\ref{tab:tuned}. With some more fine-tuning it might have been possible to achieve a better curve fit.
\begin{figure}[h]
  \centering
  \noindent\makebox[\textwidth]{\includegraphics[width=\textwidth]{Task3_sc.pdf}}
  \caption{The scematic that was used to produce the results presented in this report.}
  \label{fig:scematic}
\end{figure}
\begin{table}[h]
  \centering
  \begin{tabular}{|l|l|}\hline
    $I_{pk0}$ & $\SI{1.025}{\ampere}$ \\
    $V_{pks}$ & $\SI{50}{\milli\volt}$ \\
    $P_1$ & 0.154 \\
    $I_j$ & $\SI{440}{\milli\ampere}$ \\
    $P_g$ & 0.05 \\
    $V_{jg}$ & $\SI{1.048}{\volt}$ \\
    $R_s$ & $\SI{561}{\milli\ohm}$ \\
    $R_d$ & $\SI{283}{\milli\ohm}$ \\
    $R_g$ & $\SI{1.05}{\ohm}$ \\ \hline
  \end{tabular}
  \caption{The values of the tuned parameters}
  \label{tab:tuned}
\end{table}

The resulting graphs can be seen in Figure~\ref{fig:id_vds} and Figure~\ref{fig:id_vgs}.
\begin{figure}[h]
  \centering
  \noindent\makebox[\textwidth]{\includegraphics[width=\textwidth]{Task3.pdf}}
  \caption{The $I_d$ vs. $V_{ds}$ plot after some tuning. The blue curves represent the data from the .midf-file and the red curves represents the simulated data.}
  \label{fig:id_vds}
\end{figure}
\begin{figure}[h]
  \centering
  \noindent\makebox[\textwidth]{\includegraphics[width=\textwidth]{Task3_2.pdf}}
  \caption{The $I_d$ vs. $V_{gs}$ plot after some tuning. The blue curves represent the data from the .midf-file and the red curves represents the simulated data.}
  \label{fig:id_vgs}
\end{figure}

\end{document}