\documentclass[12pt,a4paper]{article}

%\pdfoutput=1

\usepackage[utf8]{inputenc}
\usepackage[T1]{fontenc}
\usepackage[english]{babel}
\usepackage{amsmath}
\usepackage{mathabx}
\usepackage{lmodern}
\usepackage{units}
\usepackage{siunitx}
\usepackage{icomma}
\usepackage{graphicx}
\usepackage{caption}
\usepackage{subcaption}
\usepackage{color}
\usepackage{pgf}
\DeclareMathOperator{\acosh}{arccosh}
\newcommand{\N}{\ensuremath{\mathbbm{N}}}
\newcommand{\Z}{\ensuremath{\mathbbm{Z}}}
\newcommand{\Q}{\ensuremath{\mathbbm{Q}}}
\newcommand{\R}{\ensuremath{\mathbbm{R}}}
\newcommand{\C}{\ensuremath{\mathbbm{C}}}
\newcommand{\rd}{\ensuremath{\mathrm{d}}}
\newcommand{\id}{\ensuremath{\,\rd}}
\usepackage{hyperref}
%\usepackage{a4wide} % puts the page numbering further down the page.
\usepackage{pdfpages}
\usepackage{epstopdf}
\DeclareGraphicsExtensions{.eps}

\title{Handin 1}
\author{Marcus Malmquist, marmalm}
\date{\today}

\begin{document}
\maketitle

\section{README}
I solved this task using Python (and Cython where needed for performance increase) because the MATLAB license is too restrictive for me to use. The Cython module ``ha1utils.pyx'' can be compiled by running ``setup.py.'' If needed I can demonstrate the code.

\section{Task 1}\label{sec:1}
The relevant values can be seen in Table~\ref{tab:task1} and the plots in Figure~\ref{fig:task1_ut}, Figure~\ref{fig:task1_st}, Figure~\ref{fig:task1_sr}.
The pulse width was calculated using (\ref{eq:pulse_width}).

The bandwidth was calculated using \textit{findbw} in ``ha1utils.pyx'' which locates peak values, separates them into bands, finds the center points in the band and then calculates the bandwidth given a lower threshold.

The Nyquist rate was calculated using $f_\text{Nyquist}=2f_\text{max}$. where $f_\text{max}$ is the highest frequency inside the band ($\approx \SI{35}{\mega\hertz}$).

The apparent carrier frequency is lower than the actual carrier frequency because of the FFT-shift.
\begin{equation}
  \label{eq:pulse_width}
  |u(0)|=\frac{1}{|\sqrt{t_p}|}
\end{equation}
\begin{table}[h]
  \centering
  \begin{tabular}{|l|l|}\hline
    $t_p$ & $\SI{6.7}{\micro\second}$ \\
    $B$ & $\SI{73.2}{\mega\hertz}$ \\
    $B$ & $\SI{73.2}{\mega\hertz}$ \\
    $t_pB$ & 487.7 \\
    $f_\text{Nyquist}$ & $\SI{73.1}{\mega\hertz}$ \\
    $f_s$ & $\SI{195}{\mega\hertz}$ \\
    $f_{ca}$ & $\SI{48.7}{\mega\hertz}$ \\ \hline
  \end{tabular}
  \caption{The requested values for task 1}
  \label{tab:task1}
\end{table}
\begin{figure}[h]
  \centering
  \noindent\makebox[\textwidth]{\scalebox{0.90}{%% Creator: Matplotlib, PGF backend
%%
%% To include the figure in your LaTeX document, write
%%   \input{<filename>.pgf}
%%
%% Make sure the required packages are loaded in your preamble
%%   \usepackage{pgf}
%%
%% Figures using additional raster images can only be included by \input if
%% they are in the same directory as the main LaTeX file. For loading figures
%% from other directories you can use the `import` package
%%   \usepackage{import}
%% and then include the figures with
%%   \import{<path to file>}{<filename>.pgf}
%%
%% Matplotlib used the following preamble
%%   \usepackage{fontspec}
%%   \setmainfont{DejaVu Serif}
%%   \setsansfont{DejaVu Sans}
%%   \setmonofont{DejaVu Sans Mono}
%%
\begingroup%
\makeatletter%
\begin{pgfpicture}%
\pgfpathrectangle{\pgfpointorigin}{\pgfqpoint{6.400000in}{4.800000in}}%
\pgfusepath{use as bounding box, clip}%
\begin{pgfscope}%
\pgfsetbuttcap%
\pgfsetmiterjoin%
\definecolor{currentfill}{rgb}{1.000000,1.000000,1.000000}%
\pgfsetfillcolor{currentfill}%
\pgfsetlinewidth{0.000000pt}%
\definecolor{currentstroke}{rgb}{1.000000,1.000000,1.000000}%
\pgfsetstrokecolor{currentstroke}%
\pgfsetdash{}{0pt}%
\pgfpathmoveto{\pgfqpoint{0.000000in}{0.000000in}}%
\pgfpathlineto{\pgfqpoint{6.400000in}{0.000000in}}%
\pgfpathlineto{\pgfqpoint{6.400000in}{4.800000in}}%
\pgfpathlineto{\pgfqpoint{0.000000in}{4.800000in}}%
\pgfpathclose%
\pgfusepath{fill}%
\end{pgfscope}%
\begin{pgfscope}%
\pgfsetbuttcap%
\pgfsetmiterjoin%
\definecolor{currentfill}{rgb}{1.000000,1.000000,1.000000}%
\pgfsetfillcolor{currentfill}%
\pgfsetlinewidth{0.000000pt}%
\definecolor{currentstroke}{rgb}{0.000000,0.000000,0.000000}%
\pgfsetstrokecolor{currentstroke}%
\pgfsetstrokeopacity{0.000000}%
\pgfsetdash{}{0pt}%
\pgfpathmoveto{\pgfqpoint{0.800000in}{2.544000in}}%
\pgfpathlineto{\pgfqpoint{5.760000in}{2.544000in}}%
\pgfpathlineto{\pgfqpoint{5.760000in}{4.224000in}}%
\pgfpathlineto{\pgfqpoint{0.800000in}{4.224000in}}%
\pgfpathclose%
\pgfusepath{fill}%
\end{pgfscope}%
\begin{pgfscope}%
\pgfsetbuttcap%
\pgfsetroundjoin%
\definecolor{currentfill}{rgb}{0.000000,0.000000,0.000000}%
\pgfsetfillcolor{currentfill}%
\pgfsetlinewidth{0.803000pt}%
\definecolor{currentstroke}{rgb}{0.000000,0.000000,0.000000}%
\pgfsetstrokecolor{currentstroke}%
\pgfsetdash{}{0pt}%
\pgfsys@defobject{currentmarker}{\pgfqpoint{0.000000in}{-0.048611in}}{\pgfqpoint{0.000000in}{0.000000in}}{%
\pgfpathmoveto{\pgfqpoint{0.000000in}{0.000000in}}%
\pgfpathlineto{\pgfqpoint{0.000000in}{-0.048611in}}%
\pgfusepath{stroke,fill}%
}%
\begin{pgfscope}%
\pgfsys@transformshift{1.476364in}{2.544000in}%
\pgfsys@useobject{currentmarker}{}%
\end{pgfscope}%
\end{pgfscope}%
\begin{pgfscope}%
\pgftext[x=1.476364in,y=2.446778in,,top]{\sffamily\fontsize{10.000000}{12.000000}\selectfont -4}%
\end{pgfscope}%
\begin{pgfscope}%
\pgfsetbuttcap%
\pgfsetroundjoin%
\definecolor{currentfill}{rgb}{0.000000,0.000000,0.000000}%
\pgfsetfillcolor{currentfill}%
\pgfsetlinewidth{0.803000pt}%
\definecolor{currentstroke}{rgb}{0.000000,0.000000,0.000000}%
\pgfsetstrokecolor{currentstroke}%
\pgfsetdash{}{0pt}%
\pgfsys@defobject{currentmarker}{\pgfqpoint{0.000000in}{-0.048611in}}{\pgfqpoint{0.000000in}{0.000000in}}{%
\pgfpathmoveto{\pgfqpoint{0.000000in}{0.000000in}}%
\pgfpathlineto{\pgfqpoint{0.000000in}{-0.048611in}}%
\pgfusepath{stroke,fill}%
}%
\begin{pgfscope}%
\pgfsys@transformshift{2.378182in}{2.544000in}%
\pgfsys@useobject{currentmarker}{}%
\end{pgfscope}%
\end{pgfscope}%
\begin{pgfscope}%
\pgftext[x=2.378182in,y=2.446778in,,top]{\sffamily\fontsize{10.000000}{12.000000}\selectfont -2}%
\end{pgfscope}%
\begin{pgfscope}%
\pgfsetbuttcap%
\pgfsetroundjoin%
\definecolor{currentfill}{rgb}{0.000000,0.000000,0.000000}%
\pgfsetfillcolor{currentfill}%
\pgfsetlinewidth{0.803000pt}%
\definecolor{currentstroke}{rgb}{0.000000,0.000000,0.000000}%
\pgfsetstrokecolor{currentstroke}%
\pgfsetdash{}{0pt}%
\pgfsys@defobject{currentmarker}{\pgfqpoint{0.000000in}{-0.048611in}}{\pgfqpoint{0.000000in}{0.000000in}}{%
\pgfpathmoveto{\pgfqpoint{0.000000in}{0.000000in}}%
\pgfpathlineto{\pgfqpoint{0.000000in}{-0.048611in}}%
\pgfusepath{stroke,fill}%
}%
\begin{pgfscope}%
\pgfsys@transformshift{3.280000in}{2.544000in}%
\pgfsys@useobject{currentmarker}{}%
\end{pgfscope}%
\end{pgfscope}%
\begin{pgfscope}%
\pgftext[x=3.280000in,y=2.446778in,,top]{\sffamily\fontsize{10.000000}{12.000000}\selectfont 0}%
\end{pgfscope}%
\begin{pgfscope}%
\pgfsetbuttcap%
\pgfsetroundjoin%
\definecolor{currentfill}{rgb}{0.000000,0.000000,0.000000}%
\pgfsetfillcolor{currentfill}%
\pgfsetlinewidth{0.803000pt}%
\definecolor{currentstroke}{rgb}{0.000000,0.000000,0.000000}%
\pgfsetstrokecolor{currentstroke}%
\pgfsetdash{}{0pt}%
\pgfsys@defobject{currentmarker}{\pgfqpoint{0.000000in}{-0.048611in}}{\pgfqpoint{0.000000in}{0.000000in}}{%
\pgfpathmoveto{\pgfqpoint{0.000000in}{0.000000in}}%
\pgfpathlineto{\pgfqpoint{0.000000in}{-0.048611in}}%
\pgfusepath{stroke,fill}%
}%
\begin{pgfscope}%
\pgfsys@transformshift{4.181818in}{2.544000in}%
\pgfsys@useobject{currentmarker}{}%
\end{pgfscope}%
\end{pgfscope}%
\begin{pgfscope}%
\pgftext[x=4.181818in,y=2.446778in,,top]{\sffamily\fontsize{10.000000}{12.000000}\selectfont 2}%
\end{pgfscope}%
\begin{pgfscope}%
\pgfsetbuttcap%
\pgfsetroundjoin%
\definecolor{currentfill}{rgb}{0.000000,0.000000,0.000000}%
\pgfsetfillcolor{currentfill}%
\pgfsetlinewidth{0.803000pt}%
\definecolor{currentstroke}{rgb}{0.000000,0.000000,0.000000}%
\pgfsetstrokecolor{currentstroke}%
\pgfsetdash{}{0pt}%
\pgfsys@defobject{currentmarker}{\pgfqpoint{0.000000in}{-0.048611in}}{\pgfqpoint{0.000000in}{0.000000in}}{%
\pgfpathmoveto{\pgfqpoint{0.000000in}{0.000000in}}%
\pgfpathlineto{\pgfqpoint{0.000000in}{-0.048611in}}%
\pgfusepath{stroke,fill}%
}%
\begin{pgfscope}%
\pgfsys@transformshift{5.083636in}{2.544000in}%
\pgfsys@useobject{currentmarker}{}%
\end{pgfscope}%
\end{pgfscope}%
\begin{pgfscope}%
\pgftext[x=5.083636in,y=2.446778in,,top]{\sffamily\fontsize{10.000000}{12.000000}\selectfont 4}%
\end{pgfscope}%
\begin{pgfscope}%
\pgftext[x=3.280000in,y=2.256809in,,top]{\sffamily\fontsize{10.000000}{12.000000}\selectfont Time [\(\displaystyle \mu\)s]}%
\end{pgfscope}%
\begin{pgfscope}%
\pgfsetbuttcap%
\pgfsetroundjoin%
\definecolor{currentfill}{rgb}{0.000000,0.000000,0.000000}%
\pgfsetfillcolor{currentfill}%
\pgfsetlinewidth{0.803000pt}%
\definecolor{currentstroke}{rgb}{0.000000,0.000000,0.000000}%
\pgfsetstrokecolor{currentstroke}%
\pgfsetdash{}{0pt}%
\pgfsys@defobject{currentmarker}{\pgfqpoint{-0.048611in}{0.000000in}}{\pgfqpoint{0.000000in}{0.000000in}}{%
\pgfpathmoveto{\pgfqpoint{0.000000in}{0.000000in}}%
\pgfpathlineto{\pgfqpoint{-0.048611in}{0.000000in}}%
\pgfusepath{stroke,fill}%
}%
\begin{pgfscope}%
\pgfsys@transformshift{0.800000in}{2.595320in}%
\pgfsys@useobject{currentmarker}{}%
\end{pgfscope}%
\end{pgfscope}%
\begin{pgfscope}%
\pgftext[x=0.321308in,y=2.542558in,left,base]{\sffamily\fontsize{10.000000}{12.000000}\selectfont -400}%
\end{pgfscope}%
\begin{pgfscope}%
\pgfsetbuttcap%
\pgfsetroundjoin%
\definecolor{currentfill}{rgb}{0.000000,0.000000,0.000000}%
\pgfsetfillcolor{currentfill}%
\pgfsetlinewidth{0.803000pt}%
\definecolor{currentstroke}{rgb}{0.000000,0.000000,0.000000}%
\pgfsetstrokecolor{currentstroke}%
\pgfsetdash{}{0pt}%
\pgfsys@defobject{currentmarker}{\pgfqpoint{-0.048611in}{0.000000in}}{\pgfqpoint{0.000000in}{0.000000in}}{%
\pgfpathmoveto{\pgfqpoint{0.000000in}{0.000000in}}%
\pgfpathlineto{\pgfqpoint{-0.048611in}{0.000000in}}%
\pgfusepath{stroke,fill}%
}%
\begin{pgfscope}%
\pgfsys@transformshift{0.800000in}{2.989660in}%
\pgfsys@useobject{currentmarker}{}%
\end{pgfscope}%
\end{pgfscope}%
\begin{pgfscope}%
\pgftext[x=0.321308in,y=2.936898in,left,base]{\sffamily\fontsize{10.000000}{12.000000}\selectfont -200}%
\end{pgfscope}%
\begin{pgfscope}%
\pgfsetbuttcap%
\pgfsetroundjoin%
\definecolor{currentfill}{rgb}{0.000000,0.000000,0.000000}%
\pgfsetfillcolor{currentfill}%
\pgfsetlinewidth{0.803000pt}%
\definecolor{currentstroke}{rgb}{0.000000,0.000000,0.000000}%
\pgfsetstrokecolor{currentstroke}%
\pgfsetdash{}{0pt}%
\pgfsys@defobject{currentmarker}{\pgfqpoint{-0.048611in}{0.000000in}}{\pgfqpoint{0.000000in}{0.000000in}}{%
\pgfpathmoveto{\pgfqpoint{0.000000in}{0.000000in}}%
\pgfpathlineto{\pgfqpoint{-0.048611in}{0.000000in}}%
\pgfusepath{stroke,fill}%
}%
\begin{pgfscope}%
\pgfsys@transformshift{0.800000in}{3.384000in}%
\pgfsys@useobject{currentmarker}{}%
\end{pgfscope}%
\end{pgfscope}%
\begin{pgfscope}%
\pgftext[x=0.614413in,y=3.331238in,left,base]{\sffamily\fontsize{10.000000}{12.000000}\selectfont 0}%
\end{pgfscope}%
\begin{pgfscope}%
\pgfsetbuttcap%
\pgfsetroundjoin%
\definecolor{currentfill}{rgb}{0.000000,0.000000,0.000000}%
\pgfsetfillcolor{currentfill}%
\pgfsetlinewidth{0.803000pt}%
\definecolor{currentstroke}{rgb}{0.000000,0.000000,0.000000}%
\pgfsetstrokecolor{currentstroke}%
\pgfsetdash{}{0pt}%
\pgfsys@defobject{currentmarker}{\pgfqpoint{-0.048611in}{0.000000in}}{\pgfqpoint{0.000000in}{0.000000in}}{%
\pgfpathmoveto{\pgfqpoint{0.000000in}{0.000000in}}%
\pgfpathlineto{\pgfqpoint{-0.048611in}{0.000000in}}%
\pgfusepath{stroke,fill}%
}%
\begin{pgfscope}%
\pgfsys@transformshift{0.800000in}{3.778340in}%
\pgfsys@useobject{currentmarker}{}%
\end{pgfscope}%
\end{pgfscope}%
\begin{pgfscope}%
\pgftext[x=0.437682in,y=3.725579in,left,base]{\sffamily\fontsize{10.000000}{12.000000}\selectfont 200}%
\end{pgfscope}%
\begin{pgfscope}%
\pgfsetbuttcap%
\pgfsetroundjoin%
\definecolor{currentfill}{rgb}{0.000000,0.000000,0.000000}%
\pgfsetfillcolor{currentfill}%
\pgfsetlinewidth{0.803000pt}%
\definecolor{currentstroke}{rgb}{0.000000,0.000000,0.000000}%
\pgfsetstrokecolor{currentstroke}%
\pgfsetdash{}{0pt}%
\pgfsys@defobject{currentmarker}{\pgfqpoint{-0.048611in}{0.000000in}}{\pgfqpoint{0.000000in}{0.000000in}}{%
\pgfpathmoveto{\pgfqpoint{0.000000in}{0.000000in}}%
\pgfpathlineto{\pgfqpoint{-0.048611in}{0.000000in}}%
\pgfusepath{stroke,fill}%
}%
\begin{pgfscope}%
\pgfsys@transformshift{0.800000in}{4.172680in}%
\pgfsys@useobject{currentmarker}{}%
\end{pgfscope}%
\end{pgfscope}%
\begin{pgfscope}%
\pgftext[x=0.437682in,y=4.119919in,left,base]{\sffamily\fontsize{10.000000}{12.000000}\selectfont 400}%
\end{pgfscope}%
\begin{pgfscope}%
\pgftext[x=0.265752in,y=3.384000in,,bottom,rotate=90.000000]{\sffamily\fontsize{10.000000}{12.000000}\selectfont Real part of the signal}%
\end{pgfscope}%
\begin{pgfscope}%
\pgfpathrectangle{\pgfqpoint{0.800000in}{2.544000in}}{\pgfqpoint{4.960000in}{1.680000in}} %
\pgfusepath{clip}%
\pgfsetrectcap%
\pgfsetroundjoin%
\pgfsetlinewidth{1.505625pt}%
\definecolor{currentstroke}{rgb}{0.121569,0.466667,0.705882}%
\pgfsetstrokecolor{currentstroke}%
\pgfsetdash{}{0pt}%
\pgfpathmoveto{\pgfqpoint{1.025455in}{3.384000in}}%
\pgfpathlineto{\pgfqpoint{1.774657in}{3.384000in}}%
\pgfpathlineto{\pgfqpoint{1.776970in}{2.620364in}}%
\pgfpathlineto{\pgfqpoint{1.779282in}{3.112547in}}%
\pgfpathlineto{\pgfqpoint{1.781594in}{3.953703in}}%
\pgfpathlineto{\pgfqpoint{1.783907in}{4.063111in}}%
\pgfpathlineto{\pgfqpoint{1.788531in}{2.647000in}}%
\pgfpathlineto{\pgfqpoint{1.790844in}{2.929423in}}%
\pgfpathlineto{\pgfqpoint{1.793156in}{3.788732in}}%
\pgfpathlineto{\pgfqpoint{1.795469in}{4.135000in}}%
\pgfpathlineto{\pgfqpoint{1.800093in}{2.743690in}}%
\pgfpathlineto{\pgfqpoint{1.802406in}{2.759190in}}%
\pgfpathlineto{\pgfqpoint{1.807030in}{4.138235in}}%
\pgfpathlineto{\pgfqpoint{1.809343in}{3.779665in}}%
\pgfpathlineto{\pgfqpoint{1.811655in}{2.929994in}}%
\pgfpathlineto{\pgfqpoint{1.813967in}{2.643238in}}%
\pgfpathlineto{\pgfqpoint{1.818592in}{4.037873in}}%
\pgfpathlineto{\pgfqpoint{1.820904in}{4.000541in}}%
\pgfpathlineto{\pgfqpoint{1.825529in}{2.631145in}}%
\pgfpathlineto{\pgfqpoint{1.827841in}{2.971473in}}%
\pgfpathlineto{\pgfqpoint{1.830154in}{3.812523in}}%
\pgfpathlineto{\pgfqpoint{1.832466in}{4.133950in}}%
\pgfpathlineto{\pgfqpoint{1.837091in}{2.766205in}}%
\pgfpathlineto{\pgfqpoint{1.839403in}{2.725409in}}%
\pgfpathlineto{\pgfqpoint{1.844028in}{4.113899in}}%
\pgfpathlineto{\pgfqpoint{1.846340in}{3.885050in}}%
\pgfpathlineto{\pgfqpoint{1.850965in}{2.620448in}}%
\pgfpathlineto{\pgfqpoint{1.853277in}{3.080984in}}%
\pgfpathlineto{\pgfqpoint{1.855590in}{3.897779in}}%
\pgfpathlineto{\pgfqpoint{1.857902in}{4.110917in}}%
\pgfpathlineto{\pgfqpoint{1.862527in}{2.731968in}}%
\pgfpathlineto{\pgfqpoint{1.864839in}{2.749948in}}%
\pgfpathlineto{\pgfqpoint{1.869464in}{4.118141in}}%
\pgfpathlineto{\pgfqpoint{1.871776in}{3.885585in}}%
\pgfpathlineto{\pgfqpoint{1.876401in}{2.620641in}}%
\pgfpathlineto{\pgfqpoint{1.878713in}{3.037949in}}%
\pgfpathlineto{\pgfqpoint{1.881026in}{3.849341in}}%
\pgfpathlineto{\pgfqpoint{1.883338in}{4.131145in}}%
\pgfpathlineto{\pgfqpoint{1.890275in}{2.688810in}}%
\pgfpathlineto{\pgfqpoint{1.894900in}{4.059836in}}%
\pgfpathlineto{\pgfqpoint{1.897212in}{4.001795in}}%
\pgfpathlineto{\pgfqpoint{1.901837in}{2.653066in}}%
\pgfpathlineto{\pgfqpoint{1.904149in}{2.859292in}}%
\pgfpathlineto{\pgfqpoint{1.908774in}{4.142154in}}%
\pgfpathlineto{\pgfqpoint{1.911086in}{3.808401in}}%
\pgfpathlineto{\pgfqpoint{1.913399in}{3.011026in}}%
\pgfpathlineto{\pgfqpoint{1.915711in}{2.620891in}}%
\pgfpathlineto{\pgfqpoint{1.918023in}{3.060273in}}%
\pgfpathlineto{\pgfqpoint{1.920336in}{3.849904in}}%
\pgfpathlineto{\pgfqpoint{1.922648in}{4.135634in}}%
\pgfpathlineto{\pgfqpoint{1.929585in}{2.654730in}}%
\pgfpathlineto{\pgfqpoint{1.934210in}{3.981161in}}%
\pgfpathlineto{\pgfqpoint{1.936522in}{4.084944in}}%
\pgfpathlineto{\pgfqpoint{1.941147in}{2.744077in}}%
\pgfpathlineto{\pgfqpoint{1.943459in}{2.713186in}}%
\pgfpathlineto{\pgfqpoint{1.948084in}{4.054814in}}%
\pgfpathlineto{\pgfqpoint{1.950396in}{4.026237in}}%
\pgfpathlineto{\pgfqpoint{1.955021in}{2.691777in}}%
\pgfpathlineto{\pgfqpoint{1.957333in}{2.766205in}}%
\pgfpathlineto{\pgfqpoint{1.961958in}{4.090185in}}%
\pgfpathlineto{\pgfqpoint{1.964270in}{3.983367in}}%
\pgfpathlineto{\pgfqpoint{1.968895in}{2.669736in}}%
\pgfpathlineto{\pgfqpoint{1.971207in}{2.795789in}}%
\pgfpathlineto{\pgfqpoint{1.975832in}{4.101476in}}%
\pgfpathlineto{\pgfqpoint{1.978145in}{3.969030in}}%
\pgfpathlineto{\pgfqpoint{1.982769in}{2.667747in}}%
\pgfpathlineto{\pgfqpoint{1.985082in}{2.793983in}}%
\pgfpathlineto{\pgfqpoint{1.989706in}{4.094428in}}%
\pgfpathlineto{\pgfqpoint{1.992019in}{3.986869in}}%
\pgfpathlineto{\pgfqpoint{1.996643in}{2.684757in}}%
\pgfpathlineto{\pgfqpoint{1.998956in}{2.761237in}}%
\pgfpathlineto{\pgfqpoint{2.003580in}{4.065369in}}%
\pgfpathlineto{\pgfqpoint{2.005893in}{4.032310in}}%
\pgfpathlineto{\pgfqpoint{2.010517in}{2.729030in}}%
\pgfpathlineto{\pgfqpoint{2.012830in}{2.706519in}}%
\pgfpathlineto{\pgfqpoint{2.017455in}{4.001795in}}%
\pgfpathlineto{\pgfqpoint{2.019767in}{4.091528in}}%
\pgfpathlineto{\pgfqpoint{2.024392in}{2.816666in}}%
\pgfpathlineto{\pgfqpoint{2.026704in}{2.649082in}}%
\pgfpathlineto{\pgfqpoint{2.033641in}{4.139312in}}%
\pgfpathlineto{\pgfqpoint{2.035953in}{3.696759in}}%
\pgfpathlineto{\pgfqpoint{2.038266in}{2.967300in}}%
\pgfpathlineto{\pgfqpoint{2.040578in}{2.620369in}}%
\pgfpathlineto{\pgfqpoint{2.042890in}{2.970876in}}%
\pgfpathlineto{\pgfqpoint{2.047515in}{4.138235in}}%
\pgfpathlineto{\pgfqpoint{2.049828in}{3.897779in}}%
\pgfpathlineto{\pgfqpoint{2.054452in}{2.662715in}}%
\pgfpathlineto{\pgfqpoint{2.056765in}{2.775941in}}%
\pgfpathlineto{\pgfqpoint{2.061389in}{4.043308in}}%
\pgfpathlineto{\pgfqpoint{2.063702in}{4.071660in}}%
\pgfpathlineto{\pgfqpoint{2.068326in}{2.820004in}}%
\pgfpathlineto{\pgfqpoint{2.070639in}{2.641051in}}%
\pgfpathlineto{\pgfqpoint{2.072951in}{3.110558in}}%
\pgfpathlineto{\pgfqpoint{2.075263in}{3.817211in}}%
\pgfpathlineto{\pgfqpoint{2.077576in}{4.147636in}}%
\pgfpathlineto{\pgfqpoint{2.079888in}{3.818379in}}%
\pgfpathlineto{\pgfqpoint{2.084513in}{2.644108in}}%
\pgfpathlineto{\pgfqpoint{2.086825in}{2.804943in}}%
\pgfpathlineto{\pgfqpoint{2.091450in}{4.047916in}}%
\pgfpathlineto{\pgfqpoint{2.093762in}{4.075622in}}%
\pgfpathlineto{\pgfqpoint{2.100699in}{2.628792in}}%
\pgfpathlineto{\pgfqpoint{2.103012in}{3.034789in}}%
\pgfpathlineto{\pgfqpoint{2.107636in}{4.138235in}}%
\pgfpathlineto{\pgfqpoint{2.109949in}{3.924474in}}%
\pgfpathlineto{\pgfqpoint{2.114573in}{2.706519in}}%
\pgfpathlineto{\pgfqpoint{2.116886in}{2.698515in}}%
\pgfpathlineto{\pgfqpoint{2.123823in}{4.142891in}}%
\pgfpathlineto{\pgfqpoint{2.126135in}{3.770521in}}%
\pgfpathlineto{\pgfqpoint{2.130760in}{2.644108in}}%
\pgfpathlineto{\pgfqpoint{2.133072in}{2.786839in}}%
\pgfpathlineto{\pgfqpoint{2.137697in}{4.001795in}}%
\pgfpathlineto{\pgfqpoint{2.140009in}{4.115954in}}%
\pgfpathlineto{\pgfqpoint{2.146946in}{2.625109in}}%
\pgfpathlineto{\pgfqpoint{2.149259in}{2.851099in}}%
\pgfpathlineto{\pgfqpoint{2.153883in}{4.043308in}}%
\pgfpathlineto{\pgfqpoint{2.156196in}{4.094428in}}%
\pgfpathlineto{\pgfqpoint{2.163133in}{2.621934in}}%
\pgfpathlineto{\pgfqpoint{2.165445in}{2.870221in}}%
\pgfpathlineto{\pgfqpoint{2.170070in}{4.046508in}}%
\pgfpathlineto{\pgfqpoint{2.172382in}{4.095722in}}%
\pgfpathlineto{\pgfqpoint{2.179319in}{2.624798in}}%
\pgfpathlineto{\pgfqpoint{2.181632in}{2.839032in}}%
\pgfpathlineto{\pgfqpoint{2.186256in}{4.012863in}}%
\pgfpathlineto{\pgfqpoint{2.188569in}{4.118918in}}%
\pgfpathlineto{\pgfqpoint{2.190881in}{3.694815in}}%
\pgfpathlineto{\pgfqpoint{2.195506in}{2.642724in}}%
\pgfpathlineto{\pgfqpoint{2.197818in}{2.766205in}}%
\pgfpathlineto{\pgfqpoint{2.202443in}{3.928470in}}%
\pgfpathlineto{\pgfqpoint{2.204755in}{4.144661in}}%
\pgfpathlineto{\pgfqpoint{2.207068in}{3.827093in}}%
\pgfpathlineto{\pgfqpoint{2.211692in}{2.702631in}}%
\pgfpathlineto{\pgfqpoint{2.214005in}{2.677545in}}%
\pgfpathlineto{\pgfqpoint{2.220942in}{4.133950in}}%
\pgfpathlineto{\pgfqpoint{2.223254in}{3.986869in}}%
\pgfpathlineto{\pgfqpoint{2.227879in}{2.844028in}}%
\pgfpathlineto{\pgfqpoint{2.230191in}{2.621714in}}%
\pgfpathlineto{\pgfqpoint{2.232503in}{2.910819in}}%
\pgfpathlineto{\pgfqpoint{2.237128in}{4.031559in}}%
\pgfpathlineto{\pgfqpoint{2.239441in}{4.118141in}}%
\pgfpathlineto{\pgfqpoint{2.241753in}{3.720527in}}%
\pgfpathlineto{\pgfqpoint{2.246378in}{2.669736in}}%
\pgfpathlineto{\pgfqpoint{2.248690in}{2.702631in}}%
\pgfpathlineto{\pgfqpoint{2.255627in}{4.133273in}}%
\pgfpathlineto{\pgfqpoint{2.257939in}{4.001795in}}%
\pgfpathlineto{\pgfqpoint{2.264876in}{2.621591in}}%
\pgfpathlineto{\pgfqpoint{2.267189in}{2.829671in}}%
\pgfpathlineto{\pgfqpoint{2.271814in}{3.938329in}}%
\pgfpathlineto{\pgfqpoint{2.274126in}{4.146639in}}%
\pgfpathlineto{\pgfqpoint{2.276438in}{3.882904in}}%
\pgfpathlineto{\pgfqpoint{2.281063in}{2.786839in}}%
\pgfpathlineto{\pgfqpoint{2.283375in}{2.626728in}}%
\pgfpathlineto{\pgfqpoint{2.285688in}{2.927145in}}%
\pgfpathlineto{\pgfqpoint{2.290312in}{4.006763in}}%
\pgfpathlineto{\pgfqpoint{2.292625in}{4.135634in}}%
\pgfpathlineto{\pgfqpoint{2.294937in}{3.815455in}}%
\pgfpathlineto{\pgfqpoint{2.299562in}{2.749948in}}%
\pgfpathlineto{\pgfqpoint{2.301874in}{2.635002in}}%
\pgfpathlineto{\pgfqpoint{2.304186in}{2.959599in}}%
\pgfpathlineto{\pgfqpoint{2.308811in}{4.016465in}}%
\pgfpathlineto{\pgfqpoint{2.311124in}{4.134611in}}%
\pgfpathlineto{\pgfqpoint{2.313436in}{3.820128in}}%
\pgfpathlineto{\pgfqpoint{2.318061in}{2.766205in}}%
\pgfpathlineto{\pgfqpoint{2.320373in}{2.628275in}}%
\pgfpathlineto{\pgfqpoint{2.322685in}{2.918096in}}%
\pgfpathlineto{\pgfqpoint{2.327310in}{3.972211in}}%
\pgfpathlineto{\pgfqpoint{2.329622in}{4.145579in}}%
\pgfpathlineto{\pgfqpoint{2.331935in}{3.895675in}}%
\pgfpathlineto{\pgfqpoint{2.336559in}{2.843526in}}%
\pgfpathlineto{\pgfqpoint{2.338872in}{2.620641in}}%
\pgfpathlineto{\pgfqpoint{2.341184in}{2.814297in}}%
\pgfpathlineto{\pgfqpoint{2.348121in}{4.138235in}}%
\pgfpathlineto{\pgfqpoint{2.350434in}{4.018052in}}%
\pgfpathlineto{\pgfqpoint{2.357371in}{2.658399in}}%
\pgfpathlineto{\pgfqpoint{2.359683in}{2.687932in}}%
\pgfpathlineto{\pgfqpoint{2.368932in}{4.128082in}}%
\pgfpathlineto{\pgfqpoint{2.371245in}{3.823037in}}%
\pgfpathlineto{\pgfqpoint{2.375869in}{2.812881in}}%
\pgfpathlineto{\pgfqpoint{2.378182in}{2.620364in}}%
\pgfpathlineto{\pgfqpoint{2.380494in}{2.811940in}}%
\pgfpathlineto{\pgfqpoint{2.387431in}{4.122647in}}%
\pgfpathlineto{\pgfqpoint{2.389744in}{4.068229in}}%
\pgfpathlineto{\pgfqpoint{2.398993in}{2.630908in}}%
\pgfpathlineto{\pgfqpoint{2.401305in}{2.893748in}}%
\pgfpathlineto{\pgfqpoint{2.405930in}{3.882904in}}%
\pgfpathlineto{\pgfqpoint{2.408242in}{4.138235in}}%
\pgfpathlineto{\pgfqpoint{2.410555in}{4.037873in}}%
\pgfpathlineto{\pgfqpoint{2.417492in}{2.717644in}}%
\pgfpathlineto{\pgfqpoint{2.419804in}{2.633000in}}%
\pgfpathlineto{\pgfqpoint{2.422117in}{2.893204in}}%
\pgfpathlineto{\pgfqpoint{2.429054in}{4.131145in}}%
\pgfpathlineto{\pgfqpoint{2.431366in}{4.063111in}}%
\pgfpathlineto{\pgfqpoint{2.435991in}{3.184072in}}%
\pgfpathlineto{\pgfqpoint{2.438303in}{2.766205in}}%
\pgfpathlineto{\pgfqpoint{2.440615in}{2.621221in}}%
\pgfpathlineto{\pgfqpoint{2.442928in}{2.810531in}}%
\pgfpathlineto{\pgfqpoint{2.449865in}{4.084379in}}%
\pgfpathlineto{\pgfqpoint{2.452177in}{4.122647in}}%
\pgfpathlineto{\pgfqpoint{2.454490in}{3.845958in}}%
\pgfpathlineto{\pgfqpoint{2.461427in}{2.641879in}}%
\pgfpathlineto{\pgfqpoint{2.463739in}{2.686480in}}%
\pgfpathlineto{\pgfqpoint{2.468364in}{3.503459in}}%
\pgfpathlineto{\pgfqpoint{2.470676in}{3.936373in}}%
\pgfpathlineto{\pgfqpoint{2.472988in}{4.142812in}}%
\pgfpathlineto{\pgfqpoint{2.475301in}{4.040788in}}%
\pgfpathlineto{\pgfqpoint{2.484550in}{2.620791in}}%
\pgfpathlineto{\pgfqpoint{2.486862in}{2.755943in}}%
\pgfpathlineto{\pgfqpoint{2.496112in}{4.147636in}}%
\pgfpathlineto{\pgfqpoint{2.498424in}{4.001795in}}%
\pgfpathlineto{\pgfqpoint{2.507674in}{2.620448in}}%
\pgfpathlineto{\pgfqpoint{2.509986in}{2.755943in}}%
\pgfpathlineto{\pgfqpoint{2.519235in}{4.145473in}}%
\pgfpathlineto{\pgfqpoint{2.521548in}{4.040788in}}%
\pgfpathlineto{\pgfqpoint{2.526172in}{3.247049in}}%
\pgfpathlineto{\pgfqpoint{2.530797in}{2.633000in}}%
\pgfpathlineto{\pgfqpoint{2.533110in}{2.686480in}}%
\pgfpathlineto{\pgfqpoint{2.537734in}{3.418767in}}%
\pgfpathlineto{\pgfqpoint{2.542359in}{4.105051in}}%
\pgfpathlineto{\pgfqpoint{2.544671in}{4.122647in}}%
\pgfpathlineto{\pgfqpoint{2.546984in}{3.889853in}}%
\pgfpathlineto{\pgfqpoint{2.553921in}{2.727212in}}%
\pgfpathlineto{\pgfqpoint{2.556233in}{2.621221in}}%
\pgfpathlineto{\pgfqpoint{2.558545in}{2.766205in}}%
\pgfpathlineto{\pgfqpoint{2.567795in}{4.131145in}}%
\pgfpathlineto{\pgfqpoint{2.570107in}{4.097760in}}%
\pgfpathlineto{\pgfqpoint{2.572420in}{3.838577in}}%
\pgfpathlineto{\pgfqpoint{2.579357in}{2.717644in}}%
\pgfpathlineto{\pgfqpoint{2.581669in}{2.621398in}}%
\pgfpathlineto{\pgfqpoint{2.583981in}{2.759190in}}%
\pgfpathlineto{\pgfqpoint{2.593231in}{4.112210in}}%
\pgfpathlineto{\pgfqpoint{2.595543in}{4.124762in}}%
\pgfpathlineto{\pgfqpoint{2.597855in}{3.918930in}}%
\pgfpathlineto{\pgfqpoint{2.607105in}{2.631145in}}%
\pgfpathlineto{\pgfqpoint{2.609417in}{2.672535in}}%
\pgfpathlineto{\pgfqpoint{2.611730in}{2.915289in}}%
\pgfpathlineto{\pgfqpoint{2.618667in}{4.001795in}}%
\pgfpathlineto{\pgfqpoint{2.620979in}{4.144003in}}%
\pgfpathlineto{\pgfqpoint{2.623291in}{4.077714in}}%
\pgfpathlineto{\pgfqpoint{2.627916in}{3.450634in}}%
\pgfpathlineto{\pgfqpoint{2.632541in}{2.759599in}}%
\pgfpathlineto{\pgfqpoint{2.634853in}{2.623726in}}%
\pgfpathlineto{\pgfqpoint{2.637166in}{2.687932in}}%
\pgfpathlineto{\pgfqpoint{2.641790in}{3.294773in}}%
\pgfpathlineto{\pgfqpoint{2.646415in}{3.984684in}}%
\pgfpathlineto{\pgfqpoint{2.648727in}{4.138235in}}%
\pgfpathlineto{\pgfqpoint{2.651040in}{4.101476in}}%
\pgfpathlineto{\pgfqpoint{2.653352in}{3.885585in}}%
\pgfpathlineto{\pgfqpoint{2.662601in}{2.651643in}}%
\pgfpathlineto{\pgfqpoint{2.664914in}{2.636855in}}%
\pgfpathlineto{\pgfqpoint{2.667226in}{2.800799in}}%
\pgfpathlineto{\pgfqpoint{2.676476in}{4.059836in}}%
\pgfpathlineto{\pgfqpoint{2.678788in}{4.147636in}}%
\pgfpathlineto{\pgfqpoint{2.681100in}{4.060496in}}%
\pgfpathlineto{\pgfqpoint{2.685725in}{3.482384in}}%
\pgfpathlineto{\pgfqpoint{2.690350in}{2.824330in}}%
\pgfpathlineto{\pgfqpoint{2.692662in}{2.649082in}}%
\pgfpathlineto{\pgfqpoint{2.694974in}{2.635002in}}%
\pgfpathlineto{\pgfqpoint{2.697287in}{2.783316in}}%
\pgfpathlineto{\pgfqpoint{2.708848in}{4.138235in}}%
\pgfpathlineto{\pgfqpoint{2.711161in}{4.113270in}}%
\pgfpathlineto{\pgfqpoint{2.713473in}{3.938817in}}%
\pgfpathlineto{\pgfqpoint{2.725035in}{2.626820in}}%
\pgfpathlineto{\pgfqpoint{2.727347in}{2.657958in}}%
\pgfpathlineto{\pgfqpoint{2.729660in}{2.829671in}}%
\pgfpathlineto{\pgfqpoint{2.741221in}{4.133273in}}%
\pgfpathlineto{\pgfqpoint{2.743534in}{4.126121in}}%
\pgfpathlineto{\pgfqpoint{2.745846in}{3.983367in}}%
\pgfpathlineto{\pgfqpoint{2.750471in}{3.420185in}}%
\pgfpathlineto{\pgfqpoint{2.755096in}{2.836057in}}%
\pgfpathlineto{\pgfqpoint{2.757408in}{2.666524in}}%
\pgfpathlineto{\pgfqpoint{2.759720in}{2.622218in}}%
\pgfpathlineto{\pgfqpoint{2.762033in}{2.709158in}}%
\pgfpathlineto{\pgfqpoint{2.766657in}{3.191618in}}%
\pgfpathlineto{\pgfqpoint{2.773594in}{4.016067in}}%
\pgfpathlineto{\pgfqpoint{2.775907in}{4.133950in}}%
\pgfpathlineto{\pgfqpoint{2.778219in}{4.130102in}}%
\pgfpathlineto{\pgfqpoint{2.780531in}{4.006763in}}%
\pgfpathlineto{\pgfqpoint{2.785156in}{3.501355in}}%
\pgfpathlineto{\pgfqpoint{2.792093in}{2.729030in}}%
\pgfpathlineto{\pgfqpoint{2.794406in}{2.628792in}}%
\pgfpathlineto{\pgfqpoint{2.796718in}{2.642384in}}%
\pgfpathlineto{\pgfqpoint{2.799030in}{2.766205in}}%
\pgfpathlineto{\pgfqpoint{2.803655in}{3.252639in}}%
\pgfpathlineto{\pgfqpoint{2.810592in}{4.012863in}}%
\pgfpathlineto{\pgfqpoint{2.812904in}{4.128082in}}%
\pgfpathlineto{\pgfqpoint{2.815217in}{4.139312in}}%
\pgfpathlineto{\pgfqpoint{2.817529in}{4.046508in}}%
\pgfpathlineto{\pgfqpoint{2.822154in}{3.617275in}}%
\pgfpathlineto{\pgfqpoint{2.829091in}{2.844028in}}%
\pgfpathlineto{\pgfqpoint{2.831403in}{2.687932in}}%
\pgfpathlineto{\pgfqpoint{2.833716in}{2.621934in}}%
\pgfpathlineto{\pgfqpoint{2.836028in}{2.653066in}}%
\pgfpathlineto{\pgfqpoint{2.838340in}{2.775941in}}%
\pgfpathlineto{\pgfqpoint{2.842965in}{3.221299in}}%
\pgfpathlineto{\pgfqpoint{2.849902in}{3.947996in}}%
\pgfpathlineto{\pgfqpoint{2.852214in}{4.088827in}}%
\pgfpathlineto{\pgfqpoint{2.854527in}{4.146602in}}%
\pgfpathlineto{\pgfqpoint{2.856839in}{4.115954in}}%
\pgfpathlineto{\pgfqpoint{2.859152in}{4.001795in}}%
\pgfpathlineto{\pgfqpoint{2.863776in}{3.587351in}}%
\pgfpathlineto{\pgfqpoint{2.870713in}{2.877084in}}%
\pgfpathlineto{\pgfqpoint{2.875338in}{2.634184in}}%
\pgfpathlineto{\pgfqpoint{2.877650in}{2.627193in}}%
\pgfpathlineto{\pgfqpoint{2.879963in}{2.698515in}}%
\pgfpathlineto{\pgfqpoint{2.884587in}{3.034789in}}%
\pgfpathlineto{\pgfqpoint{2.893837in}{3.921961in}}%
\pgfpathlineto{\pgfqpoint{2.898462in}{4.136375in}}%
\pgfpathlineto{\pgfqpoint{2.900774in}{4.140807in}}%
\pgfpathlineto{\pgfqpoint{2.903086in}{4.075622in}}%
\pgfpathlineto{\pgfqpoint{2.907711in}{3.770521in}}%
\pgfpathlineto{\pgfqpoint{2.919273in}{2.766205in}}%
\pgfpathlineto{\pgfqpoint{2.921585in}{2.663889in}}%
\pgfpathlineto{\pgfqpoint{2.923897in}{2.621398in}}%
\pgfpathlineto{\pgfqpoint{2.926210in}{2.641051in}}%
\pgfpathlineto{\pgfqpoint{2.928522in}{2.720084in}}%
\pgfpathlineto{\pgfqpoint{2.933147in}{3.022851in}}%
\pgfpathlineto{\pgfqpoint{2.944709in}{3.971758in}}%
\pgfpathlineto{\pgfqpoint{2.949333in}{4.138235in}}%
\pgfpathlineto{\pgfqpoint{2.951646in}{4.143202in}}%
\pgfpathlineto{\pgfqpoint{2.953958in}{4.095722in}}%
\pgfpathlineto{\pgfqpoint{2.958583in}{3.863836in}}%
\pgfpathlineto{\pgfqpoint{2.974769in}{2.709158in}}%
\pgfpathlineto{\pgfqpoint{2.977082in}{2.642724in}}%
\pgfpathlineto{\pgfqpoint{2.979394in}{2.620364in}}%
\pgfpathlineto{\pgfqpoint{2.981706in}{2.642384in}}%
\pgfpathlineto{\pgfqpoint{2.984019in}{2.706519in}}%
\pgfpathlineto{\pgfqpoint{2.988643in}{2.940907in}}%
\pgfpathlineto{\pgfqpoint{3.004830in}{4.016067in}}%
\pgfpathlineto{\pgfqpoint{3.009455in}{4.138235in}}%
\pgfpathlineto{\pgfqpoint{3.011767in}{4.146286in}}%
\pgfpathlineto{\pgfqpoint{3.014079in}{4.119110in}}%
\pgfpathlineto{\pgfqpoint{3.016392in}{4.058842in}}%
\pgfpathlineto{\pgfqpoint{3.021016in}{3.854390in}}%
\pgfpathlineto{\pgfqpoint{3.030266in}{3.266645in}}%
\pgfpathlineto{\pgfqpoint{3.037203in}{2.864471in}}%
\pgfpathlineto{\pgfqpoint{3.041828in}{2.691777in}}%
\pgfpathlineto{\pgfqpoint{3.044140in}{2.643238in}}%
\pgfpathlineto{\pgfqpoint{3.046452in}{2.621591in}}%
\pgfpathlineto{\pgfqpoint{3.048765in}{2.626820in}}%
\pgfpathlineto{\pgfqpoint{3.051077in}{2.657958in}}%
\pgfpathlineto{\pgfqpoint{3.055702in}{2.789948in}}%
\pgfpathlineto{\pgfqpoint{3.062639in}{3.115867in}}%
\pgfpathlineto{\pgfqpoint{3.076513in}{3.849904in}}%
\pgfpathlineto{\pgfqpoint{3.081138in}{4.018052in}}%
\pgfpathlineto{\pgfqpoint{3.085762in}{4.118918in}}%
\pgfpathlineto{\pgfqpoint{3.088075in}{4.142154in}}%
\pgfpathlineto{\pgfqpoint{3.090387in}{4.147209in}}%
\pgfpathlineto{\pgfqpoint{3.092699in}{4.134611in}}%
\pgfpathlineto{\pgfqpoint{3.095012in}{4.105285in}}%
\pgfpathlineto{\pgfqpoint{3.099636in}{4.001795in}}%
\pgfpathlineto{\pgfqpoint{3.106573in}{3.760684in}}%
\pgfpathlineto{\pgfqpoint{3.129697in}{2.844028in}}%
\pgfpathlineto{\pgfqpoint{3.136634in}{2.690286in}}%
\pgfpathlineto{\pgfqpoint{3.141259in}{2.636273in}}%
\pgfpathlineto{\pgfqpoint{3.143571in}{2.623726in}}%
\pgfpathlineto{\pgfqpoint{3.145883in}{2.620448in}}%
\pgfpathlineto{\pgfqpoint{3.148196in}{2.626017in}}%
\pgfpathlineto{\pgfqpoint{3.150508in}{2.639918in}}%
\pgfpathlineto{\pgfqpoint{3.155133in}{2.690286in}}%
\pgfpathlineto{\pgfqpoint{3.162070in}{2.811940in}}%
\pgfpathlineto{\pgfqpoint{3.171319in}{3.029749in}}%
\pgfpathlineto{\pgfqpoint{3.194443in}{3.608472in}}%
\pgfpathlineto{\pgfqpoint{3.203692in}{3.788732in}}%
\pgfpathlineto{\pgfqpoint{3.212942in}{3.925976in}}%
\pgfpathlineto{\pgfqpoint{3.219879in}{4.001795in}}%
\pgfpathlineto{\pgfqpoint{3.226816in}{4.057174in}}%
\pgfpathlineto{\pgfqpoint{3.233753in}{4.095465in}}%
\pgfpathlineto{\pgfqpoint{3.240690in}{4.120252in}}%
\pgfpathlineto{\pgfqpoint{3.247627in}{4.135000in}}%
\pgfpathlineto{\pgfqpoint{3.254564in}{4.142812in}}%
\pgfpathlineto{\pgfqpoint{3.261501in}{4.146286in}}%
\pgfpathlineto{\pgfqpoint{3.270751in}{4.147552in}}%
\pgfpathlineto{\pgfqpoint{3.293874in}{4.147209in}}%
\pgfpathlineto{\pgfqpoint{3.300811in}{4.145473in}}%
\pgfpathlineto{\pgfqpoint{3.305436in}{4.142812in}}%
\pgfpathlineto{\pgfqpoint{3.310061in}{4.138235in}}%
\pgfpathlineto{\pgfqpoint{3.314685in}{4.130998in}}%
\pgfpathlineto{\pgfqpoint{3.319310in}{4.120252in}}%
\pgfpathlineto{\pgfqpoint{3.323935in}{4.105051in}}%
\pgfpathlineto{\pgfqpoint{3.330872in}{4.071660in}}%
\pgfpathlineto{\pgfqpoint{3.337809in}{4.022370in}}%
\pgfpathlineto{\pgfqpoint{3.344746in}{3.953703in}}%
\pgfpathlineto{\pgfqpoint{3.351683in}{3.862731in}}%
\pgfpathlineto{\pgfqpoint{3.360932in}{3.703865in}}%
\pgfpathlineto{\pgfqpoint{3.370182in}{3.503459in}}%
\pgfpathlineto{\pgfqpoint{3.386368in}{3.089475in}}%
\pgfpathlineto{\pgfqpoint{3.397930in}{2.811940in}}%
\pgfpathlineto{\pgfqpoint{3.404867in}{2.690286in}}%
\pgfpathlineto{\pgfqpoint{3.409492in}{2.639918in}}%
\pgfpathlineto{\pgfqpoint{3.411804in}{2.626017in}}%
\pgfpathlineto{\pgfqpoint{3.414117in}{2.620448in}}%
\pgfpathlineto{\pgfqpoint{3.416429in}{2.623726in}}%
\pgfpathlineto{\pgfqpoint{3.418741in}{2.636273in}}%
\pgfpathlineto{\pgfqpoint{3.423366in}{2.690286in}}%
\pgfpathlineto{\pgfqpoint{3.427991in}{2.783316in}}%
\pgfpathlineto{\pgfqpoint{3.434928in}{2.991374in}}%
\pgfpathlineto{\pgfqpoint{3.444177in}{3.363419in}}%
\pgfpathlineto{\pgfqpoint{3.458051in}{3.930953in}}%
\pgfpathlineto{\pgfqpoint{3.462676in}{4.060496in}}%
\pgfpathlineto{\pgfqpoint{3.467301in}{4.134611in}}%
\pgfpathlineto{\pgfqpoint{3.469613in}{4.147209in}}%
\pgfpathlineto{\pgfqpoint{3.471925in}{4.142154in}}%
\pgfpathlineto{\pgfqpoint{3.474238in}{4.118918in}}%
\pgfpathlineto{\pgfqpoint{3.478862in}{4.018052in}}%
\pgfpathlineto{\pgfqpoint{3.483487in}{3.849904in}}%
\pgfpathlineto{\pgfqpoint{3.490424in}{3.503459in}}%
\pgfpathlineto{\pgfqpoint{3.501986in}{2.885096in}}%
\pgfpathlineto{\pgfqpoint{3.506611in}{2.713186in}}%
\pgfpathlineto{\pgfqpoint{3.511235in}{2.626820in}}%
\pgfpathlineto{\pgfqpoint{3.513548in}{2.621591in}}%
\pgfpathlineto{\pgfqpoint{3.515860in}{2.643238in}}%
\pgfpathlineto{\pgfqpoint{3.518172in}{2.691777in}}%
\pgfpathlineto{\pgfqpoint{3.522797in}{2.864471in}}%
\pgfpathlineto{\pgfqpoint{3.529734in}{3.266645in}}%
\pgfpathlineto{\pgfqpoint{3.541296in}{3.969030in}}%
\pgfpathlineto{\pgfqpoint{3.545921in}{4.119110in}}%
\pgfpathlineto{\pgfqpoint{3.548233in}{4.146286in}}%
\pgfpathlineto{\pgfqpoint{3.550545in}{4.138235in}}%
\pgfpathlineto{\pgfqpoint{3.552858in}{4.094428in}}%
\pgfpathlineto{\pgfqpoint{3.557483in}{3.906124in}}%
\pgfpathlineto{\pgfqpoint{3.564420in}{3.441437in}}%
\pgfpathlineto{\pgfqpoint{3.573669in}{2.808194in}}%
\pgfpathlineto{\pgfqpoint{3.578294in}{2.642384in}}%
\pgfpathlineto{\pgfqpoint{3.580606in}{2.620364in}}%
\pgfpathlineto{\pgfqpoint{3.582918in}{2.642724in}}%
\pgfpathlineto{\pgfqpoint{3.585231in}{2.709158in}}%
\pgfpathlineto{\pgfqpoint{3.589855in}{2.959599in}}%
\pgfpathlineto{\pgfqpoint{3.606042in}{4.095722in}}%
\pgfpathlineto{\pgfqpoint{3.608354in}{4.143202in}}%
\pgfpathlineto{\pgfqpoint{3.610667in}{4.138235in}}%
\pgfpathlineto{\pgfqpoint{3.612979in}{4.080068in}}%
\pgfpathlineto{\pgfqpoint{3.617604in}{3.820128in}}%
\pgfpathlineto{\pgfqpoint{3.631478in}{2.720084in}}%
\pgfpathlineto{\pgfqpoint{3.633790in}{2.641051in}}%
\pgfpathlineto{\pgfqpoint{3.636103in}{2.621398in}}%
\pgfpathlineto{\pgfqpoint{3.638415in}{2.663889in}}%
\pgfpathlineto{\pgfqpoint{3.640727in}{2.766205in}}%
\pgfpathlineto{\pgfqpoint{3.645352in}{3.115867in}}%
\pgfpathlineto{\pgfqpoint{3.654601in}{3.947996in}}%
\pgfpathlineto{\pgfqpoint{3.656914in}{4.075622in}}%
\pgfpathlineto{\pgfqpoint{3.659226in}{4.140807in}}%
\pgfpathlineto{\pgfqpoint{3.661538in}{4.136375in}}%
\pgfpathlineto{\pgfqpoint{3.663851in}{4.061481in}}%
\pgfpathlineto{\pgfqpoint{3.668476in}{3.730051in}}%
\pgfpathlineto{\pgfqpoint{3.680037in}{2.698515in}}%
\pgfpathlineto{\pgfqpoint{3.682350in}{2.627193in}}%
\pgfpathlineto{\pgfqpoint{3.684662in}{2.634184in}}%
\pgfpathlineto{\pgfqpoint{3.686974in}{2.720084in}}%
\pgfpathlineto{\pgfqpoint{3.691599in}{3.089475in}}%
\pgfpathlineto{\pgfqpoint{3.700848in}{4.001795in}}%
\pgfpathlineto{\pgfqpoint{3.703161in}{4.115954in}}%
\pgfpathlineto{\pgfqpoint{3.705473in}{4.146602in}}%
\pgfpathlineto{\pgfqpoint{3.707786in}{4.088827in}}%
\pgfpathlineto{\pgfqpoint{3.712410in}{3.739508in}}%
\pgfpathlineto{\pgfqpoint{3.721660in}{2.775941in}}%
\pgfpathlineto{\pgfqpoint{3.723972in}{2.653066in}}%
\pgfpathlineto{\pgfqpoint{3.726284in}{2.621934in}}%
\pgfpathlineto{\pgfqpoint{3.728597in}{2.687932in}}%
\pgfpathlineto{\pgfqpoint{3.733221in}{3.071241in}}%
\pgfpathlineto{\pgfqpoint{3.742471in}{4.046508in}}%
\pgfpathlineto{\pgfqpoint{3.744783in}{4.139312in}}%
\pgfpathlineto{\pgfqpoint{3.747096in}{4.128082in}}%
\pgfpathlineto{\pgfqpoint{3.749408in}{4.012863in}}%
\pgfpathlineto{\pgfqpoint{3.754033in}{3.542537in}}%
\pgfpathlineto{\pgfqpoint{3.760970in}{2.766205in}}%
\pgfpathlineto{\pgfqpoint{3.763282in}{2.642384in}}%
\pgfpathlineto{\pgfqpoint{3.765594in}{2.628792in}}%
\pgfpathlineto{\pgfqpoint{3.767907in}{2.729030in}}%
\pgfpathlineto{\pgfqpoint{3.772531in}{3.200561in}}%
\pgfpathlineto{\pgfqpoint{3.779469in}{4.006763in}}%
\pgfpathlineto{\pgfqpoint{3.781781in}{4.130102in}}%
\pgfpathlineto{\pgfqpoint{3.784093in}{4.133950in}}%
\pgfpathlineto{\pgfqpoint{3.786406in}{4.016067in}}%
\pgfpathlineto{\pgfqpoint{3.791030in}{3.503459in}}%
\pgfpathlineto{\pgfqpoint{3.797967in}{2.709158in}}%
\pgfpathlineto{\pgfqpoint{3.800280in}{2.622218in}}%
\pgfpathlineto{\pgfqpoint{3.802592in}{2.666524in}}%
\pgfpathlineto{\pgfqpoint{3.804904in}{2.836057in}}%
\pgfpathlineto{\pgfqpoint{3.816466in}{4.126121in}}%
\pgfpathlineto{\pgfqpoint{3.818779in}{4.133273in}}%
\pgfpathlineto{\pgfqpoint{3.821091in}{4.001795in}}%
\pgfpathlineto{\pgfqpoint{3.825716in}{3.436482in}}%
\pgfpathlineto{\pgfqpoint{3.830340in}{2.829671in}}%
\pgfpathlineto{\pgfqpoint{3.832653in}{2.657958in}}%
\pgfpathlineto{\pgfqpoint{3.834965in}{2.626820in}}%
\pgfpathlineto{\pgfqpoint{3.837277in}{2.744077in}}%
\pgfpathlineto{\pgfqpoint{3.841902in}{3.312418in}}%
\pgfpathlineto{\pgfqpoint{3.846527in}{3.938817in}}%
\pgfpathlineto{\pgfqpoint{3.848839in}{4.113270in}}%
\pgfpathlineto{\pgfqpoint{3.851152in}{4.138235in}}%
\pgfpathlineto{\pgfqpoint{3.853464in}{4.006763in}}%
\pgfpathlineto{\pgfqpoint{3.858089in}{3.405290in}}%
\pgfpathlineto{\pgfqpoint{3.862713in}{2.783316in}}%
\pgfpathlineto{\pgfqpoint{3.865026in}{2.635002in}}%
\pgfpathlineto{\pgfqpoint{3.867338in}{2.649082in}}%
\pgfpathlineto{\pgfqpoint{3.869650in}{2.824330in}}%
\pgfpathlineto{\pgfqpoint{3.878900in}{4.060496in}}%
\pgfpathlineto{\pgfqpoint{3.881212in}{4.147636in}}%
\pgfpathlineto{\pgfqpoint{3.883524in}{4.059836in}}%
\pgfpathlineto{\pgfqpoint{3.888149in}{3.469702in}}%
\pgfpathlineto{\pgfqpoint{3.892774in}{2.800799in}}%
\pgfpathlineto{\pgfqpoint{3.895086in}{2.636855in}}%
\pgfpathlineto{\pgfqpoint{3.897399in}{2.651643in}}%
\pgfpathlineto{\pgfqpoint{3.899711in}{2.843526in}}%
\pgfpathlineto{\pgfqpoint{3.908960in}{4.101476in}}%
\pgfpathlineto{\pgfqpoint{3.911273in}{4.138235in}}%
\pgfpathlineto{\pgfqpoint{3.913585in}{3.984684in}}%
\pgfpathlineto{\pgfqpoint{3.922834in}{2.687932in}}%
\pgfpathlineto{\pgfqpoint{3.925147in}{2.623726in}}%
\pgfpathlineto{\pgfqpoint{3.927459in}{2.759599in}}%
\pgfpathlineto{\pgfqpoint{3.932084in}{3.450634in}}%
\pgfpathlineto{\pgfqpoint{3.936709in}{4.077714in}}%
\pgfpathlineto{\pgfqpoint{3.939021in}{4.144003in}}%
\pgfpathlineto{\pgfqpoint{3.941333in}{4.001795in}}%
\pgfpathlineto{\pgfqpoint{3.950583in}{2.672535in}}%
\pgfpathlineto{\pgfqpoint{3.952895in}{2.631145in}}%
\pgfpathlineto{\pgfqpoint{3.955207in}{2.804943in}}%
\pgfpathlineto{\pgfqpoint{3.964457in}{4.124762in}}%
\pgfpathlineto{\pgfqpoint{3.966769in}{4.112210in}}%
\pgfpathlineto{\pgfqpoint{3.969082in}{3.882904in}}%
\pgfpathlineto{\pgfqpoint{3.976019in}{2.759190in}}%
\pgfpathlineto{\pgfqpoint{3.978331in}{2.621398in}}%
\pgfpathlineto{\pgfqpoint{3.980643in}{2.717644in}}%
\pgfpathlineto{\pgfqpoint{3.985268in}{3.437190in}}%
\pgfpathlineto{\pgfqpoint{3.989893in}{4.097760in}}%
\pgfpathlineto{\pgfqpoint{3.992205in}{4.131145in}}%
\pgfpathlineto{\pgfqpoint{3.994517in}{3.925976in}}%
\pgfpathlineto{\pgfqpoint{4.001455in}{2.766205in}}%
\pgfpathlineto{\pgfqpoint{4.003767in}{2.621221in}}%
\pgfpathlineto{\pgfqpoint{4.006079in}{2.727212in}}%
\pgfpathlineto{\pgfqpoint{4.010704in}{3.487309in}}%
\pgfpathlineto{\pgfqpoint{4.015329in}{4.122647in}}%
\pgfpathlineto{\pgfqpoint{4.017641in}{4.105051in}}%
\pgfpathlineto{\pgfqpoint{4.019953in}{3.840855in}}%
\pgfpathlineto{\pgfqpoint{4.026890in}{2.686480in}}%
\pgfpathlineto{\pgfqpoint{4.029203in}{2.633000in}}%
\pgfpathlineto{\pgfqpoint{4.031515in}{2.844028in}}%
\pgfpathlineto{\pgfqpoint{4.038452in}{4.040788in}}%
\pgfpathlineto{\pgfqpoint{4.040765in}{4.145473in}}%
\pgfpathlineto{\pgfqpoint{4.043077in}{3.974017in}}%
\pgfpathlineto{\pgfqpoint{4.050014in}{2.755943in}}%
\pgfpathlineto{\pgfqpoint{4.052326in}{2.620448in}}%
\pgfpathlineto{\pgfqpoint{4.054639in}{2.769982in}}%
\pgfpathlineto{\pgfqpoint{4.063888in}{4.147636in}}%
\pgfpathlineto{\pgfqpoint{4.066200in}{4.000122in}}%
\pgfpathlineto{\pgfqpoint{4.075450in}{2.620791in}}%
\pgfpathlineto{\pgfqpoint{4.077762in}{2.787281in}}%
\pgfpathlineto{\pgfqpoint{4.084699in}{4.040788in}}%
\pgfpathlineto{\pgfqpoint{4.087012in}{4.142812in}}%
\pgfpathlineto{\pgfqpoint{4.089324in}{3.936373in}}%
\pgfpathlineto{\pgfqpoint{4.096261in}{2.686480in}}%
\pgfpathlineto{\pgfqpoint{4.098573in}{2.641879in}}%
\pgfpathlineto{\pgfqpoint{4.100886in}{2.907484in}}%
\pgfpathlineto{\pgfqpoint{4.107823in}{4.122647in}}%
\pgfpathlineto{\pgfqpoint{4.110135in}{4.084379in}}%
\pgfpathlineto{\pgfqpoint{4.114760in}{3.249144in}}%
\pgfpathlineto{\pgfqpoint{4.117072in}{2.810531in}}%
\pgfpathlineto{\pgfqpoint{4.119385in}{2.621221in}}%
\pgfpathlineto{\pgfqpoint{4.121697in}{2.766205in}}%
\pgfpathlineto{\pgfqpoint{4.128634in}{4.063111in}}%
\pgfpathlineto{\pgfqpoint{4.130946in}{4.131145in}}%
\pgfpathlineto{\pgfqpoint{4.133259in}{3.862731in}}%
\pgfpathlineto{\pgfqpoint{4.140196in}{2.633000in}}%
\pgfpathlineto{\pgfqpoint{4.142508in}{2.717644in}}%
\pgfpathlineto{\pgfqpoint{4.151758in}{4.138235in}}%
\pgfpathlineto{\pgfqpoint{4.154070in}{3.882904in}}%
\pgfpathlineto{\pgfqpoint{4.161007in}{2.630908in}}%
\pgfpathlineto{\pgfqpoint{4.163319in}{2.730127in}}%
\pgfpathlineto{\pgfqpoint{4.170256in}{4.068229in}}%
\pgfpathlineto{\pgfqpoint{4.172569in}{4.122647in}}%
\pgfpathlineto{\pgfqpoint{4.174881in}{3.812523in}}%
\pgfpathlineto{\pgfqpoint{4.179506in}{2.811940in}}%
\pgfpathlineto{\pgfqpoint{4.181818in}{2.620364in}}%
\pgfpathlineto{\pgfqpoint{4.184131in}{2.812881in}}%
\pgfpathlineto{\pgfqpoint{4.191068in}{4.128082in}}%
\pgfpathlineto{\pgfqpoint{4.193380in}{4.051738in}}%
\pgfpathlineto{\pgfqpoint{4.200317in}{2.687932in}}%
\pgfpathlineto{\pgfqpoint{4.202629in}{2.658399in}}%
\pgfpathlineto{\pgfqpoint{4.204942in}{3.010407in}}%
\pgfpathlineto{\pgfqpoint{4.209566in}{4.018052in}}%
\pgfpathlineto{\pgfqpoint{4.211879in}{4.138235in}}%
\pgfpathlineto{\pgfqpoint{4.214191in}{3.854390in}}%
\pgfpathlineto{\pgfqpoint{4.218816in}{2.814297in}}%
\pgfpathlineto{\pgfqpoint{4.221128in}{2.620641in}}%
\pgfpathlineto{\pgfqpoint{4.223441in}{2.843526in}}%
\pgfpathlineto{\pgfqpoint{4.228065in}{3.895675in}}%
\pgfpathlineto{\pgfqpoint{4.230378in}{4.145579in}}%
\pgfpathlineto{\pgfqpoint{4.232690in}{3.972211in}}%
\pgfpathlineto{\pgfqpoint{4.239627in}{2.628275in}}%
\pgfpathlineto{\pgfqpoint{4.241939in}{2.766205in}}%
\pgfpathlineto{\pgfqpoint{4.248876in}{4.134611in}}%
\pgfpathlineto{\pgfqpoint{4.251189in}{4.016465in}}%
\pgfpathlineto{\pgfqpoint{4.258126in}{2.635002in}}%
\pgfpathlineto{\pgfqpoint{4.260438in}{2.749948in}}%
\pgfpathlineto{\pgfqpoint{4.267375in}{4.135634in}}%
\pgfpathlineto{\pgfqpoint{4.269688in}{4.006763in}}%
\pgfpathlineto{\pgfqpoint{4.276625in}{2.626728in}}%
\pgfpathlineto{\pgfqpoint{4.278937in}{2.786839in}}%
\pgfpathlineto{\pgfqpoint{4.285874in}{4.146639in}}%
\pgfpathlineto{\pgfqpoint{4.288186in}{3.938329in}}%
\pgfpathlineto{\pgfqpoint{4.292811in}{2.829671in}}%
\pgfpathlineto{\pgfqpoint{4.295124in}{2.621591in}}%
\pgfpathlineto{\pgfqpoint{4.297436in}{2.893748in}}%
\pgfpathlineto{\pgfqpoint{4.302061in}{4.001795in}}%
\pgfpathlineto{\pgfqpoint{4.304373in}{4.133273in}}%
\pgfpathlineto{\pgfqpoint{4.306685in}{3.784511in}}%
\pgfpathlineto{\pgfqpoint{4.311310in}{2.702631in}}%
\pgfpathlineto{\pgfqpoint{4.313622in}{2.669736in}}%
\pgfpathlineto{\pgfqpoint{4.320559in}{4.118141in}}%
\pgfpathlineto{\pgfqpoint{4.322872in}{4.031559in}}%
\pgfpathlineto{\pgfqpoint{4.329809in}{2.621714in}}%
\pgfpathlineto{\pgfqpoint{4.332121in}{2.844028in}}%
\pgfpathlineto{\pgfqpoint{4.336746in}{3.986869in}}%
\pgfpathlineto{\pgfqpoint{4.339058in}{4.133950in}}%
\pgfpathlineto{\pgfqpoint{4.341371in}{3.769296in}}%
\pgfpathlineto{\pgfqpoint{4.345995in}{2.677545in}}%
\pgfpathlineto{\pgfqpoint{4.348308in}{2.702631in}}%
\pgfpathlineto{\pgfqpoint{4.355245in}{4.144661in}}%
\pgfpathlineto{\pgfqpoint{4.357557in}{3.928470in}}%
\pgfpathlineto{\pgfqpoint{4.362182in}{2.766205in}}%
\pgfpathlineto{\pgfqpoint{4.364494in}{2.642724in}}%
\pgfpathlineto{\pgfqpoint{4.366807in}{3.048110in}}%
\pgfpathlineto{\pgfqpoint{4.371431in}{4.118918in}}%
\pgfpathlineto{\pgfqpoint{4.373744in}{4.012863in}}%
\pgfpathlineto{\pgfqpoint{4.378368in}{2.839032in}}%
\pgfpathlineto{\pgfqpoint{4.380681in}{2.624798in}}%
\pgfpathlineto{\pgfqpoint{4.382993in}{2.967300in}}%
\pgfpathlineto{\pgfqpoint{4.387618in}{4.095722in}}%
\pgfpathlineto{\pgfqpoint{4.389930in}{4.046508in}}%
\pgfpathlineto{\pgfqpoint{4.396867in}{2.621934in}}%
\pgfpathlineto{\pgfqpoint{4.399179in}{2.947872in}}%
\pgfpathlineto{\pgfqpoint{4.403804in}{4.094428in}}%
\pgfpathlineto{\pgfqpoint{4.406117in}{4.043308in}}%
\pgfpathlineto{\pgfqpoint{4.413054in}{2.625109in}}%
\pgfpathlineto{\pgfqpoint{4.415366in}{2.986515in}}%
\pgfpathlineto{\pgfqpoint{4.419991in}{4.115954in}}%
\pgfpathlineto{\pgfqpoint{4.422303in}{4.001795in}}%
\pgfpathlineto{\pgfqpoint{4.426928in}{2.786839in}}%
\pgfpathlineto{\pgfqpoint{4.429240in}{2.644108in}}%
\pgfpathlineto{\pgfqpoint{4.431552in}{3.089475in}}%
\pgfpathlineto{\pgfqpoint{4.436177in}{4.142891in}}%
\pgfpathlineto{\pgfqpoint{4.438490in}{3.905606in}}%
\pgfpathlineto{\pgfqpoint{4.443114in}{2.698515in}}%
\pgfpathlineto{\pgfqpoint{4.445427in}{2.706519in}}%
\pgfpathlineto{\pgfqpoint{4.452364in}{4.138235in}}%
\pgfpathlineto{\pgfqpoint{4.454676in}{3.730051in}}%
\pgfpathlineto{\pgfqpoint{4.459301in}{2.628792in}}%
\pgfpathlineto{\pgfqpoint{4.461613in}{2.852117in}}%
\pgfpathlineto{\pgfqpoint{4.466238in}{4.075622in}}%
\pgfpathlineto{\pgfqpoint{4.468550in}{4.047916in}}%
\pgfpathlineto{\pgfqpoint{4.473175in}{2.804943in}}%
\pgfpathlineto{\pgfqpoint{4.475487in}{2.644108in}}%
\pgfpathlineto{\pgfqpoint{4.477800in}{3.115867in}}%
\pgfpathlineto{\pgfqpoint{4.480112in}{3.818379in}}%
\pgfpathlineto{\pgfqpoint{4.482424in}{4.147636in}}%
\pgfpathlineto{\pgfqpoint{4.484737in}{3.817211in}}%
\pgfpathlineto{\pgfqpoint{4.489361in}{2.641051in}}%
\pgfpathlineto{\pgfqpoint{4.491674in}{2.820004in}}%
\pgfpathlineto{\pgfqpoint{4.496298in}{4.071660in}}%
\pgfpathlineto{\pgfqpoint{4.498611in}{4.043308in}}%
\pgfpathlineto{\pgfqpoint{4.503235in}{2.775941in}}%
\pgfpathlineto{\pgfqpoint{4.505548in}{2.662715in}}%
\pgfpathlineto{\pgfqpoint{4.512485in}{4.138235in}}%
\pgfpathlineto{\pgfqpoint{4.514797in}{3.696759in}}%
\pgfpathlineto{\pgfqpoint{4.517110in}{2.970876in}}%
\pgfpathlineto{\pgfqpoint{4.519422in}{2.620369in}}%
\pgfpathlineto{\pgfqpoint{4.521734in}{2.967300in}}%
\pgfpathlineto{\pgfqpoint{4.526359in}{4.139312in}}%
\pgfpathlineto{\pgfqpoint{4.528671in}{3.885050in}}%
\pgfpathlineto{\pgfqpoint{4.533296in}{2.649082in}}%
\pgfpathlineto{\pgfqpoint{4.535608in}{2.816666in}}%
\pgfpathlineto{\pgfqpoint{4.540233in}{4.091528in}}%
\pgfpathlineto{\pgfqpoint{4.542545in}{4.001795in}}%
\pgfpathlineto{\pgfqpoint{4.547170in}{2.706519in}}%
\pgfpathlineto{\pgfqpoint{4.549483in}{2.729030in}}%
\pgfpathlineto{\pgfqpoint{4.554107in}{4.032310in}}%
\pgfpathlineto{\pgfqpoint{4.556420in}{4.065369in}}%
\pgfpathlineto{\pgfqpoint{4.561044in}{2.761237in}}%
\pgfpathlineto{\pgfqpoint{4.563357in}{2.684757in}}%
\pgfpathlineto{\pgfqpoint{4.567981in}{3.986869in}}%
\pgfpathlineto{\pgfqpoint{4.570294in}{4.094428in}}%
\pgfpathlineto{\pgfqpoint{4.574918in}{2.793983in}}%
\pgfpathlineto{\pgfqpoint{4.577231in}{2.667747in}}%
\pgfpathlineto{\pgfqpoint{4.581855in}{3.969030in}}%
\pgfpathlineto{\pgfqpoint{4.584168in}{4.101476in}}%
\pgfpathlineto{\pgfqpoint{4.588793in}{2.795789in}}%
\pgfpathlineto{\pgfqpoint{4.591105in}{2.669736in}}%
\pgfpathlineto{\pgfqpoint{4.595730in}{3.983367in}}%
\pgfpathlineto{\pgfqpoint{4.598042in}{4.090185in}}%
\pgfpathlineto{\pgfqpoint{4.602667in}{2.766205in}}%
\pgfpathlineto{\pgfqpoint{4.604979in}{2.691777in}}%
\pgfpathlineto{\pgfqpoint{4.609604in}{4.026237in}}%
\pgfpathlineto{\pgfqpoint{4.611916in}{4.054814in}}%
\pgfpathlineto{\pgfqpoint{4.616541in}{2.713186in}}%
\pgfpathlineto{\pgfqpoint{4.618853in}{2.744077in}}%
\pgfpathlineto{\pgfqpoint{4.623478in}{4.084944in}}%
\pgfpathlineto{\pgfqpoint{4.625790in}{3.981161in}}%
\pgfpathlineto{\pgfqpoint{4.630415in}{2.654730in}}%
\pgfpathlineto{\pgfqpoint{4.632727in}{2.844028in}}%
\pgfpathlineto{\pgfqpoint{4.637352in}{4.135634in}}%
\pgfpathlineto{\pgfqpoint{4.639664in}{3.849904in}}%
\pgfpathlineto{\pgfqpoint{4.644289in}{2.620891in}}%
\pgfpathlineto{\pgfqpoint{4.646601in}{3.011026in}}%
\pgfpathlineto{\pgfqpoint{4.648914in}{3.808401in}}%
\pgfpathlineto{\pgfqpoint{4.651226in}{4.142154in}}%
\pgfpathlineto{\pgfqpoint{4.653538in}{3.644141in}}%
\pgfpathlineto{\pgfqpoint{4.655851in}{2.859292in}}%
\pgfpathlineto{\pgfqpoint{4.658163in}{2.653066in}}%
\pgfpathlineto{\pgfqpoint{4.662788in}{4.001795in}}%
\pgfpathlineto{\pgfqpoint{4.665100in}{4.059836in}}%
\pgfpathlineto{\pgfqpoint{4.669725in}{2.688810in}}%
\pgfpathlineto{\pgfqpoint{4.672037in}{2.795789in}}%
\pgfpathlineto{\pgfqpoint{4.676662in}{4.131145in}}%
\pgfpathlineto{\pgfqpoint{4.678974in}{3.849341in}}%
\pgfpathlineto{\pgfqpoint{4.681287in}{3.037949in}}%
\pgfpathlineto{\pgfqpoint{4.683599in}{2.620641in}}%
\pgfpathlineto{\pgfqpoint{4.685911in}{3.076430in}}%
\pgfpathlineto{\pgfqpoint{4.688224in}{3.885585in}}%
\pgfpathlineto{\pgfqpoint{4.690536in}{4.118141in}}%
\pgfpathlineto{\pgfqpoint{4.695161in}{2.749948in}}%
\pgfpathlineto{\pgfqpoint{4.697473in}{2.731968in}}%
\pgfpathlineto{\pgfqpoint{4.702098in}{4.110917in}}%
\pgfpathlineto{\pgfqpoint{4.704410in}{3.897779in}}%
\pgfpathlineto{\pgfqpoint{4.709035in}{2.620448in}}%
\pgfpathlineto{\pgfqpoint{4.711347in}{3.061559in}}%
\pgfpathlineto{\pgfqpoint{4.713660in}{3.885050in}}%
\pgfpathlineto{\pgfqpoint{4.715972in}{4.113899in}}%
\pgfpathlineto{\pgfqpoint{4.720597in}{2.725409in}}%
\pgfpathlineto{\pgfqpoint{4.722909in}{2.766205in}}%
\pgfpathlineto{\pgfqpoint{4.727534in}{4.133950in}}%
\pgfpathlineto{\pgfqpoint{4.729846in}{3.812523in}}%
\pgfpathlineto{\pgfqpoint{4.732159in}{2.971473in}}%
\pgfpathlineto{\pgfqpoint{4.734471in}{2.631145in}}%
\pgfpathlineto{\pgfqpoint{4.739096in}{4.000541in}}%
\pgfpathlineto{\pgfqpoint{4.741408in}{4.037873in}}%
\pgfpathlineto{\pgfqpoint{4.746033in}{2.643238in}}%
\pgfpathlineto{\pgfqpoint{4.748345in}{2.929994in}}%
\pgfpathlineto{\pgfqpoint{4.750657in}{3.779665in}}%
\pgfpathlineto{\pgfqpoint{4.752970in}{4.138235in}}%
\pgfpathlineto{\pgfqpoint{4.757594in}{2.759190in}}%
\pgfpathlineto{\pgfqpoint{4.759907in}{2.743690in}}%
\pgfpathlineto{\pgfqpoint{4.764531in}{4.135000in}}%
\pgfpathlineto{\pgfqpoint{4.766844in}{3.788732in}}%
\pgfpathlineto{\pgfqpoint{4.769156in}{2.929423in}}%
\pgfpathlineto{\pgfqpoint{4.771469in}{2.647000in}}%
\pgfpathlineto{\pgfqpoint{4.776093in}{4.063111in}}%
\pgfpathlineto{\pgfqpoint{4.778406in}{3.953703in}}%
\pgfpathlineto{\pgfqpoint{4.783030in}{2.620364in}}%
\pgfpathlineto{\pgfqpoint{4.785343in}{3.384000in}}%
\pgfpathlineto{\pgfqpoint{5.534545in}{3.384000in}}%
\pgfpathlineto{\pgfqpoint{5.534545in}{3.384000in}}%
\pgfusepath{stroke}%
\end{pgfscope}%
\begin{pgfscope}%
\pgfsetrectcap%
\pgfsetmiterjoin%
\pgfsetlinewidth{0.803000pt}%
\definecolor{currentstroke}{rgb}{0.000000,0.000000,0.000000}%
\pgfsetstrokecolor{currentstroke}%
\pgfsetdash{}{0pt}%
\pgfpathmoveto{\pgfqpoint{0.800000in}{2.544000in}}%
\pgfpathlineto{\pgfqpoint{0.800000in}{4.224000in}}%
\pgfusepath{stroke}%
\end{pgfscope}%
\begin{pgfscope}%
\pgfsetrectcap%
\pgfsetmiterjoin%
\pgfsetlinewidth{0.803000pt}%
\definecolor{currentstroke}{rgb}{0.000000,0.000000,0.000000}%
\pgfsetstrokecolor{currentstroke}%
\pgfsetdash{}{0pt}%
\pgfpathmoveto{\pgfqpoint{5.760000in}{2.544000in}}%
\pgfpathlineto{\pgfqpoint{5.760000in}{4.224000in}}%
\pgfusepath{stroke}%
\end{pgfscope}%
\begin{pgfscope}%
\pgfsetrectcap%
\pgfsetmiterjoin%
\pgfsetlinewidth{0.803000pt}%
\definecolor{currentstroke}{rgb}{0.000000,0.000000,0.000000}%
\pgfsetstrokecolor{currentstroke}%
\pgfsetdash{}{0pt}%
\pgfpathmoveto{\pgfqpoint{0.800000in}{2.544000in}}%
\pgfpathlineto{\pgfqpoint{5.760000in}{2.544000in}}%
\pgfusepath{stroke}%
\end{pgfscope}%
\begin{pgfscope}%
\pgfsetrectcap%
\pgfsetmiterjoin%
\pgfsetlinewidth{0.803000pt}%
\definecolor{currentstroke}{rgb}{0.000000,0.000000,0.000000}%
\pgfsetstrokecolor{currentstroke}%
\pgfsetdash{}{0pt}%
\pgfpathmoveto{\pgfqpoint{0.800000in}{4.224000in}}%
\pgfpathlineto{\pgfqpoint{5.760000in}{4.224000in}}%
\pgfusepath{stroke}%
\end{pgfscope}%
\begin{pgfscope}%
\pgfsetbuttcap%
\pgfsetmiterjoin%
\definecolor{currentfill}{rgb}{1.000000,1.000000,1.000000}%
\pgfsetfillcolor{currentfill}%
\pgfsetlinewidth{0.000000pt}%
\definecolor{currentstroke}{rgb}{0.000000,0.000000,0.000000}%
\pgfsetstrokecolor{currentstroke}%
\pgfsetstrokeopacity{0.000000}%
\pgfsetdash{}{0pt}%
\pgfpathmoveto{\pgfqpoint{0.800000in}{0.528000in}}%
\pgfpathlineto{\pgfqpoint{5.760000in}{0.528000in}}%
\pgfpathlineto{\pgfqpoint{5.760000in}{2.208000in}}%
\pgfpathlineto{\pgfqpoint{0.800000in}{2.208000in}}%
\pgfpathclose%
\pgfusepath{fill}%
\end{pgfscope}%
\begin{pgfscope}%
\pgfpathrectangle{\pgfqpoint{0.800000in}{0.528000in}}{\pgfqpoint{4.960000in}{1.680000in}} %
\pgfusepath{clip}%
\pgfsetrectcap%
\pgfsetroundjoin%
\pgfsetlinewidth{0.803000pt}%
\definecolor{currentstroke}{rgb}{0.690196,0.690196,0.690196}%
\pgfsetstrokecolor{currentstroke}%
\pgfsetdash{}{0pt}%
\pgfpathmoveto{\pgfqpoint{0.967616in}{0.528000in}}%
\pgfpathlineto{\pgfqpoint{0.967616in}{2.208000in}}%
\pgfusepath{stroke}%
\end{pgfscope}%
\begin{pgfscope}%
\pgfsetbuttcap%
\pgfsetroundjoin%
\definecolor{currentfill}{rgb}{0.000000,0.000000,0.000000}%
\pgfsetfillcolor{currentfill}%
\pgfsetlinewidth{0.803000pt}%
\definecolor{currentstroke}{rgb}{0.000000,0.000000,0.000000}%
\pgfsetstrokecolor{currentstroke}%
\pgfsetdash{}{0pt}%
\pgfsys@defobject{currentmarker}{\pgfqpoint{0.000000in}{-0.048611in}}{\pgfqpoint{0.000000in}{0.000000in}}{%
\pgfpathmoveto{\pgfqpoint{0.000000in}{0.000000in}}%
\pgfpathlineto{\pgfqpoint{0.000000in}{-0.048611in}}%
\pgfusepath{stroke,fill}%
}%
\begin{pgfscope}%
\pgfsys@transformshift{0.967616in}{0.528000in}%
\pgfsys@useobject{currentmarker}{}%
\end{pgfscope}%
\end{pgfscope}%
\begin{pgfscope}%
\pgftext[x=0.967616in,y=0.430778in,,top]{\sffamily\fontsize{10.000000}{12.000000}\selectfont -100}%
\end{pgfscope}%
\begin{pgfscope}%
\pgfpathrectangle{\pgfqpoint{0.800000in}{0.528000in}}{\pgfqpoint{4.960000in}{1.680000in}} %
\pgfusepath{clip}%
\pgfsetrectcap%
\pgfsetroundjoin%
\pgfsetlinewidth{0.803000pt}%
\definecolor{currentstroke}{rgb}{0.690196,0.690196,0.690196}%
\pgfsetstrokecolor{currentstroke}%
\pgfsetdash{}{0pt}%
\pgfpathmoveto{\pgfqpoint{1.546001in}{0.528000in}}%
\pgfpathlineto{\pgfqpoint{1.546001in}{2.208000in}}%
\pgfusepath{stroke}%
\end{pgfscope}%
\begin{pgfscope}%
\pgfsetbuttcap%
\pgfsetroundjoin%
\definecolor{currentfill}{rgb}{0.000000,0.000000,0.000000}%
\pgfsetfillcolor{currentfill}%
\pgfsetlinewidth{0.803000pt}%
\definecolor{currentstroke}{rgb}{0.000000,0.000000,0.000000}%
\pgfsetstrokecolor{currentstroke}%
\pgfsetdash{}{0pt}%
\pgfsys@defobject{currentmarker}{\pgfqpoint{0.000000in}{-0.048611in}}{\pgfqpoint{0.000000in}{0.000000in}}{%
\pgfpathmoveto{\pgfqpoint{0.000000in}{0.000000in}}%
\pgfpathlineto{\pgfqpoint{0.000000in}{-0.048611in}}%
\pgfusepath{stroke,fill}%
}%
\begin{pgfscope}%
\pgfsys@transformshift{1.546001in}{0.528000in}%
\pgfsys@useobject{currentmarker}{}%
\end{pgfscope}%
\end{pgfscope}%
\begin{pgfscope}%
\pgftext[x=1.546001in,y=0.430778in,,top]{\sffamily\fontsize{10.000000}{12.000000}\selectfont -75}%
\end{pgfscope}%
\begin{pgfscope}%
\pgfpathrectangle{\pgfqpoint{0.800000in}{0.528000in}}{\pgfqpoint{4.960000in}{1.680000in}} %
\pgfusepath{clip}%
\pgfsetrectcap%
\pgfsetroundjoin%
\pgfsetlinewidth{0.803000pt}%
\definecolor{currentstroke}{rgb}{0.690196,0.690196,0.690196}%
\pgfsetstrokecolor{currentstroke}%
\pgfsetdash{}{0pt}%
\pgfpathmoveto{\pgfqpoint{2.124386in}{0.528000in}}%
\pgfpathlineto{\pgfqpoint{2.124386in}{2.208000in}}%
\pgfusepath{stroke}%
\end{pgfscope}%
\begin{pgfscope}%
\pgfsetbuttcap%
\pgfsetroundjoin%
\definecolor{currentfill}{rgb}{0.000000,0.000000,0.000000}%
\pgfsetfillcolor{currentfill}%
\pgfsetlinewidth{0.803000pt}%
\definecolor{currentstroke}{rgb}{0.000000,0.000000,0.000000}%
\pgfsetstrokecolor{currentstroke}%
\pgfsetdash{}{0pt}%
\pgfsys@defobject{currentmarker}{\pgfqpoint{0.000000in}{-0.048611in}}{\pgfqpoint{0.000000in}{0.000000in}}{%
\pgfpathmoveto{\pgfqpoint{0.000000in}{0.000000in}}%
\pgfpathlineto{\pgfqpoint{0.000000in}{-0.048611in}}%
\pgfusepath{stroke,fill}%
}%
\begin{pgfscope}%
\pgfsys@transformshift{2.124386in}{0.528000in}%
\pgfsys@useobject{currentmarker}{}%
\end{pgfscope}%
\end{pgfscope}%
\begin{pgfscope}%
\pgftext[x=2.124386in,y=0.430778in,,top]{\sffamily\fontsize{10.000000}{12.000000}\selectfont -50}%
\end{pgfscope}%
\begin{pgfscope}%
\pgfpathrectangle{\pgfqpoint{0.800000in}{0.528000in}}{\pgfqpoint{4.960000in}{1.680000in}} %
\pgfusepath{clip}%
\pgfsetrectcap%
\pgfsetroundjoin%
\pgfsetlinewidth{0.803000pt}%
\definecolor{currentstroke}{rgb}{0.690196,0.690196,0.690196}%
\pgfsetstrokecolor{currentstroke}%
\pgfsetdash{}{0pt}%
\pgfpathmoveto{\pgfqpoint{2.702771in}{0.528000in}}%
\pgfpathlineto{\pgfqpoint{2.702771in}{2.208000in}}%
\pgfusepath{stroke}%
\end{pgfscope}%
\begin{pgfscope}%
\pgfsetbuttcap%
\pgfsetroundjoin%
\definecolor{currentfill}{rgb}{0.000000,0.000000,0.000000}%
\pgfsetfillcolor{currentfill}%
\pgfsetlinewidth{0.803000pt}%
\definecolor{currentstroke}{rgb}{0.000000,0.000000,0.000000}%
\pgfsetstrokecolor{currentstroke}%
\pgfsetdash{}{0pt}%
\pgfsys@defobject{currentmarker}{\pgfqpoint{0.000000in}{-0.048611in}}{\pgfqpoint{0.000000in}{0.000000in}}{%
\pgfpathmoveto{\pgfqpoint{0.000000in}{0.000000in}}%
\pgfpathlineto{\pgfqpoint{0.000000in}{-0.048611in}}%
\pgfusepath{stroke,fill}%
}%
\begin{pgfscope}%
\pgfsys@transformshift{2.702771in}{0.528000in}%
\pgfsys@useobject{currentmarker}{}%
\end{pgfscope}%
\end{pgfscope}%
\begin{pgfscope}%
\pgftext[x=2.702771in,y=0.430778in,,top]{\sffamily\fontsize{10.000000}{12.000000}\selectfont -25}%
\end{pgfscope}%
\begin{pgfscope}%
\pgfpathrectangle{\pgfqpoint{0.800000in}{0.528000in}}{\pgfqpoint{4.960000in}{1.680000in}} %
\pgfusepath{clip}%
\pgfsetrectcap%
\pgfsetroundjoin%
\pgfsetlinewidth{0.803000pt}%
\definecolor{currentstroke}{rgb}{0.690196,0.690196,0.690196}%
\pgfsetstrokecolor{currentstroke}%
\pgfsetdash{}{0pt}%
\pgfpathmoveto{\pgfqpoint{3.281156in}{0.528000in}}%
\pgfpathlineto{\pgfqpoint{3.281156in}{2.208000in}}%
\pgfusepath{stroke}%
\end{pgfscope}%
\begin{pgfscope}%
\pgfsetbuttcap%
\pgfsetroundjoin%
\definecolor{currentfill}{rgb}{0.000000,0.000000,0.000000}%
\pgfsetfillcolor{currentfill}%
\pgfsetlinewidth{0.803000pt}%
\definecolor{currentstroke}{rgb}{0.000000,0.000000,0.000000}%
\pgfsetstrokecolor{currentstroke}%
\pgfsetdash{}{0pt}%
\pgfsys@defobject{currentmarker}{\pgfqpoint{0.000000in}{-0.048611in}}{\pgfqpoint{0.000000in}{0.000000in}}{%
\pgfpathmoveto{\pgfqpoint{0.000000in}{0.000000in}}%
\pgfpathlineto{\pgfqpoint{0.000000in}{-0.048611in}}%
\pgfusepath{stroke,fill}%
}%
\begin{pgfscope}%
\pgfsys@transformshift{3.281156in}{0.528000in}%
\pgfsys@useobject{currentmarker}{}%
\end{pgfscope}%
\end{pgfscope}%
\begin{pgfscope}%
\pgftext[x=3.281156in,y=0.430778in,,top]{\sffamily\fontsize{10.000000}{12.000000}\selectfont 0}%
\end{pgfscope}%
\begin{pgfscope}%
\pgfpathrectangle{\pgfqpoint{0.800000in}{0.528000in}}{\pgfqpoint{4.960000in}{1.680000in}} %
\pgfusepath{clip}%
\pgfsetrectcap%
\pgfsetroundjoin%
\pgfsetlinewidth{0.803000pt}%
\definecolor{currentstroke}{rgb}{0.690196,0.690196,0.690196}%
\pgfsetstrokecolor{currentstroke}%
\pgfsetdash{}{0pt}%
\pgfpathmoveto{\pgfqpoint{3.859541in}{0.528000in}}%
\pgfpathlineto{\pgfqpoint{3.859541in}{2.208000in}}%
\pgfusepath{stroke}%
\end{pgfscope}%
\begin{pgfscope}%
\pgfsetbuttcap%
\pgfsetroundjoin%
\definecolor{currentfill}{rgb}{0.000000,0.000000,0.000000}%
\pgfsetfillcolor{currentfill}%
\pgfsetlinewidth{0.803000pt}%
\definecolor{currentstroke}{rgb}{0.000000,0.000000,0.000000}%
\pgfsetstrokecolor{currentstroke}%
\pgfsetdash{}{0pt}%
\pgfsys@defobject{currentmarker}{\pgfqpoint{0.000000in}{-0.048611in}}{\pgfqpoint{0.000000in}{0.000000in}}{%
\pgfpathmoveto{\pgfqpoint{0.000000in}{0.000000in}}%
\pgfpathlineto{\pgfqpoint{0.000000in}{-0.048611in}}%
\pgfusepath{stroke,fill}%
}%
\begin{pgfscope}%
\pgfsys@transformshift{3.859541in}{0.528000in}%
\pgfsys@useobject{currentmarker}{}%
\end{pgfscope}%
\end{pgfscope}%
\begin{pgfscope}%
\pgftext[x=3.859541in,y=0.430778in,,top]{\sffamily\fontsize{10.000000}{12.000000}\selectfont 25}%
\end{pgfscope}%
\begin{pgfscope}%
\pgfpathrectangle{\pgfqpoint{0.800000in}{0.528000in}}{\pgfqpoint{4.960000in}{1.680000in}} %
\pgfusepath{clip}%
\pgfsetrectcap%
\pgfsetroundjoin%
\pgfsetlinewidth{0.803000pt}%
\definecolor{currentstroke}{rgb}{0.690196,0.690196,0.690196}%
\pgfsetstrokecolor{currentstroke}%
\pgfsetdash{}{0pt}%
\pgfpathmoveto{\pgfqpoint{4.437926in}{0.528000in}}%
\pgfpathlineto{\pgfqpoint{4.437926in}{2.208000in}}%
\pgfusepath{stroke}%
\end{pgfscope}%
\begin{pgfscope}%
\pgfsetbuttcap%
\pgfsetroundjoin%
\definecolor{currentfill}{rgb}{0.000000,0.000000,0.000000}%
\pgfsetfillcolor{currentfill}%
\pgfsetlinewidth{0.803000pt}%
\definecolor{currentstroke}{rgb}{0.000000,0.000000,0.000000}%
\pgfsetstrokecolor{currentstroke}%
\pgfsetdash{}{0pt}%
\pgfsys@defobject{currentmarker}{\pgfqpoint{0.000000in}{-0.048611in}}{\pgfqpoint{0.000000in}{0.000000in}}{%
\pgfpathmoveto{\pgfqpoint{0.000000in}{0.000000in}}%
\pgfpathlineto{\pgfqpoint{0.000000in}{-0.048611in}}%
\pgfusepath{stroke,fill}%
}%
\begin{pgfscope}%
\pgfsys@transformshift{4.437926in}{0.528000in}%
\pgfsys@useobject{currentmarker}{}%
\end{pgfscope}%
\end{pgfscope}%
\begin{pgfscope}%
\pgftext[x=4.437926in,y=0.430778in,,top]{\sffamily\fontsize{10.000000}{12.000000}\selectfont 50}%
\end{pgfscope}%
\begin{pgfscope}%
\pgfpathrectangle{\pgfqpoint{0.800000in}{0.528000in}}{\pgfqpoint{4.960000in}{1.680000in}} %
\pgfusepath{clip}%
\pgfsetrectcap%
\pgfsetroundjoin%
\pgfsetlinewidth{0.803000pt}%
\definecolor{currentstroke}{rgb}{0.690196,0.690196,0.690196}%
\pgfsetstrokecolor{currentstroke}%
\pgfsetdash{}{0pt}%
\pgfpathmoveto{\pgfqpoint{5.016311in}{0.528000in}}%
\pgfpathlineto{\pgfqpoint{5.016311in}{2.208000in}}%
\pgfusepath{stroke}%
\end{pgfscope}%
\begin{pgfscope}%
\pgfsetbuttcap%
\pgfsetroundjoin%
\definecolor{currentfill}{rgb}{0.000000,0.000000,0.000000}%
\pgfsetfillcolor{currentfill}%
\pgfsetlinewidth{0.803000pt}%
\definecolor{currentstroke}{rgb}{0.000000,0.000000,0.000000}%
\pgfsetstrokecolor{currentstroke}%
\pgfsetdash{}{0pt}%
\pgfsys@defobject{currentmarker}{\pgfqpoint{0.000000in}{-0.048611in}}{\pgfqpoint{0.000000in}{0.000000in}}{%
\pgfpathmoveto{\pgfqpoint{0.000000in}{0.000000in}}%
\pgfpathlineto{\pgfqpoint{0.000000in}{-0.048611in}}%
\pgfusepath{stroke,fill}%
}%
\begin{pgfscope}%
\pgfsys@transformshift{5.016311in}{0.528000in}%
\pgfsys@useobject{currentmarker}{}%
\end{pgfscope}%
\end{pgfscope}%
\begin{pgfscope}%
\pgftext[x=5.016311in,y=0.430778in,,top]{\sffamily\fontsize{10.000000}{12.000000}\selectfont 75}%
\end{pgfscope}%
\begin{pgfscope}%
\pgfpathrectangle{\pgfqpoint{0.800000in}{0.528000in}}{\pgfqpoint{4.960000in}{1.680000in}} %
\pgfusepath{clip}%
\pgfsetrectcap%
\pgfsetroundjoin%
\pgfsetlinewidth{0.803000pt}%
\definecolor{currentstroke}{rgb}{0.690196,0.690196,0.690196}%
\pgfsetstrokecolor{currentstroke}%
\pgfsetdash{}{0pt}%
\pgfpathmoveto{\pgfqpoint{5.594696in}{0.528000in}}%
\pgfpathlineto{\pgfqpoint{5.594696in}{2.208000in}}%
\pgfusepath{stroke}%
\end{pgfscope}%
\begin{pgfscope}%
\pgfsetbuttcap%
\pgfsetroundjoin%
\definecolor{currentfill}{rgb}{0.000000,0.000000,0.000000}%
\pgfsetfillcolor{currentfill}%
\pgfsetlinewidth{0.803000pt}%
\definecolor{currentstroke}{rgb}{0.000000,0.000000,0.000000}%
\pgfsetstrokecolor{currentstroke}%
\pgfsetdash{}{0pt}%
\pgfsys@defobject{currentmarker}{\pgfqpoint{0.000000in}{-0.048611in}}{\pgfqpoint{0.000000in}{0.000000in}}{%
\pgfpathmoveto{\pgfqpoint{0.000000in}{0.000000in}}%
\pgfpathlineto{\pgfqpoint{0.000000in}{-0.048611in}}%
\pgfusepath{stroke,fill}%
}%
\begin{pgfscope}%
\pgfsys@transformshift{5.594696in}{0.528000in}%
\pgfsys@useobject{currentmarker}{}%
\end{pgfscope}%
\end{pgfscope}%
\begin{pgfscope}%
\pgftext[x=5.594696in,y=0.430778in,,top]{\sffamily\fontsize{10.000000}{12.000000}\selectfont 100}%
\end{pgfscope}%
\begin{pgfscope}%
\pgftext[x=3.280000in,y=0.240809in,,top]{\sffamily\fontsize{10.000000}{12.000000}\selectfont Frequency [MHz]}%
\end{pgfscope}%
\begin{pgfscope}%
\pgfpathrectangle{\pgfqpoint{0.800000in}{0.528000in}}{\pgfqpoint{4.960000in}{1.680000in}} %
\pgfusepath{clip}%
\pgfsetrectcap%
\pgfsetroundjoin%
\pgfsetlinewidth{0.803000pt}%
\definecolor{currentstroke}{rgb}{0.690196,0.690196,0.690196}%
\pgfsetstrokecolor{currentstroke}%
\pgfsetdash{}{0pt}%
\pgfpathmoveto{\pgfqpoint{0.800000in}{0.591331in}}%
\pgfpathlineto{\pgfqpoint{5.760000in}{0.591331in}}%
\pgfusepath{stroke}%
\end{pgfscope}%
\begin{pgfscope}%
\pgfsetbuttcap%
\pgfsetroundjoin%
\definecolor{currentfill}{rgb}{0.000000,0.000000,0.000000}%
\pgfsetfillcolor{currentfill}%
\pgfsetlinewidth{0.803000pt}%
\definecolor{currentstroke}{rgb}{0.000000,0.000000,0.000000}%
\pgfsetstrokecolor{currentstroke}%
\pgfsetdash{}{0pt}%
\pgfsys@defobject{currentmarker}{\pgfqpoint{-0.048611in}{0.000000in}}{\pgfqpoint{0.000000in}{0.000000in}}{%
\pgfpathmoveto{\pgfqpoint{0.000000in}{0.000000in}}%
\pgfpathlineto{\pgfqpoint{-0.048611in}{0.000000in}}%
\pgfusepath{stroke,fill}%
}%
\begin{pgfscope}%
\pgfsys@transformshift{0.800000in}{0.591331in}%
\pgfsys@useobject{currentmarker}{}%
\end{pgfscope}%
\end{pgfscope}%
\begin{pgfscope}%
\pgftext[x=0.481898in,y=0.538569in,left,base]{\sffamily\fontsize{10.000000}{12.000000}\selectfont 0.0}%
\end{pgfscope}%
\begin{pgfscope}%
\pgfpathrectangle{\pgfqpoint{0.800000in}{0.528000in}}{\pgfqpoint{4.960000in}{1.680000in}} %
\pgfusepath{clip}%
\pgfsetrectcap%
\pgfsetroundjoin%
\pgfsetlinewidth{0.803000pt}%
\definecolor{currentstroke}{rgb}{0.690196,0.690196,0.690196}%
\pgfsetstrokecolor{currentstroke}%
\pgfsetdash{}{0pt}%
\pgfpathmoveto{\pgfqpoint{0.800000in}{0.899392in}}%
\pgfpathlineto{\pgfqpoint{5.760000in}{0.899392in}}%
\pgfusepath{stroke}%
\end{pgfscope}%
\begin{pgfscope}%
\pgfsetbuttcap%
\pgfsetroundjoin%
\definecolor{currentfill}{rgb}{0.000000,0.000000,0.000000}%
\pgfsetfillcolor{currentfill}%
\pgfsetlinewidth{0.803000pt}%
\definecolor{currentstroke}{rgb}{0.000000,0.000000,0.000000}%
\pgfsetstrokecolor{currentstroke}%
\pgfsetdash{}{0pt}%
\pgfsys@defobject{currentmarker}{\pgfqpoint{-0.048611in}{0.000000in}}{\pgfqpoint{0.000000in}{0.000000in}}{%
\pgfpathmoveto{\pgfqpoint{0.000000in}{0.000000in}}%
\pgfpathlineto{\pgfqpoint{-0.048611in}{0.000000in}}%
\pgfusepath{stroke,fill}%
}%
\begin{pgfscope}%
\pgfsys@transformshift{0.800000in}{0.899392in}%
\pgfsys@useobject{currentmarker}{}%
\end{pgfscope}%
\end{pgfscope}%
\begin{pgfscope}%
\pgftext[x=0.481898in,y=0.846630in,left,base]{\sffamily\fontsize{10.000000}{12.000000}\selectfont 0.2}%
\end{pgfscope}%
\begin{pgfscope}%
\pgfpathrectangle{\pgfqpoint{0.800000in}{0.528000in}}{\pgfqpoint{4.960000in}{1.680000in}} %
\pgfusepath{clip}%
\pgfsetrectcap%
\pgfsetroundjoin%
\pgfsetlinewidth{0.803000pt}%
\definecolor{currentstroke}{rgb}{0.690196,0.690196,0.690196}%
\pgfsetstrokecolor{currentstroke}%
\pgfsetdash{}{0pt}%
\pgfpathmoveto{\pgfqpoint{0.800000in}{1.207453in}}%
\pgfpathlineto{\pgfqpoint{5.760000in}{1.207453in}}%
\pgfusepath{stroke}%
\end{pgfscope}%
\begin{pgfscope}%
\pgfsetbuttcap%
\pgfsetroundjoin%
\definecolor{currentfill}{rgb}{0.000000,0.000000,0.000000}%
\pgfsetfillcolor{currentfill}%
\pgfsetlinewidth{0.803000pt}%
\definecolor{currentstroke}{rgb}{0.000000,0.000000,0.000000}%
\pgfsetstrokecolor{currentstroke}%
\pgfsetdash{}{0pt}%
\pgfsys@defobject{currentmarker}{\pgfqpoint{-0.048611in}{0.000000in}}{\pgfqpoint{0.000000in}{0.000000in}}{%
\pgfpathmoveto{\pgfqpoint{0.000000in}{0.000000in}}%
\pgfpathlineto{\pgfqpoint{-0.048611in}{0.000000in}}%
\pgfusepath{stroke,fill}%
}%
\begin{pgfscope}%
\pgfsys@transformshift{0.800000in}{1.207453in}%
\pgfsys@useobject{currentmarker}{}%
\end{pgfscope}%
\end{pgfscope}%
\begin{pgfscope}%
\pgftext[x=0.481898in,y=1.154691in,left,base]{\sffamily\fontsize{10.000000}{12.000000}\selectfont 0.4}%
\end{pgfscope}%
\begin{pgfscope}%
\pgfpathrectangle{\pgfqpoint{0.800000in}{0.528000in}}{\pgfqpoint{4.960000in}{1.680000in}} %
\pgfusepath{clip}%
\pgfsetrectcap%
\pgfsetroundjoin%
\pgfsetlinewidth{0.803000pt}%
\definecolor{currentstroke}{rgb}{0.690196,0.690196,0.690196}%
\pgfsetstrokecolor{currentstroke}%
\pgfsetdash{}{0pt}%
\pgfpathmoveto{\pgfqpoint{0.800000in}{1.515514in}}%
\pgfpathlineto{\pgfqpoint{5.760000in}{1.515514in}}%
\pgfusepath{stroke}%
\end{pgfscope}%
\begin{pgfscope}%
\pgfsetbuttcap%
\pgfsetroundjoin%
\definecolor{currentfill}{rgb}{0.000000,0.000000,0.000000}%
\pgfsetfillcolor{currentfill}%
\pgfsetlinewidth{0.803000pt}%
\definecolor{currentstroke}{rgb}{0.000000,0.000000,0.000000}%
\pgfsetstrokecolor{currentstroke}%
\pgfsetdash{}{0pt}%
\pgfsys@defobject{currentmarker}{\pgfqpoint{-0.048611in}{0.000000in}}{\pgfqpoint{0.000000in}{0.000000in}}{%
\pgfpathmoveto{\pgfqpoint{0.000000in}{0.000000in}}%
\pgfpathlineto{\pgfqpoint{-0.048611in}{0.000000in}}%
\pgfusepath{stroke,fill}%
}%
\begin{pgfscope}%
\pgfsys@transformshift{0.800000in}{1.515514in}%
\pgfsys@useobject{currentmarker}{}%
\end{pgfscope}%
\end{pgfscope}%
\begin{pgfscope}%
\pgftext[x=0.481898in,y=1.462753in,left,base]{\sffamily\fontsize{10.000000}{12.000000}\selectfont 0.6}%
\end{pgfscope}%
\begin{pgfscope}%
\pgfpathrectangle{\pgfqpoint{0.800000in}{0.528000in}}{\pgfqpoint{4.960000in}{1.680000in}} %
\pgfusepath{clip}%
\pgfsetrectcap%
\pgfsetroundjoin%
\pgfsetlinewidth{0.803000pt}%
\definecolor{currentstroke}{rgb}{0.690196,0.690196,0.690196}%
\pgfsetstrokecolor{currentstroke}%
\pgfsetdash{}{0pt}%
\pgfpathmoveto{\pgfqpoint{0.800000in}{1.823575in}}%
\pgfpathlineto{\pgfqpoint{5.760000in}{1.823575in}}%
\pgfusepath{stroke}%
\end{pgfscope}%
\begin{pgfscope}%
\pgfsetbuttcap%
\pgfsetroundjoin%
\definecolor{currentfill}{rgb}{0.000000,0.000000,0.000000}%
\pgfsetfillcolor{currentfill}%
\pgfsetlinewidth{0.803000pt}%
\definecolor{currentstroke}{rgb}{0.000000,0.000000,0.000000}%
\pgfsetstrokecolor{currentstroke}%
\pgfsetdash{}{0pt}%
\pgfsys@defobject{currentmarker}{\pgfqpoint{-0.048611in}{0.000000in}}{\pgfqpoint{0.000000in}{0.000000in}}{%
\pgfpathmoveto{\pgfqpoint{0.000000in}{0.000000in}}%
\pgfpathlineto{\pgfqpoint{-0.048611in}{0.000000in}}%
\pgfusepath{stroke,fill}%
}%
\begin{pgfscope}%
\pgfsys@transformshift{0.800000in}{1.823575in}%
\pgfsys@useobject{currentmarker}{}%
\end{pgfscope}%
\end{pgfscope}%
\begin{pgfscope}%
\pgftext[x=0.481898in,y=1.770814in,left,base]{\sffamily\fontsize{10.000000}{12.000000}\selectfont 0.8}%
\end{pgfscope}%
\begin{pgfscope}%
\pgfpathrectangle{\pgfqpoint{0.800000in}{0.528000in}}{\pgfqpoint{4.960000in}{1.680000in}} %
\pgfusepath{clip}%
\pgfsetrectcap%
\pgfsetroundjoin%
\pgfsetlinewidth{0.803000pt}%
\definecolor{currentstroke}{rgb}{0.690196,0.690196,0.690196}%
\pgfsetstrokecolor{currentstroke}%
\pgfsetdash{}{0pt}%
\pgfpathmoveto{\pgfqpoint{0.800000in}{2.131636in}}%
\pgfpathlineto{\pgfqpoint{5.760000in}{2.131636in}}%
\pgfusepath{stroke}%
\end{pgfscope}%
\begin{pgfscope}%
\pgfsetbuttcap%
\pgfsetroundjoin%
\definecolor{currentfill}{rgb}{0.000000,0.000000,0.000000}%
\pgfsetfillcolor{currentfill}%
\pgfsetlinewidth{0.803000pt}%
\definecolor{currentstroke}{rgb}{0.000000,0.000000,0.000000}%
\pgfsetstrokecolor{currentstroke}%
\pgfsetdash{}{0pt}%
\pgfsys@defobject{currentmarker}{\pgfqpoint{-0.048611in}{0.000000in}}{\pgfqpoint{0.000000in}{0.000000in}}{%
\pgfpathmoveto{\pgfqpoint{0.000000in}{0.000000in}}%
\pgfpathlineto{\pgfqpoint{-0.048611in}{0.000000in}}%
\pgfusepath{stroke,fill}%
}%
\begin{pgfscope}%
\pgfsys@transformshift{0.800000in}{2.131636in}%
\pgfsys@useobject{currentmarker}{}%
\end{pgfscope}%
\end{pgfscope}%
\begin{pgfscope}%
\pgftext[x=0.481898in,y=2.078875in,left,base]{\sffamily\fontsize{10.000000}{12.000000}\selectfont 1.0}%
\end{pgfscope}%
\begin{pgfscope}%
\pgftext[x=0.426343in,y=1.368000in,,bottom,rotate=90.000000]{\sffamily\fontsize{10.000000}{12.000000}\selectfont Magnitude of the Fourier transform.}%
\end{pgfscope}%
\begin{pgfscope}%
\pgfpathrectangle{\pgfqpoint{0.800000in}{0.528000in}}{\pgfqpoint{4.960000in}{1.680000in}} %
\pgfusepath{clip}%
\pgfsetrectcap%
\pgfsetroundjoin%
\pgfsetlinewidth{1.505625pt}%
\definecolor{currentstroke}{rgb}{0.121569,0.466667,0.705882}%
\pgfsetstrokecolor{currentstroke}%
\pgfsetdash{}{0pt}%
\pgfpathmoveto{\pgfqpoint{1.025455in}{0.605016in}}%
\pgfpathlineto{\pgfqpoint{1.027767in}{0.618714in}}%
\pgfpathlineto{\pgfqpoint{1.030079in}{0.605054in}}%
\pgfpathlineto{\pgfqpoint{1.032392in}{0.604978in}}%
\pgfpathlineto{\pgfqpoint{1.034704in}{0.618715in}}%
\pgfpathlineto{\pgfqpoint{1.037016in}{0.605094in}}%
\pgfpathlineto{\pgfqpoint{1.039329in}{0.604942in}}%
\pgfpathlineto{\pgfqpoint{1.041641in}{0.618717in}}%
\pgfpathlineto{\pgfqpoint{1.043953in}{0.605134in}}%
\pgfpathlineto{\pgfqpoint{1.046266in}{0.604907in}}%
\pgfpathlineto{\pgfqpoint{1.048578in}{0.618721in}}%
\pgfpathlineto{\pgfqpoint{1.050890in}{0.605176in}}%
\pgfpathlineto{\pgfqpoint{1.053203in}{0.604872in}}%
\pgfpathlineto{\pgfqpoint{1.055515in}{0.618725in}}%
\pgfpathlineto{\pgfqpoint{1.057828in}{0.605218in}}%
\pgfpathlineto{\pgfqpoint{1.060140in}{0.604839in}}%
\pgfpathlineto{\pgfqpoint{1.062452in}{0.618731in}}%
\pgfpathlineto{\pgfqpoint{1.064765in}{0.605262in}}%
\pgfpathlineto{\pgfqpoint{1.067077in}{0.604807in}}%
\pgfpathlineto{\pgfqpoint{1.069389in}{0.618738in}}%
\pgfpathlineto{\pgfqpoint{1.071702in}{0.605307in}}%
\pgfpathlineto{\pgfqpoint{1.074014in}{0.604776in}}%
\pgfpathlineto{\pgfqpoint{1.076326in}{0.618747in}}%
\pgfpathlineto{\pgfqpoint{1.078639in}{0.605352in}}%
\pgfpathlineto{\pgfqpoint{1.080951in}{0.604746in}}%
\pgfpathlineto{\pgfqpoint{1.083263in}{0.618756in}}%
\pgfpathlineto{\pgfqpoint{1.085576in}{0.605399in}}%
\pgfpathlineto{\pgfqpoint{1.087888in}{0.604717in}}%
\pgfpathlineto{\pgfqpoint{1.090200in}{0.618767in}}%
\pgfpathlineto{\pgfqpoint{1.092513in}{0.605447in}}%
\pgfpathlineto{\pgfqpoint{1.094825in}{0.604689in}}%
\pgfpathlineto{\pgfqpoint{1.097138in}{0.618779in}}%
\pgfpathlineto{\pgfqpoint{1.099450in}{0.605496in}}%
\pgfpathlineto{\pgfqpoint{1.101762in}{0.604663in}}%
\pgfpathlineto{\pgfqpoint{1.104075in}{0.618792in}}%
\pgfpathlineto{\pgfqpoint{1.106387in}{0.605546in}}%
\pgfpathlineto{\pgfqpoint{1.108699in}{0.604637in}}%
\pgfpathlineto{\pgfqpoint{1.111012in}{0.618806in}}%
\pgfpathlineto{\pgfqpoint{1.113324in}{0.605597in}}%
\pgfpathlineto{\pgfqpoint{1.115636in}{0.604612in}}%
\pgfpathlineto{\pgfqpoint{1.117949in}{0.618821in}}%
\pgfpathlineto{\pgfqpoint{1.120261in}{0.605649in}}%
\pgfpathlineto{\pgfqpoint{1.122573in}{0.604589in}}%
\pgfpathlineto{\pgfqpoint{1.124886in}{0.618838in}}%
\pgfpathlineto{\pgfqpoint{1.127198in}{0.605702in}}%
\pgfpathlineto{\pgfqpoint{1.129510in}{0.604567in}}%
\pgfpathlineto{\pgfqpoint{1.131823in}{0.618856in}}%
\pgfpathlineto{\pgfqpoint{1.134135in}{0.605756in}}%
\pgfpathlineto{\pgfqpoint{1.136448in}{0.604545in}}%
\pgfpathlineto{\pgfqpoint{1.138760in}{0.618875in}}%
\pgfpathlineto{\pgfqpoint{1.141072in}{0.605812in}}%
\pgfpathlineto{\pgfqpoint{1.143385in}{0.604525in}}%
\pgfpathlineto{\pgfqpoint{1.145697in}{0.618895in}}%
\pgfpathlineto{\pgfqpoint{1.148009in}{0.605868in}}%
\pgfpathlineto{\pgfqpoint{1.150322in}{0.604506in}}%
\pgfpathlineto{\pgfqpoint{1.152634in}{0.618917in}}%
\pgfpathlineto{\pgfqpoint{1.154946in}{0.605925in}}%
\pgfpathlineto{\pgfqpoint{1.157259in}{0.604488in}}%
\pgfpathlineto{\pgfqpoint{1.159571in}{0.618940in}}%
\pgfpathlineto{\pgfqpoint{1.161883in}{0.605984in}}%
\pgfpathlineto{\pgfqpoint{1.164196in}{0.604472in}}%
\pgfpathlineto{\pgfqpoint{1.166508in}{0.618964in}}%
\pgfpathlineto{\pgfqpoint{1.168821in}{0.606044in}}%
\pgfpathlineto{\pgfqpoint{1.171133in}{0.604456in}}%
\pgfpathlineto{\pgfqpoint{1.173445in}{0.618989in}}%
\pgfpathlineto{\pgfqpoint{1.175758in}{0.606104in}}%
\pgfpathlineto{\pgfqpoint{1.178070in}{0.604442in}}%
\pgfpathlineto{\pgfqpoint{1.180382in}{0.619015in}}%
\pgfpathlineto{\pgfqpoint{1.182695in}{0.606166in}}%
\pgfpathlineto{\pgfqpoint{1.185007in}{0.604428in}}%
\pgfpathlineto{\pgfqpoint{1.187319in}{0.619043in}}%
\pgfpathlineto{\pgfqpoint{1.189632in}{0.606229in}}%
\pgfpathlineto{\pgfqpoint{1.191944in}{0.604416in}}%
\pgfpathlineto{\pgfqpoint{1.194256in}{0.619072in}}%
\pgfpathlineto{\pgfqpoint{1.196569in}{0.606293in}}%
\pgfpathlineto{\pgfqpoint{1.198881in}{0.604406in}}%
\pgfpathlineto{\pgfqpoint{1.201193in}{0.619103in}}%
\pgfpathlineto{\pgfqpoint{1.203506in}{0.606359in}}%
\pgfpathlineto{\pgfqpoint{1.205818in}{0.604396in}}%
\pgfpathlineto{\pgfqpoint{1.208131in}{0.619134in}}%
\pgfpathlineto{\pgfqpoint{1.210443in}{0.606425in}}%
\pgfpathlineto{\pgfqpoint{1.212755in}{0.604388in}}%
\pgfpathlineto{\pgfqpoint{1.215068in}{0.619167in}}%
\pgfpathlineto{\pgfqpoint{1.217380in}{0.606493in}}%
\pgfpathlineto{\pgfqpoint{1.219692in}{0.604380in}}%
\pgfpathlineto{\pgfqpoint{1.222005in}{0.619202in}}%
\pgfpathlineto{\pgfqpoint{1.224317in}{0.606562in}}%
\pgfpathlineto{\pgfqpoint{1.226629in}{0.604374in}}%
\pgfpathlineto{\pgfqpoint{1.228942in}{0.619237in}}%
\pgfpathlineto{\pgfqpoint{1.231254in}{0.606632in}}%
\pgfpathlineto{\pgfqpoint{1.233566in}{0.604370in}}%
\pgfpathlineto{\pgfqpoint{1.235879in}{0.619274in}}%
\pgfpathlineto{\pgfqpoint{1.238191in}{0.606703in}}%
\pgfpathlineto{\pgfqpoint{1.240503in}{0.604366in}}%
\pgfpathlineto{\pgfqpoint{1.242816in}{0.619313in}}%
\pgfpathlineto{\pgfqpoint{1.245128in}{0.606775in}}%
\pgfpathlineto{\pgfqpoint{1.247441in}{0.604364in}}%
\pgfpathlineto{\pgfqpoint{1.249753in}{0.619352in}}%
\pgfpathlineto{\pgfqpoint{1.252065in}{0.606849in}}%
\pgfpathlineto{\pgfqpoint{1.254378in}{0.604364in}}%
\pgfpathlineto{\pgfqpoint{1.256690in}{0.619393in}}%
\pgfpathlineto{\pgfqpoint{1.259002in}{0.606923in}}%
\pgfpathlineto{\pgfqpoint{1.261315in}{0.604364in}}%
\pgfpathlineto{\pgfqpoint{1.263627in}{0.619436in}}%
\pgfpathlineto{\pgfqpoint{1.265939in}{0.606999in}}%
\pgfpathlineto{\pgfqpoint{1.268252in}{0.604366in}}%
\pgfpathlineto{\pgfqpoint{1.270564in}{0.619480in}}%
\pgfpathlineto{\pgfqpoint{1.272876in}{0.607077in}}%
\pgfpathlineto{\pgfqpoint{1.275189in}{0.604369in}}%
\pgfpathlineto{\pgfqpoint{1.277501in}{0.619525in}}%
\pgfpathlineto{\pgfqpoint{1.279814in}{0.607155in}}%
\pgfpathlineto{\pgfqpoint{1.282126in}{0.604374in}}%
\pgfpathlineto{\pgfqpoint{1.284438in}{0.619572in}}%
\pgfpathlineto{\pgfqpoint{1.286751in}{0.607235in}}%
\pgfpathlineto{\pgfqpoint{1.289063in}{0.604380in}}%
\pgfpathlineto{\pgfqpoint{1.291375in}{0.619620in}}%
\pgfpathlineto{\pgfqpoint{1.293688in}{0.607316in}}%
\pgfpathlineto{\pgfqpoint{1.296000in}{0.604388in}}%
\pgfpathlineto{\pgfqpoint{1.298312in}{0.619669in}}%
\pgfpathlineto{\pgfqpoint{1.300625in}{0.607399in}}%
\pgfpathlineto{\pgfqpoint{1.302937in}{0.604396in}}%
\pgfpathlineto{\pgfqpoint{1.305249in}{0.619720in}}%
\pgfpathlineto{\pgfqpoint{1.307562in}{0.607482in}}%
\pgfpathlineto{\pgfqpoint{1.309874in}{0.604407in}}%
\pgfpathlineto{\pgfqpoint{1.312186in}{0.619773in}}%
\pgfpathlineto{\pgfqpoint{1.314499in}{0.607568in}}%
\pgfpathlineto{\pgfqpoint{1.316811in}{0.604419in}}%
\pgfpathlineto{\pgfqpoint{1.319124in}{0.619827in}}%
\pgfpathlineto{\pgfqpoint{1.321436in}{0.607654in}}%
\pgfpathlineto{\pgfqpoint{1.323748in}{0.604432in}}%
\pgfpathlineto{\pgfqpoint{1.326061in}{0.619883in}}%
\pgfpathlineto{\pgfqpoint{1.328373in}{0.607742in}}%
\pgfpathlineto{\pgfqpoint{1.330685in}{0.604447in}}%
\pgfpathlineto{\pgfqpoint{1.332998in}{0.619940in}}%
\pgfpathlineto{\pgfqpoint{1.335310in}{0.607831in}}%
\pgfpathlineto{\pgfqpoint{1.337622in}{0.604463in}}%
\pgfpathlineto{\pgfqpoint{1.339935in}{0.619999in}}%
\pgfpathlineto{\pgfqpoint{1.342247in}{0.607922in}}%
\pgfpathlineto{\pgfqpoint{1.344559in}{0.604481in}}%
\pgfpathlineto{\pgfqpoint{1.346872in}{0.620059in}}%
\pgfpathlineto{\pgfqpoint{1.349184in}{0.608014in}}%
\pgfpathlineto{\pgfqpoint{1.351497in}{0.604501in}}%
\pgfpathlineto{\pgfqpoint{1.353809in}{0.620121in}}%
\pgfpathlineto{\pgfqpoint{1.356121in}{0.608108in}}%
\pgfpathlineto{\pgfqpoint{1.358434in}{0.604522in}}%
\pgfpathlineto{\pgfqpoint{1.360746in}{0.620184in}}%
\pgfpathlineto{\pgfqpoint{1.363058in}{0.608203in}}%
\pgfpathlineto{\pgfqpoint{1.365371in}{0.604545in}}%
\pgfpathlineto{\pgfqpoint{1.367683in}{0.620250in}}%
\pgfpathlineto{\pgfqpoint{1.369995in}{0.608300in}}%
\pgfpathlineto{\pgfqpoint{1.372308in}{0.604569in}}%
\pgfpathlineto{\pgfqpoint{1.374620in}{0.620316in}}%
\pgfpathlineto{\pgfqpoint{1.376932in}{0.608398in}}%
\pgfpathlineto{\pgfqpoint{1.379245in}{0.604596in}}%
\pgfpathlineto{\pgfqpoint{1.381557in}{0.620385in}}%
\pgfpathlineto{\pgfqpoint{1.383869in}{0.608498in}}%
\pgfpathlineto{\pgfqpoint{1.386182in}{0.604623in}}%
\pgfpathlineto{\pgfqpoint{1.388494in}{0.620455in}}%
\pgfpathlineto{\pgfqpoint{1.390807in}{0.608599in}}%
\pgfpathlineto{\pgfqpoint{1.393119in}{0.604653in}}%
\pgfpathlineto{\pgfqpoint{1.395431in}{0.620527in}}%
\pgfpathlineto{\pgfqpoint{1.397744in}{0.608702in}}%
\pgfpathlineto{\pgfqpoint{1.400056in}{0.604685in}}%
\pgfpathlineto{\pgfqpoint{1.402368in}{0.620601in}}%
\pgfpathlineto{\pgfqpoint{1.404681in}{0.608807in}}%
\pgfpathlineto{\pgfqpoint{1.406993in}{0.604718in}}%
\pgfpathlineto{\pgfqpoint{1.409305in}{0.620677in}}%
\pgfpathlineto{\pgfqpoint{1.411618in}{0.608913in}}%
\pgfpathlineto{\pgfqpoint{1.413930in}{0.604753in}}%
\pgfpathlineto{\pgfqpoint{1.416242in}{0.620754in}}%
\pgfpathlineto{\pgfqpoint{1.418555in}{0.609021in}}%
\pgfpathlineto{\pgfqpoint{1.420867in}{0.604790in}}%
\pgfpathlineto{\pgfqpoint{1.423179in}{0.620834in}}%
\pgfpathlineto{\pgfqpoint{1.425492in}{0.609131in}}%
\pgfpathlineto{\pgfqpoint{1.427804in}{0.604829in}}%
\pgfpathlineto{\pgfqpoint{1.430117in}{0.620915in}}%
\pgfpathlineto{\pgfqpoint{1.432429in}{0.609243in}}%
\pgfpathlineto{\pgfqpoint{1.434741in}{0.604870in}}%
\pgfpathlineto{\pgfqpoint{1.437054in}{0.620998in}}%
\pgfpathlineto{\pgfqpoint{1.439366in}{0.609356in}}%
\pgfpathlineto{\pgfqpoint{1.441678in}{0.604913in}}%
\pgfpathlineto{\pgfqpoint{1.443991in}{0.621083in}}%
\pgfpathlineto{\pgfqpoint{1.446303in}{0.609471in}}%
\pgfpathlineto{\pgfqpoint{1.448615in}{0.604958in}}%
\pgfpathlineto{\pgfqpoint{1.450928in}{0.621170in}}%
\pgfpathlineto{\pgfqpoint{1.453240in}{0.609588in}}%
\pgfpathlineto{\pgfqpoint{1.455552in}{0.605004in}}%
\pgfpathlineto{\pgfqpoint{1.457865in}{0.621259in}}%
\pgfpathlineto{\pgfqpoint{1.460177in}{0.609707in}}%
\pgfpathlineto{\pgfqpoint{1.462490in}{0.605054in}}%
\pgfpathlineto{\pgfqpoint{1.464802in}{0.621350in}}%
\pgfpathlineto{\pgfqpoint{1.467114in}{0.609828in}}%
\pgfpathlineto{\pgfqpoint{1.469427in}{0.605105in}}%
\pgfpathlineto{\pgfqpoint{1.471739in}{0.621444in}}%
\pgfpathlineto{\pgfqpoint{1.474051in}{0.609950in}}%
\pgfpathlineto{\pgfqpoint{1.476364in}{0.605158in}}%
\pgfpathlineto{\pgfqpoint{1.478676in}{0.621539in}}%
\pgfpathlineto{\pgfqpoint{1.480988in}{0.610075in}}%
\pgfpathlineto{\pgfqpoint{1.483301in}{0.605214in}}%
\pgfpathlineto{\pgfqpoint{1.485613in}{0.621636in}}%
\pgfpathlineto{\pgfqpoint{1.487925in}{0.610202in}}%
\pgfpathlineto{\pgfqpoint{1.490238in}{0.605271in}}%
\pgfpathlineto{\pgfqpoint{1.492550in}{0.621736in}}%
\pgfpathlineto{\pgfqpoint{1.494862in}{0.610331in}}%
\pgfpathlineto{\pgfqpoint{1.497175in}{0.605332in}}%
\pgfpathlineto{\pgfqpoint{1.499487in}{0.621838in}}%
\pgfpathlineto{\pgfqpoint{1.501800in}{0.610462in}}%
\pgfpathlineto{\pgfqpoint{1.504112in}{0.605394in}}%
\pgfpathlineto{\pgfqpoint{1.506424in}{0.621943in}}%
\pgfpathlineto{\pgfqpoint{1.508737in}{0.610595in}}%
\pgfpathlineto{\pgfqpoint{1.511049in}{0.605459in}}%
\pgfpathlineto{\pgfqpoint{1.513361in}{0.622049in}}%
\pgfpathlineto{\pgfqpoint{1.515674in}{0.610731in}}%
\pgfpathlineto{\pgfqpoint{1.517986in}{0.605526in}}%
\pgfpathlineto{\pgfqpoint{1.520298in}{0.622158in}}%
\pgfpathlineto{\pgfqpoint{1.522611in}{0.610869in}}%
\pgfpathlineto{\pgfqpoint{1.524923in}{0.605596in}}%
\pgfpathlineto{\pgfqpoint{1.527235in}{0.622270in}}%
\pgfpathlineto{\pgfqpoint{1.529548in}{0.611009in}}%
\pgfpathlineto{\pgfqpoint{1.531860in}{0.605669in}}%
\pgfpathlineto{\pgfqpoint{1.534172in}{0.622384in}}%
\pgfpathlineto{\pgfqpoint{1.536485in}{0.611151in}}%
\pgfpathlineto{\pgfqpoint{1.538797in}{0.605744in}}%
\pgfpathlineto{\pgfqpoint{1.541110in}{0.622500in}}%
\pgfpathlineto{\pgfqpoint{1.543422in}{0.611296in}}%
\pgfpathlineto{\pgfqpoint{1.545734in}{0.605821in}}%
\pgfpathlineto{\pgfqpoint{1.548047in}{0.622619in}}%
\pgfpathlineto{\pgfqpoint{1.550359in}{0.611443in}}%
\pgfpathlineto{\pgfqpoint{1.552671in}{0.605902in}}%
\pgfpathlineto{\pgfqpoint{1.554984in}{0.622741in}}%
\pgfpathlineto{\pgfqpoint{1.557296in}{0.611593in}}%
\pgfpathlineto{\pgfqpoint{1.559608in}{0.605985in}}%
\pgfpathlineto{\pgfqpoint{1.561921in}{0.622865in}}%
\pgfpathlineto{\pgfqpoint{1.566545in}{0.606071in}}%
\pgfpathlineto{\pgfqpoint{1.568858in}{0.622992in}}%
\pgfpathlineto{\pgfqpoint{1.573483in}{0.606160in}}%
\pgfpathlineto{\pgfqpoint{1.575795in}{0.623122in}}%
\pgfpathlineto{\pgfqpoint{1.580420in}{0.606252in}}%
\pgfpathlineto{\pgfqpoint{1.582732in}{0.623255in}}%
\pgfpathlineto{\pgfqpoint{1.587357in}{0.606347in}}%
\pgfpathlineto{\pgfqpoint{1.589669in}{0.623391in}}%
\pgfpathlineto{\pgfqpoint{1.594294in}{0.606445in}}%
\pgfpathlineto{\pgfqpoint{1.596606in}{0.623530in}}%
\pgfpathlineto{\pgfqpoint{1.601231in}{0.606546in}}%
\pgfpathlineto{\pgfqpoint{1.603543in}{0.623672in}}%
\pgfpathlineto{\pgfqpoint{1.608168in}{0.606651in}}%
\pgfpathlineto{\pgfqpoint{1.610480in}{0.623817in}}%
\pgfpathlineto{\pgfqpoint{1.615105in}{0.606759in}}%
\pgfpathlineto{\pgfqpoint{1.617417in}{0.623965in}}%
\pgfpathlineto{\pgfqpoint{1.622042in}{0.606870in}}%
\pgfpathlineto{\pgfqpoint{1.624354in}{0.624116in}}%
\pgfpathlineto{\pgfqpoint{1.628979in}{0.606985in}}%
\pgfpathlineto{\pgfqpoint{1.631291in}{0.624271in}}%
\pgfpathlineto{\pgfqpoint{1.635916in}{0.607103in}}%
\pgfpathlineto{\pgfqpoint{1.638228in}{0.624430in}}%
\pgfpathlineto{\pgfqpoint{1.642853in}{0.607225in}}%
\pgfpathlineto{\pgfqpoint{1.645166in}{0.624592in}}%
\pgfpathlineto{\pgfqpoint{1.649790in}{0.607351in}}%
\pgfpathlineto{\pgfqpoint{1.652103in}{0.624757in}}%
\pgfpathlineto{\pgfqpoint{1.656727in}{0.607481in}}%
\pgfpathlineto{\pgfqpoint{1.659040in}{0.624927in}}%
\pgfpathlineto{\pgfqpoint{1.663664in}{0.607614in}}%
\pgfpathlineto{\pgfqpoint{1.665977in}{0.625100in}}%
\pgfpathlineto{\pgfqpoint{1.670601in}{0.607752in}}%
\pgfpathlineto{\pgfqpoint{1.672914in}{0.625277in}}%
\pgfpathlineto{\pgfqpoint{1.677538in}{0.607893in}}%
\pgfpathlineto{\pgfqpoint{1.679851in}{0.625458in}}%
\pgfpathlineto{\pgfqpoint{1.684476in}{0.608039in}}%
\pgfpathlineto{\pgfqpoint{1.686788in}{0.625643in}}%
\pgfpathlineto{\pgfqpoint{1.691413in}{0.608190in}}%
\pgfpathlineto{\pgfqpoint{1.693725in}{0.625833in}}%
\pgfpathlineto{\pgfqpoint{1.698350in}{0.608345in}}%
\pgfpathlineto{\pgfqpoint{1.700662in}{0.626027in}}%
\pgfpathlineto{\pgfqpoint{1.705287in}{0.608504in}}%
\pgfpathlineto{\pgfqpoint{1.707599in}{0.626225in}}%
\pgfpathlineto{\pgfqpoint{1.712224in}{0.608669in}}%
\pgfpathlineto{\pgfqpoint{1.714536in}{0.626428in}}%
\pgfpathlineto{\pgfqpoint{1.719161in}{0.608838in}}%
\pgfpathlineto{\pgfqpoint{1.721473in}{0.626636in}}%
\pgfpathlineto{\pgfqpoint{1.726098in}{0.609012in}}%
\pgfpathlineto{\pgfqpoint{1.728410in}{0.626849in}}%
\pgfpathlineto{\pgfqpoint{1.733035in}{0.609191in}}%
\pgfpathlineto{\pgfqpoint{1.735347in}{0.627066in}}%
\pgfpathlineto{\pgfqpoint{1.739972in}{0.609376in}}%
\pgfpathlineto{\pgfqpoint{1.742284in}{0.627289in}}%
\pgfpathlineto{\pgfqpoint{1.746909in}{0.609566in}}%
\pgfpathlineto{\pgfqpoint{1.749221in}{0.627517in}}%
\pgfpathlineto{\pgfqpoint{1.753846in}{0.609761in}}%
\pgfpathlineto{\pgfqpoint{1.756159in}{0.627751in}}%
\pgfpathlineto{\pgfqpoint{1.760783in}{0.609963in}}%
\pgfpathlineto{\pgfqpoint{1.763096in}{0.627991in}}%
\pgfpathlineto{\pgfqpoint{1.767720in}{0.610170in}}%
\pgfpathlineto{\pgfqpoint{1.770033in}{0.628236in}}%
\pgfpathlineto{\pgfqpoint{1.774657in}{0.610384in}}%
\pgfpathlineto{\pgfqpoint{1.776970in}{0.628487in}}%
\pgfpathlineto{\pgfqpoint{1.781594in}{0.610604in}}%
\pgfpathlineto{\pgfqpoint{1.783907in}{0.628745in}}%
\pgfpathlineto{\pgfqpoint{1.788531in}{0.610830in}}%
\pgfpathlineto{\pgfqpoint{1.790844in}{0.629009in}}%
\pgfpathlineto{\pgfqpoint{1.795469in}{0.611063in}}%
\pgfpathlineto{\pgfqpoint{1.797781in}{0.629279in}}%
\pgfpathlineto{\pgfqpoint{1.802406in}{0.611303in}}%
\pgfpathlineto{\pgfqpoint{1.804718in}{0.629557in}}%
\pgfpathlineto{\pgfqpoint{1.809343in}{0.611551in}}%
\pgfpathlineto{\pgfqpoint{1.811655in}{0.629841in}}%
\pgfpathlineto{\pgfqpoint{1.816280in}{0.611805in}}%
\pgfpathlineto{\pgfqpoint{1.818592in}{0.630133in}}%
\pgfpathlineto{\pgfqpoint{1.823217in}{0.612068in}}%
\pgfpathlineto{\pgfqpoint{1.825529in}{0.630433in}}%
\pgfpathlineto{\pgfqpoint{1.830154in}{0.612338in}}%
\pgfpathlineto{\pgfqpoint{1.832466in}{0.630740in}}%
\pgfpathlineto{\pgfqpoint{1.837091in}{0.612617in}}%
\pgfpathlineto{\pgfqpoint{1.839403in}{0.631055in}}%
\pgfpathlineto{\pgfqpoint{1.844028in}{0.612904in}}%
\pgfpathlineto{\pgfqpoint{1.846340in}{0.631379in}}%
\pgfpathlineto{\pgfqpoint{1.850965in}{0.613200in}}%
\pgfpathlineto{\pgfqpoint{1.853277in}{0.631712in}}%
\pgfpathlineto{\pgfqpoint{1.857902in}{0.613505in}}%
\pgfpathlineto{\pgfqpoint{1.860214in}{0.632054in}}%
\pgfpathlineto{\pgfqpoint{1.864839in}{0.613819in}}%
\pgfpathlineto{\pgfqpoint{1.867152in}{0.632405in}}%
\pgfpathlineto{\pgfqpoint{1.871776in}{0.614143in}}%
\pgfpathlineto{\pgfqpoint{1.874089in}{0.632766in}}%
\pgfpathlineto{\pgfqpoint{1.878713in}{0.614478in}}%
\pgfpathlineto{\pgfqpoint{1.881026in}{0.633137in}}%
\pgfpathlineto{\pgfqpoint{1.885650in}{0.614823in}}%
\pgfpathlineto{\pgfqpoint{1.887963in}{0.633518in}}%
\pgfpathlineto{\pgfqpoint{1.892587in}{0.615178in}}%
\pgfpathlineto{\pgfqpoint{1.894900in}{0.633911in}}%
\pgfpathlineto{\pgfqpoint{1.899524in}{0.615546in}}%
\pgfpathlineto{\pgfqpoint{1.901837in}{0.634314in}}%
\pgfpathlineto{\pgfqpoint{1.906462in}{0.615925in}}%
\pgfpathlineto{\pgfqpoint{1.908774in}{0.634730in}}%
\pgfpathlineto{\pgfqpoint{1.913399in}{0.616316in}}%
\pgfpathlineto{\pgfqpoint{1.915711in}{0.635158in}}%
\pgfpathlineto{\pgfqpoint{1.920336in}{0.616721in}}%
\pgfpathlineto{\pgfqpoint{1.922648in}{0.635599in}}%
\pgfpathlineto{\pgfqpoint{1.927273in}{0.617139in}}%
\pgfpathlineto{\pgfqpoint{1.929585in}{0.636053in}}%
\pgfpathlineto{\pgfqpoint{1.934210in}{0.617570in}}%
\pgfpathlineto{\pgfqpoint{1.936522in}{0.636522in}}%
\pgfpathlineto{\pgfqpoint{1.941147in}{0.618017in}}%
\pgfpathlineto{\pgfqpoint{1.943459in}{0.637004in}}%
\pgfpathlineto{\pgfqpoint{1.948084in}{0.618478in}}%
\pgfpathlineto{\pgfqpoint{1.950396in}{0.637503in}}%
\pgfpathlineto{\pgfqpoint{1.955021in}{0.618955in}}%
\pgfpathlineto{\pgfqpoint{1.957333in}{0.638017in}}%
\pgfpathlineto{\pgfqpoint{1.961958in}{0.619449in}}%
\pgfpathlineto{\pgfqpoint{1.964270in}{0.638547in}}%
\pgfpathlineto{\pgfqpoint{1.968895in}{0.619960in}}%
\pgfpathlineto{\pgfqpoint{1.971207in}{0.639096in}}%
\pgfpathlineto{\pgfqpoint{1.975832in}{0.620490in}}%
\pgfpathlineto{\pgfqpoint{1.978145in}{0.639662in}}%
\pgfpathlineto{\pgfqpoint{1.982769in}{0.621038in}}%
\pgfpathlineto{\pgfqpoint{1.985082in}{0.640248in}}%
\pgfpathlineto{\pgfqpoint{1.989706in}{0.621607in}}%
\pgfpathlineto{\pgfqpoint{1.992019in}{0.640854in}}%
\pgfpathlineto{\pgfqpoint{1.996643in}{0.622196in}}%
\pgfpathlineto{\pgfqpoint{1.998956in}{0.641481in}}%
\pgfpathlineto{\pgfqpoint{2.003580in}{0.622807in}}%
\pgfpathlineto{\pgfqpoint{2.005893in}{0.642130in}}%
\pgfpathlineto{\pgfqpoint{2.010517in}{0.623442in}}%
\pgfpathlineto{\pgfqpoint{2.012830in}{0.642803in}}%
\pgfpathlineto{\pgfqpoint{2.017455in}{0.624101in}}%
\pgfpathlineto{\pgfqpoint{2.019767in}{0.643500in}}%
\pgfpathlineto{\pgfqpoint{2.024392in}{0.624785in}}%
\pgfpathlineto{\pgfqpoint{2.026704in}{0.644223in}}%
\pgfpathlineto{\pgfqpoint{2.031329in}{0.625496in}}%
\pgfpathlineto{\pgfqpoint{2.033641in}{0.644973in}}%
\pgfpathlineto{\pgfqpoint{2.038266in}{0.626236in}}%
\pgfpathlineto{\pgfqpoint{2.040578in}{0.645753in}}%
\pgfpathlineto{\pgfqpoint{2.045203in}{0.627006in}}%
\pgfpathlineto{\pgfqpoint{2.047515in}{0.646562in}}%
\pgfpathlineto{\pgfqpoint{2.052140in}{0.627808in}}%
\pgfpathlineto{\pgfqpoint{2.054452in}{0.647404in}}%
\pgfpathlineto{\pgfqpoint{2.059077in}{0.628644in}}%
\pgfpathlineto{\pgfqpoint{2.061389in}{0.648281in}}%
\pgfpathlineto{\pgfqpoint{2.066014in}{0.629515in}}%
\pgfpathlineto{\pgfqpoint{2.068326in}{0.649193in}}%
\pgfpathlineto{\pgfqpoint{2.072951in}{0.630424in}}%
\pgfpathlineto{\pgfqpoint{2.075263in}{0.650144in}}%
\pgfpathlineto{\pgfqpoint{2.079888in}{0.631373in}}%
\pgfpathlineto{\pgfqpoint{2.082200in}{0.651136in}}%
\pgfpathlineto{\pgfqpoint{2.086825in}{0.632365in}}%
\pgfpathlineto{\pgfqpoint{2.089138in}{0.652172in}}%
\pgfpathlineto{\pgfqpoint{2.093762in}{0.633402in}}%
\pgfpathlineto{\pgfqpoint{2.096075in}{0.653254in}}%
\pgfpathlineto{\pgfqpoint{2.100699in}{0.634488in}}%
\pgfpathlineto{\pgfqpoint{2.103012in}{0.654385in}}%
\pgfpathlineto{\pgfqpoint{2.107636in}{0.635626in}}%
\pgfpathlineto{\pgfqpoint{2.109949in}{0.655569in}}%
\pgfpathlineto{\pgfqpoint{2.114573in}{0.636820in}}%
\pgfpathlineto{\pgfqpoint{2.116886in}{0.656810in}}%
\pgfpathlineto{\pgfqpoint{2.121510in}{0.638072in}}%
\pgfpathlineto{\pgfqpoint{2.123823in}{0.658112in}}%
\pgfpathlineto{\pgfqpoint{2.128448in}{0.639389in}}%
\pgfpathlineto{\pgfqpoint{2.130760in}{0.659479in}}%
\pgfpathlineto{\pgfqpoint{2.135385in}{0.640774in}}%
\pgfpathlineto{\pgfqpoint{2.137697in}{0.660915in}}%
\pgfpathlineto{\pgfqpoint{2.142322in}{0.642233in}}%
\pgfpathlineto{\pgfqpoint{2.144634in}{0.662428in}}%
\pgfpathlineto{\pgfqpoint{2.149259in}{0.643771in}}%
\pgfpathlineto{\pgfqpoint{2.151571in}{0.664022in}}%
\pgfpathlineto{\pgfqpoint{2.156196in}{0.645395in}}%
\pgfpathlineto{\pgfqpoint{2.158508in}{0.665704in}}%
\pgfpathlineto{\pgfqpoint{2.163133in}{0.647112in}}%
\pgfpathlineto{\pgfqpoint{2.165445in}{0.667481in}}%
\pgfpathlineto{\pgfqpoint{2.170070in}{0.648930in}}%
\pgfpathlineto{\pgfqpoint{2.172382in}{0.669362in}}%
\pgfpathlineto{\pgfqpoint{2.177007in}{0.650858in}}%
\pgfpathlineto{\pgfqpoint{2.179319in}{0.671356in}}%
\pgfpathlineto{\pgfqpoint{2.183944in}{0.652906in}}%
\pgfpathlineto{\pgfqpoint{2.186256in}{0.673474in}}%
\pgfpathlineto{\pgfqpoint{2.190881in}{0.655084in}}%
\pgfpathlineto{\pgfqpoint{2.193193in}{0.675726in}}%
\pgfpathlineto{\pgfqpoint{2.197818in}{0.657406in}}%
\pgfpathlineto{\pgfqpoint{2.200131in}{0.678126in}}%
\pgfpathlineto{\pgfqpoint{2.204755in}{0.659885in}}%
\pgfpathlineto{\pgfqpoint{2.207068in}{0.680688in}}%
\pgfpathlineto{\pgfqpoint{2.211692in}{0.662538in}}%
\pgfpathlineto{\pgfqpoint{2.214005in}{0.683429in}}%
\pgfpathlineto{\pgfqpoint{2.218629in}{0.665382in}}%
\pgfpathlineto{\pgfqpoint{2.220942in}{0.686369in}}%
\pgfpathlineto{\pgfqpoint{2.225566in}{0.668439in}}%
\pgfpathlineto{\pgfqpoint{2.227879in}{0.689529in}}%
\pgfpathlineto{\pgfqpoint{2.232503in}{0.671732in}}%
\pgfpathlineto{\pgfqpoint{2.234816in}{0.692934in}}%
\pgfpathlineto{\pgfqpoint{2.239441in}{0.675289in}}%
\pgfpathlineto{\pgfqpoint{2.241753in}{0.696613in}}%
\pgfpathlineto{\pgfqpoint{2.246378in}{0.679142in}}%
\pgfpathlineto{\pgfqpoint{2.248690in}{0.700599in}}%
\pgfpathlineto{\pgfqpoint{2.253315in}{0.683327in}}%
\pgfpathlineto{\pgfqpoint{2.255627in}{0.704932in}}%
\pgfpathlineto{\pgfqpoint{2.260252in}{0.687887in}}%
\pgfpathlineto{\pgfqpoint{2.262564in}{0.709656in}}%
\pgfpathlineto{\pgfqpoint{2.267189in}{0.692874in}}%
\pgfpathlineto{\pgfqpoint{2.269501in}{0.714825in}}%
\pgfpathlineto{\pgfqpoint{2.274126in}{0.698347in}}%
\pgfpathlineto{\pgfqpoint{2.276438in}{0.720501in}}%
\pgfpathlineto{\pgfqpoint{2.281063in}{0.704375in}}%
\pgfpathlineto{\pgfqpoint{2.283375in}{0.726760in}}%
\pgfpathlineto{\pgfqpoint{2.288000in}{0.711043in}}%
\pgfpathlineto{\pgfqpoint{2.290312in}{0.733689in}}%
\pgfpathlineto{\pgfqpoint{2.294937in}{0.718450in}}%
\pgfpathlineto{\pgfqpoint{2.297249in}{0.741395in}}%
\pgfpathlineto{\pgfqpoint{2.301874in}{0.726716in}}%
\pgfpathlineto{\pgfqpoint{2.304186in}{0.750004in}}%
\pgfpathlineto{\pgfqpoint{2.308811in}{0.735984in}}%
\pgfpathlineto{\pgfqpoint{2.311124in}{0.759669in}}%
\pgfpathlineto{\pgfqpoint{2.315748in}{0.746430in}}%
\pgfpathlineto{\pgfqpoint{2.318061in}{0.770575in}}%
\pgfpathlineto{\pgfqpoint{2.322685in}{0.758262in}}%
\pgfpathlineto{\pgfqpoint{2.324998in}{0.782944in}}%
\pgfpathlineto{\pgfqpoint{2.329622in}{0.771735in}}%
\pgfpathlineto{\pgfqpoint{2.331935in}{0.797046in}}%
\pgfpathlineto{\pgfqpoint{2.336559in}{0.787158in}}%
\pgfpathlineto{\pgfqpoint{2.338872in}{0.813206in}}%
\pgfpathlineto{\pgfqpoint{2.343497in}{0.804904in}}%
\pgfpathlineto{\pgfqpoint{2.345809in}{0.831819in}}%
\pgfpathlineto{\pgfqpoint{2.348121in}{0.827104in}}%
\pgfpathlineto{\pgfqpoint{2.350434in}{0.825422in}}%
\pgfpathlineto{\pgfqpoint{2.352746in}{0.853356in}}%
\pgfpathlineto{\pgfqpoint{2.355058in}{0.849264in}}%
\pgfpathlineto{\pgfqpoint{2.357371in}{0.849250in}}%
\pgfpathlineto{\pgfqpoint{2.359683in}{0.878376in}}%
\pgfpathlineto{\pgfqpoint{2.361995in}{0.875029in}}%
\pgfpathlineto{\pgfqpoint{2.364308in}{0.877025in}}%
\pgfpathlineto{\pgfqpoint{2.366620in}{0.907538in}}%
\pgfpathlineto{\pgfqpoint{2.368932in}{0.905080in}}%
\pgfpathlineto{\pgfqpoint{2.371245in}{0.909492in}}%
\pgfpathlineto{\pgfqpoint{2.373557in}{0.941606in}}%
\pgfpathlineto{\pgfqpoint{2.375869in}{0.940206in}}%
\pgfpathlineto{\pgfqpoint{2.378182in}{0.947507in}}%
\pgfpathlineto{\pgfqpoint{2.380494in}{0.981440in}}%
\pgfpathlineto{\pgfqpoint{2.382807in}{0.981296in}}%
\pgfpathlineto{\pgfqpoint{2.385119in}{0.992024in}}%
\pgfpathlineto{\pgfqpoint{2.387431in}{1.027987in}}%
\pgfpathlineto{\pgfqpoint{2.389744in}{1.029321in}}%
\pgfpathlineto{\pgfqpoint{2.392056in}{1.044072in}}%
\pgfpathlineto{\pgfqpoint{2.394368in}{1.082238in}}%
\pgfpathlineto{\pgfqpoint{2.396681in}{1.085296in}}%
\pgfpathlineto{\pgfqpoint{2.398993in}{1.104703in}}%
\pgfpathlineto{\pgfqpoint{2.401305in}{1.145165in}}%
\pgfpathlineto{\pgfqpoint{2.403618in}{1.150204in}}%
\pgfpathlineto{\pgfqpoint{2.405930in}{1.174898in}}%
\pgfpathlineto{\pgfqpoint{2.408242in}{1.217616in}}%
\pgfpathlineto{\pgfqpoint{2.410555in}{1.224888in}}%
\pgfpathlineto{\pgfqpoint{2.419804in}{1.346688in}}%
\pgfpathlineto{\pgfqpoint{2.422117in}{1.392871in}}%
\pgfpathlineto{\pgfqpoint{2.424429in}{1.405197in}}%
\pgfpathlineto{\pgfqpoint{2.429054in}{1.495058in}}%
\pgfpathlineto{\pgfqpoint{2.431366in}{1.510001in}}%
\pgfpathlineto{\pgfqpoint{2.435991in}{1.604927in}}%
\pgfpathlineto{\pgfqpoint{2.438303in}{1.622303in}}%
\pgfpathlineto{\pgfqpoint{2.442928in}{1.719222in}}%
\pgfpathlineto{\pgfqpoint{2.445240in}{1.738561in}}%
\pgfpathlineto{\pgfqpoint{2.449865in}{1.832890in}}%
\pgfpathlineto{\pgfqpoint{2.452177in}{1.853346in}}%
\pgfpathlineto{\pgfqpoint{2.454490in}{1.910490in}}%
\pgfpathlineto{\pgfqpoint{2.459114in}{1.959150in}}%
\pgfpathlineto{\pgfqpoint{2.461427in}{2.011967in}}%
\pgfpathlineto{\pgfqpoint{2.466051in}{2.046496in}}%
\pgfpathlineto{\pgfqpoint{2.468364in}{2.089523in}}%
\pgfpathlineto{\pgfqpoint{2.470676in}{2.090359in}}%
\pgfpathlineto{\pgfqpoint{2.472988in}{2.104605in}}%
\pgfpathlineto{\pgfqpoint{2.475301in}{2.131636in}}%
\pgfpathlineto{\pgfqpoint{2.477613in}{2.115137in}}%
\pgfpathlineto{\pgfqpoint{2.482238in}{2.128006in}}%
\pgfpathlineto{\pgfqpoint{2.484550in}{2.094648in}}%
\pgfpathlineto{\pgfqpoint{2.486862in}{2.093936in}}%
\pgfpathlineto{\pgfqpoint{2.489175in}{2.073225in}}%
\pgfpathlineto{\pgfqpoint{2.491487in}{2.027734in}}%
\pgfpathlineto{\pgfqpoint{2.493800in}{2.017678in}}%
\pgfpathlineto{\pgfqpoint{2.498424in}{1.925919in}}%
\pgfpathlineto{\pgfqpoint{2.500737in}{1.908329in}}%
\pgfpathlineto{\pgfqpoint{2.505361in}{1.819980in}}%
\pgfpathlineto{\pgfqpoint{2.507674in}{1.801059in}}%
\pgfpathlineto{\pgfqpoint{2.509986in}{1.754598in}}%
\pgfpathlineto{\pgfqpoint{2.512298in}{1.758763in}}%
\pgfpathlineto{\pgfqpoint{2.516923in}{1.742815in}}%
\pgfpathlineto{\pgfqpoint{2.519235in}{1.781629in}}%
\pgfpathlineto{\pgfqpoint{2.521548in}{1.787201in}}%
\pgfpathlineto{\pgfqpoint{2.533110in}{1.975173in}}%
\pgfpathlineto{\pgfqpoint{2.535422in}{1.992505in}}%
\pgfpathlineto{\pgfqpoint{2.537734in}{2.031096in}}%
\pgfpathlineto{\pgfqpoint{2.540047in}{2.016540in}}%
\pgfpathlineto{\pgfqpoint{2.542359in}{2.022660in}}%
\pgfpathlineto{\pgfqpoint{2.544671in}{2.015173in}}%
\pgfpathlineto{\pgfqpoint{2.546984in}{1.970042in}}%
\pgfpathlineto{\pgfqpoint{2.549296in}{1.960224in}}%
\pgfpathlineto{\pgfqpoint{2.553921in}{1.867921in}}%
\pgfpathlineto{\pgfqpoint{2.556233in}{1.849578in}}%
\pgfpathlineto{\pgfqpoint{2.558545in}{1.798642in}}%
\pgfpathlineto{\pgfqpoint{2.560858in}{1.803546in}}%
\pgfpathlineto{\pgfqpoint{2.563170in}{1.795669in}}%
\pgfpathlineto{\pgfqpoint{2.565483in}{1.802858in}}%
\pgfpathlineto{\pgfqpoint{2.567795in}{1.850192in}}%
\pgfpathlineto{\pgfqpoint{2.570107in}{1.862274in}}%
\pgfpathlineto{\pgfqpoint{2.574732in}{1.950858in}}%
\pgfpathlineto{\pgfqpoint{2.577044in}{1.967870in}}%
\pgfpathlineto{\pgfqpoint{2.579357in}{2.006158in}}%
\pgfpathlineto{\pgfqpoint{2.581669in}{1.985211in}}%
\pgfpathlineto{\pgfqpoint{2.583981in}{1.987349in}}%
\pgfpathlineto{\pgfqpoint{2.586294in}{1.961994in}}%
\pgfpathlineto{\pgfqpoint{2.588606in}{1.912326in}}%
\pgfpathlineto{\pgfqpoint{2.590918in}{1.896731in}}%
\pgfpathlineto{\pgfqpoint{2.593231in}{1.838004in}}%
\pgfpathlineto{\pgfqpoint{2.597855in}{1.816756in}}%
\pgfpathlineto{\pgfqpoint{2.600168in}{1.814624in}}%
\pgfpathlineto{\pgfqpoint{2.602480in}{1.861693in}}%
\pgfpathlineto{\pgfqpoint{2.604793in}{1.873380in}}%
\pgfpathlineto{\pgfqpoint{2.607105in}{1.931803in}}%
\pgfpathlineto{\pgfqpoint{2.611730in}{1.970563in}}%
\pgfpathlineto{\pgfqpoint{2.614042in}{1.992271in}}%
\pgfpathlineto{\pgfqpoint{2.616354in}{1.953965in}}%
\pgfpathlineto{\pgfqpoint{2.618667in}{1.947655in}}%
\pgfpathlineto{\pgfqpoint{2.623291in}{1.860001in}}%
\pgfpathlineto{\pgfqpoint{2.625604in}{1.844125in}}%
\pgfpathlineto{\pgfqpoint{2.627916in}{1.813848in}}%
\pgfpathlineto{\pgfqpoint{2.630228in}{1.850554in}}%
\pgfpathlineto{\pgfqpoint{2.632541in}{1.856918in}}%
\pgfpathlineto{\pgfqpoint{2.637166in}{1.942945in}}%
\pgfpathlineto{\pgfqpoint{2.639478in}{1.957872in}}%
\pgfpathlineto{\pgfqpoint{2.641790in}{1.983233in}}%
\pgfpathlineto{\pgfqpoint{2.644103in}{1.944059in}}%
\pgfpathlineto{\pgfqpoint{2.646415in}{1.936957in}}%
\pgfpathlineto{\pgfqpoint{2.648727in}{1.881807in}}%
\pgfpathlineto{\pgfqpoint{2.651040in}{1.853793in}}%
\pgfpathlineto{\pgfqpoint{2.655664in}{1.829983in}}%
\pgfpathlineto{\pgfqpoint{2.657977in}{1.878271in}}%
\pgfpathlineto{\pgfqpoint{2.660289in}{1.890440in}}%
\pgfpathlineto{\pgfqpoint{2.662601in}{1.950079in}}%
\pgfpathlineto{\pgfqpoint{2.667226in}{1.963014in}}%
\pgfpathlineto{\pgfqpoint{2.669538in}{1.947452in}}%
\pgfpathlineto{\pgfqpoint{2.671851in}{1.896503in}}%
\pgfpathlineto{\pgfqpoint{2.674163in}{1.881393in}}%
\pgfpathlineto{\pgfqpoint{2.676476in}{1.830801in}}%
\pgfpathlineto{\pgfqpoint{2.678788in}{1.854387in}}%
\pgfpathlineto{\pgfqpoint{2.681100in}{1.856801in}}%
\pgfpathlineto{\pgfqpoint{2.685725in}{1.939330in}}%
\pgfpathlineto{\pgfqpoint{2.690350in}{1.964693in}}%
\pgfpathlineto{\pgfqpoint{2.692662in}{1.915611in}}%
\pgfpathlineto{\pgfqpoint{2.694974in}{1.903491in}}%
\pgfpathlineto{\pgfqpoint{2.697287in}{1.844873in}}%
\pgfpathlineto{\pgfqpoint{2.699599in}{1.854338in}}%
\pgfpathlineto{\pgfqpoint{2.701911in}{1.852953in}}%
\pgfpathlineto{\pgfqpoint{2.706536in}{1.933552in}}%
\pgfpathlineto{\pgfqpoint{2.711161in}{1.961372in}}%
\pgfpathlineto{\pgfqpoint{2.713473in}{1.911607in}}%
\pgfpathlineto{\pgfqpoint{2.715786in}{1.899354in}}%
\pgfpathlineto{\pgfqpoint{2.718098in}{1.842529in}}%
\pgfpathlineto{\pgfqpoint{2.720410in}{1.860974in}}%
\pgfpathlineto{\pgfqpoint{2.722723in}{1.862783in}}%
\pgfpathlineto{\pgfqpoint{2.725035in}{1.915465in}}%
\pgfpathlineto{\pgfqpoint{2.727347in}{1.940136in}}%
\pgfpathlineto{\pgfqpoint{2.729660in}{1.949973in}}%
\pgfpathlineto{\pgfqpoint{2.731972in}{1.940794in}}%
\pgfpathlineto{\pgfqpoint{2.734284in}{1.889258in}}%
\pgfpathlineto{\pgfqpoint{2.736597in}{1.875178in}}%
\pgfpathlineto{\pgfqpoint{2.738909in}{1.838530in}}%
\pgfpathlineto{\pgfqpoint{2.741221in}{1.882839in}}%
\pgfpathlineto{\pgfqpoint{2.743534in}{1.892852in}}%
\pgfpathlineto{\pgfqpoint{2.745846in}{1.951946in}}%
\pgfpathlineto{\pgfqpoint{2.748159in}{1.938552in}}%
\pgfpathlineto{\pgfqpoint{2.750471in}{1.938161in}}%
\pgfpathlineto{\pgfqpoint{2.752783in}{1.887373in}}%
\pgfpathlineto{\pgfqpoint{2.755096in}{1.863660in}}%
\pgfpathlineto{\pgfqpoint{2.757408in}{1.855590in}}%
\pgfpathlineto{\pgfqpoint{2.759720in}{1.878883in}}%
\pgfpathlineto{\pgfqpoint{2.762033in}{1.924556in}}%
\pgfpathlineto{\pgfqpoint{2.766657in}{1.951358in}}%
\pgfpathlineto{\pgfqpoint{2.768970in}{1.898846in}}%
\pgfpathlineto{\pgfqpoint{2.771282in}{1.886144in}}%
\pgfpathlineto{\pgfqpoint{2.773594in}{1.842464in}}%
\pgfpathlineto{\pgfqpoint{2.775907in}{1.885576in}}%
\pgfpathlineto{\pgfqpoint{2.778219in}{1.895104in}}%
\pgfpathlineto{\pgfqpoint{2.780531in}{1.952955in}}%
\pgfpathlineto{\pgfqpoint{2.782844in}{1.929889in}}%
\pgfpathlineto{\pgfqpoint{2.785156in}{1.926175in}}%
\pgfpathlineto{\pgfqpoint{2.787469in}{1.867085in}}%
\pgfpathlineto{\pgfqpoint{2.792093in}{1.865524in}}%
\pgfpathlineto{\pgfqpoint{2.794406in}{1.916629in}}%
\pgfpathlineto{\pgfqpoint{2.796718in}{1.935248in}}%
\pgfpathlineto{\pgfqpoint{2.799030in}{1.941114in}}%
\pgfpathlineto{\pgfqpoint{2.803655in}{1.870936in}}%
\pgfpathlineto{\pgfqpoint{2.805967in}{1.861896in}}%
\pgfpathlineto{\pgfqpoint{2.808280in}{1.881461in}}%
\pgfpathlineto{\pgfqpoint{2.810592in}{1.925854in}}%
\pgfpathlineto{\pgfqpoint{2.812904in}{1.936562in}}%
\pgfpathlineto{\pgfqpoint{2.815217in}{1.933878in}}%
\pgfpathlineto{\pgfqpoint{2.817529in}{1.882609in}}%
\pgfpathlineto{\pgfqpoint{2.822154in}{1.862034in}}%
\pgfpathlineto{\pgfqpoint{2.824466in}{1.915103in}}%
\pgfpathlineto{\pgfqpoint{2.826779in}{1.926901in}}%
\pgfpathlineto{\pgfqpoint{2.829091in}{1.945383in}}%
\pgfpathlineto{\pgfqpoint{2.831403in}{1.891460in}}%
\pgfpathlineto{\pgfqpoint{2.833716in}{1.879775in}}%
\pgfpathlineto{\pgfqpoint{2.836028in}{1.855203in}}%
\pgfpathlineto{\pgfqpoint{2.838340in}{1.909146in}}%
\pgfpathlineto{\pgfqpoint{2.840653in}{1.920637in}}%
\pgfpathlineto{\pgfqpoint{2.842965in}{1.948033in}}%
\pgfpathlineto{\pgfqpoint{2.845277in}{1.894397in}}%
\pgfpathlineto{\pgfqpoint{2.847590in}{1.883169in}}%
\pgfpathlineto{\pgfqpoint{2.849902in}{1.855120in}}%
\pgfpathlineto{\pgfqpoint{2.852214in}{1.909186in}}%
\pgfpathlineto{\pgfqpoint{2.854527in}{1.920333in}}%
\pgfpathlineto{\pgfqpoint{2.856839in}{1.945718in}}%
\pgfpathlineto{\pgfqpoint{2.859152in}{1.891448in}}%
\pgfpathlineto{\pgfqpoint{2.861464in}{1.880455in}}%
\pgfpathlineto{\pgfqpoint{2.863776in}{1.860328in}}%
\pgfpathlineto{\pgfqpoint{2.866089in}{1.914563in}}%
\pgfpathlineto{\pgfqpoint{2.870713in}{1.937241in}}%
\pgfpathlineto{\pgfqpoint{2.873026in}{1.883918in}}%
\pgfpathlineto{\pgfqpoint{2.875338in}{1.873659in}}%
\pgfpathlineto{\pgfqpoint{2.877650in}{1.874249in}}%
\pgfpathlineto{\pgfqpoint{2.879963in}{1.922903in}}%
\pgfpathlineto{\pgfqpoint{2.882275in}{1.931950in}}%
\pgfpathlineto{\pgfqpoint{2.884587in}{1.917895in}}%
\pgfpathlineto{\pgfqpoint{2.886900in}{1.875787in}}%
\pgfpathlineto{\pgfqpoint{2.889212in}{1.868544in}}%
\pgfpathlineto{\pgfqpoint{2.893837in}{1.928098in}}%
\pgfpathlineto{\pgfqpoint{2.896149in}{1.932477in}}%
\pgfpathlineto{\pgfqpoint{2.898462in}{1.886127in}}%
\pgfpathlineto{\pgfqpoint{2.900774in}{1.875778in}}%
\pgfpathlineto{\pgfqpoint{2.903086in}{1.875227in}}%
\pgfpathlineto{\pgfqpoint{2.905399in}{1.933878in}}%
\pgfpathlineto{\pgfqpoint{2.907711in}{1.920408in}}%
\pgfpathlineto{\pgfqpoint{2.910023in}{1.917087in}}%
\pgfpathlineto{\pgfqpoint{2.912336in}{1.856867in}}%
\pgfpathlineto{\pgfqpoint{2.914648in}{1.892788in}}%
\pgfpathlineto{\pgfqpoint{2.916960in}{1.900088in}}%
\pgfpathlineto{\pgfqpoint{2.919273in}{1.948010in}}%
\pgfpathlineto{\pgfqpoint{2.921585in}{1.896264in}}%
\pgfpathlineto{\pgfqpoint{2.923897in}{1.887073in}}%
\pgfpathlineto{\pgfqpoint{2.926210in}{1.865124in}}%
\pgfpathlineto{\pgfqpoint{2.928522in}{1.919022in}}%
\pgfpathlineto{\pgfqpoint{2.930834in}{1.927541in}}%
\pgfpathlineto{\pgfqpoint{2.933147in}{1.914580in}}%
\pgfpathlineto{\pgfqpoint{2.935459in}{1.876453in}}%
\pgfpathlineto{\pgfqpoint{2.937772in}{1.871670in}}%
\pgfpathlineto{\pgfqpoint{2.940084in}{1.918974in}}%
\pgfpathlineto{\pgfqpoint{2.942396in}{1.922353in}}%
\pgfpathlineto{\pgfqpoint{2.944709in}{1.921365in}}%
\pgfpathlineto{\pgfqpoint{2.947021in}{1.860540in}}%
\pgfpathlineto{\pgfqpoint{2.949333in}{1.894024in}}%
\pgfpathlineto{\pgfqpoint{2.951646in}{1.900701in}}%
\pgfpathlineto{\pgfqpoint{2.953958in}{1.945101in}}%
\pgfpathlineto{\pgfqpoint{2.956270in}{1.890117in}}%
\pgfpathlineto{\pgfqpoint{2.958583in}{1.881609in}}%
\pgfpathlineto{\pgfqpoint{2.960895in}{1.880420in}}%
\pgfpathlineto{\pgfqpoint{2.963207in}{1.924095in}}%
\pgfpathlineto{\pgfqpoint{2.965520in}{1.929283in}}%
\pgfpathlineto{\pgfqpoint{2.967832in}{1.885470in}}%
\pgfpathlineto{\pgfqpoint{2.970145in}{1.882222in}}%
\pgfpathlineto{\pgfqpoint{2.972457in}{1.884019in}}%
\pgfpathlineto{\pgfqpoint{2.974769in}{1.944057in}}%
\pgfpathlineto{\pgfqpoint{2.977082in}{1.901316in}}%
\pgfpathlineto{\pgfqpoint{2.979394in}{1.894131in}}%
\pgfpathlineto{\pgfqpoint{2.981706in}{1.866330in}}%
\pgfpathlineto{\pgfqpoint{2.984019in}{1.920310in}}%
\pgfpathlineto{\pgfqpoint{2.986331in}{1.926920in}}%
\pgfpathlineto{\pgfqpoint{2.990956in}{1.880644in}}%
\pgfpathlineto{\pgfqpoint{2.993268in}{1.880825in}}%
\pgfpathlineto{\pgfqpoint{2.995580in}{1.941620in}}%
\pgfpathlineto{\pgfqpoint{2.997893in}{1.901994in}}%
\pgfpathlineto{\pgfqpoint{3.000205in}{1.895444in}}%
\pgfpathlineto{\pgfqpoint{3.002517in}{1.867883in}}%
\pgfpathlineto{\pgfqpoint{3.004830in}{1.921073in}}%
\pgfpathlineto{\pgfqpoint{3.007142in}{1.926791in}}%
\pgfpathlineto{\pgfqpoint{3.009455in}{1.889361in}}%
\pgfpathlineto{\pgfqpoint{3.011767in}{1.884577in}}%
\pgfpathlineto{\pgfqpoint{3.014079in}{1.886558in}}%
\pgfpathlineto{\pgfqpoint{3.016392in}{1.944338in}}%
\pgfpathlineto{\pgfqpoint{3.018704in}{1.892905in}}%
\pgfpathlineto{\pgfqpoint{3.021016in}{1.886034in}}%
\pgfpathlineto{\pgfqpoint{3.023329in}{1.885514in}}%
\pgfpathlineto{\pgfqpoint{3.025641in}{1.922698in}}%
\pgfpathlineto{\pgfqpoint{3.027953in}{1.925058in}}%
\pgfpathlineto{\pgfqpoint{3.030266in}{1.867040in}}%
\pgfpathlineto{\pgfqpoint{3.032578in}{1.898579in}}%
\pgfpathlineto{\pgfqpoint{3.034890in}{1.904063in}}%
\pgfpathlineto{\pgfqpoint{3.037203in}{1.933679in}}%
\pgfpathlineto{\pgfqpoint{3.039515in}{1.880703in}}%
\pgfpathlineto{\pgfqpoint{3.041828in}{1.876227in}}%
\pgfpathlineto{\pgfqpoint{3.044140in}{1.923947in}}%
\pgfpathlineto{\pgfqpoint{3.046452in}{1.910231in}}%
\pgfpathlineto{\pgfqpoint{3.048765in}{1.906555in}}%
\pgfpathlineto{\pgfqpoint{3.051077in}{1.862922in}}%
\pgfpathlineto{\pgfqpoint{3.053389in}{1.919423in}}%
\pgfpathlineto{\pgfqpoint{3.055702in}{1.924445in}}%
\pgfpathlineto{\pgfqpoint{3.058014in}{1.885922in}}%
\pgfpathlineto{\pgfqpoint{3.062639in}{1.893273in}}%
\pgfpathlineto{\pgfqpoint{3.064951in}{1.940325in}}%
\pgfpathlineto{\pgfqpoint{3.067263in}{1.883197in}}%
\pgfpathlineto{\pgfqpoint{3.069576in}{1.878374in}}%
\pgfpathlineto{\pgfqpoint{3.071888in}{1.918418in}}%
\pgfpathlineto{\pgfqpoint{3.074200in}{1.910432in}}%
\pgfpathlineto{\pgfqpoint{3.076513in}{1.907385in}}%
\pgfpathlineto{\pgfqpoint{3.078825in}{1.864681in}}%
\pgfpathlineto{\pgfqpoint{3.081138in}{1.920457in}}%
\pgfpathlineto{\pgfqpoint{3.083450in}{1.924377in}}%
\pgfpathlineto{\pgfqpoint{3.085762in}{1.876593in}}%
\pgfpathlineto{\pgfqpoint{3.088075in}{1.897722in}}%
\pgfpathlineto{\pgfqpoint{3.090387in}{1.901715in}}%
\pgfpathlineto{\pgfqpoint{3.092699in}{1.929287in}}%
\pgfpathlineto{\pgfqpoint{3.095012in}{1.880775in}}%
\pgfpathlineto{\pgfqpoint{3.097324in}{1.878588in}}%
\pgfpathlineto{\pgfqpoint{3.099636in}{1.937529in}}%
\pgfpathlineto{\pgfqpoint{3.101949in}{1.894912in}}%
\pgfpathlineto{\pgfqpoint{3.104261in}{1.890316in}}%
\pgfpathlineto{\pgfqpoint{3.106573in}{1.893302in}}%
\pgfpathlineto{\pgfqpoint{3.108886in}{1.917317in}}%
\pgfpathlineto{\pgfqpoint{3.111198in}{1.916851in}}%
\pgfpathlineto{\pgfqpoint{3.113510in}{1.860436in}}%
\pgfpathlineto{\pgfqpoint{3.115823in}{1.918092in}}%
\pgfpathlineto{\pgfqpoint{3.118135in}{1.921926in}}%
\pgfpathlineto{\pgfqpoint{3.120448in}{1.880996in}}%
\pgfpathlineto{\pgfqpoint{3.122760in}{1.898416in}}%
\pgfpathlineto{\pgfqpoint{3.125072in}{1.901701in}}%
\pgfpathlineto{\pgfqpoint{3.127385in}{1.924630in}}%
\pgfpathlineto{\pgfqpoint{3.129697in}{1.882267in}}%
\pgfpathlineto{\pgfqpoint{3.132009in}{1.881467in}}%
\pgfpathlineto{\pgfqpoint{3.134322in}{1.941538in}}%
\pgfpathlineto{\pgfqpoint{3.136634in}{1.886233in}}%
\pgfpathlineto{\pgfqpoint{3.138946in}{1.882612in}}%
\pgfpathlineto{\pgfqpoint{3.141259in}{1.918560in}}%
\pgfpathlineto{\pgfqpoint{3.143571in}{1.904153in}}%
\pgfpathlineto{\pgfqpoint{3.145883in}{1.901429in}}%
\pgfpathlineto{\pgfqpoint{3.148196in}{1.880462in}}%
\pgfpathlineto{\pgfqpoint{3.150508in}{1.918672in}}%
\pgfpathlineto{\pgfqpoint{3.152821in}{1.918976in}}%
\pgfpathlineto{\pgfqpoint{3.155133in}{1.860862in}}%
\pgfpathlineto{\pgfqpoint{3.157445in}{1.919031in}}%
\pgfpathlineto{\pgfqpoint{3.159758in}{1.921717in}}%
\pgfpathlineto{\pgfqpoint{3.162070in}{1.872837in}}%
\pgfpathlineto{\pgfqpoint{3.164382in}{1.906952in}}%
\pgfpathlineto{\pgfqpoint{3.166695in}{1.909906in}}%
\pgfpathlineto{\pgfqpoint{3.169007in}{1.903349in}}%
\pgfpathlineto{\pgfqpoint{3.171319in}{1.891759in}}%
\pgfpathlineto{\pgfqpoint{3.173632in}{1.893167in}}%
\pgfpathlineto{\pgfqpoint{3.175944in}{1.930704in}}%
\pgfpathlineto{\pgfqpoint{3.178256in}{1.882515in}}%
\pgfpathlineto{\pgfqpoint{3.180569in}{1.881856in}}%
\pgfpathlineto{\pgfqpoint{3.182881in}{1.941169in}}%
\pgfpathlineto{\pgfqpoint{3.185193in}{1.882929in}}%
\pgfpathlineto{\pgfqpoint{3.187506in}{1.880881in}}%
\pgfpathlineto{\pgfqpoint{3.189818in}{1.933088in}}%
\pgfpathlineto{\pgfqpoint{3.192131in}{1.891068in}}%
\pgfpathlineto{\pgfqpoint{3.194443in}{1.888758in}}%
\pgfpathlineto{\pgfqpoint{3.196755in}{1.913065in}}%
\pgfpathlineto{\pgfqpoint{3.199068in}{1.902268in}}%
\pgfpathlineto{\pgfqpoint{3.201380in}{1.900584in}}%
\pgfpathlineto{\pgfqpoint{3.203692in}{1.890337in}}%
\pgfpathlineto{\pgfqpoint{3.206005in}{1.912222in}}%
\pgfpathlineto{\pgfqpoint{3.208317in}{1.911531in}}%
\pgfpathlineto{\pgfqpoint{3.210629in}{1.872382in}}%
\pgfpathlineto{\pgfqpoint{3.212942in}{1.918501in}}%
\pgfpathlineto{\pgfqpoint{3.215254in}{1.918734in}}%
\pgfpathlineto{\pgfqpoint{3.217566in}{1.862856in}}%
\pgfpathlineto{\pgfqpoint{3.219879in}{1.920608in}}%
\pgfpathlineto{\pgfqpoint{3.222191in}{1.921468in}}%
\pgfpathlineto{\pgfqpoint{3.224503in}{1.861743in}}%
\pgfpathlineto{\pgfqpoint{3.226816in}{1.919304in}}%
\pgfpathlineto{\pgfqpoint{3.229128in}{1.920452in}}%
\pgfpathlineto{\pgfqpoint{3.231441in}{1.866811in}}%
\pgfpathlineto{\pgfqpoint{3.233753in}{1.915854in}}%
\pgfpathlineto{\pgfqpoint{3.236065in}{1.917020in}}%
\pgfpathlineto{\pgfqpoint{3.238378in}{1.875205in}}%
\pgfpathlineto{\pgfqpoint{3.240690in}{1.911491in}}%
\pgfpathlineto{\pgfqpoint{3.243002in}{1.912507in}}%
\pgfpathlineto{\pgfqpoint{3.245315in}{1.884488in}}%
\pgfpathlineto{\pgfqpoint{3.247627in}{1.907144in}}%
\pgfpathlineto{\pgfqpoint{3.249939in}{1.907941in}}%
\pgfpathlineto{\pgfqpoint{3.252252in}{1.893029in}}%
\pgfpathlineto{\pgfqpoint{3.254564in}{1.903392in}}%
\pgfpathlineto{\pgfqpoint{3.256876in}{1.903961in}}%
\pgfpathlineto{\pgfqpoint{3.259189in}{1.899953in}}%
\pgfpathlineto{\pgfqpoint{3.263814in}{1.900885in}}%
\pgfpathlineto{\pgfqpoint{3.266126in}{1.904918in}}%
\pgfpathlineto{\pgfqpoint{3.268438in}{1.898617in}}%
\pgfpathlineto{\pgfqpoint{3.270751in}{1.898821in}}%
\pgfpathlineto{\pgfqpoint{3.273063in}{1.907867in}}%
\pgfpathlineto{\pgfqpoint{3.275375in}{1.897703in}}%
\pgfpathlineto{\pgfqpoint{3.277688in}{1.897768in}}%
\pgfpathlineto{\pgfqpoint{3.280000in}{1.908841in}}%
\pgfpathlineto{\pgfqpoint{3.282312in}{1.897768in}}%
\pgfpathlineto{\pgfqpoint{3.284625in}{1.897703in}}%
\pgfpathlineto{\pgfqpoint{3.286937in}{1.907867in}}%
\pgfpathlineto{\pgfqpoint{3.289249in}{1.898821in}}%
\pgfpathlineto{\pgfqpoint{3.291562in}{1.898617in}}%
\pgfpathlineto{\pgfqpoint{3.293874in}{1.904918in}}%
\pgfpathlineto{\pgfqpoint{3.296186in}{1.900885in}}%
\pgfpathlineto{\pgfqpoint{3.300811in}{1.899953in}}%
\pgfpathlineto{\pgfqpoint{3.303124in}{1.903961in}}%
\pgfpathlineto{\pgfqpoint{3.305436in}{1.903392in}}%
\pgfpathlineto{\pgfqpoint{3.307748in}{1.893029in}}%
\pgfpathlineto{\pgfqpoint{3.310061in}{1.907941in}}%
\pgfpathlineto{\pgfqpoint{3.312373in}{1.907144in}}%
\pgfpathlineto{\pgfqpoint{3.314685in}{1.884488in}}%
\pgfpathlineto{\pgfqpoint{3.316998in}{1.912507in}}%
\pgfpathlineto{\pgfqpoint{3.319310in}{1.911491in}}%
\pgfpathlineto{\pgfqpoint{3.321622in}{1.875205in}}%
\pgfpathlineto{\pgfqpoint{3.323935in}{1.917020in}}%
\pgfpathlineto{\pgfqpoint{3.326247in}{1.915854in}}%
\pgfpathlineto{\pgfqpoint{3.328559in}{1.866811in}}%
\pgfpathlineto{\pgfqpoint{3.330872in}{1.920452in}}%
\pgfpathlineto{\pgfqpoint{3.333184in}{1.919304in}}%
\pgfpathlineto{\pgfqpoint{3.335497in}{1.861743in}}%
\pgfpathlineto{\pgfqpoint{3.337809in}{1.921468in}}%
\pgfpathlineto{\pgfqpoint{3.340121in}{1.920608in}}%
\pgfpathlineto{\pgfqpoint{3.342434in}{1.862856in}}%
\pgfpathlineto{\pgfqpoint{3.344746in}{1.918734in}}%
\pgfpathlineto{\pgfqpoint{3.347058in}{1.918501in}}%
\pgfpathlineto{\pgfqpoint{3.349371in}{1.872382in}}%
\pgfpathlineto{\pgfqpoint{3.351683in}{1.911531in}}%
\pgfpathlineto{\pgfqpoint{3.353995in}{1.912222in}}%
\pgfpathlineto{\pgfqpoint{3.356308in}{1.890337in}}%
\pgfpathlineto{\pgfqpoint{3.358620in}{1.900584in}}%
\pgfpathlineto{\pgfqpoint{3.360932in}{1.902268in}}%
\pgfpathlineto{\pgfqpoint{3.363245in}{1.913065in}}%
\pgfpathlineto{\pgfqpoint{3.365557in}{1.888758in}}%
\pgfpathlineto{\pgfqpoint{3.367869in}{1.891068in}}%
\pgfpathlineto{\pgfqpoint{3.370182in}{1.933088in}}%
\pgfpathlineto{\pgfqpoint{3.372494in}{1.880881in}}%
\pgfpathlineto{\pgfqpoint{3.374807in}{1.882929in}}%
\pgfpathlineto{\pgfqpoint{3.377119in}{1.941169in}}%
\pgfpathlineto{\pgfqpoint{3.379431in}{1.881856in}}%
\pgfpathlineto{\pgfqpoint{3.381744in}{1.882515in}}%
\pgfpathlineto{\pgfqpoint{3.384056in}{1.930704in}}%
\pgfpathlineto{\pgfqpoint{3.386368in}{1.893167in}}%
\pgfpathlineto{\pgfqpoint{3.388681in}{1.891759in}}%
\pgfpathlineto{\pgfqpoint{3.393305in}{1.909906in}}%
\pgfpathlineto{\pgfqpoint{3.395618in}{1.906952in}}%
\pgfpathlineto{\pgfqpoint{3.397930in}{1.872837in}}%
\pgfpathlineto{\pgfqpoint{3.400242in}{1.921717in}}%
\pgfpathlineto{\pgfqpoint{3.402555in}{1.919031in}}%
\pgfpathlineto{\pgfqpoint{3.404867in}{1.860862in}}%
\pgfpathlineto{\pgfqpoint{3.407179in}{1.918976in}}%
\pgfpathlineto{\pgfqpoint{3.409492in}{1.918672in}}%
\pgfpathlineto{\pgfqpoint{3.411804in}{1.880462in}}%
\pgfpathlineto{\pgfqpoint{3.414117in}{1.901429in}}%
\pgfpathlineto{\pgfqpoint{3.416429in}{1.904153in}}%
\pgfpathlineto{\pgfqpoint{3.418741in}{1.918560in}}%
\pgfpathlineto{\pgfqpoint{3.421054in}{1.882612in}}%
\pgfpathlineto{\pgfqpoint{3.423366in}{1.886233in}}%
\pgfpathlineto{\pgfqpoint{3.425678in}{1.941538in}}%
\pgfpathlineto{\pgfqpoint{3.427991in}{1.881467in}}%
\pgfpathlineto{\pgfqpoint{3.430303in}{1.882267in}}%
\pgfpathlineto{\pgfqpoint{3.432615in}{1.924630in}}%
\pgfpathlineto{\pgfqpoint{3.434928in}{1.901701in}}%
\pgfpathlineto{\pgfqpoint{3.437240in}{1.898416in}}%
\pgfpathlineto{\pgfqpoint{3.439552in}{1.880996in}}%
\pgfpathlineto{\pgfqpoint{3.441865in}{1.921926in}}%
\pgfpathlineto{\pgfqpoint{3.444177in}{1.918092in}}%
\pgfpathlineto{\pgfqpoint{3.446490in}{1.860436in}}%
\pgfpathlineto{\pgfqpoint{3.448802in}{1.916851in}}%
\pgfpathlineto{\pgfqpoint{3.451114in}{1.917317in}}%
\pgfpathlineto{\pgfqpoint{3.453427in}{1.893302in}}%
\pgfpathlineto{\pgfqpoint{3.455739in}{1.890316in}}%
\pgfpathlineto{\pgfqpoint{3.458051in}{1.894912in}}%
\pgfpathlineto{\pgfqpoint{3.460364in}{1.937529in}}%
\pgfpathlineto{\pgfqpoint{3.462676in}{1.878588in}}%
\pgfpathlineto{\pgfqpoint{3.464988in}{1.880775in}}%
\pgfpathlineto{\pgfqpoint{3.467301in}{1.929287in}}%
\pgfpathlineto{\pgfqpoint{3.469613in}{1.901715in}}%
\pgfpathlineto{\pgfqpoint{3.471925in}{1.897722in}}%
\pgfpathlineto{\pgfqpoint{3.474238in}{1.876593in}}%
\pgfpathlineto{\pgfqpoint{3.476550in}{1.924377in}}%
\pgfpathlineto{\pgfqpoint{3.478862in}{1.920457in}}%
\pgfpathlineto{\pgfqpoint{3.481175in}{1.864681in}}%
\pgfpathlineto{\pgfqpoint{3.483487in}{1.907385in}}%
\pgfpathlineto{\pgfqpoint{3.485800in}{1.910432in}}%
\pgfpathlineto{\pgfqpoint{3.488112in}{1.918418in}}%
\pgfpathlineto{\pgfqpoint{3.490424in}{1.878374in}}%
\pgfpathlineto{\pgfqpoint{3.492737in}{1.883197in}}%
\pgfpathlineto{\pgfqpoint{3.495049in}{1.940325in}}%
\pgfpathlineto{\pgfqpoint{3.497361in}{1.893273in}}%
\pgfpathlineto{\pgfqpoint{3.501986in}{1.885922in}}%
\pgfpathlineto{\pgfqpoint{3.504298in}{1.924445in}}%
\pgfpathlineto{\pgfqpoint{3.506611in}{1.919423in}}%
\pgfpathlineto{\pgfqpoint{3.508923in}{1.862922in}}%
\pgfpathlineto{\pgfqpoint{3.511235in}{1.906555in}}%
\pgfpathlineto{\pgfqpoint{3.513548in}{1.910231in}}%
\pgfpathlineto{\pgfqpoint{3.515860in}{1.923947in}}%
\pgfpathlineto{\pgfqpoint{3.518172in}{1.876227in}}%
\pgfpathlineto{\pgfqpoint{3.520485in}{1.880703in}}%
\pgfpathlineto{\pgfqpoint{3.522797in}{1.933679in}}%
\pgfpathlineto{\pgfqpoint{3.525110in}{1.904063in}}%
\pgfpathlineto{\pgfqpoint{3.527422in}{1.898579in}}%
\pgfpathlineto{\pgfqpoint{3.529734in}{1.867040in}}%
\pgfpathlineto{\pgfqpoint{3.532047in}{1.925058in}}%
\pgfpathlineto{\pgfqpoint{3.534359in}{1.922698in}}%
\pgfpathlineto{\pgfqpoint{3.536671in}{1.885514in}}%
\pgfpathlineto{\pgfqpoint{3.538984in}{1.886034in}}%
\pgfpathlineto{\pgfqpoint{3.541296in}{1.892905in}}%
\pgfpathlineto{\pgfqpoint{3.543608in}{1.944338in}}%
\pgfpathlineto{\pgfqpoint{3.545921in}{1.886558in}}%
\pgfpathlineto{\pgfqpoint{3.548233in}{1.884577in}}%
\pgfpathlineto{\pgfqpoint{3.550545in}{1.889361in}}%
\pgfpathlineto{\pgfqpoint{3.552858in}{1.926791in}}%
\pgfpathlineto{\pgfqpoint{3.555170in}{1.921073in}}%
\pgfpathlineto{\pgfqpoint{3.557483in}{1.867883in}}%
\pgfpathlineto{\pgfqpoint{3.559795in}{1.895444in}}%
\pgfpathlineto{\pgfqpoint{3.562107in}{1.901994in}}%
\pgfpathlineto{\pgfqpoint{3.564420in}{1.941620in}}%
\pgfpathlineto{\pgfqpoint{3.566732in}{1.880825in}}%
\pgfpathlineto{\pgfqpoint{3.569044in}{1.880644in}}%
\pgfpathlineto{\pgfqpoint{3.571357in}{1.897263in}}%
\pgfpathlineto{\pgfqpoint{3.573669in}{1.926920in}}%
\pgfpathlineto{\pgfqpoint{3.575981in}{1.920310in}}%
\pgfpathlineto{\pgfqpoint{3.578294in}{1.866330in}}%
\pgfpathlineto{\pgfqpoint{3.580606in}{1.894131in}}%
\pgfpathlineto{\pgfqpoint{3.582918in}{1.901316in}}%
\pgfpathlineto{\pgfqpoint{3.585231in}{1.944057in}}%
\pgfpathlineto{\pgfqpoint{3.587543in}{1.884019in}}%
\pgfpathlineto{\pgfqpoint{3.589855in}{1.882222in}}%
\pgfpathlineto{\pgfqpoint{3.592168in}{1.885470in}}%
\pgfpathlineto{\pgfqpoint{3.594480in}{1.929283in}}%
\pgfpathlineto{\pgfqpoint{3.596793in}{1.924095in}}%
\pgfpathlineto{\pgfqpoint{3.599105in}{1.880420in}}%
\pgfpathlineto{\pgfqpoint{3.601417in}{1.881609in}}%
\pgfpathlineto{\pgfqpoint{3.603730in}{1.890117in}}%
\pgfpathlineto{\pgfqpoint{3.606042in}{1.945101in}}%
\pgfpathlineto{\pgfqpoint{3.608354in}{1.900701in}}%
\pgfpathlineto{\pgfqpoint{3.610667in}{1.894024in}}%
\pgfpathlineto{\pgfqpoint{3.612979in}{1.860540in}}%
\pgfpathlineto{\pgfqpoint{3.615291in}{1.921365in}}%
\pgfpathlineto{\pgfqpoint{3.617604in}{1.922353in}}%
\pgfpathlineto{\pgfqpoint{3.619916in}{1.918974in}}%
\pgfpathlineto{\pgfqpoint{3.622228in}{1.871670in}}%
\pgfpathlineto{\pgfqpoint{3.624541in}{1.876453in}}%
\pgfpathlineto{\pgfqpoint{3.626853in}{1.914580in}}%
\pgfpathlineto{\pgfqpoint{3.629166in}{1.927541in}}%
\pgfpathlineto{\pgfqpoint{3.631478in}{1.919022in}}%
\pgfpathlineto{\pgfqpoint{3.633790in}{1.865124in}}%
\pgfpathlineto{\pgfqpoint{3.636103in}{1.887073in}}%
\pgfpathlineto{\pgfqpoint{3.638415in}{1.896264in}}%
\pgfpathlineto{\pgfqpoint{3.640727in}{1.948010in}}%
\pgfpathlineto{\pgfqpoint{3.643040in}{1.900088in}}%
\pgfpathlineto{\pgfqpoint{3.645352in}{1.892788in}}%
\pgfpathlineto{\pgfqpoint{3.647664in}{1.856867in}}%
\pgfpathlineto{\pgfqpoint{3.649977in}{1.917087in}}%
\pgfpathlineto{\pgfqpoint{3.652289in}{1.920408in}}%
\pgfpathlineto{\pgfqpoint{3.654601in}{1.933878in}}%
\pgfpathlineto{\pgfqpoint{3.656914in}{1.875227in}}%
\pgfpathlineto{\pgfqpoint{3.659226in}{1.875778in}}%
\pgfpathlineto{\pgfqpoint{3.661538in}{1.886127in}}%
\pgfpathlineto{\pgfqpoint{3.663851in}{1.932477in}}%
\pgfpathlineto{\pgfqpoint{3.666163in}{1.928098in}}%
\pgfpathlineto{\pgfqpoint{3.670788in}{1.868544in}}%
\pgfpathlineto{\pgfqpoint{3.673100in}{1.875787in}}%
\pgfpathlineto{\pgfqpoint{3.675413in}{1.917895in}}%
\pgfpathlineto{\pgfqpoint{3.677725in}{1.931950in}}%
\pgfpathlineto{\pgfqpoint{3.680037in}{1.922903in}}%
\pgfpathlineto{\pgfqpoint{3.682350in}{1.874249in}}%
\pgfpathlineto{\pgfqpoint{3.684662in}{1.873659in}}%
\pgfpathlineto{\pgfqpoint{3.686974in}{1.883918in}}%
\pgfpathlineto{\pgfqpoint{3.689287in}{1.937241in}}%
\pgfpathlineto{\pgfqpoint{3.693911in}{1.914563in}}%
\pgfpathlineto{\pgfqpoint{3.696224in}{1.860328in}}%
\pgfpathlineto{\pgfqpoint{3.700848in}{1.891448in}}%
\pgfpathlineto{\pgfqpoint{3.703161in}{1.945718in}}%
\pgfpathlineto{\pgfqpoint{3.705473in}{1.920333in}}%
\pgfpathlineto{\pgfqpoint{3.707786in}{1.909186in}}%
\pgfpathlineto{\pgfqpoint{3.710098in}{1.855120in}}%
\pgfpathlineto{\pgfqpoint{3.712410in}{1.883169in}}%
\pgfpathlineto{\pgfqpoint{3.714723in}{1.894397in}}%
\pgfpathlineto{\pgfqpoint{3.717035in}{1.948033in}}%
\pgfpathlineto{\pgfqpoint{3.719347in}{1.920637in}}%
\pgfpathlineto{\pgfqpoint{3.721660in}{1.909146in}}%
\pgfpathlineto{\pgfqpoint{3.723972in}{1.855203in}}%
\pgfpathlineto{\pgfqpoint{3.726284in}{1.879775in}}%
\pgfpathlineto{\pgfqpoint{3.728597in}{1.891460in}}%
\pgfpathlineto{\pgfqpoint{3.730909in}{1.945383in}}%
\pgfpathlineto{\pgfqpoint{3.735534in}{1.915103in}}%
\pgfpathlineto{\pgfqpoint{3.737846in}{1.862034in}}%
\pgfpathlineto{\pgfqpoint{3.742471in}{1.882609in}}%
\pgfpathlineto{\pgfqpoint{3.744783in}{1.933878in}}%
\pgfpathlineto{\pgfqpoint{3.747096in}{1.936562in}}%
\pgfpathlineto{\pgfqpoint{3.749408in}{1.925854in}}%
\pgfpathlineto{\pgfqpoint{3.751720in}{1.881461in}}%
\pgfpathlineto{\pgfqpoint{3.754033in}{1.861896in}}%
\pgfpathlineto{\pgfqpoint{3.756345in}{1.870936in}}%
\pgfpathlineto{\pgfqpoint{3.760970in}{1.941114in}}%
\pgfpathlineto{\pgfqpoint{3.763282in}{1.935248in}}%
\pgfpathlineto{\pgfqpoint{3.765594in}{1.916629in}}%
\pgfpathlineto{\pgfqpoint{3.767907in}{1.865524in}}%
\pgfpathlineto{\pgfqpoint{3.772531in}{1.867085in}}%
\pgfpathlineto{\pgfqpoint{3.774844in}{1.926175in}}%
\pgfpathlineto{\pgfqpoint{3.777156in}{1.929889in}}%
\pgfpathlineto{\pgfqpoint{3.779469in}{1.952955in}}%
\pgfpathlineto{\pgfqpoint{3.781781in}{1.895104in}}%
\pgfpathlineto{\pgfqpoint{3.784093in}{1.885576in}}%
\pgfpathlineto{\pgfqpoint{3.786406in}{1.842464in}}%
\pgfpathlineto{\pgfqpoint{3.788718in}{1.886144in}}%
\pgfpathlineto{\pgfqpoint{3.791030in}{1.898846in}}%
\pgfpathlineto{\pgfqpoint{3.793343in}{1.951358in}}%
\pgfpathlineto{\pgfqpoint{3.797967in}{1.924556in}}%
\pgfpathlineto{\pgfqpoint{3.800280in}{1.878883in}}%
\pgfpathlineto{\pgfqpoint{3.802592in}{1.855590in}}%
\pgfpathlineto{\pgfqpoint{3.804904in}{1.863660in}}%
\pgfpathlineto{\pgfqpoint{3.807217in}{1.887373in}}%
\pgfpathlineto{\pgfqpoint{3.809529in}{1.938161in}}%
\pgfpathlineto{\pgfqpoint{3.811841in}{1.938552in}}%
\pgfpathlineto{\pgfqpoint{3.814154in}{1.951946in}}%
\pgfpathlineto{\pgfqpoint{3.816466in}{1.892852in}}%
\pgfpathlineto{\pgfqpoint{3.818779in}{1.882839in}}%
\pgfpathlineto{\pgfqpoint{3.821091in}{1.838530in}}%
\pgfpathlineto{\pgfqpoint{3.823403in}{1.875178in}}%
\pgfpathlineto{\pgfqpoint{3.825716in}{1.889258in}}%
\pgfpathlineto{\pgfqpoint{3.828028in}{1.940794in}}%
\pgfpathlineto{\pgfqpoint{3.830340in}{1.949973in}}%
\pgfpathlineto{\pgfqpoint{3.832653in}{1.940136in}}%
\pgfpathlineto{\pgfqpoint{3.834965in}{1.915465in}}%
\pgfpathlineto{\pgfqpoint{3.837277in}{1.862783in}}%
\pgfpathlineto{\pgfqpoint{3.839590in}{1.860974in}}%
\pgfpathlineto{\pgfqpoint{3.841902in}{1.842529in}}%
\pgfpathlineto{\pgfqpoint{3.844214in}{1.899354in}}%
\pgfpathlineto{\pgfqpoint{3.846527in}{1.911607in}}%
\pgfpathlineto{\pgfqpoint{3.848839in}{1.961372in}}%
\pgfpathlineto{\pgfqpoint{3.853464in}{1.933552in}}%
\pgfpathlineto{\pgfqpoint{3.858089in}{1.852953in}}%
\pgfpathlineto{\pgfqpoint{3.860401in}{1.854338in}}%
\pgfpathlineto{\pgfqpoint{3.862713in}{1.844873in}}%
\pgfpathlineto{\pgfqpoint{3.865026in}{1.903491in}}%
\pgfpathlineto{\pgfqpoint{3.867338in}{1.915611in}}%
\pgfpathlineto{\pgfqpoint{3.869650in}{1.964693in}}%
\pgfpathlineto{\pgfqpoint{3.874275in}{1.939330in}}%
\pgfpathlineto{\pgfqpoint{3.876587in}{1.907212in}}%
\pgfpathlineto{\pgfqpoint{3.878900in}{1.856801in}}%
\pgfpathlineto{\pgfqpoint{3.881212in}{1.854387in}}%
\pgfpathlineto{\pgfqpoint{3.883524in}{1.830801in}}%
\pgfpathlineto{\pgfqpoint{3.885837in}{1.881393in}}%
\pgfpathlineto{\pgfqpoint{3.888149in}{1.896503in}}%
\pgfpathlineto{\pgfqpoint{3.890462in}{1.947452in}}%
\pgfpathlineto{\pgfqpoint{3.892774in}{1.963014in}}%
\pgfpathlineto{\pgfqpoint{3.897399in}{1.950079in}}%
\pgfpathlineto{\pgfqpoint{3.899711in}{1.890440in}}%
\pgfpathlineto{\pgfqpoint{3.902023in}{1.878271in}}%
\pgfpathlineto{\pgfqpoint{3.904336in}{1.829983in}}%
\pgfpathlineto{\pgfqpoint{3.908960in}{1.853793in}}%
\pgfpathlineto{\pgfqpoint{3.911273in}{1.881807in}}%
\pgfpathlineto{\pgfqpoint{3.913585in}{1.936957in}}%
\pgfpathlineto{\pgfqpoint{3.915897in}{1.944059in}}%
\pgfpathlineto{\pgfqpoint{3.918210in}{1.983233in}}%
\pgfpathlineto{\pgfqpoint{3.925147in}{1.910466in}}%
\pgfpathlineto{\pgfqpoint{3.927459in}{1.856918in}}%
\pgfpathlineto{\pgfqpoint{3.929772in}{1.850554in}}%
\pgfpathlineto{\pgfqpoint{3.932084in}{1.813848in}}%
\pgfpathlineto{\pgfqpoint{3.945958in}{1.992271in}}%
\pgfpathlineto{\pgfqpoint{3.952895in}{1.931803in}}%
\pgfpathlineto{\pgfqpoint{3.955207in}{1.873380in}}%
\pgfpathlineto{\pgfqpoint{3.957520in}{1.861693in}}%
\pgfpathlineto{\pgfqpoint{3.959832in}{1.814624in}}%
\pgfpathlineto{\pgfqpoint{3.962145in}{1.816756in}}%
\pgfpathlineto{\pgfqpoint{3.966769in}{1.838004in}}%
\pgfpathlineto{\pgfqpoint{3.969082in}{1.896731in}}%
\pgfpathlineto{\pgfqpoint{3.971394in}{1.912326in}}%
\pgfpathlineto{\pgfqpoint{3.973706in}{1.961994in}}%
\pgfpathlineto{\pgfqpoint{3.976019in}{1.987349in}}%
\pgfpathlineto{\pgfqpoint{3.978331in}{1.985211in}}%
\pgfpathlineto{\pgfqpoint{3.980643in}{2.006158in}}%
\pgfpathlineto{\pgfqpoint{3.982956in}{1.967870in}}%
\pgfpathlineto{\pgfqpoint{3.985268in}{1.950858in}}%
\pgfpathlineto{\pgfqpoint{3.987580in}{1.919171in}}%
\pgfpathlineto{\pgfqpoint{3.989893in}{1.862274in}}%
\pgfpathlineto{\pgfqpoint{3.992205in}{1.850192in}}%
\pgfpathlineto{\pgfqpoint{3.994517in}{1.802858in}}%
\pgfpathlineto{\pgfqpoint{3.996830in}{1.795669in}}%
\pgfpathlineto{\pgfqpoint{3.999142in}{1.803546in}}%
\pgfpathlineto{\pgfqpoint{4.001455in}{1.798642in}}%
\pgfpathlineto{\pgfqpoint{4.003767in}{1.849578in}}%
\pgfpathlineto{\pgfqpoint{4.006079in}{1.867921in}}%
\pgfpathlineto{\pgfqpoint{4.010704in}{1.960224in}}%
\pgfpathlineto{\pgfqpoint{4.013016in}{1.970042in}}%
\pgfpathlineto{\pgfqpoint{4.015329in}{2.015173in}}%
\pgfpathlineto{\pgfqpoint{4.017641in}{2.022660in}}%
\pgfpathlineto{\pgfqpoint{4.019953in}{2.016540in}}%
\pgfpathlineto{\pgfqpoint{4.022266in}{2.031096in}}%
\pgfpathlineto{\pgfqpoint{4.024578in}{1.992505in}}%
\pgfpathlineto{\pgfqpoint{4.029203in}{1.950736in}}%
\pgfpathlineto{\pgfqpoint{4.031515in}{1.892174in}}%
\pgfpathlineto{\pgfqpoint{4.033828in}{1.874921in}}%
\pgfpathlineto{\pgfqpoint{4.038452in}{1.787201in}}%
\pgfpathlineto{\pgfqpoint{4.040765in}{1.781629in}}%
\pgfpathlineto{\pgfqpoint{4.043077in}{1.742815in}}%
\pgfpathlineto{\pgfqpoint{4.045389in}{1.748637in}}%
\pgfpathlineto{\pgfqpoint{4.047702in}{1.758763in}}%
\pgfpathlineto{\pgfqpoint{4.050014in}{1.754598in}}%
\pgfpathlineto{\pgfqpoint{4.052326in}{1.801059in}}%
\pgfpathlineto{\pgfqpoint{4.054639in}{1.819980in}}%
\pgfpathlineto{\pgfqpoint{4.056951in}{1.850212in}}%
\pgfpathlineto{\pgfqpoint{4.059263in}{1.908329in}}%
\pgfpathlineto{\pgfqpoint{4.061576in}{1.925919in}}%
\pgfpathlineto{\pgfqpoint{4.066200in}{2.017678in}}%
\pgfpathlineto{\pgfqpoint{4.068513in}{2.027734in}}%
\pgfpathlineto{\pgfqpoint{4.070825in}{2.073225in}}%
\pgfpathlineto{\pgfqpoint{4.073138in}{2.093936in}}%
\pgfpathlineto{\pgfqpoint{4.075450in}{2.094648in}}%
\pgfpathlineto{\pgfqpoint{4.077762in}{2.128006in}}%
\pgfpathlineto{\pgfqpoint{4.080075in}{2.122926in}}%
\pgfpathlineto{\pgfqpoint{4.082387in}{2.115137in}}%
\pgfpathlineto{\pgfqpoint{4.084699in}{2.131636in}}%
\pgfpathlineto{\pgfqpoint{4.089324in}{2.090359in}}%
\pgfpathlineto{\pgfqpoint{4.091636in}{2.089523in}}%
\pgfpathlineto{\pgfqpoint{4.093949in}{2.046496in}}%
\pgfpathlineto{\pgfqpoint{4.098573in}{2.011967in}}%
\pgfpathlineto{\pgfqpoint{4.100886in}{1.959150in}}%
\pgfpathlineto{\pgfqpoint{4.105510in}{1.910490in}}%
\pgfpathlineto{\pgfqpoint{4.107823in}{1.853346in}}%
\pgfpathlineto{\pgfqpoint{4.110135in}{1.832890in}}%
\pgfpathlineto{\pgfqpoint{4.112448in}{1.795735in}}%
\pgfpathlineto{\pgfqpoint{4.114760in}{1.738561in}}%
\pgfpathlineto{\pgfqpoint{4.117072in}{1.719222in}}%
\pgfpathlineto{\pgfqpoint{4.128634in}{1.510001in}}%
\pgfpathlineto{\pgfqpoint{4.130946in}{1.495058in}}%
\pgfpathlineto{\pgfqpoint{4.135571in}{1.405197in}}%
\pgfpathlineto{\pgfqpoint{4.137883in}{1.392871in}}%
\pgfpathlineto{\pgfqpoint{4.142508in}{1.309885in}}%
\pgfpathlineto{\pgfqpoint{4.144821in}{1.300159in}}%
\pgfpathlineto{\pgfqpoint{4.149445in}{1.224888in}}%
\pgfpathlineto{\pgfqpoint{4.151758in}{1.217616in}}%
\pgfpathlineto{\pgfqpoint{4.156382in}{1.150204in}}%
\pgfpathlineto{\pgfqpoint{4.158695in}{1.145165in}}%
\pgfpathlineto{\pgfqpoint{4.161007in}{1.104703in}}%
\pgfpathlineto{\pgfqpoint{4.163319in}{1.085296in}}%
\pgfpathlineto{\pgfqpoint{4.165632in}{1.082238in}}%
\pgfpathlineto{\pgfqpoint{4.167944in}{1.044072in}}%
\pgfpathlineto{\pgfqpoint{4.170256in}{1.029321in}}%
\pgfpathlineto{\pgfqpoint{4.172569in}{1.027987in}}%
\pgfpathlineto{\pgfqpoint{4.174881in}{0.992024in}}%
\pgfpathlineto{\pgfqpoint{4.177193in}{0.981296in}}%
\pgfpathlineto{\pgfqpoint{4.179506in}{0.981440in}}%
\pgfpathlineto{\pgfqpoint{4.181818in}{0.947507in}}%
\pgfpathlineto{\pgfqpoint{4.184131in}{0.940206in}}%
\pgfpathlineto{\pgfqpoint{4.186443in}{0.941606in}}%
\pgfpathlineto{\pgfqpoint{4.188755in}{0.909492in}}%
\pgfpathlineto{\pgfqpoint{4.191068in}{0.905080in}}%
\pgfpathlineto{\pgfqpoint{4.193380in}{0.907538in}}%
\pgfpathlineto{\pgfqpoint{4.195692in}{0.877025in}}%
\pgfpathlineto{\pgfqpoint{4.198005in}{0.875029in}}%
\pgfpathlineto{\pgfqpoint{4.200317in}{0.878376in}}%
\pgfpathlineto{\pgfqpoint{4.202629in}{0.849250in}}%
\pgfpathlineto{\pgfqpoint{4.204942in}{0.849264in}}%
\pgfpathlineto{\pgfqpoint{4.207254in}{0.853356in}}%
\pgfpathlineto{\pgfqpoint{4.209566in}{0.825422in}}%
\pgfpathlineto{\pgfqpoint{4.211879in}{0.827104in}}%
\pgfpathlineto{\pgfqpoint{4.214191in}{0.831819in}}%
\pgfpathlineto{\pgfqpoint{4.216503in}{0.804904in}}%
\pgfpathlineto{\pgfqpoint{4.218816in}{0.807967in}}%
\pgfpathlineto{\pgfqpoint{4.221128in}{0.813206in}}%
\pgfpathlineto{\pgfqpoint{4.223441in}{0.787158in}}%
\pgfpathlineto{\pgfqpoint{4.228065in}{0.797046in}}%
\pgfpathlineto{\pgfqpoint{4.230378in}{0.771735in}}%
\pgfpathlineto{\pgfqpoint{4.235002in}{0.782944in}}%
\pgfpathlineto{\pgfqpoint{4.237315in}{0.758262in}}%
\pgfpathlineto{\pgfqpoint{4.241939in}{0.770575in}}%
\pgfpathlineto{\pgfqpoint{4.244252in}{0.746430in}}%
\pgfpathlineto{\pgfqpoint{4.248876in}{0.759669in}}%
\pgfpathlineto{\pgfqpoint{4.251189in}{0.735984in}}%
\pgfpathlineto{\pgfqpoint{4.255814in}{0.750004in}}%
\pgfpathlineto{\pgfqpoint{4.258126in}{0.726716in}}%
\pgfpathlineto{\pgfqpoint{4.262751in}{0.741395in}}%
\pgfpathlineto{\pgfqpoint{4.265063in}{0.718450in}}%
\pgfpathlineto{\pgfqpoint{4.269688in}{0.733689in}}%
\pgfpathlineto{\pgfqpoint{4.272000in}{0.711043in}}%
\pgfpathlineto{\pgfqpoint{4.276625in}{0.726760in}}%
\pgfpathlineto{\pgfqpoint{4.278937in}{0.704375in}}%
\pgfpathlineto{\pgfqpoint{4.283562in}{0.720501in}}%
\pgfpathlineto{\pgfqpoint{4.285874in}{0.698347in}}%
\pgfpathlineto{\pgfqpoint{4.290499in}{0.714825in}}%
\pgfpathlineto{\pgfqpoint{4.292811in}{0.692874in}}%
\pgfpathlineto{\pgfqpoint{4.297436in}{0.709656in}}%
\pgfpathlineto{\pgfqpoint{4.299748in}{0.687887in}}%
\pgfpathlineto{\pgfqpoint{4.304373in}{0.704932in}}%
\pgfpathlineto{\pgfqpoint{4.306685in}{0.683327in}}%
\pgfpathlineto{\pgfqpoint{4.311310in}{0.700599in}}%
\pgfpathlineto{\pgfqpoint{4.313622in}{0.679142in}}%
\pgfpathlineto{\pgfqpoint{4.318247in}{0.696613in}}%
\pgfpathlineto{\pgfqpoint{4.320559in}{0.675289in}}%
\pgfpathlineto{\pgfqpoint{4.325184in}{0.692934in}}%
\pgfpathlineto{\pgfqpoint{4.327497in}{0.671732in}}%
\pgfpathlineto{\pgfqpoint{4.332121in}{0.689529in}}%
\pgfpathlineto{\pgfqpoint{4.334434in}{0.668439in}}%
\pgfpathlineto{\pgfqpoint{4.339058in}{0.686369in}}%
\pgfpathlineto{\pgfqpoint{4.341371in}{0.665382in}}%
\pgfpathlineto{\pgfqpoint{4.345995in}{0.683429in}}%
\pgfpathlineto{\pgfqpoint{4.348308in}{0.662538in}}%
\pgfpathlineto{\pgfqpoint{4.352932in}{0.680688in}}%
\pgfpathlineto{\pgfqpoint{4.355245in}{0.659885in}}%
\pgfpathlineto{\pgfqpoint{4.359869in}{0.678126in}}%
\pgfpathlineto{\pgfqpoint{4.362182in}{0.657406in}}%
\pgfpathlineto{\pgfqpoint{4.366807in}{0.675726in}}%
\pgfpathlineto{\pgfqpoint{4.369119in}{0.655084in}}%
\pgfpathlineto{\pgfqpoint{4.373744in}{0.673474in}}%
\pgfpathlineto{\pgfqpoint{4.376056in}{0.652906in}}%
\pgfpathlineto{\pgfqpoint{4.380681in}{0.671356in}}%
\pgfpathlineto{\pgfqpoint{4.382993in}{0.650858in}}%
\pgfpathlineto{\pgfqpoint{4.387618in}{0.669362in}}%
\pgfpathlineto{\pgfqpoint{4.389930in}{0.648930in}}%
\pgfpathlineto{\pgfqpoint{4.394555in}{0.667481in}}%
\pgfpathlineto{\pgfqpoint{4.396867in}{0.647112in}}%
\pgfpathlineto{\pgfqpoint{4.401492in}{0.665704in}}%
\pgfpathlineto{\pgfqpoint{4.403804in}{0.645395in}}%
\pgfpathlineto{\pgfqpoint{4.408429in}{0.664022in}}%
\pgfpathlineto{\pgfqpoint{4.410741in}{0.643771in}}%
\pgfpathlineto{\pgfqpoint{4.415366in}{0.662428in}}%
\pgfpathlineto{\pgfqpoint{4.417678in}{0.642233in}}%
\pgfpathlineto{\pgfqpoint{4.422303in}{0.660915in}}%
\pgfpathlineto{\pgfqpoint{4.424615in}{0.640774in}}%
\pgfpathlineto{\pgfqpoint{4.429240in}{0.659479in}}%
\pgfpathlineto{\pgfqpoint{4.431552in}{0.639389in}}%
\pgfpathlineto{\pgfqpoint{4.436177in}{0.658112in}}%
\pgfpathlineto{\pgfqpoint{4.438490in}{0.638072in}}%
\pgfpathlineto{\pgfqpoint{4.443114in}{0.656810in}}%
\pgfpathlineto{\pgfqpoint{4.445427in}{0.636820in}}%
\pgfpathlineto{\pgfqpoint{4.450051in}{0.655569in}}%
\pgfpathlineto{\pgfqpoint{4.452364in}{0.635626in}}%
\pgfpathlineto{\pgfqpoint{4.456988in}{0.654385in}}%
\pgfpathlineto{\pgfqpoint{4.459301in}{0.634488in}}%
\pgfpathlineto{\pgfqpoint{4.463925in}{0.653254in}}%
\pgfpathlineto{\pgfqpoint{4.466238in}{0.633402in}}%
\pgfpathlineto{\pgfqpoint{4.470862in}{0.652172in}}%
\pgfpathlineto{\pgfqpoint{4.473175in}{0.632365in}}%
\pgfpathlineto{\pgfqpoint{4.477800in}{0.651136in}}%
\pgfpathlineto{\pgfqpoint{4.480112in}{0.631373in}}%
\pgfpathlineto{\pgfqpoint{4.484737in}{0.650144in}}%
\pgfpathlineto{\pgfqpoint{4.487049in}{0.630424in}}%
\pgfpathlineto{\pgfqpoint{4.491674in}{0.649193in}}%
\pgfpathlineto{\pgfqpoint{4.493986in}{0.629515in}}%
\pgfpathlineto{\pgfqpoint{4.498611in}{0.648281in}}%
\pgfpathlineto{\pgfqpoint{4.500923in}{0.628644in}}%
\pgfpathlineto{\pgfqpoint{4.505548in}{0.647404in}}%
\pgfpathlineto{\pgfqpoint{4.507860in}{0.627808in}}%
\pgfpathlineto{\pgfqpoint{4.512485in}{0.646562in}}%
\pgfpathlineto{\pgfqpoint{4.514797in}{0.627006in}}%
\pgfpathlineto{\pgfqpoint{4.519422in}{0.645753in}}%
\pgfpathlineto{\pgfqpoint{4.521734in}{0.626236in}}%
\pgfpathlineto{\pgfqpoint{4.526359in}{0.644973in}}%
\pgfpathlineto{\pgfqpoint{4.528671in}{0.625496in}}%
\pgfpathlineto{\pgfqpoint{4.533296in}{0.644223in}}%
\pgfpathlineto{\pgfqpoint{4.535608in}{0.624785in}}%
\pgfpathlineto{\pgfqpoint{4.540233in}{0.643500in}}%
\pgfpathlineto{\pgfqpoint{4.542545in}{0.624101in}}%
\pgfpathlineto{\pgfqpoint{4.547170in}{0.642803in}}%
\pgfpathlineto{\pgfqpoint{4.549483in}{0.623442in}}%
\pgfpathlineto{\pgfqpoint{4.554107in}{0.642130in}}%
\pgfpathlineto{\pgfqpoint{4.556420in}{0.622807in}}%
\pgfpathlineto{\pgfqpoint{4.561044in}{0.641481in}}%
\pgfpathlineto{\pgfqpoint{4.563357in}{0.622196in}}%
\pgfpathlineto{\pgfqpoint{4.567981in}{0.640854in}}%
\pgfpathlineto{\pgfqpoint{4.570294in}{0.621607in}}%
\pgfpathlineto{\pgfqpoint{4.574918in}{0.640248in}}%
\pgfpathlineto{\pgfqpoint{4.577231in}{0.621038in}}%
\pgfpathlineto{\pgfqpoint{4.581855in}{0.639662in}}%
\pgfpathlineto{\pgfqpoint{4.584168in}{0.620490in}}%
\pgfpathlineto{\pgfqpoint{4.588793in}{0.639096in}}%
\pgfpathlineto{\pgfqpoint{4.591105in}{0.619960in}}%
\pgfpathlineto{\pgfqpoint{4.595730in}{0.638547in}}%
\pgfpathlineto{\pgfqpoint{4.598042in}{0.619449in}}%
\pgfpathlineto{\pgfqpoint{4.602667in}{0.638017in}}%
\pgfpathlineto{\pgfqpoint{4.604979in}{0.618955in}}%
\pgfpathlineto{\pgfqpoint{4.609604in}{0.637503in}}%
\pgfpathlineto{\pgfqpoint{4.611916in}{0.618478in}}%
\pgfpathlineto{\pgfqpoint{4.616541in}{0.637004in}}%
\pgfpathlineto{\pgfqpoint{4.618853in}{0.618017in}}%
\pgfpathlineto{\pgfqpoint{4.623478in}{0.636522in}}%
\pgfpathlineto{\pgfqpoint{4.625790in}{0.617570in}}%
\pgfpathlineto{\pgfqpoint{4.630415in}{0.636053in}}%
\pgfpathlineto{\pgfqpoint{4.632727in}{0.617139in}}%
\pgfpathlineto{\pgfqpoint{4.637352in}{0.635599in}}%
\pgfpathlineto{\pgfqpoint{4.639664in}{0.616721in}}%
\pgfpathlineto{\pgfqpoint{4.644289in}{0.635158in}}%
\pgfpathlineto{\pgfqpoint{4.646601in}{0.616316in}}%
\pgfpathlineto{\pgfqpoint{4.651226in}{0.634730in}}%
\pgfpathlineto{\pgfqpoint{4.653538in}{0.615925in}}%
\pgfpathlineto{\pgfqpoint{4.658163in}{0.634314in}}%
\pgfpathlineto{\pgfqpoint{4.660476in}{0.615546in}}%
\pgfpathlineto{\pgfqpoint{4.665100in}{0.633911in}}%
\pgfpathlineto{\pgfqpoint{4.667413in}{0.615178in}}%
\pgfpathlineto{\pgfqpoint{4.672037in}{0.633518in}}%
\pgfpathlineto{\pgfqpoint{4.674350in}{0.614823in}}%
\pgfpathlineto{\pgfqpoint{4.678974in}{0.633137in}}%
\pgfpathlineto{\pgfqpoint{4.681287in}{0.614478in}}%
\pgfpathlineto{\pgfqpoint{4.685911in}{0.632766in}}%
\pgfpathlineto{\pgfqpoint{4.688224in}{0.614143in}}%
\pgfpathlineto{\pgfqpoint{4.692848in}{0.632405in}}%
\pgfpathlineto{\pgfqpoint{4.695161in}{0.613819in}}%
\pgfpathlineto{\pgfqpoint{4.699786in}{0.632054in}}%
\pgfpathlineto{\pgfqpoint{4.702098in}{0.613505in}}%
\pgfpathlineto{\pgfqpoint{4.706723in}{0.631712in}}%
\pgfpathlineto{\pgfqpoint{4.709035in}{0.613200in}}%
\pgfpathlineto{\pgfqpoint{4.713660in}{0.631379in}}%
\pgfpathlineto{\pgfqpoint{4.715972in}{0.612904in}}%
\pgfpathlineto{\pgfqpoint{4.720597in}{0.631055in}}%
\pgfpathlineto{\pgfqpoint{4.722909in}{0.612617in}}%
\pgfpathlineto{\pgfqpoint{4.727534in}{0.630740in}}%
\pgfpathlineto{\pgfqpoint{4.729846in}{0.612338in}}%
\pgfpathlineto{\pgfqpoint{4.734471in}{0.630433in}}%
\pgfpathlineto{\pgfqpoint{4.736783in}{0.612068in}}%
\pgfpathlineto{\pgfqpoint{4.741408in}{0.630133in}}%
\pgfpathlineto{\pgfqpoint{4.743720in}{0.611805in}}%
\pgfpathlineto{\pgfqpoint{4.748345in}{0.629841in}}%
\pgfpathlineto{\pgfqpoint{4.750657in}{0.611551in}}%
\pgfpathlineto{\pgfqpoint{4.755282in}{0.629557in}}%
\pgfpathlineto{\pgfqpoint{4.757594in}{0.611303in}}%
\pgfpathlineto{\pgfqpoint{4.762219in}{0.629279in}}%
\pgfpathlineto{\pgfqpoint{4.764531in}{0.611063in}}%
\pgfpathlineto{\pgfqpoint{4.769156in}{0.629009in}}%
\pgfpathlineto{\pgfqpoint{4.771469in}{0.610830in}}%
\pgfpathlineto{\pgfqpoint{4.776093in}{0.628745in}}%
\pgfpathlineto{\pgfqpoint{4.778406in}{0.610604in}}%
\pgfpathlineto{\pgfqpoint{4.783030in}{0.628487in}}%
\pgfpathlineto{\pgfqpoint{4.785343in}{0.610384in}}%
\pgfpathlineto{\pgfqpoint{4.789967in}{0.628236in}}%
\pgfpathlineto{\pgfqpoint{4.792280in}{0.610170in}}%
\pgfpathlineto{\pgfqpoint{4.796904in}{0.627991in}}%
\pgfpathlineto{\pgfqpoint{4.799217in}{0.609963in}}%
\pgfpathlineto{\pgfqpoint{4.803841in}{0.627751in}}%
\pgfpathlineto{\pgfqpoint{4.806154in}{0.609761in}}%
\pgfpathlineto{\pgfqpoint{4.810779in}{0.627517in}}%
\pgfpathlineto{\pgfqpoint{4.813091in}{0.609566in}}%
\pgfpathlineto{\pgfqpoint{4.817716in}{0.627289in}}%
\pgfpathlineto{\pgfqpoint{4.820028in}{0.609376in}}%
\pgfpathlineto{\pgfqpoint{4.824653in}{0.627066in}}%
\pgfpathlineto{\pgfqpoint{4.826965in}{0.609191in}}%
\pgfpathlineto{\pgfqpoint{4.831590in}{0.626849in}}%
\pgfpathlineto{\pgfqpoint{4.833902in}{0.609012in}}%
\pgfpathlineto{\pgfqpoint{4.838527in}{0.626636in}}%
\pgfpathlineto{\pgfqpoint{4.840839in}{0.608838in}}%
\pgfpathlineto{\pgfqpoint{4.845464in}{0.626428in}}%
\pgfpathlineto{\pgfqpoint{4.847776in}{0.608669in}}%
\pgfpathlineto{\pgfqpoint{4.850089in}{0.615673in}}%
\pgfpathlineto{\pgfqpoint{4.852401in}{0.626225in}}%
\pgfpathlineto{\pgfqpoint{4.854713in}{0.608504in}}%
\pgfpathlineto{\pgfqpoint{4.857026in}{0.615449in}}%
\pgfpathlineto{\pgfqpoint{4.859338in}{0.626027in}}%
\pgfpathlineto{\pgfqpoint{4.861650in}{0.608345in}}%
\pgfpathlineto{\pgfqpoint{4.863963in}{0.615228in}}%
\pgfpathlineto{\pgfqpoint{4.866275in}{0.625833in}}%
\pgfpathlineto{\pgfqpoint{4.868587in}{0.608190in}}%
\pgfpathlineto{\pgfqpoint{4.870900in}{0.615013in}}%
\pgfpathlineto{\pgfqpoint{4.873212in}{0.625643in}}%
\pgfpathlineto{\pgfqpoint{4.875524in}{0.608039in}}%
\pgfpathlineto{\pgfqpoint{4.877837in}{0.614801in}}%
\pgfpathlineto{\pgfqpoint{4.880149in}{0.625458in}}%
\pgfpathlineto{\pgfqpoint{4.882462in}{0.607893in}}%
\pgfpathlineto{\pgfqpoint{4.884774in}{0.614593in}}%
\pgfpathlineto{\pgfqpoint{4.887086in}{0.625277in}}%
\pgfpathlineto{\pgfqpoint{4.889399in}{0.607752in}}%
\pgfpathlineto{\pgfqpoint{4.891711in}{0.614390in}}%
\pgfpathlineto{\pgfqpoint{4.894023in}{0.625100in}}%
\pgfpathlineto{\pgfqpoint{4.896336in}{0.607614in}}%
\pgfpathlineto{\pgfqpoint{4.898648in}{0.614190in}}%
\pgfpathlineto{\pgfqpoint{4.900960in}{0.624927in}}%
\pgfpathlineto{\pgfqpoint{4.903273in}{0.607481in}}%
\pgfpathlineto{\pgfqpoint{4.905585in}{0.613994in}}%
\pgfpathlineto{\pgfqpoint{4.907897in}{0.624757in}}%
\pgfpathlineto{\pgfqpoint{4.910210in}{0.607351in}}%
\pgfpathlineto{\pgfqpoint{4.912522in}{0.613801in}}%
\pgfpathlineto{\pgfqpoint{4.914834in}{0.624592in}}%
\pgfpathlineto{\pgfqpoint{4.917147in}{0.607225in}}%
\pgfpathlineto{\pgfqpoint{4.919459in}{0.613612in}}%
\pgfpathlineto{\pgfqpoint{4.921772in}{0.624430in}}%
\pgfpathlineto{\pgfqpoint{4.924084in}{0.607103in}}%
\pgfpathlineto{\pgfqpoint{4.926396in}{0.613427in}}%
\pgfpathlineto{\pgfqpoint{4.928709in}{0.624271in}}%
\pgfpathlineto{\pgfqpoint{4.931021in}{0.606985in}}%
\pgfpathlineto{\pgfqpoint{4.933333in}{0.613245in}}%
\pgfpathlineto{\pgfqpoint{4.935646in}{0.624116in}}%
\pgfpathlineto{\pgfqpoint{4.937958in}{0.606870in}}%
\pgfpathlineto{\pgfqpoint{4.940270in}{0.613066in}}%
\pgfpathlineto{\pgfqpoint{4.942583in}{0.623965in}}%
\pgfpathlineto{\pgfqpoint{4.944895in}{0.606759in}}%
\pgfpathlineto{\pgfqpoint{4.947207in}{0.612891in}}%
\pgfpathlineto{\pgfqpoint{4.949520in}{0.623817in}}%
\pgfpathlineto{\pgfqpoint{4.951832in}{0.606651in}}%
\pgfpathlineto{\pgfqpoint{4.954145in}{0.612718in}}%
\pgfpathlineto{\pgfqpoint{4.956457in}{0.623672in}}%
\pgfpathlineto{\pgfqpoint{4.958769in}{0.606546in}}%
\pgfpathlineto{\pgfqpoint{4.961082in}{0.612549in}}%
\pgfpathlineto{\pgfqpoint{4.963394in}{0.623530in}}%
\pgfpathlineto{\pgfqpoint{4.965706in}{0.606445in}}%
\pgfpathlineto{\pgfqpoint{4.968019in}{0.612383in}}%
\pgfpathlineto{\pgfqpoint{4.970331in}{0.623391in}}%
\pgfpathlineto{\pgfqpoint{4.972643in}{0.606347in}}%
\pgfpathlineto{\pgfqpoint{4.974956in}{0.612219in}}%
\pgfpathlineto{\pgfqpoint{4.977268in}{0.623255in}}%
\pgfpathlineto{\pgfqpoint{4.979580in}{0.606252in}}%
\pgfpathlineto{\pgfqpoint{4.981893in}{0.612059in}}%
\pgfpathlineto{\pgfqpoint{4.984205in}{0.623122in}}%
\pgfpathlineto{\pgfqpoint{4.986517in}{0.606160in}}%
\pgfpathlineto{\pgfqpoint{4.988830in}{0.611901in}}%
\pgfpathlineto{\pgfqpoint{4.991142in}{0.622992in}}%
\pgfpathlineto{\pgfqpoint{4.993455in}{0.606071in}}%
\pgfpathlineto{\pgfqpoint{4.995767in}{0.611746in}}%
\pgfpathlineto{\pgfqpoint{4.998079in}{0.622865in}}%
\pgfpathlineto{\pgfqpoint{5.000392in}{0.605985in}}%
\pgfpathlineto{\pgfqpoint{5.002704in}{0.611593in}}%
\pgfpathlineto{\pgfqpoint{5.005016in}{0.622741in}}%
\pgfpathlineto{\pgfqpoint{5.007329in}{0.605902in}}%
\pgfpathlineto{\pgfqpoint{5.009641in}{0.611443in}}%
\pgfpathlineto{\pgfqpoint{5.011953in}{0.622619in}}%
\pgfpathlineto{\pgfqpoint{5.014266in}{0.605821in}}%
\pgfpathlineto{\pgfqpoint{5.016578in}{0.611296in}}%
\pgfpathlineto{\pgfqpoint{5.018890in}{0.622500in}}%
\pgfpathlineto{\pgfqpoint{5.021203in}{0.605744in}}%
\pgfpathlineto{\pgfqpoint{5.023515in}{0.611151in}}%
\pgfpathlineto{\pgfqpoint{5.025828in}{0.622384in}}%
\pgfpathlineto{\pgfqpoint{5.028140in}{0.605669in}}%
\pgfpathlineto{\pgfqpoint{5.030452in}{0.611009in}}%
\pgfpathlineto{\pgfqpoint{5.032765in}{0.622270in}}%
\pgfpathlineto{\pgfqpoint{5.035077in}{0.605596in}}%
\pgfpathlineto{\pgfqpoint{5.037389in}{0.610869in}}%
\pgfpathlineto{\pgfqpoint{5.039702in}{0.622158in}}%
\pgfpathlineto{\pgfqpoint{5.042014in}{0.605526in}}%
\pgfpathlineto{\pgfqpoint{5.044326in}{0.610731in}}%
\pgfpathlineto{\pgfqpoint{5.046639in}{0.622049in}}%
\pgfpathlineto{\pgfqpoint{5.048951in}{0.605459in}}%
\pgfpathlineto{\pgfqpoint{5.051263in}{0.610595in}}%
\pgfpathlineto{\pgfqpoint{5.053576in}{0.621943in}}%
\pgfpathlineto{\pgfqpoint{5.055888in}{0.605394in}}%
\pgfpathlineto{\pgfqpoint{5.058200in}{0.610462in}}%
\pgfpathlineto{\pgfqpoint{5.060513in}{0.621838in}}%
\pgfpathlineto{\pgfqpoint{5.062825in}{0.605332in}}%
\pgfpathlineto{\pgfqpoint{5.065138in}{0.610331in}}%
\pgfpathlineto{\pgfqpoint{5.067450in}{0.621736in}}%
\pgfpathlineto{\pgfqpoint{5.069762in}{0.605271in}}%
\pgfpathlineto{\pgfqpoint{5.072075in}{0.610202in}}%
\pgfpathlineto{\pgfqpoint{5.074387in}{0.621636in}}%
\pgfpathlineto{\pgfqpoint{5.076699in}{0.605214in}}%
\pgfpathlineto{\pgfqpoint{5.079012in}{0.610075in}}%
\pgfpathlineto{\pgfqpoint{5.081324in}{0.621539in}}%
\pgfpathlineto{\pgfqpoint{5.083636in}{0.605158in}}%
\pgfpathlineto{\pgfqpoint{5.085949in}{0.609950in}}%
\pgfpathlineto{\pgfqpoint{5.088261in}{0.621444in}}%
\pgfpathlineto{\pgfqpoint{5.090573in}{0.605105in}}%
\pgfpathlineto{\pgfqpoint{5.092886in}{0.609828in}}%
\pgfpathlineto{\pgfqpoint{5.095198in}{0.621350in}}%
\pgfpathlineto{\pgfqpoint{5.097510in}{0.605054in}}%
\pgfpathlineto{\pgfqpoint{5.099823in}{0.609707in}}%
\pgfpathlineto{\pgfqpoint{5.102135in}{0.621259in}}%
\pgfpathlineto{\pgfqpoint{5.104448in}{0.605004in}}%
\pgfpathlineto{\pgfqpoint{5.106760in}{0.609588in}}%
\pgfpathlineto{\pgfqpoint{5.109072in}{0.621170in}}%
\pgfpathlineto{\pgfqpoint{5.111385in}{0.604958in}}%
\pgfpathlineto{\pgfqpoint{5.113697in}{0.609471in}}%
\pgfpathlineto{\pgfqpoint{5.116009in}{0.621083in}}%
\pgfpathlineto{\pgfqpoint{5.118322in}{0.604913in}}%
\pgfpathlineto{\pgfqpoint{5.120634in}{0.609356in}}%
\pgfpathlineto{\pgfqpoint{5.122946in}{0.620998in}}%
\pgfpathlineto{\pgfqpoint{5.125259in}{0.604870in}}%
\pgfpathlineto{\pgfqpoint{5.127571in}{0.609243in}}%
\pgfpathlineto{\pgfqpoint{5.129883in}{0.620915in}}%
\pgfpathlineto{\pgfqpoint{5.132196in}{0.604829in}}%
\pgfpathlineto{\pgfqpoint{5.134508in}{0.609131in}}%
\pgfpathlineto{\pgfqpoint{5.136821in}{0.620834in}}%
\pgfpathlineto{\pgfqpoint{5.139133in}{0.604790in}}%
\pgfpathlineto{\pgfqpoint{5.141445in}{0.609021in}}%
\pgfpathlineto{\pgfqpoint{5.143758in}{0.620754in}}%
\pgfpathlineto{\pgfqpoint{5.146070in}{0.604753in}}%
\pgfpathlineto{\pgfqpoint{5.148382in}{0.608913in}}%
\pgfpathlineto{\pgfqpoint{5.150695in}{0.620677in}}%
\pgfpathlineto{\pgfqpoint{5.153007in}{0.604718in}}%
\pgfpathlineto{\pgfqpoint{5.155319in}{0.608807in}}%
\pgfpathlineto{\pgfqpoint{5.157632in}{0.620601in}}%
\pgfpathlineto{\pgfqpoint{5.159944in}{0.604685in}}%
\pgfpathlineto{\pgfqpoint{5.162256in}{0.608702in}}%
\pgfpathlineto{\pgfqpoint{5.164569in}{0.620527in}}%
\pgfpathlineto{\pgfqpoint{5.166881in}{0.604653in}}%
\pgfpathlineto{\pgfqpoint{5.169193in}{0.608599in}}%
\pgfpathlineto{\pgfqpoint{5.171506in}{0.620455in}}%
\pgfpathlineto{\pgfqpoint{5.173818in}{0.604623in}}%
\pgfpathlineto{\pgfqpoint{5.176131in}{0.608498in}}%
\pgfpathlineto{\pgfqpoint{5.178443in}{0.620385in}}%
\pgfpathlineto{\pgfqpoint{5.180755in}{0.604596in}}%
\pgfpathlineto{\pgfqpoint{5.183068in}{0.608398in}}%
\pgfpathlineto{\pgfqpoint{5.185380in}{0.620316in}}%
\pgfpathlineto{\pgfqpoint{5.187692in}{0.604569in}}%
\pgfpathlineto{\pgfqpoint{5.190005in}{0.608300in}}%
\pgfpathlineto{\pgfqpoint{5.192317in}{0.620250in}}%
\pgfpathlineto{\pgfqpoint{5.194629in}{0.604545in}}%
\pgfpathlineto{\pgfqpoint{5.196942in}{0.608203in}}%
\pgfpathlineto{\pgfqpoint{5.199254in}{0.620184in}}%
\pgfpathlineto{\pgfqpoint{5.201566in}{0.604522in}}%
\pgfpathlineto{\pgfqpoint{5.203879in}{0.608108in}}%
\pgfpathlineto{\pgfqpoint{5.206191in}{0.620121in}}%
\pgfpathlineto{\pgfqpoint{5.208503in}{0.604501in}}%
\pgfpathlineto{\pgfqpoint{5.210816in}{0.608014in}}%
\pgfpathlineto{\pgfqpoint{5.213128in}{0.620059in}}%
\pgfpathlineto{\pgfqpoint{5.215441in}{0.604481in}}%
\pgfpathlineto{\pgfqpoint{5.217753in}{0.607922in}}%
\pgfpathlineto{\pgfqpoint{5.220065in}{0.619999in}}%
\pgfpathlineto{\pgfqpoint{5.222378in}{0.604463in}}%
\pgfpathlineto{\pgfqpoint{5.224690in}{0.607831in}}%
\pgfpathlineto{\pgfqpoint{5.227002in}{0.619940in}}%
\pgfpathlineto{\pgfqpoint{5.229315in}{0.604447in}}%
\pgfpathlineto{\pgfqpoint{5.231627in}{0.607742in}}%
\pgfpathlineto{\pgfqpoint{5.233939in}{0.619883in}}%
\pgfpathlineto{\pgfqpoint{5.236252in}{0.604432in}}%
\pgfpathlineto{\pgfqpoint{5.238564in}{0.607654in}}%
\pgfpathlineto{\pgfqpoint{5.240876in}{0.619827in}}%
\pgfpathlineto{\pgfqpoint{5.243189in}{0.604419in}}%
\pgfpathlineto{\pgfqpoint{5.245501in}{0.607568in}}%
\pgfpathlineto{\pgfqpoint{5.247814in}{0.619773in}}%
\pgfpathlineto{\pgfqpoint{5.250126in}{0.604407in}}%
\pgfpathlineto{\pgfqpoint{5.252438in}{0.607482in}}%
\pgfpathlineto{\pgfqpoint{5.254751in}{0.619720in}}%
\pgfpathlineto{\pgfqpoint{5.257063in}{0.604396in}}%
\pgfpathlineto{\pgfqpoint{5.259375in}{0.607399in}}%
\pgfpathlineto{\pgfqpoint{5.261688in}{0.619669in}}%
\pgfpathlineto{\pgfqpoint{5.264000in}{0.604388in}}%
\pgfpathlineto{\pgfqpoint{5.266312in}{0.607316in}}%
\pgfpathlineto{\pgfqpoint{5.268625in}{0.619620in}}%
\pgfpathlineto{\pgfqpoint{5.270937in}{0.604380in}}%
\pgfpathlineto{\pgfqpoint{5.273249in}{0.607235in}}%
\pgfpathlineto{\pgfqpoint{5.275562in}{0.619572in}}%
\pgfpathlineto{\pgfqpoint{5.277874in}{0.604374in}}%
\pgfpathlineto{\pgfqpoint{5.280186in}{0.607155in}}%
\pgfpathlineto{\pgfqpoint{5.282499in}{0.619525in}}%
\pgfpathlineto{\pgfqpoint{5.284811in}{0.604369in}}%
\pgfpathlineto{\pgfqpoint{5.287124in}{0.607077in}}%
\pgfpathlineto{\pgfqpoint{5.289436in}{0.619480in}}%
\pgfpathlineto{\pgfqpoint{5.291748in}{0.604366in}}%
\pgfpathlineto{\pgfqpoint{5.294061in}{0.606999in}}%
\pgfpathlineto{\pgfqpoint{5.296373in}{0.619436in}}%
\pgfpathlineto{\pgfqpoint{5.298685in}{0.604364in}}%
\pgfpathlineto{\pgfqpoint{5.300998in}{0.606923in}}%
\pgfpathlineto{\pgfqpoint{5.303310in}{0.619393in}}%
\pgfpathlineto{\pgfqpoint{5.305622in}{0.604364in}}%
\pgfpathlineto{\pgfqpoint{5.307935in}{0.606849in}}%
\pgfpathlineto{\pgfqpoint{5.310247in}{0.619352in}}%
\pgfpathlineto{\pgfqpoint{5.312559in}{0.604364in}}%
\pgfpathlineto{\pgfqpoint{5.314872in}{0.606775in}}%
\pgfpathlineto{\pgfqpoint{5.317184in}{0.619313in}}%
\pgfpathlineto{\pgfqpoint{5.319497in}{0.604366in}}%
\pgfpathlineto{\pgfqpoint{5.321809in}{0.606703in}}%
\pgfpathlineto{\pgfqpoint{5.324121in}{0.619274in}}%
\pgfpathlineto{\pgfqpoint{5.326434in}{0.604370in}}%
\pgfpathlineto{\pgfqpoint{5.328746in}{0.606632in}}%
\pgfpathlineto{\pgfqpoint{5.331058in}{0.619237in}}%
\pgfpathlineto{\pgfqpoint{5.333371in}{0.604374in}}%
\pgfpathlineto{\pgfqpoint{5.335683in}{0.606562in}}%
\pgfpathlineto{\pgfqpoint{5.337995in}{0.619202in}}%
\pgfpathlineto{\pgfqpoint{5.340308in}{0.604380in}}%
\pgfpathlineto{\pgfqpoint{5.342620in}{0.606493in}}%
\pgfpathlineto{\pgfqpoint{5.344932in}{0.619167in}}%
\pgfpathlineto{\pgfqpoint{5.347245in}{0.604388in}}%
\pgfpathlineto{\pgfqpoint{5.349557in}{0.606425in}}%
\pgfpathlineto{\pgfqpoint{5.351869in}{0.619134in}}%
\pgfpathlineto{\pgfqpoint{5.354182in}{0.604396in}}%
\pgfpathlineto{\pgfqpoint{5.356494in}{0.606359in}}%
\pgfpathlineto{\pgfqpoint{5.358807in}{0.619103in}}%
\pgfpathlineto{\pgfqpoint{5.361119in}{0.604406in}}%
\pgfpathlineto{\pgfqpoint{5.363431in}{0.606293in}}%
\pgfpathlineto{\pgfqpoint{5.365744in}{0.619072in}}%
\pgfpathlineto{\pgfqpoint{5.368056in}{0.604416in}}%
\pgfpathlineto{\pgfqpoint{5.370368in}{0.606229in}}%
\pgfpathlineto{\pgfqpoint{5.372681in}{0.619043in}}%
\pgfpathlineto{\pgfqpoint{5.374993in}{0.604428in}}%
\pgfpathlineto{\pgfqpoint{5.377305in}{0.606166in}}%
\pgfpathlineto{\pgfqpoint{5.379618in}{0.619015in}}%
\pgfpathlineto{\pgfqpoint{5.381930in}{0.604442in}}%
\pgfpathlineto{\pgfqpoint{5.384242in}{0.606104in}}%
\pgfpathlineto{\pgfqpoint{5.386555in}{0.618989in}}%
\pgfpathlineto{\pgfqpoint{5.388867in}{0.604456in}}%
\pgfpathlineto{\pgfqpoint{5.391179in}{0.606044in}}%
\pgfpathlineto{\pgfqpoint{5.393492in}{0.618964in}}%
\pgfpathlineto{\pgfqpoint{5.395804in}{0.604472in}}%
\pgfpathlineto{\pgfqpoint{5.398117in}{0.605984in}}%
\pgfpathlineto{\pgfqpoint{5.400429in}{0.618940in}}%
\pgfpathlineto{\pgfqpoint{5.402741in}{0.604488in}}%
\pgfpathlineto{\pgfqpoint{5.405054in}{0.605925in}}%
\pgfpathlineto{\pgfqpoint{5.407366in}{0.618917in}}%
\pgfpathlineto{\pgfqpoint{5.409678in}{0.604506in}}%
\pgfpathlineto{\pgfqpoint{5.411991in}{0.605868in}}%
\pgfpathlineto{\pgfqpoint{5.414303in}{0.618895in}}%
\pgfpathlineto{\pgfqpoint{5.416615in}{0.604525in}}%
\pgfpathlineto{\pgfqpoint{5.418928in}{0.605812in}}%
\pgfpathlineto{\pgfqpoint{5.421240in}{0.618875in}}%
\pgfpathlineto{\pgfqpoint{5.423552in}{0.604545in}}%
\pgfpathlineto{\pgfqpoint{5.425865in}{0.605756in}}%
\pgfpathlineto{\pgfqpoint{5.428177in}{0.618856in}}%
\pgfpathlineto{\pgfqpoint{5.430490in}{0.604567in}}%
\pgfpathlineto{\pgfqpoint{5.432802in}{0.605702in}}%
\pgfpathlineto{\pgfqpoint{5.435114in}{0.618838in}}%
\pgfpathlineto{\pgfqpoint{5.437427in}{0.604589in}}%
\pgfpathlineto{\pgfqpoint{5.439739in}{0.605649in}}%
\pgfpathlineto{\pgfqpoint{5.442051in}{0.618821in}}%
\pgfpathlineto{\pgfqpoint{5.444364in}{0.604612in}}%
\pgfpathlineto{\pgfqpoint{5.446676in}{0.605597in}}%
\pgfpathlineto{\pgfqpoint{5.448988in}{0.618806in}}%
\pgfpathlineto{\pgfqpoint{5.451301in}{0.604637in}}%
\pgfpathlineto{\pgfqpoint{5.453613in}{0.605546in}}%
\pgfpathlineto{\pgfqpoint{5.455925in}{0.618792in}}%
\pgfpathlineto{\pgfqpoint{5.458238in}{0.604663in}}%
\pgfpathlineto{\pgfqpoint{5.460550in}{0.605496in}}%
\pgfpathlineto{\pgfqpoint{5.462862in}{0.618779in}}%
\pgfpathlineto{\pgfqpoint{5.465175in}{0.604689in}}%
\pgfpathlineto{\pgfqpoint{5.467487in}{0.605447in}}%
\pgfpathlineto{\pgfqpoint{5.469800in}{0.618767in}}%
\pgfpathlineto{\pgfqpoint{5.472112in}{0.604717in}}%
\pgfpathlineto{\pgfqpoint{5.474424in}{0.605399in}}%
\pgfpathlineto{\pgfqpoint{5.476737in}{0.618756in}}%
\pgfpathlineto{\pgfqpoint{5.479049in}{0.604746in}}%
\pgfpathlineto{\pgfqpoint{5.481361in}{0.605352in}}%
\pgfpathlineto{\pgfqpoint{5.483674in}{0.618747in}}%
\pgfpathlineto{\pgfqpoint{5.485986in}{0.604776in}}%
\pgfpathlineto{\pgfqpoint{5.488298in}{0.605307in}}%
\pgfpathlineto{\pgfqpoint{5.490611in}{0.618738in}}%
\pgfpathlineto{\pgfqpoint{5.492923in}{0.604807in}}%
\pgfpathlineto{\pgfqpoint{5.495235in}{0.605262in}}%
\pgfpathlineto{\pgfqpoint{5.497548in}{0.618731in}}%
\pgfpathlineto{\pgfqpoint{5.499860in}{0.604839in}}%
\pgfpathlineto{\pgfqpoint{5.502172in}{0.605218in}}%
\pgfpathlineto{\pgfqpoint{5.504485in}{0.618725in}}%
\pgfpathlineto{\pgfqpoint{5.506797in}{0.604872in}}%
\pgfpathlineto{\pgfqpoint{5.509110in}{0.605176in}}%
\pgfpathlineto{\pgfqpoint{5.511422in}{0.618721in}}%
\pgfpathlineto{\pgfqpoint{5.513734in}{0.604907in}}%
\pgfpathlineto{\pgfqpoint{5.516047in}{0.605134in}}%
\pgfpathlineto{\pgfqpoint{5.518359in}{0.618717in}}%
\pgfpathlineto{\pgfqpoint{5.520671in}{0.604942in}}%
\pgfpathlineto{\pgfqpoint{5.522984in}{0.605094in}}%
\pgfpathlineto{\pgfqpoint{5.525296in}{0.618715in}}%
\pgfpathlineto{\pgfqpoint{5.527608in}{0.604978in}}%
\pgfpathlineto{\pgfqpoint{5.529921in}{0.605054in}}%
\pgfpathlineto{\pgfqpoint{5.532233in}{0.618714in}}%
\pgfpathlineto{\pgfqpoint{5.534545in}{0.605016in}}%
\pgfpathlineto{\pgfqpoint{5.534545in}{0.605016in}}%
\pgfusepath{stroke}%
\end{pgfscope}%
\begin{pgfscope}%
\pgfsetrectcap%
\pgfsetmiterjoin%
\pgfsetlinewidth{0.803000pt}%
\definecolor{currentstroke}{rgb}{0.000000,0.000000,0.000000}%
\pgfsetstrokecolor{currentstroke}%
\pgfsetdash{}{0pt}%
\pgfpathmoveto{\pgfqpoint{0.800000in}{0.528000in}}%
\pgfpathlineto{\pgfqpoint{0.800000in}{2.208000in}}%
\pgfusepath{stroke}%
\end{pgfscope}%
\begin{pgfscope}%
\pgfsetrectcap%
\pgfsetmiterjoin%
\pgfsetlinewidth{0.803000pt}%
\definecolor{currentstroke}{rgb}{0.000000,0.000000,0.000000}%
\pgfsetstrokecolor{currentstroke}%
\pgfsetdash{}{0pt}%
\pgfpathmoveto{\pgfqpoint{5.760000in}{0.528000in}}%
\pgfpathlineto{\pgfqpoint{5.760000in}{2.208000in}}%
\pgfusepath{stroke}%
\end{pgfscope}%
\begin{pgfscope}%
\pgfsetrectcap%
\pgfsetmiterjoin%
\pgfsetlinewidth{0.803000pt}%
\definecolor{currentstroke}{rgb}{0.000000,0.000000,0.000000}%
\pgfsetstrokecolor{currentstroke}%
\pgfsetdash{}{0pt}%
\pgfpathmoveto{\pgfqpoint{0.800000in}{0.528000in}}%
\pgfpathlineto{\pgfqpoint{5.760000in}{0.528000in}}%
\pgfusepath{stroke}%
\end{pgfscope}%
\begin{pgfscope}%
\pgfsetrectcap%
\pgfsetmiterjoin%
\pgfsetlinewidth{0.803000pt}%
\definecolor{currentstroke}{rgb}{0.000000,0.000000,0.000000}%
\pgfsetstrokecolor{currentstroke}%
\pgfsetdash{}{0pt}%
\pgfpathmoveto{\pgfqpoint{0.800000in}{2.208000in}}%
\pgfpathlineto{\pgfqpoint{5.760000in}{2.208000in}}%
\pgfusepath{stroke}%
\end{pgfscope}%
\end{pgfpicture}%
\makeatother%
\endgroup%
}}
  \caption{Transmitted envelope.}
  \label{fig:task1_ut}
\end{figure}
\begin{figure}[h]
  \centering
  \noindent\makebox[\textwidth]{\scalebox{0.90}{%% Creator: Matplotlib, PGF backend
%%
%% To include the figure in your LaTeX document, write
%%   \input{<filename>.pgf}
%%
%% Make sure the required packages are loaded in your preamble
%%   \usepackage{pgf}
%%
%% Figures using additional raster images can only be included by \input if
%% they are in the same directory as the main LaTeX file. For loading figures
%% from other directories you can use the `import` package
%%   \usepackage{import}
%% and then include the figures with
%%   \import{<path to file>}{<filename>.pgf}
%%
%% Matplotlib used the following preamble
%%   \usepackage{fontspec}
%%   \setmainfont{DejaVu Serif}
%%   \setsansfont{DejaVu Sans}
%%   \setmonofont{DejaVu Sans Mono}
%%
\begingroup%
\makeatletter%
\begin{pgfpicture}%
\pgfpathrectangle{\pgfpointorigin}{\pgfqpoint{6.400000in}{4.800000in}}%
\pgfusepath{use as bounding box, clip}%
\begin{pgfscope}%
\pgfsetbuttcap%
\pgfsetmiterjoin%
\definecolor{currentfill}{rgb}{1.000000,1.000000,1.000000}%
\pgfsetfillcolor{currentfill}%
\pgfsetlinewidth{0.000000pt}%
\definecolor{currentstroke}{rgb}{1.000000,1.000000,1.000000}%
\pgfsetstrokecolor{currentstroke}%
\pgfsetdash{}{0pt}%
\pgfpathmoveto{\pgfqpoint{0.000000in}{0.000000in}}%
\pgfpathlineto{\pgfqpoint{6.400000in}{0.000000in}}%
\pgfpathlineto{\pgfqpoint{6.400000in}{4.800000in}}%
\pgfpathlineto{\pgfqpoint{0.000000in}{4.800000in}}%
\pgfpathclose%
\pgfusepath{fill}%
\end{pgfscope}%
\begin{pgfscope}%
\pgfsetbuttcap%
\pgfsetmiterjoin%
\definecolor{currentfill}{rgb}{1.000000,1.000000,1.000000}%
\pgfsetfillcolor{currentfill}%
\pgfsetlinewidth{0.000000pt}%
\definecolor{currentstroke}{rgb}{0.000000,0.000000,0.000000}%
\pgfsetstrokecolor{currentstroke}%
\pgfsetstrokeopacity{0.000000}%
\pgfsetdash{}{0pt}%
\pgfpathmoveto{\pgfqpoint{0.800000in}{2.544000in}}%
\pgfpathlineto{\pgfqpoint{5.760000in}{2.544000in}}%
\pgfpathlineto{\pgfqpoint{5.760000in}{4.224000in}}%
\pgfpathlineto{\pgfqpoint{0.800000in}{4.224000in}}%
\pgfpathclose%
\pgfusepath{fill}%
\end{pgfscope}%
\begin{pgfscope}%
\pgfsetbuttcap%
\pgfsetroundjoin%
\definecolor{currentfill}{rgb}{0.000000,0.000000,0.000000}%
\pgfsetfillcolor{currentfill}%
\pgfsetlinewidth{0.803000pt}%
\definecolor{currentstroke}{rgb}{0.000000,0.000000,0.000000}%
\pgfsetstrokecolor{currentstroke}%
\pgfsetdash{}{0pt}%
\pgfsys@defobject{currentmarker}{\pgfqpoint{0.000000in}{-0.048611in}}{\pgfqpoint{0.000000in}{0.000000in}}{%
\pgfpathmoveto{\pgfqpoint{0.000000in}{0.000000in}}%
\pgfpathlineto{\pgfqpoint{0.000000in}{-0.048611in}}%
\pgfusepath{stroke,fill}%
}%
\begin{pgfscope}%
\pgfsys@transformshift{1.476364in}{2.544000in}%
\pgfsys@useobject{currentmarker}{}%
\end{pgfscope}%
\end{pgfscope}%
\begin{pgfscope}%
\pgftext[x=1.476364in,y=2.446778in,,top]{\sffamily\fontsize{10.000000}{12.000000}\selectfont -4}%
\end{pgfscope}%
\begin{pgfscope}%
\pgfsetbuttcap%
\pgfsetroundjoin%
\definecolor{currentfill}{rgb}{0.000000,0.000000,0.000000}%
\pgfsetfillcolor{currentfill}%
\pgfsetlinewidth{0.803000pt}%
\definecolor{currentstroke}{rgb}{0.000000,0.000000,0.000000}%
\pgfsetstrokecolor{currentstroke}%
\pgfsetdash{}{0pt}%
\pgfsys@defobject{currentmarker}{\pgfqpoint{0.000000in}{-0.048611in}}{\pgfqpoint{0.000000in}{0.000000in}}{%
\pgfpathmoveto{\pgfqpoint{0.000000in}{0.000000in}}%
\pgfpathlineto{\pgfqpoint{0.000000in}{-0.048611in}}%
\pgfusepath{stroke,fill}%
}%
\begin{pgfscope}%
\pgfsys@transformshift{2.378182in}{2.544000in}%
\pgfsys@useobject{currentmarker}{}%
\end{pgfscope}%
\end{pgfscope}%
\begin{pgfscope}%
\pgftext[x=2.378182in,y=2.446778in,,top]{\sffamily\fontsize{10.000000}{12.000000}\selectfont -2}%
\end{pgfscope}%
\begin{pgfscope}%
\pgfsetbuttcap%
\pgfsetroundjoin%
\definecolor{currentfill}{rgb}{0.000000,0.000000,0.000000}%
\pgfsetfillcolor{currentfill}%
\pgfsetlinewidth{0.803000pt}%
\definecolor{currentstroke}{rgb}{0.000000,0.000000,0.000000}%
\pgfsetstrokecolor{currentstroke}%
\pgfsetdash{}{0pt}%
\pgfsys@defobject{currentmarker}{\pgfqpoint{0.000000in}{-0.048611in}}{\pgfqpoint{0.000000in}{0.000000in}}{%
\pgfpathmoveto{\pgfqpoint{0.000000in}{0.000000in}}%
\pgfpathlineto{\pgfqpoint{0.000000in}{-0.048611in}}%
\pgfusepath{stroke,fill}%
}%
\begin{pgfscope}%
\pgfsys@transformshift{3.280000in}{2.544000in}%
\pgfsys@useobject{currentmarker}{}%
\end{pgfscope}%
\end{pgfscope}%
\begin{pgfscope}%
\pgftext[x=3.280000in,y=2.446778in,,top]{\sffamily\fontsize{10.000000}{12.000000}\selectfont 0}%
\end{pgfscope}%
\begin{pgfscope}%
\pgfsetbuttcap%
\pgfsetroundjoin%
\definecolor{currentfill}{rgb}{0.000000,0.000000,0.000000}%
\pgfsetfillcolor{currentfill}%
\pgfsetlinewidth{0.803000pt}%
\definecolor{currentstroke}{rgb}{0.000000,0.000000,0.000000}%
\pgfsetstrokecolor{currentstroke}%
\pgfsetdash{}{0pt}%
\pgfsys@defobject{currentmarker}{\pgfqpoint{0.000000in}{-0.048611in}}{\pgfqpoint{0.000000in}{0.000000in}}{%
\pgfpathmoveto{\pgfqpoint{0.000000in}{0.000000in}}%
\pgfpathlineto{\pgfqpoint{0.000000in}{-0.048611in}}%
\pgfusepath{stroke,fill}%
}%
\begin{pgfscope}%
\pgfsys@transformshift{4.181818in}{2.544000in}%
\pgfsys@useobject{currentmarker}{}%
\end{pgfscope}%
\end{pgfscope}%
\begin{pgfscope}%
\pgftext[x=4.181818in,y=2.446778in,,top]{\sffamily\fontsize{10.000000}{12.000000}\selectfont 2}%
\end{pgfscope}%
\begin{pgfscope}%
\pgfsetbuttcap%
\pgfsetroundjoin%
\definecolor{currentfill}{rgb}{0.000000,0.000000,0.000000}%
\pgfsetfillcolor{currentfill}%
\pgfsetlinewidth{0.803000pt}%
\definecolor{currentstroke}{rgb}{0.000000,0.000000,0.000000}%
\pgfsetstrokecolor{currentstroke}%
\pgfsetdash{}{0pt}%
\pgfsys@defobject{currentmarker}{\pgfqpoint{0.000000in}{-0.048611in}}{\pgfqpoint{0.000000in}{0.000000in}}{%
\pgfpathmoveto{\pgfqpoint{0.000000in}{0.000000in}}%
\pgfpathlineto{\pgfqpoint{0.000000in}{-0.048611in}}%
\pgfusepath{stroke,fill}%
}%
\begin{pgfscope}%
\pgfsys@transformshift{5.083636in}{2.544000in}%
\pgfsys@useobject{currentmarker}{}%
\end{pgfscope}%
\end{pgfscope}%
\begin{pgfscope}%
\pgftext[x=5.083636in,y=2.446778in,,top]{\sffamily\fontsize{10.000000}{12.000000}\selectfont 4}%
\end{pgfscope}%
\begin{pgfscope}%
\pgftext[x=3.280000in,y=2.256809in,,top]{\sffamily\fontsize{10.000000}{12.000000}\selectfont Time [\(\displaystyle \mu\)s]}%
\end{pgfscope}%
\begin{pgfscope}%
\pgfsetbuttcap%
\pgfsetroundjoin%
\definecolor{currentfill}{rgb}{0.000000,0.000000,0.000000}%
\pgfsetfillcolor{currentfill}%
\pgfsetlinewidth{0.803000pt}%
\definecolor{currentstroke}{rgb}{0.000000,0.000000,0.000000}%
\pgfsetstrokecolor{currentstroke}%
\pgfsetdash{}{0pt}%
\pgfsys@defobject{currentmarker}{\pgfqpoint{-0.048611in}{0.000000in}}{\pgfqpoint{0.000000in}{0.000000in}}{%
\pgfpathmoveto{\pgfqpoint{0.000000in}{0.000000in}}%
\pgfpathlineto{\pgfqpoint{-0.048611in}{0.000000in}}%
\pgfusepath{stroke,fill}%
}%
\begin{pgfscope}%
\pgfsys@transformshift{0.800000in}{2.595319in}%
\pgfsys@useobject{currentmarker}{}%
\end{pgfscope}%
\end{pgfscope}%
\begin{pgfscope}%
\pgftext[x=0.321308in,y=2.542558in,left,base]{\sffamily\fontsize{10.000000}{12.000000}\selectfont -400}%
\end{pgfscope}%
\begin{pgfscope}%
\pgfsetbuttcap%
\pgfsetroundjoin%
\definecolor{currentfill}{rgb}{0.000000,0.000000,0.000000}%
\pgfsetfillcolor{currentfill}%
\pgfsetlinewidth{0.803000pt}%
\definecolor{currentstroke}{rgb}{0.000000,0.000000,0.000000}%
\pgfsetstrokecolor{currentstroke}%
\pgfsetdash{}{0pt}%
\pgfsys@defobject{currentmarker}{\pgfqpoint{-0.048611in}{0.000000in}}{\pgfqpoint{0.000000in}{0.000000in}}{%
\pgfpathmoveto{\pgfqpoint{0.000000in}{0.000000in}}%
\pgfpathlineto{\pgfqpoint{-0.048611in}{0.000000in}}%
\pgfusepath{stroke,fill}%
}%
\begin{pgfscope}%
\pgfsys@transformshift{0.800000in}{2.989660in}%
\pgfsys@useobject{currentmarker}{}%
\end{pgfscope}%
\end{pgfscope}%
\begin{pgfscope}%
\pgftext[x=0.321308in,y=2.936898in,left,base]{\sffamily\fontsize{10.000000}{12.000000}\selectfont -200}%
\end{pgfscope}%
\begin{pgfscope}%
\pgfsetbuttcap%
\pgfsetroundjoin%
\definecolor{currentfill}{rgb}{0.000000,0.000000,0.000000}%
\pgfsetfillcolor{currentfill}%
\pgfsetlinewidth{0.803000pt}%
\definecolor{currentstroke}{rgb}{0.000000,0.000000,0.000000}%
\pgfsetstrokecolor{currentstroke}%
\pgfsetdash{}{0pt}%
\pgfsys@defobject{currentmarker}{\pgfqpoint{-0.048611in}{0.000000in}}{\pgfqpoint{0.000000in}{0.000000in}}{%
\pgfpathmoveto{\pgfqpoint{0.000000in}{0.000000in}}%
\pgfpathlineto{\pgfqpoint{-0.048611in}{0.000000in}}%
\pgfusepath{stroke,fill}%
}%
\begin{pgfscope}%
\pgfsys@transformshift{0.800000in}{3.384000in}%
\pgfsys@useobject{currentmarker}{}%
\end{pgfscope}%
\end{pgfscope}%
\begin{pgfscope}%
\pgftext[x=0.614413in,y=3.331238in,left,base]{\sffamily\fontsize{10.000000}{12.000000}\selectfont 0}%
\end{pgfscope}%
\begin{pgfscope}%
\pgfsetbuttcap%
\pgfsetroundjoin%
\definecolor{currentfill}{rgb}{0.000000,0.000000,0.000000}%
\pgfsetfillcolor{currentfill}%
\pgfsetlinewidth{0.803000pt}%
\definecolor{currentstroke}{rgb}{0.000000,0.000000,0.000000}%
\pgfsetstrokecolor{currentstroke}%
\pgfsetdash{}{0pt}%
\pgfsys@defobject{currentmarker}{\pgfqpoint{-0.048611in}{0.000000in}}{\pgfqpoint{0.000000in}{0.000000in}}{%
\pgfpathmoveto{\pgfqpoint{0.000000in}{0.000000in}}%
\pgfpathlineto{\pgfqpoint{-0.048611in}{0.000000in}}%
\pgfusepath{stroke,fill}%
}%
\begin{pgfscope}%
\pgfsys@transformshift{0.800000in}{3.778340in}%
\pgfsys@useobject{currentmarker}{}%
\end{pgfscope}%
\end{pgfscope}%
\begin{pgfscope}%
\pgftext[x=0.437682in,y=3.725579in,left,base]{\sffamily\fontsize{10.000000}{12.000000}\selectfont 200}%
\end{pgfscope}%
\begin{pgfscope}%
\pgfsetbuttcap%
\pgfsetroundjoin%
\definecolor{currentfill}{rgb}{0.000000,0.000000,0.000000}%
\pgfsetfillcolor{currentfill}%
\pgfsetlinewidth{0.803000pt}%
\definecolor{currentstroke}{rgb}{0.000000,0.000000,0.000000}%
\pgfsetstrokecolor{currentstroke}%
\pgfsetdash{}{0pt}%
\pgfsys@defobject{currentmarker}{\pgfqpoint{-0.048611in}{0.000000in}}{\pgfqpoint{0.000000in}{0.000000in}}{%
\pgfpathmoveto{\pgfqpoint{0.000000in}{0.000000in}}%
\pgfpathlineto{\pgfqpoint{-0.048611in}{0.000000in}}%
\pgfusepath{stroke,fill}%
}%
\begin{pgfscope}%
\pgfsys@transformshift{0.800000in}{4.172680in}%
\pgfsys@useobject{currentmarker}{}%
\end{pgfscope}%
\end{pgfscope}%
\begin{pgfscope}%
\pgftext[x=0.437682in,y=4.119919in,left,base]{\sffamily\fontsize{10.000000}{12.000000}\selectfont 400}%
\end{pgfscope}%
\begin{pgfscope}%
\pgftext[x=0.265752in,y=3.384000in,,bottom,rotate=90.000000]{\sffamily\fontsize{10.000000}{12.000000}\selectfont Real part of the signal}%
\end{pgfscope}%
\begin{pgfscope}%
\pgfpathrectangle{\pgfqpoint{0.800000in}{2.544000in}}{\pgfqpoint{4.960000in}{1.680000in}} %
\pgfusepath{clip}%
\pgfsetrectcap%
\pgfsetroundjoin%
\pgfsetlinewidth{1.505625pt}%
\definecolor{currentstroke}{rgb}{0.121569,0.466667,0.705882}%
\pgfsetstrokecolor{currentstroke}%
\pgfsetdash{}{0pt}%
\pgfpathmoveto{\pgfqpoint{1.025455in}{3.384000in}}%
\pgfpathlineto{\pgfqpoint{1.774657in}{3.384000in}}%
\pgfpathlineto{\pgfqpoint{1.776970in}{4.147636in}}%
\pgfpathlineto{\pgfqpoint{1.779282in}{4.097760in}}%
\pgfpathlineto{\pgfqpoint{1.781594in}{3.953703in}}%
\pgfpathlineto{\pgfqpoint{1.786219in}{3.464763in}}%
\pgfpathlineto{\pgfqpoint{1.793156in}{2.736441in}}%
\pgfpathlineto{\pgfqpoint{1.795469in}{2.633000in}}%
\pgfpathlineto{\pgfqpoint{1.797781in}{2.635002in}}%
\pgfpathlineto{\pgfqpoint{1.800093in}{2.743689in}}%
\pgfpathlineto{\pgfqpoint{1.804718in}{3.210913in}}%
\pgfpathlineto{\pgfqpoint{1.811655in}{3.998018in}}%
\pgfpathlineto{\pgfqpoint{1.813967in}{4.124762in}}%
\pgfpathlineto{\pgfqpoint{1.816280in}{4.139312in}}%
\pgfpathlineto{\pgfqpoint{1.818592in}{4.037873in}}%
\pgfpathlineto{\pgfqpoint{1.823217in}{3.559851in}}%
\pgfpathlineto{\pgfqpoint{1.830154in}{2.751933in}}%
\pgfpathlineto{\pgfqpoint{1.832466in}{2.634049in}}%
\pgfpathlineto{\pgfqpoint{1.834779in}{2.638507in}}%
\pgfpathlineto{\pgfqpoint{1.837091in}{2.766205in}}%
\pgfpathlineto{\pgfqpoint{1.841716in}{3.294773in}}%
\pgfpathlineto{\pgfqpoint{1.846340in}{3.885050in}}%
\pgfpathlineto{\pgfqpoint{1.848653in}{4.076223in}}%
\pgfpathlineto{\pgfqpoint{1.850965in}{4.147552in}}%
\pgfpathlineto{\pgfqpoint{1.853277in}{4.084943in}}%
\pgfpathlineto{\pgfqpoint{1.855590in}{3.897779in}}%
\pgfpathlineto{\pgfqpoint{1.867152in}{2.629765in}}%
\pgfpathlineto{\pgfqpoint{1.869464in}{2.649858in}}%
\pgfpathlineto{\pgfqpoint{1.871776in}{2.808194in}}%
\pgfpathlineto{\pgfqpoint{1.883338in}{4.131145in}}%
\pgfpathlineto{\pgfqpoint{1.885650in}{4.125616in}}%
\pgfpathlineto{\pgfqpoint{1.887963in}{3.972211in}}%
\pgfpathlineto{\pgfqpoint{1.899524in}{2.631144in}}%
\pgfpathlineto{\pgfqpoint{1.901837in}{2.653065in}}%
\pgfpathlineto{\pgfqpoint{1.904149in}{2.829183in}}%
\pgfpathlineto{\pgfqpoint{1.913399in}{4.050356in}}%
\pgfpathlineto{\pgfqpoint{1.915711in}{4.147109in}}%
\pgfpathlineto{\pgfqpoint{1.918023in}{4.075622in}}%
\pgfpathlineto{\pgfqpoint{1.922648in}{3.518856in}}%
\pgfpathlineto{\pgfqpoint{1.927273in}{2.844027in}}%
\pgfpathlineto{\pgfqpoint{1.929585in}{2.654729in}}%
\pgfpathlineto{\pgfqpoint{1.931897in}{2.633389in}}%
\pgfpathlineto{\pgfqpoint{1.934210in}{2.786838in}}%
\pgfpathlineto{\pgfqpoint{1.945772in}{4.147636in}}%
\pgfpathlineto{\pgfqpoint{1.948084in}{4.054814in}}%
\pgfpathlineto{\pgfqpoint{1.952709in}{3.436482in}}%
\pgfpathlineto{\pgfqpoint{1.957333in}{2.766205in}}%
\pgfpathlineto{\pgfqpoint{1.959646in}{2.625846in}}%
\pgfpathlineto{\pgfqpoint{1.961958in}{2.677815in}}%
\pgfpathlineto{\pgfqpoint{1.964270in}{2.910819in}}%
\pgfpathlineto{\pgfqpoint{1.971207in}{3.972211in}}%
\pgfpathlineto{\pgfqpoint{1.973520in}{4.136375in}}%
\pgfpathlineto{\pgfqpoint{1.975832in}{4.101475in}}%
\pgfpathlineto{\pgfqpoint{1.978145in}{3.874796in}}%
\pgfpathlineto{\pgfqpoint{1.985082in}{2.793982in}}%
\pgfpathlineto{\pgfqpoint{1.987394in}{2.629765in}}%
\pgfpathlineto{\pgfqpoint{1.989706in}{2.673571in}}%
\pgfpathlineto{\pgfqpoint{1.992019in}{2.915289in}}%
\pgfpathlineto{\pgfqpoint{1.998956in}{4.006763in}}%
\pgfpathlineto{\pgfqpoint{2.001268in}{4.145473in}}%
\pgfpathlineto{\pgfqpoint{2.003580in}{4.065369in}}%
\pgfpathlineto{\pgfqpoint{2.008205in}{3.391097in}}%
\pgfpathlineto{\pgfqpoint{2.012830in}{2.706519in}}%
\pgfpathlineto{\pgfqpoint{2.015142in}{2.622421in}}%
\pgfpathlineto{\pgfqpoint{2.017455in}{2.766205in}}%
\pgfpathlineto{\pgfqpoint{2.026704in}{4.118918in}}%
\pgfpathlineto{\pgfqpoint{2.029016in}{4.115954in}}%
\pgfpathlineto{\pgfqpoint{2.031329in}{3.885050in}}%
\pgfpathlineto{\pgfqpoint{2.038266in}{2.744076in}}%
\pgfpathlineto{\pgfqpoint{2.040578in}{2.620369in}}%
\pgfpathlineto{\pgfqpoint{2.042890in}{2.741762in}}%
\pgfpathlineto{\pgfqpoint{2.047515in}{3.503459in}}%
\pgfpathlineto{\pgfqpoint{2.052140in}{4.123892in}}%
\pgfpathlineto{\pgfqpoint{2.054452in}{4.105285in}}%
\pgfpathlineto{\pgfqpoint{2.056765in}{3.845958in}}%
\pgfpathlineto{\pgfqpoint{2.063702in}{2.696339in}}%
\pgfpathlineto{\pgfqpoint{2.066014in}{2.628274in}}%
\pgfpathlineto{\pgfqpoint{2.068326in}{2.820003in}}%
\pgfpathlineto{\pgfqpoint{2.075263in}{4.012863in}}%
\pgfpathlineto{\pgfqpoint{2.077576in}{4.147636in}}%
\pgfpathlineto{\pgfqpoint{2.079888in}{4.012057in}}%
\pgfpathlineto{\pgfqpoint{2.089138in}{2.623997in}}%
\pgfpathlineto{\pgfqpoint{2.091450in}{2.720084in}}%
\pgfpathlineto{\pgfqpoint{2.096075in}{3.520951in}}%
\pgfpathlineto{\pgfqpoint{2.098387in}{3.931943in}}%
\pgfpathlineto{\pgfqpoint{2.100699in}{4.139208in}}%
\pgfpathlineto{\pgfqpoint{2.103012in}{4.063111in}}%
\pgfpathlineto{\pgfqpoint{2.107636in}{3.264541in}}%
\pgfpathlineto{\pgfqpoint{2.109949in}{2.843526in}}%
\pgfpathlineto{\pgfqpoint{2.112261in}{2.629326in}}%
\pgfpathlineto{\pgfqpoint{2.114573in}{2.706519in}}%
\pgfpathlineto{\pgfqpoint{2.119198in}{3.520951in}}%
\pgfpathlineto{\pgfqpoint{2.121510in}{3.941735in}}%
\pgfpathlineto{\pgfqpoint{2.123823in}{4.142891in}}%
\pgfpathlineto{\pgfqpoint{2.126135in}{4.042591in}}%
\pgfpathlineto{\pgfqpoint{2.135385in}{2.620569in}}%
\pgfpathlineto{\pgfqpoint{2.137697in}{2.766205in}}%
\pgfpathlineto{\pgfqpoint{2.144634in}{4.036032in}}%
\pgfpathlineto{\pgfqpoint{2.146946in}{4.142891in}}%
\pgfpathlineto{\pgfqpoint{2.149259in}{3.930953in}}%
\pgfpathlineto{\pgfqpoint{2.156196in}{2.673571in}}%
\pgfpathlineto{\pgfqpoint{2.158508in}{2.652860in}}%
\pgfpathlineto{\pgfqpoint{2.160821in}{2.947871in}}%
\pgfpathlineto{\pgfqpoint{2.165445in}{3.897779in}}%
\pgfpathlineto{\pgfqpoint{2.167758in}{4.138235in}}%
\pgfpathlineto{\pgfqpoint{2.170070in}{4.046508in}}%
\pgfpathlineto{\pgfqpoint{2.179319in}{2.624798in}}%
\pgfpathlineto{\pgfqpoint{2.181632in}{2.849069in}}%
\pgfpathlineto{\pgfqpoint{2.188569in}{4.118918in}}%
\pgfpathlineto{\pgfqpoint{2.190881in}{4.081520in}}%
\pgfpathlineto{\pgfqpoint{2.195506in}{3.200561in}}%
\pgfpathlineto{\pgfqpoint{2.197818in}{2.766205in}}%
\pgfpathlineto{\pgfqpoint{2.200131in}{2.622421in}}%
\pgfpathlineto{\pgfqpoint{2.202443in}{2.839529in}}%
\pgfpathlineto{\pgfqpoint{2.209380in}{4.125276in}}%
\pgfpathlineto{\pgfqpoint{2.211692in}{4.065369in}}%
\pgfpathlineto{\pgfqpoint{2.220942in}{2.634049in}}%
\pgfpathlineto{\pgfqpoint{2.223254in}{2.915289in}}%
\pgfpathlineto{\pgfqpoint{2.227879in}{3.923972in}}%
\pgfpathlineto{\pgfqpoint{2.230191in}{4.146286in}}%
\pgfpathlineto{\pgfqpoint{2.232503in}{3.983367in}}%
\pgfpathlineto{\pgfqpoint{2.239441in}{2.649858in}}%
\pgfpathlineto{\pgfqpoint{2.241753in}{2.698514in}}%
\pgfpathlineto{\pgfqpoint{2.248690in}{4.065369in}}%
\pgfpathlineto{\pgfqpoint{2.251002in}{4.119110in}}%
\pgfpathlineto{\pgfqpoint{2.253315in}{3.784511in}}%
\pgfpathlineto{\pgfqpoint{2.257939in}{2.766205in}}%
\pgfpathlineto{\pgfqpoint{2.260252in}{2.626017in}}%
\pgfpathlineto{\pgfqpoint{2.262564in}{2.893748in}}%
\pgfpathlineto{\pgfqpoint{2.267189in}{3.938329in}}%
\pgfpathlineto{\pgfqpoint{2.269501in}{4.147636in}}%
\pgfpathlineto{\pgfqpoint{2.271814in}{3.938329in}}%
\pgfpathlineto{\pgfqpoint{2.278751in}{2.623726in}}%
\pgfpathlineto{\pgfqpoint{2.281063in}{2.786838in}}%
\pgfpathlineto{\pgfqpoint{2.288000in}{4.138235in}}%
\pgfpathlineto{\pgfqpoint{2.290312in}{4.006763in}}%
\pgfpathlineto{\pgfqpoint{2.297249in}{2.634184in}}%
\pgfpathlineto{\pgfqpoint{2.299562in}{2.749948in}}%
\pgfpathlineto{\pgfqpoint{2.306499in}{4.133273in}}%
\pgfpathlineto{\pgfqpoint{2.308811in}{4.016465in}}%
\pgfpathlineto{\pgfqpoint{2.315748in}{2.631144in}}%
\pgfpathlineto{\pgfqpoint{2.318061in}{2.766205in}}%
\pgfpathlineto{\pgfqpoint{2.324998in}{4.142812in}}%
\pgfpathlineto{\pgfqpoint{2.327310in}{3.972211in}}%
\pgfpathlineto{\pgfqpoint{2.334247in}{2.620641in}}%
\pgfpathlineto{\pgfqpoint{2.336559in}{2.843526in}}%
\pgfpathlineto{\pgfqpoint{2.341184in}{3.953703in}}%
\pgfpathlineto{\pgfqpoint{2.343497in}{4.144661in}}%
\pgfpathlineto{\pgfqpoint{2.345809in}{3.854390in}}%
\pgfpathlineto{\pgfqpoint{2.350434in}{2.749948in}}%
\pgfpathlineto{\pgfqpoint{2.352746in}{2.641050in}}%
\pgfpathlineto{\pgfqpoint{2.355058in}{3.010406in}}%
\pgfpathlineto{\pgfqpoint{2.359683in}{4.080068in}}%
\pgfpathlineto{\pgfqpoint{2.361995in}{4.084943in}}%
\pgfpathlineto{\pgfqpoint{2.368932in}{2.639918in}}%
\pgfpathlineto{\pgfqpoint{2.371245in}{2.759190in}}%
\pgfpathlineto{\pgfqpoint{2.378182in}{4.147636in}}%
\pgfpathlineto{\pgfqpoint{2.380494in}{3.889853in}}%
\pgfpathlineto{\pgfqpoint{2.385119in}{2.751933in}}%
\pgfpathlineto{\pgfqpoint{2.387431in}{2.645353in}}%
\pgfpathlineto{\pgfqpoint{2.389744in}{3.044926in}}%
\pgfpathlineto{\pgfqpoint{2.394368in}{4.109601in}}%
\pgfpathlineto{\pgfqpoint{2.396681in}{4.037873in}}%
\pgfpathlineto{\pgfqpoint{2.403618in}{2.620390in}}%
\pgfpathlineto{\pgfqpoint{2.405930in}{2.885096in}}%
\pgfpathlineto{\pgfqpoint{2.410555in}{4.037873in}}%
\pgfpathlineto{\pgfqpoint{2.412867in}{4.106442in}}%
\pgfpathlineto{\pgfqpoint{2.419804in}{2.633000in}}%
\pgfpathlineto{\pgfqpoint{2.422117in}{2.798970in}}%
\pgfpathlineto{\pgfqpoint{2.426741in}{3.978943in}}%
\pgfpathlineto{\pgfqpoint{2.429054in}{4.131145in}}%
\pgfpathlineto{\pgfqpoint{2.431366in}{3.733211in}}%
\pgfpathlineto{\pgfqpoint{2.435991in}{2.646999in}}%
\pgfpathlineto{\pgfqpoint{2.438303in}{2.766205in}}%
\pgfpathlineto{\pgfqpoint{2.442928in}{3.957469in}}%
\pgfpathlineto{\pgfqpoint{2.445240in}{4.135634in}}%
\pgfpathlineto{\pgfqpoint{2.447552in}{3.745149in}}%
\pgfpathlineto{\pgfqpoint{2.452177in}{2.645353in}}%
\pgfpathlineto{\pgfqpoint{2.454490in}{2.775941in}}%
\pgfpathlineto{\pgfqpoint{2.459114in}{3.980719in}}%
\pgfpathlineto{\pgfqpoint{2.461427in}{4.126121in}}%
\pgfpathlineto{\pgfqpoint{2.463739in}{3.694815in}}%
\pgfpathlineto{\pgfqpoint{2.468364in}{2.629765in}}%
\pgfpathlineto{\pgfqpoint{2.470676in}{2.831627in}}%
\pgfpathlineto{\pgfqpoint{2.475301in}{4.040788in}}%
\pgfpathlineto{\pgfqpoint{2.477613in}{4.090455in}}%
\pgfpathlineto{\pgfqpoint{2.484550in}{2.620791in}}%
\pgfpathlineto{\pgfqpoint{2.486862in}{2.949621in}}%
\pgfpathlineto{\pgfqpoint{2.491487in}{4.112210in}}%
\pgfpathlineto{\pgfqpoint{2.493800in}{4.000122in}}%
\pgfpathlineto{\pgfqpoint{2.498424in}{2.766205in}}%
\pgfpathlineto{\pgfqpoint{2.500737in}{2.657958in}}%
\pgfpathlineto{\pgfqpoint{2.507674in}{4.147552in}}%
\pgfpathlineto{\pgfqpoint{2.509986in}{3.818379in}}%
\pgfpathlineto{\pgfqpoint{2.514611in}{2.647748in}}%
\pgfpathlineto{\pgfqpoint{2.516923in}{2.793982in}}%
\pgfpathlineto{\pgfqpoint{2.521548in}{4.040788in}}%
\pgfpathlineto{\pgfqpoint{2.523860in}{4.079190in}}%
\pgfpathlineto{\pgfqpoint{2.530797in}{2.633000in}}%
\pgfpathlineto{\pgfqpoint{2.533110in}{3.073185in}}%
\pgfpathlineto{\pgfqpoint{2.535422in}{3.784511in}}%
\pgfpathlineto{\pgfqpoint{2.537734in}{4.146845in}}%
\pgfpathlineto{\pgfqpoint{2.540047in}{3.840855in}}%
\pgfpathlineto{\pgfqpoint{2.544671in}{2.645353in}}%
\pgfpathlineto{\pgfqpoint{2.546984in}{2.811939in}}%
\pgfpathlineto{\pgfqpoint{2.551608in}{4.071351in}}%
\pgfpathlineto{\pgfqpoint{2.553921in}{4.040788in}}%
\pgfpathlineto{\pgfqpoint{2.558545in}{2.766205in}}%
\pgfpathlineto{\pgfqpoint{2.560858in}{2.670239in}}%
\pgfpathlineto{\pgfqpoint{2.567795in}{4.131145in}}%
\pgfpathlineto{\pgfqpoint{2.570107in}{3.655453in}}%
\pgfpathlineto{\pgfqpoint{2.572420in}{2.929423in}}%
\pgfpathlineto{\pgfqpoint{2.574732in}{2.622218in}}%
\pgfpathlineto{\pgfqpoint{2.577044in}{3.020352in}}%
\pgfpathlineto{\pgfqpoint{2.581669in}{4.146602in}}%
\pgfpathlineto{\pgfqpoint{2.583981in}{3.823037in}}%
\pgfpathlineto{\pgfqpoint{2.588606in}{2.629765in}}%
\pgfpathlineto{\pgfqpoint{2.590918in}{2.885096in}}%
\pgfpathlineto{\pgfqpoint{2.595543in}{4.124762in}}%
\pgfpathlineto{\pgfqpoint{2.597855in}{3.928968in}}%
\pgfpathlineto{\pgfqpoint{2.602480in}{2.658399in}}%
\pgfpathlineto{\pgfqpoint{2.604793in}{2.804942in}}%
\pgfpathlineto{\pgfqpoint{2.609417in}{4.095465in}}%
\pgfpathlineto{\pgfqpoint{2.611730in}{3.986869in}}%
\pgfpathlineto{\pgfqpoint{2.616354in}{2.683621in}}%
\pgfpathlineto{\pgfqpoint{2.618667in}{2.766205in}}%
\pgfpathlineto{\pgfqpoint{2.623291in}{4.077714in}}%
\pgfpathlineto{\pgfqpoint{2.625604in}{4.008810in}}%
\pgfpathlineto{\pgfqpoint{2.630228in}{2.691777in}}%
\pgfpathlineto{\pgfqpoint{2.632541in}{2.759598in}}%
\pgfpathlineto{\pgfqpoint{2.637166in}{4.080068in}}%
\pgfpathlineto{\pgfqpoint{2.639478in}{4.000541in}}%
\pgfpathlineto{\pgfqpoint{2.644103in}{2.679173in}}%
\pgfpathlineto{\pgfqpoint{2.646415in}{2.783316in}}%
\pgfpathlineto{\pgfqpoint{2.651040in}{4.101475in}}%
\pgfpathlineto{\pgfqpoint{2.653352in}{3.959806in}}%
\pgfpathlineto{\pgfqpoint{2.657977in}{2.651642in}}%
\pgfpathlineto{\pgfqpoint{2.660289in}{2.843526in}}%
\pgfpathlineto{\pgfqpoint{2.664914in}{4.131145in}}%
\pgfpathlineto{\pgfqpoint{2.667226in}{3.876967in}}%
\pgfpathlineto{\pgfqpoint{2.671851in}{2.625188in}}%
\pgfpathlineto{\pgfqpoint{2.674163in}{2.952544in}}%
\pgfpathlineto{\pgfqpoint{2.676476in}{3.739508in}}%
\pgfpathlineto{\pgfqpoint{2.678788in}{4.147636in}}%
\pgfpathlineto{\pgfqpoint{2.681100in}{3.738251in}}%
\pgfpathlineto{\pgfqpoint{2.683413in}{2.947871in}}%
\pgfpathlineto{\pgfqpoint{2.685725in}{2.626728in}}%
\pgfpathlineto{\pgfqpoint{2.688037in}{3.123859in}}%
\pgfpathlineto{\pgfqpoint{2.690350in}{3.903528in}}%
\pgfpathlineto{\pgfqpoint{2.692662in}{4.118918in}}%
\pgfpathlineto{\pgfqpoint{2.697287in}{2.783316in}}%
\pgfpathlineto{\pgfqpoint{2.699599in}{2.692377in}}%
\pgfpathlineto{\pgfqpoint{2.704224in}{4.057174in}}%
\pgfpathlineto{\pgfqpoint{2.706536in}{4.006763in}}%
\pgfpathlineto{\pgfqpoint{2.711161in}{2.654729in}}%
\pgfpathlineto{\pgfqpoint{2.713473in}{2.859292in}}%
\pgfpathlineto{\pgfqpoint{2.718098in}{4.144274in}}%
\pgfpathlineto{\pgfqpoint{2.720410in}{3.779665in}}%
\pgfpathlineto{\pgfqpoint{2.722723in}{2.967300in}}%
\pgfpathlineto{\pgfqpoint{2.725035in}{2.626819in}}%
\pgfpathlineto{\pgfqpoint{2.727347in}{3.147348in}}%
\pgfpathlineto{\pgfqpoint{2.729660in}{3.938329in}}%
\pgfpathlineto{\pgfqpoint{2.731972in}{4.095722in}}%
\pgfpathlineto{\pgfqpoint{2.736597in}{2.716261in}}%
\pgfpathlineto{\pgfqpoint{2.738909in}{2.766205in}}%
\pgfpathlineto{\pgfqpoint{2.743534in}{4.126121in}}%
\pgfpathlineto{\pgfqpoint{2.745846in}{3.857181in}}%
\pgfpathlineto{\pgfqpoint{2.748159in}{3.035420in}}%
\pgfpathlineto{\pgfqpoint{2.750471in}{2.621221in}}%
\pgfpathlineto{\pgfqpoint{2.752783in}{3.102623in}}%
\pgfpathlineto{\pgfqpoint{2.755096in}{3.915883in}}%
\pgfpathlineto{\pgfqpoint{2.757408in}{4.101475in}}%
\pgfpathlineto{\pgfqpoint{2.762033in}{2.709157in}}%
\pgfpathlineto{\pgfqpoint{2.764345in}{2.784633in}}%
\pgfpathlineto{\pgfqpoint{2.768970in}{4.138235in}}%
\pgfpathlineto{\pgfqpoint{2.771282in}{3.794135in}}%
\pgfpathlineto{\pgfqpoint{2.773594in}{2.955477in}}%
\pgfpathlineto{\pgfqpoint{2.775907in}{2.634049in}}%
\pgfpathlineto{\pgfqpoint{2.780531in}{4.006763in}}%
\pgfpathlineto{\pgfqpoint{2.782844in}{4.033806in}}%
\pgfpathlineto{\pgfqpoint{2.787469in}{2.642723in}}%
\pgfpathlineto{\pgfqpoint{2.789781in}{2.929423in}}%
\pgfpathlineto{\pgfqpoint{2.792093in}{3.776626in}}%
\pgfpathlineto{\pgfqpoint{2.794406in}{4.139208in}}%
\pgfpathlineto{\pgfqpoint{2.799030in}{2.766205in}}%
\pgfpathlineto{\pgfqpoint{2.801343in}{2.735690in}}%
\pgfpathlineto{\pgfqpoint{2.805967in}{4.130998in}}%
\pgfpathlineto{\pgfqpoint{2.808280in}{3.808401in}}%
\pgfpathlineto{\pgfqpoint{2.810592in}{2.950789in}}%
\pgfpathlineto{\pgfqpoint{2.812904in}{2.639918in}}%
\pgfpathlineto{\pgfqpoint{2.817529in}{4.046508in}}%
\pgfpathlineto{\pgfqpoint{2.819841in}{3.978052in}}%
\pgfpathlineto{\pgfqpoint{2.824466in}{2.621590in}}%
\pgfpathlineto{\pgfqpoint{2.826779in}{3.071241in}}%
\pgfpathlineto{\pgfqpoint{2.829091in}{3.923972in}}%
\pgfpathlineto{\pgfqpoint{2.831403in}{4.080068in}}%
\pgfpathlineto{\pgfqpoint{2.836028in}{2.653065in}}%
\pgfpathlineto{\pgfqpoint{2.838340in}{2.922042in}}%
\pgfpathlineto{\pgfqpoint{2.840653in}{3.794135in}}%
\pgfpathlineto{\pgfqpoint{2.842965in}{4.130102in}}%
\pgfpathlineto{\pgfqpoint{2.847590in}{2.708163in}}%
\pgfpathlineto{\pgfqpoint{2.849902in}{2.820003in}}%
\pgfpathlineto{\pgfqpoint{2.854527in}{4.146602in}}%
\pgfpathlineto{\pgfqpoint{2.856839in}{3.601678in}}%
\pgfpathlineto{\pgfqpoint{2.859152in}{2.766205in}}%
\pgfpathlineto{\pgfqpoint{2.861464in}{2.755943in}}%
\pgfpathlineto{\pgfqpoint{2.866089in}{4.146066in}}%
\pgfpathlineto{\pgfqpoint{2.868401in}{3.678524in}}%
\pgfpathlineto{\pgfqpoint{2.870713in}{2.812881in}}%
\pgfpathlineto{\pgfqpoint{2.873026in}{2.720084in}}%
\pgfpathlineto{\pgfqpoint{2.877650in}{4.140807in}}%
\pgfpathlineto{\pgfqpoint{2.879963in}{3.720527in}}%
\pgfpathlineto{\pgfqpoint{2.882275in}{2.839529in}}%
\pgfpathlineto{\pgfqpoint{2.884587in}{2.704888in}}%
\pgfpathlineto{\pgfqpoint{2.889212in}{4.138235in}}%
\pgfpathlineto{\pgfqpoint{2.891524in}{3.730051in}}%
\pgfpathlineto{\pgfqpoint{2.893837in}{2.842023in}}%
\pgfpathlineto{\pgfqpoint{2.896149in}{2.706519in}}%
\pgfpathlineto{\pgfqpoint{2.900774in}{4.140807in}}%
\pgfpathlineto{\pgfqpoint{2.903086in}{3.707727in}}%
\pgfpathlineto{\pgfqpoint{2.905399in}{2.820003in}}%
\pgfpathlineto{\pgfqpoint{2.907711in}{2.725409in}}%
\pgfpathlineto{\pgfqpoint{2.912336in}{4.146066in}}%
\pgfpathlineto{\pgfqpoint{2.914648in}{3.652133in}}%
\pgfpathlineto{\pgfqpoint{2.916960in}{2.776801in}}%
\pgfpathlineto{\pgfqpoint{2.919273in}{2.766205in}}%
\pgfpathlineto{\pgfqpoint{2.923897in}{4.146602in}}%
\pgfpathlineto{\pgfqpoint{2.928522in}{2.720084in}}%
\pgfpathlineto{\pgfqpoint{2.930834in}{2.837047in}}%
\pgfpathlineto{\pgfqpoint{2.933147in}{3.745149in}}%
\pgfpathlineto{\pgfqpoint{2.935459in}{4.130102in}}%
\pgfpathlineto{\pgfqpoint{2.940084in}{2.662949in}}%
\pgfpathlineto{\pgfqpoint{2.942396in}{2.947871in}}%
\pgfpathlineto{\pgfqpoint{2.944709in}{3.871525in}}%
\pgfpathlineto{\pgfqpoint{2.947021in}{4.080068in}}%
\pgfpathlineto{\pgfqpoint{2.951646in}{2.624798in}}%
\pgfpathlineto{\pgfqpoint{2.953958in}{3.107247in}}%
\pgfpathlineto{\pgfqpoint{2.956270in}{4.000122in}}%
\pgfpathlineto{\pgfqpoint{2.958583in}{3.978052in}}%
\pgfpathlineto{\pgfqpoint{2.960895in}{3.071241in}}%
\pgfpathlineto{\pgfqpoint{2.963207in}{2.630908in}}%
\pgfpathlineto{\pgfqpoint{2.967832in}{4.104111in}}%
\pgfpathlineto{\pgfqpoint{2.970145in}{3.808401in}}%
\pgfpathlineto{\pgfqpoint{2.972457in}{2.872851in}}%
\pgfpathlineto{\pgfqpoint{2.974769in}{2.709157in}}%
\pgfpathlineto{\pgfqpoint{2.979394in}{4.147636in}}%
\pgfpathlineto{\pgfqpoint{2.984019in}{2.706519in}}%
\pgfpathlineto{\pgfqpoint{2.986331in}{2.882414in}}%
\pgfpathlineto{\pgfqpoint{2.988643in}{3.827093in}}%
\pgfpathlineto{\pgfqpoint{2.990956in}{4.091528in}}%
\pgfpathlineto{\pgfqpoint{2.995580in}{2.622526in}}%
\pgfpathlineto{\pgfqpoint{2.997893in}{3.156138in}}%
\pgfpathlineto{\pgfqpoint{3.000205in}{4.043308in}}%
\pgfpathlineto{\pgfqpoint{3.002517in}{3.906124in}}%
\pgfpathlineto{\pgfqpoint{3.004830in}{2.955477in}}%
\pgfpathlineto{\pgfqpoint{3.007142in}{2.673571in}}%
\pgfpathlineto{\pgfqpoint{3.011767in}{4.146286in}}%
\pgfpathlineto{\pgfqpoint{3.014079in}{3.590769in}}%
\pgfpathlineto{\pgfqpoint{3.016392in}{2.709157in}}%
\pgfpathlineto{\pgfqpoint{3.018704in}{2.893204in}}%
\pgfpathlineto{\pgfqpoint{3.021016in}{3.854390in}}%
\pgfpathlineto{\pgfqpoint{3.023329in}{4.069485in}}%
\pgfpathlineto{\pgfqpoint{3.027953in}{2.621221in}}%
\pgfpathlineto{\pgfqpoint{3.032578in}{4.100253in}}%
\pgfpathlineto{\pgfqpoint{3.034890in}{3.784511in}}%
\pgfpathlineto{\pgfqpoint{3.037203in}{2.824330in}}%
\pgfpathlineto{\pgfqpoint{3.039515in}{2.766205in}}%
\pgfpathlineto{\pgfqpoint{3.041828in}{3.706441in}}%
\pgfpathlineto{\pgfqpoint{3.044140in}{4.124762in}}%
\pgfpathlineto{\pgfqpoint{3.048765in}{2.626819in}}%
\pgfpathlineto{\pgfqpoint{3.051077in}{3.147348in}}%
\pgfpathlineto{\pgfqpoint{3.053389in}{4.054814in}}%
\pgfpathlineto{\pgfqpoint{3.055702in}{3.863836in}}%
\pgfpathlineto{\pgfqpoint{3.058014in}{2.885096in}}%
\pgfpathlineto{\pgfqpoint{3.060326in}{2.726850in}}%
\pgfpathlineto{\pgfqpoint{3.064951in}{4.134611in}}%
\pgfpathlineto{\pgfqpoint{3.069576in}{2.629765in}}%
\pgfpathlineto{\pgfqpoint{3.071888in}{3.139938in}}%
\pgfpathlineto{\pgfqpoint{3.074200in}{4.057174in}}%
\pgfpathlineto{\pgfqpoint{3.076513in}{3.849904in}}%
\pgfpathlineto{\pgfqpoint{3.078825in}{2.862394in}}%
\pgfpathlineto{\pgfqpoint{3.081138in}{2.749948in}}%
\pgfpathlineto{\pgfqpoint{3.083450in}{3.703865in}}%
\pgfpathlineto{\pgfqpoint{3.085762in}{4.118918in}}%
\pgfpathlineto{\pgfqpoint{3.090387in}{2.620791in}}%
\pgfpathlineto{\pgfqpoint{3.095012in}{4.105285in}}%
\pgfpathlineto{\pgfqpoint{3.097324in}{3.738251in}}%
\pgfpathlineto{\pgfqpoint{3.099636in}{2.766205in}}%
\pgfpathlineto{\pgfqpoint{3.101949in}{2.851099in}}%
\pgfpathlineto{\pgfqpoint{3.104261in}{3.849904in}}%
\pgfpathlineto{\pgfqpoint{3.106573in}{4.048266in}}%
\pgfpathlineto{\pgfqpoint{3.108886in}{3.102623in}}%
\pgfpathlineto{\pgfqpoint{3.111198in}{2.642384in}}%
\pgfpathlineto{\pgfqpoint{3.115823in}{4.147359in}}%
\pgfpathlineto{\pgfqpoint{3.120448in}{2.651642in}}%
\pgfpathlineto{\pgfqpoint{3.122760in}{3.076430in}}%
\pgfpathlineto{\pgfqpoint{3.125072in}{4.038970in}}%
\pgfpathlineto{\pgfqpoint{3.127385in}{3.854390in}}%
\pgfpathlineto{\pgfqpoint{3.129697in}{2.844027in}}%
\pgfpathlineto{\pgfqpoint{3.132009in}{2.783316in}}%
\pgfpathlineto{\pgfqpoint{3.134322in}{3.781485in}}%
\pgfpathlineto{\pgfqpoint{3.136634in}{4.077714in}}%
\pgfpathlineto{\pgfqpoint{3.141259in}{2.636273in}}%
\pgfpathlineto{\pgfqpoint{3.145883in}{4.147552in}}%
\pgfpathlineto{\pgfqpoint{3.150508in}{2.639918in}}%
\pgfpathlineto{\pgfqpoint{3.152821in}{3.136578in}}%
\pgfpathlineto{\pgfqpoint{3.155133in}{4.077714in}}%
\pgfpathlineto{\pgfqpoint{3.157445in}{3.770521in}}%
\pgfpathlineto{\pgfqpoint{3.159758in}{2.766205in}}%
\pgfpathlineto{\pgfqpoint{3.162070in}{2.878147in}}%
\pgfpathlineto{\pgfqpoint{3.164382in}{3.906124in}}%
\pgfpathlineto{\pgfqpoint{3.166695in}{3.986869in}}%
\pgfpathlineto{\pgfqpoint{3.169007in}{2.971473in}}%
\pgfpathlineto{\pgfqpoint{3.171319in}{2.707504in}}%
\pgfpathlineto{\pgfqpoint{3.173632in}{3.678524in}}%
\pgfpathlineto{\pgfqpoint{3.175944in}{4.110917in}}%
\pgfpathlineto{\pgfqpoint{3.180569in}{2.628687in}}%
\pgfpathlineto{\pgfqpoint{3.185193in}{4.147610in}}%
\pgfpathlineto{\pgfqpoint{3.189818in}{2.629765in}}%
\pgfpathlineto{\pgfqpoint{3.192131in}{3.210913in}}%
\pgfpathlineto{\pgfqpoint{3.194443in}{4.113899in}}%
\pgfpathlineto{\pgfqpoint{3.196755in}{3.657442in}}%
\pgfpathlineto{\pgfqpoint{3.199068in}{2.690583in}}%
\pgfpathlineto{\pgfqpoint{3.201380in}{3.020352in}}%
\pgfpathlineto{\pgfqpoint{3.203692in}{4.031559in}}%
\pgfpathlineto{\pgfqpoint{3.206005in}{3.827093in}}%
\pgfpathlineto{\pgfqpoint{3.208317in}{2.789057in}}%
\pgfpathlineto{\pgfqpoint{3.210629in}{2.872324in}}%
\pgfpathlineto{\pgfqpoint{3.212942in}{3.921961in}}%
\pgfpathlineto{\pgfqpoint{3.215254in}{3.953703in}}%
\pgfpathlineto{\pgfqpoint{3.217566in}{2.905269in}}%
\pgfpathlineto{\pgfqpoint{3.219879in}{2.766205in}}%
\pgfpathlineto{\pgfqpoint{3.222191in}{3.803076in}}%
\pgfpathlineto{\pgfqpoint{3.224503in}{4.040788in}}%
\pgfpathlineto{\pgfqpoint{3.226816in}{3.023476in}}%
\pgfpathlineto{\pgfqpoint{3.229128in}{2.696339in}}%
\pgfpathlineto{\pgfqpoint{3.231441in}{3.688318in}}%
\pgfpathlineto{\pgfqpoint{3.233753in}{4.095465in}}%
\pgfpathlineto{\pgfqpoint{3.236065in}{3.132552in}}%
\pgfpathlineto{\pgfqpoint{3.238378in}{2.654729in}}%
\pgfpathlineto{\pgfqpoint{3.243002in}{4.126121in}}%
\pgfpathlineto{\pgfqpoint{3.247627in}{2.633000in}}%
\pgfpathlineto{\pgfqpoint{3.252252in}{4.140807in}}%
\pgfpathlineto{\pgfqpoint{3.256876in}{2.623659in}}%
\pgfpathlineto{\pgfqpoint{3.261501in}{4.146286in}}%
\pgfpathlineto{\pgfqpoint{3.266126in}{2.620791in}}%
\pgfpathlineto{\pgfqpoint{3.270751in}{4.147552in}}%
\pgfpathlineto{\pgfqpoint{3.275375in}{2.620369in}}%
\pgfpathlineto{\pgfqpoint{3.280000in}{4.147636in}}%
\pgfpathlineto{\pgfqpoint{3.284625in}{2.620369in}}%
\pgfpathlineto{\pgfqpoint{3.289249in}{4.147552in}}%
\pgfpathlineto{\pgfqpoint{3.293874in}{2.620791in}}%
\pgfpathlineto{\pgfqpoint{3.298499in}{4.146286in}}%
\pgfpathlineto{\pgfqpoint{3.303124in}{2.623659in}}%
\pgfpathlineto{\pgfqpoint{3.307748in}{4.140807in}}%
\pgfpathlineto{\pgfqpoint{3.312373in}{2.633000in}}%
\pgfpathlineto{\pgfqpoint{3.316998in}{4.126121in}}%
\pgfpathlineto{\pgfqpoint{3.321622in}{2.654729in}}%
\pgfpathlineto{\pgfqpoint{3.323935in}{3.635447in}}%
\pgfpathlineto{\pgfqpoint{3.326247in}{4.095465in}}%
\pgfpathlineto{\pgfqpoint{3.328559in}{3.079681in}}%
\pgfpathlineto{\pgfqpoint{3.330872in}{2.696339in}}%
\pgfpathlineto{\pgfqpoint{3.333184in}{3.744523in}}%
\pgfpathlineto{\pgfqpoint{3.335497in}{4.040788in}}%
\pgfpathlineto{\pgfqpoint{3.337809in}{2.964924in}}%
\pgfpathlineto{\pgfqpoint{3.340121in}{2.766205in}}%
\pgfpathlineto{\pgfqpoint{3.342434in}{3.862731in}}%
\pgfpathlineto{\pgfqpoint{3.344746in}{3.953703in}}%
\pgfpathlineto{\pgfqpoint{3.347058in}{2.846039in}}%
\pgfpathlineto{\pgfqpoint{3.349371in}{2.872324in}}%
\pgfpathlineto{\pgfqpoint{3.351683in}{3.978943in}}%
\pgfpathlineto{\pgfqpoint{3.353995in}{3.827093in}}%
\pgfpathlineto{\pgfqpoint{3.356308in}{2.736441in}}%
\pgfpathlineto{\pgfqpoint{3.358620in}{3.020352in}}%
\pgfpathlineto{\pgfqpoint{3.360932in}{4.077417in}}%
\pgfpathlineto{\pgfqpoint{3.363245in}{3.657442in}}%
\pgfpathlineto{\pgfqpoint{3.365557in}{2.654101in}}%
\pgfpathlineto{\pgfqpoint{3.367869in}{3.210913in}}%
\pgfpathlineto{\pgfqpoint{3.370182in}{4.138235in}}%
\pgfpathlineto{\pgfqpoint{3.374807in}{2.620390in}}%
\pgfpathlineto{\pgfqpoint{3.379431in}{4.139312in}}%
\pgfpathlineto{\pgfqpoint{3.384056in}{2.657083in}}%
\pgfpathlineto{\pgfqpoint{3.386368in}{3.678524in}}%
\pgfpathlineto{\pgfqpoint{3.388681in}{4.060496in}}%
\pgfpathlineto{\pgfqpoint{3.390993in}{2.971473in}}%
\pgfpathlineto{\pgfqpoint{3.393305in}{2.781131in}}%
\pgfpathlineto{\pgfqpoint{3.395618in}{3.906124in}}%
\pgfpathlineto{\pgfqpoint{3.397930in}{3.889853in}}%
\pgfpathlineto{\pgfqpoint{3.400242in}{2.766205in}}%
\pgfpathlineto{\pgfqpoint{3.402555in}{2.997479in}}%
\pgfpathlineto{\pgfqpoint{3.404867in}{4.077714in}}%
\pgfpathlineto{\pgfqpoint{3.407179in}{3.631422in}}%
\pgfpathlineto{\pgfqpoint{3.409492in}{2.639918in}}%
\pgfpathlineto{\pgfqpoint{3.414117in}{4.147552in}}%
\pgfpathlineto{\pgfqpoint{3.418741in}{2.636273in}}%
\pgfpathlineto{\pgfqpoint{3.421054in}{3.622001in}}%
\pgfpathlineto{\pgfqpoint{3.423366in}{4.077714in}}%
\pgfpathlineto{\pgfqpoint{3.425678in}{2.986515in}}%
\pgfpathlineto{\pgfqpoint{3.427991in}{2.783316in}}%
\pgfpathlineto{\pgfqpoint{3.430303in}{3.923972in}}%
\pgfpathlineto{\pgfqpoint{3.432615in}{3.854390in}}%
\pgfpathlineto{\pgfqpoint{3.434928in}{2.729030in}}%
\pgfpathlineto{\pgfqpoint{3.437240in}{3.076430in}}%
\pgfpathlineto{\pgfqpoint{3.439552in}{4.116357in}}%
\pgfpathlineto{\pgfqpoint{3.444177in}{2.620641in}}%
\pgfpathlineto{\pgfqpoint{3.448802in}{4.125616in}}%
\pgfpathlineto{\pgfqpoint{3.451114in}{3.102623in}}%
\pgfpathlineto{\pgfqpoint{3.453427in}{2.719734in}}%
\pgfpathlineto{\pgfqpoint{3.455739in}{3.849904in}}%
\pgfpathlineto{\pgfqpoint{3.458051in}{3.916900in}}%
\pgfpathlineto{\pgfqpoint{3.460364in}{2.766205in}}%
\pgfpathlineto{\pgfqpoint{3.462676in}{3.029749in}}%
\pgfpathlineto{\pgfqpoint{3.464988in}{4.105285in}}%
\pgfpathlineto{\pgfqpoint{3.467301in}{3.524441in}}%
\pgfpathlineto{\pgfqpoint{3.469613in}{2.620791in}}%
\pgfpathlineto{\pgfqpoint{3.474238in}{4.118918in}}%
\pgfpathlineto{\pgfqpoint{3.476550in}{3.064135in}}%
\pgfpathlineto{\pgfqpoint{3.478862in}{2.749948in}}%
\pgfpathlineto{\pgfqpoint{3.481175in}{3.905606in}}%
\pgfpathlineto{\pgfqpoint{3.483487in}{3.849904in}}%
\pgfpathlineto{\pgfqpoint{3.485800in}{2.710826in}}%
\pgfpathlineto{\pgfqpoint{3.488112in}{3.139938in}}%
\pgfpathlineto{\pgfqpoint{3.490424in}{4.138235in}}%
\pgfpathlineto{\pgfqpoint{3.495049in}{2.633389in}}%
\pgfpathlineto{\pgfqpoint{3.497361in}{3.652133in}}%
\pgfpathlineto{\pgfqpoint{3.499674in}{4.041149in}}%
\pgfpathlineto{\pgfqpoint{3.501986in}{2.885096in}}%
\pgfpathlineto{\pgfqpoint{3.504298in}{2.904164in}}%
\pgfpathlineto{\pgfqpoint{3.506611in}{4.054814in}}%
\pgfpathlineto{\pgfqpoint{3.508923in}{3.620651in}}%
\pgfpathlineto{\pgfqpoint{3.511235in}{2.626819in}}%
\pgfpathlineto{\pgfqpoint{3.515860in}{4.124762in}}%
\pgfpathlineto{\pgfqpoint{3.518172in}{3.061559in}}%
\pgfpathlineto{\pgfqpoint{3.520485in}{2.766205in}}%
\pgfpathlineto{\pgfqpoint{3.522797in}{3.943670in}}%
\pgfpathlineto{\pgfqpoint{3.525110in}{3.784511in}}%
\pgfpathlineto{\pgfqpoint{3.527422in}{2.667747in}}%
\pgfpathlineto{\pgfqpoint{3.532047in}{4.146779in}}%
\pgfpathlineto{\pgfqpoint{3.536671in}{2.698514in}}%
\pgfpathlineto{\pgfqpoint{3.538984in}{3.854390in}}%
\pgfpathlineto{\pgfqpoint{3.541296in}{3.874796in}}%
\pgfpathlineto{\pgfqpoint{3.543608in}{2.709157in}}%
\pgfpathlineto{\pgfqpoint{3.545921in}{3.177230in}}%
\pgfpathlineto{\pgfqpoint{3.548233in}{4.146286in}}%
\pgfpathlineto{\pgfqpoint{3.552858in}{2.673571in}}%
\pgfpathlineto{\pgfqpoint{3.555170in}{3.812523in}}%
\pgfpathlineto{\pgfqpoint{3.557483in}{3.906124in}}%
\pgfpathlineto{\pgfqpoint{3.559795in}{2.724691in}}%
\pgfpathlineto{\pgfqpoint{3.562107in}{3.156138in}}%
\pgfpathlineto{\pgfqpoint{3.564420in}{4.145473in}}%
\pgfpathlineto{\pgfqpoint{3.569044in}{2.676472in}}%
\pgfpathlineto{\pgfqpoint{3.571357in}{3.827093in}}%
\pgfpathlineto{\pgfqpoint{3.573669in}{3.885585in}}%
\pgfpathlineto{\pgfqpoint{3.575981in}{2.706519in}}%
\pgfpathlineto{\pgfqpoint{3.578294in}{3.201939in}}%
\pgfpathlineto{\pgfqpoint{3.580606in}{4.147636in}}%
\pgfpathlineto{\pgfqpoint{3.582918in}{3.200561in}}%
\pgfpathlineto{\pgfqpoint{3.585231in}{2.709157in}}%
\pgfpathlineto{\pgfqpoint{3.587543in}{3.895148in}}%
\pgfpathlineto{\pgfqpoint{3.589855in}{3.808401in}}%
\pgfpathlineto{\pgfqpoint{3.592168in}{2.663888in}}%
\pgfpathlineto{\pgfqpoint{3.596793in}{4.137092in}}%
\pgfpathlineto{\pgfqpoint{3.599105in}{3.071241in}}%
\pgfpathlineto{\pgfqpoint{3.601417in}{2.789948in}}%
\pgfpathlineto{\pgfqpoint{3.603730in}{4.000122in}}%
\pgfpathlineto{\pgfqpoint{3.606042in}{3.660752in}}%
\pgfpathlineto{\pgfqpoint{3.608354in}{2.624798in}}%
\pgfpathlineto{\pgfqpoint{3.612979in}{4.080068in}}%
\pgfpathlineto{\pgfqpoint{3.615291in}{2.896474in}}%
\pgfpathlineto{\pgfqpoint{3.617604in}{2.947871in}}%
\pgfpathlineto{\pgfqpoint{3.619916in}{4.105051in}}%
\pgfpathlineto{\pgfqpoint{3.624541in}{2.637897in}}%
\pgfpathlineto{\pgfqpoint{3.626853in}{3.745149in}}%
\pgfpathlineto{\pgfqpoint{3.629166in}{3.930953in}}%
\pgfpathlineto{\pgfqpoint{3.631478in}{2.720084in}}%
\pgfpathlineto{\pgfqpoint{3.633790in}{3.207458in}}%
\pgfpathlineto{\pgfqpoint{3.636103in}{4.146602in}}%
\pgfpathlineto{\pgfqpoint{3.638415in}{3.129873in}}%
\pgfpathlineto{\pgfqpoint{3.640727in}{2.766205in}}%
\pgfpathlineto{\pgfqpoint{3.643040in}{3.991199in}}%
\pgfpathlineto{\pgfqpoint{3.645352in}{3.652133in}}%
\pgfpathlineto{\pgfqpoint{3.647664in}{2.621933in}}%
\pgfpathlineto{\pgfqpoint{3.649977in}{3.559851in}}%
\pgfpathlineto{\pgfqpoint{3.652289in}{4.042591in}}%
\pgfpathlineto{\pgfqpoint{3.654601in}{2.820003in}}%
\pgfpathlineto{\pgfqpoint{3.656914in}{3.060273in}}%
\pgfpathlineto{\pgfqpoint{3.659226in}{4.140807in}}%
\pgfpathlineto{\pgfqpoint{3.663851in}{2.706519in}}%
\pgfpathlineto{\pgfqpoint{3.666163in}{3.925976in}}%
\pgfpathlineto{\pgfqpoint{3.668476in}{3.730051in}}%
\pgfpathlineto{\pgfqpoint{3.670788in}{2.629765in}}%
\pgfpathlineto{\pgfqpoint{3.675413in}{4.063111in}}%
\pgfpathlineto{\pgfqpoint{3.677725in}{2.839529in}}%
\pgfpathlineto{\pgfqpoint{3.680037in}{3.047473in}}%
\pgfpathlineto{\pgfqpoint{3.682350in}{4.140807in}}%
\pgfpathlineto{\pgfqpoint{3.686974in}{2.720084in}}%
\pgfpathlineto{\pgfqpoint{3.689287in}{3.955119in}}%
\pgfpathlineto{\pgfqpoint{3.691599in}{3.678524in}}%
\pgfpathlineto{\pgfqpoint{3.693911in}{2.621933in}}%
\pgfpathlineto{\pgfqpoint{3.696224in}{3.587351in}}%
\pgfpathlineto{\pgfqpoint{3.698536in}{4.012057in}}%
\pgfpathlineto{\pgfqpoint{3.700848in}{2.766205in}}%
\pgfpathlineto{\pgfqpoint{3.703161in}{3.166322in}}%
\pgfpathlineto{\pgfqpoint{3.705473in}{4.146602in}}%
\pgfpathlineto{\pgfqpoint{3.707786in}{3.090130in}}%
\pgfpathlineto{\pgfqpoint{3.710098in}{2.820003in}}%
\pgfpathlineto{\pgfqpoint{3.712410in}{4.059836in}}%
\pgfpathlineto{\pgfqpoint{3.714723in}{3.487309in}}%
\pgfpathlineto{\pgfqpoint{3.717035in}{2.637897in}}%
\pgfpathlineto{\pgfqpoint{3.719347in}{3.794135in}}%
\pgfpathlineto{\pgfqpoint{3.721660in}{3.845958in}}%
\pgfpathlineto{\pgfqpoint{3.723972in}{2.653065in}}%
\pgfpathlineto{\pgfqpoint{3.728597in}{4.080068in}}%
\pgfpathlineto{\pgfqpoint{3.730909in}{2.844027in}}%
\pgfpathlineto{\pgfqpoint{3.733221in}{3.071241in}}%
\pgfpathlineto{\pgfqpoint{3.735534in}{4.146409in}}%
\pgfpathlineto{\pgfqpoint{3.737846in}{3.150725in}}%
\pgfpathlineto{\pgfqpoint{3.740159in}{2.789948in}}%
\pgfpathlineto{\pgfqpoint{3.742471in}{4.046508in}}%
\pgfpathlineto{\pgfqpoint{3.744783in}{3.496443in}}%
\pgfpathlineto{\pgfqpoint{3.747096in}{2.639918in}}%
\pgfpathlineto{\pgfqpoint{3.749408in}{3.817211in}}%
\pgfpathlineto{\pgfqpoint{3.751720in}{3.808401in}}%
\pgfpathlineto{\pgfqpoint{3.754033in}{2.637001in}}%
\pgfpathlineto{\pgfqpoint{3.758657in}{4.032310in}}%
\pgfpathlineto{\pgfqpoint{3.760970in}{2.766205in}}%
\pgfpathlineto{\pgfqpoint{3.763282in}{3.201939in}}%
\pgfpathlineto{\pgfqpoint{3.765594in}{4.139208in}}%
\pgfpathlineto{\pgfqpoint{3.767907in}{2.991374in}}%
\pgfpathlineto{\pgfqpoint{3.770219in}{2.929423in}}%
\pgfpathlineto{\pgfqpoint{3.772531in}{4.125276in}}%
\pgfpathlineto{\pgfqpoint{3.777156in}{2.734194in}}%
\pgfpathlineto{\pgfqpoint{3.779469in}{4.006763in}}%
\pgfpathlineto{\pgfqpoint{3.781781in}{3.546701in}}%
\pgfpathlineto{\pgfqpoint{3.784093in}{2.634049in}}%
\pgfpathlineto{\pgfqpoint{3.786406in}{3.812523in}}%
\pgfpathlineto{\pgfqpoint{3.788718in}{3.794135in}}%
\pgfpathlineto{\pgfqpoint{3.791030in}{2.629765in}}%
\pgfpathlineto{\pgfqpoint{3.793343in}{3.576382in}}%
\pgfpathlineto{\pgfqpoint{3.795655in}{3.983367in}}%
\pgfpathlineto{\pgfqpoint{3.797967in}{2.709157in}}%
\pgfpathlineto{\pgfqpoint{3.802592in}{4.101475in}}%
\pgfpathlineto{\pgfqpoint{3.804904in}{2.852117in}}%
\pgfpathlineto{\pgfqpoint{3.807217in}{3.102623in}}%
\pgfpathlineto{\pgfqpoint{3.809529in}{4.146779in}}%
\pgfpathlineto{\pgfqpoint{3.811841in}{3.035420in}}%
\pgfpathlineto{\pgfqpoint{3.814154in}{2.910819in}}%
\pgfpathlineto{\pgfqpoint{3.816466in}{4.126121in}}%
\pgfpathlineto{\pgfqpoint{3.821091in}{2.766205in}}%
\pgfpathlineto{\pgfqpoint{3.823403in}{4.051738in}}%
\pgfpathlineto{\pgfqpoint{3.828028in}{2.672277in}}%
\pgfpathlineto{\pgfqpoint{3.830340in}{3.938329in}}%
\pgfpathlineto{\pgfqpoint{3.832653in}{3.620651in}}%
\pgfpathlineto{\pgfqpoint{3.834965in}{2.626819in}}%
\pgfpathlineto{\pgfqpoint{3.837277in}{3.800699in}}%
\pgfpathlineto{\pgfqpoint{3.839590in}{3.779665in}}%
\pgfpathlineto{\pgfqpoint{3.841902in}{2.623726in}}%
\pgfpathlineto{\pgfqpoint{3.844214in}{3.652133in}}%
\pgfpathlineto{\pgfqpoint{3.846527in}{3.908708in}}%
\pgfpathlineto{\pgfqpoint{3.848839in}{2.654729in}}%
\pgfpathlineto{\pgfqpoint{3.853464in}{4.006763in}}%
\pgfpathlineto{\pgfqpoint{3.855776in}{2.710826in}}%
\pgfpathlineto{\pgfqpoint{3.860401in}{4.075622in}}%
\pgfpathlineto{\pgfqpoint{3.862713in}{2.783316in}}%
\pgfpathlineto{\pgfqpoint{3.865026in}{3.235196in}}%
\pgfpathlineto{\pgfqpoint{3.867338in}{4.118918in}}%
\pgfpathlineto{\pgfqpoint{3.869650in}{2.864471in}}%
\pgfpathlineto{\pgfqpoint{3.871963in}{3.123859in}}%
\pgfpathlineto{\pgfqpoint{3.874275in}{4.141272in}}%
\pgfpathlineto{\pgfqpoint{3.876587in}{2.947871in}}%
\pgfpathlineto{\pgfqpoint{3.878900in}{3.029749in}}%
\pgfpathlineto{\pgfqpoint{3.881212in}{4.147636in}}%
\pgfpathlineto{\pgfqpoint{3.883524in}{3.028492in}}%
\pgfpathlineto{\pgfqpoint{3.885837in}{2.952544in}}%
\pgfpathlineto{\pgfqpoint{3.888149in}{4.142812in}}%
\pgfpathlineto{\pgfqpoint{3.890462in}{3.102623in}}%
\pgfpathlineto{\pgfqpoint{3.892774in}{2.891032in}}%
\pgfpathlineto{\pgfqpoint{3.895086in}{4.131145in}}%
\pgfpathlineto{\pgfqpoint{3.897399in}{3.167683in}}%
\pgfpathlineto{\pgfqpoint{3.899711in}{2.843526in}}%
\pgfpathlineto{\pgfqpoint{3.902023in}{4.116357in}}%
\pgfpathlineto{\pgfqpoint{3.904336in}{3.221992in}}%
\pgfpathlineto{\pgfqpoint{3.906648in}{2.808194in}}%
\pgfpathlineto{\pgfqpoint{3.908960in}{4.101475in}}%
\pgfpathlineto{\pgfqpoint{3.913585in}{2.783316in}}%
\pgfpathlineto{\pgfqpoint{3.915897in}{4.088827in}}%
\pgfpathlineto{\pgfqpoint{3.920522in}{2.767459in}}%
\pgfpathlineto{\pgfqpoint{3.922834in}{4.080068in}}%
\pgfpathlineto{\pgfqpoint{3.927459in}{2.759598in}}%
\pgfpathlineto{\pgfqpoint{3.929772in}{4.076223in}}%
\pgfpathlineto{\pgfqpoint{3.934396in}{2.759190in}}%
\pgfpathlineto{\pgfqpoint{3.936709in}{4.077714in}}%
\pgfpathlineto{\pgfqpoint{3.941333in}{2.766205in}}%
\pgfpathlineto{\pgfqpoint{3.943646in}{4.084379in}}%
\pgfpathlineto{\pgfqpoint{3.948270in}{2.781131in}}%
\pgfpathlineto{\pgfqpoint{3.950583in}{4.095465in}}%
\pgfpathlineto{\pgfqpoint{3.955207in}{2.804942in}}%
\pgfpathlineto{\pgfqpoint{3.957520in}{4.109601in}}%
\pgfpathlineto{\pgfqpoint{3.959832in}{3.210913in}}%
\pgfpathlineto{\pgfqpoint{3.962145in}{2.839032in}}%
\pgfpathlineto{\pgfqpoint{3.964457in}{4.124762in}}%
\pgfpathlineto{\pgfqpoint{3.966769in}{3.154107in}}%
\pgfpathlineto{\pgfqpoint{3.969082in}{2.885096in}}%
\pgfpathlineto{\pgfqpoint{3.971394in}{4.138235in}}%
\pgfpathlineto{\pgfqpoint{3.973706in}{3.086858in}}%
\pgfpathlineto{\pgfqpoint{3.976019in}{2.944963in}}%
\pgfpathlineto{\pgfqpoint{3.978331in}{4.146602in}}%
\pgfpathlineto{\pgfqpoint{3.980643in}{3.011026in}}%
\pgfpathlineto{\pgfqpoint{3.982956in}{3.020352in}}%
\pgfpathlineto{\pgfqpoint{3.985268in}{4.145782in}}%
\pgfpathlineto{\pgfqpoint{3.987580in}{2.929423in}}%
\pgfpathlineto{\pgfqpoint{3.989893in}{3.112547in}}%
\pgfpathlineto{\pgfqpoint{3.992205in}{4.131145in}}%
\pgfpathlineto{\pgfqpoint{3.994517in}{2.846039in}}%
\pgfpathlineto{\pgfqpoint{3.996830in}{3.221992in}}%
\pgfpathlineto{\pgfqpoint{3.999142in}{4.097760in}}%
\pgfpathlineto{\pgfqpoint{4.001455in}{2.766205in}}%
\pgfpathlineto{\pgfqpoint{4.006079in}{4.040788in}}%
\pgfpathlineto{\pgfqpoint{4.008392in}{2.696648in}}%
\pgfpathlineto{\pgfqpoint{4.013016in}{3.956060in}}%
\pgfpathlineto{\pgfqpoint{4.015329in}{2.645353in}}%
\pgfpathlineto{\pgfqpoint{4.017641in}{3.635447in}}%
\pgfpathlineto{\pgfqpoint{4.019953in}{3.840855in}}%
\pgfpathlineto{\pgfqpoint{4.022266in}{2.621155in}}%
\pgfpathlineto{\pgfqpoint{4.024578in}{3.784511in}}%
\pgfpathlineto{\pgfqpoint{4.026890in}{3.694815in}}%
\pgfpathlineto{\pgfqpoint{4.029203in}{2.633000in}}%
\pgfpathlineto{\pgfqpoint{4.031515in}{3.923972in}}%
\pgfpathlineto{\pgfqpoint{4.033828in}{3.520951in}}%
\pgfpathlineto{\pgfqpoint{4.036140in}{2.688810in}}%
\pgfpathlineto{\pgfqpoint{4.038452in}{4.040788in}}%
\pgfpathlineto{\pgfqpoint{4.043077in}{2.793982in}}%
\pgfpathlineto{\pgfqpoint{4.045389in}{4.120252in}}%
\pgfpathlineto{\pgfqpoint{4.047702in}{3.123859in}}%
\pgfpathlineto{\pgfqpoint{4.050014in}{2.949621in}}%
\pgfpathlineto{\pgfqpoint{4.052326in}{4.147552in}}%
\pgfpathlineto{\pgfqpoint{4.054639in}{2.929993in}}%
\pgfpathlineto{\pgfqpoint{4.056951in}{3.150725in}}%
\pgfpathlineto{\pgfqpoint{4.059263in}{4.110042in}}%
\pgfpathlineto{\pgfqpoint{4.061576in}{2.766205in}}%
\pgfpathlineto{\pgfqpoint{4.066200in}{4.000122in}}%
\pgfpathlineto{\pgfqpoint{4.068513in}{2.655790in}}%
\pgfpathlineto{\pgfqpoint{4.070825in}{3.630751in}}%
\pgfpathlineto{\pgfqpoint{4.073138in}{3.818379in}}%
\pgfpathlineto{\pgfqpoint{4.075450in}{2.620791in}}%
\pgfpathlineto{\pgfqpoint{4.077762in}{3.860516in}}%
\pgfpathlineto{\pgfqpoint{4.080075in}{3.576382in}}%
\pgfpathlineto{\pgfqpoint{4.082387in}{2.677545in}}%
\pgfpathlineto{\pgfqpoint{4.084699in}{4.040788in}}%
\pgfpathlineto{\pgfqpoint{4.089324in}{2.831627in}}%
\pgfpathlineto{\pgfqpoint{4.091636in}{4.138235in}}%
\pgfpathlineto{\pgfqpoint{4.093949in}{3.020352in}}%
\pgfpathlineto{\pgfqpoint{4.096261in}{3.073185in}}%
\pgfpathlineto{\pgfqpoint{4.098573in}{4.126121in}}%
\pgfpathlineto{\pgfqpoint{4.100886in}{2.787281in}}%
\pgfpathlineto{\pgfqpoint{4.105510in}{3.992059in}}%
\pgfpathlineto{\pgfqpoint{4.107823in}{2.645353in}}%
\pgfpathlineto{\pgfqpoint{4.110135in}{3.688318in}}%
\pgfpathlineto{\pgfqpoint{4.112448in}{3.745149in}}%
\pgfpathlineto{\pgfqpoint{4.114760in}{2.632365in}}%
\pgfpathlineto{\pgfqpoint{4.117072in}{3.957469in}}%
\pgfpathlineto{\pgfqpoint{4.121697in}{2.766205in}}%
\pgfpathlineto{\pgfqpoint{4.124009in}{4.121000in}}%
\pgfpathlineto{\pgfqpoint{4.126322in}{3.076430in}}%
\pgfpathlineto{\pgfqpoint{4.128634in}{3.034789in}}%
\pgfpathlineto{\pgfqpoint{4.130946in}{4.131145in}}%
\pgfpathlineto{\pgfqpoint{4.133259in}{2.789057in}}%
\pgfpathlineto{\pgfqpoint{4.137883in}{3.969030in}}%
\pgfpathlineto{\pgfqpoint{4.140196in}{2.633000in}}%
\pgfpathlineto{\pgfqpoint{4.142508in}{3.756974in}}%
\pgfpathlineto{\pgfqpoint{4.144821in}{3.657442in}}%
\pgfpathlineto{\pgfqpoint{4.147133in}{2.661557in}}%
\pgfpathlineto{\pgfqpoint{4.149445in}{4.037873in}}%
\pgfpathlineto{\pgfqpoint{4.151758in}{3.264541in}}%
\pgfpathlineto{\pgfqpoint{4.154070in}{2.885096in}}%
\pgfpathlineto{\pgfqpoint{4.156382in}{4.147610in}}%
\pgfpathlineto{\pgfqpoint{4.158695in}{2.893748in}}%
\pgfpathlineto{\pgfqpoint{4.161007in}{3.257536in}}%
\pgfpathlineto{\pgfqpoint{4.163319in}{4.037873in}}%
\pgfpathlineto{\pgfqpoint{4.165632in}{2.658399in}}%
\pgfpathlineto{\pgfqpoint{4.167944in}{3.678524in}}%
\pgfpathlineto{\pgfqpoint{4.170256in}{3.723073in}}%
\pgfpathlineto{\pgfqpoint{4.172569in}{2.645353in}}%
\pgfpathlineto{\pgfqpoint{4.174881in}{4.016067in}}%
\pgfpathlineto{\pgfqpoint{4.179506in}{2.878147in}}%
\pgfpathlineto{\pgfqpoint{4.181818in}{4.147636in}}%
\pgfpathlineto{\pgfqpoint{4.184131in}{2.877084in}}%
\pgfpathlineto{\pgfqpoint{4.186443in}{3.294773in}}%
\pgfpathlineto{\pgfqpoint{4.188755in}{4.008810in}}%
\pgfpathlineto{\pgfqpoint{4.191068in}{2.639918in}}%
\pgfpathlineto{\pgfqpoint{4.193380in}{3.754494in}}%
\pgfpathlineto{\pgfqpoint{4.195692in}{3.630751in}}%
\pgfpathlineto{\pgfqpoint{4.198005in}{2.683056in}}%
\pgfpathlineto{\pgfqpoint{4.200317in}{4.080068in}}%
\pgfpathlineto{\pgfqpoint{4.202629in}{3.145999in}}%
\pgfpathlineto{\pgfqpoint{4.204942in}{3.010406in}}%
\pgfpathlineto{\pgfqpoint{4.207254in}{4.126949in}}%
\pgfpathlineto{\pgfqpoint{4.209566in}{2.749948in}}%
\pgfpathlineto{\pgfqpoint{4.211879in}{3.503459in}}%
\pgfpathlineto{\pgfqpoint{4.214191in}{3.854390in}}%
\pgfpathlineto{\pgfqpoint{4.216503in}{2.623338in}}%
\pgfpathlineto{\pgfqpoint{4.218816in}{3.953703in}}%
\pgfpathlineto{\pgfqpoint{4.223441in}{2.843526in}}%
\pgfpathlineto{\pgfqpoint{4.225753in}{4.147359in}}%
\pgfpathlineto{\pgfqpoint{4.228065in}{2.872324in}}%
\pgfpathlineto{\pgfqpoint{4.232690in}{3.972211in}}%
\pgfpathlineto{\pgfqpoint{4.235002in}{2.625188in}}%
\pgfpathlineto{\pgfqpoint{4.237315in}{3.849904in}}%
\pgfpathlineto{\pgfqpoint{4.239627in}{3.493635in}}%
\pgfpathlineto{\pgfqpoint{4.241939in}{2.766205in}}%
\pgfpathlineto{\pgfqpoint{4.244252in}{4.136855in}}%
\pgfpathlineto{\pgfqpoint{4.246564in}{2.947871in}}%
\pgfpathlineto{\pgfqpoint{4.248876in}{3.243559in}}%
\pgfpathlineto{\pgfqpoint{4.251189in}{4.016465in}}%
\pgfpathlineto{\pgfqpoint{4.253501in}{2.634726in}}%
\pgfpathlineto{\pgfqpoint{4.255814in}{3.808401in}}%
\pgfpathlineto{\pgfqpoint{4.258126in}{3.532804in}}%
\pgfpathlineto{\pgfqpoint{4.260438in}{2.749948in}}%
\pgfpathlineto{\pgfqpoint{4.262751in}{4.133816in}}%
\pgfpathlineto{\pgfqpoint{4.265063in}{2.952544in}}%
\pgfpathlineto{\pgfqpoint{4.267375in}{3.249144in}}%
\pgfpathlineto{\pgfqpoint{4.269688in}{4.006763in}}%
\pgfpathlineto{\pgfqpoint{4.272000in}{2.629765in}}%
\pgfpathlineto{\pgfqpoint{4.274312in}{3.840855in}}%
\pgfpathlineto{\pgfqpoint{4.276625in}{3.482384in}}%
\pgfpathlineto{\pgfqpoint{4.278937in}{2.786838in}}%
\pgfpathlineto{\pgfqpoint{4.281249in}{4.144274in}}%
\pgfpathlineto{\pgfqpoint{4.283562in}{2.885096in}}%
\pgfpathlineto{\pgfqpoint{4.288186in}{3.938329in}}%
\pgfpathlineto{\pgfqpoint{4.290499in}{2.620364in}}%
\pgfpathlineto{\pgfqpoint{4.292811in}{3.938329in}}%
\pgfpathlineto{\pgfqpoint{4.297436in}{2.893748in}}%
\pgfpathlineto{\pgfqpoint{4.299748in}{4.141983in}}%
\pgfpathlineto{\pgfqpoint{4.302061in}{2.766205in}}%
\pgfpathlineto{\pgfqpoint{4.304373in}{3.531411in}}%
\pgfpathlineto{\pgfqpoint{4.306685in}{3.784511in}}%
\pgfpathlineto{\pgfqpoint{4.308998in}{2.648890in}}%
\pgfpathlineto{\pgfqpoint{4.311310in}{4.065369in}}%
\pgfpathlineto{\pgfqpoint{4.313622in}{3.113874in}}%
\pgfpathlineto{\pgfqpoint{4.315935in}{3.102623in}}%
\pgfpathlineto{\pgfqpoint{4.318247in}{4.069485in}}%
\pgfpathlineto{\pgfqpoint{4.320559in}{2.649858in}}%
\pgfpathlineto{\pgfqpoint{4.322872in}{3.788732in}}%
\pgfpathlineto{\pgfqpoint{4.325184in}{3.515361in}}%
\pgfpathlineto{\pgfqpoint{4.327497in}{2.784633in}}%
\pgfpathlineto{\pgfqpoint{4.329809in}{4.146286in}}%
\pgfpathlineto{\pgfqpoint{4.332121in}{2.844027in}}%
\pgfpathlineto{\pgfqpoint{4.336746in}{3.852711in}}%
\pgfpathlineto{\pgfqpoint{4.339058in}{2.634049in}}%
\pgfpathlineto{\pgfqpoint{4.341371in}{4.043308in}}%
\pgfpathlineto{\pgfqpoint{4.343683in}{3.139938in}}%
\pgfpathlineto{\pgfqpoint{4.345995in}{3.094065in}}%
\pgfpathlineto{\pgfqpoint{4.348308in}{4.065369in}}%
\pgfpathlineto{\pgfqpoint{4.350620in}{2.642723in}}%
\pgfpathlineto{\pgfqpoint{4.352932in}{3.827093in}}%
\pgfpathlineto{\pgfqpoint{4.355245in}{3.451341in}}%
\pgfpathlineto{\pgfqpoint{4.357557in}{2.839529in}}%
\pgfpathlineto{\pgfqpoint{4.359869in}{4.145579in}}%
\pgfpathlineto{\pgfqpoint{4.362182in}{2.766205in}}%
\pgfpathlineto{\pgfqpoint{4.364494in}{3.567439in}}%
\pgfpathlineto{\pgfqpoint{4.366807in}{3.719890in}}%
\pgfpathlineto{\pgfqpoint{4.369119in}{2.686479in}}%
\pgfpathlineto{\pgfqpoint{4.371431in}{4.118918in}}%
\pgfpathlineto{\pgfqpoint{4.373744in}{2.950789in}}%
\pgfpathlineto{\pgfqpoint{4.376056in}{3.317366in}}%
\pgfpathlineto{\pgfqpoint{4.378368in}{3.918930in}}%
\pgfpathlineto{\pgfqpoint{4.380681in}{2.624798in}}%
\pgfpathlineto{\pgfqpoint{4.382993in}{4.023923in}}%
\pgfpathlineto{\pgfqpoint{4.385305in}{3.150725in}}%
\pgfpathlineto{\pgfqpoint{4.387618in}{3.107247in}}%
\pgfpathlineto{\pgfqpoint{4.389930in}{4.046508in}}%
\pgfpathlineto{\pgfqpoint{4.392242in}{2.629765in}}%
\pgfpathlineto{\pgfqpoint{4.394555in}{3.897779in}}%
\pgfpathlineto{\pgfqpoint{4.399179in}{2.947871in}}%
\pgfpathlineto{\pgfqpoint{4.401492in}{4.115140in}}%
\pgfpathlineto{\pgfqpoint{4.403804in}{2.673571in}}%
\pgfpathlineto{\pgfqpoint{4.406117in}{3.769296in}}%
\pgfpathlineto{\pgfqpoint{4.408429in}{3.487309in}}%
\pgfpathlineto{\pgfqpoint{4.410741in}{2.837047in}}%
\pgfpathlineto{\pgfqpoint{4.413054in}{4.142891in}}%
\pgfpathlineto{\pgfqpoint{4.415366in}{2.731967in}}%
\pgfpathlineto{\pgfqpoint{4.417678in}{3.657442in}}%
\pgfpathlineto{\pgfqpoint{4.419991in}{3.601678in}}%
\pgfpathlineto{\pgfqpoint{4.422303in}{2.766205in}}%
\pgfpathlineto{\pgfqpoint{4.424615in}{4.147430in}}%
\pgfpathlineto{\pgfqpoint{4.426928in}{2.786838in}}%
\pgfpathlineto{\pgfqpoint{4.429240in}{3.572946in}}%
\pgfpathlineto{\pgfqpoint{4.431552in}{3.678524in}}%
\pgfpathlineto{\pgfqpoint{4.433865in}{2.725409in}}%
\pgfpathlineto{\pgfqpoint{4.436177in}{4.142891in}}%
\pgfpathlineto{\pgfqpoint{4.438490in}{2.826265in}}%
\pgfpathlineto{\pgfqpoint{4.440802in}{3.520951in}}%
\pgfpathlineto{\pgfqpoint{4.443114in}{3.720527in}}%
\pgfpathlineto{\pgfqpoint{4.445427in}{2.706519in}}%
\pgfpathlineto{\pgfqpoint{4.447739in}{4.138674in}}%
\pgfpathlineto{\pgfqpoint{4.450051in}{2.843526in}}%
\pgfpathlineto{\pgfqpoint{4.452364in}{3.503459in}}%
\pgfpathlineto{\pgfqpoint{4.454676in}{3.730051in}}%
\pgfpathlineto{\pgfqpoint{4.456988in}{2.704888in}}%
\pgfpathlineto{\pgfqpoint{4.459301in}{4.139208in}}%
\pgfpathlineto{\pgfqpoint{4.461613in}{2.836057in}}%
\pgfpathlineto{\pgfqpoint{4.463925in}{3.520951in}}%
\pgfpathlineto{\pgfqpoint{4.466238in}{3.707727in}}%
\pgfpathlineto{\pgfqpoint{4.468550in}{2.720084in}}%
\pgfpathlineto{\pgfqpoint{4.470862in}{4.144003in}}%
\pgfpathlineto{\pgfqpoint{4.473175in}{2.804942in}}%
\pgfpathlineto{\pgfqpoint{4.475487in}{3.572946in}}%
\pgfpathlineto{\pgfqpoint{4.477800in}{3.652133in}}%
\pgfpathlineto{\pgfqpoint{4.480112in}{2.755943in}}%
\pgfpathlineto{\pgfqpoint{4.482424in}{4.147636in}}%
\pgfpathlineto{\pgfqpoint{4.484737in}{2.755136in}}%
\pgfpathlineto{\pgfqpoint{4.487049in}{3.657442in}}%
\pgfpathlineto{\pgfqpoint{4.489361in}{3.560541in}}%
\pgfpathlineto{\pgfqpoint{4.491674in}{2.820003in}}%
\pgfpathlineto{\pgfqpoint{4.493986in}{4.139725in}}%
\pgfpathlineto{\pgfqpoint{4.496298in}{2.696339in}}%
\pgfpathlineto{\pgfqpoint{4.498611in}{3.769296in}}%
\pgfpathlineto{\pgfqpoint{4.500923in}{3.430816in}}%
\pgfpathlineto{\pgfqpoint{4.503235in}{2.922042in}}%
\pgfpathlineto{\pgfqpoint{4.505548in}{4.105285in}}%
\pgfpathlineto{\pgfqpoint{4.507860in}{2.644108in}}%
\pgfpathlineto{\pgfqpoint{4.510172in}{3.897779in}}%
\pgfpathlineto{\pgfqpoint{4.512485in}{3.264541in}}%
\pgfpathlineto{\pgfqpoint{4.514797in}{3.071241in}}%
\pgfpathlineto{\pgfqpoint{4.517110in}{4.026237in}}%
\pgfpathlineto{\pgfqpoint{4.519422in}{2.620369in}}%
\pgfpathlineto{\pgfqpoint{4.521734in}{4.023923in}}%
\pgfpathlineto{\pgfqpoint{4.524047in}{3.071241in}}%
\pgfpathlineto{\pgfqpoint{4.526359in}{3.271556in}}%
\pgfpathlineto{\pgfqpoint{4.528671in}{3.885050in}}%
\pgfpathlineto{\pgfqpoint{4.530984in}{2.652046in}}%
\pgfpathlineto{\pgfqpoint{4.533296in}{4.118918in}}%
\pgfpathlineto{\pgfqpoint{4.535608in}{2.872851in}}%
\pgfpathlineto{\pgfqpoint{4.537921in}{3.515361in}}%
\pgfpathlineto{\pgfqpoint{4.540233in}{3.671306in}}%
\pgfpathlineto{\pgfqpoint{4.542545in}{2.766205in}}%
\pgfpathlineto{\pgfqpoint{4.544858in}{4.145579in}}%
\pgfpathlineto{\pgfqpoint{4.547170in}{2.706519in}}%
\pgfpathlineto{\pgfqpoint{4.549483in}{3.776626in}}%
\pgfpathlineto{\pgfqpoint{4.554107in}{2.980472in}}%
\pgfpathlineto{\pgfqpoint{4.556420in}{4.065369in}}%
\pgfpathlineto{\pgfqpoint{4.558732in}{2.622526in}}%
\pgfpathlineto{\pgfqpoint{4.561044in}{4.006763in}}%
\pgfpathlineto{\pgfqpoint{4.563357in}{3.077079in}}%
\pgfpathlineto{\pgfqpoint{4.565669in}{3.289136in}}%
\pgfpathlineto{\pgfqpoint{4.567981in}{3.852711in}}%
\pgfpathlineto{\pgfqpoint{4.570294in}{2.673571in}}%
\pgfpathlineto{\pgfqpoint{4.572606in}{4.138235in}}%
\pgfpathlineto{\pgfqpoint{4.574918in}{2.793982in}}%
\pgfpathlineto{\pgfqpoint{4.577231in}{3.648807in}}%
\pgfpathlineto{\pgfqpoint{4.579543in}{3.515361in}}%
\pgfpathlineto{\pgfqpoint{4.581855in}{2.893204in}}%
\pgfpathlineto{\pgfqpoint{4.584168in}{4.101475in}}%
\pgfpathlineto{\pgfqpoint{4.586480in}{2.631625in}}%
\pgfpathlineto{\pgfqpoint{4.588793in}{3.972211in}}%
\pgfpathlineto{\pgfqpoint{4.591105in}{3.113874in}}%
\pgfpathlineto{\pgfqpoint{4.593417in}{3.266644in}}%
\pgfpathlineto{\pgfqpoint{4.595730in}{3.857181in}}%
\pgfpathlineto{\pgfqpoint{4.598042in}{2.677815in}}%
\pgfpathlineto{\pgfqpoint{4.600354in}{4.142154in}}%
\pgfpathlineto{\pgfqpoint{4.602667in}{2.766205in}}%
\pgfpathlineto{\pgfqpoint{4.604979in}{3.706441in}}%
\pgfpathlineto{\pgfqpoint{4.607291in}{3.436482in}}%
\pgfpathlineto{\pgfqpoint{4.609604in}{2.970875in}}%
\pgfpathlineto{\pgfqpoint{4.611916in}{4.054814in}}%
\pgfpathlineto{\pgfqpoint{4.614228in}{2.620364in}}%
\pgfpathlineto{\pgfqpoint{4.616541in}{4.054814in}}%
\pgfpathlineto{\pgfqpoint{4.618853in}{2.967300in}}%
\pgfpathlineto{\pgfqpoint{4.621166in}{3.447805in}}%
\pgfpathlineto{\pgfqpoint{4.623478in}{3.687016in}}%
\pgfpathlineto{\pgfqpoint{4.625790in}{2.786838in}}%
\pgfpathlineto{\pgfqpoint{4.628103in}{4.134611in}}%
\pgfpathlineto{\pgfqpoint{4.630415in}{2.654729in}}%
\pgfpathlineto{\pgfqpoint{4.632727in}{3.923972in}}%
\pgfpathlineto{\pgfqpoint{4.635040in}{3.156138in}}%
\pgfpathlineto{\pgfqpoint{4.637352in}{3.249144in}}%
\pgfpathlineto{\pgfqpoint{4.639664in}{3.849904in}}%
\pgfpathlineto{\pgfqpoint{4.641977in}{2.692377in}}%
\pgfpathlineto{\pgfqpoint{4.644289in}{4.147109in}}%
\pgfpathlineto{\pgfqpoint{4.646601in}{2.717643in}}%
\pgfpathlineto{\pgfqpoint{4.648914in}{3.808401in}}%
\pgfpathlineto{\pgfqpoint{4.651226in}{3.292658in}}%
\pgfpathlineto{\pgfqpoint{4.653538in}{3.123859in}}%
\pgfpathlineto{\pgfqpoint{4.655851in}{3.938817in}}%
\pgfpathlineto{\pgfqpoint{4.658163in}{2.653065in}}%
\pgfpathlineto{\pgfqpoint{4.660476in}{4.136855in}}%
\pgfpathlineto{\pgfqpoint{4.662788in}{2.766205in}}%
\pgfpathlineto{\pgfqpoint{4.665100in}{3.739508in}}%
\pgfpathlineto{\pgfqpoint{4.669725in}{3.068007in}}%
\pgfpathlineto{\pgfqpoint{4.672037in}{3.972211in}}%
\pgfpathlineto{\pgfqpoint{4.674350in}{2.642384in}}%
\pgfpathlineto{\pgfqpoint{4.676662in}{4.131145in}}%
\pgfpathlineto{\pgfqpoint{4.678974in}{2.778526in}}%
\pgfpathlineto{\pgfqpoint{4.681287in}{3.730051in}}%
\pgfpathlineto{\pgfqpoint{4.685911in}{3.076430in}}%
\pgfpathlineto{\pgfqpoint{4.688224in}{3.959806in}}%
\pgfpathlineto{\pgfqpoint{4.690536in}{2.649858in}}%
\pgfpathlineto{\pgfqpoint{4.692848in}{4.138235in}}%
\pgfpathlineto{\pgfqpoint{4.695161in}{2.749948in}}%
\pgfpathlineto{\pgfqpoint{4.697473in}{3.781485in}}%
\pgfpathlineto{\pgfqpoint{4.699786in}{3.294773in}}%
\pgfpathlineto{\pgfqpoint{4.702098in}{3.150049in}}%
\pgfpathlineto{\pgfqpoint{4.704410in}{3.897779in}}%
\pgfpathlineto{\pgfqpoint{4.706723in}{2.683056in}}%
\pgfpathlineto{\pgfqpoint{4.709035in}{4.147552in}}%
\pgfpathlineto{\pgfqpoint{4.711347in}{2.691777in}}%
\pgfpathlineto{\pgfqpoint{4.713660in}{3.885050in}}%
\pgfpathlineto{\pgfqpoint{4.715972in}{3.159528in}}%
\pgfpathlineto{\pgfqpoint{4.718284in}{3.294773in}}%
\pgfpathlineto{\pgfqpoint{4.720597in}{3.770521in}}%
\pgfpathlineto{\pgfqpoint{4.722909in}{2.766205in}}%
\pgfpathlineto{\pgfqpoint{4.725221in}{4.129492in}}%
\pgfpathlineto{\pgfqpoint{4.727534in}{2.634049in}}%
\pgfpathlineto{\pgfqpoint{4.729846in}{4.016067in}}%
\pgfpathlineto{\pgfqpoint{4.732159in}{2.971473in}}%
\pgfpathlineto{\pgfqpoint{4.734471in}{3.511864in}}%
\pgfpathlineto{\pgfqpoint{4.736783in}{3.559851in}}%
\pgfpathlineto{\pgfqpoint{4.739096in}{2.933425in}}%
\pgfpathlineto{\pgfqpoint{4.741408in}{4.037873in}}%
\pgfpathlineto{\pgfqpoint{4.743720in}{2.628687in}}%
\pgfpathlineto{\pgfqpoint{4.746033in}{4.124762in}}%
\pgfpathlineto{\pgfqpoint{4.748345in}{2.769981in}}%
\pgfpathlineto{\pgfqpoint{4.750657in}{3.779665in}}%
\pgfpathlineto{\pgfqpoint{4.752970in}{3.264541in}}%
\pgfpathlineto{\pgfqpoint{4.755282in}{3.210913in}}%
\pgfpathlineto{\pgfqpoint{4.757594in}{3.823037in}}%
\pgfpathlineto{\pgfqpoint{4.759907in}{2.743689in}}%
\pgfpathlineto{\pgfqpoint{4.762219in}{4.132998in}}%
\pgfpathlineto{\pgfqpoint{4.764531in}{2.633000in}}%
\pgfpathlineto{\pgfqpoint{4.766844in}{4.031559in}}%
\pgfpathlineto{\pgfqpoint{4.769156in}{2.929423in}}%
\pgfpathlineto{\pgfqpoint{4.771469in}{3.583928in}}%
\pgfpathlineto{\pgfqpoint{4.773781in}{3.464763in}}%
\pgfpathlineto{\pgfqpoint{4.776093in}{3.034789in}}%
\pgfpathlineto{\pgfqpoint{4.778406in}{3.953703in}}%
\pgfpathlineto{\pgfqpoint{4.780718in}{2.670239in}}%
\pgfpathlineto{\pgfqpoint{4.783030in}{4.147636in}}%
\pgfpathlineto{\pgfqpoint{4.785343in}{3.384000in}}%
\pgfpathlineto{\pgfqpoint{5.534545in}{3.384000in}}%
\pgfpathlineto{\pgfqpoint{5.534545in}{3.384000in}}%
\pgfusepath{stroke}%
\end{pgfscope}%
\begin{pgfscope}%
\pgfsetrectcap%
\pgfsetmiterjoin%
\pgfsetlinewidth{0.803000pt}%
\definecolor{currentstroke}{rgb}{0.000000,0.000000,0.000000}%
\pgfsetstrokecolor{currentstroke}%
\pgfsetdash{}{0pt}%
\pgfpathmoveto{\pgfqpoint{0.800000in}{2.544000in}}%
\pgfpathlineto{\pgfqpoint{0.800000in}{4.224000in}}%
\pgfusepath{stroke}%
\end{pgfscope}%
\begin{pgfscope}%
\pgfsetrectcap%
\pgfsetmiterjoin%
\pgfsetlinewidth{0.803000pt}%
\definecolor{currentstroke}{rgb}{0.000000,0.000000,0.000000}%
\pgfsetstrokecolor{currentstroke}%
\pgfsetdash{}{0pt}%
\pgfpathmoveto{\pgfqpoint{5.760000in}{2.544000in}}%
\pgfpathlineto{\pgfqpoint{5.760000in}{4.224000in}}%
\pgfusepath{stroke}%
\end{pgfscope}%
\begin{pgfscope}%
\pgfsetrectcap%
\pgfsetmiterjoin%
\pgfsetlinewidth{0.803000pt}%
\definecolor{currentstroke}{rgb}{0.000000,0.000000,0.000000}%
\pgfsetstrokecolor{currentstroke}%
\pgfsetdash{}{0pt}%
\pgfpathmoveto{\pgfqpoint{0.800000in}{2.544000in}}%
\pgfpathlineto{\pgfqpoint{5.760000in}{2.544000in}}%
\pgfusepath{stroke}%
\end{pgfscope}%
\begin{pgfscope}%
\pgfsetrectcap%
\pgfsetmiterjoin%
\pgfsetlinewidth{0.803000pt}%
\definecolor{currentstroke}{rgb}{0.000000,0.000000,0.000000}%
\pgfsetstrokecolor{currentstroke}%
\pgfsetdash{}{0pt}%
\pgfpathmoveto{\pgfqpoint{0.800000in}{4.224000in}}%
\pgfpathlineto{\pgfqpoint{5.760000in}{4.224000in}}%
\pgfusepath{stroke}%
\end{pgfscope}%
\begin{pgfscope}%
\pgfsetbuttcap%
\pgfsetmiterjoin%
\definecolor{currentfill}{rgb}{1.000000,1.000000,1.000000}%
\pgfsetfillcolor{currentfill}%
\pgfsetlinewidth{0.000000pt}%
\definecolor{currentstroke}{rgb}{0.000000,0.000000,0.000000}%
\pgfsetstrokecolor{currentstroke}%
\pgfsetstrokeopacity{0.000000}%
\pgfsetdash{}{0pt}%
\pgfpathmoveto{\pgfqpoint{0.800000in}{0.528000in}}%
\pgfpathlineto{\pgfqpoint{5.760000in}{0.528000in}}%
\pgfpathlineto{\pgfqpoint{5.760000in}{2.208000in}}%
\pgfpathlineto{\pgfqpoint{0.800000in}{2.208000in}}%
\pgfpathclose%
\pgfusepath{fill}%
\end{pgfscope}%
\begin{pgfscope}%
\pgfpathrectangle{\pgfqpoint{0.800000in}{0.528000in}}{\pgfqpoint{4.960000in}{1.680000in}} %
\pgfusepath{clip}%
\pgfsetrectcap%
\pgfsetroundjoin%
\pgfsetlinewidth{0.803000pt}%
\definecolor{currentstroke}{rgb}{0.690196,0.690196,0.690196}%
\pgfsetstrokecolor{currentstroke}%
\pgfsetdash{}{0pt}%
\pgfpathmoveto{\pgfqpoint{0.967616in}{0.528000in}}%
\pgfpathlineto{\pgfqpoint{0.967616in}{2.208000in}}%
\pgfusepath{stroke}%
\end{pgfscope}%
\begin{pgfscope}%
\pgfsetbuttcap%
\pgfsetroundjoin%
\definecolor{currentfill}{rgb}{0.000000,0.000000,0.000000}%
\pgfsetfillcolor{currentfill}%
\pgfsetlinewidth{0.803000pt}%
\definecolor{currentstroke}{rgb}{0.000000,0.000000,0.000000}%
\pgfsetstrokecolor{currentstroke}%
\pgfsetdash{}{0pt}%
\pgfsys@defobject{currentmarker}{\pgfqpoint{0.000000in}{-0.048611in}}{\pgfqpoint{0.000000in}{0.000000in}}{%
\pgfpathmoveto{\pgfqpoint{0.000000in}{0.000000in}}%
\pgfpathlineto{\pgfqpoint{0.000000in}{-0.048611in}}%
\pgfusepath{stroke,fill}%
}%
\begin{pgfscope}%
\pgfsys@transformshift{0.967616in}{0.528000in}%
\pgfsys@useobject{currentmarker}{}%
\end{pgfscope}%
\end{pgfscope}%
\begin{pgfscope}%
\pgftext[x=0.967616in,y=0.430778in,,top]{\sffamily\fontsize{10.000000}{12.000000}\selectfont -100}%
\end{pgfscope}%
\begin{pgfscope}%
\pgfpathrectangle{\pgfqpoint{0.800000in}{0.528000in}}{\pgfqpoint{4.960000in}{1.680000in}} %
\pgfusepath{clip}%
\pgfsetrectcap%
\pgfsetroundjoin%
\pgfsetlinewidth{0.803000pt}%
\definecolor{currentstroke}{rgb}{0.690196,0.690196,0.690196}%
\pgfsetstrokecolor{currentstroke}%
\pgfsetdash{}{0pt}%
\pgfpathmoveto{\pgfqpoint{1.546001in}{0.528000in}}%
\pgfpathlineto{\pgfqpoint{1.546001in}{2.208000in}}%
\pgfusepath{stroke}%
\end{pgfscope}%
\begin{pgfscope}%
\pgfsetbuttcap%
\pgfsetroundjoin%
\definecolor{currentfill}{rgb}{0.000000,0.000000,0.000000}%
\pgfsetfillcolor{currentfill}%
\pgfsetlinewidth{0.803000pt}%
\definecolor{currentstroke}{rgb}{0.000000,0.000000,0.000000}%
\pgfsetstrokecolor{currentstroke}%
\pgfsetdash{}{0pt}%
\pgfsys@defobject{currentmarker}{\pgfqpoint{0.000000in}{-0.048611in}}{\pgfqpoint{0.000000in}{0.000000in}}{%
\pgfpathmoveto{\pgfqpoint{0.000000in}{0.000000in}}%
\pgfpathlineto{\pgfqpoint{0.000000in}{-0.048611in}}%
\pgfusepath{stroke,fill}%
}%
\begin{pgfscope}%
\pgfsys@transformshift{1.546001in}{0.528000in}%
\pgfsys@useobject{currentmarker}{}%
\end{pgfscope}%
\end{pgfscope}%
\begin{pgfscope}%
\pgftext[x=1.546001in,y=0.430778in,,top]{\sffamily\fontsize{10.000000}{12.000000}\selectfont -75}%
\end{pgfscope}%
\begin{pgfscope}%
\pgfpathrectangle{\pgfqpoint{0.800000in}{0.528000in}}{\pgfqpoint{4.960000in}{1.680000in}} %
\pgfusepath{clip}%
\pgfsetrectcap%
\pgfsetroundjoin%
\pgfsetlinewidth{0.803000pt}%
\definecolor{currentstroke}{rgb}{0.690196,0.690196,0.690196}%
\pgfsetstrokecolor{currentstroke}%
\pgfsetdash{}{0pt}%
\pgfpathmoveto{\pgfqpoint{2.124386in}{0.528000in}}%
\pgfpathlineto{\pgfqpoint{2.124386in}{2.208000in}}%
\pgfusepath{stroke}%
\end{pgfscope}%
\begin{pgfscope}%
\pgfsetbuttcap%
\pgfsetroundjoin%
\definecolor{currentfill}{rgb}{0.000000,0.000000,0.000000}%
\pgfsetfillcolor{currentfill}%
\pgfsetlinewidth{0.803000pt}%
\definecolor{currentstroke}{rgb}{0.000000,0.000000,0.000000}%
\pgfsetstrokecolor{currentstroke}%
\pgfsetdash{}{0pt}%
\pgfsys@defobject{currentmarker}{\pgfqpoint{0.000000in}{-0.048611in}}{\pgfqpoint{0.000000in}{0.000000in}}{%
\pgfpathmoveto{\pgfqpoint{0.000000in}{0.000000in}}%
\pgfpathlineto{\pgfqpoint{0.000000in}{-0.048611in}}%
\pgfusepath{stroke,fill}%
}%
\begin{pgfscope}%
\pgfsys@transformshift{2.124386in}{0.528000in}%
\pgfsys@useobject{currentmarker}{}%
\end{pgfscope}%
\end{pgfscope}%
\begin{pgfscope}%
\pgftext[x=2.124386in,y=0.430778in,,top]{\sffamily\fontsize{10.000000}{12.000000}\selectfont -50}%
\end{pgfscope}%
\begin{pgfscope}%
\pgfpathrectangle{\pgfqpoint{0.800000in}{0.528000in}}{\pgfqpoint{4.960000in}{1.680000in}} %
\pgfusepath{clip}%
\pgfsetrectcap%
\pgfsetroundjoin%
\pgfsetlinewidth{0.803000pt}%
\definecolor{currentstroke}{rgb}{0.690196,0.690196,0.690196}%
\pgfsetstrokecolor{currentstroke}%
\pgfsetdash{}{0pt}%
\pgfpathmoveto{\pgfqpoint{2.702771in}{0.528000in}}%
\pgfpathlineto{\pgfqpoint{2.702771in}{2.208000in}}%
\pgfusepath{stroke}%
\end{pgfscope}%
\begin{pgfscope}%
\pgfsetbuttcap%
\pgfsetroundjoin%
\definecolor{currentfill}{rgb}{0.000000,0.000000,0.000000}%
\pgfsetfillcolor{currentfill}%
\pgfsetlinewidth{0.803000pt}%
\definecolor{currentstroke}{rgb}{0.000000,0.000000,0.000000}%
\pgfsetstrokecolor{currentstroke}%
\pgfsetdash{}{0pt}%
\pgfsys@defobject{currentmarker}{\pgfqpoint{0.000000in}{-0.048611in}}{\pgfqpoint{0.000000in}{0.000000in}}{%
\pgfpathmoveto{\pgfqpoint{0.000000in}{0.000000in}}%
\pgfpathlineto{\pgfqpoint{0.000000in}{-0.048611in}}%
\pgfusepath{stroke,fill}%
}%
\begin{pgfscope}%
\pgfsys@transformshift{2.702771in}{0.528000in}%
\pgfsys@useobject{currentmarker}{}%
\end{pgfscope}%
\end{pgfscope}%
\begin{pgfscope}%
\pgftext[x=2.702771in,y=0.430778in,,top]{\sffamily\fontsize{10.000000}{12.000000}\selectfont -25}%
\end{pgfscope}%
\begin{pgfscope}%
\pgfpathrectangle{\pgfqpoint{0.800000in}{0.528000in}}{\pgfqpoint{4.960000in}{1.680000in}} %
\pgfusepath{clip}%
\pgfsetrectcap%
\pgfsetroundjoin%
\pgfsetlinewidth{0.803000pt}%
\definecolor{currentstroke}{rgb}{0.690196,0.690196,0.690196}%
\pgfsetstrokecolor{currentstroke}%
\pgfsetdash{}{0pt}%
\pgfpathmoveto{\pgfqpoint{3.281156in}{0.528000in}}%
\pgfpathlineto{\pgfqpoint{3.281156in}{2.208000in}}%
\pgfusepath{stroke}%
\end{pgfscope}%
\begin{pgfscope}%
\pgfsetbuttcap%
\pgfsetroundjoin%
\definecolor{currentfill}{rgb}{0.000000,0.000000,0.000000}%
\pgfsetfillcolor{currentfill}%
\pgfsetlinewidth{0.803000pt}%
\definecolor{currentstroke}{rgb}{0.000000,0.000000,0.000000}%
\pgfsetstrokecolor{currentstroke}%
\pgfsetdash{}{0pt}%
\pgfsys@defobject{currentmarker}{\pgfqpoint{0.000000in}{-0.048611in}}{\pgfqpoint{0.000000in}{0.000000in}}{%
\pgfpathmoveto{\pgfqpoint{0.000000in}{0.000000in}}%
\pgfpathlineto{\pgfqpoint{0.000000in}{-0.048611in}}%
\pgfusepath{stroke,fill}%
}%
\begin{pgfscope}%
\pgfsys@transformshift{3.281156in}{0.528000in}%
\pgfsys@useobject{currentmarker}{}%
\end{pgfscope}%
\end{pgfscope}%
\begin{pgfscope}%
\pgftext[x=3.281156in,y=0.430778in,,top]{\sffamily\fontsize{10.000000}{12.000000}\selectfont 0}%
\end{pgfscope}%
\begin{pgfscope}%
\pgfpathrectangle{\pgfqpoint{0.800000in}{0.528000in}}{\pgfqpoint{4.960000in}{1.680000in}} %
\pgfusepath{clip}%
\pgfsetrectcap%
\pgfsetroundjoin%
\pgfsetlinewidth{0.803000pt}%
\definecolor{currentstroke}{rgb}{0.690196,0.690196,0.690196}%
\pgfsetstrokecolor{currentstroke}%
\pgfsetdash{}{0pt}%
\pgfpathmoveto{\pgfqpoint{3.859541in}{0.528000in}}%
\pgfpathlineto{\pgfqpoint{3.859541in}{2.208000in}}%
\pgfusepath{stroke}%
\end{pgfscope}%
\begin{pgfscope}%
\pgfsetbuttcap%
\pgfsetroundjoin%
\definecolor{currentfill}{rgb}{0.000000,0.000000,0.000000}%
\pgfsetfillcolor{currentfill}%
\pgfsetlinewidth{0.803000pt}%
\definecolor{currentstroke}{rgb}{0.000000,0.000000,0.000000}%
\pgfsetstrokecolor{currentstroke}%
\pgfsetdash{}{0pt}%
\pgfsys@defobject{currentmarker}{\pgfqpoint{0.000000in}{-0.048611in}}{\pgfqpoint{0.000000in}{0.000000in}}{%
\pgfpathmoveto{\pgfqpoint{0.000000in}{0.000000in}}%
\pgfpathlineto{\pgfqpoint{0.000000in}{-0.048611in}}%
\pgfusepath{stroke,fill}%
}%
\begin{pgfscope}%
\pgfsys@transformshift{3.859541in}{0.528000in}%
\pgfsys@useobject{currentmarker}{}%
\end{pgfscope}%
\end{pgfscope}%
\begin{pgfscope}%
\pgftext[x=3.859541in,y=0.430778in,,top]{\sffamily\fontsize{10.000000}{12.000000}\selectfont 25}%
\end{pgfscope}%
\begin{pgfscope}%
\pgfpathrectangle{\pgfqpoint{0.800000in}{0.528000in}}{\pgfqpoint{4.960000in}{1.680000in}} %
\pgfusepath{clip}%
\pgfsetrectcap%
\pgfsetroundjoin%
\pgfsetlinewidth{0.803000pt}%
\definecolor{currentstroke}{rgb}{0.690196,0.690196,0.690196}%
\pgfsetstrokecolor{currentstroke}%
\pgfsetdash{}{0pt}%
\pgfpathmoveto{\pgfqpoint{4.437926in}{0.528000in}}%
\pgfpathlineto{\pgfqpoint{4.437926in}{2.208000in}}%
\pgfusepath{stroke}%
\end{pgfscope}%
\begin{pgfscope}%
\pgfsetbuttcap%
\pgfsetroundjoin%
\definecolor{currentfill}{rgb}{0.000000,0.000000,0.000000}%
\pgfsetfillcolor{currentfill}%
\pgfsetlinewidth{0.803000pt}%
\definecolor{currentstroke}{rgb}{0.000000,0.000000,0.000000}%
\pgfsetstrokecolor{currentstroke}%
\pgfsetdash{}{0pt}%
\pgfsys@defobject{currentmarker}{\pgfqpoint{0.000000in}{-0.048611in}}{\pgfqpoint{0.000000in}{0.000000in}}{%
\pgfpathmoveto{\pgfqpoint{0.000000in}{0.000000in}}%
\pgfpathlineto{\pgfqpoint{0.000000in}{-0.048611in}}%
\pgfusepath{stroke,fill}%
}%
\begin{pgfscope}%
\pgfsys@transformshift{4.437926in}{0.528000in}%
\pgfsys@useobject{currentmarker}{}%
\end{pgfscope}%
\end{pgfscope}%
\begin{pgfscope}%
\pgftext[x=4.437926in,y=0.430778in,,top]{\sffamily\fontsize{10.000000}{12.000000}\selectfont 50}%
\end{pgfscope}%
\begin{pgfscope}%
\pgfpathrectangle{\pgfqpoint{0.800000in}{0.528000in}}{\pgfqpoint{4.960000in}{1.680000in}} %
\pgfusepath{clip}%
\pgfsetrectcap%
\pgfsetroundjoin%
\pgfsetlinewidth{0.803000pt}%
\definecolor{currentstroke}{rgb}{0.690196,0.690196,0.690196}%
\pgfsetstrokecolor{currentstroke}%
\pgfsetdash{}{0pt}%
\pgfpathmoveto{\pgfqpoint{5.016311in}{0.528000in}}%
\pgfpathlineto{\pgfqpoint{5.016311in}{2.208000in}}%
\pgfusepath{stroke}%
\end{pgfscope}%
\begin{pgfscope}%
\pgfsetbuttcap%
\pgfsetroundjoin%
\definecolor{currentfill}{rgb}{0.000000,0.000000,0.000000}%
\pgfsetfillcolor{currentfill}%
\pgfsetlinewidth{0.803000pt}%
\definecolor{currentstroke}{rgb}{0.000000,0.000000,0.000000}%
\pgfsetstrokecolor{currentstroke}%
\pgfsetdash{}{0pt}%
\pgfsys@defobject{currentmarker}{\pgfqpoint{0.000000in}{-0.048611in}}{\pgfqpoint{0.000000in}{0.000000in}}{%
\pgfpathmoveto{\pgfqpoint{0.000000in}{0.000000in}}%
\pgfpathlineto{\pgfqpoint{0.000000in}{-0.048611in}}%
\pgfusepath{stroke,fill}%
}%
\begin{pgfscope}%
\pgfsys@transformshift{5.016311in}{0.528000in}%
\pgfsys@useobject{currentmarker}{}%
\end{pgfscope}%
\end{pgfscope}%
\begin{pgfscope}%
\pgftext[x=5.016311in,y=0.430778in,,top]{\sffamily\fontsize{10.000000}{12.000000}\selectfont 75}%
\end{pgfscope}%
\begin{pgfscope}%
\pgfpathrectangle{\pgfqpoint{0.800000in}{0.528000in}}{\pgfqpoint{4.960000in}{1.680000in}} %
\pgfusepath{clip}%
\pgfsetrectcap%
\pgfsetroundjoin%
\pgfsetlinewidth{0.803000pt}%
\definecolor{currentstroke}{rgb}{0.690196,0.690196,0.690196}%
\pgfsetstrokecolor{currentstroke}%
\pgfsetdash{}{0pt}%
\pgfpathmoveto{\pgfqpoint{5.594696in}{0.528000in}}%
\pgfpathlineto{\pgfqpoint{5.594696in}{2.208000in}}%
\pgfusepath{stroke}%
\end{pgfscope}%
\begin{pgfscope}%
\pgfsetbuttcap%
\pgfsetroundjoin%
\definecolor{currentfill}{rgb}{0.000000,0.000000,0.000000}%
\pgfsetfillcolor{currentfill}%
\pgfsetlinewidth{0.803000pt}%
\definecolor{currentstroke}{rgb}{0.000000,0.000000,0.000000}%
\pgfsetstrokecolor{currentstroke}%
\pgfsetdash{}{0pt}%
\pgfsys@defobject{currentmarker}{\pgfqpoint{0.000000in}{-0.048611in}}{\pgfqpoint{0.000000in}{0.000000in}}{%
\pgfpathmoveto{\pgfqpoint{0.000000in}{0.000000in}}%
\pgfpathlineto{\pgfqpoint{0.000000in}{-0.048611in}}%
\pgfusepath{stroke,fill}%
}%
\begin{pgfscope}%
\pgfsys@transformshift{5.594696in}{0.528000in}%
\pgfsys@useobject{currentmarker}{}%
\end{pgfscope}%
\end{pgfscope}%
\begin{pgfscope}%
\pgftext[x=5.594696in,y=0.430778in,,top]{\sffamily\fontsize{10.000000}{12.000000}\selectfont 100}%
\end{pgfscope}%
\begin{pgfscope}%
\pgftext[x=3.280000in,y=0.240809in,,top]{\sffamily\fontsize{10.000000}{12.000000}\selectfont Frequency [MHz]}%
\end{pgfscope}%
\begin{pgfscope}%
\pgfpathrectangle{\pgfqpoint{0.800000in}{0.528000in}}{\pgfqpoint{4.960000in}{1.680000in}} %
\pgfusepath{clip}%
\pgfsetrectcap%
\pgfsetroundjoin%
\pgfsetlinewidth{0.803000pt}%
\definecolor{currentstroke}{rgb}{0.690196,0.690196,0.690196}%
\pgfsetstrokecolor{currentstroke}%
\pgfsetdash{}{0pt}%
\pgfpathmoveto{\pgfqpoint{0.800000in}{0.591271in}}%
\pgfpathlineto{\pgfqpoint{5.760000in}{0.591271in}}%
\pgfusepath{stroke}%
\end{pgfscope}%
\begin{pgfscope}%
\pgfsetbuttcap%
\pgfsetroundjoin%
\definecolor{currentfill}{rgb}{0.000000,0.000000,0.000000}%
\pgfsetfillcolor{currentfill}%
\pgfsetlinewidth{0.803000pt}%
\definecolor{currentstroke}{rgb}{0.000000,0.000000,0.000000}%
\pgfsetstrokecolor{currentstroke}%
\pgfsetdash{}{0pt}%
\pgfsys@defobject{currentmarker}{\pgfqpoint{-0.048611in}{0.000000in}}{\pgfqpoint{0.000000in}{0.000000in}}{%
\pgfpathmoveto{\pgfqpoint{0.000000in}{0.000000in}}%
\pgfpathlineto{\pgfqpoint{-0.048611in}{0.000000in}}%
\pgfusepath{stroke,fill}%
}%
\begin{pgfscope}%
\pgfsys@transformshift{0.800000in}{0.591271in}%
\pgfsys@useobject{currentmarker}{}%
\end{pgfscope}%
\end{pgfscope}%
\begin{pgfscope}%
\pgftext[x=0.481898in,y=0.538510in,left,base]{\sffamily\fontsize{10.000000}{12.000000}\selectfont 0.0}%
\end{pgfscope}%
\begin{pgfscope}%
\pgfpathrectangle{\pgfqpoint{0.800000in}{0.528000in}}{\pgfqpoint{4.960000in}{1.680000in}} %
\pgfusepath{clip}%
\pgfsetrectcap%
\pgfsetroundjoin%
\pgfsetlinewidth{0.803000pt}%
\definecolor{currentstroke}{rgb}{0.690196,0.690196,0.690196}%
\pgfsetstrokecolor{currentstroke}%
\pgfsetdash{}{0pt}%
\pgfpathmoveto{\pgfqpoint{0.800000in}{0.899344in}}%
\pgfpathlineto{\pgfqpoint{5.760000in}{0.899344in}}%
\pgfusepath{stroke}%
\end{pgfscope}%
\begin{pgfscope}%
\pgfsetbuttcap%
\pgfsetroundjoin%
\definecolor{currentfill}{rgb}{0.000000,0.000000,0.000000}%
\pgfsetfillcolor{currentfill}%
\pgfsetlinewidth{0.803000pt}%
\definecolor{currentstroke}{rgb}{0.000000,0.000000,0.000000}%
\pgfsetstrokecolor{currentstroke}%
\pgfsetdash{}{0pt}%
\pgfsys@defobject{currentmarker}{\pgfqpoint{-0.048611in}{0.000000in}}{\pgfqpoint{0.000000in}{0.000000in}}{%
\pgfpathmoveto{\pgfqpoint{0.000000in}{0.000000in}}%
\pgfpathlineto{\pgfqpoint{-0.048611in}{0.000000in}}%
\pgfusepath{stroke,fill}%
}%
\begin{pgfscope}%
\pgfsys@transformshift{0.800000in}{0.899344in}%
\pgfsys@useobject{currentmarker}{}%
\end{pgfscope}%
\end{pgfscope}%
\begin{pgfscope}%
\pgftext[x=0.481898in,y=0.846583in,left,base]{\sffamily\fontsize{10.000000}{12.000000}\selectfont 0.2}%
\end{pgfscope}%
\begin{pgfscope}%
\pgfpathrectangle{\pgfqpoint{0.800000in}{0.528000in}}{\pgfqpoint{4.960000in}{1.680000in}} %
\pgfusepath{clip}%
\pgfsetrectcap%
\pgfsetroundjoin%
\pgfsetlinewidth{0.803000pt}%
\definecolor{currentstroke}{rgb}{0.690196,0.690196,0.690196}%
\pgfsetstrokecolor{currentstroke}%
\pgfsetdash{}{0pt}%
\pgfpathmoveto{\pgfqpoint{0.800000in}{1.207417in}}%
\pgfpathlineto{\pgfqpoint{5.760000in}{1.207417in}}%
\pgfusepath{stroke}%
\end{pgfscope}%
\begin{pgfscope}%
\pgfsetbuttcap%
\pgfsetroundjoin%
\definecolor{currentfill}{rgb}{0.000000,0.000000,0.000000}%
\pgfsetfillcolor{currentfill}%
\pgfsetlinewidth{0.803000pt}%
\definecolor{currentstroke}{rgb}{0.000000,0.000000,0.000000}%
\pgfsetstrokecolor{currentstroke}%
\pgfsetdash{}{0pt}%
\pgfsys@defobject{currentmarker}{\pgfqpoint{-0.048611in}{0.000000in}}{\pgfqpoint{0.000000in}{0.000000in}}{%
\pgfpathmoveto{\pgfqpoint{0.000000in}{0.000000in}}%
\pgfpathlineto{\pgfqpoint{-0.048611in}{0.000000in}}%
\pgfusepath{stroke,fill}%
}%
\begin{pgfscope}%
\pgfsys@transformshift{0.800000in}{1.207417in}%
\pgfsys@useobject{currentmarker}{}%
\end{pgfscope}%
\end{pgfscope}%
\begin{pgfscope}%
\pgftext[x=0.481898in,y=1.154656in,left,base]{\sffamily\fontsize{10.000000}{12.000000}\selectfont 0.4}%
\end{pgfscope}%
\begin{pgfscope}%
\pgfpathrectangle{\pgfqpoint{0.800000in}{0.528000in}}{\pgfqpoint{4.960000in}{1.680000in}} %
\pgfusepath{clip}%
\pgfsetrectcap%
\pgfsetroundjoin%
\pgfsetlinewidth{0.803000pt}%
\definecolor{currentstroke}{rgb}{0.690196,0.690196,0.690196}%
\pgfsetstrokecolor{currentstroke}%
\pgfsetdash{}{0pt}%
\pgfpathmoveto{\pgfqpoint{0.800000in}{1.515490in}}%
\pgfpathlineto{\pgfqpoint{5.760000in}{1.515490in}}%
\pgfusepath{stroke}%
\end{pgfscope}%
\begin{pgfscope}%
\pgfsetbuttcap%
\pgfsetroundjoin%
\definecolor{currentfill}{rgb}{0.000000,0.000000,0.000000}%
\pgfsetfillcolor{currentfill}%
\pgfsetlinewidth{0.803000pt}%
\definecolor{currentstroke}{rgb}{0.000000,0.000000,0.000000}%
\pgfsetstrokecolor{currentstroke}%
\pgfsetdash{}{0pt}%
\pgfsys@defobject{currentmarker}{\pgfqpoint{-0.048611in}{0.000000in}}{\pgfqpoint{0.000000in}{0.000000in}}{%
\pgfpathmoveto{\pgfqpoint{0.000000in}{0.000000in}}%
\pgfpathlineto{\pgfqpoint{-0.048611in}{0.000000in}}%
\pgfusepath{stroke,fill}%
}%
\begin{pgfscope}%
\pgfsys@transformshift{0.800000in}{1.515490in}%
\pgfsys@useobject{currentmarker}{}%
\end{pgfscope}%
\end{pgfscope}%
\begin{pgfscope}%
\pgftext[x=0.481898in,y=1.462729in,left,base]{\sffamily\fontsize{10.000000}{12.000000}\selectfont 0.6}%
\end{pgfscope}%
\begin{pgfscope}%
\pgfpathrectangle{\pgfqpoint{0.800000in}{0.528000in}}{\pgfqpoint{4.960000in}{1.680000in}} %
\pgfusepath{clip}%
\pgfsetrectcap%
\pgfsetroundjoin%
\pgfsetlinewidth{0.803000pt}%
\definecolor{currentstroke}{rgb}{0.690196,0.690196,0.690196}%
\pgfsetstrokecolor{currentstroke}%
\pgfsetdash{}{0pt}%
\pgfpathmoveto{\pgfqpoint{0.800000in}{1.823563in}}%
\pgfpathlineto{\pgfqpoint{5.760000in}{1.823563in}}%
\pgfusepath{stroke}%
\end{pgfscope}%
\begin{pgfscope}%
\pgfsetbuttcap%
\pgfsetroundjoin%
\definecolor{currentfill}{rgb}{0.000000,0.000000,0.000000}%
\pgfsetfillcolor{currentfill}%
\pgfsetlinewidth{0.803000pt}%
\definecolor{currentstroke}{rgb}{0.000000,0.000000,0.000000}%
\pgfsetstrokecolor{currentstroke}%
\pgfsetdash{}{0pt}%
\pgfsys@defobject{currentmarker}{\pgfqpoint{-0.048611in}{0.000000in}}{\pgfqpoint{0.000000in}{0.000000in}}{%
\pgfpathmoveto{\pgfqpoint{0.000000in}{0.000000in}}%
\pgfpathlineto{\pgfqpoint{-0.048611in}{0.000000in}}%
\pgfusepath{stroke,fill}%
}%
\begin{pgfscope}%
\pgfsys@transformshift{0.800000in}{1.823563in}%
\pgfsys@useobject{currentmarker}{}%
\end{pgfscope}%
\end{pgfscope}%
\begin{pgfscope}%
\pgftext[x=0.481898in,y=1.770802in,left,base]{\sffamily\fontsize{10.000000}{12.000000}\selectfont 0.8}%
\end{pgfscope}%
\begin{pgfscope}%
\pgfpathrectangle{\pgfqpoint{0.800000in}{0.528000in}}{\pgfqpoint{4.960000in}{1.680000in}} %
\pgfusepath{clip}%
\pgfsetrectcap%
\pgfsetroundjoin%
\pgfsetlinewidth{0.803000pt}%
\definecolor{currentstroke}{rgb}{0.690196,0.690196,0.690196}%
\pgfsetstrokecolor{currentstroke}%
\pgfsetdash{}{0pt}%
\pgfpathmoveto{\pgfqpoint{0.800000in}{2.131636in}}%
\pgfpathlineto{\pgfqpoint{5.760000in}{2.131636in}}%
\pgfusepath{stroke}%
\end{pgfscope}%
\begin{pgfscope}%
\pgfsetbuttcap%
\pgfsetroundjoin%
\definecolor{currentfill}{rgb}{0.000000,0.000000,0.000000}%
\pgfsetfillcolor{currentfill}%
\pgfsetlinewidth{0.803000pt}%
\definecolor{currentstroke}{rgb}{0.000000,0.000000,0.000000}%
\pgfsetstrokecolor{currentstroke}%
\pgfsetdash{}{0pt}%
\pgfsys@defobject{currentmarker}{\pgfqpoint{-0.048611in}{0.000000in}}{\pgfqpoint{0.000000in}{0.000000in}}{%
\pgfpathmoveto{\pgfqpoint{0.000000in}{0.000000in}}%
\pgfpathlineto{\pgfqpoint{-0.048611in}{0.000000in}}%
\pgfusepath{stroke,fill}%
}%
\begin{pgfscope}%
\pgfsys@transformshift{0.800000in}{2.131636in}%
\pgfsys@useobject{currentmarker}{}%
\end{pgfscope}%
\end{pgfscope}%
\begin{pgfscope}%
\pgftext[x=0.481898in,y=2.078875in,left,base]{\sffamily\fontsize{10.000000}{12.000000}\selectfont 1.0}%
\end{pgfscope}%
\begin{pgfscope}%
\pgftext[x=0.426343in,y=1.368000in,,bottom,rotate=90.000000]{\sffamily\fontsize{10.000000}{12.000000}\selectfont Magnitude of the Fourier transform.}%
\end{pgfscope}%
\begin{pgfscope}%
\pgfpathrectangle{\pgfqpoint{0.800000in}{0.528000in}}{\pgfqpoint{4.960000in}{1.680000in}} %
\pgfusepath{clip}%
\pgfsetrectcap%
\pgfsetroundjoin%
\pgfsetlinewidth{1.505625pt}%
\definecolor{currentstroke}{rgb}{0.121569,0.466667,0.705882}%
\pgfsetstrokecolor{currentstroke}%
\pgfsetdash{}{0pt}%
\pgfpathmoveto{\pgfqpoint{1.025455in}{0.613694in}}%
\pgfpathlineto{\pgfqpoint{1.027767in}{0.636942in}}%
\pgfpathlineto{\pgfqpoint{1.030079in}{0.616422in}}%
\pgfpathlineto{\pgfqpoint{1.032392in}{0.611130in}}%
\pgfpathlineto{\pgfqpoint{1.034704in}{0.637193in}}%
\pgfpathlineto{\pgfqpoint{1.039329in}{0.608792in}}%
\pgfpathlineto{\pgfqpoint{1.041641in}{0.637697in}}%
\pgfpathlineto{\pgfqpoint{1.046266in}{0.606786in}}%
\pgfpathlineto{\pgfqpoint{1.048578in}{0.638453in}}%
\pgfpathlineto{\pgfqpoint{1.050890in}{0.625301in}}%
\pgfpathlineto{\pgfqpoint{1.053203in}{0.605272in}}%
\pgfpathlineto{\pgfqpoint{1.055515in}{0.639461in}}%
\pgfpathlineto{\pgfqpoint{1.057828in}{0.628464in}}%
\pgfpathlineto{\pgfqpoint{1.060140in}{0.604453in}}%
\pgfpathlineto{\pgfqpoint{1.062452in}{0.640724in}}%
\pgfpathlineto{\pgfqpoint{1.064765in}{0.631732in}}%
\pgfpathlineto{\pgfqpoint{1.067077in}{0.604507in}}%
\pgfpathlineto{\pgfqpoint{1.069389in}{0.642244in}}%
\pgfpathlineto{\pgfqpoint{1.071702in}{0.635115in}}%
\pgfpathlineto{\pgfqpoint{1.074014in}{0.605487in}}%
\pgfpathlineto{\pgfqpoint{1.076326in}{0.644025in}}%
\pgfpathlineto{\pgfqpoint{1.078639in}{0.638624in}}%
\pgfpathlineto{\pgfqpoint{1.080951in}{0.607295in}}%
\pgfpathlineto{\pgfqpoint{1.083263in}{0.646073in}}%
\pgfpathlineto{\pgfqpoint{1.085576in}{0.642274in}}%
\pgfpathlineto{\pgfqpoint{1.087888in}{0.609771in}}%
\pgfpathlineto{\pgfqpoint{1.090200in}{0.648398in}}%
\pgfpathlineto{\pgfqpoint{1.092513in}{0.646085in}}%
\pgfpathlineto{\pgfqpoint{1.094825in}{0.612771in}}%
\pgfpathlineto{\pgfqpoint{1.097138in}{0.651011in}}%
\pgfpathlineto{\pgfqpoint{1.099450in}{0.650076in}}%
\pgfpathlineto{\pgfqpoint{1.101762in}{0.616201in}}%
\pgfpathlineto{\pgfqpoint{1.104075in}{0.653927in}}%
\pgfpathlineto{\pgfqpoint{1.106387in}{0.654273in}}%
\pgfpathlineto{\pgfqpoint{1.108699in}{0.620009in}}%
\pgfpathlineto{\pgfqpoint{1.111012in}{0.657167in}}%
\pgfpathlineto{\pgfqpoint{1.113324in}{0.658704in}}%
\pgfpathlineto{\pgfqpoint{1.115636in}{0.624174in}}%
\pgfpathlineto{\pgfqpoint{1.117949in}{0.660755in}}%
\pgfpathlineto{\pgfqpoint{1.120261in}{0.663403in}}%
\pgfpathlineto{\pgfqpoint{1.122573in}{0.628699in}}%
\pgfpathlineto{\pgfqpoint{1.124886in}{0.664722in}}%
\pgfpathlineto{\pgfqpoint{1.127198in}{0.668406in}}%
\pgfpathlineto{\pgfqpoint{1.129510in}{0.633602in}}%
\pgfpathlineto{\pgfqpoint{1.131823in}{0.669106in}}%
\pgfpathlineto{\pgfqpoint{1.134135in}{0.673759in}}%
\pgfpathlineto{\pgfqpoint{1.136448in}{0.638916in}}%
\pgfpathlineto{\pgfqpoint{1.138760in}{0.673953in}}%
\pgfpathlineto{\pgfqpoint{1.141072in}{0.679514in}}%
\pgfpathlineto{\pgfqpoint{1.143385in}{0.644685in}}%
\pgfpathlineto{\pgfqpoint{1.145697in}{0.679316in}}%
\pgfpathlineto{\pgfqpoint{1.148009in}{0.685732in}}%
\pgfpathlineto{\pgfqpoint{1.150322in}{0.650967in}}%
\pgfpathlineto{\pgfqpoint{1.152634in}{0.685265in}}%
\pgfpathlineto{\pgfqpoint{1.154946in}{0.692484in}}%
\pgfpathlineto{\pgfqpoint{1.157259in}{0.657835in}}%
\pgfpathlineto{\pgfqpoint{1.159571in}{0.691878in}}%
\pgfpathlineto{\pgfqpoint{1.161883in}{0.699856in}}%
\pgfpathlineto{\pgfqpoint{1.164196in}{0.665375in}}%
\pgfpathlineto{\pgfqpoint{1.166508in}{0.699255in}}%
\pgfpathlineto{\pgfqpoint{1.168821in}{0.707950in}}%
\pgfpathlineto{\pgfqpoint{1.171133in}{0.673696in}}%
\pgfpathlineto{\pgfqpoint{1.173445in}{0.707513in}}%
\pgfpathlineto{\pgfqpoint{1.175758in}{0.716889in}}%
\pgfpathlineto{\pgfqpoint{1.178070in}{0.682926in}}%
\pgfpathlineto{\pgfqpoint{1.180382in}{0.716795in}}%
\pgfpathlineto{\pgfqpoint{1.182695in}{0.726818in}}%
\pgfpathlineto{\pgfqpoint{1.185007in}{0.693223in}}%
\pgfpathlineto{\pgfqpoint{1.187319in}{0.727275in}}%
\pgfpathlineto{\pgfqpoint{1.189632in}{0.737914in}}%
\pgfpathlineto{\pgfqpoint{1.191944in}{0.704778in}}%
\pgfpathlineto{\pgfqpoint{1.194256in}{0.739164in}}%
\pgfpathlineto{\pgfqpoint{1.196569in}{0.750393in}}%
\pgfpathlineto{\pgfqpoint{1.198881in}{0.717822in}}%
\pgfpathlineto{\pgfqpoint{1.201193in}{0.752720in}}%
\pgfpathlineto{\pgfqpoint{1.203506in}{0.764511in}}%
\pgfpathlineto{\pgfqpoint{1.205818in}{0.732640in}}%
\pgfpathlineto{\pgfqpoint{1.208131in}{0.768253in}}%
\pgfpathlineto{\pgfqpoint{1.210443in}{0.780583in}}%
\pgfpathlineto{\pgfqpoint{1.212755in}{0.749573in}}%
\pgfpathlineto{\pgfqpoint{1.215068in}{0.786142in}}%
\pgfpathlineto{\pgfqpoint{1.217380in}{0.798987in}}%
\pgfpathlineto{\pgfqpoint{1.219692in}{0.769037in}}%
\pgfpathlineto{\pgfqpoint{1.222005in}{0.806840in}}%
\pgfpathlineto{\pgfqpoint{1.224317in}{0.820177in}}%
\pgfpathlineto{\pgfqpoint{1.226629in}{0.791535in}}%
\pgfpathlineto{\pgfqpoint{1.228942in}{0.830896in}}%
\pgfpathlineto{\pgfqpoint{1.231254in}{0.844697in}}%
\pgfpathlineto{\pgfqpoint{1.233566in}{0.817666in}}%
\pgfpathlineto{\pgfqpoint{1.235879in}{0.858953in}}%
\pgfpathlineto{\pgfqpoint{1.238191in}{0.873187in}}%
\pgfpathlineto{\pgfqpoint{1.240503in}{0.848141in}}%
\pgfpathlineto{\pgfqpoint{1.242816in}{0.891771in}}%
\pgfpathlineto{\pgfqpoint{1.245128in}{0.906394in}}%
\pgfpathlineto{\pgfqpoint{1.247441in}{0.883791in}}%
\pgfpathlineto{\pgfqpoint{1.249753in}{0.930215in}}%
\pgfpathlineto{\pgfqpoint{1.252065in}{0.945169in}}%
\pgfpathlineto{\pgfqpoint{1.254378in}{0.925561in}}%
\pgfpathlineto{\pgfqpoint{1.256690in}{0.975256in}}%
\pgfpathlineto{\pgfqpoint{1.259002in}{0.990456in}}%
\pgfpathlineto{\pgfqpoint{1.261315in}{0.974504in}}%
\pgfpathlineto{\pgfqpoint{1.263627in}{1.027935in}}%
\pgfpathlineto{\pgfqpoint{1.265939in}{1.043258in}}%
\pgfpathlineto{\pgfqpoint{1.268252in}{1.031734in}}%
\pgfpathlineto{\pgfqpoint{1.270564in}{1.089312in}}%
\pgfpathlineto{\pgfqpoint{1.272876in}{1.104581in}}%
\pgfpathlineto{\pgfqpoint{1.275189in}{1.098367in}}%
\pgfpathlineto{\pgfqpoint{1.277501in}{1.160374in}}%
\pgfpathlineto{\pgfqpoint{1.279814in}{1.175338in}}%
\pgfpathlineto{\pgfqpoint{1.282126in}{1.175400in}}%
\pgfpathlineto{\pgfqpoint{1.284438in}{1.241887in}}%
\pgfpathlineto{\pgfqpoint{1.286751in}{1.256199in}}%
\pgfpathlineto{\pgfqpoint{1.289063in}{1.263540in}}%
\pgfpathlineto{\pgfqpoint{1.291375in}{1.334196in}}%
\pgfpathlineto{\pgfqpoint{1.296000in}{1.362966in}}%
\pgfpathlineto{\pgfqpoint{1.298312in}{1.436947in}}%
\pgfpathlineto{\pgfqpoint{1.300625in}{1.448425in}}%
\pgfpathlineto{\pgfqpoint{1.302937in}{1.473003in}}%
\pgfpathlineto{\pgfqpoint{1.305249in}{1.548743in}}%
\pgfpathlineto{\pgfqpoint{1.307562in}{1.557767in}}%
\pgfpathlineto{\pgfqpoint{1.309874in}{1.591730in}}%
\pgfpathlineto{\pgfqpoint{1.312186in}{1.666748in}}%
\pgfpathlineto{\pgfqpoint{1.314499in}{1.672461in}}%
\pgfpathlineto{\pgfqpoint{1.316811in}{1.715543in}}%
\pgfpathlineto{\pgfqpoint{1.319124in}{1.786281in}}%
\pgfpathlineto{\pgfqpoint{1.321436in}{1.787768in}}%
\pgfpathlineto{\pgfqpoint{1.326061in}{1.900493in}}%
\pgfpathlineto{\pgfqpoint{1.328373in}{1.896897in}}%
\pgfpathlineto{\pgfqpoint{1.332998in}{2.000271in}}%
\pgfpathlineto{\pgfqpoint{1.335310in}{1.990990in}}%
\pgfpathlineto{\pgfqpoint{1.337622in}{2.048375in}}%
\pgfpathlineto{\pgfqpoint{1.339935in}{2.074582in}}%
\pgfpathlineto{\pgfqpoint{1.342247in}{2.059560in}}%
\pgfpathlineto{\pgfqpoint{1.344559in}{2.112047in}}%
\pgfpathlineto{\pgfqpoint{1.346872in}{2.111565in}}%
\pgfpathlineto{\pgfqpoint{1.349184in}{2.091682in}}%
\pgfpathlineto{\pgfqpoint{1.351497in}{2.131636in}}%
\pgfpathlineto{\pgfqpoint{1.356121in}{2.078306in}}%
\pgfpathlineto{\pgfqpoint{1.358434in}{2.097300in}}%
\pgfpathlineto{\pgfqpoint{1.360746in}{2.037133in}}%
\pgfpathlineto{\pgfqpoint{1.363058in}{2.016175in}}%
\pgfpathlineto{\pgfqpoint{1.365371in}{2.006993in}}%
\pgfpathlineto{\pgfqpoint{1.367683in}{1.927229in}}%
\pgfpathlineto{\pgfqpoint{1.369995in}{1.913791in}}%
\pgfpathlineto{\pgfqpoint{1.372308in}{1.874597in}}%
\pgfpathlineto{\pgfqpoint{1.374620in}{1.798256in}}%
\pgfpathlineto{\pgfqpoint{1.376932in}{1.798955in}}%
\pgfpathlineto{\pgfqpoint{1.381557in}{1.704205in}}%
\pgfpathlineto{\pgfqpoint{1.383869in}{1.721373in}}%
\pgfpathlineto{\pgfqpoint{1.386182in}{1.673935in}}%
\pgfpathlineto{\pgfqpoint{1.390807in}{1.729752in}}%
\pgfpathlineto{\pgfqpoint{1.393119in}{1.726219in}}%
\pgfpathlineto{\pgfqpoint{1.395431in}{1.806215in}}%
\pgfpathlineto{\pgfqpoint{1.397744in}{1.821401in}}%
\pgfpathlineto{\pgfqpoint{1.400056in}{1.864061in}}%
\pgfpathlineto{\pgfqpoint{1.402368in}{1.937690in}}%
\pgfpathlineto{\pgfqpoint{1.404681in}{1.933469in}}%
\pgfpathlineto{\pgfqpoint{1.406993in}{1.993169in}}%
\pgfpathlineto{\pgfqpoint{1.409305in}{2.013872in}}%
\pgfpathlineto{\pgfqpoint{1.411618in}{1.992478in}}%
\pgfpathlineto{\pgfqpoint{1.413930in}{2.031017in}}%
\pgfpathlineto{\pgfqpoint{1.418555in}{1.958550in}}%
\pgfpathlineto{\pgfqpoint{1.420867in}{1.947969in}}%
\pgfpathlineto{\pgfqpoint{1.423179in}{1.861833in}}%
\pgfpathlineto{\pgfqpoint{1.425492in}{1.852529in}}%
\pgfpathlineto{\pgfqpoint{1.430117in}{1.747894in}}%
\pgfpathlineto{\pgfqpoint{1.432429in}{1.766076in}}%
\pgfpathlineto{\pgfqpoint{1.434741in}{1.721544in}}%
\pgfpathlineto{\pgfqpoint{1.439366in}{1.795665in}}%
\pgfpathlineto{\pgfqpoint{1.441678in}{1.817960in}}%
\pgfpathlineto{\pgfqpoint{1.443991in}{1.904141in}}%
\pgfpathlineto{\pgfqpoint{1.446303in}{1.905554in}}%
\pgfpathlineto{\pgfqpoint{1.448615in}{1.966229in}}%
\pgfpathlineto{\pgfqpoint{1.450928in}{1.988010in}}%
\pgfpathlineto{\pgfqpoint{1.453240in}{1.963467in}}%
\pgfpathlineto{\pgfqpoint{1.455552in}{1.992524in}}%
\pgfpathlineto{\pgfqpoint{1.457865in}{1.924106in}}%
\pgfpathlineto{\pgfqpoint{1.460177in}{1.901540in}}%
\pgfpathlineto{\pgfqpoint{1.462490in}{1.862419in}}%
\pgfpathlineto{\pgfqpoint{1.464802in}{1.786343in}}%
\pgfpathlineto{\pgfqpoint{1.467114in}{1.797419in}}%
\pgfpathlineto{\pgfqpoint{1.469427in}{1.743440in}}%
\pgfpathlineto{\pgfqpoint{1.474051in}{1.807577in}}%
\pgfpathlineto{\pgfqpoint{1.476364in}{1.830163in}}%
\pgfpathlineto{\pgfqpoint{1.478676in}{1.917112in}}%
\pgfpathlineto{\pgfqpoint{1.480988in}{1.914135in}}%
\pgfpathlineto{\pgfqpoint{1.483301in}{1.974378in}}%
\pgfpathlineto{\pgfqpoint{1.485613in}{1.968698in}}%
\pgfpathlineto{\pgfqpoint{1.487925in}{1.938982in}}%
\pgfpathlineto{\pgfqpoint{1.490238in}{1.939773in}}%
\pgfpathlineto{\pgfqpoint{1.492550in}{1.848966in}}%
\pgfpathlineto{\pgfqpoint{1.494862in}{1.841862in}}%
\pgfpathlineto{\pgfqpoint{1.497175in}{1.780464in}}%
\pgfpathlineto{\pgfqpoint{1.499487in}{1.770499in}}%
\pgfpathlineto{\pgfqpoint{1.501800in}{1.800642in}}%
\pgfpathlineto{\pgfqpoint{1.504112in}{1.801337in}}%
\pgfpathlineto{\pgfqpoint{1.506424in}{1.893682in}}%
\pgfpathlineto{\pgfqpoint{1.508737in}{1.897206in}}%
\pgfpathlineto{\pgfqpoint{1.511049in}{1.958936in}}%
\pgfpathlineto{\pgfqpoint{1.513361in}{1.960099in}}%
\pgfpathlineto{\pgfqpoint{1.515674in}{1.929011in}}%
\pgfpathlineto{\pgfqpoint{1.517986in}{1.926114in}}%
\pgfpathlineto{\pgfqpoint{1.520298in}{1.832887in}}%
\pgfpathlineto{\pgfqpoint{1.522611in}{1.831727in}}%
\pgfpathlineto{\pgfqpoint{1.524923in}{1.770901in}}%
\pgfpathlineto{\pgfqpoint{1.529548in}{1.824421in}}%
\pgfpathlineto{\pgfqpoint{1.531860in}{1.853215in}}%
\pgfpathlineto{\pgfqpoint{1.534172in}{1.935530in}}%
\pgfpathlineto{\pgfqpoint{1.536485in}{1.921426in}}%
\pgfpathlineto{\pgfqpoint{1.538797in}{1.970027in}}%
\pgfpathlineto{\pgfqpoint{1.541110in}{1.911544in}}%
\pgfpathlineto{\pgfqpoint{1.543422in}{1.885253in}}%
\pgfpathlineto{\pgfqpoint{1.548047in}{1.782452in}}%
\pgfpathlineto{\pgfqpoint{1.550359in}{1.810423in}}%
\pgfpathlineto{\pgfqpoint{1.552671in}{1.795102in}}%
\pgfpathlineto{\pgfqpoint{1.554984in}{1.889362in}}%
\pgfpathlineto{\pgfqpoint{1.557296in}{1.893752in}}%
\pgfpathlineto{\pgfqpoint{1.559608in}{1.955430in}}%
\pgfpathlineto{\pgfqpoint{1.561921in}{1.936752in}}%
\pgfpathlineto{\pgfqpoint{1.566545in}{1.873709in}}%
\pgfpathlineto{\pgfqpoint{1.568858in}{1.794579in}}%
\pgfpathlineto{\pgfqpoint{1.571170in}{1.816407in}}%
\pgfpathlineto{\pgfqpoint{1.573483in}{1.785827in}}%
\pgfpathlineto{\pgfqpoint{1.575795in}{1.876241in}}%
\pgfpathlineto{\pgfqpoint{1.578107in}{1.885740in}}%
\pgfpathlineto{\pgfqpoint{1.580420in}{1.948199in}}%
\pgfpathlineto{\pgfqpoint{1.582732in}{1.933661in}}%
\pgfpathlineto{\pgfqpoint{1.587357in}{1.866636in}}%
\pgfpathlineto{\pgfqpoint{1.589669in}{1.793054in}}%
\pgfpathlineto{\pgfqpoint{1.591981in}{1.819595in}}%
\pgfpathlineto{\pgfqpoint{1.594294in}{1.801183in}}%
\pgfpathlineto{\pgfqpoint{1.596606in}{1.898295in}}%
\pgfpathlineto{\pgfqpoint{1.598918in}{1.897683in}}%
\pgfpathlineto{\pgfqpoint{1.601231in}{1.955661in}}%
\pgfpathlineto{\pgfqpoint{1.608168in}{1.822675in}}%
\pgfpathlineto{\pgfqpoint{1.610480in}{1.795630in}}%
\pgfpathlineto{\pgfqpoint{1.615105in}{1.854623in}}%
\pgfpathlineto{\pgfqpoint{1.617417in}{1.936273in}}%
\pgfpathlineto{\pgfqpoint{1.619730in}{1.911593in}}%
\pgfpathlineto{\pgfqpoint{1.622042in}{1.936772in}}%
\pgfpathlineto{\pgfqpoint{1.624354in}{1.839858in}}%
\pgfpathlineto{\pgfqpoint{1.626667in}{1.839815in}}%
\pgfpathlineto{\pgfqpoint{1.628979in}{1.784217in}}%
\pgfpathlineto{\pgfqpoint{1.631291in}{1.853857in}}%
\pgfpathlineto{\pgfqpoint{1.633604in}{1.873514in}}%
\pgfpathlineto{\pgfqpoint{1.635916in}{1.936040in}}%
\pgfpathlineto{\pgfqpoint{1.638228in}{1.922442in}}%
\pgfpathlineto{\pgfqpoint{1.642853in}{1.841502in}}%
\pgfpathlineto{\pgfqpoint{1.645166in}{1.798044in}}%
\pgfpathlineto{\pgfqpoint{1.649790in}{1.858077in}}%
\pgfpathlineto{\pgfqpoint{1.652103in}{1.936460in}}%
\pgfpathlineto{\pgfqpoint{1.654415in}{1.906499in}}%
\pgfpathlineto{\pgfqpoint{1.656727in}{1.917133in}}%
\pgfpathlineto{\pgfqpoint{1.659040in}{1.817319in}}%
\pgfpathlineto{\pgfqpoint{1.661352in}{1.833521in}}%
\pgfpathlineto{\pgfqpoint{1.663664in}{1.800072in}}%
\pgfpathlineto{\pgfqpoint{1.665977in}{1.899138in}}%
\pgfpathlineto{\pgfqpoint{1.668289in}{1.895148in}}%
\pgfpathlineto{\pgfqpoint{1.670601in}{1.946074in}}%
\pgfpathlineto{\pgfqpoint{1.672914in}{1.864068in}}%
\pgfpathlineto{\pgfqpoint{1.675226in}{1.852171in}}%
\pgfpathlineto{\pgfqpoint{1.677538in}{1.791692in}}%
\pgfpathlineto{\pgfqpoint{1.679851in}{1.856024in}}%
\pgfpathlineto{\pgfqpoint{1.682163in}{1.875615in}}%
\pgfpathlineto{\pgfqpoint{1.684476in}{1.938414in}}%
\pgfpathlineto{\pgfqpoint{1.691413in}{1.808921in}}%
\pgfpathlineto{\pgfqpoint{1.693725in}{1.828938in}}%
\pgfpathlineto{\pgfqpoint{1.696037in}{1.861543in}}%
\pgfpathlineto{\pgfqpoint{1.698350in}{1.920513in}}%
\pgfpathlineto{\pgfqpoint{1.700662in}{1.916583in}}%
\pgfpathlineto{\pgfqpoint{1.707599in}{1.817765in}}%
\pgfpathlineto{\pgfqpoint{1.714536in}{1.921537in}}%
\pgfpathlineto{\pgfqpoint{1.721473in}{1.816874in}}%
\pgfpathlineto{\pgfqpoint{1.728410in}{1.918413in}}%
\pgfpathlineto{\pgfqpoint{1.733035in}{1.826171in}}%
\pgfpathlineto{\pgfqpoint{1.735347in}{1.824676in}}%
\pgfpathlineto{\pgfqpoint{1.737660in}{1.860928in}}%
\pgfpathlineto{\pgfqpoint{1.739972in}{1.919544in}}%
\pgfpathlineto{\pgfqpoint{1.742284in}{1.905900in}}%
\pgfpathlineto{\pgfqpoint{1.744597in}{1.871594in}}%
\pgfpathlineto{\pgfqpoint{1.746909in}{1.810817in}}%
\pgfpathlineto{\pgfqpoint{1.751534in}{1.871463in}}%
\pgfpathlineto{\pgfqpoint{1.753846in}{1.934044in}}%
\pgfpathlineto{\pgfqpoint{1.756159in}{1.879212in}}%
\pgfpathlineto{\pgfqpoint{1.758471in}{1.858991in}}%
\pgfpathlineto{\pgfqpoint{1.760783in}{1.798694in}}%
\pgfpathlineto{\pgfqpoint{1.763096in}{1.879467in}}%
\pgfpathlineto{\pgfqpoint{1.765408in}{1.884689in}}%
\pgfpathlineto{\pgfqpoint{1.767720in}{1.935961in}}%
\pgfpathlineto{\pgfqpoint{1.770033in}{1.839093in}}%
\pgfpathlineto{\pgfqpoint{1.772345in}{1.846747in}}%
\pgfpathlineto{\pgfqpoint{1.774657in}{1.812701in}}%
\pgfpathlineto{\pgfqpoint{1.776970in}{1.917843in}}%
\pgfpathlineto{\pgfqpoint{1.779282in}{1.892369in}}%
\pgfpathlineto{\pgfqpoint{1.781594in}{1.903785in}}%
\pgfpathlineto{\pgfqpoint{1.783907in}{1.808214in}}%
\pgfpathlineto{\pgfqpoint{1.790844in}{1.926516in}}%
\pgfpathlineto{\pgfqpoint{1.795469in}{1.837765in}}%
\pgfpathlineto{\pgfqpoint{1.797781in}{1.829929in}}%
\pgfpathlineto{\pgfqpoint{1.800093in}{1.866022in}}%
\pgfpathlineto{\pgfqpoint{1.802406in}{1.927696in}}%
\pgfpathlineto{\pgfqpoint{1.804718in}{1.875530in}}%
\pgfpathlineto{\pgfqpoint{1.807030in}{1.857892in}}%
\pgfpathlineto{\pgfqpoint{1.809343in}{1.802395in}}%
\pgfpathlineto{\pgfqpoint{1.811655in}{1.901238in}}%
\pgfpathlineto{\pgfqpoint{1.813967in}{1.887874in}}%
\pgfpathlineto{\pgfqpoint{1.816280in}{1.914222in}}%
\pgfpathlineto{\pgfqpoint{1.818592in}{1.811376in}}%
\pgfpathlineto{\pgfqpoint{1.825529in}{1.922036in}}%
\pgfpathlineto{\pgfqpoint{1.830154in}{1.823762in}}%
\pgfpathlineto{\pgfqpoint{1.834779in}{1.874964in}}%
\pgfpathlineto{\pgfqpoint{1.837091in}{1.933228in}}%
\pgfpathlineto{\pgfqpoint{1.839403in}{1.838534in}}%
\pgfpathlineto{\pgfqpoint{1.841716in}{1.850581in}}%
\pgfpathlineto{\pgfqpoint{1.844028in}{1.830069in}}%
\pgfpathlineto{\pgfqpoint{1.846340in}{1.926707in}}%
\pgfpathlineto{\pgfqpoint{1.850965in}{1.852076in}}%
\pgfpathlineto{\pgfqpoint{1.853277in}{1.830199in}}%
\pgfpathlineto{\pgfqpoint{1.855590in}{1.867689in}}%
\pgfpathlineto{\pgfqpoint{1.857902in}{1.929319in}}%
\pgfpathlineto{\pgfqpoint{1.860214in}{1.853347in}}%
\pgfpathlineto{\pgfqpoint{1.862527in}{1.853911in}}%
\pgfpathlineto{\pgfqpoint{1.864839in}{1.821664in}}%
\pgfpathlineto{\pgfqpoint{1.867152in}{1.924936in}}%
\pgfpathlineto{\pgfqpoint{1.871776in}{1.854619in}}%
\pgfpathlineto{\pgfqpoint{1.874089in}{1.832119in}}%
\pgfpathlineto{\pgfqpoint{1.876401in}{1.868867in}}%
\pgfpathlineto{\pgfqpoint{1.878713in}{1.930020in}}%
\pgfpathlineto{\pgfqpoint{1.881026in}{1.843459in}}%
\pgfpathlineto{\pgfqpoint{1.883338in}{1.853506in}}%
\pgfpathlineto{\pgfqpoint{1.885650in}{1.835243in}}%
\pgfpathlineto{\pgfqpoint{1.887963in}{1.925538in}}%
\pgfpathlineto{\pgfqpoint{1.892587in}{1.831298in}}%
\pgfpathlineto{\pgfqpoint{1.897212in}{1.876168in}}%
\pgfpathlineto{\pgfqpoint{1.899524in}{1.926153in}}%
\pgfpathlineto{\pgfqpoint{1.901837in}{1.817588in}}%
\pgfpathlineto{\pgfqpoint{1.906462in}{1.878178in}}%
\pgfpathlineto{\pgfqpoint{1.908774in}{1.904960in}}%
\pgfpathlineto{\pgfqpoint{1.911086in}{1.866988in}}%
\pgfpathlineto{\pgfqpoint{1.913399in}{1.807072in}}%
\pgfpathlineto{\pgfqpoint{1.915711in}{1.906861in}}%
\pgfpathlineto{\pgfqpoint{1.918023in}{1.881843in}}%
\pgfpathlineto{\pgfqpoint{1.920336in}{1.880759in}}%
\pgfpathlineto{\pgfqpoint{1.922648in}{1.822488in}}%
\pgfpathlineto{\pgfqpoint{1.927273in}{1.927027in}}%
\pgfpathlineto{\pgfqpoint{1.929585in}{1.839337in}}%
\pgfpathlineto{\pgfqpoint{1.931897in}{1.855910in}}%
\pgfpathlineto{\pgfqpoint{1.934210in}{1.851723in}}%
\pgfpathlineto{\pgfqpoint{1.936522in}{1.916904in}}%
\pgfpathlineto{\pgfqpoint{1.941147in}{1.810597in}}%
\pgfpathlineto{\pgfqpoint{1.943459in}{1.900280in}}%
\pgfpathlineto{\pgfqpoint{1.945772in}{1.879952in}}%
\pgfpathlineto{\pgfqpoint{1.948084in}{1.882633in}}%
\pgfpathlineto{\pgfqpoint{1.950396in}{1.824893in}}%
\pgfpathlineto{\pgfqpoint{1.955021in}{1.927783in}}%
\pgfpathlineto{\pgfqpoint{1.957333in}{1.828076in}}%
\pgfpathlineto{\pgfqpoint{1.961958in}{1.873540in}}%
\pgfpathlineto{\pgfqpoint{1.964270in}{1.899233in}}%
\pgfpathlineto{\pgfqpoint{1.966583in}{1.865118in}}%
\pgfpathlineto{\pgfqpoint{1.968895in}{1.811440in}}%
\pgfpathlineto{\pgfqpoint{1.971207in}{1.921427in}}%
\pgfpathlineto{\pgfqpoint{1.975832in}{1.838512in}}%
\pgfpathlineto{\pgfqpoint{1.978145in}{1.867521in}}%
\pgfpathlineto{\pgfqpoint{1.980457in}{1.875679in}}%
\pgfpathlineto{\pgfqpoint{1.982769in}{1.907241in}}%
\pgfpathlineto{\pgfqpoint{1.985082in}{1.815205in}}%
\pgfpathlineto{\pgfqpoint{1.989706in}{1.923825in}}%
\pgfpathlineto{\pgfqpoint{1.992019in}{1.833366in}}%
\pgfpathlineto{\pgfqpoint{1.996643in}{1.874299in}}%
\pgfpathlineto{\pgfqpoint{1.998956in}{1.892736in}}%
\pgfpathlineto{\pgfqpoint{2.001268in}{1.864246in}}%
\pgfpathlineto{\pgfqpoint{2.003580in}{1.818905in}}%
\pgfpathlineto{\pgfqpoint{2.005893in}{1.923814in}}%
\pgfpathlineto{\pgfqpoint{2.010517in}{1.817775in}}%
\pgfpathlineto{\pgfqpoint{2.012830in}{1.900342in}}%
\pgfpathlineto{\pgfqpoint{2.015142in}{1.875894in}}%
\pgfpathlineto{\pgfqpoint{2.017455in}{1.865986in}}%
\pgfpathlineto{\pgfqpoint{2.019767in}{1.848662in}}%
\pgfpathlineto{\pgfqpoint{2.022079in}{1.872376in}}%
\pgfpathlineto{\pgfqpoint{2.024392in}{1.913695in}}%
\pgfpathlineto{\pgfqpoint{2.026704in}{1.814796in}}%
\pgfpathlineto{\pgfqpoint{2.031329in}{1.924577in}}%
\pgfpathlineto{\pgfqpoint{2.033641in}{1.823630in}}%
\pgfpathlineto{\pgfqpoint{2.038266in}{1.896702in}}%
\pgfpathlineto{\pgfqpoint{2.040578in}{1.862766in}}%
\pgfpathlineto{\pgfqpoint{2.042890in}{1.862855in}}%
\pgfpathlineto{\pgfqpoint{2.045203in}{1.852315in}}%
\pgfpathlineto{\pgfqpoint{2.047515in}{1.902850in}}%
\pgfpathlineto{\pgfqpoint{2.052140in}{1.818876in}}%
\pgfpathlineto{\pgfqpoint{2.054452in}{1.922744in}}%
\pgfpathlineto{\pgfqpoint{2.059077in}{1.812333in}}%
\pgfpathlineto{\pgfqpoint{2.061389in}{1.917306in}}%
\pgfpathlineto{\pgfqpoint{2.066014in}{1.830998in}}%
\pgfpathlineto{\pgfqpoint{2.068326in}{1.893297in}}%
\pgfpathlineto{\pgfqpoint{2.072951in}{1.862132in}}%
\pgfpathlineto{\pgfqpoint{2.075263in}{1.862394in}}%
\pgfpathlineto{\pgfqpoint{2.077576in}{1.871634in}}%
\pgfpathlineto{\pgfqpoint{2.079888in}{1.892422in}}%
\pgfpathlineto{\pgfqpoint{2.082200in}{1.835308in}}%
\pgfpathlineto{\pgfqpoint{2.086825in}{1.913738in}}%
\pgfpathlineto{\pgfqpoint{2.089138in}{1.818381in}}%
\pgfpathlineto{\pgfqpoint{2.093762in}{1.923641in}}%
\pgfpathlineto{\pgfqpoint{2.096075in}{1.812956in}}%
\pgfpathlineto{\pgfqpoint{2.100699in}{1.923521in}}%
\pgfpathlineto{\pgfqpoint{2.103012in}{1.816810in}}%
\pgfpathlineto{\pgfqpoint{2.107636in}{1.916509in}}%
\pgfpathlineto{\pgfqpoint{2.109949in}{1.826287in}}%
\pgfpathlineto{\pgfqpoint{2.114573in}{1.905957in}}%
\pgfpathlineto{\pgfqpoint{2.116886in}{1.837931in}}%
\pgfpathlineto{\pgfqpoint{2.121510in}{1.894611in}}%
\pgfpathlineto{\pgfqpoint{2.123823in}{1.849239in}}%
\pgfpathlineto{\pgfqpoint{2.128448in}{1.884341in}}%
\pgfpathlineto{\pgfqpoint{2.130760in}{1.858736in}}%
\pgfpathlineto{\pgfqpoint{2.135385in}{1.876207in}}%
\pgfpathlineto{\pgfqpoint{2.137697in}{1.865723in}}%
\pgfpathlineto{\pgfqpoint{2.142322in}{1.870691in}}%
\pgfpathlineto{\pgfqpoint{2.144634in}{1.869963in}}%
\pgfpathlineto{\pgfqpoint{2.146946in}{1.868046in}}%
\pgfpathlineto{\pgfqpoint{2.149259in}{1.867949in}}%
\pgfpathlineto{\pgfqpoint{2.151571in}{1.871416in}}%
\pgfpathlineto{\pgfqpoint{2.153883in}{1.868397in}}%
\pgfpathlineto{\pgfqpoint{2.156196in}{1.868004in}}%
\pgfpathlineto{\pgfqpoint{2.158508in}{1.870081in}}%
\pgfpathlineto{\pgfqpoint{2.160821in}{1.868753in}}%
\pgfpathlineto{\pgfqpoint{2.163133in}{1.870845in}}%
\pgfpathlineto{\pgfqpoint{2.165445in}{1.865941in}}%
\pgfpathlineto{\pgfqpoint{2.167758in}{1.869129in}}%
\pgfpathlineto{\pgfqpoint{2.170070in}{1.876431in}}%
\pgfpathlineto{\pgfqpoint{2.172382in}{1.859012in}}%
\pgfpathlineto{\pgfqpoint{2.177007in}{1.884578in}}%
\pgfpathlineto{\pgfqpoint{2.179319in}{1.849503in}}%
\pgfpathlineto{\pgfqpoint{2.183944in}{1.894775in}}%
\pgfpathlineto{\pgfqpoint{2.186256in}{1.838077in}}%
\pgfpathlineto{\pgfqpoint{2.190881in}{1.905934in}}%
\pgfpathlineto{\pgfqpoint{2.193193in}{1.826179in}}%
\pgfpathlineto{\pgfqpoint{2.197818in}{1.916172in}}%
\pgfpathlineto{\pgfqpoint{2.200131in}{1.816305in}}%
\pgfpathlineto{\pgfqpoint{2.204755in}{1.922760in}}%
\pgfpathlineto{\pgfqpoint{2.207068in}{1.811961in}}%
\pgfpathlineto{\pgfqpoint{2.211692in}{1.922418in}}%
\pgfpathlineto{\pgfqpoint{2.214005in}{1.816939in}}%
\pgfpathlineto{\pgfqpoint{2.218629in}{1.912159in}}%
\pgfpathlineto{\pgfqpoint{2.220942in}{1.833673in}}%
\pgfpathlineto{\pgfqpoint{2.225566in}{1.890799in}}%
\pgfpathlineto{\pgfqpoint{2.227879in}{1.861041in}}%
\pgfpathlineto{\pgfqpoint{2.230191in}{1.864295in}}%
\pgfpathlineto{\pgfqpoint{2.232503in}{1.860997in}}%
\pgfpathlineto{\pgfqpoint{2.234816in}{1.892792in}}%
\pgfpathlineto{\pgfqpoint{2.239441in}{1.830957in}}%
\pgfpathlineto{\pgfqpoint{2.241753in}{1.918060in}}%
\pgfpathlineto{\pgfqpoint{2.246378in}{1.813714in}}%
\pgfpathlineto{\pgfqpoint{2.248690in}{1.924719in}}%
\pgfpathlineto{\pgfqpoint{2.253315in}{1.821262in}}%
\pgfpathlineto{\pgfqpoint{2.255627in}{1.905335in}}%
\pgfpathlineto{\pgfqpoint{2.260252in}{1.854463in}}%
\pgfpathlineto{\pgfqpoint{2.264876in}{1.874502in}}%
\pgfpathlineto{\pgfqpoint{2.267189in}{1.897193in}}%
\pgfpathlineto{\pgfqpoint{2.269501in}{1.823130in}}%
\pgfpathlineto{\pgfqpoint{2.274126in}{1.922815in}}%
\pgfpathlineto{\pgfqpoint{2.276438in}{1.812121in}}%
\pgfpathlineto{\pgfqpoint{2.281063in}{1.910604in}}%
\pgfpathlineto{\pgfqpoint{2.283375in}{1.845844in}}%
\pgfpathlineto{\pgfqpoint{2.285688in}{1.861815in}}%
\pgfpathlineto{\pgfqpoint{2.288000in}{1.863903in}}%
\pgfpathlineto{\pgfqpoint{2.290312in}{1.899957in}}%
\pgfpathlineto{\pgfqpoint{2.294937in}{1.818903in}}%
\pgfpathlineto{\pgfqpoint{2.297249in}{1.926516in}}%
\pgfpathlineto{\pgfqpoint{2.301874in}{1.822473in}}%
\pgfpathlineto{\pgfqpoint{2.304186in}{1.896202in}}%
\pgfpathlineto{\pgfqpoint{2.306499in}{1.875818in}}%
\pgfpathlineto{\pgfqpoint{2.308811in}{1.876373in}}%
\pgfpathlineto{\pgfqpoint{2.311124in}{1.833719in}}%
\pgfpathlineto{\pgfqpoint{2.315748in}{1.921796in}}%
\pgfpathlineto{\pgfqpoint{2.318061in}{1.811554in}}%
\pgfpathlineto{\pgfqpoint{2.322685in}{1.903167in}}%
\pgfpathlineto{\pgfqpoint{2.324998in}{1.864685in}}%
\pgfpathlineto{\pgfqpoint{2.327310in}{1.858499in}}%
\pgfpathlineto{\pgfqpoint{2.329622in}{1.837692in}}%
\pgfpathlineto{\pgfqpoint{2.331935in}{1.923545in}}%
\pgfpathlineto{\pgfqpoint{2.336559in}{1.815526in}}%
\pgfpathlineto{\pgfqpoint{2.338872in}{1.903660in}}%
\pgfpathlineto{\pgfqpoint{2.341184in}{1.877263in}}%
\pgfpathlineto{\pgfqpoint{2.343497in}{1.876110in}}%
\pgfpathlineto{\pgfqpoint{2.345809in}{1.827943in}}%
\pgfpathlineto{\pgfqpoint{2.350434in}{1.924439in}}%
\pgfpathlineto{\pgfqpoint{2.352746in}{1.820027in}}%
\pgfpathlineto{\pgfqpoint{2.355058in}{1.861019in}}%
\pgfpathlineto{\pgfqpoint{2.359683in}{1.899621in}}%
\pgfpathlineto{\pgfqpoint{2.364308in}{1.813399in}}%
\pgfpathlineto{\pgfqpoint{2.366620in}{1.921978in}}%
\pgfpathlineto{\pgfqpoint{2.368932in}{1.875195in}}%
\pgfpathlineto{\pgfqpoint{2.373557in}{1.840611in}}%
\pgfpathlineto{\pgfqpoint{2.378182in}{1.923843in}}%
\pgfpathlineto{\pgfqpoint{2.380494in}{1.816968in}}%
\pgfpathlineto{\pgfqpoint{2.382807in}{1.859969in}}%
\pgfpathlineto{\pgfqpoint{2.385119in}{1.876274in}}%
\pgfpathlineto{\pgfqpoint{2.387431in}{1.906904in}}%
\pgfpathlineto{\pgfqpoint{2.392056in}{1.811368in}}%
\pgfpathlineto{\pgfqpoint{2.394368in}{1.910827in}}%
\pgfpathlineto{\pgfqpoint{2.396681in}{1.880151in}}%
\pgfpathlineto{\pgfqpoint{2.398993in}{1.881081in}}%
\pgfpathlineto{\pgfqpoint{2.401305in}{1.815669in}}%
\pgfpathlineto{\pgfqpoint{2.405930in}{1.920337in}}%
\pgfpathlineto{\pgfqpoint{2.408242in}{1.852751in}}%
\pgfpathlineto{\pgfqpoint{2.410555in}{1.853192in}}%
\pgfpathlineto{\pgfqpoint{2.412867in}{1.830878in}}%
\pgfpathlineto{\pgfqpoint{2.415179in}{1.930621in}}%
\pgfpathlineto{\pgfqpoint{2.419804in}{1.841352in}}%
\pgfpathlineto{\pgfqpoint{2.422117in}{1.845621in}}%
\pgfpathlineto{\pgfqpoint{2.426741in}{1.925715in}}%
\pgfpathlineto{\pgfqpoint{2.429054in}{1.825444in}}%
\pgfpathlineto{\pgfqpoint{2.431366in}{1.854052in}}%
\pgfpathlineto{\pgfqpoint{2.433678in}{1.851423in}}%
\pgfpathlineto{\pgfqpoint{2.435991in}{1.928823in}}%
\pgfpathlineto{\pgfqpoint{2.440615in}{1.828659in}}%
\pgfpathlineto{\pgfqpoint{2.447552in}{1.925273in}}%
\pgfpathlineto{\pgfqpoint{2.449865in}{1.822974in}}%
\pgfpathlineto{\pgfqpoint{2.452177in}{1.852689in}}%
\pgfpathlineto{\pgfqpoint{2.454490in}{1.848909in}}%
\pgfpathlineto{\pgfqpoint{2.456802in}{1.931549in}}%
\pgfpathlineto{\pgfqpoint{2.461427in}{1.837316in}}%
\pgfpathlineto{\pgfqpoint{2.463739in}{1.840431in}}%
\pgfpathlineto{\pgfqpoint{2.468364in}{1.927068in}}%
\pgfpathlineto{\pgfqpoint{2.470676in}{1.844927in}}%
\pgfpathlineto{\pgfqpoint{2.472988in}{1.848592in}}%
\pgfpathlineto{\pgfqpoint{2.475301in}{1.824238in}}%
\pgfpathlineto{\pgfqpoint{2.477613in}{1.929609in}}%
\pgfpathlineto{\pgfqpoint{2.479925in}{1.881526in}}%
\pgfpathlineto{\pgfqpoint{2.482238in}{1.873836in}}%
\pgfpathlineto{\pgfqpoint{2.484550in}{1.807511in}}%
\pgfpathlineto{\pgfqpoint{2.489175in}{1.906032in}}%
\pgfpathlineto{\pgfqpoint{2.491487in}{1.899961in}}%
\pgfpathlineto{\pgfqpoint{2.496112in}{1.809697in}}%
\pgfpathlineto{\pgfqpoint{2.498424in}{1.882204in}}%
\pgfpathlineto{\pgfqpoint{2.500737in}{1.889952in}}%
\pgfpathlineto{\pgfqpoint{2.503049in}{1.924117in}}%
\pgfpathlineto{\pgfqpoint{2.505361in}{1.821339in}}%
\pgfpathlineto{\pgfqpoint{2.507674in}{1.847510in}}%
\pgfpathlineto{\pgfqpoint{2.509986in}{1.835871in}}%
\pgfpathlineto{\pgfqpoint{2.512298in}{1.934468in}}%
\pgfpathlineto{\pgfqpoint{2.514611in}{1.882063in}}%
\pgfpathlineto{\pgfqpoint{2.516923in}{1.872883in}}%
\pgfpathlineto{\pgfqpoint{2.519235in}{1.803061in}}%
\pgfpathlineto{\pgfqpoint{2.521548in}{1.865360in}}%
\pgfpathlineto{\pgfqpoint{2.526172in}{1.919627in}}%
\pgfpathlineto{\pgfqpoint{2.530797in}{1.822011in}}%
\pgfpathlineto{\pgfqpoint{2.533110in}{1.841779in}}%
\pgfpathlineto{\pgfqpoint{2.537734in}{1.927376in}}%
\pgfpathlineto{\pgfqpoint{2.542359in}{1.846092in}}%
\pgfpathlineto{\pgfqpoint{2.544671in}{1.805700in}}%
\pgfpathlineto{\pgfqpoint{2.546984in}{1.887179in}}%
\pgfpathlineto{\pgfqpoint{2.549296in}{1.894047in}}%
\pgfpathlineto{\pgfqpoint{2.551608in}{1.928903in}}%
\pgfpathlineto{\pgfqpoint{2.553921in}{1.835766in}}%
\pgfpathlineto{\pgfqpoint{2.556233in}{1.841177in}}%
\pgfpathlineto{\pgfqpoint{2.558545in}{1.813520in}}%
\pgfpathlineto{\pgfqpoint{2.560858in}{1.915886in}}%
\pgfpathlineto{\pgfqpoint{2.563170in}{1.895608in}}%
\pgfpathlineto{\pgfqpoint{2.565483in}{1.917948in}}%
\pgfpathlineto{\pgfqpoint{2.567795in}{1.814410in}}%
\pgfpathlineto{\pgfqpoint{2.570107in}{1.841429in}}%
\pgfpathlineto{\pgfqpoint{2.572420in}{1.825649in}}%
\pgfpathlineto{\pgfqpoint{2.574732in}{1.928864in}}%
\pgfpathlineto{\pgfqpoint{2.577044in}{1.894972in}}%
\pgfpathlineto{\pgfqpoint{2.579357in}{1.909215in}}%
\pgfpathlineto{\pgfqpoint{2.581669in}{1.806233in}}%
\pgfpathlineto{\pgfqpoint{2.583981in}{1.841633in}}%
\pgfpathlineto{\pgfqpoint{2.586294in}{1.830780in}}%
\pgfpathlineto{\pgfqpoint{2.588606in}{1.932285in}}%
\pgfpathlineto{\pgfqpoint{2.590918in}{1.895948in}}%
\pgfpathlineto{\pgfqpoint{2.593231in}{1.909339in}}%
\pgfpathlineto{\pgfqpoint{2.595543in}{1.806730in}}%
\pgfpathlineto{\pgfqpoint{2.597855in}{1.839308in}}%
\pgfpathlineto{\pgfqpoint{2.600168in}{1.825408in}}%
\pgfpathlineto{\pgfqpoint{2.602480in}{1.927809in}}%
\pgfpathlineto{\pgfqpoint{2.604793in}{1.899367in}}%
\pgfpathlineto{\pgfqpoint{2.607105in}{1.919008in}}%
\pgfpathlineto{\pgfqpoint{2.609417in}{1.817798in}}%
\pgfpathlineto{\pgfqpoint{2.611730in}{1.835368in}}%
\pgfpathlineto{\pgfqpoint{2.614042in}{1.811578in}}%
\pgfpathlineto{\pgfqpoint{2.616354in}{1.910349in}}%
\pgfpathlineto{\pgfqpoint{2.618667in}{1.902318in}}%
\pgfpathlineto{\pgfqpoint{2.620979in}{1.932718in}}%
\pgfpathlineto{\pgfqpoint{2.623291in}{1.846811in}}%
\pgfpathlineto{\pgfqpoint{2.625604in}{1.835336in}}%
\pgfpathlineto{\pgfqpoint{2.627916in}{1.799813in}}%
\pgfpathlineto{\pgfqpoint{2.630228in}{1.871909in}}%
\pgfpathlineto{\pgfqpoint{2.632541in}{1.897134in}}%
\pgfpathlineto{\pgfqpoint{2.634853in}{1.935482in}}%
\pgfpathlineto{\pgfqpoint{2.641790in}{1.812288in}}%
\pgfpathlineto{\pgfqpoint{2.644103in}{1.816777in}}%
\pgfpathlineto{\pgfqpoint{2.648727in}{1.905092in}}%
\pgfpathlineto{\pgfqpoint{2.651040in}{1.943062in}}%
\pgfpathlineto{\pgfqpoint{2.653352in}{1.880616in}}%
\pgfpathlineto{\pgfqpoint{2.655664in}{1.866544in}}%
\pgfpathlineto{\pgfqpoint{2.657977in}{1.787050in}}%
\pgfpathlineto{\pgfqpoint{2.660289in}{1.839504in}}%
\pgfpathlineto{\pgfqpoint{2.662601in}{1.837373in}}%
\pgfpathlineto{\pgfqpoint{2.664914in}{1.935367in}}%
\pgfpathlineto{\pgfqpoint{2.667226in}{1.908470in}}%
\pgfpathlineto{\pgfqpoint{2.669538in}{1.931771in}}%
\pgfpathlineto{\pgfqpoint{2.671851in}{1.843544in}}%
\pgfpathlineto{\pgfqpoint{2.674163in}{1.830145in}}%
\pgfpathlineto{\pgfqpoint{2.676476in}{1.794790in}}%
\pgfpathlineto{\pgfqpoint{2.683413in}{1.923358in}}%
\pgfpathlineto{\pgfqpoint{2.685725in}{1.941676in}}%
\pgfpathlineto{\pgfqpoint{2.688037in}{1.878913in}}%
\pgfpathlineto{\pgfqpoint{2.690350in}{1.863976in}}%
\pgfpathlineto{\pgfqpoint{2.692662in}{1.783137in}}%
\pgfpathlineto{\pgfqpoint{2.694974in}{1.830601in}}%
\pgfpathlineto{\pgfqpoint{2.697287in}{1.821282in}}%
\pgfpathlineto{\pgfqpoint{2.699599in}{1.919041in}}%
\pgfpathlineto{\pgfqpoint{2.701911in}{1.913884in}}%
\pgfpathlineto{\pgfqpoint{2.704224in}{1.945775in}}%
\pgfpathlineto{\pgfqpoint{2.711161in}{1.815274in}}%
\pgfpathlineto{\pgfqpoint{2.713473in}{1.786072in}}%
\pgfpathlineto{\pgfqpoint{2.715786in}{1.848354in}}%
\pgfpathlineto{\pgfqpoint{2.718098in}{1.858100in}}%
\pgfpathlineto{\pgfqpoint{2.720410in}{1.948059in}}%
\pgfpathlineto{\pgfqpoint{2.722723in}{1.918409in}}%
\pgfpathlineto{\pgfqpoint{2.725035in}{1.942016in}}%
\pgfpathlineto{\pgfqpoint{2.727347in}{1.868059in}}%
\pgfpathlineto{\pgfqpoint{2.731972in}{1.800557in}}%
\pgfpathlineto{\pgfqpoint{2.734284in}{1.789395in}}%
\pgfpathlineto{\pgfqpoint{2.736597in}{1.851828in}}%
\pgfpathlineto{\pgfqpoint{2.738909in}{1.863971in}}%
\pgfpathlineto{\pgfqpoint{2.741221in}{1.951763in}}%
\pgfpathlineto{\pgfqpoint{2.743534in}{1.923357in}}%
\pgfpathlineto{\pgfqpoint{2.745846in}{1.947924in}}%
\pgfpathlineto{\pgfqpoint{2.750471in}{1.840219in}}%
\pgfpathlineto{\pgfqpoint{2.755096in}{1.772657in}}%
\pgfpathlineto{\pgfqpoint{2.757408in}{1.832093in}}%
\pgfpathlineto{\pgfqpoint{2.759720in}{1.831784in}}%
\pgfpathlineto{\pgfqpoint{2.762033in}{1.926494in}}%
\pgfpathlineto{\pgfqpoint{2.764345in}{1.925067in}}%
\pgfpathlineto{\pgfqpoint{2.766657in}{1.956574in}}%
\pgfpathlineto{\pgfqpoint{2.768970in}{1.938066in}}%
\pgfpathlineto{\pgfqpoint{2.771282in}{1.877123in}}%
\pgfpathlineto{\pgfqpoint{2.773594in}{1.863770in}}%
\pgfpathlineto{\pgfqpoint{2.775907in}{1.777140in}}%
\pgfpathlineto{\pgfqpoint{2.778219in}{1.803843in}}%
\pgfpathlineto{\pgfqpoint{2.780531in}{1.779457in}}%
\pgfpathlineto{\pgfqpoint{2.785156in}{1.886244in}}%
\pgfpathlineto{\pgfqpoint{2.787469in}{1.910750in}}%
\pgfpathlineto{\pgfqpoint{2.789781in}{1.975235in}}%
\pgfpathlineto{\pgfqpoint{2.792093in}{1.933424in}}%
\pgfpathlineto{\pgfqpoint{2.794406in}{1.952692in}}%
\pgfpathlineto{\pgfqpoint{2.799030in}{1.841587in}}%
\pgfpathlineto{\pgfqpoint{2.801343in}{1.817114in}}%
\pgfpathlineto{\pgfqpoint{2.803655in}{1.755392in}}%
\pgfpathlineto{\pgfqpoint{2.805967in}{1.800877in}}%
\pgfpathlineto{\pgfqpoint{2.808280in}{1.784693in}}%
\pgfpathlineto{\pgfqpoint{2.812904in}{1.898112in}}%
\pgfpathlineto{\pgfqpoint{2.815217in}{1.923150in}}%
\pgfpathlineto{\pgfqpoint{2.817529in}{1.984766in}}%
\pgfpathlineto{\pgfqpoint{2.819841in}{1.945093in}}%
\pgfpathlineto{\pgfqpoint{2.822154in}{1.965936in}}%
\pgfpathlineto{\pgfqpoint{2.826779in}{1.859993in}}%
\pgfpathlineto{\pgfqpoint{2.829091in}{1.843665in}}%
\pgfpathlineto{\pgfqpoint{2.831403in}{1.761531in}}%
\pgfpathlineto{\pgfqpoint{2.833716in}{1.782525in}}%
\pgfpathlineto{\pgfqpoint{2.836028in}{1.756542in}}%
\pgfpathlineto{\pgfqpoint{2.838340in}{1.784685in}}%
\pgfpathlineto{\pgfqpoint{2.840653in}{1.844346in}}%
\pgfpathlineto{\pgfqpoint{2.842965in}{1.852361in}}%
\pgfpathlineto{\pgfqpoint{2.845277in}{1.941264in}}%
\pgfpathlineto{\pgfqpoint{2.847590in}{1.946291in}}%
\pgfpathlineto{\pgfqpoint{2.852214in}{2.001235in}}%
\pgfpathlineto{\pgfqpoint{2.854527in}{1.948209in}}%
\pgfpathlineto{\pgfqpoint{2.856839in}{1.961753in}}%
\pgfpathlineto{\pgfqpoint{2.861464in}{1.848654in}}%
\pgfpathlineto{\pgfqpoint{2.863776in}{1.832641in}}%
\pgfpathlineto{\pgfqpoint{2.866089in}{1.751141in}}%
\pgfpathlineto{\pgfqpoint{2.868401in}{1.764528in}}%
\pgfpathlineto{\pgfqpoint{2.870713in}{1.738154in}}%
\pgfpathlineto{\pgfqpoint{2.873026in}{1.739578in}}%
\pgfpathlineto{\pgfqpoint{2.875338in}{1.799234in}}%
\pgfpathlineto{\pgfqpoint{2.877650in}{1.794636in}}%
\pgfpathlineto{\pgfqpoint{2.879963in}{1.873689in}}%
\pgfpathlineto{\pgfqpoint{2.884587in}{1.930131in}}%
\pgfpathlineto{\pgfqpoint{2.886900in}{2.003589in}}%
\pgfpathlineto{\pgfqpoint{2.889212in}{1.987514in}}%
\pgfpathlineto{\pgfqpoint{2.893837in}{2.026930in}}%
\pgfpathlineto{\pgfqpoint{2.896149in}{1.973248in}}%
\pgfpathlineto{\pgfqpoint{2.898462in}{1.987082in}}%
\pgfpathlineto{\pgfqpoint{2.903086in}{1.879868in}}%
\pgfpathlineto{\pgfqpoint{2.905399in}{1.873888in}}%
\pgfpathlineto{\pgfqpoint{2.907711in}{1.789891in}}%
\pgfpathlineto{\pgfqpoint{2.912336in}{1.746351in}}%
\pgfpathlineto{\pgfqpoint{2.914648in}{1.684968in}}%
\pgfpathlineto{\pgfqpoint{2.916960in}{1.713262in}}%
\pgfpathlineto{\pgfqpoint{2.919273in}{1.690489in}}%
\pgfpathlineto{\pgfqpoint{2.921585in}{1.694293in}}%
\pgfpathlineto{\pgfqpoint{2.923897in}{1.751883in}}%
\pgfpathlineto{\pgfqpoint{2.926210in}{1.742651in}}%
\pgfpathlineto{\pgfqpoint{2.930834in}{1.855262in}}%
\pgfpathlineto{\pgfqpoint{2.933147in}{1.862364in}}%
\pgfpathlineto{\pgfqpoint{2.935459in}{1.944271in}}%
\pgfpathlineto{\pgfqpoint{2.940084in}{1.986863in}}%
\pgfpathlineto{\pgfqpoint{2.942396in}{2.058768in}}%
\pgfpathlineto{\pgfqpoint{2.944709in}{2.053165in}}%
\pgfpathlineto{\pgfqpoint{2.947021in}{2.075518in}}%
\pgfpathlineto{\pgfqpoint{2.949333in}{2.121598in}}%
\pgfpathlineto{\pgfqpoint{2.951646in}{2.091080in}}%
\pgfpathlineto{\pgfqpoint{2.956270in}{2.128078in}}%
\pgfpathlineto{\pgfqpoint{2.958583in}{2.080823in}}%
\pgfpathlineto{\pgfqpoint{2.960895in}{2.098461in}}%
\pgfpathlineto{\pgfqpoint{2.963207in}{2.084957in}}%
\pgfpathlineto{\pgfqpoint{2.965520in}{2.029159in}}%
\pgfpathlineto{\pgfqpoint{2.967832in}{2.041351in}}%
\pgfpathlineto{\pgfqpoint{2.970145in}{2.003917in}}%
\pgfpathlineto{\pgfqpoint{2.972457in}{1.946450in}}%
\pgfpathlineto{\pgfqpoint{2.974769in}{1.952840in}}%
\pgfpathlineto{\pgfqpoint{2.979394in}{1.843612in}}%
\pgfpathlineto{\pgfqpoint{2.981706in}{1.844570in}}%
\pgfpathlineto{\pgfqpoint{2.986331in}{1.730412in}}%
\pgfpathlineto{\pgfqpoint{2.988643in}{1.726698in}}%
\pgfpathlineto{\pgfqpoint{2.993268in}{1.614696in}}%
\pgfpathlineto{\pgfqpoint{2.995580in}{1.607214in}}%
\pgfpathlineto{\pgfqpoint{2.997893in}{1.531461in}}%
\pgfpathlineto{\pgfqpoint{3.000205in}{1.502204in}}%
\pgfpathlineto{\pgfqpoint{3.002517in}{1.491853in}}%
\pgfpathlineto{\pgfqpoint{3.004830in}{1.416744in}}%
\pgfpathlineto{\pgfqpoint{3.009455in}{1.384325in}}%
\pgfpathlineto{\pgfqpoint{3.011767in}{1.311861in}}%
\pgfpathlineto{\pgfqpoint{3.016392in}{1.286694in}}%
\pgfpathlineto{\pgfqpoint{3.018704in}{1.218054in}}%
\pgfpathlineto{\pgfqpoint{3.021016in}{1.214478in}}%
\pgfpathlineto{\pgfqpoint{3.023329in}{1.199790in}}%
\pgfpathlineto{\pgfqpoint{3.025641in}{1.135530in}}%
\pgfpathlineto{\pgfqpoint{3.027953in}{1.138731in}}%
\pgfpathlineto{\pgfqpoint{3.030266in}{1.123578in}}%
\pgfpathlineto{\pgfqpoint{3.032578in}{1.063809in}}%
\pgfpathlineto{\pgfqpoint{3.034890in}{1.072794in}}%
\pgfpathlineto{\pgfqpoint{3.037203in}{1.057471in}}%
\pgfpathlineto{\pgfqpoint{3.039515in}{1.002012in}}%
\pgfpathlineto{\pgfqpoint{3.041828in}{1.015854in}}%
\pgfpathlineto{\pgfqpoint{3.044140in}{1.000574in}}%
\pgfpathlineto{\pgfqpoint{3.046452in}{0.949067in}}%
\pgfpathlineto{\pgfqpoint{3.048765in}{0.966937in}}%
\pgfpathlineto{\pgfqpoint{3.051077in}{0.951847in}}%
\pgfpathlineto{\pgfqpoint{3.053389in}{0.903847in}}%
\pgfpathlineto{\pgfqpoint{3.055702in}{0.925029in}}%
\pgfpathlineto{\pgfqpoint{3.058014in}{0.910231in}}%
\pgfpathlineto{\pgfqpoint{3.060326in}{0.865263in}}%
\pgfpathlineto{\pgfqpoint{3.062639in}{0.889150in}}%
\pgfpathlineto{\pgfqpoint{3.064951in}{0.874715in}}%
\pgfpathlineto{\pgfqpoint{3.067263in}{0.832311in}}%
\pgfpathlineto{\pgfqpoint{3.069576in}{0.858401in}}%
\pgfpathlineto{\pgfqpoint{3.071888in}{0.844379in}}%
\pgfpathlineto{\pgfqpoint{3.074200in}{0.804104in}}%
\pgfpathlineto{\pgfqpoint{3.076513in}{0.831983in}}%
\pgfpathlineto{\pgfqpoint{3.078825in}{0.818411in}}%
\pgfpathlineto{\pgfqpoint{3.081138in}{0.779872in}}%
\pgfpathlineto{\pgfqpoint{3.083450in}{0.809202in}}%
\pgfpathlineto{\pgfqpoint{3.085762in}{0.796108in}}%
\pgfpathlineto{\pgfqpoint{3.088075in}{0.758960in}}%
\pgfpathlineto{\pgfqpoint{3.090387in}{0.789467in}}%
\pgfpathlineto{\pgfqpoint{3.092699in}{0.776876in}}%
\pgfpathlineto{\pgfqpoint{3.095012in}{0.740818in}}%
\pgfpathlineto{\pgfqpoint{3.097324in}{0.772281in}}%
\pgfpathlineto{\pgfqpoint{3.099636in}{0.760217in}}%
\pgfpathlineto{\pgfqpoint{3.101949in}{0.724990in}}%
\pgfpathlineto{\pgfqpoint{3.104261in}{0.757229in}}%
\pgfpathlineto{\pgfqpoint{3.106573in}{0.745716in}}%
\pgfpathlineto{\pgfqpoint{3.108886in}{0.711097in}}%
\pgfpathlineto{\pgfqpoint{3.111198in}{0.743966in}}%
\pgfpathlineto{\pgfqpoint{3.113510in}{0.733028in}}%
\pgfpathlineto{\pgfqpoint{3.115823in}{0.698829in}}%
\pgfpathlineto{\pgfqpoint{3.118135in}{0.732208in}}%
\pgfpathlineto{\pgfqpoint{3.120448in}{0.721873in}}%
\pgfpathlineto{\pgfqpoint{3.122760in}{0.687930in}}%
\pgfpathlineto{\pgfqpoint{3.125072in}{0.721719in}}%
\pgfpathlineto{\pgfqpoint{3.127385in}{0.712015in}}%
\pgfpathlineto{\pgfqpoint{3.129697in}{0.678188in}}%
\pgfpathlineto{\pgfqpoint{3.132009in}{0.712305in}}%
\pgfpathlineto{\pgfqpoint{3.134322in}{0.703265in}}%
\pgfpathlineto{\pgfqpoint{3.136634in}{0.669431in}}%
\pgfpathlineto{\pgfqpoint{3.138946in}{0.703806in}}%
\pgfpathlineto{\pgfqpoint{3.141259in}{0.695464in}}%
\pgfpathlineto{\pgfqpoint{3.143571in}{0.661515in}}%
\pgfpathlineto{\pgfqpoint{3.145883in}{0.696086in}}%
\pgfpathlineto{\pgfqpoint{3.148196in}{0.688483in}}%
\pgfpathlineto{\pgfqpoint{3.150508in}{0.654323in}}%
\pgfpathlineto{\pgfqpoint{3.152821in}{0.689036in}}%
\pgfpathlineto{\pgfqpoint{3.155133in}{0.682213in}}%
\pgfpathlineto{\pgfqpoint{3.157445in}{0.647758in}}%
\pgfpathlineto{\pgfqpoint{3.159758in}{0.682561in}}%
\pgfpathlineto{\pgfqpoint{3.162070in}{0.676566in}}%
\pgfpathlineto{\pgfqpoint{3.164382in}{0.641740in}}%
\pgfpathlineto{\pgfqpoint{3.166695in}{0.676583in}}%
\pgfpathlineto{\pgfqpoint{3.169007in}{0.671468in}}%
\pgfpathlineto{\pgfqpoint{3.171319in}{0.636205in}}%
\pgfpathlineto{\pgfqpoint{3.173632in}{0.671036in}}%
\pgfpathlineto{\pgfqpoint{3.175944in}{0.666860in}}%
\pgfpathlineto{\pgfqpoint{3.178256in}{0.631102in}}%
\pgfpathlineto{\pgfqpoint{3.180569in}{0.665864in}}%
\pgfpathlineto{\pgfqpoint{3.182881in}{0.662689in}}%
\pgfpathlineto{\pgfqpoint{3.185193in}{0.626391in}}%
\pgfpathlineto{\pgfqpoint{3.187506in}{0.661018in}}%
\pgfpathlineto{\pgfqpoint{3.189818in}{0.658916in}}%
\pgfpathlineto{\pgfqpoint{3.192131in}{0.622047in}}%
\pgfpathlineto{\pgfqpoint{3.194443in}{0.656458in}}%
\pgfpathlineto{\pgfqpoint{3.196755in}{0.655505in}}%
\pgfpathlineto{\pgfqpoint{3.199068in}{0.618059in}}%
\pgfpathlineto{\pgfqpoint{3.201380in}{0.652147in}}%
\pgfpathlineto{\pgfqpoint{3.203692in}{0.652430in}}%
\pgfpathlineto{\pgfqpoint{3.206005in}{0.614436in}}%
\pgfpathlineto{\pgfqpoint{3.208317in}{0.648056in}}%
\pgfpathlineto{\pgfqpoint{3.210629in}{0.649667in}}%
\pgfpathlineto{\pgfqpoint{3.212942in}{0.611212in}}%
\pgfpathlineto{\pgfqpoint{3.215254in}{0.644158in}}%
\pgfpathlineto{\pgfqpoint{3.217566in}{0.647201in}}%
\pgfpathlineto{\pgfqpoint{3.219879in}{0.608459in}}%
\pgfpathlineto{\pgfqpoint{3.222191in}{0.640430in}}%
\pgfpathlineto{\pgfqpoint{3.224503in}{0.645015in}}%
\pgfpathlineto{\pgfqpoint{3.226816in}{0.606297in}}%
\pgfpathlineto{\pgfqpoint{3.229128in}{0.636853in}}%
\pgfpathlineto{\pgfqpoint{3.231441in}{0.643102in}}%
\pgfpathlineto{\pgfqpoint{3.233753in}{0.604884in}}%
\pgfpathlineto{\pgfqpoint{3.236065in}{0.633409in}}%
\pgfpathlineto{\pgfqpoint{3.238378in}{0.641452in}}%
\pgfpathlineto{\pgfqpoint{3.240690in}{0.604364in}}%
\pgfpathlineto{\pgfqpoint{3.243002in}{0.630085in}}%
\pgfpathlineto{\pgfqpoint{3.245315in}{0.640061in}}%
\pgfpathlineto{\pgfqpoint{3.247627in}{0.604763in}}%
\pgfpathlineto{\pgfqpoint{3.252252in}{0.638925in}}%
\pgfpathlineto{\pgfqpoint{3.254564in}{0.605956in}}%
\pgfpathlineto{\pgfqpoint{3.259189in}{0.638043in}}%
\pgfpathlineto{\pgfqpoint{3.261501in}{0.607739in}}%
\pgfpathlineto{\pgfqpoint{3.266126in}{0.637414in}}%
\pgfpathlineto{\pgfqpoint{3.268438in}{0.609927in}}%
\pgfpathlineto{\pgfqpoint{3.270751in}{0.617834in}}%
\pgfpathlineto{\pgfqpoint{3.273063in}{0.637036in}}%
\pgfpathlineto{\pgfqpoint{3.275375in}{0.612388in}}%
\pgfpathlineto{\pgfqpoint{3.277688in}{0.615040in}}%
\pgfpathlineto{\pgfqpoint{3.280000in}{0.636910in}}%
\pgfpathlineto{\pgfqpoint{3.282312in}{0.615040in}}%
\pgfpathlineto{\pgfqpoint{3.284625in}{0.612388in}}%
\pgfpathlineto{\pgfqpoint{3.286937in}{0.637036in}}%
\pgfpathlineto{\pgfqpoint{3.289249in}{0.617834in}}%
\pgfpathlineto{\pgfqpoint{3.291562in}{0.609927in}}%
\pgfpathlineto{\pgfqpoint{3.293874in}{0.637414in}}%
\pgfpathlineto{\pgfqpoint{3.298499in}{0.607739in}}%
\pgfpathlineto{\pgfqpoint{3.300811in}{0.638043in}}%
\pgfpathlineto{\pgfqpoint{3.305436in}{0.605956in}}%
\pgfpathlineto{\pgfqpoint{3.307748in}{0.638925in}}%
\pgfpathlineto{\pgfqpoint{3.310061in}{0.626870in}}%
\pgfpathlineto{\pgfqpoint{3.312373in}{0.604763in}}%
\pgfpathlineto{\pgfqpoint{3.314685in}{0.640061in}}%
\pgfpathlineto{\pgfqpoint{3.316998in}{0.630085in}}%
\pgfpathlineto{\pgfqpoint{3.319310in}{0.604364in}}%
\pgfpathlineto{\pgfqpoint{3.321622in}{0.641452in}}%
\pgfpathlineto{\pgfqpoint{3.323935in}{0.633409in}}%
\pgfpathlineto{\pgfqpoint{3.326247in}{0.604884in}}%
\pgfpathlineto{\pgfqpoint{3.328559in}{0.643102in}}%
\pgfpathlineto{\pgfqpoint{3.330872in}{0.636853in}}%
\pgfpathlineto{\pgfqpoint{3.333184in}{0.606297in}}%
\pgfpathlineto{\pgfqpoint{3.335497in}{0.645015in}}%
\pgfpathlineto{\pgfqpoint{3.337809in}{0.640430in}}%
\pgfpathlineto{\pgfqpoint{3.340121in}{0.608459in}}%
\pgfpathlineto{\pgfqpoint{3.342434in}{0.647201in}}%
\pgfpathlineto{\pgfqpoint{3.344746in}{0.644158in}}%
\pgfpathlineto{\pgfqpoint{3.347058in}{0.611212in}}%
\pgfpathlineto{\pgfqpoint{3.349371in}{0.649667in}}%
\pgfpathlineto{\pgfqpoint{3.351683in}{0.648056in}}%
\pgfpathlineto{\pgfqpoint{3.353995in}{0.614436in}}%
\pgfpathlineto{\pgfqpoint{3.356308in}{0.652430in}}%
\pgfpathlineto{\pgfqpoint{3.358620in}{0.652147in}}%
\pgfpathlineto{\pgfqpoint{3.360932in}{0.618059in}}%
\pgfpathlineto{\pgfqpoint{3.363245in}{0.655505in}}%
\pgfpathlineto{\pgfqpoint{3.365557in}{0.656458in}}%
\pgfpathlineto{\pgfqpoint{3.367869in}{0.622047in}}%
\pgfpathlineto{\pgfqpoint{3.370182in}{0.658916in}}%
\pgfpathlineto{\pgfqpoint{3.372494in}{0.661018in}}%
\pgfpathlineto{\pgfqpoint{3.374807in}{0.626391in}}%
\pgfpathlineto{\pgfqpoint{3.377119in}{0.662689in}}%
\pgfpathlineto{\pgfqpoint{3.379431in}{0.665864in}}%
\pgfpathlineto{\pgfqpoint{3.381744in}{0.631102in}}%
\pgfpathlineto{\pgfqpoint{3.384056in}{0.666860in}}%
\pgfpathlineto{\pgfqpoint{3.386368in}{0.671036in}}%
\pgfpathlineto{\pgfqpoint{3.388681in}{0.636205in}}%
\pgfpathlineto{\pgfqpoint{3.390993in}{0.671468in}}%
\pgfpathlineto{\pgfqpoint{3.393305in}{0.676583in}}%
\pgfpathlineto{\pgfqpoint{3.395618in}{0.641740in}}%
\pgfpathlineto{\pgfqpoint{3.397930in}{0.676566in}}%
\pgfpathlineto{\pgfqpoint{3.400242in}{0.682561in}}%
\pgfpathlineto{\pgfqpoint{3.402555in}{0.647758in}}%
\pgfpathlineto{\pgfqpoint{3.404867in}{0.682213in}}%
\pgfpathlineto{\pgfqpoint{3.407179in}{0.689036in}}%
\pgfpathlineto{\pgfqpoint{3.409492in}{0.654323in}}%
\pgfpathlineto{\pgfqpoint{3.411804in}{0.688483in}}%
\pgfpathlineto{\pgfqpoint{3.414117in}{0.696086in}}%
\pgfpathlineto{\pgfqpoint{3.416429in}{0.661515in}}%
\pgfpathlineto{\pgfqpoint{3.418741in}{0.695464in}}%
\pgfpathlineto{\pgfqpoint{3.421054in}{0.703806in}}%
\pgfpathlineto{\pgfqpoint{3.423366in}{0.669431in}}%
\pgfpathlineto{\pgfqpoint{3.425678in}{0.703265in}}%
\pgfpathlineto{\pgfqpoint{3.427991in}{0.712305in}}%
\pgfpathlineto{\pgfqpoint{3.430303in}{0.678188in}}%
\pgfpathlineto{\pgfqpoint{3.432615in}{0.712015in}}%
\pgfpathlineto{\pgfqpoint{3.434928in}{0.721719in}}%
\pgfpathlineto{\pgfqpoint{3.437240in}{0.687930in}}%
\pgfpathlineto{\pgfqpoint{3.439552in}{0.721873in}}%
\pgfpathlineto{\pgfqpoint{3.441865in}{0.732208in}}%
\pgfpathlineto{\pgfqpoint{3.444177in}{0.698829in}}%
\pgfpathlineto{\pgfqpoint{3.446490in}{0.733028in}}%
\pgfpathlineto{\pgfqpoint{3.448802in}{0.743966in}}%
\pgfpathlineto{\pgfqpoint{3.451114in}{0.711097in}}%
\pgfpathlineto{\pgfqpoint{3.453427in}{0.745716in}}%
\pgfpathlineto{\pgfqpoint{3.455739in}{0.757229in}}%
\pgfpathlineto{\pgfqpoint{3.458051in}{0.724990in}}%
\pgfpathlineto{\pgfqpoint{3.460364in}{0.760217in}}%
\pgfpathlineto{\pgfqpoint{3.462676in}{0.772281in}}%
\pgfpathlineto{\pgfqpoint{3.464988in}{0.740818in}}%
\pgfpathlineto{\pgfqpoint{3.467301in}{0.776876in}}%
\pgfpathlineto{\pgfqpoint{3.469613in}{0.789467in}}%
\pgfpathlineto{\pgfqpoint{3.471925in}{0.758960in}}%
\pgfpathlineto{\pgfqpoint{3.474238in}{0.796108in}}%
\pgfpathlineto{\pgfqpoint{3.476550in}{0.809202in}}%
\pgfpathlineto{\pgfqpoint{3.478862in}{0.779872in}}%
\pgfpathlineto{\pgfqpoint{3.481175in}{0.818411in}}%
\pgfpathlineto{\pgfqpoint{3.483487in}{0.831983in}}%
\pgfpathlineto{\pgfqpoint{3.485800in}{0.804104in}}%
\pgfpathlineto{\pgfqpoint{3.488112in}{0.844379in}}%
\pgfpathlineto{\pgfqpoint{3.490424in}{0.858401in}}%
\pgfpathlineto{\pgfqpoint{3.492737in}{0.832311in}}%
\pgfpathlineto{\pgfqpoint{3.495049in}{0.874715in}}%
\pgfpathlineto{\pgfqpoint{3.497361in}{0.889150in}}%
\pgfpathlineto{\pgfqpoint{3.499674in}{0.865263in}}%
\pgfpathlineto{\pgfqpoint{3.501986in}{0.910231in}}%
\pgfpathlineto{\pgfqpoint{3.504298in}{0.925029in}}%
\pgfpathlineto{\pgfqpoint{3.506611in}{0.903847in}}%
\pgfpathlineto{\pgfqpoint{3.508923in}{0.951847in}}%
\pgfpathlineto{\pgfqpoint{3.511235in}{0.966937in}}%
\pgfpathlineto{\pgfqpoint{3.513548in}{0.949067in}}%
\pgfpathlineto{\pgfqpoint{3.515860in}{1.000574in}}%
\pgfpathlineto{\pgfqpoint{3.518172in}{1.015854in}}%
\pgfpathlineto{\pgfqpoint{3.520485in}{1.002012in}}%
\pgfpathlineto{\pgfqpoint{3.522797in}{1.057471in}}%
\pgfpathlineto{\pgfqpoint{3.525110in}{1.072794in}}%
\pgfpathlineto{\pgfqpoint{3.527422in}{1.063809in}}%
\pgfpathlineto{\pgfqpoint{3.529734in}{1.123578in}}%
\pgfpathlineto{\pgfqpoint{3.532047in}{1.138731in}}%
\pgfpathlineto{\pgfqpoint{3.534359in}{1.135530in}}%
\pgfpathlineto{\pgfqpoint{3.536671in}{1.199790in}}%
\pgfpathlineto{\pgfqpoint{3.538984in}{1.214478in}}%
\pgfpathlineto{\pgfqpoint{3.541296in}{1.218054in}}%
\pgfpathlineto{\pgfqpoint{3.543608in}{1.286694in}}%
\pgfpathlineto{\pgfqpoint{3.548233in}{1.311861in}}%
\pgfpathlineto{\pgfqpoint{3.550545in}{1.384325in}}%
\pgfpathlineto{\pgfqpoint{3.552858in}{1.396744in}}%
\pgfpathlineto{\pgfqpoint{3.555170in}{1.416744in}}%
\pgfpathlineto{\pgfqpoint{3.557483in}{1.491853in}}%
\pgfpathlineto{\pgfqpoint{3.559795in}{1.502204in}}%
\pgfpathlineto{\pgfqpoint{3.562107in}{1.531461in}}%
\pgfpathlineto{\pgfqpoint{3.564420in}{1.607214in}}%
\pgfpathlineto{\pgfqpoint{3.566732in}{1.614696in}}%
\pgfpathlineto{\pgfqpoint{3.569044in}{1.653304in}}%
\pgfpathlineto{\pgfqpoint{3.571357in}{1.726698in}}%
\pgfpathlineto{\pgfqpoint{3.573669in}{1.730412in}}%
\pgfpathlineto{\pgfqpoint{3.578294in}{1.844570in}}%
\pgfpathlineto{\pgfqpoint{3.580606in}{1.843612in}}%
\pgfpathlineto{\pgfqpoint{3.585231in}{1.952840in}}%
\pgfpathlineto{\pgfqpoint{3.587543in}{1.946450in}}%
\pgfpathlineto{\pgfqpoint{3.592168in}{2.041351in}}%
\pgfpathlineto{\pgfqpoint{3.594480in}{2.029159in}}%
\pgfpathlineto{\pgfqpoint{3.596793in}{2.084957in}}%
\pgfpathlineto{\pgfqpoint{3.599105in}{2.098461in}}%
\pgfpathlineto{\pgfqpoint{3.601417in}{2.080823in}}%
\pgfpathlineto{\pgfqpoint{3.603730in}{2.128078in}}%
\pgfpathlineto{\pgfqpoint{3.608354in}{2.091080in}}%
\pgfpathlineto{\pgfqpoint{3.610667in}{2.121598in}}%
\pgfpathlineto{\pgfqpoint{3.612979in}{2.075518in}}%
\pgfpathlineto{\pgfqpoint{3.615291in}{2.053165in}}%
\pgfpathlineto{\pgfqpoint{3.617604in}{2.058768in}}%
\pgfpathlineto{\pgfqpoint{3.619916in}{1.986863in}}%
\pgfpathlineto{\pgfqpoint{3.624541in}{1.944271in}}%
\pgfpathlineto{\pgfqpoint{3.626853in}{1.862364in}}%
\pgfpathlineto{\pgfqpoint{3.629166in}{1.855262in}}%
\pgfpathlineto{\pgfqpoint{3.633790in}{1.742651in}}%
\pgfpathlineto{\pgfqpoint{3.636103in}{1.751883in}}%
\pgfpathlineto{\pgfqpoint{3.638415in}{1.694293in}}%
\pgfpathlineto{\pgfqpoint{3.640727in}{1.690489in}}%
\pgfpathlineto{\pgfqpoint{3.643040in}{1.713262in}}%
\pgfpathlineto{\pgfqpoint{3.645352in}{1.684968in}}%
\pgfpathlineto{\pgfqpoint{3.647664in}{1.746351in}}%
\pgfpathlineto{\pgfqpoint{3.652289in}{1.789891in}}%
\pgfpathlineto{\pgfqpoint{3.654601in}{1.873888in}}%
\pgfpathlineto{\pgfqpoint{3.656914in}{1.879868in}}%
\pgfpathlineto{\pgfqpoint{3.661538in}{1.987082in}}%
\pgfpathlineto{\pgfqpoint{3.663851in}{1.973248in}}%
\pgfpathlineto{\pgfqpoint{3.666163in}{2.026930in}}%
\pgfpathlineto{\pgfqpoint{3.668476in}{2.013090in}}%
\pgfpathlineto{\pgfqpoint{3.670788in}{1.987514in}}%
\pgfpathlineto{\pgfqpoint{3.673100in}{2.003589in}}%
\pgfpathlineto{\pgfqpoint{3.675413in}{1.930131in}}%
\pgfpathlineto{\pgfqpoint{3.677725in}{1.910324in}}%
\pgfpathlineto{\pgfqpoint{3.680037in}{1.873689in}}%
\pgfpathlineto{\pgfqpoint{3.682350in}{1.794636in}}%
\pgfpathlineto{\pgfqpoint{3.684662in}{1.799234in}}%
\pgfpathlineto{\pgfqpoint{3.686974in}{1.739578in}}%
\pgfpathlineto{\pgfqpoint{3.689287in}{1.738154in}}%
\pgfpathlineto{\pgfqpoint{3.691599in}{1.764528in}}%
\pgfpathlineto{\pgfqpoint{3.693911in}{1.751141in}}%
\pgfpathlineto{\pgfqpoint{3.696224in}{1.832641in}}%
\pgfpathlineto{\pgfqpoint{3.698536in}{1.848654in}}%
\pgfpathlineto{\pgfqpoint{3.703161in}{1.961753in}}%
\pgfpathlineto{\pgfqpoint{3.705473in}{1.948209in}}%
\pgfpathlineto{\pgfqpoint{3.707786in}{2.001235in}}%
\pgfpathlineto{\pgfqpoint{3.712410in}{1.946291in}}%
\pgfpathlineto{\pgfqpoint{3.714723in}{1.941264in}}%
\pgfpathlineto{\pgfqpoint{3.717035in}{1.852361in}}%
\pgfpathlineto{\pgfqpoint{3.719347in}{1.844346in}}%
\pgfpathlineto{\pgfqpoint{3.721660in}{1.784685in}}%
\pgfpathlineto{\pgfqpoint{3.723972in}{1.756542in}}%
\pgfpathlineto{\pgfqpoint{3.726284in}{1.782525in}}%
\pgfpathlineto{\pgfqpoint{3.728597in}{1.761531in}}%
\pgfpathlineto{\pgfqpoint{3.730909in}{1.843665in}}%
\pgfpathlineto{\pgfqpoint{3.733221in}{1.859993in}}%
\pgfpathlineto{\pgfqpoint{3.737846in}{1.965936in}}%
\pgfpathlineto{\pgfqpoint{3.740159in}{1.945093in}}%
\pgfpathlineto{\pgfqpoint{3.742471in}{1.984766in}}%
\pgfpathlineto{\pgfqpoint{3.744783in}{1.923150in}}%
\pgfpathlineto{\pgfqpoint{3.747096in}{1.898112in}}%
\pgfpathlineto{\pgfqpoint{3.749408in}{1.857919in}}%
\pgfpathlineto{\pgfqpoint{3.751720in}{1.784693in}}%
\pgfpathlineto{\pgfqpoint{3.754033in}{1.800877in}}%
\pgfpathlineto{\pgfqpoint{3.756345in}{1.755392in}}%
\pgfpathlineto{\pgfqpoint{3.758657in}{1.817114in}}%
\pgfpathlineto{\pgfqpoint{3.760970in}{1.841587in}}%
\pgfpathlineto{\pgfqpoint{3.765594in}{1.952692in}}%
\pgfpathlineto{\pgfqpoint{3.767907in}{1.933424in}}%
\pgfpathlineto{\pgfqpoint{3.770219in}{1.975235in}}%
\pgfpathlineto{\pgfqpoint{3.772531in}{1.910750in}}%
\pgfpathlineto{\pgfqpoint{3.774844in}{1.886244in}}%
\pgfpathlineto{\pgfqpoint{3.779469in}{1.779457in}}%
\pgfpathlineto{\pgfqpoint{3.781781in}{1.803843in}}%
\pgfpathlineto{\pgfqpoint{3.784093in}{1.777140in}}%
\pgfpathlineto{\pgfqpoint{3.786406in}{1.863770in}}%
\pgfpathlineto{\pgfqpoint{3.788718in}{1.877123in}}%
\pgfpathlineto{\pgfqpoint{3.791030in}{1.938066in}}%
\pgfpathlineto{\pgfqpoint{3.793343in}{1.956574in}}%
\pgfpathlineto{\pgfqpoint{3.795655in}{1.925067in}}%
\pgfpathlineto{\pgfqpoint{3.797967in}{1.926494in}}%
\pgfpathlineto{\pgfqpoint{3.800280in}{1.831784in}}%
\pgfpathlineto{\pgfqpoint{3.802592in}{1.832093in}}%
\pgfpathlineto{\pgfqpoint{3.804904in}{1.772657in}}%
\pgfpathlineto{\pgfqpoint{3.811841in}{1.882960in}}%
\pgfpathlineto{\pgfqpoint{3.814154in}{1.947924in}}%
\pgfpathlineto{\pgfqpoint{3.816466in}{1.923357in}}%
\pgfpathlineto{\pgfqpoint{3.818779in}{1.951763in}}%
\pgfpathlineto{\pgfqpoint{3.821091in}{1.863971in}}%
\pgfpathlineto{\pgfqpoint{3.823403in}{1.851828in}}%
\pgfpathlineto{\pgfqpoint{3.825716in}{1.789395in}}%
\pgfpathlineto{\pgfqpoint{3.828028in}{1.800557in}}%
\pgfpathlineto{\pgfqpoint{3.832653in}{1.868059in}}%
\pgfpathlineto{\pgfqpoint{3.834965in}{1.942016in}}%
\pgfpathlineto{\pgfqpoint{3.837277in}{1.918409in}}%
\pgfpathlineto{\pgfqpoint{3.839590in}{1.948059in}}%
\pgfpathlineto{\pgfqpoint{3.841902in}{1.858100in}}%
\pgfpathlineto{\pgfqpoint{3.844214in}{1.848354in}}%
\pgfpathlineto{\pgfqpoint{3.846527in}{1.786072in}}%
\pgfpathlineto{\pgfqpoint{3.851152in}{1.846867in}}%
\pgfpathlineto{\pgfqpoint{3.855776in}{1.945775in}}%
\pgfpathlineto{\pgfqpoint{3.858089in}{1.913884in}}%
\pgfpathlineto{\pgfqpoint{3.860401in}{1.919041in}}%
\pgfpathlineto{\pgfqpoint{3.862713in}{1.821282in}}%
\pgfpathlineto{\pgfqpoint{3.865026in}{1.830601in}}%
\pgfpathlineto{\pgfqpoint{3.867338in}{1.783137in}}%
\pgfpathlineto{\pgfqpoint{3.869650in}{1.863976in}}%
\pgfpathlineto{\pgfqpoint{3.871963in}{1.878913in}}%
\pgfpathlineto{\pgfqpoint{3.874275in}{1.941676in}}%
\pgfpathlineto{\pgfqpoint{3.876587in}{1.923358in}}%
\pgfpathlineto{\pgfqpoint{3.881212in}{1.844739in}}%
\pgfpathlineto{\pgfqpoint{3.883524in}{1.794790in}}%
\pgfpathlineto{\pgfqpoint{3.885837in}{1.830145in}}%
\pgfpathlineto{\pgfqpoint{3.888149in}{1.843544in}}%
\pgfpathlineto{\pgfqpoint{3.890462in}{1.931771in}}%
\pgfpathlineto{\pgfqpoint{3.892774in}{1.908470in}}%
\pgfpathlineto{\pgfqpoint{3.895086in}{1.935367in}}%
\pgfpathlineto{\pgfqpoint{3.897399in}{1.837373in}}%
\pgfpathlineto{\pgfqpoint{3.899711in}{1.839504in}}%
\pgfpathlineto{\pgfqpoint{3.902023in}{1.787050in}}%
\pgfpathlineto{\pgfqpoint{3.904336in}{1.866544in}}%
\pgfpathlineto{\pgfqpoint{3.906648in}{1.880616in}}%
\pgfpathlineto{\pgfqpoint{3.908960in}{1.943062in}}%
\pgfpathlineto{\pgfqpoint{3.915897in}{1.816777in}}%
\pgfpathlineto{\pgfqpoint{3.918210in}{1.812288in}}%
\pgfpathlineto{\pgfqpoint{3.925147in}{1.935482in}}%
\pgfpathlineto{\pgfqpoint{3.929772in}{1.871909in}}%
\pgfpathlineto{\pgfqpoint{3.932084in}{1.799813in}}%
\pgfpathlineto{\pgfqpoint{3.934396in}{1.835336in}}%
\pgfpathlineto{\pgfqpoint{3.936709in}{1.846811in}}%
\pgfpathlineto{\pgfqpoint{3.939021in}{1.932718in}}%
\pgfpathlineto{\pgfqpoint{3.941333in}{1.902318in}}%
\pgfpathlineto{\pgfqpoint{3.943646in}{1.910349in}}%
\pgfpathlineto{\pgfqpoint{3.945958in}{1.811578in}}%
\pgfpathlineto{\pgfqpoint{3.948270in}{1.835368in}}%
\pgfpathlineto{\pgfqpoint{3.950583in}{1.817798in}}%
\pgfpathlineto{\pgfqpoint{3.952895in}{1.919008in}}%
\pgfpathlineto{\pgfqpoint{3.955207in}{1.899367in}}%
\pgfpathlineto{\pgfqpoint{3.957520in}{1.927809in}}%
\pgfpathlineto{\pgfqpoint{3.959832in}{1.825408in}}%
\pgfpathlineto{\pgfqpoint{3.962145in}{1.839308in}}%
\pgfpathlineto{\pgfqpoint{3.964457in}{1.806730in}}%
\pgfpathlineto{\pgfqpoint{3.966769in}{1.909339in}}%
\pgfpathlineto{\pgfqpoint{3.969082in}{1.895948in}}%
\pgfpathlineto{\pgfqpoint{3.971394in}{1.932285in}}%
\pgfpathlineto{\pgfqpoint{3.973706in}{1.830780in}}%
\pgfpathlineto{\pgfqpoint{3.976019in}{1.841633in}}%
\pgfpathlineto{\pgfqpoint{3.978331in}{1.806233in}}%
\pgfpathlineto{\pgfqpoint{3.980643in}{1.909215in}}%
\pgfpathlineto{\pgfqpoint{3.982956in}{1.894972in}}%
\pgfpathlineto{\pgfqpoint{3.985268in}{1.928864in}}%
\pgfpathlineto{\pgfqpoint{3.987580in}{1.825649in}}%
\pgfpathlineto{\pgfqpoint{3.989893in}{1.841429in}}%
\pgfpathlineto{\pgfqpoint{3.992205in}{1.814410in}}%
\pgfpathlineto{\pgfqpoint{3.994517in}{1.917948in}}%
\pgfpathlineto{\pgfqpoint{3.996830in}{1.895608in}}%
\pgfpathlineto{\pgfqpoint{3.999142in}{1.915886in}}%
\pgfpathlineto{\pgfqpoint{4.001455in}{1.813520in}}%
\pgfpathlineto{\pgfqpoint{4.003767in}{1.841177in}}%
\pgfpathlineto{\pgfqpoint{4.006079in}{1.835766in}}%
\pgfpathlineto{\pgfqpoint{4.008392in}{1.928903in}}%
\pgfpathlineto{\pgfqpoint{4.010704in}{1.894047in}}%
\pgfpathlineto{\pgfqpoint{4.013016in}{1.887179in}}%
\pgfpathlineto{\pgfqpoint{4.015329in}{1.805700in}}%
\pgfpathlineto{\pgfqpoint{4.022266in}{1.927376in}}%
\pgfpathlineto{\pgfqpoint{4.026890in}{1.841779in}}%
\pgfpathlineto{\pgfqpoint{4.029203in}{1.822011in}}%
\pgfpathlineto{\pgfqpoint{4.031515in}{1.861069in}}%
\pgfpathlineto{\pgfqpoint{4.033828in}{1.919627in}}%
\pgfpathlineto{\pgfqpoint{4.038452in}{1.865360in}}%
\pgfpathlineto{\pgfqpoint{4.040765in}{1.803061in}}%
\pgfpathlineto{\pgfqpoint{4.043077in}{1.872883in}}%
\pgfpathlineto{\pgfqpoint{4.045389in}{1.882063in}}%
\pgfpathlineto{\pgfqpoint{4.047702in}{1.934468in}}%
\pgfpathlineto{\pgfqpoint{4.050014in}{1.835871in}}%
\pgfpathlineto{\pgfqpoint{4.052326in}{1.847510in}}%
\pgfpathlineto{\pgfqpoint{4.054639in}{1.821339in}}%
\pgfpathlineto{\pgfqpoint{4.056951in}{1.924117in}}%
\pgfpathlineto{\pgfqpoint{4.059263in}{1.889952in}}%
\pgfpathlineto{\pgfqpoint{4.061576in}{1.882204in}}%
\pgfpathlineto{\pgfqpoint{4.063888in}{1.809697in}}%
\pgfpathlineto{\pgfqpoint{4.068513in}{1.899961in}}%
\pgfpathlineto{\pgfqpoint{4.070825in}{1.906032in}}%
\pgfpathlineto{\pgfqpoint{4.073138in}{1.869358in}}%
\pgfpathlineto{\pgfqpoint{4.075450in}{1.807511in}}%
\pgfpathlineto{\pgfqpoint{4.077762in}{1.873836in}}%
\pgfpathlineto{\pgfqpoint{4.080075in}{1.881526in}}%
\pgfpathlineto{\pgfqpoint{4.082387in}{1.929609in}}%
\pgfpathlineto{\pgfqpoint{4.084699in}{1.824238in}}%
\pgfpathlineto{\pgfqpoint{4.087012in}{1.848592in}}%
\pgfpathlineto{\pgfqpoint{4.089324in}{1.844927in}}%
\pgfpathlineto{\pgfqpoint{4.091636in}{1.927068in}}%
\pgfpathlineto{\pgfqpoint{4.096261in}{1.840431in}}%
\pgfpathlineto{\pgfqpoint{4.098573in}{1.837316in}}%
\pgfpathlineto{\pgfqpoint{4.100886in}{1.870185in}}%
\pgfpathlineto{\pgfqpoint{4.103198in}{1.931549in}}%
\pgfpathlineto{\pgfqpoint{4.105510in}{1.848909in}}%
\pgfpathlineto{\pgfqpoint{4.107823in}{1.852689in}}%
\pgfpathlineto{\pgfqpoint{4.110135in}{1.822974in}}%
\pgfpathlineto{\pgfqpoint{4.112448in}{1.925273in}}%
\pgfpathlineto{\pgfqpoint{4.117072in}{1.856891in}}%
\pgfpathlineto{\pgfqpoint{4.119385in}{1.828659in}}%
\pgfpathlineto{\pgfqpoint{4.121697in}{1.867335in}}%
\pgfpathlineto{\pgfqpoint{4.124009in}{1.928823in}}%
\pgfpathlineto{\pgfqpoint{4.126322in}{1.851423in}}%
\pgfpathlineto{\pgfqpoint{4.128634in}{1.854052in}}%
\pgfpathlineto{\pgfqpoint{4.130946in}{1.825444in}}%
\pgfpathlineto{\pgfqpoint{4.133259in}{1.925715in}}%
\pgfpathlineto{\pgfqpoint{4.137883in}{1.845621in}}%
\pgfpathlineto{\pgfqpoint{4.140196in}{1.841352in}}%
\pgfpathlineto{\pgfqpoint{4.142508in}{1.871996in}}%
\pgfpathlineto{\pgfqpoint{4.144821in}{1.930621in}}%
\pgfpathlineto{\pgfqpoint{4.147133in}{1.830878in}}%
\pgfpathlineto{\pgfqpoint{4.149445in}{1.853192in}}%
\pgfpathlineto{\pgfqpoint{4.151758in}{1.852751in}}%
\pgfpathlineto{\pgfqpoint{4.154070in}{1.920337in}}%
\pgfpathlineto{\pgfqpoint{4.158695in}{1.815669in}}%
\pgfpathlineto{\pgfqpoint{4.161007in}{1.881081in}}%
\pgfpathlineto{\pgfqpoint{4.163319in}{1.880151in}}%
\pgfpathlineto{\pgfqpoint{4.165632in}{1.910827in}}%
\pgfpathlineto{\pgfqpoint{4.167944in}{1.811368in}}%
\pgfpathlineto{\pgfqpoint{4.172569in}{1.906904in}}%
\pgfpathlineto{\pgfqpoint{4.179506in}{1.816968in}}%
\pgfpathlineto{\pgfqpoint{4.181818in}{1.923843in}}%
\pgfpathlineto{\pgfqpoint{4.186443in}{1.840611in}}%
\pgfpathlineto{\pgfqpoint{4.191068in}{1.875195in}}%
\pgfpathlineto{\pgfqpoint{4.193380in}{1.921978in}}%
\pgfpathlineto{\pgfqpoint{4.195692in}{1.813399in}}%
\pgfpathlineto{\pgfqpoint{4.200317in}{1.899621in}}%
\pgfpathlineto{\pgfqpoint{4.204942in}{1.861019in}}%
\pgfpathlineto{\pgfqpoint{4.207254in}{1.820027in}}%
\pgfpathlineto{\pgfqpoint{4.209566in}{1.924439in}}%
\pgfpathlineto{\pgfqpoint{4.214191in}{1.827943in}}%
\pgfpathlineto{\pgfqpoint{4.216503in}{1.876110in}}%
\pgfpathlineto{\pgfqpoint{4.218816in}{1.877263in}}%
\pgfpathlineto{\pgfqpoint{4.221128in}{1.903660in}}%
\pgfpathlineto{\pgfqpoint{4.223441in}{1.815526in}}%
\pgfpathlineto{\pgfqpoint{4.228065in}{1.923545in}}%
\pgfpathlineto{\pgfqpoint{4.230378in}{1.837692in}}%
\pgfpathlineto{\pgfqpoint{4.232690in}{1.858499in}}%
\pgfpathlineto{\pgfqpoint{4.235002in}{1.864685in}}%
\pgfpathlineto{\pgfqpoint{4.237315in}{1.903167in}}%
\pgfpathlineto{\pgfqpoint{4.241939in}{1.811554in}}%
\pgfpathlineto{\pgfqpoint{4.244252in}{1.921796in}}%
\pgfpathlineto{\pgfqpoint{4.248876in}{1.833719in}}%
\pgfpathlineto{\pgfqpoint{4.251189in}{1.876373in}}%
\pgfpathlineto{\pgfqpoint{4.253501in}{1.875818in}}%
\pgfpathlineto{\pgfqpoint{4.255814in}{1.896202in}}%
\pgfpathlineto{\pgfqpoint{4.258126in}{1.822473in}}%
\pgfpathlineto{\pgfqpoint{4.262751in}{1.926516in}}%
\pgfpathlineto{\pgfqpoint{4.265063in}{1.818903in}}%
\pgfpathlineto{\pgfqpoint{4.269688in}{1.899957in}}%
\pgfpathlineto{\pgfqpoint{4.272000in}{1.863903in}}%
\pgfpathlineto{\pgfqpoint{4.274312in}{1.861815in}}%
\pgfpathlineto{\pgfqpoint{4.276625in}{1.845844in}}%
\pgfpathlineto{\pgfqpoint{4.278937in}{1.910604in}}%
\pgfpathlineto{\pgfqpoint{4.283562in}{1.812121in}}%
\pgfpathlineto{\pgfqpoint{4.285874in}{1.922815in}}%
\pgfpathlineto{\pgfqpoint{4.290499in}{1.823130in}}%
\pgfpathlineto{\pgfqpoint{4.292811in}{1.897193in}}%
\pgfpathlineto{\pgfqpoint{4.295124in}{1.874502in}}%
\pgfpathlineto{\pgfqpoint{4.299748in}{1.854463in}}%
\pgfpathlineto{\pgfqpoint{4.302061in}{1.872286in}}%
\pgfpathlineto{\pgfqpoint{4.304373in}{1.905335in}}%
\pgfpathlineto{\pgfqpoint{4.306685in}{1.821262in}}%
\pgfpathlineto{\pgfqpoint{4.311310in}{1.924719in}}%
\pgfpathlineto{\pgfqpoint{4.313622in}{1.813714in}}%
\pgfpathlineto{\pgfqpoint{4.318247in}{1.918060in}}%
\pgfpathlineto{\pgfqpoint{4.320559in}{1.830957in}}%
\pgfpathlineto{\pgfqpoint{4.325184in}{1.892792in}}%
\pgfpathlineto{\pgfqpoint{4.327497in}{1.860997in}}%
\pgfpathlineto{\pgfqpoint{4.329809in}{1.864295in}}%
\pgfpathlineto{\pgfqpoint{4.332121in}{1.861041in}}%
\pgfpathlineto{\pgfqpoint{4.334434in}{1.890799in}}%
\pgfpathlineto{\pgfqpoint{4.339058in}{1.833673in}}%
\pgfpathlineto{\pgfqpoint{4.341371in}{1.912159in}}%
\pgfpathlineto{\pgfqpoint{4.345995in}{1.816939in}}%
\pgfpathlineto{\pgfqpoint{4.348308in}{1.922418in}}%
\pgfpathlineto{\pgfqpoint{4.352932in}{1.811961in}}%
\pgfpathlineto{\pgfqpoint{4.355245in}{1.922760in}}%
\pgfpathlineto{\pgfqpoint{4.359869in}{1.816305in}}%
\pgfpathlineto{\pgfqpoint{4.362182in}{1.916172in}}%
\pgfpathlineto{\pgfqpoint{4.366807in}{1.826179in}}%
\pgfpathlineto{\pgfqpoint{4.369119in}{1.905934in}}%
\pgfpathlineto{\pgfqpoint{4.373744in}{1.838077in}}%
\pgfpathlineto{\pgfqpoint{4.376056in}{1.894775in}}%
\pgfpathlineto{\pgfqpoint{4.380681in}{1.849503in}}%
\pgfpathlineto{\pgfqpoint{4.382993in}{1.884578in}}%
\pgfpathlineto{\pgfqpoint{4.387618in}{1.859012in}}%
\pgfpathlineto{\pgfqpoint{4.389930in}{1.876431in}}%
\pgfpathlineto{\pgfqpoint{4.392242in}{1.869129in}}%
\pgfpathlineto{\pgfqpoint{4.394555in}{1.865941in}}%
\pgfpathlineto{\pgfqpoint{4.396867in}{1.870845in}}%
\pgfpathlineto{\pgfqpoint{4.399179in}{1.868753in}}%
\pgfpathlineto{\pgfqpoint{4.401492in}{1.870081in}}%
\pgfpathlineto{\pgfqpoint{4.403804in}{1.868004in}}%
\pgfpathlineto{\pgfqpoint{4.406117in}{1.868397in}}%
\pgfpathlineto{\pgfqpoint{4.408429in}{1.871416in}}%
\pgfpathlineto{\pgfqpoint{4.410741in}{1.867949in}}%
\pgfpathlineto{\pgfqpoint{4.413054in}{1.868046in}}%
\pgfpathlineto{\pgfqpoint{4.417678in}{1.870691in}}%
\pgfpathlineto{\pgfqpoint{4.422303in}{1.865723in}}%
\pgfpathlineto{\pgfqpoint{4.424615in}{1.876207in}}%
\pgfpathlineto{\pgfqpoint{4.429240in}{1.858736in}}%
\pgfpathlineto{\pgfqpoint{4.431552in}{1.884341in}}%
\pgfpathlineto{\pgfqpoint{4.436177in}{1.849239in}}%
\pgfpathlineto{\pgfqpoint{4.438490in}{1.894611in}}%
\pgfpathlineto{\pgfqpoint{4.443114in}{1.837931in}}%
\pgfpathlineto{\pgfqpoint{4.445427in}{1.905957in}}%
\pgfpathlineto{\pgfqpoint{4.450051in}{1.826287in}}%
\pgfpathlineto{\pgfqpoint{4.452364in}{1.916509in}}%
\pgfpathlineto{\pgfqpoint{4.456988in}{1.816810in}}%
\pgfpathlineto{\pgfqpoint{4.459301in}{1.923521in}}%
\pgfpathlineto{\pgfqpoint{4.463925in}{1.812956in}}%
\pgfpathlineto{\pgfqpoint{4.466238in}{1.923641in}}%
\pgfpathlineto{\pgfqpoint{4.470862in}{1.818381in}}%
\pgfpathlineto{\pgfqpoint{4.473175in}{1.913738in}}%
\pgfpathlineto{\pgfqpoint{4.477800in}{1.835308in}}%
\pgfpathlineto{\pgfqpoint{4.480112in}{1.892422in}}%
\pgfpathlineto{\pgfqpoint{4.482424in}{1.871634in}}%
\pgfpathlineto{\pgfqpoint{4.484737in}{1.862394in}}%
\pgfpathlineto{\pgfqpoint{4.487049in}{1.862132in}}%
\pgfpathlineto{\pgfqpoint{4.489361in}{1.872752in}}%
\pgfpathlineto{\pgfqpoint{4.491674in}{1.893297in}}%
\pgfpathlineto{\pgfqpoint{4.493986in}{1.830998in}}%
\pgfpathlineto{\pgfqpoint{4.498611in}{1.917306in}}%
\pgfpathlineto{\pgfqpoint{4.500923in}{1.812333in}}%
\pgfpathlineto{\pgfqpoint{4.505548in}{1.922744in}}%
\pgfpathlineto{\pgfqpoint{4.507860in}{1.818876in}}%
\pgfpathlineto{\pgfqpoint{4.512485in}{1.902850in}}%
\pgfpathlineto{\pgfqpoint{4.514797in}{1.852315in}}%
\pgfpathlineto{\pgfqpoint{4.517110in}{1.862855in}}%
\pgfpathlineto{\pgfqpoint{4.519422in}{1.862766in}}%
\pgfpathlineto{\pgfqpoint{4.521734in}{1.896702in}}%
\pgfpathlineto{\pgfqpoint{4.526359in}{1.823630in}}%
\pgfpathlineto{\pgfqpoint{4.528671in}{1.924577in}}%
\pgfpathlineto{\pgfqpoint{4.533296in}{1.814796in}}%
\pgfpathlineto{\pgfqpoint{4.535608in}{1.913695in}}%
\pgfpathlineto{\pgfqpoint{4.540233in}{1.848662in}}%
\pgfpathlineto{\pgfqpoint{4.547170in}{1.900342in}}%
\pgfpathlineto{\pgfqpoint{4.549483in}{1.817775in}}%
\pgfpathlineto{\pgfqpoint{4.554107in}{1.923814in}}%
\pgfpathlineto{\pgfqpoint{4.556420in}{1.818905in}}%
\pgfpathlineto{\pgfqpoint{4.561044in}{1.892736in}}%
\pgfpathlineto{\pgfqpoint{4.567981in}{1.833366in}}%
\pgfpathlineto{\pgfqpoint{4.570294in}{1.923825in}}%
\pgfpathlineto{\pgfqpoint{4.574918in}{1.815205in}}%
\pgfpathlineto{\pgfqpoint{4.577231in}{1.907241in}}%
\pgfpathlineto{\pgfqpoint{4.579543in}{1.875679in}}%
\pgfpathlineto{\pgfqpoint{4.581855in}{1.867521in}}%
\pgfpathlineto{\pgfqpoint{4.584168in}{1.838512in}}%
\pgfpathlineto{\pgfqpoint{4.588793in}{1.921427in}}%
\pgfpathlineto{\pgfqpoint{4.591105in}{1.811440in}}%
\pgfpathlineto{\pgfqpoint{4.595730in}{1.899233in}}%
\pgfpathlineto{\pgfqpoint{4.602667in}{1.828076in}}%
\pgfpathlineto{\pgfqpoint{4.604979in}{1.927783in}}%
\pgfpathlineto{\pgfqpoint{4.609604in}{1.824893in}}%
\pgfpathlineto{\pgfqpoint{4.611916in}{1.882633in}}%
\pgfpathlineto{\pgfqpoint{4.614228in}{1.879952in}}%
\pgfpathlineto{\pgfqpoint{4.616541in}{1.900280in}}%
\pgfpathlineto{\pgfqpoint{4.618853in}{1.810597in}}%
\pgfpathlineto{\pgfqpoint{4.623478in}{1.916904in}}%
\pgfpathlineto{\pgfqpoint{4.625790in}{1.851723in}}%
\pgfpathlineto{\pgfqpoint{4.628103in}{1.855910in}}%
\pgfpathlineto{\pgfqpoint{4.630415in}{1.839337in}}%
\pgfpathlineto{\pgfqpoint{4.632727in}{1.927027in}}%
\pgfpathlineto{\pgfqpoint{4.637352in}{1.822488in}}%
\pgfpathlineto{\pgfqpoint{4.639664in}{1.880759in}}%
\pgfpathlineto{\pgfqpoint{4.641977in}{1.881843in}}%
\pgfpathlineto{\pgfqpoint{4.644289in}{1.906861in}}%
\pgfpathlineto{\pgfqpoint{4.646601in}{1.807072in}}%
\pgfpathlineto{\pgfqpoint{4.651226in}{1.904960in}}%
\pgfpathlineto{\pgfqpoint{4.658163in}{1.817588in}}%
\pgfpathlineto{\pgfqpoint{4.660476in}{1.926153in}}%
\pgfpathlineto{\pgfqpoint{4.662788in}{1.876168in}}%
\pgfpathlineto{\pgfqpoint{4.667413in}{1.831298in}}%
\pgfpathlineto{\pgfqpoint{4.672037in}{1.925538in}}%
\pgfpathlineto{\pgfqpoint{4.674350in}{1.835243in}}%
\pgfpathlineto{\pgfqpoint{4.676662in}{1.853506in}}%
\pgfpathlineto{\pgfqpoint{4.678974in}{1.843459in}}%
\pgfpathlineto{\pgfqpoint{4.681287in}{1.930020in}}%
\pgfpathlineto{\pgfqpoint{4.685911in}{1.832119in}}%
\pgfpathlineto{\pgfqpoint{4.690536in}{1.883336in}}%
\pgfpathlineto{\pgfqpoint{4.692848in}{1.924936in}}%
\pgfpathlineto{\pgfqpoint{4.695161in}{1.821664in}}%
\pgfpathlineto{\pgfqpoint{4.697473in}{1.853911in}}%
\pgfpathlineto{\pgfqpoint{4.699786in}{1.853347in}}%
\pgfpathlineto{\pgfqpoint{4.702098in}{1.929319in}}%
\pgfpathlineto{\pgfqpoint{4.706723in}{1.830199in}}%
\pgfpathlineto{\pgfqpoint{4.711347in}{1.883972in}}%
\pgfpathlineto{\pgfqpoint{4.713660in}{1.926707in}}%
\pgfpathlineto{\pgfqpoint{4.715972in}{1.830069in}}%
\pgfpathlineto{\pgfqpoint{4.718284in}{1.850581in}}%
\pgfpathlineto{\pgfqpoint{4.720597in}{1.838534in}}%
\pgfpathlineto{\pgfqpoint{4.722909in}{1.933228in}}%
\pgfpathlineto{\pgfqpoint{4.725221in}{1.874964in}}%
\pgfpathlineto{\pgfqpoint{4.729846in}{1.823762in}}%
\pgfpathlineto{\pgfqpoint{4.734471in}{1.922036in}}%
\pgfpathlineto{\pgfqpoint{4.736783in}{1.869029in}}%
\pgfpathlineto{\pgfqpoint{4.739096in}{1.848791in}}%
\pgfpathlineto{\pgfqpoint{4.741408in}{1.811376in}}%
\pgfpathlineto{\pgfqpoint{4.743720in}{1.914222in}}%
\pgfpathlineto{\pgfqpoint{4.746033in}{1.887874in}}%
\pgfpathlineto{\pgfqpoint{4.748345in}{1.901238in}}%
\pgfpathlineto{\pgfqpoint{4.750657in}{1.802395in}}%
\pgfpathlineto{\pgfqpoint{4.752970in}{1.857892in}}%
\pgfpathlineto{\pgfqpoint{4.755282in}{1.875530in}}%
\pgfpathlineto{\pgfqpoint{4.757594in}{1.927696in}}%
\pgfpathlineto{\pgfqpoint{4.762219in}{1.829929in}}%
\pgfpathlineto{\pgfqpoint{4.764531in}{1.837765in}}%
\pgfpathlineto{\pgfqpoint{4.769156in}{1.926516in}}%
\pgfpathlineto{\pgfqpoint{4.771469in}{1.867435in}}%
\pgfpathlineto{\pgfqpoint{4.773781in}{1.846418in}}%
\pgfpathlineto{\pgfqpoint{4.776093in}{1.808214in}}%
\pgfpathlineto{\pgfqpoint{4.778406in}{1.903785in}}%
\pgfpathlineto{\pgfqpoint{4.780718in}{1.892369in}}%
\pgfpathlineto{\pgfqpoint{4.783030in}{1.917843in}}%
\pgfpathlineto{\pgfqpoint{4.785343in}{1.812701in}}%
\pgfpathlineto{\pgfqpoint{4.787655in}{1.846747in}}%
\pgfpathlineto{\pgfqpoint{4.789967in}{1.839093in}}%
\pgfpathlineto{\pgfqpoint{4.792280in}{1.935961in}}%
\pgfpathlineto{\pgfqpoint{4.794592in}{1.884689in}}%
\pgfpathlineto{\pgfqpoint{4.796904in}{1.879467in}}%
\pgfpathlineto{\pgfqpoint{4.799217in}{1.798694in}}%
\pgfpathlineto{\pgfqpoint{4.801529in}{1.858991in}}%
\pgfpathlineto{\pgfqpoint{4.803841in}{1.879212in}}%
\pgfpathlineto{\pgfqpoint{4.806154in}{1.934044in}}%
\pgfpathlineto{\pgfqpoint{4.808466in}{1.871463in}}%
\pgfpathlineto{\pgfqpoint{4.813091in}{1.810817in}}%
\pgfpathlineto{\pgfqpoint{4.817716in}{1.905900in}}%
\pgfpathlineto{\pgfqpoint{4.820028in}{1.919544in}}%
\pgfpathlineto{\pgfqpoint{4.824653in}{1.824676in}}%
\pgfpathlineto{\pgfqpoint{4.826965in}{1.826171in}}%
\pgfpathlineto{\pgfqpoint{4.831590in}{1.918413in}}%
\pgfpathlineto{\pgfqpoint{4.833902in}{1.908742in}}%
\pgfpathlineto{\pgfqpoint{4.838527in}{1.816874in}}%
\pgfpathlineto{\pgfqpoint{4.840839in}{1.832534in}}%
\pgfpathlineto{\pgfqpoint{4.845464in}{1.921537in}}%
\pgfpathlineto{\pgfqpoint{4.847776in}{1.908764in}}%
\pgfpathlineto{\pgfqpoint{4.852401in}{1.817765in}}%
\pgfpathlineto{\pgfqpoint{4.854713in}{1.826189in}}%
\pgfpathlineto{\pgfqpoint{4.859338in}{1.916583in}}%
\pgfpathlineto{\pgfqpoint{4.861650in}{1.920513in}}%
\pgfpathlineto{\pgfqpoint{4.866275in}{1.828938in}}%
\pgfpathlineto{\pgfqpoint{4.868587in}{1.808921in}}%
\pgfpathlineto{\pgfqpoint{4.870900in}{1.869939in}}%
\pgfpathlineto{\pgfqpoint{4.875524in}{1.938414in}}%
\pgfpathlineto{\pgfqpoint{4.877837in}{1.875615in}}%
\pgfpathlineto{\pgfqpoint{4.880149in}{1.856024in}}%
\pgfpathlineto{\pgfqpoint{4.882462in}{1.791692in}}%
\pgfpathlineto{\pgfqpoint{4.884774in}{1.852171in}}%
\pgfpathlineto{\pgfqpoint{4.887086in}{1.864068in}}%
\pgfpathlineto{\pgfqpoint{4.889399in}{1.946074in}}%
\pgfpathlineto{\pgfqpoint{4.891711in}{1.895148in}}%
\pgfpathlineto{\pgfqpoint{4.894023in}{1.899138in}}%
\pgfpathlineto{\pgfqpoint{4.896336in}{1.800072in}}%
\pgfpathlineto{\pgfqpoint{4.898648in}{1.833521in}}%
\pgfpathlineto{\pgfqpoint{4.900960in}{1.817319in}}%
\pgfpathlineto{\pgfqpoint{4.903273in}{1.917133in}}%
\pgfpathlineto{\pgfqpoint{4.905585in}{1.906499in}}%
\pgfpathlineto{\pgfqpoint{4.907897in}{1.936460in}}%
\pgfpathlineto{\pgfqpoint{4.910210in}{1.858077in}}%
\pgfpathlineto{\pgfqpoint{4.912522in}{1.835146in}}%
\pgfpathlineto{\pgfqpoint{4.914834in}{1.798044in}}%
\pgfpathlineto{\pgfqpoint{4.921772in}{1.922442in}}%
\pgfpathlineto{\pgfqpoint{4.924084in}{1.936040in}}%
\pgfpathlineto{\pgfqpoint{4.926396in}{1.873514in}}%
\pgfpathlineto{\pgfqpoint{4.928709in}{1.853857in}}%
\pgfpathlineto{\pgfqpoint{4.931021in}{1.784217in}}%
\pgfpathlineto{\pgfqpoint{4.933333in}{1.839815in}}%
\pgfpathlineto{\pgfqpoint{4.935646in}{1.839858in}}%
\pgfpathlineto{\pgfqpoint{4.937958in}{1.936772in}}%
\pgfpathlineto{\pgfqpoint{4.940270in}{1.911593in}}%
\pgfpathlineto{\pgfqpoint{4.942583in}{1.936273in}}%
\pgfpathlineto{\pgfqpoint{4.944895in}{1.854623in}}%
\pgfpathlineto{\pgfqpoint{4.947207in}{1.831528in}}%
\pgfpathlineto{\pgfqpoint{4.949520in}{1.795630in}}%
\pgfpathlineto{\pgfqpoint{4.951832in}{1.822675in}}%
\pgfpathlineto{\pgfqpoint{4.954145in}{1.877528in}}%
\pgfpathlineto{\pgfqpoint{4.956457in}{1.905775in}}%
\pgfpathlineto{\pgfqpoint{4.958769in}{1.955661in}}%
\pgfpathlineto{\pgfqpoint{4.961082in}{1.897683in}}%
\pgfpathlineto{\pgfqpoint{4.963394in}{1.898295in}}%
\pgfpathlineto{\pgfqpoint{4.965706in}{1.801183in}}%
\pgfpathlineto{\pgfqpoint{4.968019in}{1.819595in}}%
\pgfpathlineto{\pgfqpoint{4.970331in}{1.793054in}}%
\pgfpathlineto{\pgfqpoint{4.972643in}{1.866636in}}%
\pgfpathlineto{\pgfqpoint{4.977268in}{1.933661in}}%
\pgfpathlineto{\pgfqpoint{4.979580in}{1.948199in}}%
\pgfpathlineto{\pgfqpoint{4.981893in}{1.885740in}}%
\pgfpathlineto{\pgfqpoint{4.984205in}{1.876241in}}%
\pgfpathlineto{\pgfqpoint{4.986517in}{1.785827in}}%
\pgfpathlineto{\pgfqpoint{4.988830in}{1.816407in}}%
\pgfpathlineto{\pgfqpoint{4.991142in}{1.794579in}}%
\pgfpathlineto{\pgfqpoint{4.993455in}{1.873709in}}%
\pgfpathlineto{\pgfqpoint{5.000392in}{1.955430in}}%
\pgfpathlineto{\pgfqpoint{5.002704in}{1.893752in}}%
\pgfpathlineto{\pgfqpoint{5.005016in}{1.889362in}}%
\pgfpathlineto{\pgfqpoint{5.007329in}{1.795102in}}%
\pgfpathlineto{\pgfqpoint{5.009641in}{1.810423in}}%
\pgfpathlineto{\pgfqpoint{5.011953in}{1.782452in}}%
\pgfpathlineto{\pgfqpoint{5.016578in}{1.885253in}}%
\pgfpathlineto{\pgfqpoint{5.018890in}{1.911544in}}%
\pgfpathlineto{\pgfqpoint{5.021203in}{1.970027in}}%
\pgfpathlineto{\pgfqpoint{5.023515in}{1.921426in}}%
\pgfpathlineto{\pgfqpoint{5.025828in}{1.935530in}}%
\pgfpathlineto{\pgfqpoint{5.028140in}{1.853215in}}%
\pgfpathlineto{\pgfqpoint{5.035077in}{1.770901in}}%
\pgfpathlineto{\pgfqpoint{5.037389in}{1.831727in}}%
\pgfpathlineto{\pgfqpoint{5.039702in}{1.832887in}}%
\pgfpathlineto{\pgfqpoint{5.042014in}{1.926114in}}%
\pgfpathlineto{\pgfqpoint{5.044326in}{1.929011in}}%
\pgfpathlineto{\pgfqpoint{5.046639in}{1.960099in}}%
\pgfpathlineto{\pgfqpoint{5.048951in}{1.958936in}}%
\pgfpathlineto{\pgfqpoint{5.051263in}{1.897206in}}%
\pgfpathlineto{\pgfqpoint{5.053576in}{1.893682in}}%
\pgfpathlineto{\pgfqpoint{5.055888in}{1.801337in}}%
\pgfpathlineto{\pgfqpoint{5.058200in}{1.800642in}}%
\pgfpathlineto{\pgfqpoint{5.060513in}{1.770499in}}%
\pgfpathlineto{\pgfqpoint{5.062825in}{1.780464in}}%
\pgfpathlineto{\pgfqpoint{5.065138in}{1.841862in}}%
\pgfpathlineto{\pgfqpoint{5.067450in}{1.848966in}}%
\pgfpathlineto{\pgfqpoint{5.069762in}{1.939773in}}%
\pgfpathlineto{\pgfqpoint{5.072075in}{1.938982in}}%
\pgfpathlineto{\pgfqpoint{5.074387in}{1.968698in}}%
\pgfpathlineto{\pgfqpoint{5.076699in}{1.974378in}}%
\pgfpathlineto{\pgfqpoint{5.079012in}{1.914135in}}%
\pgfpathlineto{\pgfqpoint{5.081324in}{1.917112in}}%
\pgfpathlineto{\pgfqpoint{5.083636in}{1.830163in}}%
\pgfpathlineto{\pgfqpoint{5.088261in}{1.779428in}}%
\pgfpathlineto{\pgfqpoint{5.090573in}{1.743440in}}%
\pgfpathlineto{\pgfqpoint{5.092886in}{1.797419in}}%
\pgfpathlineto{\pgfqpoint{5.095198in}{1.786343in}}%
\pgfpathlineto{\pgfqpoint{5.097510in}{1.862419in}}%
\pgfpathlineto{\pgfqpoint{5.104448in}{1.992524in}}%
\pgfpathlineto{\pgfqpoint{5.106760in}{1.963467in}}%
\pgfpathlineto{\pgfqpoint{5.109072in}{1.988010in}}%
\pgfpathlineto{\pgfqpoint{5.111385in}{1.966229in}}%
\pgfpathlineto{\pgfqpoint{5.113697in}{1.905554in}}%
\pgfpathlineto{\pgfqpoint{5.116009in}{1.904141in}}%
\pgfpathlineto{\pgfqpoint{5.118322in}{1.817960in}}%
\pgfpathlineto{\pgfqpoint{5.122946in}{1.770118in}}%
\pgfpathlineto{\pgfqpoint{5.125259in}{1.721544in}}%
\pgfpathlineto{\pgfqpoint{5.127571in}{1.766076in}}%
\pgfpathlineto{\pgfqpoint{5.129883in}{1.747894in}}%
\pgfpathlineto{\pgfqpoint{5.134508in}{1.852529in}}%
\pgfpathlineto{\pgfqpoint{5.136821in}{1.861833in}}%
\pgfpathlineto{\pgfqpoint{5.139133in}{1.947969in}}%
\pgfpathlineto{\pgfqpoint{5.141445in}{1.958550in}}%
\pgfpathlineto{\pgfqpoint{5.143758in}{1.983826in}}%
\pgfpathlineto{\pgfqpoint{5.146070in}{2.031017in}}%
\pgfpathlineto{\pgfqpoint{5.148382in}{1.992478in}}%
\pgfpathlineto{\pgfqpoint{5.150695in}{2.013872in}}%
\pgfpathlineto{\pgfqpoint{5.153007in}{1.993169in}}%
\pgfpathlineto{\pgfqpoint{5.155319in}{1.933469in}}%
\pgfpathlineto{\pgfqpoint{5.157632in}{1.937690in}}%
\pgfpathlineto{\pgfqpoint{5.162256in}{1.821401in}}%
\pgfpathlineto{\pgfqpoint{5.164569in}{1.806215in}}%
\pgfpathlineto{\pgfqpoint{5.166881in}{1.726219in}}%
\pgfpathlineto{\pgfqpoint{5.169193in}{1.729752in}}%
\pgfpathlineto{\pgfqpoint{5.173818in}{1.673935in}}%
\pgfpathlineto{\pgfqpoint{5.176131in}{1.721373in}}%
\pgfpathlineto{\pgfqpoint{5.178443in}{1.704205in}}%
\pgfpathlineto{\pgfqpoint{5.180755in}{1.740692in}}%
\pgfpathlineto{\pgfqpoint{5.183068in}{1.798955in}}%
\pgfpathlineto{\pgfqpoint{5.185380in}{1.798256in}}%
\pgfpathlineto{\pgfqpoint{5.187692in}{1.874597in}}%
\pgfpathlineto{\pgfqpoint{5.190005in}{1.913791in}}%
\pgfpathlineto{\pgfqpoint{5.192317in}{1.927229in}}%
\pgfpathlineto{\pgfqpoint{5.194629in}{2.006993in}}%
\pgfpathlineto{\pgfqpoint{5.196942in}{2.016175in}}%
\pgfpathlineto{\pgfqpoint{5.199254in}{2.037133in}}%
\pgfpathlineto{\pgfqpoint{5.201566in}{2.097300in}}%
\pgfpathlineto{\pgfqpoint{5.203879in}{2.078306in}}%
\pgfpathlineto{\pgfqpoint{5.208503in}{2.131636in}}%
\pgfpathlineto{\pgfqpoint{5.210816in}{2.091682in}}%
\pgfpathlineto{\pgfqpoint{5.213128in}{2.111565in}}%
\pgfpathlineto{\pgfqpoint{5.215441in}{2.112047in}}%
\pgfpathlineto{\pgfqpoint{5.217753in}{2.059560in}}%
\pgfpathlineto{\pgfqpoint{5.220065in}{2.074582in}}%
\pgfpathlineto{\pgfqpoint{5.222378in}{2.048375in}}%
\pgfpathlineto{\pgfqpoint{5.224690in}{1.990990in}}%
\pgfpathlineto{\pgfqpoint{5.227002in}{2.000271in}}%
\pgfpathlineto{\pgfqpoint{5.231627in}{1.896897in}}%
\pgfpathlineto{\pgfqpoint{5.233939in}{1.900493in}}%
\pgfpathlineto{\pgfqpoint{5.238564in}{1.787768in}}%
\pgfpathlineto{\pgfqpoint{5.240876in}{1.786281in}}%
\pgfpathlineto{\pgfqpoint{5.245501in}{1.672461in}}%
\pgfpathlineto{\pgfqpoint{5.247814in}{1.666748in}}%
\pgfpathlineto{\pgfqpoint{5.250126in}{1.591730in}}%
\pgfpathlineto{\pgfqpoint{5.252438in}{1.557767in}}%
\pgfpathlineto{\pgfqpoint{5.254751in}{1.548743in}}%
\pgfpathlineto{\pgfqpoint{5.257063in}{1.473003in}}%
\pgfpathlineto{\pgfqpoint{5.259375in}{1.448425in}}%
\pgfpathlineto{\pgfqpoint{5.261688in}{1.436947in}}%
\pgfpathlineto{\pgfqpoint{5.264000in}{1.362966in}}%
\pgfpathlineto{\pgfqpoint{5.268625in}{1.334196in}}%
\pgfpathlineto{\pgfqpoint{5.270937in}{1.263540in}}%
\pgfpathlineto{\pgfqpoint{5.273249in}{1.256199in}}%
\pgfpathlineto{\pgfqpoint{5.275562in}{1.241887in}}%
\pgfpathlineto{\pgfqpoint{5.277874in}{1.175400in}}%
\pgfpathlineto{\pgfqpoint{5.280186in}{1.175338in}}%
\pgfpathlineto{\pgfqpoint{5.282499in}{1.160374in}}%
\pgfpathlineto{\pgfqpoint{5.284811in}{1.098367in}}%
\pgfpathlineto{\pgfqpoint{5.287124in}{1.104581in}}%
\pgfpathlineto{\pgfqpoint{5.289436in}{1.089312in}}%
\pgfpathlineto{\pgfqpoint{5.291748in}{1.031734in}}%
\pgfpathlineto{\pgfqpoint{5.294061in}{1.043258in}}%
\pgfpathlineto{\pgfqpoint{5.296373in}{1.027935in}}%
\pgfpathlineto{\pgfqpoint{5.298685in}{0.974504in}}%
\pgfpathlineto{\pgfqpoint{5.300998in}{0.990456in}}%
\pgfpathlineto{\pgfqpoint{5.303310in}{0.975256in}}%
\pgfpathlineto{\pgfqpoint{5.305622in}{0.925561in}}%
\pgfpathlineto{\pgfqpoint{5.307935in}{0.945169in}}%
\pgfpathlineto{\pgfqpoint{5.310247in}{0.930215in}}%
\pgfpathlineto{\pgfqpoint{5.312559in}{0.883791in}}%
\pgfpathlineto{\pgfqpoint{5.314872in}{0.906394in}}%
\pgfpathlineto{\pgfqpoint{5.317184in}{0.891771in}}%
\pgfpathlineto{\pgfqpoint{5.319497in}{0.848141in}}%
\pgfpathlineto{\pgfqpoint{5.321809in}{0.873187in}}%
\pgfpathlineto{\pgfqpoint{5.324121in}{0.858953in}}%
\pgfpathlineto{\pgfqpoint{5.326434in}{0.817666in}}%
\pgfpathlineto{\pgfqpoint{5.328746in}{0.844697in}}%
\pgfpathlineto{\pgfqpoint{5.331058in}{0.830896in}}%
\pgfpathlineto{\pgfqpoint{5.333371in}{0.791535in}}%
\pgfpathlineto{\pgfqpoint{5.335683in}{0.820177in}}%
\pgfpathlineto{\pgfqpoint{5.337995in}{0.806840in}}%
\pgfpathlineto{\pgfqpoint{5.340308in}{0.769037in}}%
\pgfpathlineto{\pgfqpoint{5.342620in}{0.798987in}}%
\pgfpathlineto{\pgfqpoint{5.344932in}{0.786142in}}%
\pgfpathlineto{\pgfqpoint{5.347245in}{0.749573in}}%
\pgfpathlineto{\pgfqpoint{5.349557in}{0.780583in}}%
\pgfpathlineto{\pgfqpoint{5.351869in}{0.768253in}}%
\pgfpathlineto{\pgfqpoint{5.354182in}{0.732640in}}%
\pgfpathlineto{\pgfqpoint{5.356494in}{0.764511in}}%
\pgfpathlineto{\pgfqpoint{5.358807in}{0.752720in}}%
\pgfpathlineto{\pgfqpoint{5.361119in}{0.717822in}}%
\pgfpathlineto{\pgfqpoint{5.363431in}{0.750393in}}%
\pgfpathlineto{\pgfqpoint{5.365744in}{0.739164in}}%
\pgfpathlineto{\pgfqpoint{5.368056in}{0.704778in}}%
\pgfpathlineto{\pgfqpoint{5.370368in}{0.737914in}}%
\pgfpathlineto{\pgfqpoint{5.372681in}{0.727275in}}%
\pgfpathlineto{\pgfqpoint{5.374993in}{0.693223in}}%
\pgfpathlineto{\pgfqpoint{5.377305in}{0.726818in}}%
\pgfpathlineto{\pgfqpoint{5.379618in}{0.716795in}}%
\pgfpathlineto{\pgfqpoint{5.381930in}{0.682926in}}%
\pgfpathlineto{\pgfqpoint{5.384242in}{0.716889in}}%
\pgfpathlineto{\pgfqpoint{5.386555in}{0.707513in}}%
\pgfpathlineto{\pgfqpoint{5.388867in}{0.673696in}}%
\pgfpathlineto{\pgfqpoint{5.391179in}{0.707950in}}%
\pgfpathlineto{\pgfqpoint{5.393492in}{0.699255in}}%
\pgfpathlineto{\pgfqpoint{5.395804in}{0.665375in}}%
\pgfpathlineto{\pgfqpoint{5.398117in}{0.699856in}}%
\pgfpathlineto{\pgfqpoint{5.400429in}{0.691878in}}%
\pgfpathlineto{\pgfqpoint{5.402741in}{0.657835in}}%
\pgfpathlineto{\pgfqpoint{5.405054in}{0.692484in}}%
\pgfpathlineto{\pgfqpoint{5.407366in}{0.685265in}}%
\pgfpathlineto{\pgfqpoint{5.409678in}{0.650967in}}%
\pgfpathlineto{\pgfqpoint{5.411991in}{0.685732in}}%
\pgfpathlineto{\pgfqpoint{5.414303in}{0.679316in}}%
\pgfpathlineto{\pgfqpoint{5.416615in}{0.644685in}}%
\pgfpathlineto{\pgfqpoint{5.418928in}{0.679514in}}%
\pgfpathlineto{\pgfqpoint{5.421240in}{0.673953in}}%
\pgfpathlineto{\pgfqpoint{5.423552in}{0.638916in}}%
\pgfpathlineto{\pgfqpoint{5.425865in}{0.673759in}}%
\pgfpathlineto{\pgfqpoint{5.428177in}{0.669106in}}%
\pgfpathlineto{\pgfqpoint{5.430490in}{0.633602in}}%
\pgfpathlineto{\pgfqpoint{5.432802in}{0.668406in}}%
\pgfpathlineto{\pgfqpoint{5.435114in}{0.664722in}}%
\pgfpathlineto{\pgfqpoint{5.437427in}{0.628699in}}%
\pgfpathlineto{\pgfqpoint{5.439739in}{0.663403in}}%
\pgfpathlineto{\pgfqpoint{5.442051in}{0.660755in}}%
\pgfpathlineto{\pgfqpoint{5.444364in}{0.624174in}}%
\pgfpathlineto{\pgfqpoint{5.446676in}{0.658704in}}%
\pgfpathlineto{\pgfqpoint{5.448988in}{0.657167in}}%
\pgfpathlineto{\pgfqpoint{5.451301in}{0.620009in}}%
\pgfpathlineto{\pgfqpoint{5.453613in}{0.654273in}}%
\pgfpathlineto{\pgfqpoint{5.455925in}{0.653927in}}%
\pgfpathlineto{\pgfqpoint{5.458238in}{0.616201in}}%
\pgfpathlineto{\pgfqpoint{5.460550in}{0.650076in}}%
\pgfpathlineto{\pgfqpoint{5.462862in}{0.651011in}}%
\pgfpathlineto{\pgfqpoint{5.465175in}{0.612771in}}%
\pgfpathlineto{\pgfqpoint{5.467487in}{0.646085in}}%
\pgfpathlineto{\pgfqpoint{5.469800in}{0.648398in}}%
\pgfpathlineto{\pgfqpoint{5.472112in}{0.609771in}}%
\pgfpathlineto{\pgfqpoint{5.474424in}{0.642274in}}%
\pgfpathlineto{\pgfqpoint{5.476737in}{0.646073in}}%
\pgfpathlineto{\pgfqpoint{5.479049in}{0.607295in}}%
\pgfpathlineto{\pgfqpoint{5.481361in}{0.638624in}}%
\pgfpathlineto{\pgfqpoint{5.483674in}{0.644025in}}%
\pgfpathlineto{\pgfqpoint{5.485986in}{0.605487in}}%
\pgfpathlineto{\pgfqpoint{5.488298in}{0.635115in}}%
\pgfpathlineto{\pgfqpoint{5.490611in}{0.642244in}}%
\pgfpathlineto{\pgfqpoint{5.492923in}{0.604507in}}%
\pgfpathlineto{\pgfqpoint{5.495235in}{0.631732in}}%
\pgfpathlineto{\pgfqpoint{5.497548in}{0.640724in}}%
\pgfpathlineto{\pgfqpoint{5.499860in}{0.604453in}}%
\pgfpathlineto{\pgfqpoint{5.502172in}{0.628464in}}%
\pgfpathlineto{\pgfqpoint{5.504485in}{0.639461in}}%
\pgfpathlineto{\pgfqpoint{5.506797in}{0.605272in}}%
\pgfpathlineto{\pgfqpoint{5.511422in}{0.638453in}}%
\pgfpathlineto{\pgfqpoint{5.513734in}{0.606786in}}%
\pgfpathlineto{\pgfqpoint{5.518359in}{0.637697in}}%
\pgfpathlineto{\pgfqpoint{5.520671in}{0.608792in}}%
\pgfpathlineto{\pgfqpoint{5.522984in}{0.619276in}}%
\pgfpathlineto{\pgfqpoint{5.525296in}{0.637193in}}%
\pgfpathlineto{\pgfqpoint{5.527608in}{0.611130in}}%
\pgfpathlineto{\pgfqpoint{5.529921in}{0.616422in}}%
\pgfpathlineto{\pgfqpoint{5.532233in}{0.636942in}}%
\pgfpathlineto{\pgfqpoint{5.534545in}{0.613694in}}%
\pgfpathlineto{\pgfqpoint{5.534545in}{0.613694in}}%
\pgfusepath{stroke}%
\end{pgfscope}%
\begin{pgfscope}%
\pgfpathrectangle{\pgfqpoint{0.800000in}{0.528000in}}{\pgfqpoint{4.960000in}{1.680000in}} %
\pgfusepath{clip}%
\pgfsetbuttcap%
\pgfsetroundjoin%
\pgfsetlinewidth{1.003750pt}%
\definecolor{currentstroke}{rgb}{1.000000,0.000000,0.000000}%
\pgfsetstrokecolor{currentstroke}%
\pgfsetdash{}{0pt}%
\pgfsys@defobject{currentmarker}{\pgfqpoint{-0.034722in}{-0.034722in}}{\pgfqpoint{0.034722in}{0.034722in}}{%
\pgfpathmoveto{\pgfqpoint{0.000000in}{-0.034722in}}%
\pgfpathcurveto{\pgfqpoint{0.009208in}{-0.034722in}}{\pgfqpoint{0.018041in}{-0.031064in}}{\pgfqpoint{0.024552in}{-0.024552in}}%
\pgfpathcurveto{\pgfqpoint{0.031064in}{-0.018041in}}{\pgfqpoint{0.034722in}{-0.009208in}}{\pgfqpoint{0.034722in}{0.000000in}}%
\pgfpathcurveto{\pgfqpoint{0.034722in}{0.009208in}}{\pgfqpoint{0.031064in}{0.018041in}}{\pgfqpoint{0.024552in}{0.024552in}}%
\pgfpathcurveto{\pgfqpoint{0.018041in}{0.031064in}}{\pgfqpoint{0.009208in}{0.034722in}}{\pgfqpoint{0.000000in}{0.034722in}}%
\pgfpathcurveto{\pgfqpoint{-0.009208in}{0.034722in}}{\pgfqpoint{-0.018041in}{0.031064in}}{\pgfqpoint{-0.024552in}{0.024552in}}%
\pgfpathcurveto{\pgfqpoint{-0.031064in}{0.018041in}}{\pgfqpoint{-0.034722in}{0.009208in}}{\pgfqpoint{-0.034722in}{0.000000in}}%
\pgfpathcurveto{\pgfqpoint{-0.034722in}{-0.009208in}}{\pgfqpoint{-0.031064in}{-0.018041in}}{\pgfqpoint{-0.024552in}{-0.024552in}}%
\pgfpathcurveto{\pgfqpoint{-0.018041in}{-0.031064in}}{\pgfqpoint{-0.009208in}{-0.034722in}}{\pgfqpoint{0.000000in}{-0.034722in}}%
\pgfpathclose%
\pgfusepath{stroke}%
}%
\begin{pgfscope}%
\pgfsys@transformshift{3.000205in}{1.502204in}%
\pgfsys@useobject{currentmarker}{}%
\end{pgfscope}%
\begin{pgfscope}%
\pgfsys@transformshift{1.305249in}{1.548743in}%
\pgfsys@useobject{currentmarker}{}%
\end{pgfscope}%
\end{pgfscope}%
\begin{pgfscope}%
\pgfpathrectangle{\pgfqpoint{0.800000in}{0.528000in}}{\pgfqpoint{4.960000in}{1.680000in}} %
\pgfusepath{clip}%
\pgfsetbuttcap%
\pgfsetroundjoin%
\pgfsetlinewidth{1.003750pt}%
\definecolor{currentstroke}{rgb}{1.000000,0.000000,0.000000}%
\pgfsetstrokecolor{currentstroke}%
\pgfsetdash{}{0pt}%
\pgfsys@defobject{currentmarker}{\pgfqpoint{-0.034722in}{-0.034722in}}{\pgfqpoint{0.034722in}{0.034722in}}{%
\pgfpathmoveto{\pgfqpoint{0.000000in}{-0.034722in}}%
\pgfpathcurveto{\pgfqpoint{0.009208in}{-0.034722in}}{\pgfqpoint{0.018041in}{-0.031064in}}{\pgfqpoint{0.024552in}{-0.024552in}}%
\pgfpathcurveto{\pgfqpoint{0.031064in}{-0.018041in}}{\pgfqpoint{0.034722in}{-0.009208in}}{\pgfqpoint{0.034722in}{0.000000in}}%
\pgfpathcurveto{\pgfqpoint{0.034722in}{0.009208in}}{\pgfqpoint{0.031064in}{0.018041in}}{\pgfqpoint{0.024552in}{0.024552in}}%
\pgfpathcurveto{\pgfqpoint{0.018041in}{0.031064in}}{\pgfqpoint{0.009208in}{0.034722in}}{\pgfqpoint{0.000000in}{0.034722in}}%
\pgfpathcurveto{\pgfqpoint{-0.009208in}{0.034722in}}{\pgfqpoint{-0.018041in}{0.031064in}}{\pgfqpoint{-0.024552in}{0.024552in}}%
\pgfpathcurveto{\pgfqpoint{-0.031064in}{0.018041in}}{\pgfqpoint{-0.034722in}{0.009208in}}{\pgfqpoint{-0.034722in}{0.000000in}}%
\pgfpathcurveto{\pgfqpoint{-0.034722in}{-0.009208in}}{\pgfqpoint{-0.031064in}{-0.018041in}}{\pgfqpoint{-0.024552in}{-0.024552in}}%
\pgfpathcurveto{\pgfqpoint{-0.018041in}{-0.031064in}}{\pgfqpoint{-0.009208in}{-0.034722in}}{\pgfqpoint{0.000000in}{-0.034722in}}%
\pgfpathclose%
\pgfusepath{stroke}%
}%
\begin{pgfscope}%
\pgfsys@transformshift{5.254751in}{1.548743in}%
\pgfsys@useobject{currentmarker}{}%
\end{pgfscope}%
\begin{pgfscope}%
\pgfsys@transformshift{3.559795in}{1.502204in}%
\pgfsys@useobject{currentmarker}{}%
\end{pgfscope}%
\end{pgfscope}%
\begin{pgfscope}%
\pgfsetrectcap%
\pgfsetmiterjoin%
\pgfsetlinewidth{0.803000pt}%
\definecolor{currentstroke}{rgb}{0.000000,0.000000,0.000000}%
\pgfsetstrokecolor{currentstroke}%
\pgfsetdash{}{0pt}%
\pgfpathmoveto{\pgfqpoint{0.800000in}{0.528000in}}%
\pgfpathlineto{\pgfqpoint{0.800000in}{2.208000in}}%
\pgfusepath{stroke}%
\end{pgfscope}%
\begin{pgfscope}%
\pgfsetrectcap%
\pgfsetmiterjoin%
\pgfsetlinewidth{0.803000pt}%
\definecolor{currentstroke}{rgb}{0.000000,0.000000,0.000000}%
\pgfsetstrokecolor{currentstroke}%
\pgfsetdash{}{0pt}%
\pgfpathmoveto{\pgfqpoint{5.760000in}{0.528000in}}%
\pgfpathlineto{\pgfqpoint{5.760000in}{2.208000in}}%
\pgfusepath{stroke}%
\end{pgfscope}%
\begin{pgfscope}%
\pgfsetrectcap%
\pgfsetmiterjoin%
\pgfsetlinewidth{0.803000pt}%
\definecolor{currentstroke}{rgb}{0.000000,0.000000,0.000000}%
\pgfsetstrokecolor{currentstroke}%
\pgfsetdash{}{0pt}%
\pgfpathmoveto{\pgfqpoint{0.800000in}{0.528000in}}%
\pgfpathlineto{\pgfqpoint{5.760000in}{0.528000in}}%
\pgfusepath{stroke}%
\end{pgfscope}%
\begin{pgfscope}%
\pgfsetrectcap%
\pgfsetmiterjoin%
\pgfsetlinewidth{0.803000pt}%
\definecolor{currentstroke}{rgb}{0.000000,0.000000,0.000000}%
\pgfsetstrokecolor{currentstroke}%
\pgfsetdash{}{0pt}%
\pgfpathmoveto{\pgfqpoint{0.800000in}{2.208000in}}%
\pgfpathlineto{\pgfqpoint{5.760000in}{2.208000in}}%
\pgfusepath{stroke}%
\end{pgfscope}%
\end{pgfpicture}%
\makeatother%
\endgroup%
}}
  \caption{Transmitted signal.}
  \label{fig:task1_st}
\end{figure}
\begin{figure}[h]
  \centering
  \noindent\makebox[\textwidth]{\scalebox{0.90}{%% Creator: Matplotlib, PGF backend
%%
%% To include the figure in your LaTeX document, write
%%   \input{<filename>.pgf}
%%
%% Make sure the required packages are loaded in your preamble
%%   \usepackage{pgf}
%%
%% Figures using additional raster images can only be included by \input if
%% they are in the same directory as the main LaTeX file. For loading figures
%% from other directories you can use the `import` package
%%   \usepackage{import}
%% and then include the figures with
%%   \import{<path to file>}{<filename>.pgf}
%%
%% Matplotlib used the following preamble
%%   \usepackage{fontspec}
%%   \setmainfont{DejaVu Serif}
%%   \setsansfont{DejaVu Sans}
%%   \setmonofont{DejaVu Sans Mono}
%%
\begingroup%
\makeatletter%
\begin{pgfpicture}%
\pgfpathrectangle{\pgfpointorigin}{\pgfqpoint{6.400000in}{4.800000in}}%
\pgfusepath{use as bounding box, clip}%
\begin{pgfscope}%
\pgfsetbuttcap%
\pgfsetmiterjoin%
\definecolor{currentfill}{rgb}{1.000000,1.000000,1.000000}%
\pgfsetfillcolor{currentfill}%
\pgfsetlinewidth{0.000000pt}%
\definecolor{currentstroke}{rgb}{1.000000,1.000000,1.000000}%
\pgfsetstrokecolor{currentstroke}%
\pgfsetdash{}{0pt}%
\pgfpathmoveto{\pgfqpoint{0.000000in}{0.000000in}}%
\pgfpathlineto{\pgfqpoint{6.400000in}{0.000000in}}%
\pgfpathlineto{\pgfqpoint{6.400000in}{4.800000in}}%
\pgfpathlineto{\pgfqpoint{0.000000in}{4.800000in}}%
\pgfpathclose%
\pgfusepath{fill}%
\end{pgfscope}%
\begin{pgfscope}%
\pgfsetbuttcap%
\pgfsetmiterjoin%
\definecolor{currentfill}{rgb}{1.000000,1.000000,1.000000}%
\pgfsetfillcolor{currentfill}%
\pgfsetlinewidth{0.000000pt}%
\definecolor{currentstroke}{rgb}{0.000000,0.000000,0.000000}%
\pgfsetstrokecolor{currentstroke}%
\pgfsetstrokeopacity{0.000000}%
\pgfsetdash{}{0pt}%
\pgfpathmoveto{\pgfqpoint{0.800000in}{2.544000in}}%
\pgfpathlineto{\pgfqpoint{5.760000in}{2.544000in}}%
\pgfpathlineto{\pgfqpoint{5.760000in}{4.224000in}}%
\pgfpathlineto{\pgfqpoint{0.800000in}{4.224000in}}%
\pgfpathclose%
\pgfusepath{fill}%
\end{pgfscope}%
\begin{pgfscope}%
\pgfsetbuttcap%
\pgfsetroundjoin%
\definecolor{currentfill}{rgb}{0.000000,0.000000,0.000000}%
\pgfsetfillcolor{currentfill}%
\pgfsetlinewidth{0.803000pt}%
\definecolor{currentstroke}{rgb}{0.000000,0.000000,0.000000}%
\pgfsetstrokecolor{currentstroke}%
\pgfsetdash{}{0pt}%
\pgfsys@defobject{currentmarker}{\pgfqpoint{0.000000in}{-0.048611in}}{\pgfqpoint{0.000000in}{0.000000in}}{%
\pgfpathmoveto{\pgfqpoint{0.000000in}{0.000000in}}%
\pgfpathlineto{\pgfqpoint{0.000000in}{-0.048611in}}%
\pgfusepath{stroke,fill}%
}%
\begin{pgfscope}%
\pgfsys@transformshift{1.589091in}{2.544000in}%
\pgfsys@useobject{currentmarker}{}%
\end{pgfscope}%
\end{pgfscope}%
\begin{pgfscope}%
\pgftext[x=1.589091in,y=2.446778in,,top]{\sffamily\fontsize{10.000000}{12.000000}\selectfont 50}%
\end{pgfscope}%
\begin{pgfscope}%
\pgfsetbuttcap%
\pgfsetroundjoin%
\definecolor{currentfill}{rgb}{0.000000,0.000000,0.000000}%
\pgfsetfillcolor{currentfill}%
\pgfsetlinewidth{0.803000pt}%
\definecolor{currentstroke}{rgb}{0.000000,0.000000,0.000000}%
\pgfsetstrokecolor{currentstroke}%
\pgfsetdash{}{0pt}%
\pgfsys@defobject{currentmarker}{\pgfqpoint{0.000000in}{-0.048611in}}{\pgfqpoint{0.000000in}{0.000000in}}{%
\pgfpathmoveto{\pgfqpoint{0.000000in}{0.000000in}}%
\pgfpathlineto{\pgfqpoint{0.000000in}{-0.048611in}}%
\pgfusepath{stroke,fill}%
}%
\begin{pgfscope}%
\pgfsys@transformshift{2.434545in}{2.544000in}%
\pgfsys@useobject{currentmarker}{}%
\end{pgfscope}%
\end{pgfscope}%
\begin{pgfscope}%
\pgftext[x=2.434545in,y=2.446778in,,top]{\sffamily\fontsize{10.000000}{12.000000}\selectfont 55}%
\end{pgfscope}%
\begin{pgfscope}%
\pgfsetbuttcap%
\pgfsetroundjoin%
\definecolor{currentfill}{rgb}{0.000000,0.000000,0.000000}%
\pgfsetfillcolor{currentfill}%
\pgfsetlinewidth{0.803000pt}%
\definecolor{currentstroke}{rgb}{0.000000,0.000000,0.000000}%
\pgfsetstrokecolor{currentstroke}%
\pgfsetdash{}{0pt}%
\pgfsys@defobject{currentmarker}{\pgfqpoint{0.000000in}{-0.048611in}}{\pgfqpoint{0.000000in}{0.000000in}}{%
\pgfpathmoveto{\pgfqpoint{0.000000in}{0.000000in}}%
\pgfpathlineto{\pgfqpoint{0.000000in}{-0.048611in}}%
\pgfusepath{stroke,fill}%
}%
\begin{pgfscope}%
\pgfsys@transformshift{3.280000in}{2.544000in}%
\pgfsys@useobject{currentmarker}{}%
\end{pgfscope}%
\end{pgfscope}%
\begin{pgfscope}%
\pgftext[x=3.280000in,y=2.446778in,,top]{\sffamily\fontsize{10.000000}{12.000000}\selectfont 60}%
\end{pgfscope}%
\begin{pgfscope}%
\pgfsetbuttcap%
\pgfsetroundjoin%
\definecolor{currentfill}{rgb}{0.000000,0.000000,0.000000}%
\pgfsetfillcolor{currentfill}%
\pgfsetlinewidth{0.803000pt}%
\definecolor{currentstroke}{rgb}{0.000000,0.000000,0.000000}%
\pgfsetstrokecolor{currentstroke}%
\pgfsetdash{}{0pt}%
\pgfsys@defobject{currentmarker}{\pgfqpoint{0.000000in}{-0.048611in}}{\pgfqpoint{0.000000in}{0.000000in}}{%
\pgfpathmoveto{\pgfqpoint{0.000000in}{0.000000in}}%
\pgfpathlineto{\pgfqpoint{0.000000in}{-0.048611in}}%
\pgfusepath{stroke,fill}%
}%
\begin{pgfscope}%
\pgfsys@transformshift{4.125455in}{2.544000in}%
\pgfsys@useobject{currentmarker}{}%
\end{pgfscope}%
\end{pgfscope}%
\begin{pgfscope}%
\pgftext[x=4.125455in,y=2.446778in,,top]{\sffamily\fontsize{10.000000}{12.000000}\selectfont 65}%
\end{pgfscope}%
\begin{pgfscope}%
\pgfsetbuttcap%
\pgfsetroundjoin%
\definecolor{currentfill}{rgb}{0.000000,0.000000,0.000000}%
\pgfsetfillcolor{currentfill}%
\pgfsetlinewidth{0.803000pt}%
\definecolor{currentstroke}{rgb}{0.000000,0.000000,0.000000}%
\pgfsetstrokecolor{currentstroke}%
\pgfsetdash{}{0pt}%
\pgfsys@defobject{currentmarker}{\pgfqpoint{0.000000in}{-0.048611in}}{\pgfqpoint{0.000000in}{0.000000in}}{%
\pgfpathmoveto{\pgfqpoint{0.000000in}{0.000000in}}%
\pgfpathlineto{\pgfqpoint{0.000000in}{-0.048611in}}%
\pgfusepath{stroke,fill}%
}%
\begin{pgfscope}%
\pgfsys@transformshift{4.970909in}{2.544000in}%
\pgfsys@useobject{currentmarker}{}%
\end{pgfscope}%
\end{pgfscope}%
\begin{pgfscope}%
\pgftext[x=4.970909in,y=2.446778in,,top]{\sffamily\fontsize{10.000000}{12.000000}\selectfont 70}%
\end{pgfscope}%
\begin{pgfscope}%
\pgftext[x=3.280000in,y=2.256809in,,top]{\sffamily\fontsize{10.000000}{12.000000}\selectfont Time [\(\displaystyle \mu\)s]}%
\end{pgfscope}%
\begin{pgfscope}%
\pgfsetbuttcap%
\pgfsetroundjoin%
\definecolor{currentfill}{rgb}{0.000000,0.000000,0.000000}%
\pgfsetfillcolor{currentfill}%
\pgfsetlinewidth{0.803000pt}%
\definecolor{currentstroke}{rgb}{0.000000,0.000000,0.000000}%
\pgfsetstrokecolor{currentstroke}%
\pgfsetdash{}{0pt}%
\pgfsys@defobject{currentmarker}{\pgfqpoint{-0.048611in}{0.000000in}}{\pgfqpoint{0.000000in}{0.000000in}}{%
\pgfpathmoveto{\pgfqpoint{0.000000in}{0.000000in}}%
\pgfpathlineto{\pgfqpoint{-0.048611in}{0.000000in}}%
\pgfusepath{stroke,fill}%
}%
\begin{pgfscope}%
\pgfsys@transformshift{0.800000in}{2.647793in}%
\pgfsys@useobject{currentmarker}{}%
\end{pgfscope}%
\end{pgfscope}%
\begin{pgfscope}%
\pgftext[x=0.232943in,y=2.595032in,left,base]{\sffamily\fontsize{10.000000}{12.000000}\selectfont -2000}%
\end{pgfscope}%
\begin{pgfscope}%
\pgfsetbuttcap%
\pgfsetroundjoin%
\definecolor{currentfill}{rgb}{0.000000,0.000000,0.000000}%
\pgfsetfillcolor{currentfill}%
\pgfsetlinewidth{0.803000pt}%
\definecolor{currentstroke}{rgb}{0.000000,0.000000,0.000000}%
\pgfsetstrokecolor{currentstroke}%
\pgfsetdash{}{0pt}%
\pgfsys@defobject{currentmarker}{\pgfqpoint{-0.048611in}{0.000000in}}{\pgfqpoint{0.000000in}{0.000000in}}{%
\pgfpathmoveto{\pgfqpoint{0.000000in}{0.000000in}}%
\pgfpathlineto{\pgfqpoint{-0.048611in}{0.000000in}}%
\pgfusepath{stroke,fill}%
}%
\begin{pgfscope}%
\pgfsys@transformshift{0.800000in}{3.019725in}%
\pgfsys@useobject{currentmarker}{}%
\end{pgfscope}%
\end{pgfscope}%
\begin{pgfscope}%
\pgftext[x=0.232943in,y=2.966963in,left,base]{\sffamily\fontsize{10.000000}{12.000000}\selectfont -1000}%
\end{pgfscope}%
\begin{pgfscope}%
\pgfsetbuttcap%
\pgfsetroundjoin%
\definecolor{currentfill}{rgb}{0.000000,0.000000,0.000000}%
\pgfsetfillcolor{currentfill}%
\pgfsetlinewidth{0.803000pt}%
\definecolor{currentstroke}{rgb}{0.000000,0.000000,0.000000}%
\pgfsetstrokecolor{currentstroke}%
\pgfsetdash{}{0pt}%
\pgfsys@defobject{currentmarker}{\pgfqpoint{-0.048611in}{0.000000in}}{\pgfqpoint{0.000000in}{0.000000in}}{%
\pgfpathmoveto{\pgfqpoint{0.000000in}{0.000000in}}%
\pgfpathlineto{\pgfqpoint{-0.048611in}{0.000000in}}%
\pgfusepath{stroke,fill}%
}%
\begin{pgfscope}%
\pgfsys@transformshift{0.800000in}{3.391656in}%
\pgfsys@useobject{currentmarker}{}%
\end{pgfscope}%
\end{pgfscope}%
\begin{pgfscope}%
\pgftext[x=0.614413in,y=3.338895in,left,base]{\sffamily\fontsize{10.000000}{12.000000}\selectfont 0}%
\end{pgfscope}%
\begin{pgfscope}%
\pgfsetbuttcap%
\pgfsetroundjoin%
\definecolor{currentfill}{rgb}{0.000000,0.000000,0.000000}%
\pgfsetfillcolor{currentfill}%
\pgfsetlinewidth{0.803000pt}%
\definecolor{currentstroke}{rgb}{0.000000,0.000000,0.000000}%
\pgfsetstrokecolor{currentstroke}%
\pgfsetdash{}{0pt}%
\pgfsys@defobject{currentmarker}{\pgfqpoint{-0.048611in}{0.000000in}}{\pgfqpoint{0.000000in}{0.000000in}}{%
\pgfpathmoveto{\pgfqpoint{0.000000in}{0.000000in}}%
\pgfpathlineto{\pgfqpoint{-0.048611in}{0.000000in}}%
\pgfusepath{stroke,fill}%
}%
\begin{pgfscope}%
\pgfsys@transformshift{0.800000in}{3.763588in}%
\pgfsys@useobject{currentmarker}{}%
\end{pgfscope}%
\end{pgfscope}%
\begin{pgfscope}%
\pgftext[x=0.349316in,y=3.710826in,left,base]{\sffamily\fontsize{10.000000}{12.000000}\selectfont 1000}%
\end{pgfscope}%
\begin{pgfscope}%
\pgfsetbuttcap%
\pgfsetroundjoin%
\definecolor{currentfill}{rgb}{0.000000,0.000000,0.000000}%
\pgfsetfillcolor{currentfill}%
\pgfsetlinewidth{0.803000pt}%
\definecolor{currentstroke}{rgb}{0.000000,0.000000,0.000000}%
\pgfsetstrokecolor{currentstroke}%
\pgfsetdash{}{0pt}%
\pgfsys@defobject{currentmarker}{\pgfqpoint{-0.048611in}{0.000000in}}{\pgfqpoint{0.000000in}{0.000000in}}{%
\pgfpathmoveto{\pgfqpoint{0.000000in}{0.000000in}}%
\pgfpathlineto{\pgfqpoint{-0.048611in}{0.000000in}}%
\pgfusepath{stroke,fill}%
}%
\begin{pgfscope}%
\pgfsys@transformshift{0.800000in}{4.135519in}%
\pgfsys@useobject{currentmarker}{}%
\end{pgfscope}%
\end{pgfscope}%
\begin{pgfscope}%
\pgftext[x=0.349316in,y=4.082757in,left,base]{\sffamily\fontsize{10.000000}{12.000000}\selectfont 2000}%
\end{pgfscope}%
\begin{pgfscope}%
\pgftext[x=0.177387in,y=3.384000in,,bottom,rotate=90.000000]{\sffamily\fontsize{10.000000}{12.000000}\selectfont Real part of the signal}%
\end{pgfscope}%
\begin{pgfscope}%
\pgfpathrectangle{\pgfqpoint{0.800000in}{2.544000in}}{\pgfqpoint{4.960000in}{1.680000in}} %
\pgfusepath{clip}%
\pgfsetrectcap%
\pgfsetroundjoin%
\pgfsetlinewidth{1.505625pt}%
\definecolor{currentstroke}{rgb}{0.121569,0.466667,0.705882}%
\pgfsetstrokecolor{currentstroke}%
\pgfsetdash{}{0pt}%
\pgfpathmoveto{\pgfqpoint{1.025455in}{3.044176in}}%
\pgfpathlineto{\pgfqpoint{1.026322in}{3.334720in}}%
\pgfpathlineto{\pgfqpoint{1.026322in}{3.334720in}}%
\pgfpathlineto{\pgfqpoint{1.026322in}{3.334720in}}%
\pgfpathlineto{\pgfqpoint{1.027189in}{3.331002in}}%
\pgfpathlineto{\pgfqpoint{1.028056in}{3.125922in}}%
\pgfpathlineto{\pgfqpoint{1.029790in}{3.580742in}}%
\pgfpathlineto{\pgfqpoint{1.030657in}{3.520282in}}%
\pgfpathlineto{\pgfqpoint{1.031524in}{3.412901in}}%
\pgfpathlineto{\pgfqpoint{1.032392in}{3.086114in}}%
\pgfpathlineto{\pgfqpoint{1.034126in}{3.440849in}}%
\pgfpathlineto{\pgfqpoint{1.034993in}{3.359801in}}%
\pgfpathlineto{\pgfqpoint{1.035860in}{3.824841in}}%
\pgfpathlineto{\pgfqpoint{1.036727in}{3.322738in}}%
\pgfpathlineto{\pgfqpoint{1.038462in}{3.733729in}}%
\pgfpathlineto{\pgfqpoint{1.040196in}{3.168924in}}%
\pgfpathlineto{\pgfqpoint{1.041063in}{3.290067in}}%
\pgfpathlineto{\pgfqpoint{1.041930in}{3.741766in}}%
\pgfpathlineto{\pgfqpoint{1.042797in}{3.271435in}}%
\pgfpathlineto{\pgfqpoint{1.043664in}{3.538903in}}%
\pgfpathlineto{\pgfqpoint{1.044531in}{3.310389in}}%
\pgfpathlineto{\pgfqpoint{1.045399in}{3.606567in}}%
\pgfpathlineto{\pgfqpoint{1.047133in}{3.180762in}}%
\pgfpathlineto{\pgfqpoint{1.048867in}{3.604237in}}%
\pgfpathlineto{\pgfqpoint{1.049734in}{3.300462in}}%
\pgfpathlineto{\pgfqpoint{1.050601in}{3.461014in}}%
\pgfpathlineto{\pgfqpoint{1.051469in}{3.370565in}}%
\pgfpathlineto{\pgfqpoint{1.053203in}{2.907547in}}%
\pgfpathlineto{\pgfqpoint{1.054937in}{3.360864in}}%
\pgfpathlineto{\pgfqpoint{1.055804in}{3.127451in}}%
\pgfpathlineto{\pgfqpoint{1.056671in}{3.331658in}}%
\pgfpathlineto{\pgfqpoint{1.058406in}{2.966546in}}%
\pgfpathlineto{\pgfqpoint{1.059273in}{3.271268in}}%
\pgfpathlineto{\pgfqpoint{1.060140in}{3.236050in}}%
\pgfpathlineto{\pgfqpoint{1.061007in}{3.434910in}}%
\pgfpathlineto{\pgfqpoint{1.061874in}{3.376451in}}%
\pgfpathlineto{\pgfqpoint{1.062741in}{3.514875in}}%
\pgfpathlineto{\pgfqpoint{1.063608in}{3.502685in}}%
\pgfpathlineto{\pgfqpoint{1.064476in}{3.477063in}}%
\pgfpathlineto{\pgfqpoint{1.065343in}{3.604110in}}%
\pgfpathlineto{\pgfqpoint{1.067077in}{3.167461in}}%
\pgfpathlineto{\pgfqpoint{1.067944in}{3.460480in}}%
\pgfpathlineto{\pgfqpoint{1.068811in}{3.442200in}}%
\pgfpathlineto{\pgfqpoint{1.069678in}{3.633330in}}%
\pgfpathlineto{\pgfqpoint{1.070545in}{3.612753in}}%
\pgfpathlineto{\pgfqpoint{1.072280in}{3.496997in}}%
\pgfpathlineto{\pgfqpoint{1.073147in}{3.530406in}}%
\pgfpathlineto{\pgfqpoint{1.075748in}{3.080296in}}%
\pgfpathlineto{\pgfqpoint{1.077483in}{3.876180in}}%
\pgfpathlineto{\pgfqpoint{1.078350in}{3.456284in}}%
\pgfpathlineto{\pgfqpoint{1.079217in}{3.476534in}}%
\pgfpathlineto{\pgfqpoint{1.080084in}{3.298604in}}%
\pgfpathlineto{\pgfqpoint{1.080951in}{3.334702in}}%
\pgfpathlineto{\pgfqpoint{1.081818in}{3.509491in}}%
\pgfpathlineto{\pgfqpoint{1.082685in}{3.113898in}}%
\pgfpathlineto{\pgfqpoint{1.083552in}{3.200304in}}%
\pgfpathlineto{\pgfqpoint{1.084420in}{3.517436in}}%
\pgfpathlineto{\pgfqpoint{1.086154in}{3.118988in}}%
\pgfpathlineto{\pgfqpoint{1.087021in}{3.252949in}}%
\pgfpathlineto{\pgfqpoint{1.087888in}{3.654254in}}%
\pgfpathlineto{\pgfqpoint{1.088755in}{3.470530in}}%
\pgfpathlineto{\pgfqpoint{1.089622in}{3.511904in}}%
\pgfpathlineto{\pgfqpoint{1.090490in}{3.518762in}}%
\pgfpathlineto{\pgfqpoint{1.091357in}{3.265510in}}%
\pgfpathlineto{\pgfqpoint{1.093958in}{3.598863in}}%
\pgfpathlineto{\pgfqpoint{1.095692in}{3.183640in}}%
\pgfpathlineto{\pgfqpoint{1.096559in}{3.537652in}}%
\pgfpathlineto{\pgfqpoint{1.097427in}{3.041285in}}%
\pgfpathlineto{\pgfqpoint{1.098294in}{3.381217in}}%
\pgfpathlineto{\pgfqpoint{1.099161in}{3.275469in}}%
\pgfpathlineto{\pgfqpoint{1.100028in}{3.557835in}}%
\pgfpathlineto{\pgfqpoint{1.100895in}{3.553180in}}%
\pgfpathlineto{\pgfqpoint{1.103497in}{3.119779in}}%
\pgfpathlineto{\pgfqpoint{1.104364in}{3.572853in}}%
\pgfpathlineto{\pgfqpoint{1.105231in}{3.453166in}}%
\pgfpathlineto{\pgfqpoint{1.106098in}{3.058648in}}%
\pgfpathlineto{\pgfqpoint{1.107832in}{3.378960in}}%
\pgfpathlineto{\pgfqpoint{1.108699in}{3.360606in}}%
\pgfpathlineto{\pgfqpoint{1.109566in}{3.393837in}}%
\pgfpathlineto{\pgfqpoint{1.110434in}{3.311111in}}%
\pgfpathlineto{\pgfqpoint{1.111301in}{3.414257in}}%
\pgfpathlineto{\pgfqpoint{1.112168in}{3.244610in}}%
\pgfpathlineto{\pgfqpoint{1.113035in}{3.279310in}}%
\pgfpathlineto{\pgfqpoint{1.113902in}{3.311269in}}%
\pgfpathlineto{\pgfqpoint{1.114769in}{3.510994in}}%
\pgfpathlineto{\pgfqpoint{1.116503in}{3.200942in}}%
\pgfpathlineto{\pgfqpoint{1.117371in}{3.493861in}}%
\pgfpathlineto{\pgfqpoint{1.119105in}{3.010765in}}%
\pgfpathlineto{\pgfqpoint{1.119972in}{3.396333in}}%
\pgfpathlineto{\pgfqpoint{1.120839in}{3.090333in}}%
\pgfpathlineto{\pgfqpoint{1.121706in}{3.343548in}}%
\pgfpathlineto{\pgfqpoint{1.122573in}{3.340987in}}%
\pgfpathlineto{\pgfqpoint{1.123441in}{3.441415in}}%
\pgfpathlineto{\pgfqpoint{1.124308in}{3.214499in}}%
\pgfpathlineto{\pgfqpoint{1.125175in}{3.552070in}}%
\pgfpathlineto{\pgfqpoint{1.126909in}{3.187786in}}%
\pgfpathlineto{\pgfqpoint{1.127776in}{3.489963in}}%
\pgfpathlineto{\pgfqpoint{1.128643in}{3.077263in}}%
\pgfpathlineto{\pgfqpoint{1.129510in}{3.223838in}}%
\pgfpathlineto{\pgfqpoint{1.130378in}{3.590164in}}%
\pgfpathlineto{\pgfqpoint{1.131245in}{3.456574in}}%
\pgfpathlineto{\pgfqpoint{1.132112in}{3.694392in}}%
\pgfpathlineto{\pgfqpoint{1.132979in}{3.541157in}}%
\pgfpathlineto{\pgfqpoint{1.133846in}{3.546279in}}%
\pgfpathlineto{\pgfqpoint{1.134713in}{3.616054in}}%
\pgfpathlineto{\pgfqpoint{1.135580in}{3.536546in}}%
\pgfpathlineto{\pgfqpoint{1.137315in}{3.134781in}}%
\pgfpathlineto{\pgfqpoint{1.138182in}{3.796184in}}%
\pgfpathlineto{\pgfqpoint{1.139049in}{3.333953in}}%
\pgfpathlineto{\pgfqpoint{1.139916in}{3.618465in}}%
\pgfpathlineto{\pgfqpoint{1.140783in}{3.423678in}}%
\pgfpathlineto{\pgfqpoint{1.141650in}{3.438823in}}%
\pgfpathlineto{\pgfqpoint{1.142517in}{3.616441in}}%
\pgfpathlineto{\pgfqpoint{1.144252in}{3.253379in}}%
\pgfpathlineto{\pgfqpoint{1.145119in}{3.429052in}}%
\pgfpathlineto{\pgfqpoint{1.145986in}{3.298038in}}%
\pgfpathlineto{\pgfqpoint{1.146853in}{3.504785in}}%
\pgfpathlineto{\pgfqpoint{1.147720in}{3.285579in}}%
\pgfpathlineto{\pgfqpoint{1.148587in}{3.365286in}}%
\pgfpathlineto{\pgfqpoint{1.149455in}{3.266555in}}%
\pgfpathlineto{\pgfqpoint{1.151189in}{3.617542in}}%
\pgfpathlineto{\pgfqpoint{1.152056in}{3.736632in}}%
\pgfpathlineto{\pgfqpoint{1.152923in}{3.098152in}}%
\pgfpathlineto{\pgfqpoint{1.153790in}{3.249300in}}%
\pgfpathlineto{\pgfqpoint{1.154657in}{3.328792in}}%
\pgfpathlineto{\pgfqpoint{1.156392in}{3.166140in}}%
\pgfpathlineto{\pgfqpoint{1.157259in}{3.225586in}}%
\pgfpathlineto{\pgfqpoint{1.158126in}{3.608242in}}%
\pgfpathlineto{\pgfqpoint{1.158993in}{3.050968in}}%
\pgfpathlineto{\pgfqpoint{1.159860in}{3.581355in}}%
\pgfpathlineto{\pgfqpoint{1.162462in}{3.115479in}}%
\pgfpathlineto{\pgfqpoint{1.163329in}{3.086896in}}%
\pgfpathlineto{\pgfqpoint{1.164196in}{3.722109in}}%
\pgfpathlineto{\pgfqpoint{1.165063in}{3.304585in}}%
\pgfpathlineto{\pgfqpoint{1.165930in}{3.385129in}}%
\pgfpathlineto{\pgfqpoint{1.166797in}{3.351604in}}%
\pgfpathlineto{\pgfqpoint{1.167664in}{3.466357in}}%
\pgfpathlineto{\pgfqpoint{1.168531in}{3.769355in}}%
\pgfpathlineto{\pgfqpoint{1.171133in}{3.162031in}}%
\pgfpathlineto{\pgfqpoint{1.172000in}{3.537508in}}%
\pgfpathlineto{\pgfqpoint{1.173734in}{3.140639in}}%
\pgfpathlineto{\pgfqpoint{1.174601in}{3.423338in}}%
\pgfpathlineto{\pgfqpoint{1.175469in}{3.135405in}}%
\pgfpathlineto{\pgfqpoint{1.177203in}{3.792763in}}%
\pgfpathlineto{\pgfqpoint{1.178070in}{3.505322in}}%
\pgfpathlineto{\pgfqpoint{1.178937in}{3.752086in}}%
\pgfpathlineto{\pgfqpoint{1.180671in}{3.225453in}}%
\pgfpathlineto{\pgfqpoint{1.181538in}{3.394224in}}%
\pgfpathlineto{\pgfqpoint{1.182406in}{3.032199in}}%
\pgfpathlineto{\pgfqpoint{1.183273in}{3.499261in}}%
\pgfpathlineto{\pgfqpoint{1.184140in}{3.240132in}}%
\pgfpathlineto{\pgfqpoint{1.185007in}{3.316384in}}%
\pgfpathlineto{\pgfqpoint{1.185874in}{3.195257in}}%
\pgfpathlineto{\pgfqpoint{1.186741in}{3.375255in}}%
\pgfpathlineto{\pgfqpoint{1.187608in}{3.225305in}}%
\pgfpathlineto{\pgfqpoint{1.189343in}{3.531581in}}%
\pgfpathlineto{\pgfqpoint{1.191077in}{3.243934in}}%
\pgfpathlineto{\pgfqpoint{1.191944in}{3.539288in}}%
\pgfpathlineto{\pgfqpoint{1.192811in}{3.145587in}}%
\pgfpathlineto{\pgfqpoint{1.193678in}{3.293506in}}%
\pgfpathlineto{\pgfqpoint{1.194545in}{3.205143in}}%
\pgfpathlineto{\pgfqpoint{1.196280in}{3.500916in}}%
\pgfpathlineto{\pgfqpoint{1.197147in}{3.339132in}}%
\pgfpathlineto{\pgfqpoint{1.198881in}{3.599499in}}%
\pgfpathlineto{\pgfqpoint{1.200615in}{3.460255in}}%
\pgfpathlineto{\pgfqpoint{1.201483in}{3.418034in}}%
\pgfpathlineto{\pgfqpoint{1.202350in}{2.986697in}}%
\pgfpathlineto{\pgfqpoint{1.204084in}{3.424148in}}%
\pgfpathlineto{\pgfqpoint{1.205818in}{3.189813in}}%
\pgfpathlineto{\pgfqpoint{1.206685in}{3.356899in}}%
\pgfpathlineto{\pgfqpoint{1.207552in}{3.337765in}}%
\pgfpathlineto{\pgfqpoint{1.208420in}{3.244223in}}%
\pgfpathlineto{\pgfqpoint{1.209287in}{3.639214in}}%
\pgfpathlineto{\pgfqpoint{1.210154in}{3.279201in}}%
\pgfpathlineto{\pgfqpoint{1.211888in}{3.562137in}}%
\pgfpathlineto{\pgfqpoint{1.212755in}{3.423771in}}%
\pgfpathlineto{\pgfqpoint{1.213622in}{3.558237in}}%
\pgfpathlineto{\pgfqpoint{1.214490in}{3.288434in}}%
\pgfpathlineto{\pgfqpoint{1.215357in}{3.501493in}}%
\pgfpathlineto{\pgfqpoint{1.216224in}{3.255937in}}%
\pgfpathlineto{\pgfqpoint{1.217091in}{3.344524in}}%
\pgfpathlineto{\pgfqpoint{1.217958in}{3.266107in}}%
\pgfpathlineto{\pgfqpoint{1.218825in}{3.304213in}}%
\pgfpathlineto{\pgfqpoint{1.220559in}{3.275777in}}%
\pgfpathlineto{\pgfqpoint{1.221427in}{3.501136in}}%
\pgfpathlineto{\pgfqpoint{1.222294in}{3.235912in}}%
\pgfpathlineto{\pgfqpoint{1.224895in}{3.518973in}}%
\pgfpathlineto{\pgfqpoint{1.225762in}{3.587452in}}%
\pgfpathlineto{\pgfqpoint{1.227497in}{3.295625in}}%
\pgfpathlineto{\pgfqpoint{1.228364in}{3.297279in}}%
\pgfpathlineto{\pgfqpoint{1.229231in}{3.337967in}}%
\pgfpathlineto{\pgfqpoint{1.230098in}{3.566610in}}%
\pgfpathlineto{\pgfqpoint{1.230965in}{3.183802in}}%
\pgfpathlineto{\pgfqpoint{1.231832in}{3.690168in}}%
\pgfpathlineto{\pgfqpoint{1.233566in}{3.220791in}}%
\pgfpathlineto{\pgfqpoint{1.234434in}{3.547745in}}%
\pgfpathlineto{\pgfqpoint{1.235301in}{3.176142in}}%
\pgfpathlineto{\pgfqpoint{1.236168in}{3.422137in}}%
\pgfpathlineto{\pgfqpoint{1.237035in}{3.367120in}}%
\pgfpathlineto{\pgfqpoint{1.237902in}{3.446707in}}%
\pgfpathlineto{\pgfqpoint{1.238769in}{3.697955in}}%
\pgfpathlineto{\pgfqpoint{1.240503in}{3.399339in}}%
\pgfpathlineto{\pgfqpoint{1.241371in}{3.448655in}}%
\pgfpathlineto{\pgfqpoint{1.242238in}{3.400170in}}%
\pgfpathlineto{\pgfqpoint{1.243105in}{3.025244in}}%
\pgfpathlineto{\pgfqpoint{1.243972in}{3.594676in}}%
\pgfpathlineto{\pgfqpoint{1.244839in}{3.299625in}}%
\pgfpathlineto{\pgfqpoint{1.245706in}{3.491841in}}%
\pgfpathlineto{\pgfqpoint{1.246573in}{3.439237in}}%
\pgfpathlineto{\pgfqpoint{1.247441in}{3.239623in}}%
\pgfpathlineto{\pgfqpoint{1.248308in}{3.585505in}}%
\pgfpathlineto{\pgfqpoint{1.249175in}{3.367826in}}%
\pgfpathlineto{\pgfqpoint{1.250042in}{3.569898in}}%
\pgfpathlineto{\pgfqpoint{1.250909in}{3.548210in}}%
\pgfpathlineto{\pgfqpoint{1.252643in}{3.743789in}}%
\pgfpathlineto{\pgfqpoint{1.254378in}{3.387948in}}%
\pgfpathlineto{\pgfqpoint{1.255245in}{3.520811in}}%
\pgfpathlineto{\pgfqpoint{1.256979in}{3.293302in}}%
\pgfpathlineto{\pgfqpoint{1.257846in}{3.370193in}}%
\pgfpathlineto{\pgfqpoint{1.258713in}{3.360643in}}%
\pgfpathlineto{\pgfqpoint{1.259580in}{3.354739in}}%
\pgfpathlineto{\pgfqpoint{1.261315in}{3.521373in}}%
\pgfpathlineto{\pgfqpoint{1.263049in}{3.303833in}}%
\pgfpathlineto{\pgfqpoint{1.263916in}{3.661190in}}%
\pgfpathlineto{\pgfqpoint{1.265650in}{3.072207in}}%
\pgfpathlineto{\pgfqpoint{1.266517in}{3.493234in}}%
\pgfpathlineto{\pgfqpoint{1.268252in}{3.242724in}}%
\pgfpathlineto{\pgfqpoint{1.269119in}{3.354194in}}%
\pgfpathlineto{\pgfqpoint{1.269986in}{3.317652in}}%
\pgfpathlineto{\pgfqpoint{1.270853in}{3.329860in}}%
\pgfpathlineto{\pgfqpoint{1.271720in}{3.586208in}}%
\pgfpathlineto{\pgfqpoint{1.272587in}{3.079559in}}%
\pgfpathlineto{\pgfqpoint{1.273455in}{3.752468in}}%
\pgfpathlineto{\pgfqpoint{1.275189in}{3.177347in}}%
\pgfpathlineto{\pgfqpoint{1.276056in}{3.557178in}}%
\pgfpathlineto{\pgfqpoint{1.276923in}{2.972117in}}%
\pgfpathlineto{\pgfqpoint{1.278657in}{3.581972in}}%
\pgfpathlineto{\pgfqpoint{1.279524in}{3.252677in}}%
\pgfpathlineto{\pgfqpoint{1.283860in}{3.647334in}}%
\pgfpathlineto{\pgfqpoint{1.285594in}{3.119773in}}%
\pgfpathlineto{\pgfqpoint{1.287329in}{3.389220in}}%
\pgfpathlineto{\pgfqpoint{1.288196in}{3.406794in}}%
\pgfpathlineto{\pgfqpoint{1.289930in}{3.155346in}}%
\pgfpathlineto{\pgfqpoint{1.290797in}{3.197518in}}%
\pgfpathlineto{\pgfqpoint{1.291664in}{3.362423in}}%
\pgfpathlineto{\pgfqpoint{1.292531in}{3.292467in}}%
\pgfpathlineto{\pgfqpoint{1.293399in}{3.761077in}}%
\pgfpathlineto{\pgfqpoint{1.294266in}{3.368625in}}%
\pgfpathlineto{\pgfqpoint{1.295133in}{3.570669in}}%
\pgfpathlineto{\pgfqpoint{1.296000in}{3.176425in}}%
\pgfpathlineto{\pgfqpoint{1.296867in}{3.234144in}}%
\pgfpathlineto{\pgfqpoint{1.297734in}{3.400505in}}%
\pgfpathlineto{\pgfqpoint{1.298601in}{3.029391in}}%
\pgfpathlineto{\pgfqpoint{1.300336in}{3.564723in}}%
\pgfpathlineto{\pgfqpoint{1.301203in}{3.201635in}}%
\pgfpathlineto{\pgfqpoint{1.302070in}{3.557671in}}%
\pgfpathlineto{\pgfqpoint{1.302937in}{3.212705in}}%
\pgfpathlineto{\pgfqpoint{1.304671in}{3.485818in}}%
\pgfpathlineto{\pgfqpoint{1.306406in}{3.341745in}}%
\pgfpathlineto{\pgfqpoint{1.307273in}{3.470094in}}%
\pgfpathlineto{\pgfqpoint{1.309007in}{3.205213in}}%
\pgfpathlineto{\pgfqpoint{1.309874in}{3.675476in}}%
\pgfpathlineto{\pgfqpoint{1.311608in}{3.203315in}}%
\pgfpathlineto{\pgfqpoint{1.312476in}{3.531378in}}%
\pgfpathlineto{\pgfqpoint{1.313343in}{3.517597in}}%
\pgfpathlineto{\pgfqpoint{1.314210in}{3.601426in}}%
\pgfpathlineto{\pgfqpoint{1.315077in}{3.206788in}}%
\pgfpathlineto{\pgfqpoint{1.315944in}{3.449170in}}%
\pgfpathlineto{\pgfqpoint{1.316811in}{3.276026in}}%
\pgfpathlineto{\pgfqpoint{1.317678in}{3.363557in}}%
\pgfpathlineto{\pgfqpoint{1.318545in}{3.346511in}}%
\pgfpathlineto{\pgfqpoint{1.319413in}{3.313539in}}%
\pgfpathlineto{\pgfqpoint{1.320280in}{3.761346in}}%
\pgfpathlineto{\pgfqpoint{1.321147in}{3.379161in}}%
\pgfpathlineto{\pgfqpoint{1.322014in}{3.585408in}}%
\pgfpathlineto{\pgfqpoint{1.323748in}{3.196241in}}%
\pgfpathlineto{\pgfqpoint{1.324615in}{3.421345in}}%
\pgfpathlineto{\pgfqpoint{1.325483in}{3.189818in}}%
\pgfpathlineto{\pgfqpoint{1.327217in}{3.551433in}}%
\pgfpathlineto{\pgfqpoint{1.328084in}{3.461141in}}%
\pgfpathlineto{\pgfqpoint{1.328951in}{3.248251in}}%
\pgfpathlineto{\pgfqpoint{1.329818in}{3.523365in}}%
\pgfpathlineto{\pgfqpoint{1.332420in}{3.054668in}}%
\pgfpathlineto{\pgfqpoint{1.335021in}{3.699895in}}%
\pgfpathlineto{\pgfqpoint{1.335888in}{3.385528in}}%
\pgfpathlineto{\pgfqpoint{1.336755in}{3.503184in}}%
\pgfpathlineto{\pgfqpoint{1.337622in}{3.108851in}}%
\pgfpathlineto{\pgfqpoint{1.338490in}{3.420596in}}%
\pgfpathlineto{\pgfqpoint{1.339357in}{3.323529in}}%
\pgfpathlineto{\pgfqpoint{1.340224in}{3.406479in}}%
\pgfpathlineto{\pgfqpoint{1.341091in}{3.263024in}}%
\pgfpathlineto{\pgfqpoint{1.341958in}{3.510986in}}%
\pgfpathlineto{\pgfqpoint{1.342825in}{3.053430in}}%
\pgfpathlineto{\pgfqpoint{1.343692in}{3.949829in}}%
\pgfpathlineto{\pgfqpoint{1.344559in}{2.968888in}}%
\pgfpathlineto{\pgfqpoint{1.346294in}{3.544136in}}%
\pgfpathlineto{\pgfqpoint{1.347161in}{3.231323in}}%
\pgfpathlineto{\pgfqpoint{1.348028in}{3.682475in}}%
\pgfpathlineto{\pgfqpoint{1.348895in}{3.330383in}}%
\pgfpathlineto{\pgfqpoint{1.349762in}{3.709586in}}%
\pgfpathlineto{\pgfqpoint{1.350629in}{3.119895in}}%
\pgfpathlineto{\pgfqpoint{1.352364in}{3.356170in}}%
\pgfpathlineto{\pgfqpoint{1.354098in}{3.233752in}}%
\pgfpathlineto{\pgfqpoint{1.354965in}{3.585598in}}%
\pgfpathlineto{\pgfqpoint{1.355832in}{3.193769in}}%
\pgfpathlineto{\pgfqpoint{1.357566in}{3.601177in}}%
\pgfpathlineto{\pgfqpoint{1.358434in}{3.213505in}}%
\pgfpathlineto{\pgfqpoint{1.359301in}{3.307422in}}%
\pgfpathlineto{\pgfqpoint{1.360168in}{3.331152in}}%
\pgfpathlineto{\pgfqpoint{1.361035in}{3.184676in}}%
\pgfpathlineto{\pgfqpoint{1.361902in}{3.521897in}}%
\pgfpathlineto{\pgfqpoint{1.362769in}{3.474752in}}%
\pgfpathlineto{\pgfqpoint{1.364503in}{3.644168in}}%
\pgfpathlineto{\pgfqpoint{1.365371in}{3.173101in}}%
\pgfpathlineto{\pgfqpoint{1.366238in}{3.303088in}}%
\pgfpathlineto{\pgfqpoint{1.367105in}{3.958381in}}%
\pgfpathlineto{\pgfqpoint{1.367972in}{2.942498in}}%
\pgfpathlineto{\pgfqpoint{1.368839in}{3.337317in}}%
\pgfpathlineto{\pgfqpoint{1.369706in}{3.057983in}}%
\pgfpathlineto{\pgfqpoint{1.370573in}{3.432487in}}%
\pgfpathlineto{\pgfqpoint{1.371441in}{3.201776in}}%
\pgfpathlineto{\pgfqpoint{1.372308in}{3.500947in}}%
\pgfpathlineto{\pgfqpoint{1.373175in}{3.369442in}}%
\pgfpathlineto{\pgfqpoint{1.374042in}{3.578281in}}%
\pgfpathlineto{\pgfqpoint{1.374909in}{3.235635in}}%
\pgfpathlineto{\pgfqpoint{1.375776in}{3.629307in}}%
\pgfpathlineto{\pgfqpoint{1.377510in}{3.361538in}}%
\pgfpathlineto{\pgfqpoint{1.379245in}{3.418460in}}%
\pgfpathlineto{\pgfqpoint{1.380112in}{3.519757in}}%
\pgfpathlineto{\pgfqpoint{1.380979in}{3.332869in}}%
\pgfpathlineto{\pgfqpoint{1.381846in}{3.537909in}}%
\pgfpathlineto{\pgfqpoint{1.382713in}{3.519898in}}%
\pgfpathlineto{\pgfqpoint{1.383580in}{3.289312in}}%
\pgfpathlineto{\pgfqpoint{1.384448in}{3.386708in}}%
\pgfpathlineto{\pgfqpoint{1.385315in}{3.627169in}}%
\pgfpathlineto{\pgfqpoint{1.387049in}{3.312602in}}%
\pgfpathlineto{\pgfqpoint{1.388783in}{3.616165in}}%
\pgfpathlineto{\pgfqpoint{1.390517in}{3.350344in}}%
\pgfpathlineto{\pgfqpoint{1.391385in}{3.349832in}}%
\pgfpathlineto{\pgfqpoint{1.392252in}{3.242593in}}%
\pgfpathlineto{\pgfqpoint{1.393119in}{3.307214in}}%
\pgfpathlineto{\pgfqpoint{1.393986in}{3.091874in}}%
\pgfpathlineto{\pgfqpoint{1.394853in}{3.485444in}}%
\pgfpathlineto{\pgfqpoint{1.396587in}{3.268961in}}%
\pgfpathlineto{\pgfqpoint{1.397455in}{3.558786in}}%
\pgfpathlineto{\pgfqpoint{1.398322in}{3.356080in}}%
\pgfpathlineto{\pgfqpoint{1.399189in}{3.868838in}}%
\pgfpathlineto{\pgfqpoint{1.400056in}{2.891021in}}%
\pgfpathlineto{\pgfqpoint{1.400923in}{3.435256in}}%
\pgfpathlineto{\pgfqpoint{1.401790in}{3.269434in}}%
\pgfpathlineto{\pgfqpoint{1.404392in}{3.762598in}}%
\pgfpathlineto{\pgfqpoint{1.406993in}{3.216967in}}%
\pgfpathlineto{\pgfqpoint{1.407860in}{3.559791in}}%
\pgfpathlineto{\pgfqpoint{1.408727in}{3.534123in}}%
\pgfpathlineto{\pgfqpoint{1.409594in}{3.591436in}}%
\pgfpathlineto{\pgfqpoint{1.410462in}{3.314444in}}%
\pgfpathlineto{\pgfqpoint{1.411329in}{3.335563in}}%
\pgfpathlineto{\pgfqpoint{1.412196in}{3.318188in}}%
\pgfpathlineto{\pgfqpoint{1.414797in}{3.579474in}}%
\pgfpathlineto{\pgfqpoint{1.415664in}{3.385609in}}%
\pgfpathlineto{\pgfqpoint{1.416531in}{3.437747in}}%
\pgfpathlineto{\pgfqpoint{1.417399in}{3.284373in}}%
\pgfpathlineto{\pgfqpoint{1.418266in}{3.560075in}}%
\pgfpathlineto{\pgfqpoint{1.419133in}{3.327573in}}%
\pgfpathlineto{\pgfqpoint{1.420000in}{3.685018in}}%
\pgfpathlineto{\pgfqpoint{1.420867in}{3.343395in}}%
\pgfpathlineto{\pgfqpoint{1.422601in}{3.791102in}}%
\pgfpathlineto{\pgfqpoint{1.424336in}{3.205078in}}%
\pgfpathlineto{\pgfqpoint{1.426070in}{3.531282in}}%
\pgfpathlineto{\pgfqpoint{1.428671in}{3.007658in}}%
\pgfpathlineto{\pgfqpoint{1.429538in}{3.409133in}}%
\pgfpathlineto{\pgfqpoint{1.430406in}{3.359590in}}%
\pgfpathlineto{\pgfqpoint{1.431273in}{2.978575in}}%
\pgfpathlineto{\pgfqpoint{1.432140in}{3.546663in}}%
\pgfpathlineto{\pgfqpoint{1.433007in}{3.144680in}}%
\pgfpathlineto{\pgfqpoint{1.434741in}{3.416434in}}%
\pgfpathlineto{\pgfqpoint{1.435608in}{3.430585in}}%
\pgfpathlineto{\pgfqpoint{1.436476in}{3.644113in}}%
\pgfpathlineto{\pgfqpoint{1.437343in}{3.568145in}}%
\pgfpathlineto{\pgfqpoint{1.439077in}{3.048502in}}%
\pgfpathlineto{\pgfqpoint{1.439944in}{3.613910in}}%
\pgfpathlineto{\pgfqpoint{1.440811in}{3.256215in}}%
\pgfpathlineto{\pgfqpoint{1.441678in}{3.270074in}}%
\pgfpathlineto{\pgfqpoint{1.442545in}{3.490908in}}%
\pgfpathlineto{\pgfqpoint{1.443413in}{3.425363in}}%
\pgfpathlineto{\pgfqpoint{1.444280in}{3.175497in}}%
\pgfpathlineto{\pgfqpoint{1.445147in}{3.410367in}}%
\pgfpathlineto{\pgfqpoint{1.446014in}{3.226325in}}%
\pgfpathlineto{\pgfqpoint{1.446881in}{3.525972in}}%
\pgfpathlineto{\pgfqpoint{1.447748in}{3.126819in}}%
\pgfpathlineto{\pgfqpoint{1.449483in}{3.805788in}}%
\pgfpathlineto{\pgfqpoint{1.452084in}{3.033680in}}%
\pgfpathlineto{\pgfqpoint{1.453818in}{3.613002in}}%
\pgfpathlineto{\pgfqpoint{1.454685in}{3.454282in}}%
\pgfpathlineto{\pgfqpoint{1.455552in}{3.091172in}}%
\pgfpathlineto{\pgfqpoint{1.457287in}{3.485589in}}%
\pgfpathlineto{\pgfqpoint{1.458154in}{3.492431in}}%
\pgfpathlineto{\pgfqpoint{1.459021in}{3.385405in}}%
\pgfpathlineto{\pgfqpoint{1.459888in}{3.422576in}}%
\pgfpathlineto{\pgfqpoint{1.460755in}{3.534823in}}%
\pgfpathlineto{\pgfqpoint{1.463357in}{3.241417in}}%
\pgfpathlineto{\pgfqpoint{1.464224in}{3.337862in}}%
\pgfpathlineto{\pgfqpoint{1.465091in}{2.943162in}}%
\pgfpathlineto{\pgfqpoint{1.465958in}{3.494524in}}%
\pgfpathlineto{\pgfqpoint{1.466825in}{3.258453in}}%
\pgfpathlineto{\pgfqpoint{1.467692in}{3.678489in}}%
\pgfpathlineto{\pgfqpoint{1.468559in}{3.420191in}}%
\pgfpathlineto{\pgfqpoint{1.469427in}{3.534129in}}%
\pgfpathlineto{\pgfqpoint{1.470294in}{3.122057in}}%
\pgfpathlineto{\pgfqpoint{1.471161in}{3.529350in}}%
\pgfpathlineto{\pgfqpoint{1.472895in}{3.081123in}}%
\pgfpathlineto{\pgfqpoint{1.473762in}{3.546906in}}%
\pgfpathlineto{\pgfqpoint{1.474629in}{3.190989in}}%
\pgfpathlineto{\pgfqpoint{1.475497in}{3.533301in}}%
\pgfpathlineto{\pgfqpoint{1.476364in}{3.165120in}}%
\pgfpathlineto{\pgfqpoint{1.477231in}{3.537850in}}%
\pgfpathlineto{\pgfqpoint{1.478098in}{2.924748in}}%
\pgfpathlineto{\pgfqpoint{1.479832in}{3.488470in}}%
\pgfpathlineto{\pgfqpoint{1.481566in}{3.167254in}}%
\pgfpathlineto{\pgfqpoint{1.482434in}{3.467175in}}%
\pgfpathlineto{\pgfqpoint{1.483301in}{3.297357in}}%
\pgfpathlineto{\pgfqpoint{1.484168in}{3.375695in}}%
\pgfpathlineto{\pgfqpoint{1.486769in}{3.204341in}}%
\pgfpathlineto{\pgfqpoint{1.488503in}{3.600376in}}%
\pgfpathlineto{\pgfqpoint{1.489371in}{3.340387in}}%
\pgfpathlineto{\pgfqpoint{1.490238in}{3.698274in}}%
\pgfpathlineto{\pgfqpoint{1.491105in}{3.575341in}}%
\pgfpathlineto{\pgfqpoint{1.491972in}{3.287639in}}%
\pgfpathlineto{\pgfqpoint{1.492839in}{3.304476in}}%
\pgfpathlineto{\pgfqpoint{1.493706in}{3.365242in}}%
\pgfpathlineto{\pgfqpoint{1.494573in}{3.285685in}}%
\pgfpathlineto{\pgfqpoint{1.495441in}{3.422294in}}%
\pgfpathlineto{\pgfqpoint{1.496308in}{2.823964in}}%
\pgfpathlineto{\pgfqpoint{1.498042in}{3.218382in}}%
\pgfpathlineto{\pgfqpoint{1.499776in}{3.426247in}}%
\pgfpathlineto{\pgfqpoint{1.500643in}{3.171333in}}%
\pgfpathlineto{\pgfqpoint{1.503245in}{3.515388in}}%
\pgfpathlineto{\pgfqpoint{1.504112in}{3.200156in}}%
\pgfpathlineto{\pgfqpoint{1.504979in}{3.332415in}}%
\pgfpathlineto{\pgfqpoint{1.506713in}{3.170938in}}%
\pgfpathlineto{\pgfqpoint{1.507580in}{3.257961in}}%
\pgfpathlineto{\pgfqpoint{1.508448in}{3.253701in}}%
\pgfpathlineto{\pgfqpoint{1.509315in}{2.836511in}}%
\pgfpathlineto{\pgfqpoint{1.511049in}{3.575326in}}%
\pgfpathlineto{\pgfqpoint{1.512783in}{3.188848in}}%
\pgfpathlineto{\pgfqpoint{1.513650in}{3.710579in}}%
\pgfpathlineto{\pgfqpoint{1.515385in}{2.932981in}}%
\pgfpathlineto{\pgfqpoint{1.517119in}{3.653900in}}%
\pgfpathlineto{\pgfqpoint{1.517986in}{3.251929in}}%
\pgfpathlineto{\pgfqpoint{1.518853in}{3.322070in}}%
\pgfpathlineto{\pgfqpoint{1.519720in}{3.216394in}}%
\pgfpathlineto{\pgfqpoint{1.523189in}{3.514255in}}%
\pgfpathlineto{\pgfqpoint{1.524056in}{3.230000in}}%
\pgfpathlineto{\pgfqpoint{1.524923in}{3.308777in}}%
\pgfpathlineto{\pgfqpoint{1.525790in}{3.375831in}}%
\pgfpathlineto{\pgfqpoint{1.527524in}{3.186320in}}%
\pgfpathlineto{\pgfqpoint{1.528392in}{3.462430in}}%
\pgfpathlineto{\pgfqpoint{1.529259in}{3.067641in}}%
\pgfpathlineto{\pgfqpoint{1.530993in}{3.640023in}}%
\pgfpathlineto{\pgfqpoint{1.531860in}{3.637581in}}%
\pgfpathlineto{\pgfqpoint{1.532727in}{3.628642in}}%
\pgfpathlineto{\pgfqpoint{1.533594in}{3.165521in}}%
\pgfpathlineto{\pgfqpoint{1.534462in}{3.433838in}}%
\pgfpathlineto{\pgfqpoint{1.535329in}{2.959229in}}%
\pgfpathlineto{\pgfqpoint{1.537063in}{3.527317in}}%
\pgfpathlineto{\pgfqpoint{1.537930in}{3.064461in}}%
\pgfpathlineto{\pgfqpoint{1.538797in}{3.328651in}}%
\pgfpathlineto{\pgfqpoint{1.539664in}{3.110722in}}%
\pgfpathlineto{\pgfqpoint{1.540531in}{3.507037in}}%
\pgfpathlineto{\pgfqpoint{1.541399in}{3.211424in}}%
\pgfpathlineto{\pgfqpoint{1.543133in}{3.535188in}}%
\pgfpathlineto{\pgfqpoint{1.544867in}{3.148207in}}%
\pgfpathlineto{\pgfqpoint{1.545734in}{3.496550in}}%
\pgfpathlineto{\pgfqpoint{1.546601in}{3.412416in}}%
\pgfpathlineto{\pgfqpoint{1.547469in}{3.052814in}}%
\pgfpathlineto{\pgfqpoint{1.549203in}{3.512866in}}%
\pgfpathlineto{\pgfqpoint{1.550070in}{3.342864in}}%
\pgfpathlineto{\pgfqpoint{1.550937in}{3.417456in}}%
\pgfpathlineto{\pgfqpoint{1.552671in}{3.122194in}}%
\pgfpathlineto{\pgfqpoint{1.554406in}{3.385082in}}%
\pgfpathlineto{\pgfqpoint{1.555273in}{3.211760in}}%
\pgfpathlineto{\pgfqpoint{1.556140in}{3.647728in}}%
\pgfpathlineto{\pgfqpoint{1.557007in}{3.361636in}}%
\pgfpathlineto{\pgfqpoint{1.557874in}{3.502761in}}%
\pgfpathlineto{\pgfqpoint{1.558741in}{3.476529in}}%
\pgfpathlineto{\pgfqpoint{1.559608in}{3.344136in}}%
\pgfpathlineto{\pgfqpoint{1.560476in}{3.872312in}}%
\pgfpathlineto{\pgfqpoint{1.562210in}{3.275289in}}%
\pgfpathlineto{\pgfqpoint{1.563077in}{3.316654in}}%
\pgfpathlineto{\pgfqpoint{1.563944in}{3.151229in}}%
\pgfpathlineto{\pgfqpoint{1.564811in}{3.180922in}}%
\pgfpathlineto{\pgfqpoint{1.565678in}{3.795703in}}%
\pgfpathlineto{\pgfqpoint{1.567413in}{3.135512in}}%
\pgfpathlineto{\pgfqpoint{1.568280in}{3.571014in}}%
\pgfpathlineto{\pgfqpoint{1.569147in}{3.413840in}}%
\pgfpathlineto{\pgfqpoint{1.570014in}{2.888336in}}%
\pgfpathlineto{\pgfqpoint{1.570881in}{3.462287in}}%
\pgfpathlineto{\pgfqpoint{1.571748in}{3.233496in}}%
\pgfpathlineto{\pgfqpoint{1.573483in}{3.592591in}}%
\pgfpathlineto{\pgfqpoint{1.575217in}{3.216985in}}%
\pgfpathlineto{\pgfqpoint{1.576084in}{3.470817in}}%
\pgfpathlineto{\pgfqpoint{1.576951in}{3.326156in}}%
\pgfpathlineto{\pgfqpoint{1.577818in}{3.419071in}}%
\pgfpathlineto{\pgfqpoint{1.578685in}{3.251760in}}%
\pgfpathlineto{\pgfqpoint{1.581287in}{3.464592in}}%
\pgfpathlineto{\pgfqpoint{1.582154in}{3.127823in}}%
\pgfpathlineto{\pgfqpoint{1.583888in}{3.716480in}}%
\pgfpathlineto{\pgfqpoint{1.584755in}{3.136202in}}%
\pgfpathlineto{\pgfqpoint{1.586490in}{3.802187in}}%
\pgfpathlineto{\pgfqpoint{1.587357in}{3.399406in}}%
\pgfpathlineto{\pgfqpoint{1.588224in}{3.696546in}}%
\pgfpathlineto{\pgfqpoint{1.589958in}{3.235161in}}%
\pgfpathlineto{\pgfqpoint{1.590825in}{3.366283in}}%
\pgfpathlineto{\pgfqpoint{1.591692in}{3.485979in}}%
\pgfpathlineto{\pgfqpoint{1.592559in}{3.213771in}}%
\pgfpathlineto{\pgfqpoint{1.594294in}{3.547977in}}%
\pgfpathlineto{\pgfqpoint{1.595161in}{3.558204in}}%
\pgfpathlineto{\pgfqpoint{1.596028in}{3.278698in}}%
\pgfpathlineto{\pgfqpoint{1.597762in}{3.485547in}}%
\pgfpathlineto{\pgfqpoint{1.599497in}{3.320654in}}%
\pgfpathlineto{\pgfqpoint{1.601231in}{3.494937in}}%
\pgfpathlineto{\pgfqpoint{1.602098in}{3.237452in}}%
\pgfpathlineto{\pgfqpoint{1.603832in}{3.582931in}}%
\pgfpathlineto{\pgfqpoint{1.604699in}{3.188943in}}%
\pgfpathlineto{\pgfqpoint{1.606434in}{3.501902in}}%
\pgfpathlineto{\pgfqpoint{1.608168in}{3.231768in}}%
\pgfpathlineto{\pgfqpoint{1.609035in}{3.621153in}}%
\pgfpathlineto{\pgfqpoint{1.609902in}{3.276733in}}%
\pgfpathlineto{\pgfqpoint{1.611636in}{3.905305in}}%
\pgfpathlineto{\pgfqpoint{1.612503in}{3.426424in}}%
\pgfpathlineto{\pgfqpoint{1.613371in}{3.427258in}}%
\pgfpathlineto{\pgfqpoint{1.615972in}{3.648782in}}%
\pgfpathlineto{\pgfqpoint{1.617706in}{3.485122in}}%
\pgfpathlineto{\pgfqpoint{1.618573in}{3.573705in}}%
\pgfpathlineto{\pgfqpoint{1.619441in}{3.388504in}}%
\pgfpathlineto{\pgfqpoint{1.620308in}{3.437890in}}%
\pgfpathlineto{\pgfqpoint{1.621175in}{3.662044in}}%
\pgfpathlineto{\pgfqpoint{1.622042in}{3.100966in}}%
\pgfpathlineto{\pgfqpoint{1.622909in}{3.157061in}}%
\pgfpathlineto{\pgfqpoint{1.623776in}{3.523801in}}%
\pgfpathlineto{\pgfqpoint{1.624643in}{3.467698in}}%
\pgfpathlineto{\pgfqpoint{1.625510in}{2.921079in}}%
\pgfpathlineto{\pgfqpoint{1.626378in}{3.601795in}}%
\pgfpathlineto{\pgfqpoint{1.627245in}{3.358080in}}%
\pgfpathlineto{\pgfqpoint{1.628112in}{3.659758in}}%
\pgfpathlineto{\pgfqpoint{1.628979in}{3.186344in}}%
\pgfpathlineto{\pgfqpoint{1.629846in}{3.374353in}}%
\pgfpathlineto{\pgfqpoint{1.630713in}{3.139727in}}%
\pgfpathlineto{\pgfqpoint{1.631580in}{3.682714in}}%
\pgfpathlineto{\pgfqpoint{1.632448in}{3.476014in}}%
\pgfpathlineto{\pgfqpoint{1.633315in}{3.512541in}}%
\pgfpathlineto{\pgfqpoint{1.634182in}{3.552363in}}%
\pgfpathlineto{\pgfqpoint{1.635049in}{3.150012in}}%
\pgfpathlineto{\pgfqpoint{1.635916in}{3.607739in}}%
\pgfpathlineto{\pgfqpoint{1.636783in}{3.555028in}}%
\pgfpathlineto{\pgfqpoint{1.637650in}{3.077129in}}%
\pgfpathlineto{\pgfqpoint{1.638517in}{3.188530in}}%
\pgfpathlineto{\pgfqpoint{1.639385in}{3.428782in}}%
\pgfpathlineto{\pgfqpoint{1.640252in}{3.258221in}}%
\pgfpathlineto{\pgfqpoint{1.641119in}{3.319060in}}%
\pgfpathlineto{\pgfqpoint{1.642853in}{3.518597in}}%
\pgfpathlineto{\pgfqpoint{1.643720in}{3.272412in}}%
\pgfpathlineto{\pgfqpoint{1.645455in}{3.430765in}}%
\pgfpathlineto{\pgfqpoint{1.646322in}{3.108542in}}%
\pgfpathlineto{\pgfqpoint{1.647189in}{3.817300in}}%
\pgfpathlineto{\pgfqpoint{1.649790in}{3.303475in}}%
\pgfpathlineto{\pgfqpoint{1.650657in}{3.321672in}}%
\pgfpathlineto{\pgfqpoint{1.651524in}{3.439847in}}%
\pgfpathlineto{\pgfqpoint{1.652392in}{3.413856in}}%
\pgfpathlineto{\pgfqpoint{1.654126in}{3.226539in}}%
\pgfpathlineto{\pgfqpoint{1.654993in}{3.592027in}}%
\pgfpathlineto{\pgfqpoint{1.655860in}{3.216289in}}%
\pgfpathlineto{\pgfqpoint{1.656727in}{3.570826in}}%
\pgfpathlineto{\pgfqpoint{1.657594in}{3.537177in}}%
\pgfpathlineto{\pgfqpoint{1.658462in}{3.565146in}}%
\pgfpathlineto{\pgfqpoint{1.659329in}{3.251124in}}%
\pgfpathlineto{\pgfqpoint{1.660196in}{3.251865in}}%
\pgfpathlineto{\pgfqpoint{1.661063in}{3.054954in}}%
\pgfpathlineto{\pgfqpoint{1.662797in}{3.628628in}}%
\pgfpathlineto{\pgfqpoint{1.664531in}{3.352207in}}%
\pgfpathlineto{\pgfqpoint{1.665399in}{3.395327in}}%
\pgfpathlineto{\pgfqpoint{1.666266in}{3.506655in}}%
\pgfpathlineto{\pgfqpoint{1.667133in}{3.490667in}}%
\pgfpathlineto{\pgfqpoint{1.668000in}{3.431148in}}%
\pgfpathlineto{\pgfqpoint{1.668867in}{3.237563in}}%
\pgfpathlineto{\pgfqpoint{1.669734in}{3.340306in}}%
\pgfpathlineto{\pgfqpoint{1.671469in}{3.188246in}}%
\pgfpathlineto{\pgfqpoint{1.673203in}{3.351655in}}%
\pgfpathlineto{\pgfqpoint{1.674070in}{3.292592in}}%
\pgfpathlineto{\pgfqpoint{1.674937in}{3.486908in}}%
\pgfpathlineto{\pgfqpoint{1.675804in}{3.473565in}}%
\pgfpathlineto{\pgfqpoint{1.676671in}{3.566498in}}%
\pgfpathlineto{\pgfqpoint{1.679273in}{3.305873in}}%
\pgfpathlineto{\pgfqpoint{1.680140in}{3.343740in}}%
\pgfpathlineto{\pgfqpoint{1.681007in}{3.478681in}}%
\pgfpathlineto{\pgfqpoint{1.681874in}{3.382533in}}%
\pgfpathlineto{\pgfqpoint{1.682741in}{3.482381in}}%
\pgfpathlineto{\pgfqpoint{1.684476in}{3.232729in}}%
\pgfpathlineto{\pgfqpoint{1.685343in}{3.380998in}}%
\pgfpathlineto{\pgfqpoint{1.686210in}{3.234641in}}%
\pgfpathlineto{\pgfqpoint{1.687077in}{3.406966in}}%
\pgfpathlineto{\pgfqpoint{1.687944in}{3.274352in}}%
\pgfpathlineto{\pgfqpoint{1.688811in}{3.352801in}}%
\pgfpathlineto{\pgfqpoint{1.689678in}{3.327987in}}%
\pgfpathlineto{\pgfqpoint{1.690545in}{3.484331in}}%
\pgfpathlineto{\pgfqpoint{1.692280in}{3.308746in}}%
\pgfpathlineto{\pgfqpoint{1.693147in}{3.051047in}}%
\pgfpathlineto{\pgfqpoint{1.694014in}{3.393494in}}%
\pgfpathlineto{\pgfqpoint{1.694881in}{3.331424in}}%
\pgfpathlineto{\pgfqpoint{1.695748in}{3.325230in}}%
\pgfpathlineto{\pgfqpoint{1.696615in}{3.069710in}}%
\pgfpathlineto{\pgfqpoint{1.698350in}{3.672504in}}%
\pgfpathlineto{\pgfqpoint{1.699217in}{3.240510in}}%
\pgfpathlineto{\pgfqpoint{1.700084in}{3.409220in}}%
\pgfpathlineto{\pgfqpoint{1.700951in}{3.343307in}}%
\pgfpathlineto{\pgfqpoint{1.701818in}{3.744346in}}%
\pgfpathlineto{\pgfqpoint{1.703552in}{3.261397in}}%
\pgfpathlineto{\pgfqpoint{1.704420in}{3.715559in}}%
\pgfpathlineto{\pgfqpoint{1.705287in}{3.346576in}}%
\pgfpathlineto{\pgfqpoint{1.707021in}{3.583285in}}%
\pgfpathlineto{\pgfqpoint{1.709622in}{3.121487in}}%
\pgfpathlineto{\pgfqpoint{1.711357in}{3.350264in}}%
\pgfpathlineto{\pgfqpoint{1.712224in}{3.198842in}}%
\pgfpathlineto{\pgfqpoint{1.713958in}{3.716939in}}%
\pgfpathlineto{\pgfqpoint{1.714825in}{3.273298in}}%
\pgfpathlineto{\pgfqpoint{1.715692in}{3.415037in}}%
\pgfpathlineto{\pgfqpoint{1.716559in}{3.972330in}}%
\pgfpathlineto{\pgfqpoint{1.719161in}{3.276974in}}%
\pgfpathlineto{\pgfqpoint{1.720895in}{3.123916in}}%
\pgfpathlineto{\pgfqpoint{1.721762in}{2.985520in}}%
\pgfpathlineto{\pgfqpoint{1.724364in}{3.508451in}}%
\pgfpathlineto{\pgfqpoint{1.726098in}{3.212079in}}%
\pgfpathlineto{\pgfqpoint{1.728699in}{3.660408in}}%
\pgfpathlineto{\pgfqpoint{1.729566in}{3.314415in}}%
\pgfpathlineto{\pgfqpoint{1.730434in}{3.466751in}}%
\pgfpathlineto{\pgfqpoint{1.731301in}{3.818066in}}%
\pgfpathlineto{\pgfqpoint{1.732168in}{3.765783in}}%
\pgfpathlineto{\pgfqpoint{1.734769in}{2.788146in}}%
\pgfpathlineto{\pgfqpoint{1.735636in}{3.118340in}}%
\pgfpathlineto{\pgfqpoint{1.736503in}{3.012745in}}%
\pgfpathlineto{\pgfqpoint{1.739105in}{3.427912in}}%
\pgfpathlineto{\pgfqpoint{1.739972in}{3.142445in}}%
\pgfpathlineto{\pgfqpoint{1.741706in}{3.968331in}}%
\pgfpathlineto{\pgfqpoint{1.744308in}{3.329061in}}%
\pgfpathlineto{\pgfqpoint{1.745175in}{3.653014in}}%
\pgfpathlineto{\pgfqpoint{1.746909in}{3.105190in}}%
\pgfpathlineto{\pgfqpoint{1.747776in}{3.140254in}}%
\pgfpathlineto{\pgfqpoint{1.748643in}{3.102972in}}%
\pgfpathlineto{\pgfqpoint{1.749510in}{3.308336in}}%
\pgfpathlineto{\pgfqpoint{1.750378in}{3.131747in}}%
\pgfpathlineto{\pgfqpoint{1.752112in}{3.775035in}}%
\pgfpathlineto{\pgfqpoint{1.753846in}{3.406493in}}%
\pgfpathlineto{\pgfqpoint{1.754713in}{3.767242in}}%
\pgfpathlineto{\pgfqpoint{1.755580in}{3.397441in}}%
\pgfpathlineto{\pgfqpoint{1.756448in}{3.706679in}}%
\pgfpathlineto{\pgfqpoint{1.757315in}{3.636869in}}%
\pgfpathlineto{\pgfqpoint{1.758182in}{3.498306in}}%
\pgfpathlineto{\pgfqpoint{1.759049in}{3.621405in}}%
\pgfpathlineto{\pgfqpoint{1.759916in}{2.910386in}}%
\pgfpathlineto{\pgfqpoint{1.763385in}{3.536214in}}%
\pgfpathlineto{\pgfqpoint{1.764252in}{3.744638in}}%
\pgfpathlineto{\pgfqpoint{1.765119in}{3.682071in}}%
\pgfpathlineto{\pgfqpoint{1.765986in}{3.507026in}}%
\pgfpathlineto{\pgfqpoint{1.766853in}{3.537178in}}%
\pgfpathlineto{\pgfqpoint{1.767720in}{3.612928in}}%
\pgfpathlineto{\pgfqpoint{1.769455in}{3.235988in}}%
\pgfpathlineto{\pgfqpoint{1.770322in}{3.336640in}}%
\pgfpathlineto{\pgfqpoint{1.771189in}{3.660567in}}%
\pgfpathlineto{\pgfqpoint{1.772056in}{3.250736in}}%
\pgfpathlineto{\pgfqpoint{1.772923in}{3.740213in}}%
\pgfpathlineto{\pgfqpoint{1.773790in}{3.461245in}}%
\pgfpathlineto{\pgfqpoint{1.774657in}{3.507579in}}%
\pgfpathlineto{\pgfqpoint{1.775524in}{3.472399in}}%
\pgfpathlineto{\pgfqpoint{1.776392in}{3.563244in}}%
\pgfpathlineto{\pgfqpoint{1.777259in}{3.280512in}}%
\pgfpathlineto{\pgfqpoint{1.778126in}{3.588427in}}%
\pgfpathlineto{\pgfqpoint{1.778993in}{3.209717in}}%
\pgfpathlineto{\pgfqpoint{1.779860in}{3.456023in}}%
\pgfpathlineto{\pgfqpoint{1.781594in}{3.279505in}}%
\pgfpathlineto{\pgfqpoint{1.782462in}{3.251258in}}%
\pgfpathlineto{\pgfqpoint{1.783329in}{3.142245in}}%
\pgfpathlineto{\pgfqpoint{1.784196in}{3.673181in}}%
\pgfpathlineto{\pgfqpoint{1.785063in}{3.650301in}}%
\pgfpathlineto{\pgfqpoint{1.785930in}{3.605368in}}%
\pgfpathlineto{\pgfqpoint{1.786797in}{3.618240in}}%
\pgfpathlineto{\pgfqpoint{1.787664in}{3.037958in}}%
\pgfpathlineto{\pgfqpoint{1.788531in}{3.179938in}}%
\pgfpathlineto{\pgfqpoint{1.789399in}{3.211877in}}%
\pgfpathlineto{\pgfqpoint{1.791133in}{3.755641in}}%
\pgfpathlineto{\pgfqpoint{1.792000in}{3.140181in}}%
\pgfpathlineto{\pgfqpoint{1.792867in}{3.216076in}}%
\pgfpathlineto{\pgfqpoint{1.794601in}{3.659140in}}%
\pgfpathlineto{\pgfqpoint{1.795469in}{3.738039in}}%
\pgfpathlineto{\pgfqpoint{1.796336in}{3.607293in}}%
\pgfpathlineto{\pgfqpoint{1.797203in}{3.012018in}}%
\pgfpathlineto{\pgfqpoint{1.798937in}{3.718475in}}%
\pgfpathlineto{\pgfqpoint{1.799804in}{2.889822in}}%
\pgfpathlineto{\pgfqpoint{1.801538in}{3.534127in}}%
\pgfpathlineto{\pgfqpoint{1.802406in}{3.246050in}}%
\pgfpathlineto{\pgfqpoint{1.803273in}{3.572924in}}%
\pgfpathlineto{\pgfqpoint{1.804140in}{3.450496in}}%
\pgfpathlineto{\pgfqpoint{1.805007in}{3.528556in}}%
\pgfpathlineto{\pgfqpoint{1.805874in}{3.132754in}}%
\pgfpathlineto{\pgfqpoint{1.806741in}{3.634265in}}%
\pgfpathlineto{\pgfqpoint{1.808476in}{3.315672in}}%
\pgfpathlineto{\pgfqpoint{1.810210in}{3.652827in}}%
\pgfpathlineto{\pgfqpoint{1.811077in}{3.132580in}}%
\pgfpathlineto{\pgfqpoint{1.812811in}{3.489168in}}%
\pgfpathlineto{\pgfqpoint{1.813678in}{3.348741in}}%
\pgfpathlineto{\pgfqpoint{1.814545in}{3.770477in}}%
\pgfpathlineto{\pgfqpoint{1.815413in}{3.302016in}}%
\pgfpathlineto{\pgfqpoint{1.816280in}{3.328226in}}%
\pgfpathlineto{\pgfqpoint{1.817147in}{3.399972in}}%
\pgfpathlineto{\pgfqpoint{1.818014in}{3.351345in}}%
\pgfpathlineto{\pgfqpoint{1.819748in}{2.794712in}}%
\pgfpathlineto{\pgfqpoint{1.820615in}{3.755223in}}%
\pgfpathlineto{\pgfqpoint{1.822350in}{3.091579in}}%
\pgfpathlineto{\pgfqpoint{1.823217in}{3.455386in}}%
\pgfpathlineto{\pgfqpoint{1.824084in}{3.212866in}}%
\pgfpathlineto{\pgfqpoint{1.824951in}{3.852061in}}%
\pgfpathlineto{\pgfqpoint{1.826685in}{3.233050in}}%
\pgfpathlineto{\pgfqpoint{1.828420in}{3.639991in}}%
\pgfpathlineto{\pgfqpoint{1.830154in}{3.288753in}}%
\pgfpathlineto{\pgfqpoint{1.831888in}{3.735007in}}%
\pgfpathlineto{\pgfqpoint{1.834490in}{3.096548in}}%
\pgfpathlineto{\pgfqpoint{1.835357in}{3.277498in}}%
\pgfpathlineto{\pgfqpoint{1.836224in}{3.026091in}}%
\pgfpathlineto{\pgfqpoint{1.837958in}{3.553016in}}%
\pgfpathlineto{\pgfqpoint{1.838825in}{3.532137in}}%
\pgfpathlineto{\pgfqpoint{1.840559in}{3.149118in}}%
\pgfpathlineto{\pgfqpoint{1.841427in}{3.479280in}}%
\pgfpathlineto{\pgfqpoint{1.842294in}{3.290702in}}%
\pgfpathlineto{\pgfqpoint{1.843161in}{3.322466in}}%
\pgfpathlineto{\pgfqpoint{1.845762in}{3.499288in}}%
\pgfpathlineto{\pgfqpoint{1.846629in}{3.246644in}}%
\pgfpathlineto{\pgfqpoint{1.847497in}{3.279930in}}%
\pgfpathlineto{\pgfqpoint{1.848364in}{3.361359in}}%
\pgfpathlineto{\pgfqpoint{1.849231in}{3.649672in}}%
\pgfpathlineto{\pgfqpoint{1.850965in}{3.340258in}}%
\pgfpathlineto{\pgfqpoint{1.851832in}{3.537762in}}%
\pgfpathlineto{\pgfqpoint{1.852699in}{3.182838in}}%
\pgfpathlineto{\pgfqpoint{1.853566in}{3.287827in}}%
\pgfpathlineto{\pgfqpoint{1.855301in}{3.742087in}}%
\pgfpathlineto{\pgfqpoint{1.856168in}{3.396017in}}%
\pgfpathlineto{\pgfqpoint{1.857035in}{3.419213in}}%
\pgfpathlineto{\pgfqpoint{1.857902in}{3.886605in}}%
\pgfpathlineto{\pgfqpoint{1.858769in}{3.091134in}}%
\pgfpathlineto{\pgfqpoint{1.859636in}{3.286662in}}%
\pgfpathlineto{\pgfqpoint{1.860503in}{3.049765in}}%
\pgfpathlineto{\pgfqpoint{1.861371in}{3.671884in}}%
\pgfpathlineto{\pgfqpoint{1.862238in}{3.603288in}}%
\pgfpathlineto{\pgfqpoint{1.863972in}{3.283068in}}%
\pgfpathlineto{\pgfqpoint{1.864839in}{3.728225in}}%
\pgfpathlineto{\pgfqpoint{1.866573in}{3.212362in}}%
\pgfpathlineto{\pgfqpoint{1.867441in}{3.273523in}}%
\pgfpathlineto{\pgfqpoint{1.868308in}{3.605384in}}%
\pgfpathlineto{\pgfqpoint{1.869175in}{3.145315in}}%
\pgfpathlineto{\pgfqpoint{1.870909in}{3.785804in}}%
\pgfpathlineto{\pgfqpoint{1.871776in}{3.353980in}}%
\pgfpathlineto{\pgfqpoint{1.872643in}{3.376582in}}%
\pgfpathlineto{\pgfqpoint{1.873510in}{3.105048in}}%
\pgfpathlineto{\pgfqpoint{1.874378in}{3.546332in}}%
\pgfpathlineto{\pgfqpoint{1.876112in}{3.306812in}}%
\pgfpathlineto{\pgfqpoint{1.876979in}{3.237433in}}%
\pgfpathlineto{\pgfqpoint{1.877846in}{3.435175in}}%
\pgfpathlineto{\pgfqpoint{1.878713in}{4.002595in}}%
\pgfpathlineto{\pgfqpoint{1.879580in}{3.493709in}}%
\pgfpathlineto{\pgfqpoint{1.880448in}{3.494770in}}%
\pgfpathlineto{\pgfqpoint{1.881315in}{3.635299in}}%
\pgfpathlineto{\pgfqpoint{1.882182in}{3.205653in}}%
\pgfpathlineto{\pgfqpoint{1.883049in}{3.303573in}}%
\pgfpathlineto{\pgfqpoint{1.883916in}{3.464808in}}%
\pgfpathlineto{\pgfqpoint{1.884783in}{3.366327in}}%
\pgfpathlineto{\pgfqpoint{1.885650in}{2.808170in}}%
\pgfpathlineto{\pgfqpoint{1.886517in}{3.395838in}}%
\pgfpathlineto{\pgfqpoint{1.887385in}{3.362740in}}%
\pgfpathlineto{\pgfqpoint{1.888252in}{3.059190in}}%
\pgfpathlineto{\pgfqpoint{1.889119in}{3.390995in}}%
\pgfpathlineto{\pgfqpoint{1.889986in}{3.320855in}}%
\pgfpathlineto{\pgfqpoint{1.890853in}{3.408940in}}%
\pgfpathlineto{\pgfqpoint{1.892587in}{3.770541in}}%
\pgfpathlineto{\pgfqpoint{1.893455in}{3.725885in}}%
\pgfpathlineto{\pgfqpoint{1.895189in}{3.069908in}}%
\pgfpathlineto{\pgfqpoint{1.896056in}{3.407302in}}%
\pgfpathlineto{\pgfqpoint{1.896923in}{3.293394in}}%
\pgfpathlineto{\pgfqpoint{1.898657in}{3.421685in}}%
\pgfpathlineto{\pgfqpoint{1.900392in}{3.708674in}}%
\pgfpathlineto{\pgfqpoint{1.901259in}{3.526368in}}%
\pgfpathlineto{\pgfqpoint{1.902126in}{3.754291in}}%
\pgfpathlineto{\pgfqpoint{1.903860in}{3.257265in}}%
\pgfpathlineto{\pgfqpoint{1.904727in}{3.187132in}}%
\pgfpathlineto{\pgfqpoint{1.905594in}{3.683671in}}%
\pgfpathlineto{\pgfqpoint{1.906462in}{3.372545in}}%
\pgfpathlineto{\pgfqpoint{1.908196in}{3.649677in}}%
\pgfpathlineto{\pgfqpoint{1.909063in}{3.571334in}}%
\pgfpathlineto{\pgfqpoint{1.910797in}{3.230196in}}%
\pgfpathlineto{\pgfqpoint{1.911664in}{3.455455in}}%
\pgfpathlineto{\pgfqpoint{1.912531in}{3.333232in}}%
\pgfpathlineto{\pgfqpoint{1.914266in}{3.600276in}}%
\pgfpathlineto{\pgfqpoint{1.915133in}{3.588765in}}%
\pgfpathlineto{\pgfqpoint{1.916000in}{3.594831in}}%
\pgfpathlineto{\pgfqpoint{1.916867in}{3.092716in}}%
\pgfpathlineto{\pgfqpoint{1.917734in}{3.214689in}}%
\pgfpathlineto{\pgfqpoint{1.919469in}{3.599299in}}%
\pgfpathlineto{\pgfqpoint{1.921203in}{3.691556in}}%
\pgfpathlineto{\pgfqpoint{1.922070in}{3.710202in}}%
\pgfpathlineto{\pgfqpoint{1.922937in}{2.707678in}}%
\pgfpathlineto{\pgfqpoint{1.924671in}{3.304605in}}%
\pgfpathlineto{\pgfqpoint{1.925538in}{3.218535in}}%
\pgfpathlineto{\pgfqpoint{1.926406in}{3.693472in}}%
\pgfpathlineto{\pgfqpoint{1.927273in}{3.172004in}}%
\pgfpathlineto{\pgfqpoint{1.928140in}{3.476086in}}%
\pgfpathlineto{\pgfqpoint{1.929007in}{3.375985in}}%
\pgfpathlineto{\pgfqpoint{1.929874in}{2.967466in}}%
\pgfpathlineto{\pgfqpoint{1.931608in}{3.611826in}}%
\pgfpathlineto{\pgfqpoint{1.932476in}{3.094806in}}%
\pgfpathlineto{\pgfqpoint{1.933343in}{3.460285in}}%
\pgfpathlineto{\pgfqpoint{1.934210in}{3.378074in}}%
\pgfpathlineto{\pgfqpoint{1.935077in}{3.520113in}}%
\pgfpathlineto{\pgfqpoint{1.935944in}{3.175953in}}%
\pgfpathlineto{\pgfqpoint{1.936811in}{3.217823in}}%
\pgfpathlineto{\pgfqpoint{1.937678in}{3.384553in}}%
\pgfpathlineto{\pgfqpoint{1.938545in}{3.256268in}}%
\pgfpathlineto{\pgfqpoint{1.940280in}{3.785958in}}%
\pgfpathlineto{\pgfqpoint{1.942014in}{3.181643in}}%
\pgfpathlineto{\pgfqpoint{1.942881in}{3.449159in}}%
\pgfpathlineto{\pgfqpoint{1.943748in}{3.306560in}}%
\pgfpathlineto{\pgfqpoint{1.944615in}{3.314226in}}%
\pgfpathlineto{\pgfqpoint{1.946350in}{3.599441in}}%
\pgfpathlineto{\pgfqpoint{1.948084in}{3.218159in}}%
\pgfpathlineto{\pgfqpoint{1.948951in}{3.434466in}}%
\pgfpathlineto{\pgfqpoint{1.949818in}{2.804610in}}%
\pgfpathlineto{\pgfqpoint{1.950685in}{3.557561in}}%
\pgfpathlineto{\pgfqpoint{1.951552in}{3.452274in}}%
\pgfpathlineto{\pgfqpoint{1.952420in}{3.537612in}}%
\pgfpathlineto{\pgfqpoint{1.955021in}{3.184453in}}%
\pgfpathlineto{\pgfqpoint{1.955888in}{3.624631in}}%
\pgfpathlineto{\pgfqpoint{1.956755in}{3.408454in}}%
\pgfpathlineto{\pgfqpoint{1.957622in}{3.485849in}}%
\pgfpathlineto{\pgfqpoint{1.958490in}{3.728421in}}%
\pgfpathlineto{\pgfqpoint{1.959357in}{3.396536in}}%
\pgfpathlineto{\pgfqpoint{1.960224in}{3.553248in}}%
\pgfpathlineto{\pgfqpoint{1.961091in}{3.231474in}}%
\pgfpathlineto{\pgfqpoint{1.963692in}{3.630149in}}%
\pgfpathlineto{\pgfqpoint{1.966294in}{3.327043in}}%
\pgfpathlineto{\pgfqpoint{1.967161in}{3.338565in}}%
\pgfpathlineto{\pgfqpoint{1.968028in}{3.496937in}}%
\pgfpathlineto{\pgfqpoint{1.968895in}{3.319851in}}%
\pgfpathlineto{\pgfqpoint{1.969762in}{3.538807in}}%
\pgfpathlineto{\pgfqpoint{1.970629in}{3.489178in}}%
\pgfpathlineto{\pgfqpoint{1.972364in}{3.031924in}}%
\pgfpathlineto{\pgfqpoint{1.973231in}{3.508473in}}%
\pgfpathlineto{\pgfqpoint{1.974098in}{3.251475in}}%
\pgfpathlineto{\pgfqpoint{1.974965in}{3.288987in}}%
\pgfpathlineto{\pgfqpoint{1.975832in}{3.878405in}}%
\pgfpathlineto{\pgfqpoint{1.976699in}{3.279931in}}%
\pgfpathlineto{\pgfqpoint{1.977566in}{3.362529in}}%
\pgfpathlineto{\pgfqpoint{1.978434in}{3.360919in}}%
\pgfpathlineto{\pgfqpoint{1.979301in}{3.378013in}}%
\pgfpathlineto{\pgfqpoint{1.980168in}{3.795259in}}%
\pgfpathlineto{\pgfqpoint{1.981035in}{3.314639in}}%
\pgfpathlineto{\pgfqpoint{1.981902in}{3.551556in}}%
\pgfpathlineto{\pgfqpoint{1.982769in}{3.542781in}}%
\pgfpathlineto{\pgfqpoint{1.983636in}{3.000231in}}%
\pgfpathlineto{\pgfqpoint{1.984503in}{3.483095in}}%
\pgfpathlineto{\pgfqpoint{1.985371in}{3.380711in}}%
\pgfpathlineto{\pgfqpoint{1.986238in}{3.697340in}}%
\pgfpathlineto{\pgfqpoint{1.987105in}{3.365306in}}%
\pgfpathlineto{\pgfqpoint{1.987972in}{3.418137in}}%
\pgfpathlineto{\pgfqpoint{1.988839in}{3.492569in}}%
\pgfpathlineto{\pgfqpoint{1.989706in}{3.004735in}}%
\pgfpathlineto{\pgfqpoint{1.991441in}{3.401878in}}%
\pgfpathlineto{\pgfqpoint{1.993175in}{3.658649in}}%
\pgfpathlineto{\pgfqpoint{1.994909in}{3.112351in}}%
\pgfpathlineto{\pgfqpoint{1.996643in}{3.430986in}}%
\pgfpathlineto{\pgfqpoint{1.998378in}{3.661719in}}%
\pgfpathlineto{\pgfqpoint{2.000979in}{3.328513in}}%
\pgfpathlineto{\pgfqpoint{2.002713in}{3.548217in}}%
\pgfpathlineto{\pgfqpoint{2.003580in}{3.273595in}}%
\pgfpathlineto{\pgfqpoint{2.004448in}{3.426405in}}%
\pgfpathlineto{\pgfqpoint{2.005315in}{3.085339in}}%
\pgfpathlineto{\pgfqpoint{2.007049in}{3.565305in}}%
\pgfpathlineto{\pgfqpoint{2.007916in}{3.545744in}}%
\pgfpathlineto{\pgfqpoint{2.008783in}{3.372771in}}%
\pgfpathlineto{\pgfqpoint{2.009650in}{3.696448in}}%
\pgfpathlineto{\pgfqpoint{2.011385in}{3.149616in}}%
\pgfpathlineto{\pgfqpoint{2.012252in}{3.019526in}}%
\pgfpathlineto{\pgfqpoint{2.013119in}{3.122569in}}%
\pgfpathlineto{\pgfqpoint{2.014853in}{3.540442in}}%
\pgfpathlineto{\pgfqpoint{2.015720in}{3.584134in}}%
\pgfpathlineto{\pgfqpoint{2.017455in}{3.004225in}}%
\pgfpathlineto{\pgfqpoint{2.019189in}{3.745203in}}%
\pgfpathlineto{\pgfqpoint{2.021790in}{3.109063in}}%
\pgfpathlineto{\pgfqpoint{2.023524in}{3.503340in}}%
\pgfpathlineto{\pgfqpoint{2.024392in}{3.671669in}}%
\pgfpathlineto{\pgfqpoint{2.025259in}{3.265146in}}%
\pgfpathlineto{\pgfqpoint{2.026126in}{3.560699in}}%
\pgfpathlineto{\pgfqpoint{2.026993in}{3.217829in}}%
\pgfpathlineto{\pgfqpoint{2.030462in}{3.670464in}}%
\pgfpathlineto{\pgfqpoint{2.033063in}{3.148642in}}%
\pgfpathlineto{\pgfqpoint{2.034797in}{3.324806in}}%
\pgfpathlineto{\pgfqpoint{2.036531in}{3.210989in}}%
\pgfpathlineto{\pgfqpoint{2.037399in}{3.015303in}}%
\pgfpathlineto{\pgfqpoint{2.038266in}{3.408347in}}%
\pgfpathlineto{\pgfqpoint{2.039133in}{3.138015in}}%
\pgfpathlineto{\pgfqpoint{2.040000in}{3.747699in}}%
\pgfpathlineto{\pgfqpoint{2.041734in}{3.046937in}}%
\pgfpathlineto{\pgfqpoint{2.042601in}{3.079526in}}%
\pgfpathlineto{\pgfqpoint{2.043469in}{3.123449in}}%
\pgfpathlineto{\pgfqpoint{2.044336in}{3.704381in}}%
\pgfpathlineto{\pgfqpoint{2.045203in}{3.579999in}}%
\pgfpathlineto{\pgfqpoint{2.046070in}{3.560449in}}%
\pgfpathlineto{\pgfqpoint{2.047804in}{3.311369in}}%
\pgfpathlineto{\pgfqpoint{2.048671in}{3.065100in}}%
\pgfpathlineto{\pgfqpoint{2.051273in}{3.638617in}}%
\pgfpathlineto{\pgfqpoint{2.053007in}{3.385332in}}%
\pgfpathlineto{\pgfqpoint{2.053874in}{3.572024in}}%
\pgfpathlineto{\pgfqpoint{2.054741in}{3.565348in}}%
\pgfpathlineto{\pgfqpoint{2.056476in}{3.184029in}}%
\pgfpathlineto{\pgfqpoint{2.057343in}{3.187889in}}%
\pgfpathlineto{\pgfqpoint{2.058210in}{3.644501in}}%
\pgfpathlineto{\pgfqpoint{2.059077in}{3.593941in}}%
\pgfpathlineto{\pgfqpoint{2.061678in}{3.167719in}}%
\pgfpathlineto{\pgfqpoint{2.062545in}{3.256575in}}%
\pgfpathlineto{\pgfqpoint{2.063413in}{3.206468in}}%
\pgfpathlineto{\pgfqpoint{2.064280in}{3.616011in}}%
\pgfpathlineto{\pgfqpoint{2.066881in}{3.196983in}}%
\pgfpathlineto{\pgfqpoint{2.067748in}{3.709345in}}%
\pgfpathlineto{\pgfqpoint{2.068615in}{3.207816in}}%
\pgfpathlineto{\pgfqpoint{2.069483in}{3.531444in}}%
\pgfpathlineto{\pgfqpoint{2.070350in}{3.372944in}}%
\pgfpathlineto{\pgfqpoint{2.071217in}{3.427492in}}%
\pgfpathlineto{\pgfqpoint{2.072084in}{3.080945in}}%
\pgfpathlineto{\pgfqpoint{2.073818in}{3.620022in}}%
\pgfpathlineto{\pgfqpoint{2.075552in}{3.373288in}}%
\pgfpathlineto{\pgfqpoint{2.076420in}{3.569249in}}%
\pgfpathlineto{\pgfqpoint{2.077287in}{3.338115in}}%
\pgfpathlineto{\pgfqpoint{2.079021in}{4.019566in}}%
\pgfpathlineto{\pgfqpoint{2.079888in}{3.325594in}}%
\pgfpathlineto{\pgfqpoint{2.080755in}{3.477251in}}%
\pgfpathlineto{\pgfqpoint{2.081622in}{3.141993in}}%
\pgfpathlineto{\pgfqpoint{2.082490in}{3.256496in}}%
\pgfpathlineto{\pgfqpoint{2.083357in}{3.653729in}}%
\pgfpathlineto{\pgfqpoint{2.084224in}{2.978379in}}%
\pgfpathlineto{\pgfqpoint{2.085091in}{3.787766in}}%
\pgfpathlineto{\pgfqpoint{2.086825in}{3.045009in}}%
\pgfpathlineto{\pgfqpoint{2.087692in}{3.238478in}}%
\pgfpathlineto{\pgfqpoint{2.088559in}{3.765291in}}%
\pgfpathlineto{\pgfqpoint{2.089427in}{3.650187in}}%
\pgfpathlineto{\pgfqpoint{2.090294in}{3.063862in}}%
\pgfpathlineto{\pgfqpoint{2.092895in}{3.652975in}}%
\pgfpathlineto{\pgfqpoint{2.093762in}{3.115594in}}%
\pgfpathlineto{\pgfqpoint{2.094629in}{3.303140in}}%
\pgfpathlineto{\pgfqpoint{2.095497in}{3.296017in}}%
\pgfpathlineto{\pgfqpoint{2.098098in}{3.667669in}}%
\pgfpathlineto{\pgfqpoint{2.098965in}{3.208592in}}%
\pgfpathlineto{\pgfqpoint{2.101566in}{3.887929in}}%
\pgfpathlineto{\pgfqpoint{2.104168in}{3.208269in}}%
\pgfpathlineto{\pgfqpoint{2.105035in}{3.401688in}}%
\pgfpathlineto{\pgfqpoint{2.105902in}{3.875465in}}%
\pgfpathlineto{\pgfqpoint{2.107636in}{3.266051in}}%
\pgfpathlineto{\pgfqpoint{2.108503in}{3.324303in}}%
\pgfpathlineto{\pgfqpoint{2.109371in}{3.686301in}}%
\pgfpathlineto{\pgfqpoint{2.110238in}{3.619297in}}%
\pgfpathlineto{\pgfqpoint{2.111105in}{3.642027in}}%
\pgfpathlineto{\pgfqpoint{2.111972in}{3.192409in}}%
\pgfpathlineto{\pgfqpoint{2.112839in}{3.256010in}}%
\pgfpathlineto{\pgfqpoint{2.114573in}{3.785212in}}%
\pgfpathlineto{\pgfqpoint{2.116308in}{3.257289in}}%
\pgfpathlineto{\pgfqpoint{2.118909in}{3.776969in}}%
\pgfpathlineto{\pgfqpoint{2.120643in}{3.185440in}}%
\pgfpathlineto{\pgfqpoint{2.121510in}{3.684186in}}%
\pgfpathlineto{\pgfqpoint{2.123245in}{3.322604in}}%
\pgfpathlineto{\pgfqpoint{2.124112in}{3.381694in}}%
\pgfpathlineto{\pgfqpoint{2.124979in}{3.379107in}}%
\pgfpathlineto{\pgfqpoint{2.125846in}{3.151226in}}%
\pgfpathlineto{\pgfqpoint{2.126713in}{3.691500in}}%
\pgfpathlineto{\pgfqpoint{2.130182in}{2.917246in}}%
\pgfpathlineto{\pgfqpoint{2.131049in}{3.525535in}}%
\pgfpathlineto{\pgfqpoint{2.131916in}{3.512997in}}%
\pgfpathlineto{\pgfqpoint{2.132783in}{3.793953in}}%
\pgfpathlineto{\pgfqpoint{2.134517in}{3.062674in}}%
\pgfpathlineto{\pgfqpoint{2.135385in}{3.597497in}}%
\pgfpathlineto{\pgfqpoint{2.136252in}{3.501538in}}%
\pgfpathlineto{\pgfqpoint{2.137119in}{3.229641in}}%
\pgfpathlineto{\pgfqpoint{2.137986in}{3.414165in}}%
\pgfpathlineto{\pgfqpoint{2.138853in}{3.313068in}}%
\pgfpathlineto{\pgfqpoint{2.140587in}{3.438882in}}%
\pgfpathlineto{\pgfqpoint{2.141455in}{3.309361in}}%
\pgfpathlineto{\pgfqpoint{2.144056in}{3.627772in}}%
\pgfpathlineto{\pgfqpoint{2.144923in}{3.548119in}}%
\pgfpathlineto{\pgfqpoint{2.146657in}{3.067241in}}%
\pgfpathlineto{\pgfqpoint{2.147524in}{3.483953in}}%
\pgfpathlineto{\pgfqpoint{2.148392in}{3.322129in}}%
\pgfpathlineto{\pgfqpoint{2.149259in}{3.541489in}}%
\pgfpathlineto{\pgfqpoint{2.150126in}{3.062917in}}%
\pgfpathlineto{\pgfqpoint{2.150993in}{3.424713in}}%
\pgfpathlineto{\pgfqpoint{2.151860in}{3.153514in}}%
\pgfpathlineto{\pgfqpoint{2.153594in}{3.558435in}}%
\pgfpathlineto{\pgfqpoint{2.154462in}{3.478380in}}%
\pgfpathlineto{\pgfqpoint{2.155329in}{3.041322in}}%
\pgfpathlineto{\pgfqpoint{2.156196in}{3.891166in}}%
\pgfpathlineto{\pgfqpoint{2.157063in}{3.486804in}}%
\pgfpathlineto{\pgfqpoint{2.157930in}{3.521911in}}%
\pgfpathlineto{\pgfqpoint{2.159664in}{3.434723in}}%
\pgfpathlineto{\pgfqpoint{2.160531in}{3.626718in}}%
\pgfpathlineto{\pgfqpoint{2.161399in}{3.545444in}}%
\pgfpathlineto{\pgfqpoint{2.162266in}{3.055300in}}%
\pgfpathlineto{\pgfqpoint{2.164000in}{3.615483in}}%
\pgfpathlineto{\pgfqpoint{2.164867in}{3.239345in}}%
\pgfpathlineto{\pgfqpoint{2.165734in}{3.362684in}}%
\pgfpathlineto{\pgfqpoint{2.166601in}{3.176029in}}%
\pgfpathlineto{\pgfqpoint{2.167469in}{3.266328in}}%
\pgfpathlineto{\pgfqpoint{2.169203in}{3.713103in}}%
\pgfpathlineto{\pgfqpoint{2.170937in}{3.176532in}}%
\pgfpathlineto{\pgfqpoint{2.171804in}{3.386808in}}%
\pgfpathlineto{\pgfqpoint{2.172671in}{3.361448in}}%
\pgfpathlineto{\pgfqpoint{2.173538in}{3.392291in}}%
\pgfpathlineto{\pgfqpoint{2.175273in}{3.114538in}}%
\pgfpathlineto{\pgfqpoint{2.177007in}{3.528753in}}%
\pgfpathlineto{\pgfqpoint{2.177874in}{3.404664in}}%
\pgfpathlineto{\pgfqpoint{2.178741in}{3.434303in}}%
\pgfpathlineto{\pgfqpoint{2.179608in}{3.316824in}}%
\pgfpathlineto{\pgfqpoint{2.181343in}{3.504980in}}%
\pgfpathlineto{\pgfqpoint{2.182210in}{3.380860in}}%
\pgfpathlineto{\pgfqpoint{2.183077in}{3.045497in}}%
\pgfpathlineto{\pgfqpoint{2.184811in}{3.450962in}}%
\pgfpathlineto{\pgfqpoint{2.185678in}{3.072560in}}%
\pgfpathlineto{\pgfqpoint{2.188280in}{3.742848in}}%
\pgfpathlineto{\pgfqpoint{2.190014in}{3.382732in}}%
\pgfpathlineto{\pgfqpoint{2.190881in}{3.267447in}}%
\pgfpathlineto{\pgfqpoint{2.191748in}{3.794195in}}%
\pgfpathlineto{\pgfqpoint{2.192615in}{3.647746in}}%
\pgfpathlineto{\pgfqpoint{2.193483in}{2.981133in}}%
\pgfpathlineto{\pgfqpoint{2.195217in}{3.587558in}}%
\pgfpathlineto{\pgfqpoint{2.196084in}{3.165348in}}%
\pgfpathlineto{\pgfqpoint{2.197818in}{3.654708in}}%
\pgfpathlineto{\pgfqpoint{2.198685in}{3.363645in}}%
\pgfpathlineto{\pgfqpoint{2.199552in}{3.432496in}}%
\pgfpathlineto{\pgfqpoint{2.200420in}{3.637632in}}%
\pgfpathlineto{\pgfqpoint{2.201287in}{3.375832in}}%
\pgfpathlineto{\pgfqpoint{2.202154in}{3.406199in}}%
\pgfpathlineto{\pgfqpoint{2.203021in}{3.487807in}}%
\pgfpathlineto{\pgfqpoint{2.203888in}{3.459345in}}%
\pgfpathlineto{\pgfqpoint{2.204755in}{3.162245in}}%
\pgfpathlineto{\pgfqpoint{2.207357in}{3.559439in}}%
\pgfpathlineto{\pgfqpoint{2.208224in}{3.381558in}}%
\pgfpathlineto{\pgfqpoint{2.209091in}{3.554931in}}%
\pgfpathlineto{\pgfqpoint{2.209958in}{3.166879in}}%
\pgfpathlineto{\pgfqpoint{2.211692in}{3.762506in}}%
\pgfpathlineto{\pgfqpoint{2.212559in}{3.528651in}}%
\pgfpathlineto{\pgfqpoint{2.213427in}{3.751080in}}%
\pgfpathlineto{\pgfqpoint{2.214294in}{3.354200in}}%
\pgfpathlineto{\pgfqpoint{2.215161in}{3.551212in}}%
\pgfpathlineto{\pgfqpoint{2.216028in}{3.267819in}}%
\pgfpathlineto{\pgfqpoint{2.217762in}{3.654091in}}%
\pgfpathlineto{\pgfqpoint{2.218629in}{3.035956in}}%
\pgfpathlineto{\pgfqpoint{2.219497in}{3.674408in}}%
\pgfpathlineto{\pgfqpoint{2.220364in}{3.013352in}}%
\pgfpathlineto{\pgfqpoint{2.222965in}{3.522512in}}%
\pgfpathlineto{\pgfqpoint{2.223832in}{2.954145in}}%
\pgfpathlineto{\pgfqpoint{2.224699in}{3.036102in}}%
\pgfpathlineto{\pgfqpoint{2.225566in}{3.805049in}}%
\pgfpathlineto{\pgfqpoint{2.227301in}{3.467637in}}%
\pgfpathlineto{\pgfqpoint{2.228168in}{3.484315in}}%
\pgfpathlineto{\pgfqpoint{2.229035in}{3.664484in}}%
\pgfpathlineto{\pgfqpoint{2.230769in}{3.390452in}}%
\pgfpathlineto{\pgfqpoint{2.231636in}{3.455972in}}%
\pgfpathlineto{\pgfqpoint{2.233371in}{3.434810in}}%
\pgfpathlineto{\pgfqpoint{2.234238in}{3.416269in}}%
\pgfpathlineto{\pgfqpoint{2.235105in}{2.916753in}}%
\pgfpathlineto{\pgfqpoint{2.235972in}{3.380372in}}%
\pgfpathlineto{\pgfqpoint{2.236839in}{3.320254in}}%
\pgfpathlineto{\pgfqpoint{2.237706in}{3.304835in}}%
\pgfpathlineto{\pgfqpoint{2.238573in}{3.226053in}}%
\pgfpathlineto{\pgfqpoint{2.239441in}{3.271190in}}%
\pgfpathlineto{\pgfqpoint{2.240308in}{3.902872in}}%
\pgfpathlineto{\pgfqpoint{2.242042in}{3.211025in}}%
\pgfpathlineto{\pgfqpoint{2.242909in}{3.735101in}}%
\pgfpathlineto{\pgfqpoint{2.243776in}{3.465198in}}%
\pgfpathlineto{\pgfqpoint{2.244643in}{3.543332in}}%
\pgfpathlineto{\pgfqpoint{2.245510in}{3.164862in}}%
\pgfpathlineto{\pgfqpoint{2.246378in}{3.315084in}}%
\pgfpathlineto{\pgfqpoint{2.248112in}{4.066515in}}%
\pgfpathlineto{\pgfqpoint{2.248979in}{3.321363in}}%
\pgfpathlineto{\pgfqpoint{2.249846in}{3.545703in}}%
\pgfpathlineto{\pgfqpoint{2.250713in}{3.206161in}}%
\pgfpathlineto{\pgfqpoint{2.251580in}{3.509516in}}%
\pgfpathlineto{\pgfqpoint{2.253315in}{3.162005in}}%
\pgfpathlineto{\pgfqpoint{2.255049in}{3.607848in}}%
\pgfpathlineto{\pgfqpoint{2.257650in}{3.247440in}}%
\pgfpathlineto{\pgfqpoint{2.258517in}{3.611814in}}%
\pgfpathlineto{\pgfqpoint{2.260252in}{3.225500in}}%
\pgfpathlineto{\pgfqpoint{2.261119in}{3.577268in}}%
\pgfpathlineto{\pgfqpoint{2.262853in}{3.099875in}}%
\pgfpathlineto{\pgfqpoint{2.264587in}{3.432903in}}%
\pgfpathlineto{\pgfqpoint{2.265455in}{3.380292in}}%
\pgfpathlineto{\pgfqpoint{2.266322in}{3.408885in}}%
\pgfpathlineto{\pgfqpoint{2.267189in}{3.127353in}}%
\pgfpathlineto{\pgfqpoint{2.268923in}{3.495734in}}%
\pgfpathlineto{\pgfqpoint{2.270657in}{3.168798in}}%
\pgfpathlineto{\pgfqpoint{2.272392in}{3.836974in}}%
\pgfpathlineto{\pgfqpoint{2.273259in}{3.693335in}}%
\pgfpathlineto{\pgfqpoint{2.274993in}{3.115101in}}%
\pgfpathlineto{\pgfqpoint{2.275860in}{3.520680in}}%
\pgfpathlineto{\pgfqpoint{2.276727in}{3.009140in}}%
\pgfpathlineto{\pgfqpoint{2.278462in}{3.464631in}}%
\pgfpathlineto{\pgfqpoint{2.280196in}{3.155971in}}%
\pgfpathlineto{\pgfqpoint{2.281063in}{3.424844in}}%
\pgfpathlineto{\pgfqpoint{2.281930in}{3.151623in}}%
\pgfpathlineto{\pgfqpoint{2.282797in}{3.686185in}}%
\pgfpathlineto{\pgfqpoint{2.283664in}{3.540133in}}%
\pgfpathlineto{\pgfqpoint{2.284531in}{3.175573in}}%
\pgfpathlineto{\pgfqpoint{2.285399in}{3.555995in}}%
\pgfpathlineto{\pgfqpoint{2.286266in}{3.385818in}}%
\pgfpathlineto{\pgfqpoint{2.287133in}{3.496780in}}%
\pgfpathlineto{\pgfqpoint{2.288000in}{3.197256in}}%
\pgfpathlineto{\pgfqpoint{2.288867in}{3.637080in}}%
\pgfpathlineto{\pgfqpoint{2.289734in}{3.517259in}}%
\pgfpathlineto{\pgfqpoint{2.291469in}{3.183313in}}%
\pgfpathlineto{\pgfqpoint{2.293203in}{3.693520in}}%
\pgfpathlineto{\pgfqpoint{2.294070in}{3.131830in}}%
\pgfpathlineto{\pgfqpoint{2.295804in}{3.678099in}}%
\pgfpathlineto{\pgfqpoint{2.296671in}{3.237780in}}%
\pgfpathlineto{\pgfqpoint{2.297538in}{3.615639in}}%
\pgfpathlineto{\pgfqpoint{2.298406in}{3.588361in}}%
\pgfpathlineto{\pgfqpoint{2.299273in}{3.700393in}}%
\pgfpathlineto{\pgfqpoint{2.301007in}{2.912060in}}%
\pgfpathlineto{\pgfqpoint{2.302741in}{3.645228in}}%
\pgfpathlineto{\pgfqpoint{2.303608in}{3.083301in}}%
\pgfpathlineto{\pgfqpoint{2.305343in}{3.579065in}}%
\pgfpathlineto{\pgfqpoint{2.306210in}{3.366993in}}%
\pgfpathlineto{\pgfqpoint{2.307077in}{3.669593in}}%
\pgfpathlineto{\pgfqpoint{2.307944in}{3.107046in}}%
\pgfpathlineto{\pgfqpoint{2.309678in}{3.464433in}}%
\pgfpathlineto{\pgfqpoint{2.310545in}{3.334955in}}%
\pgfpathlineto{\pgfqpoint{2.311413in}{3.435295in}}%
\pgfpathlineto{\pgfqpoint{2.312280in}{3.129018in}}%
\pgfpathlineto{\pgfqpoint{2.313147in}{3.877188in}}%
\pgfpathlineto{\pgfqpoint{2.314014in}{3.433291in}}%
\pgfpathlineto{\pgfqpoint{2.315748in}{3.735143in}}%
\pgfpathlineto{\pgfqpoint{2.317483in}{3.128486in}}%
\pgfpathlineto{\pgfqpoint{2.318350in}{3.333259in}}%
\pgfpathlineto{\pgfqpoint{2.319217in}{3.264271in}}%
\pgfpathlineto{\pgfqpoint{2.320084in}{3.356156in}}%
\pgfpathlineto{\pgfqpoint{2.320951in}{3.210344in}}%
\pgfpathlineto{\pgfqpoint{2.321818in}{3.730111in}}%
\pgfpathlineto{\pgfqpoint{2.324420in}{3.325139in}}%
\pgfpathlineto{\pgfqpoint{2.327021in}{3.153191in}}%
\pgfpathlineto{\pgfqpoint{2.327888in}{3.659550in}}%
\pgfpathlineto{\pgfqpoint{2.328755in}{3.259869in}}%
\pgfpathlineto{\pgfqpoint{2.329622in}{3.478155in}}%
\pgfpathlineto{\pgfqpoint{2.330490in}{3.271849in}}%
\pgfpathlineto{\pgfqpoint{2.331357in}{3.347120in}}%
\pgfpathlineto{\pgfqpoint{2.333091in}{3.030446in}}%
\pgfpathlineto{\pgfqpoint{2.333958in}{3.598456in}}%
\pgfpathlineto{\pgfqpoint{2.336559in}{3.092055in}}%
\pgfpathlineto{\pgfqpoint{2.338294in}{3.550627in}}%
\pgfpathlineto{\pgfqpoint{2.339161in}{3.503875in}}%
\pgfpathlineto{\pgfqpoint{2.340028in}{3.143879in}}%
\pgfpathlineto{\pgfqpoint{2.341762in}{3.976818in}}%
\pgfpathlineto{\pgfqpoint{2.342629in}{3.266612in}}%
\pgfpathlineto{\pgfqpoint{2.343497in}{3.297501in}}%
\pgfpathlineto{\pgfqpoint{2.344364in}{3.515444in}}%
\pgfpathlineto{\pgfqpoint{2.346098in}{3.107410in}}%
\pgfpathlineto{\pgfqpoint{2.347832in}{3.844830in}}%
\pgfpathlineto{\pgfqpoint{2.349566in}{3.258918in}}%
\pgfpathlineto{\pgfqpoint{2.350434in}{3.489800in}}%
\pgfpathlineto{\pgfqpoint{2.351301in}{3.437634in}}%
\pgfpathlineto{\pgfqpoint{2.352168in}{3.252804in}}%
\pgfpathlineto{\pgfqpoint{2.353902in}{3.677156in}}%
\pgfpathlineto{\pgfqpoint{2.355636in}{3.143388in}}%
\pgfpathlineto{\pgfqpoint{2.356503in}{3.725691in}}%
\pgfpathlineto{\pgfqpoint{2.358238in}{2.967148in}}%
\pgfpathlineto{\pgfqpoint{2.359105in}{3.062036in}}%
\pgfpathlineto{\pgfqpoint{2.359972in}{3.446469in}}%
\pgfpathlineto{\pgfqpoint{2.360839in}{3.435758in}}%
\pgfpathlineto{\pgfqpoint{2.361706in}{3.218204in}}%
\pgfpathlineto{\pgfqpoint{2.362573in}{3.734345in}}%
\pgfpathlineto{\pgfqpoint{2.364308in}{3.177546in}}%
\pgfpathlineto{\pgfqpoint{2.365175in}{3.842257in}}%
\pgfpathlineto{\pgfqpoint{2.367776in}{3.109291in}}%
\pgfpathlineto{\pgfqpoint{2.369510in}{3.587154in}}%
\pgfpathlineto{\pgfqpoint{2.370378in}{3.321267in}}%
\pgfpathlineto{\pgfqpoint{2.372112in}{3.778433in}}%
\pgfpathlineto{\pgfqpoint{2.373846in}{3.038745in}}%
\pgfpathlineto{\pgfqpoint{2.374713in}{3.246476in}}%
\pgfpathlineto{\pgfqpoint{2.375580in}{3.642218in}}%
\pgfpathlineto{\pgfqpoint{2.376448in}{3.333136in}}%
\pgfpathlineto{\pgfqpoint{2.377315in}{3.404473in}}%
\pgfpathlineto{\pgfqpoint{2.378182in}{3.880120in}}%
\pgfpathlineto{\pgfqpoint{2.379916in}{3.378614in}}%
\pgfpathlineto{\pgfqpoint{2.380783in}{3.369248in}}%
\pgfpathlineto{\pgfqpoint{2.381650in}{3.598340in}}%
\pgfpathlineto{\pgfqpoint{2.383385in}{3.021793in}}%
\pgfpathlineto{\pgfqpoint{2.384252in}{3.570318in}}%
\pgfpathlineto{\pgfqpoint{2.385119in}{3.253982in}}%
\pgfpathlineto{\pgfqpoint{2.386853in}{3.602692in}}%
\pgfpathlineto{\pgfqpoint{2.388587in}{3.452549in}}%
\pgfpathlineto{\pgfqpoint{2.390322in}{3.643738in}}%
\pgfpathlineto{\pgfqpoint{2.391189in}{3.634843in}}%
\pgfpathlineto{\pgfqpoint{2.392056in}{3.359988in}}%
\pgfpathlineto{\pgfqpoint{2.392923in}{3.591767in}}%
\pgfpathlineto{\pgfqpoint{2.393790in}{3.360495in}}%
\pgfpathlineto{\pgfqpoint{2.396392in}{3.734044in}}%
\pgfpathlineto{\pgfqpoint{2.397259in}{3.134786in}}%
\pgfpathlineto{\pgfqpoint{2.398126in}{3.631935in}}%
\pgfpathlineto{\pgfqpoint{2.398993in}{3.593894in}}%
\pgfpathlineto{\pgfqpoint{2.400727in}{3.138347in}}%
\pgfpathlineto{\pgfqpoint{2.401594in}{3.531531in}}%
\pgfpathlineto{\pgfqpoint{2.403329in}{3.172601in}}%
\pgfpathlineto{\pgfqpoint{2.404196in}{3.604359in}}%
\pgfpathlineto{\pgfqpoint{2.405063in}{3.483582in}}%
\pgfpathlineto{\pgfqpoint{2.405930in}{3.514979in}}%
\pgfpathlineto{\pgfqpoint{2.406797in}{3.588488in}}%
\pgfpathlineto{\pgfqpoint{2.409399in}{3.209603in}}%
\pgfpathlineto{\pgfqpoint{2.410266in}{3.104905in}}%
\pgfpathlineto{\pgfqpoint{2.412000in}{3.290961in}}%
\pgfpathlineto{\pgfqpoint{2.412867in}{3.272366in}}%
\pgfpathlineto{\pgfqpoint{2.413734in}{3.570073in}}%
\pgfpathlineto{\pgfqpoint{2.414601in}{3.374136in}}%
\pgfpathlineto{\pgfqpoint{2.416336in}{3.640965in}}%
\pgfpathlineto{\pgfqpoint{2.418070in}{2.929131in}}%
\pgfpathlineto{\pgfqpoint{2.419804in}{3.751890in}}%
\pgfpathlineto{\pgfqpoint{2.420671in}{3.215246in}}%
\pgfpathlineto{\pgfqpoint{2.422406in}{3.486916in}}%
\pgfpathlineto{\pgfqpoint{2.423273in}{2.988118in}}%
\pgfpathlineto{\pgfqpoint{2.424140in}{3.220393in}}%
\pgfpathlineto{\pgfqpoint{2.425007in}{3.759439in}}%
\pgfpathlineto{\pgfqpoint{2.425874in}{2.991412in}}%
\pgfpathlineto{\pgfqpoint{2.426741in}{3.084462in}}%
\pgfpathlineto{\pgfqpoint{2.427608in}{3.339951in}}%
\pgfpathlineto{\pgfqpoint{2.428476in}{3.142232in}}%
\pgfpathlineto{\pgfqpoint{2.430210in}{3.602018in}}%
\pgfpathlineto{\pgfqpoint{2.431944in}{3.036692in}}%
\pgfpathlineto{\pgfqpoint{2.433678in}{3.672290in}}%
\pgfpathlineto{\pgfqpoint{2.434545in}{3.552960in}}%
\pgfpathlineto{\pgfqpoint{2.435413in}{3.208989in}}%
\pgfpathlineto{\pgfqpoint{2.436280in}{3.575739in}}%
\pgfpathlineto{\pgfqpoint{2.438014in}{3.103634in}}%
\pgfpathlineto{\pgfqpoint{2.439748in}{3.605959in}}%
\pgfpathlineto{\pgfqpoint{2.440615in}{3.097534in}}%
\pgfpathlineto{\pgfqpoint{2.441483in}{3.659150in}}%
\pgfpathlineto{\pgfqpoint{2.442350in}{3.623456in}}%
\pgfpathlineto{\pgfqpoint{2.443217in}{3.209114in}}%
\pgfpathlineto{\pgfqpoint{2.444951in}{3.612063in}}%
\pgfpathlineto{\pgfqpoint{2.448420in}{3.317066in}}%
\pgfpathlineto{\pgfqpoint{2.449287in}{3.775457in}}%
\pgfpathlineto{\pgfqpoint{2.450154in}{3.567130in}}%
\pgfpathlineto{\pgfqpoint{2.451021in}{3.629209in}}%
\pgfpathlineto{\pgfqpoint{2.451888in}{3.463448in}}%
\pgfpathlineto{\pgfqpoint{2.452755in}{3.749147in}}%
\pgfpathlineto{\pgfqpoint{2.453622in}{3.564013in}}%
\pgfpathlineto{\pgfqpoint{2.454490in}{3.049348in}}%
\pgfpathlineto{\pgfqpoint{2.455357in}{3.388582in}}%
\pgfpathlineto{\pgfqpoint{2.456224in}{3.355760in}}%
\pgfpathlineto{\pgfqpoint{2.457091in}{3.445341in}}%
\pgfpathlineto{\pgfqpoint{2.458825in}{3.348918in}}%
\pgfpathlineto{\pgfqpoint{2.459692in}{2.620364in}}%
\pgfpathlineto{\pgfqpoint{2.461427in}{3.487372in}}%
\pgfpathlineto{\pgfqpoint{2.462294in}{3.369308in}}%
\pgfpathlineto{\pgfqpoint{2.463161in}{3.492977in}}%
\pgfpathlineto{\pgfqpoint{2.464028in}{3.477982in}}%
\pgfpathlineto{\pgfqpoint{2.464895in}{3.152318in}}%
\pgfpathlineto{\pgfqpoint{2.465762in}{3.403204in}}%
\pgfpathlineto{\pgfqpoint{2.466629in}{3.985438in}}%
\pgfpathlineto{\pgfqpoint{2.468364in}{3.018977in}}%
\pgfpathlineto{\pgfqpoint{2.469231in}{3.497495in}}%
\pgfpathlineto{\pgfqpoint{2.470098in}{3.441932in}}%
\pgfpathlineto{\pgfqpoint{2.470965in}{3.142996in}}%
\pgfpathlineto{\pgfqpoint{2.471832in}{3.351746in}}%
\pgfpathlineto{\pgfqpoint{2.472699in}{3.916968in}}%
\pgfpathlineto{\pgfqpoint{2.473566in}{2.892415in}}%
\pgfpathlineto{\pgfqpoint{2.474434in}{3.506552in}}%
\pgfpathlineto{\pgfqpoint{2.475301in}{3.082505in}}%
\pgfpathlineto{\pgfqpoint{2.476168in}{3.131468in}}%
\pgfpathlineto{\pgfqpoint{2.477035in}{3.310215in}}%
\pgfpathlineto{\pgfqpoint{2.478769in}{3.139782in}}%
\pgfpathlineto{\pgfqpoint{2.479636in}{3.184722in}}%
\pgfpathlineto{\pgfqpoint{2.480503in}{3.323496in}}%
\pgfpathlineto{\pgfqpoint{2.481371in}{3.261676in}}%
\pgfpathlineto{\pgfqpoint{2.483105in}{3.974042in}}%
\pgfpathlineto{\pgfqpoint{2.483972in}{3.141333in}}%
\pgfpathlineto{\pgfqpoint{2.484839in}{3.563132in}}%
\pgfpathlineto{\pgfqpoint{2.486573in}{3.306821in}}%
\pgfpathlineto{\pgfqpoint{2.487441in}{3.291529in}}%
\pgfpathlineto{\pgfqpoint{2.488308in}{3.215677in}}%
\pgfpathlineto{\pgfqpoint{2.489175in}{3.350221in}}%
\pgfpathlineto{\pgfqpoint{2.490042in}{3.166604in}}%
\pgfpathlineto{\pgfqpoint{2.490909in}{3.462824in}}%
\pgfpathlineto{\pgfqpoint{2.491776in}{3.396989in}}%
\pgfpathlineto{\pgfqpoint{2.492643in}{3.023890in}}%
\pgfpathlineto{\pgfqpoint{2.493510in}{3.673531in}}%
\pgfpathlineto{\pgfqpoint{2.495245in}{3.332720in}}%
\pgfpathlineto{\pgfqpoint{2.496112in}{3.495962in}}%
\pgfpathlineto{\pgfqpoint{2.496979in}{3.259236in}}%
\pgfpathlineto{\pgfqpoint{2.497846in}{3.302777in}}%
\pgfpathlineto{\pgfqpoint{2.498713in}{3.871745in}}%
\pgfpathlineto{\pgfqpoint{2.500448in}{3.189229in}}%
\pgfpathlineto{\pgfqpoint{2.501315in}{3.912874in}}%
\pgfpathlineto{\pgfqpoint{2.502182in}{3.192367in}}%
\pgfpathlineto{\pgfqpoint{2.503049in}{3.300027in}}%
\pgfpathlineto{\pgfqpoint{2.503916in}{3.361770in}}%
\pgfpathlineto{\pgfqpoint{2.505650in}{3.092448in}}%
\pgfpathlineto{\pgfqpoint{2.507385in}{3.612800in}}%
\pgfpathlineto{\pgfqpoint{2.508252in}{3.429992in}}%
\pgfpathlineto{\pgfqpoint{2.509119in}{3.343086in}}%
\pgfpathlineto{\pgfqpoint{2.509986in}{3.405478in}}%
\pgfpathlineto{\pgfqpoint{2.510853in}{3.001299in}}%
\pgfpathlineto{\pgfqpoint{2.511720in}{3.499821in}}%
\pgfpathlineto{\pgfqpoint{2.513455in}{3.153038in}}%
\pgfpathlineto{\pgfqpoint{2.514322in}{3.812050in}}%
\pgfpathlineto{\pgfqpoint{2.515189in}{3.725516in}}%
\pgfpathlineto{\pgfqpoint{2.516056in}{3.542430in}}%
\pgfpathlineto{\pgfqpoint{2.516923in}{3.599320in}}%
\pgfpathlineto{\pgfqpoint{2.517790in}{3.127258in}}%
\pgfpathlineto{\pgfqpoint{2.518657in}{3.535877in}}%
\pgfpathlineto{\pgfqpoint{2.519524in}{3.381527in}}%
\pgfpathlineto{\pgfqpoint{2.520392in}{3.417351in}}%
\pgfpathlineto{\pgfqpoint{2.522126in}{3.797011in}}%
\pgfpathlineto{\pgfqpoint{2.522993in}{3.084449in}}%
\pgfpathlineto{\pgfqpoint{2.523860in}{3.158728in}}%
\pgfpathlineto{\pgfqpoint{2.524727in}{3.513796in}}%
\pgfpathlineto{\pgfqpoint{2.526462in}{3.033536in}}%
\pgfpathlineto{\pgfqpoint{2.527329in}{3.576363in}}%
\pgfpathlineto{\pgfqpoint{2.528196in}{3.516094in}}%
\pgfpathlineto{\pgfqpoint{2.529063in}{3.332120in}}%
\pgfpathlineto{\pgfqpoint{2.529930in}{3.469771in}}%
\pgfpathlineto{\pgfqpoint{2.530797in}{3.159051in}}%
\pgfpathlineto{\pgfqpoint{2.531664in}{3.262203in}}%
\pgfpathlineto{\pgfqpoint{2.532531in}{3.520983in}}%
\pgfpathlineto{\pgfqpoint{2.534266in}{3.143048in}}%
\pgfpathlineto{\pgfqpoint{2.535133in}{3.381668in}}%
\pgfpathlineto{\pgfqpoint{2.536000in}{3.213653in}}%
\pgfpathlineto{\pgfqpoint{2.536867in}{3.364823in}}%
\pgfpathlineto{\pgfqpoint{2.537734in}{3.755705in}}%
\pgfpathlineto{\pgfqpoint{2.538601in}{3.147083in}}%
\pgfpathlineto{\pgfqpoint{2.540336in}{3.669095in}}%
\pgfpathlineto{\pgfqpoint{2.542070in}{3.114529in}}%
\pgfpathlineto{\pgfqpoint{2.542937in}{3.153987in}}%
\pgfpathlineto{\pgfqpoint{2.543804in}{3.543505in}}%
\pgfpathlineto{\pgfqpoint{2.544671in}{3.258072in}}%
\pgfpathlineto{\pgfqpoint{2.546406in}{3.426259in}}%
\pgfpathlineto{\pgfqpoint{2.547273in}{3.122110in}}%
\pgfpathlineto{\pgfqpoint{2.549874in}{3.571505in}}%
\pgfpathlineto{\pgfqpoint{2.551608in}{2.999698in}}%
\pgfpathlineto{\pgfqpoint{2.552476in}{3.723262in}}%
\pgfpathlineto{\pgfqpoint{2.554210in}{3.346463in}}%
\pgfpathlineto{\pgfqpoint{2.555077in}{3.424836in}}%
\pgfpathlineto{\pgfqpoint{2.556811in}{3.262299in}}%
\pgfpathlineto{\pgfqpoint{2.557678in}{3.780354in}}%
\pgfpathlineto{\pgfqpoint{2.558545in}{3.255492in}}%
\pgfpathlineto{\pgfqpoint{2.559413in}{3.423708in}}%
\pgfpathlineto{\pgfqpoint{2.560280in}{4.035380in}}%
\pgfpathlineto{\pgfqpoint{2.561147in}{2.906069in}}%
\pgfpathlineto{\pgfqpoint{2.562014in}{3.372307in}}%
\pgfpathlineto{\pgfqpoint{2.562881in}{3.205590in}}%
\pgfpathlineto{\pgfqpoint{2.563748in}{3.262674in}}%
\pgfpathlineto{\pgfqpoint{2.564615in}{3.577414in}}%
\pgfpathlineto{\pgfqpoint{2.565483in}{3.549388in}}%
\pgfpathlineto{\pgfqpoint{2.566350in}{3.084482in}}%
\pgfpathlineto{\pgfqpoint{2.568084in}{3.621758in}}%
\pgfpathlineto{\pgfqpoint{2.568951in}{3.667195in}}%
\pgfpathlineto{\pgfqpoint{2.569818in}{3.584731in}}%
\pgfpathlineto{\pgfqpoint{2.571552in}{3.209389in}}%
\pgfpathlineto{\pgfqpoint{2.572420in}{3.139289in}}%
\pgfpathlineto{\pgfqpoint{2.574154in}{3.716695in}}%
\pgfpathlineto{\pgfqpoint{2.575888in}{3.041748in}}%
\pgfpathlineto{\pgfqpoint{2.576755in}{3.645237in}}%
\pgfpathlineto{\pgfqpoint{2.577622in}{3.375203in}}%
\pgfpathlineto{\pgfqpoint{2.578490in}{3.417895in}}%
\pgfpathlineto{\pgfqpoint{2.579357in}{3.692491in}}%
\pgfpathlineto{\pgfqpoint{2.581091in}{3.213137in}}%
\pgfpathlineto{\pgfqpoint{2.581958in}{3.233288in}}%
\pgfpathlineto{\pgfqpoint{2.582825in}{3.335432in}}%
\pgfpathlineto{\pgfqpoint{2.583692in}{3.277003in}}%
\pgfpathlineto{\pgfqpoint{2.585427in}{3.368710in}}%
\pgfpathlineto{\pgfqpoint{2.586294in}{3.687019in}}%
\pgfpathlineto{\pgfqpoint{2.588028in}{3.195514in}}%
\pgfpathlineto{\pgfqpoint{2.588895in}{3.605639in}}%
\pgfpathlineto{\pgfqpoint{2.589762in}{3.580433in}}%
\pgfpathlineto{\pgfqpoint{2.590629in}{3.187956in}}%
\pgfpathlineto{\pgfqpoint{2.591497in}{3.470035in}}%
\pgfpathlineto{\pgfqpoint{2.592364in}{3.261345in}}%
\pgfpathlineto{\pgfqpoint{2.594098in}{3.574147in}}%
\pgfpathlineto{\pgfqpoint{2.594965in}{3.498176in}}%
\pgfpathlineto{\pgfqpoint{2.595832in}{3.517264in}}%
\pgfpathlineto{\pgfqpoint{2.596699in}{3.499050in}}%
\pgfpathlineto{\pgfqpoint{2.597566in}{3.308191in}}%
\pgfpathlineto{\pgfqpoint{2.598434in}{3.573662in}}%
\pgfpathlineto{\pgfqpoint{2.599301in}{3.552981in}}%
\pgfpathlineto{\pgfqpoint{2.601035in}{3.379511in}}%
\pgfpathlineto{\pgfqpoint{2.601902in}{3.547822in}}%
\pgfpathlineto{\pgfqpoint{2.602769in}{3.377160in}}%
\pgfpathlineto{\pgfqpoint{2.603636in}{3.554545in}}%
\pgfpathlineto{\pgfqpoint{2.604503in}{3.155376in}}%
\pgfpathlineto{\pgfqpoint{2.606238in}{3.671364in}}%
\pgfpathlineto{\pgfqpoint{2.607105in}{3.072195in}}%
\pgfpathlineto{\pgfqpoint{2.608839in}{3.396053in}}%
\pgfpathlineto{\pgfqpoint{2.609706in}{3.256538in}}%
\pgfpathlineto{\pgfqpoint{2.611441in}{3.665267in}}%
\pgfpathlineto{\pgfqpoint{2.612308in}{3.303751in}}%
\pgfpathlineto{\pgfqpoint{2.613175in}{3.500922in}}%
\pgfpathlineto{\pgfqpoint{2.614042in}{3.308559in}}%
\pgfpathlineto{\pgfqpoint{2.614909in}{3.320998in}}%
\pgfpathlineto{\pgfqpoint{2.615776in}{3.273318in}}%
\pgfpathlineto{\pgfqpoint{2.616643in}{2.883592in}}%
\pgfpathlineto{\pgfqpoint{2.618378in}{3.831271in}}%
\pgfpathlineto{\pgfqpoint{2.619245in}{3.071867in}}%
\pgfpathlineto{\pgfqpoint{2.620112in}{3.154066in}}%
\pgfpathlineto{\pgfqpoint{2.620979in}{3.592259in}}%
\pgfpathlineto{\pgfqpoint{2.621846in}{3.054226in}}%
\pgfpathlineto{\pgfqpoint{2.622713in}{3.648393in}}%
\pgfpathlineto{\pgfqpoint{2.623580in}{3.288838in}}%
\pgfpathlineto{\pgfqpoint{2.625315in}{3.563276in}}%
\pgfpathlineto{\pgfqpoint{2.627916in}{3.020274in}}%
\pgfpathlineto{\pgfqpoint{2.629650in}{3.632132in}}%
\pgfpathlineto{\pgfqpoint{2.630517in}{3.142866in}}%
\pgfpathlineto{\pgfqpoint{2.632252in}{3.470058in}}%
\pgfpathlineto{\pgfqpoint{2.633119in}{3.068429in}}%
\pgfpathlineto{\pgfqpoint{2.633986in}{3.545125in}}%
\pgfpathlineto{\pgfqpoint{2.635720in}{3.187705in}}%
\pgfpathlineto{\pgfqpoint{2.636587in}{3.710038in}}%
\pgfpathlineto{\pgfqpoint{2.637455in}{2.974501in}}%
\pgfpathlineto{\pgfqpoint{2.639189in}{3.395981in}}%
\pgfpathlineto{\pgfqpoint{2.640056in}{3.283894in}}%
\pgfpathlineto{\pgfqpoint{2.641790in}{3.721781in}}%
\pgfpathlineto{\pgfqpoint{2.643524in}{3.144473in}}%
\pgfpathlineto{\pgfqpoint{2.645259in}{3.552730in}}%
\pgfpathlineto{\pgfqpoint{2.646126in}{3.363268in}}%
\pgfpathlineto{\pgfqpoint{2.646993in}{3.396674in}}%
\pgfpathlineto{\pgfqpoint{2.648727in}{2.975598in}}%
\pgfpathlineto{\pgfqpoint{2.650462in}{3.550280in}}%
\pgfpathlineto{\pgfqpoint{2.651329in}{3.437287in}}%
\pgfpathlineto{\pgfqpoint{2.652196in}{3.701110in}}%
\pgfpathlineto{\pgfqpoint{2.653930in}{3.148186in}}%
\pgfpathlineto{\pgfqpoint{2.654797in}{3.703275in}}%
\pgfpathlineto{\pgfqpoint{2.655664in}{3.226609in}}%
\pgfpathlineto{\pgfqpoint{2.656531in}{3.296189in}}%
\pgfpathlineto{\pgfqpoint{2.657399in}{3.555724in}}%
\pgfpathlineto{\pgfqpoint{2.658266in}{3.144710in}}%
\pgfpathlineto{\pgfqpoint{2.659133in}{3.675360in}}%
\pgfpathlineto{\pgfqpoint{2.660867in}{3.033998in}}%
\pgfpathlineto{\pgfqpoint{2.661734in}{3.203235in}}%
\pgfpathlineto{\pgfqpoint{2.662601in}{3.208971in}}%
\pgfpathlineto{\pgfqpoint{2.663469in}{3.667873in}}%
\pgfpathlineto{\pgfqpoint{2.665203in}{3.117768in}}%
\pgfpathlineto{\pgfqpoint{2.666070in}{3.408600in}}%
\pgfpathlineto{\pgfqpoint{2.666937in}{3.103313in}}%
\pgfpathlineto{\pgfqpoint{2.667804in}{3.511683in}}%
\pgfpathlineto{\pgfqpoint{2.668671in}{3.283076in}}%
\pgfpathlineto{\pgfqpoint{2.670406in}{3.610425in}}%
\pgfpathlineto{\pgfqpoint{2.671273in}{3.206339in}}%
\pgfpathlineto{\pgfqpoint{2.672140in}{3.210367in}}%
\pgfpathlineto{\pgfqpoint{2.673007in}{2.994078in}}%
\pgfpathlineto{\pgfqpoint{2.674741in}{4.048632in}}%
\pgfpathlineto{\pgfqpoint{2.675608in}{3.192021in}}%
\pgfpathlineto{\pgfqpoint{2.676476in}{3.544977in}}%
\pgfpathlineto{\pgfqpoint{2.677343in}{3.490492in}}%
\pgfpathlineto{\pgfqpoint{2.678210in}{3.285180in}}%
\pgfpathlineto{\pgfqpoint{2.679077in}{3.465033in}}%
\pgfpathlineto{\pgfqpoint{2.679944in}{3.413848in}}%
\pgfpathlineto{\pgfqpoint{2.680811in}{3.690964in}}%
\pgfpathlineto{\pgfqpoint{2.681678in}{3.239060in}}%
\pgfpathlineto{\pgfqpoint{2.682545in}{3.469430in}}%
\pgfpathlineto{\pgfqpoint{2.683413in}{3.368217in}}%
\pgfpathlineto{\pgfqpoint{2.684280in}{3.870186in}}%
\pgfpathlineto{\pgfqpoint{2.686014in}{3.266076in}}%
\pgfpathlineto{\pgfqpoint{2.687748in}{3.492800in}}%
\pgfpathlineto{\pgfqpoint{2.688615in}{3.401416in}}%
\pgfpathlineto{\pgfqpoint{2.689483in}{3.110446in}}%
\pgfpathlineto{\pgfqpoint{2.690350in}{3.516085in}}%
\pgfpathlineto{\pgfqpoint{2.691217in}{3.171435in}}%
\pgfpathlineto{\pgfqpoint{2.692084in}{3.670021in}}%
\pgfpathlineto{\pgfqpoint{2.692951in}{3.037888in}}%
\pgfpathlineto{\pgfqpoint{2.694685in}{3.529721in}}%
\pgfpathlineto{\pgfqpoint{2.695552in}{3.061102in}}%
\pgfpathlineto{\pgfqpoint{2.696420in}{3.324650in}}%
\pgfpathlineto{\pgfqpoint{2.697287in}{3.296664in}}%
\pgfpathlineto{\pgfqpoint{2.698154in}{3.219649in}}%
\pgfpathlineto{\pgfqpoint{2.699021in}{3.602204in}}%
\pgfpathlineto{\pgfqpoint{2.699888in}{3.217628in}}%
\pgfpathlineto{\pgfqpoint{2.700755in}{3.300024in}}%
\pgfpathlineto{\pgfqpoint{2.701622in}{3.088530in}}%
\pgfpathlineto{\pgfqpoint{2.703357in}{3.685365in}}%
\pgfpathlineto{\pgfqpoint{2.704224in}{3.370784in}}%
\pgfpathlineto{\pgfqpoint{2.705958in}{3.548180in}}%
\pgfpathlineto{\pgfqpoint{2.706825in}{3.342073in}}%
\pgfpathlineto{\pgfqpoint{2.707692in}{3.443721in}}%
\pgfpathlineto{\pgfqpoint{2.708559in}{3.138128in}}%
\pgfpathlineto{\pgfqpoint{2.709427in}{3.656020in}}%
\pgfpathlineto{\pgfqpoint{2.711161in}{3.242750in}}%
\pgfpathlineto{\pgfqpoint{2.712028in}{3.638904in}}%
\pgfpathlineto{\pgfqpoint{2.712895in}{3.320806in}}%
\pgfpathlineto{\pgfqpoint{2.713762in}{3.559992in}}%
\pgfpathlineto{\pgfqpoint{2.714629in}{3.397568in}}%
\pgfpathlineto{\pgfqpoint{2.715497in}{3.424378in}}%
\pgfpathlineto{\pgfqpoint{2.716364in}{3.558550in}}%
\pgfpathlineto{\pgfqpoint{2.717231in}{3.142936in}}%
\pgfpathlineto{\pgfqpoint{2.718098in}{3.457440in}}%
\pgfpathlineto{\pgfqpoint{2.718965in}{3.423123in}}%
\pgfpathlineto{\pgfqpoint{2.719832in}{3.017862in}}%
\pgfpathlineto{\pgfqpoint{2.720699in}{3.465334in}}%
\pgfpathlineto{\pgfqpoint{2.721566in}{3.090765in}}%
\pgfpathlineto{\pgfqpoint{2.722434in}{3.494202in}}%
\pgfpathlineto{\pgfqpoint{2.723301in}{3.389400in}}%
\pgfpathlineto{\pgfqpoint{2.724168in}{2.905505in}}%
\pgfpathlineto{\pgfqpoint{2.725035in}{3.599998in}}%
\pgfpathlineto{\pgfqpoint{2.725902in}{3.093382in}}%
\pgfpathlineto{\pgfqpoint{2.727636in}{3.517844in}}%
\pgfpathlineto{\pgfqpoint{2.728503in}{3.355078in}}%
\pgfpathlineto{\pgfqpoint{2.729371in}{3.633200in}}%
\pgfpathlineto{\pgfqpoint{2.730238in}{3.563380in}}%
\pgfpathlineto{\pgfqpoint{2.731105in}{3.718441in}}%
\pgfpathlineto{\pgfqpoint{2.731972in}{3.154406in}}%
\pgfpathlineto{\pgfqpoint{2.732839in}{3.595204in}}%
\pgfpathlineto{\pgfqpoint{2.733706in}{3.472038in}}%
\pgfpathlineto{\pgfqpoint{2.734573in}{3.291178in}}%
\pgfpathlineto{\pgfqpoint{2.735441in}{3.727907in}}%
\pgfpathlineto{\pgfqpoint{2.736308in}{3.222239in}}%
\pgfpathlineto{\pgfqpoint{2.737175in}{3.312620in}}%
\pgfpathlineto{\pgfqpoint{2.738909in}{2.974898in}}%
\pgfpathlineto{\pgfqpoint{2.741510in}{3.489250in}}%
\pgfpathlineto{\pgfqpoint{2.742378in}{3.316768in}}%
\pgfpathlineto{\pgfqpoint{2.743245in}{3.662837in}}%
\pgfpathlineto{\pgfqpoint{2.744112in}{3.607131in}}%
\pgfpathlineto{\pgfqpoint{2.744979in}{3.278413in}}%
\pgfpathlineto{\pgfqpoint{2.745846in}{4.096964in}}%
\pgfpathlineto{\pgfqpoint{2.748448in}{3.285001in}}%
\pgfpathlineto{\pgfqpoint{2.749315in}{3.339810in}}%
\pgfpathlineto{\pgfqpoint{2.750182in}{3.389223in}}%
\pgfpathlineto{\pgfqpoint{2.751916in}{2.643712in}}%
\pgfpathlineto{\pgfqpoint{2.752783in}{3.037375in}}%
\pgfpathlineto{\pgfqpoint{2.753650in}{2.838787in}}%
\pgfpathlineto{\pgfqpoint{2.754517in}{3.602164in}}%
\pgfpathlineto{\pgfqpoint{2.755385in}{3.356810in}}%
\pgfpathlineto{\pgfqpoint{2.757119in}{4.147636in}}%
\pgfpathlineto{\pgfqpoint{2.757986in}{3.831631in}}%
\pgfpathlineto{\pgfqpoint{2.759720in}{3.284233in}}%
\pgfpathlineto{\pgfqpoint{2.760587in}{3.555141in}}%
\pgfpathlineto{\pgfqpoint{2.761455in}{3.181744in}}%
\pgfpathlineto{\pgfqpoint{2.762322in}{3.384406in}}%
\pgfpathlineto{\pgfqpoint{2.764056in}{3.001463in}}%
\pgfpathlineto{\pgfqpoint{2.764923in}{3.782787in}}%
\pgfpathlineto{\pgfqpoint{2.765790in}{3.224584in}}%
\pgfpathlineto{\pgfqpoint{2.766657in}{3.734794in}}%
\pgfpathlineto{\pgfqpoint{2.767524in}{3.259722in}}%
\pgfpathlineto{\pgfqpoint{2.769259in}{3.889056in}}%
\pgfpathlineto{\pgfqpoint{2.771860in}{3.147621in}}%
\pgfpathlineto{\pgfqpoint{2.772727in}{3.412311in}}%
\pgfpathlineto{\pgfqpoint{2.773594in}{2.990923in}}%
\pgfpathlineto{\pgfqpoint{2.775329in}{3.456297in}}%
\pgfpathlineto{\pgfqpoint{2.777063in}{2.981095in}}%
\pgfpathlineto{\pgfqpoint{2.778797in}{3.754958in}}%
\pgfpathlineto{\pgfqpoint{2.780531in}{3.150304in}}%
\pgfpathlineto{\pgfqpoint{2.781399in}{3.906274in}}%
\pgfpathlineto{\pgfqpoint{2.782266in}{3.135138in}}%
\pgfpathlineto{\pgfqpoint{2.783133in}{3.593734in}}%
\pgfpathlineto{\pgfqpoint{2.784867in}{3.112249in}}%
\pgfpathlineto{\pgfqpoint{2.786601in}{3.394185in}}%
\pgfpathlineto{\pgfqpoint{2.787469in}{2.950105in}}%
\pgfpathlineto{\pgfqpoint{2.788336in}{3.312220in}}%
\pgfpathlineto{\pgfqpoint{2.789203in}{3.218661in}}%
\pgfpathlineto{\pgfqpoint{2.790070in}{3.285619in}}%
\pgfpathlineto{\pgfqpoint{2.790937in}{3.774464in}}%
\pgfpathlineto{\pgfqpoint{2.791804in}{3.753963in}}%
\pgfpathlineto{\pgfqpoint{2.793538in}{3.677772in}}%
\pgfpathlineto{\pgfqpoint{2.794406in}{3.210722in}}%
\pgfpathlineto{\pgfqpoint{2.795273in}{3.294649in}}%
\pgfpathlineto{\pgfqpoint{2.796140in}{3.191192in}}%
\pgfpathlineto{\pgfqpoint{2.797007in}{3.281215in}}%
\pgfpathlineto{\pgfqpoint{2.797874in}{2.873879in}}%
\pgfpathlineto{\pgfqpoint{2.799608in}{3.667390in}}%
\pgfpathlineto{\pgfqpoint{2.800476in}{3.266923in}}%
\pgfpathlineto{\pgfqpoint{2.801343in}{3.711243in}}%
\pgfpathlineto{\pgfqpoint{2.802210in}{3.311554in}}%
\pgfpathlineto{\pgfqpoint{2.803944in}{3.671276in}}%
\pgfpathlineto{\pgfqpoint{2.805678in}{3.122699in}}%
\pgfpathlineto{\pgfqpoint{2.806545in}{3.116178in}}%
\pgfpathlineto{\pgfqpoint{2.807413in}{3.215226in}}%
\pgfpathlineto{\pgfqpoint{2.808280in}{2.999599in}}%
\pgfpathlineto{\pgfqpoint{2.810881in}{3.757086in}}%
\pgfpathlineto{\pgfqpoint{2.811748in}{3.186142in}}%
\pgfpathlineto{\pgfqpoint{2.813483in}{3.609614in}}%
\pgfpathlineto{\pgfqpoint{2.816084in}{3.092775in}}%
\pgfpathlineto{\pgfqpoint{2.816951in}{3.667360in}}%
\pgfpathlineto{\pgfqpoint{2.817818in}{3.304038in}}%
\pgfpathlineto{\pgfqpoint{2.818685in}{3.325868in}}%
\pgfpathlineto{\pgfqpoint{2.819552in}{3.132981in}}%
\pgfpathlineto{\pgfqpoint{2.820420in}{3.165315in}}%
\pgfpathlineto{\pgfqpoint{2.821287in}{3.744311in}}%
\pgfpathlineto{\pgfqpoint{2.822154in}{3.341073in}}%
\pgfpathlineto{\pgfqpoint{2.823021in}{3.503681in}}%
\pgfpathlineto{\pgfqpoint{2.824755in}{2.988001in}}%
\pgfpathlineto{\pgfqpoint{2.825622in}{3.322845in}}%
\pgfpathlineto{\pgfqpoint{2.826490in}{3.057211in}}%
\pgfpathlineto{\pgfqpoint{2.827357in}{3.534412in}}%
\pgfpathlineto{\pgfqpoint{2.828224in}{3.245838in}}%
\pgfpathlineto{\pgfqpoint{2.829091in}{3.247295in}}%
\pgfpathlineto{\pgfqpoint{2.831692in}{3.578519in}}%
\pgfpathlineto{\pgfqpoint{2.832559in}{3.156452in}}%
\pgfpathlineto{\pgfqpoint{2.833427in}{3.211349in}}%
\pgfpathlineto{\pgfqpoint{2.834294in}{3.219105in}}%
\pgfpathlineto{\pgfqpoint{2.836028in}{3.430837in}}%
\pgfpathlineto{\pgfqpoint{2.837762in}{3.253181in}}%
\pgfpathlineto{\pgfqpoint{2.838629in}{3.715040in}}%
\pgfpathlineto{\pgfqpoint{2.840364in}{3.420267in}}%
\pgfpathlineto{\pgfqpoint{2.841231in}{3.624940in}}%
\pgfpathlineto{\pgfqpoint{2.842098in}{3.597765in}}%
\pgfpathlineto{\pgfqpoint{2.842965in}{3.212918in}}%
\pgfpathlineto{\pgfqpoint{2.843832in}{3.227248in}}%
\pgfpathlineto{\pgfqpoint{2.844699in}{3.284396in}}%
\pgfpathlineto{\pgfqpoint{2.845566in}{3.109405in}}%
\pgfpathlineto{\pgfqpoint{2.847301in}{3.640345in}}%
\pgfpathlineto{\pgfqpoint{2.849035in}{2.994167in}}%
\pgfpathlineto{\pgfqpoint{2.850769in}{3.568485in}}%
\pgfpathlineto{\pgfqpoint{2.851636in}{3.170054in}}%
\pgfpathlineto{\pgfqpoint{2.852503in}{3.179935in}}%
\pgfpathlineto{\pgfqpoint{2.853371in}{3.589051in}}%
\pgfpathlineto{\pgfqpoint{2.855105in}{3.180192in}}%
\pgfpathlineto{\pgfqpoint{2.856839in}{3.774515in}}%
\pgfpathlineto{\pgfqpoint{2.857706in}{3.361800in}}%
\pgfpathlineto{\pgfqpoint{2.858573in}{3.792955in}}%
\pgfpathlineto{\pgfqpoint{2.859441in}{3.437114in}}%
\pgfpathlineto{\pgfqpoint{2.860308in}{3.463195in}}%
\pgfpathlineto{\pgfqpoint{2.861175in}{3.577892in}}%
\pgfpathlineto{\pgfqpoint{2.862042in}{3.105032in}}%
\pgfpathlineto{\pgfqpoint{2.862909in}{3.527917in}}%
\pgfpathlineto{\pgfqpoint{2.864643in}{3.145921in}}%
\pgfpathlineto{\pgfqpoint{2.865510in}{3.507505in}}%
\pgfpathlineto{\pgfqpoint{2.868112in}{3.186549in}}%
\pgfpathlineto{\pgfqpoint{2.868979in}{3.869425in}}%
\pgfpathlineto{\pgfqpoint{2.869846in}{3.197797in}}%
\pgfpathlineto{\pgfqpoint{2.870713in}{3.499724in}}%
\pgfpathlineto{\pgfqpoint{2.872448in}{3.275890in}}%
\pgfpathlineto{\pgfqpoint{2.873315in}{3.225618in}}%
\pgfpathlineto{\pgfqpoint{2.874182in}{3.709437in}}%
\pgfpathlineto{\pgfqpoint{2.875049in}{3.581271in}}%
\pgfpathlineto{\pgfqpoint{2.878517in}{2.955591in}}%
\pgfpathlineto{\pgfqpoint{2.879385in}{3.010596in}}%
\pgfpathlineto{\pgfqpoint{2.881119in}{3.535611in}}%
\pgfpathlineto{\pgfqpoint{2.881986in}{3.371222in}}%
\pgfpathlineto{\pgfqpoint{2.882853in}{3.431751in}}%
\pgfpathlineto{\pgfqpoint{2.883720in}{3.402153in}}%
\pgfpathlineto{\pgfqpoint{2.884587in}{3.640815in}}%
\pgfpathlineto{\pgfqpoint{2.886322in}{3.296189in}}%
\pgfpathlineto{\pgfqpoint{2.888056in}{3.532490in}}%
\pgfpathlineto{\pgfqpoint{2.890657in}{3.291809in}}%
\pgfpathlineto{\pgfqpoint{2.892392in}{3.571772in}}%
\pgfpathlineto{\pgfqpoint{2.893259in}{3.497452in}}%
\pgfpathlineto{\pgfqpoint{2.894126in}{2.915229in}}%
\pgfpathlineto{\pgfqpoint{2.894993in}{3.478405in}}%
\pgfpathlineto{\pgfqpoint{2.895860in}{3.373714in}}%
\pgfpathlineto{\pgfqpoint{2.896727in}{3.239773in}}%
\pgfpathlineto{\pgfqpoint{2.897594in}{3.335795in}}%
\pgfpathlineto{\pgfqpoint{2.898462in}{3.264876in}}%
\pgfpathlineto{\pgfqpoint{2.900196in}{3.589698in}}%
\pgfpathlineto{\pgfqpoint{2.901063in}{3.354685in}}%
\pgfpathlineto{\pgfqpoint{2.901930in}{3.523874in}}%
\pgfpathlineto{\pgfqpoint{2.903664in}{3.271366in}}%
\pgfpathlineto{\pgfqpoint{2.904531in}{3.507336in}}%
\pgfpathlineto{\pgfqpoint{2.905399in}{3.409419in}}%
\pgfpathlineto{\pgfqpoint{2.906266in}{3.707151in}}%
\pgfpathlineto{\pgfqpoint{2.907133in}{3.659712in}}%
\pgfpathlineto{\pgfqpoint{2.908000in}{3.298480in}}%
\pgfpathlineto{\pgfqpoint{2.908867in}{3.429147in}}%
\pgfpathlineto{\pgfqpoint{2.909734in}{3.241428in}}%
\pgfpathlineto{\pgfqpoint{2.910601in}{3.261106in}}%
\pgfpathlineto{\pgfqpoint{2.911469in}{3.111376in}}%
\pgfpathlineto{\pgfqpoint{2.912336in}{3.196443in}}%
\pgfpathlineto{\pgfqpoint{2.913203in}{3.526334in}}%
\pgfpathlineto{\pgfqpoint{2.914070in}{3.443577in}}%
\pgfpathlineto{\pgfqpoint{2.914937in}{3.586514in}}%
\pgfpathlineto{\pgfqpoint{2.916671in}{3.206069in}}%
\pgfpathlineto{\pgfqpoint{2.917538in}{3.073254in}}%
\pgfpathlineto{\pgfqpoint{2.918406in}{3.318259in}}%
\pgfpathlineto{\pgfqpoint{2.919273in}{3.312492in}}%
\pgfpathlineto{\pgfqpoint{2.920140in}{3.238758in}}%
\pgfpathlineto{\pgfqpoint{2.921874in}{3.729252in}}%
\pgfpathlineto{\pgfqpoint{2.924476in}{3.114755in}}%
\pgfpathlineto{\pgfqpoint{2.925343in}{2.866634in}}%
\pgfpathlineto{\pgfqpoint{2.927077in}{3.256905in}}%
\pgfpathlineto{\pgfqpoint{2.930545in}{3.537082in}}%
\pgfpathlineto{\pgfqpoint{2.931413in}{3.289111in}}%
\pgfpathlineto{\pgfqpoint{2.934014in}{3.774425in}}%
\pgfpathlineto{\pgfqpoint{2.935748in}{3.276160in}}%
\pgfpathlineto{\pgfqpoint{2.936615in}{3.429244in}}%
\pgfpathlineto{\pgfqpoint{2.939217in}{2.946551in}}%
\pgfpathlineto{\pgfqpoint{2.940951in}{3.881882in}}%
\pgfpathlineto{\pgfqpoint{2.942685in}{3.420701in}}%
\pgfpathlineto{\pgfqpoint{2.943552in}{3.634174in}}%
\pgfpathlineto{\pgfqpoint{2.944420in}{3.006696in}}%
\pgfpathlineto{\pgfqpoint{2.946154in}{3.659056in}}%
\pgfpathlineto{\pgfqpoint{2.948755in}{3.106913in}}%
\pgfpathlineto{\pgfqpoint{2.949622in}{3.446788in}}%
\pgfpathlineto{\pgfqpoint{2.952224in}{3.177355in}}%
\pgfpathlineto{\pgfqpoint{2.953091in}{3.797195in}}%
\pgfpathlineto{\pgfqpoint{2.953958in}{3.790389in}}%
\pgfpathlineto{\pgfqpoint{2.954825in}{3.956068in}}%
\pgfpathlineto{\pgfqpoint{2.957427in}{3.147266in}}%
\pgfpathlineto{\pgfqpoint{2.958294in}{3.145455in}}%
\pgfpathlineto{\pgfqpoint{2.959161in}{3.663037in}}%
\pgfpathlineto{\pgfqpoint{2.960028in}{3.516105in}}%
\pgfpathlineto{\pgfqpoint{2.960895in}{3.704714in}}%
\pgfpathlineto{\pgfqpoint{2.961762in}{3.633267in}}%
\pgfpathlineto{\pgfqpoint{2.964364in}{3.032915in}}%
\pgfpathlineto{\pgfqpoint{2.965231in}{3.604481in}}%
\pgfpathlineto{\pgfqpoint{2.966098in}{3.565351in}}%
\pgfpathlineto{\pgfqpoint{2.966965in}{3.095837in}}%
\pgfpathlineto{\pgfqpoint{2.968699in}{3.554033in}}%
\pgfpathlineto{\pgfqpoint{2.969566in}{2.898349in}}%
\pgfpathlineto{\pgfqpoint{2.971301in}{3.630184in}}%
\pgfpathlineto{\pgfqpoint{2.975636in}{3.167285in}}%
\pgfpathlineto{\pgfqpoint{2.976503in}{3.640551in}}%
\pgfpathlineto{\pgfqpoint{2.977371in}{3.104422in}}%
\pgfpathlineto{\pgfqpoint{2.979105in}{3.857138in}}%
\pgfpathlineto{\pgfqpoint{2.980839in}{3.358770in}}%
\pgfpathlineto{\pgfqpoint{2.981706in}{3.393041in}}%
\pgfpathlineto{\pgfqpoint{2.983441in}{3.707147in}}%
\pgfpathlineto{\pgfqpoint{2.984308in}{3.390465in}}%
\pgfpathlineto{\pgfqpoint{2.985175in}{3.697598in}}%
\pgfpathlineto{\pgfqpoint{2.986909in}{2.993079in}}%
\pgfpathlineto{\pgfqpoint{2.987776in}{3.354240in}}%
\pgfpathlineto{\pgfqpoint{2.988643in}{3.135978in}}%
\pgfpathlineto{\pgfqpoint{2.989510in}{3.212209in}}%
\pgfpathlineto{\pgfqpoint{2.991245in}{3.828340in}}%
\pgfpathlineto{\pgfqpoint{2.992979in}{3.350992in}}%
\pgfpathlineto{\pgfqpoint{2.993846in}{3.479268in}}%
\pgfpathlineto{\pgfqpoint{2.994713in}{3.396431in}}%
\pgfpathlineto{\pgfqpoint{2.996448in}{3.792644in}}%
\pgfpathlineto{\pgfqpoint{2.998182in}{3.045704in}}%
\pgfpathlineto{\pgfqpoint{2.999049in}{3.086283in}}%
\pgfpathlineto{\pgfqpoint{2.999916in}{3.210235in}}%
\pgfpathlineto{\pgfqpoint{3.000783in}{3.571696in}}%
\pgfpathlineto{\pgfqpoint{3.004252in}{2.828584in}}%
\pgfpathlineto{\pgfqpoint{3.005986in}{3.636950in}}%
\pgfpathlineto{\pgfqpoint{3.006853in}{3.292726in}}%
\pgfpathlineto{\pgfqpoint{3.007720in}{3.633560in}}%
\pgfpathlineto{\pgfqpoint{3.010322in}{3.106123in}}%
\pgfpathlineto{\pgfqpoint{3.011189in}{3.188053in}}%
\pgfpathlineto{\pgfqpoint{3.012056in}{3.710710in}}%
\pgfpathlineto{\pgfqpoint{3.012923in}{3.398422in}}%
\pgfpathlineto{\pgfqpoint{3.013790in}{3.646285in}}%
\pgfpathlineto{\pgfqpoint{3.014657in}{2.992915in}}%
\pgfpathlineto{\pgfqpoint{3.015524in}{3.035600in}}%
\pgfpathlineto{\pgfqpoint{3.016392in}{3.017860in}}%
\pgfpathlineto{\pgfqpoint{3.018993in}{3.770585in}}%
\pgfpathlineto{\pgfqpoint{3.019860in}{3.609261in}}%
\pgfpathlineto{\pgfqpoint{3.020727in}{2.786422in}}%
\pgfpathlineto{\pgfqpoint{3.021594in}{2.879691in}}%
\pgfpathlineto{\pgfqpoint{3.023329in}{3.994324in}}%
\pgfpathlineto{\pgfqpoint{3.025930in}{2.832797in}}%
\pgfpathlineto{\pgfqpoint{3.026797in}{2.871626in}}%
\pgfpathlineto{\pgfqpoint{3.029399in}{3.732462in}}%
\pgfpathlineto{\pgfqpoint{3.032000in}{2.853110in}}%
\pgfpathlineto{\pgfqpoint{3.034601in}{3.741410in}}%
\pgfpathlineto{\pgfqpoint{3.036336in}{3.032017in}}%
\pgfpathlineto{\pgfqpoint{3.037203in}{3.066413in}}%
\pgfpathlineto{\pgfqpoint{3.039804in}{3.643326in}}%
\pgfpathlineto{\pgfqpoint{3.041538in}{3.187387in}}%
\pgfpathlineto{\pgfqpoint{3.042406in}{3.124696in}}%
\pgfpathlineto{\pgfqpoint{3.045007in}{3.569034in}}%
\pgfpathlineto{\pgfqpoint{3.046741in}{3.088750in}}%
\pgfpathlineto{\pgfqpoint{3.047608in}{3.182037in}}%
\pgfpathlineto{\pgfqpoint{3.048476in}{3.129259in}}%
\pgfpathlineto{\pgfqpoint{3.049343in}{3.685277in}}%
\pgfpathlineto{\pgfqpoint{3.050210in}{3.623101in}}%
\pgfpathlineto{\pgfqpoint{3.051077in}{3.681729in}}%
\pgfpathlineto{\pgfqpoint{3.051944in}{2.983326in}}%
\pgfpathlineto{\pgfqpoint{3.052811in}{3.468328in}}%
\pgfpathlineto{\pgfqpoint{3.053678in}{3.456339in}}%
\pgfpathlineto{\pgfqpoint{3.054545in}{3.648455in}}%
\pgfpathlineto{\pgfqpoint{3.056280in}{3.405707in}}%
\pgfpathlineto{\pgfqpoint{3.057147in}{3.296808in}}%
\pgfpathlineto{\pgfqpoint{3.058014in}{2.985140in}}%
\pgfpathlineto{\pgfqpoint{3.060615in}{3.660932in}}%
\pgfpathlineto{\pgfqpoint{3.062350in}{2.845103in}}%
\pgfpathlineto{\pgfqpoint{3.063217in}{2.638902in}}%
\pgfpathlineto{\pgfqpoint{3.064951in}{3.859093in}}%
\pgfpathlineto{\pgfqpoint{3.067552in}{2.978124in}}%
\pgfpathlineto{\pgfqpoint{3.070154in}{3.519983in}}%
\pgfpathlineto{\pgfqpoint{3.071021in}{3.447969in}}%
\pgfpathlineto{\pgfqpoint{3.072755in}{3.056229in}}%
\pgfpathlineto{\pgfqpoint{3.074490in}{3.719010in}}%
\pgfpathlineto{\pgfqpoint{3.076224in}{3.028809in}}%
\pgfpathlineto{\pgfqpoint{3.077091in}{3.471074in}}%
\pgfpathlineto{\pgfqpoint{3.077958in}{3.421112in}}%
\pgfpathlineto{\pgfqpoint{3.078825in}{3.265488in}}%
\pgfpathlineto{\pgfqpoint{3.079692in}{3.770295in}}%
\pgfpathlineto{\pgfqpoint{3.080559in}{3.251865in}}%
\pgfpathlineto{\pgfqpoint{3.081427in}{3.375673in}}%
\pgfpathlineto{\pgfqpoint{3.082294in}{3.399859in}}%
\pgfpathlineto{\pgfqpoint{3.083161in}{3.253196in}}%
\pgfpathlineto{\pgfqpoint{3.084895in}{3.680640in}}%
\pgfpathlineto{\pgfqpoint{3.086629in}{3.189352in}}%
\pgfpathlineto{\pgfqpoint{3.087497in}{3.015497in}}%
\pgfpathlineto{\pgfqpoint{3.089231in}{3.583241in}}%
\pgfpathlineto{\pgfqpoint{3.090965in}{3.338069in}}%
\pgfpathlineto{\pgfqpoint{3.091832in}{3.584009in}}%
\pgfpathlineto{\pgfqpoint{3.092699in}{3.433894in}}%
\pgfpathlineto{\pgfqpoint{3.093566in}{3.589631in}}%
\pgfpathlineto{\pgfqpoint{3.094434in}{3.450721in}}%
\pgfpathlineto{\pgfqpoint{3.095301in}{3.542712in}}%
\pgfpathlineto{\pgfqpoint{3.096168in}{3.087231in}}%
\pgfpathlineto{\pgfqpoint{3.098769in}{3.587956in}}%
\pgfpathlineto{\pgfqpoint{3.099636in}{3.257657in}}%
\pgfpathlineto{\pgfqpoint{3.100503in}{3.517997in}}%
\pgfpathlineto{\pgfqpoint{3.101371in}{3.335295in}}%
\pgfpathlineto{\pgfqpoint{3.103105in}{3.714715in}}%
\pgfpathlineto{\pgfqpoint{3.104839in}{3.324243in}}%
\pgfpathlineto{\pgfqpoint{3.105706in}{3.310771in}}%
\pgfpathlineto{\pgfqpoint{3.106573in}{3.541769in}}%
\pgfpathlineto{\pgfqpoint{3.110042in}{3.091892in}}%
\pgfpathlineto{\pgfqpoint{3.110909in}{3.583603in}}%
\pgfpathlineto{\pgfqpoint{3.111776in}{3.182912in}}%
\pgfpathlineto{\pgfqpoint{3.112643in}{3.621421in}}%
\pgfpathlineto{\pgfqpoint{3.114378in}{3.322690in}}%
\pgfpathlineto{\pgfqpoint{3.115245in}{3.550943in}}%
\pgfpathlineto{\pgfqpoint{3.116112in}{3.431979in}}%
\pgfpathlineto{\pgfqpoint{3.116979in}{3.743737in}}%
\pgfpathlineto{\pgfqpoint{3.117846in}{3.386984in}}%
\pgfpathlineto{\pgfqpoint{3.118713in}{3.492658in}}%
\pgfpathlineto{\pgfqpoint{3.119580in}{3.338536in}}%
\pgfpathlineto{\pgfqpoint{3.120448in}{3.713828in}}%
\pgfpathlineto{\pgfqpoint{3.123049in}{3.386393in}}%
\pgfpathlineto{\pgfqpoint{3.125650in}{3.601114in}}%
\pgfpathlineto{\pgfqpoint{3.126517in}{3.496922in}}%
\pgfpathlineto{\pgfqpoint{3.127385in}{3.120250in}}%
\pgfpathlineto{\pgfqpoint{3.128252in}{3.495767in}}%
\pgfpathlineto{\pgfqpoint{3.129119in}{3.296043in}}%
\pgfpathlineto{\pgfqpoint{3.129986in}{3.561543in}}%
\pgfpathlineto{\pgfqpoint{3.130853in}{3.509226in}}%
\pgfpathlineto{\pgfqpoint{3.132587in}{3.133003in}}%
\pgfpathlineto{\pgfqpoint{3.134322in}{3.603224in}}%
\pgfpathlineto{\pgfqpoint{3.136056in}{2.891625in}}%
\pgfpathlineto{\pgfqpoint{3.136923in}{3.293195in}}%
\pgfpathlineto{\pgfqpoint{3.137790in}{3.182585in}}%
\pgfpathlineto{\pgfqpoint{3.138657in}{3.189111in}}%
\pgfpathlineto{\pgfqpoint{3.139524in}{2.898011in}}%
\pgfpathlineto{\pgfqpoint{3.140392in}{3.540221in}}%
\pgfpathlineto{\pgfqpoint{3.141259in}{3.446814in}}%
\pgfpathlineto{\pgfqpoint{3.142993in}{3.354346in}}%
\pgfpathlineto{\pgfqpoint{3.143860in}{3.776333in}}%
\pgfpathlineto{\pgfqpoint{3.146462in}{2.938063in}}%
\pgfpathlineto{\pgfqpoint{3.147329in}{3.527515in}}%
\pgfpathlineto{\pgfqpoint{3.148196in}{3.299488in}}%
\pgfpathlineto{\pgfqpoint{3.149063in}{3.486724in}}%
\pgfpathlineto{\pgfqpoint{3.150797in}{3.175417in}}%
\pgfpathlineto{\pgfqpoint{3.153399in}{3.433452in}}%
\pgfpathlineto{\pgfqpoint{3.155133in}{3.079420in}}%
\pgfpathlineto{\pgfqpoint{3.157734in}{3.931980in}}%
\pgfpathlineto{\pgfqpoint{3.158601in}{3.635749in}}%
\pgfpathlineto{\pgfqpoint{3.159469in}{2.927222in}}%
\pgfpathlineto{\pgfqpoint{3.160336in}{2.958895in}}%
\pgfpathlineto{\pgfqpoint{3.161203in}{3.297896in}}%
\pgfpathlineto{\pgfqpoint{3.162070in}{3.087777in}}%
\pgfpathlineto{\pgfqpoint{3.162937in}{3.692853in}}%
\pgfpathlineto{\pgfqpoint{3.163804in}{3.145464in}}%
\pgfpathlineto{\pgfqpoint{3.164671in}{3.395728in}}%
\pgfpathlineto{\pgfqpoint{3.165538in}{3.146991in}}%
\pgfpathlineto{\pgfqpoint{3.167273in}{3.675838in}}%
\pgfpathlineto{\pgfqpoint{3.170741in}{3.399076in}}%
\pgfpathlineto{\pgfqpoint{3.171608in}{3.633928in}}%
\pgfpathlineto{\pgfqpoint{3.173343in}{3.123076in}}%
\pgfpathlineto{\pgfqpoint{3.174210in}{3.674329in}}%
\pgfpathlineto{\pgfqpoint{3.175077in}{3.240507in}}%
\pgfpathlineto{\pgfqpoint{3.175944in}{3.294772in}}%
\pgfpathlineto{\pgfqpoint{3.177678in}{3.703901in}}%
\pgfpathlineto{\pgfqpoint{3.178545in}{3.578087in}}%
\pgfpathlineto{\pgfqpoint{3.179413in}{3.513735in}}%
\pgfpathlineto{\pgfqpoint{3.180280in}{3.253461in}}%
\pgfpathlineto{\pgfqpoint{3.181147in}{3.818356in}}%
\pgfpathlineto{\pgfqpoint{3.182014in}{3.140348in}}%
\pgfpathlineto{\pgfqpoint{3.182881in}{3.758060in}}%
\pgfpathlineto{\pgfqpoint{3.183748in}{3.128075in}}%
\pgfpathlineto{\pgfqpoint{3.184615in}{3.131108in}}%
\pgfpathlineto{\pgfqpoint{3.186350in}{3.305993in}}%
\pgfpathlineto{\pgfqpoint{3.187217in}{3.609390in}}%
\pgfpathlineto{\pgfqpoint{3.188951in}{3.227034in}}%
\pgfpathlineto{\pgfqpoint{3.190685in}{3.706992in}}%
\pgfpathlineto{\pgfqpoint{3.192420in}{3.165025in}}%
\pgfpathlineto{\pgfqpoint{3.194154in}{3.559899in}}%
\pgfpathlineto{\pgfqpoint{3.196755in}{3.044333in}}%
\pgfpathlineto{\pgfqpoint{3.197622in}{3.407862in}}%
\pgfpathlineto{\pgfqpoint{3.198490in}{3.376795in}}%
\pgfpathlineto{\pgfqpoint{3.199357in}{3.368709in}}%
\pgfpathlineto{\pgfqpoint{3.200224in}{3.548698in}}%
\pgfpathlineto{\pgfqpoint{3.201091in}{3.530045in}}%
\pgfpathlineto{\pgfqpoint{3.201958in}{3.341969in}}%
\pgfpathlineto{\pgfqpoint{3.202825in}{3.604075in}}%
\pgfpathlineto{\pgfqpoint{3.203692in}{3.216827in}}%
\pgfpathlineto{\pgfqpoint{3.205427in}{3.894294in}}%
\pgfpathlineto{\pgfqpoint{3.206294in}{3.430176in}}%
\pgfpathlineto{\pgfqpoint{3.207161in}{3.623138in}}%
\pgfpathlineto{\pgfqpoint{3.208028in}{3.160677in}}%
\pgfpathlineto{\pgfqpoint{3.209762in}{3.590755in}}%
\pgfpathlineto{\pgfqpoint{3.210629in}{3.635462in}}%
\pgfpathlineto{\pgfqpoint{3.211497in}{3.142272in}}%
\pgfpathlineto{\pgfqpoint{3.212364in}{3.167389in}}%
\pgfpathlineto{\pgfqpoint{3.213231in}{3.225356in}}%
\pgfpathlineto{\pgfqpoint{3.214098in}{3.572015in}}%
\pgfpathlineto{\pgfqpoint{3.214965in}{3.480741in}}%
\pgfpathlineto{\pgfqpoint{3.215832in}{3.472433in}}%
\pgfpathlineto{\pgfqpoint{3.216699in}{3.044742in}}%
\pgfpathlineto{\pgfqpoint{3.218434in}{3.447363in}}%
\pgfpathlineto{\pgfqpoint{3.219301in}{3.384920in}}%
\pgfpathlineto{\pgfqpoint{3.220168in}{3.194134in}}%
\pgfpathlineto{\pgfqpoint{3.221035in}{3.618772in}}%
\pgfpathlineto{\pgfqpoint{3.221902in}{3.472324in}}%
\pgfpathlineto{\pgfqpoint{3.222769in}{3.633625in}}%
\pgfpathlineto{\pgfqpoint{3.223636in}{3.479002in}}%
\pgfpathlineto{\pgfqpoint{3.224503in}{3.490461in}}%
\pgfpathlineto{\pgfqpoint{3.225371in}{3.464026in}}%
\pgfpathlineto{\pgfqpoint{3.226238in}{3.543091in}}%
\pgfpathlineto{\pgfqpoint{3.227105in}{3.017167in}}%
\pgfpathlineto{\pgfqpoint{3.227972in}{3.083636in}}%
\pgfpathlineto{\pgfqpoint{3.228839in}{3.688064in}}%
\pgfpathlineto{\pgfqpoint{3.230573in}{3.236385in}}%
\pgfpathlineto{\pgfqpoint{3.231441in}{3.279291in}}%
\pgfpathlineto{\pgfqpoint{3.232308in}{3.580092in}}%
\pgfpathlineto{\pgfqpoint{3.233175in}{3.499045in}}%
\pgfpathlineto{\pgfqpoint{3.234909in}{3.045410in}}%
\pgfpathlineto{\pgfqpoint{3.235776in}{3.187183in}}%
\pgfpathlineto{\pgfqpoint{3.236643in}{3.593960in}}%
\pgfpathlineto{\pgfqpoint{3.237510in}{3.322059in}}%
\pgfpathlineto{\pgfqpoint{3.238378in}{3.603397in}}%
\pgfpathlineto{\pgfqpoint{3.239245in}{3.372002in}}%
\pgfpathlineto{\pgfqpoint{3.240112in}{3.854384in}}%
\pgfpathlineto{\pgfqpoint{3.240979in}{3.761334in}}%
\pgfpathlineto{\pgfqpoint{3.242713in}{3.065498in}}%
\pgfpathlineto{\pgfqpoint{3.244448in}{3.858244in}}%
\pgfpathlineto{\pgfqpoint{3.246182in}{2.870247in}}%
\pgfpathlineto{\pgfqpoint{3.247916in}{3.818090in}}%
\pgfpathlineto{\pgfqpoint{3.250517in}{3.121063in}}%
\pgfpathlineto{\pgfqpoint{3.251385in}{3.823670in}}%
\pgfpathlineto{\pgfqpoint{3.253119in}{2.943347in}}%
\pgfpathlineto{\pgfqpoint{3.254853in}{3.405387in}}%
\pgfpathlineto{\pgfqpoint{3.255720in}{3.459753in}}%
\pgfpathlineto{\pgfqpoint{3.256587in}{3.006880in}}%
\pgfpathlineto{\pgfqpoint{3.258322in}{3.773045in}}%
\pgfpathlineto{\pgfqpoint{3.260056in}{3.250277in}}%
\pgfpathlineto{\pgfqpoint{3.262657in}{3.911245in}}%
\pgfpathlineto{\pgfqpoint{3.263524in}{3.194453in}}%
\pgfpathlineto{\pgfqpoint{3.264392in}{3.299832in}}%
\pgfpathlineto{\pgfqpoint{3.265259in}{3.199383in}}%
\pgfpathlineto{\pgfqpoint{3.266126in}{3.638836in}}%
\pgfpathlineto{\pgfqpoint{3.267860in}{2.873373in}}%
\pgfpathlineto{\pgfqpoint{3.269594in}{3.709516in}}%
\pgfpathlineto{\pgfqpoint{3.271329in}{2.991599in}}%
\pgfpathlineto{\pgfqpoint{3.272196in}{3.665448in}}%
\pgfpathlineto{\pgfqpoint{3.273063in}{3.509509in}}%
\pgfpathlineto{\pgfqpoint{3.273930in}{3.486353in}}%
\pgfpathlineto{\pgfqpoint{3.274797in}{2.679967in}}%
\pgfpathlineto{\pgfqpoint{3.276531in}{3.502403in}}%
\pgfpathlineto{\pgfqpoint{3.277399in}{3.516685in}}%
\pgfpathlineto{\pgfqpoint{3.278266in}{2.978376in}}%
\pgfpathlineto{\pgfqpoint{3.279133in}{3.486508in}}%
\pgfpathlineto{\pgfqpoint{3.280000in}{3.405223in}}%
\pgfpathlineto{\pgfqpoint{3.280867in}{3.386843in}}%
\pgfpathlineto{\pgfqpoint{3.281734in}{3.108316in}}%
\pgfpathlineto{\pgfqpoint{3.283469in}{3.810351in}}%
\pgfpathlineto{\pgfqpoint{3.284336in}{3.582877in}}%
\pgfpathlineto{\pgfqpoint{3.285203in}{3.185705in}}%
\pgfpathlineto{\pgfqpoint{3.286070in}{3.770613in}}%
\pgfpathlineto{\pgfqpoint{3.288671in}{3.124945in}}%
\pgfpathlineto{\pgfqpoint{3.290406in}{3.662176in}}%
\pgfpathlineto{\pgfqpoint{3.292140in}{2.750858in}}%
\pgfpathlineto{\pgfqpoint{3.293874in}{4.006574in}}%
\pgfpathlineto{\pgfqpoint{3.295608in}{3.121518in}}%
\pgfpathlineto{\pgfqpoint{3.296476in}{3.767750in}}%
\pgfpathlineto{\pgfqpoint{3.297343in}{3.709597in}}%
\pgfpathlineto{\pgfqpoint{3.299077in}{2.867036in}}%
\pgfpathlineto{\pgfqpoint{3.300811in}{3.528420in}}%
\pgfpathlineto{\pgfqpoint{3.302545in}{3.083959in}}%
\pgfpathlineto{\pgfqpoint{3.303413in}{3.767829in}}%
\pgfpathlineto{\pgfqpoint{3.305147in}{3.260401in}}%
\pgfpathlineto{\pgfqpoint{3.306014in}{3.310392in}}%
\pgfpathlineto{\pgfqpoint{3.306881in}{3.753764in}}%
\pgfpathlineto{\pgfqpoint{3.307748in}{3.669690in}}%
\pgfpathlineto{\pgfqpoint{3.309483in}{3.184171in}}%
\pgfpathlineto{\pgfqpoint{3.310350in}{3.717691in}}%
\pgfpathlineto{\pgfqpoint{3.312951in}{3.089168in}}%
\pgfpathlineto{\pgfqpoint{3.313818in}{3.846053in}}%
\pgfpathlineto{\pgfqpoint{3.314685in}{3.158691in}}%
\pgfpathlineto{\pgfqpoint{3.315552in}{3.199110in}}%
\pgfpathlineto{\pgfqpoint{3.316420in}{3.331909in}}%
\pgfpathlineto{\pgfqpoint{3.317287in}{3.752776in}}%
\pgfpathlineto{\pgfqpoint{3.318154in}{3.664530in}}%
\pgfpathlineto{\pgfqpoint{3.319021in}{3.027964in}}%
\pgfpathlineto{\pgfqpoint{3.319888in}{3.558794in}}%
\pgfpathlineto{\pgfqpoint{3.320755in}{3.463546in}}%
\pgfpathlineto{\pgfqpoint{3.322490in}{3.140203in}}%
\pgfpathlineto{\pgfqpoint{3.324224in}{3.837203in}}%
\pgfpathlineto{\pgfqpoint{3.325091in}{3.064562in}}%
\pgfpathlineto{\pgfqpoint{3.326825in}{3.489931in}}%
\pgfpathlineto{\pgfqpoint{3.327692in}{3.567351in}}%
\pgfpathlineto{\pgfqpoint{3.329427in}{3.058962in}}%
\pgfpathlineto{\pgfqpoint{3.331161in}{3.668523in}}%
\pgfpathlineto{\pgfqpoint{3.332895in}{3.061677in}}%
\pgfpathlineto{\pgfqpoint{3.333762in}{3.789128in}}%
\pgfpathlineto{\pgfqpoint{3.335497in}{2.916066in}}%
\pgfpathlineto{\pgfqpoint{3.337231in}{3.863172in}}%
\pgfpathlineto{\pgfqpoint{3.338965in}{3.177569in}}%
\pgfpathlineto{\pgfqpoint{3.340699in}{3.577757in}}%
\pgfpathlineto{\pgfqpoint{3.342434in}{3.039997in}}%
\pgfpathlineto{\pgfqpoint{3.343301in}{3.576681in}}%
\pgfpathlineto{\pgfqpoint{3.345035in}{3.177290in}}%
\pgfpathlineto{\pgfqpoint{3.345902in}{3.670871in}}%
\pgfpathlineto{\pgfqpoint{3.346769in}{3.659108in}}%
\pgfpathlineto{\pgfqpoint{3.348503in}{2.951385in}}%
\pgfpathlineto{\pgfqpoint{3.350238in}{3.335425in}}%
\pgfpathlineto{\pgfqpoint{3.351105in}{3.167756in}}%
\pgfpathlineto{\pgfqpoint{3.351972in}{3.266449in}}%
\pgfpathlineto{\pgfqpoint{3.352839in}{3.642762in}}%
\pgfpathlineto{\pgfqpoint{3.353706in}{3.526606in}}%
\pgfpathlineto{\pgfqpoint{3.354573in}{2.867570in}}%
\pgfpathlineto{\pgfqpoint{3.356308in}{3.686379in}}%
\pgfpathlineto{\pgfqpoint{3.357175in}{3.140154in}}%
\pgfpathlineto{\pgfqpoint{3.359776in}{3.571753in}}%
\pgfpathlineto{\pgfqpoint{3.360643in}{3.579454in}}%
\pgfpathlineto{\pgfqpoint{3.361510in}{3.653200in}}%
\pgfpathlineto{\pgfqpoint{3.362378in}{3.343409in}}%
\pgfpathlineto{\pgfqpoint{3.363245in}{3.702703in}}%
\pgfpathlineto{\pgfqpoint{3.364112in}{2.903380in}}%
\pgfpathlineto{\pgfqpoint{3.365846in}{3.451015in}}%
\pgfpathlineto{\pgfqpoint{3.367580in}{3.135939in}}%
\pgfpathlineto{\pgfqpoint{3.369315in}{3.675636in}}%
\pgfpathlineto{\pgfqpoint{3.370182in}{3.296496in}}%
\pgfpathlineto{\pgfqpoint{3.371049in}{3.368788in}}%
\pgfpathlineto{\pgfqpoint{3.371916in}{3.421510in}}%
\pgfpathlineto{\pgfqpoint{3.374517in}{3.119013in}}%
\pgfpathlineto{\pgfqpoint{3.375385in}{3.195691in}}%
\pgfpathlineto{\pgfqpoint{3.377119in}{3.623507in}}%
\pgfpathlineto{\pgfqpoint{3.377986in}{3.558053in}}%
\pgfpathlineto{\pgfqpoint{3.379720in}{3.241939in}}%
\pgfpathlineto{\pgfqpoint{3.380587in}{3.657840in}}%
\pgfpathlineto{\pgfqpoint{3.382322in}{3.061203in}}%
\pgfpathlineto{\pgfqpoint{3.384056in}{3.369400in}}%
\pgfpathlineto{\pgfqpoint{3.384923in}{3.367445in}}%
\pgfpathlineto{\pgfqpoint{3.385790in}{3.351818in}}%
\pgfpathlineto{\pgfqpoint{3.388392in}{3.412740in}}%
\pgfpathlineto{\pgfqpoint{3.389259in}{3.625785in}}%
\pgfpathlineto{\pgfqpoint{3.390126in}{3.506778in}}%
\pgfpathlineto{\pgfqpoint{3.391860in}{3.185007in}}%
\pgfpathlineto{\pgfqpoint{3.393594in}{3.401212in}}%
\pgfpathlineto{\pgfqpoint{3.394462in}{3.211662in}}%
\pgfpathlineto{\pgfqpoint{3.396196in}{3.547402in}}%
\pgfpathlineto{\pgfqpoint{3.397063in}{3.161634in}}%
\pgfpathlineto{\pgfqpoint{3.397930in}{3.462136in}}%
\pgfpathlineto{\pgfqpoint{3.398797in}{3.149413in}}%
\pgfpathlineto{\pgfqpoint{3.399664in}{3.678514in}}%
\pgfpathlineto{\pgfqpoint{3.400531in}{3.365803in}}%
\pgfpathlineto{\pgfqpoint{3.401399in}{3.401704in}}%
\pgfpathlineto{\pgfqpoint{3.402266in}{3.506546in}}%
\pgfpathlineto{\pgfqpoint{3.403133in}{3.316099in}}%
\pgfpathlineto{\pgfqpoint{3.404000in}{3.623456in}}%
\pgfpathlineto{\pgfqpoint{3.405734in}{3.343918in}}%
\pgfpathlineto{\pgfqpoint{3.406601in}{3.303451in}}%
\pgfpathlineto{\pgfqpoint{3.407469in}{3.040272in}}%
\pgfpathlineto{\pgfqpoint{3.408336in}{3.362895in}}%
\pgfpathlineto{\pgfqpoint{3.409203in}{3.149869in}}%
\pgfpathlineto{\pgfqpoint{3.410070in}{3.731231in}}%
\pgfpathlineto{\pgfqpoint{3.410937in}{3.702937in}}%
\pgfpathlineto{\pgfqpoint{3.411804in}{3.973610in}}%
\pgfpathlineto{\pgfqpoint{3.413538in}{3.494202in}}%
\pgfpathlineto{\pgfqpoint{3.414406in}{3.636526in}}%
\pgfpathlineto{\pgfqpoint{3.415273in}{3.206448in}}%
\pgfpathlineto{\pgfqpoint{3.416140in}{3.479530in}}%
\pgfpathlineto{\pgfqpoint{3.417007in}{3.314605in}}%
\pgfpathlineto{\pgfqpoint{3.417874in}{3.645653in}}%
\pgfpathlineto{\pgfqpoint{3.418741in}{3.447919in}}%
\pgfpathlineto{\pgfqpoint{3.419608in}{3.467874in}}%
\pgfpathlineto{\pgfqpoint{3.420476in}{3.346408in}}%
\pgfpathlineto{\pgfqpoint{3.421343in}{3.507507in}}%
\pgfpathlineto{\pgfqpoint{3.422210in}{3.208469in}}%
\pgfpathlineto{\pgfqpoint{3.423077in}{3.234246in}}%
\pgfpathlineto{\pgfqpoint{3.424811in}{3.621348in}}%
\pgfpathlineto{\pgfqpoint{3.425678in}{3.455641in}}%
\pgfpathlineto{\pgfqpoint{3.427413in}{3.760228in}}%
\pgfpathlineto{\pgfqpoint{3.428280in}{3.747218in}}%
\pgfpathlineto{\pgfqpoint{3.430014in}{3.344481in}}%
\pgfpathlineto{\pgfqpoint{3.430881in}{3.627306in}}%
\pgfpathlineto{\pgfqpoint{3.431748in}{3.291409in}}%
\pgfpathlineto{\pgfqpoint{3.433483in}{3.586096in}}%
\pgfpathlineto{\pgfqpoint{3.434350in}{3.584184in}}%
\pgfpathlineto{\pgfqpoint{3.435217in}{3.185450in}}%
\pgfpathlineto{\pgfqpoint{3.436084in}{3.644616in}}%
\pgfpathlineto{\pgfqpoint{3.437818in}{3.335042in}}%
\pgfpathlineto{\pgfqpoint{3.438685in}{3.227953in}}%
\pgfpathlineto{\pgfqpoint{3.440420in}{3.519042in}}%
\pgfpathlineto{\pgfqpoint{3.441287in}{3.472556in}}%
\pgfpathlineto{\pgfqpoint{3.442154in}{3.499799in}}%
\pgfpathlineto{\pgfqpoint{3.443021in}{3.448695in}}%
\pgfpathlineto{\pgfqpoint{3.444755in}{3.131148in}}%
\pgfpathlineto{\pgfqpoint{3.445622in}{3.519695in}}%
\pgfpathlineto{\pgfqpoint{3.446490in}{3.437868in}}%
\pgfpathlineto{\pgfqpoint{3.448224in}{3.366871in}}%
\pgfpathlineto{\pgfqpoint{3.449091in}{3.480413in}}%
\pgfpathlineto{\pgfqpoint{3.450825in}{3.089811in}}%
\pgfpathlineto{\pgfqpoint{3.451692in}{3.140620in}}%
\pgfpathlineto{\pgfqpoint{3.453427in}{3.562224in}}%
\pgfpathlineto{\pgfqpoint{3.454294in}{3.232764in}}%
\pgfpathlineto{\pgfqpoint{3.455161in}{3.283989in}}%
\pgfpathlineto{\pgfqpoint{3.456028in}{3.479979in}}%
\pgfpathlineto{\pgfqpoint{3.456895in}{3.454629in}}%
\pgfpathlineto{\pgfqpoint{3.458629in}{3.563882in}}%
\pgfpathlineto{\pgfqpoint{3.460364in}{3.286496in}}%
\pgfpathlineto{\pgfqpoint{3.461231in}{3.499275in}}%
\pgfpathlineto{\pgfqpoint{3.462965in}{3.277214in}}%
\pgfpathlineto{\pgfqpoint{3.463832in}{3.360718in}}%
\pgfpathlineto{\pgfqpoint{3.464699in}{3.749200in}}%
\pgfpathlineto{\pgfqpoint{3.466434in}{3.262488in}}%
\pgfpathlineto{\pgfqpoint{3.468168in}{3.338392in}}%
\pgfpathlineto{\pgfqpoint{3.469035in}{3.169363in}}%
\pgfpathlineto{\pgfqpoint{3.470769in}{3.353190in}}%
\pgfpathlineto{\pgfqpoint{3.471636in}{3.362505in}}%
\pgfpathlineto{\pgfqpoint{3.472503in}{3.668791in}}%
\pgfpathlineto{\pgfqpoint{3.473371in}{3.138719in}}%
\pgfpathlineto{\pgfqpoint{3.475105in}{3.414855in}}%
\pgfpathlineto{\pgfqpoint{3.475972in}{3.548813in}}%
\pgfpathlineto{\pgfqpoint{3.477706in}{3.136041in}}%
\pgfpathlineto{\pgfqpoint{3.478573in}{3.601711in}}%
\pgfpathlineto{\pgfqpoint{3.479441in}{2.965481in}}%
\pgfpathlineto{\pgfqpoint{3.480308in}{3.591621in}}%
\pgfpathlineto{\pgfqpoint{3.481175in}{3.579950in}}%
\pgfpathlineto{\pgfqpoint{3.482042in}{3.490504in}}%
\pgfpathlineto{\pgfqpoint{3.482909in}{3.130314in}}%
\pgfpathlineto{\pgfqpoint{3.483776in}{3.392168in}}%
\pgfpathlineto{\pgfqpoint{3.485510in}{3.004260in}}%
\pgfpathlineto{\pgfqpoint{3.486378in}{3.784575in}}%
\pgfpathlineto{\pgfqpoint{3.488112in}{2.993972in}}%
\pgfpathlineto{\pgfqpoint{3.488979in}{3.655913in}}%
\pgfpathlineto{\pgfqpoint{3.490713in}{3.113837in}}%
\pgfpathlineto{\pgfqpoint{3.491580in}{3.668203in}}%
\pgfpathlineto{\pgfqpoint{3.493315in}{2.897202in}}%
\pgfpathlineto{\pgfqpoint{3.494182in}{3.561112in}}%
\pgfpathlineto{\pgfqpoint{3.495049in}{3.109053in}}%
\pgfpathlineto{\pgfqpoint{3.497650in}{3.639029in}}%
\pgfpathlineto{\pgfqpoint{3.498517in}{3.415277in}}%
\pgfpathlineto{\pgfqpoint{3.500252in}{3.575985in}}%
\pgfpathlineto{\pgfqpoint{3.501119in}{3.183450in}}%
\pgfpathlineto{\pgfqpoint{3.502853in}{3.866630in}}%
\pgfpathlineto{\pgfqpoint{3.503720in}{3.273231in}}%
\pgfpathlineto{\pgfqpoint{3.504587in}{3.429165in}}%
\pgfpathlineto{\pgfqpoint{3.506322in}{3.799490in}}%
\pgfpathlineto{\pgfqpoint{3.507189in}{3.343305in}}%
\pgfpathlineto{\pgfqpoint{3.508056in}{3.414445in}}%
\pgfpathlineto{\pgfqpoint{3.508923in}{3.078751in}}%
\pgfpathlineto{\pgfqpoint{3.510657in}{3.709693in}}%
\pgfpathlineto{\pgfqpoint{3.512392in}{3.117590in}}%
\pgfpathlineto{\pgfqpoint{3.513259in}{3.436351in}}%
\pgfpathlineto{\pgfqpoint{3.514126in}{3.174526in}}%
\pgfpathlineto{\pgfqpoint{3.515860in}{3.688983in}}%
\pgfpathlineto{\pgfqpoint{3.517594in}{3.393609in}}%
\pgfpathlineto{\pgfqpoint{3.518462in}{3.481197in}}%
\pgfpathlineto{\pgfqpoint{3.519329in}{3.267792in}}%
\pgfpathlineto{\pgfqpoint{3.520196in}{3.438055in}}%
\pgfpathlineto{\pgfqpoint{3.521063in}{2.959879in}}%
\pgfpathlineto{\pgfqpoint{3.523664in}{3.880735in}}%
\pgfpathlineto{\pgfqpoint{3.524531in}{3.223003in}}%
\pgfpathlineto{\pgfqpoint{3.526266in}{3.751665in}}%
\pgfpathlineto{\pgfqpoint{3.527133in}{3.267120in}}%
\pgfpathlineto{\pgfqpoint{3.528000in}{3.308412in}}%
\pgfpathlineto{\pgfqpoint{3.528867in}{3.894963in}}%
\pgfpathlineto{\pgfqpoint{3.529734in}{3.112837in}}%
\pgfpathlineto{\pgfqpoint{3.531469in}{3.565249in}}%
\pgfpathlineto{\pgfqpoint{3.532336in}{3.450793in}}%
\pgfpathlineto{\pgfqpoint{3.533203in}{2.996844in}}%
\pgfpathlineto{\pgfqpoint{3.534070in}{3.586403in}}%
\pgfpathlineto{\pgfqpoint{3.534937in}{3.183660in}}%
\pgfpathlineto{\pgfqpoint{3.535804in}{3.204311in}}%
\pgfpathlineto{\pgfqpoint{3.536671in}{3.874362in}}%
\pgfpathlineto{\pgfqpoint{3.537538in}{3.287010in}}%
\pgfpathlineto{\pgfqpoint{3.538406in}{3.322790in}}%
\pgfpathlineto{\pgfqpoint{3.539273in}{3.420973in}}%
\pgfpathlineto{\pgfqpoint{3.540140in}{3.405196in}}%
\pgfpathlineto{\pgfqpoint{3.541007in}{3.299607in}}%
\pgfpathlineto{\pgfqpoint{3.541874in}{3.732233in}}%
\pgfpathlineto{\pgfqpoint{3.542741in}{3.139049in}}%
\pgfpathlineto{\pgfqpoint{3.543608in}{3.145398in}}%
\pgfpathlineto{\pgfqpoint{3.544476in}{3.370023in}}%
\pgfpathlineto{\pgfqpoint{3.545343in}{3.119642in}}%
\pgfpathlineto{\pgfqpoint{3.547077in}{3.873429in}}%
\pgfpathlineto{\pgfqpoint{3.547944in}{3.138197in}}%
\pgfpathlineto{\pgfqpoint{3.549678in}{3.476569in}}%
\pgfpathlineto{\pgfqpoint{3.550545in}{3.272729in}}%
\pgfpathlineto{\pgfqpoint{3.552280in}{4.080389in}}%
\pgfpathlineto{\pgfqpoint{3.553147in}{2.836185in}}%
\pgfpathlineto{\pgfqpoint{3.554881in}{3.635268in}}%
\pgfpathlineto{\pgfqpoint{3.555748in}{3.006403in}}%
\pgfpathlineto{\pgfqpoint{3.557483in}{3.652490in}}%
\pgfpathlineto{\pgfqpoint{3.558350in}{3.351663in}}%
\pgfpathlineto{\pgfqpoint{3.559217in}{3.627689in}}%
\pgfpathlineto{\pgfqpoint{3.560951in}{3.061190in}}%
\pgfpathlineto{\pgfqpoint{3.561818in}{3.520893in}}%
\pgfpathlineto{\pgfqpoint{3.563552in}{2.938406in}}%
\pgfpathlineto{\pgfqpoint{3.564420in}{3.674203in}}%
\pgfpathlineto{\pgfqpoint{3.566154in}{2.924855in}}%
\pgfpathlineto{\pgfqpoint{3.567021in}{3.435902in}}%
\pgfpathlineto{\pgfqpoint{3.567888in}{3.399323in}}%
\pgfpathlineto{\pgfqpoint{3.568755in}{2.934988in}}%
\pgfpathlineto{\pgfqpoint{3.569622in}{3.819369in}}%
\pgfpathlineto{\pgfqpoint{3.570490in}{3.000819in}}%
\pgfpathlineto{\pgfqpoint{3.571357in}{3.208220in}}%
\pgfpathlineto{\pgfqpoint{3.572224in}{3.800680in}}%
\pgfpathlineto{\pgfqpoint{3.573091in}{3.112273in}}%
\pgfpathlineto{\pgfqpoint{3.573958in}{3.229646in}}%
\pgfpathlineto{\pgfqpoint{3.574825in}{3.584658in}}%
\pgfpathlineto{\pgfqpoint{3.575692in}{3.244714in}}%
\pgfpathlineto{\pgfqpoint{3.576559in}{3.320950in}}%
\pgfpathlineto{\pgfqpoint{3.577427in}{3.662652in}}%
\pgfpathlineto{\pgfqpoint{3.578294in}{3.044546in}}%
\pgfpathlineto{\pgfqpoint{3.580028in}{3.794154in}}%
\pgfpathlineto{\pgfqpoint{3.580895in}{2.961774in}}%
\pgfpathlineto{\pgfqpoint{3.582629in}{3.639591in}}%
\pgfpathlineto{\pgfqpoint{3.583497in}{2.940280in}}%
\pgfpathlineto{\pgfqpoint{3.584364in}{3.782037in}}%
\pgfpathlineto{\pgfqpoint{3.586098in}{3.054190in}}%
\pgfpathlineto{\pgfqpoint{3.586965in}{3.761387in}}%
\pgfpathlineto{\pgfqpoint{3.587832in}{3.344321in}}%
\pgfpathlineto{\pgfqpoint{3.589566in}{3.741155in}}%
\pgfpathlineto{\pgfqpoint{3.590434in}{3.036403in}}%
\pgfpathlineto{\pgfqpoint{3.591301in}{3.545669in}}%
\pgfpathlineto{\pgfqpoint{3.593035in}{3.141787in}}%
\pgfpathlineto{\pgfqpoint{3.593902in}{3.557740in}}%
\pgfpathlineto{\pgfqpoint{3.594769in}{3.506057in}}%
\pgfpathlineto{\pgfqpoint{3.595636in}{3.064986in}}%
\pgfpathlineto{\pgfqpoint{3.596503in}{3.340966in}}%
\pgfpathlineto{\pgfqpoint{3.597371in}{3.305451in}}%
\pgfpathlineto{\pgfqpoint{3.598238in}{3.299998in}}%
\pgfpathlineto{\pgfqpoint{3.599105in}{3.959390in}}%
\pgfpathlineto{\pgfqpoint{3.599972in}{3.037836in}}%
\pgfpathlineto{\pgfqpoint{3.601706in}{3.659760in}}%
\pgfpathlineto{\pgfqpoint{3.602573in}{3.198009in}}%
\pgfpathlineto{\pgfqpoint{3.603441in}{3.827530in}}%
\pgfpathlineto{\pgfqpoint{3.605175in}{3.149976in}}%
\pgfpathlineto{\pgfqpoint{3.606042in}{3.866633in}}%
\pgfpathlineto{\pgfqpoint{3.607776in}{2.964914in}}%
\pgfpathlineto{\pgfqpoint{3.608643in}{3.839484in}}%
\pgfpathlineto{\pgfqpoint{3.610378in}{3.065809in}}%
\pgfpathlineto{\pgfqpoint{3.611245in}{3.424319in}}%
\pgfpathlineto{\pgfqpoint{3.612112in}{3.178960in}}%
\pgfpathlineto{\pgfqpoint{3.612979in}{3.295207in}}%
\pgfpathlineto{\pgfqpoint{3.613846in}{3.650227in}}%
\pgfpathlineto{\pgfqpoint{3.614713in}{3.185683in}}%
\pgfpathlineto{\pgfqpoint{3.615580in}{3.670625in}}%
\pgfpathlineto{\pgfqpoint{3.617315in}{2.783054in}}%
\pgfpathlineto{\pgfqpoint{3.618182in}{3.427950in}}%
\pgfpathlineto{\pgfqpoint{3.619049in}{3.057877in}}%
\pgfpathlineto{\pgfqpoint{3.620783in}{3.688826in}}%
\pgfpathlineto{\pgfqpoint{3.622517in}{3.361464in}}%
\pgfpathlineto{\pgfqpoint{3.623385in}{3.695824in}}%
\pgfpathlineto{\pgfqpoint{3.624252in}{3.255728in}}%
\pgfpathlineto{\pgfqpoint{3.625119in}{3.476001in}}%
\pgfpathlineto{\pgfqpoint{3.625986in}{3.462695in}}%
\pgfpathlineto{\pgfqpoint{3.626853in}{3.415578in}}%
\pgfpathlineto{\pgfqpoint{3.627720in}{3.528961in}}%
\pgfpathlineto{\pgfqpoint{3.628587in}{3.487293in}}%
\pgfpathlineto{\pgfqpoint{3.629455in}{3.222901in}}%
\pgfpathlineto{\pgfqpoint{3.630322in}{3.936782in}}%
\pgfpathlineto{\pgfqpoint{3.631189in}{2.928523in}}%
\pgfpathlineto{\pgfqpoint{3.632056in}{3.166765in}}%
\pgfpathlineto{\pgfqpoint{3.632923in}{3.353929in}}%
\pgfpathlineto{\pgfqpoint{3.633790in}{3.192467in}}%
\pgfpathlineto{\pgfqpoint{3.634657in}{3.475037in}}%
\pgfpathlineto{\pgfqpoint{3.636392in}{3.303403in}}%
\pgfpathlineto{\pgfqpoint{3.637259in}{3.467136in}}%
\pgfpathlineto{\pgfqpoint{3.638993in}{3.368549in}}%
\pgfpathlineto{\pgfqpoint{3.639860in}{3.466081in}}%
\pgfpathlineto{\pgfqpoint{3.640727in}{3.116186in}}%
\pgfpathlineto{\pgfqpoint{3.642462in}{3.658421in}}%
\pgfpathlineto{\pgfqpoint{3.644196in}{3.444037in}}%
\pgfpathlineto{\pgfqpoint{3.645063in}{3.629331in}}%
\pgfpathlineto{\pgfqpoint{3.646797in}{3.266774in}}%
\pgfpathlineto{\pgfqpoint{3.647664in}{3.144454in}}%
\pgfpathlineto{\pgfqpoint{3.649399in}{3.520779in}}%
\pgfpathlineto{\pgfqpoint{3.650266in}{3.106073in}}%
\pgfpathlineto{\pgfqpoint{3.651133in}{3.484678in}}%
\pgfpathlineto{\pgfqpoint{3.652000in}{3.155839in}}%
\pgfpathlineto{\pgfqpoint{3.653734in}{3.699195in}}%
\pgfpathlineto{\pgfqpoint{3.654601in}{3.116696in}}%
\pgfpathlineto{\pgfqpoint{3.655469in}{3.789161in}}%
\pgfpathlineto{\pgfqpoint{3.656336in}{3.422793in}}%
\pgfpathlineto{\pgfqpoint{3.658070in}{3.789614in}}%
\pgfpathlineto{\pgfqpoint{3.659804in}{3.377465in}}%
\pgfpathlineto{\pgfqpoint{3.660671in}{3.568157in}}%
\pgfpathlineto{\pgfqpoint{3.661538in}{3.508827in}}%
\pgfpathlineto{\pgfqpoint{3.662406in}{3.525011in}}%
\pgfpathlineto{\pgfqpoint{3.664140in}{3.071437in}}%
\pgfpathlineto{\pgfqpoint{3.665007in}{3.622796in}}%
\pgfpathlineto{\pgfqpoint{3.665874in}{3.541858in}}%
\pgfpathlineto{\pgfqpoint{3.666741in}{3.321598in}}%
\pgfpathlineto{\pgfqpoint{3.667608in}{3.921092in}}%
\pgfpathlineto{\pgfqpoint{3.668476in}{3.195653in}}%
\pgfpathlineto{\pgfqpoint{3.669343in}{3.728169in}}%
\pgfpathlineto{\pgfqpoint{3.670210in}{3.026475in}}%
\pgfpathlineto{\pgfqpoint{3.671077in}{3.440236in}}%
\pgfpathlineto{\pgfqpoint{3.671944in}{3.433921in}}%
\pgfpathlineto{\pgfqpoint{3.673678in}{3.404636in}}%
\pgfpathlineto{\pgfqpoint{3.674545in}{3.697873in}}%
\pgfpathlineto{\pgfqpoint{3.675413in}{3.364424in}}%
\pgfpathlineto{\pgfqpoint{3.676280in}{3.680279in}}%
\pgfpathlineto{\pgfqpoint{3.677147in}{3.256098in}}%
\pgfpathlineto{\pgfqpoint{3.678014in}{3.261965in}}%
\pgfpathlineto{\pgfqpoint{3.678881in}{3.429367in}}%
\pgfpathlineto{\pgfqpoint{3.679748in}{3.317844in}}%
\pgfpathlineto{\pgfqpoint{3.680615in}{3.468099in}}%
\pgfpathlineto{\pgfqpoint{3.681483in}{3.408394in}}%
\pgfpathlineto{\pgfqpoint{3.682350in}{3.195728in}}%
\pgfpathlineto{\pgfqpoint{3.683217in}{3.566027in}}%
\pgfpathlineto{\pgfqpoint{3.684084in}{3.097494in}}%
\pgfpathlineto{\pgfqpoint{3.684951in}{3.348536in}}%
\pgfpathlineto{\pgfqpoint{3.687552in}{3.142853in}}%
\pgfpathlineto{\pgfqpoint{3.689287in}{3.262509in}}%
\pgfpathlineto{\pgfqpoint{3.690154in}{3.478762in}}%
\pgfpathlineto{\pgfqpoint{3.691021in}{3.293650in}}%
\pgfpathlineto{\pgfqpoint{3.691888in}{3.855459in}}%
\pgfpathlineto{\pgfqpoint{3.692755in}{3.000328in}}%
\pgfpathlineto{\pgfqpoint{3.694490in}{3.774038in}}%
\pgfpathlineto{\pgfqpoint{3.697091in}{3.005718in}}%
\pgfpathlineto{\pgfqpoint{3.697958in}{3.467649in}}%
\pgfpathlineto{\pgfqpoint{3.698825in}{3.409715in}}%
\pgfpathlineto{\pgfqpoint{3.699692in}{3.473039in}}%
\pgfpathlineto{\pgfqpoint{3.701427in}{3.181698in}}%
\pgfpathlineto{\pgfqpoint{3.702294in}{3.477647in}}%
\pgfpathlineto{\pgfqpoint{3.704028in}{3.194704in}}%
\pgfpathlineto{\pgfqpoint{3.704895in}{3.234916in}}%
\pgfpathlineto{\pgfqpoint{3.705762in}{3.074309in}}%
\pgfpathlineto{\pgfqpoint{3.706629in}{3.258084in}}%
\pgfpathlineto{\pgfqpoint{3.707497in}{3.737598in}}%
\pgfpathlineto{\pgfqpoint{3.708364in}{3.180656in}}%
\pgfpathlineto{\pgfqpoint{3.709231in}{3.291375in}}%
\pgfpathlineto{\pgfqpoint{3.710098in}{3.284748in}}%
\pgfpathlineto{\pgfqpoint{3.710965in}{3.605394in}}%
\pgfpathlineto{\pgfqpoint{3.711832in}{3.417732in}}%
\pgfpathlineto{\pgfqpoint{3.712699in}{3.549342in}}%
\pgfpathlineto{\pgfqpoint{3.713566in}{3.259471in}}%
\pgfpathlineto{\pgfqpoint{3.714434in}{3.674632in}}%
\pgfpathlineto{\pgfqpoint{3.715301in}{3.532559in}}%
\pgfpathlineto{\pgfqpoint{3.716168in}{3.655739in}}%
\pgfpathlineto{\pgfqpoint{3.717035in}{3.290108in}}%
\pgfpathlineto{\pgfqpoint{3.717902in}{3.321675in}}%
\pgfpathlineto{\pgfqpoint{3.718769in}{3.518972in}}%
\pgfpathlineto{\pgfqpoint{3.719636in}{3.508024in}}%
\pgfpathlineto{\pgfqpoint{3.720503in}{3.319356in}}%
\pgfpathlineto{\pgfqpoint{3.721371in}{3.434930in}}%
\pgfpathlineto{\pgfqpoint{3.723972in}{3.297551in}}%
\pgfpathlineto{\pgfqpoint{3.724839in}{3.690330in}}%
\pgfpathlineto{\pgfqpoint{3.725706in}{3.580245in}}%
\pgfpathlineto{\pgfqpoint{3.726573in}{3.557119in}}%
\pgfpathlineto{\pgfqpoint{3.727441in}{3.278754in}}%
\pgfpathlineto{\pgfqpoint{3.729175in}{3.496061in}}%
\pgfpathlineto{\pgfqpoint{3.730909in}{3.130236in}}%
\pgfpathlineto{\pgfqpoint{3.731776in}{3.449818in}}%
\pgfpathlineto{\pgfqpoint{3.732643in}{3.335697in}}%
\pgfpathlineto{\pgfqpoint{3.733510in}{3.455263in}}%
\pgfpathlineto{\pgfqpoint{3.734378in}{3.881722in}}%
\pgfpathlineto{\pgfqpoint{3.736112in}{3.383475in}}%
\pgfpathlineto{\pgfqpoint{3.736979in}{3.393877in}}%
\pgfpathlineto{\pgfqpoint{3.737846in}{3.131625in}}%
\pgfpathlineto{\pgfqpoint{3.738713in}{3.742177in}}%
\pgfpathlineto{\pgfqpoint{3.739580in}{3.592767in}}%
\pgfpathlineto{\pgfqpoint{3.741315in}{3.112466in}}%
\pgfpathlineto{\pgfqpoint{3.743049in}{3.562336in}}%
\pgfpathlineto{\pgfqpoint{3.743916in}{3.638296in}}%
\pgfpathlineto{\pgfqpoint{3.744783in}{3.248216in}}%
\pgfpathlineto{\pgfqpoint{3.745650in}{3.257093in}}%
\pgfpathlineto{\pgfqpoint{3.746517in}{3.239197in}}%
\pgfpathlineto{\pgfqpoint{3.748252in}{3.544515in}}%
\pgfpathlineto{\pgfqpoint{3.749119in}{3.593192in}}%
\pgfpathlineto{\pgfqpoint{3.749986in}{3.722762in}}%
\pgfpathlineto{\pgfqpoint{3.750853in}{3.250006in}}%
\pgfpathlineto{\pgfqpoint{3.751720in}{3.643022in}}%
\pgfpathlineto{\pgfqpoint{3.752587in}{3.207743in}}%
\pgfpathlineto{\pgfqpoint{3.754322in}{3.690043in}}%
\pgfpathlineto{\pgfqpoint{3.755189in}{3.607218in}}%
\pgfpathlineto{\pgfqpoint{3.756056in}{3.904086in}}%
\pgfpathlineto{\pgfqpoint{3.756923in}{3.117179in}}%
\pgfpathlineto{\pgfqpoint{3.757790in}{3.570479in}}%
\pgfpathlineto{\pgfqpoint{3.758657in}{3.435792in}}%
\pgfpathlineto{\pgfqpoint{3.759524in}{3.102436in}}%
\pgfpathlineto{\pgfqpoint{3.760392in}{3.343168in}}%
\pgfpathlineto{\pgfqpoint{3.761259in}{3.231023in}}%
\pgfpathlineto{\pgfqpoint{3.762126in}{3.691450in}}%
\pgfpathlineto{\pgfqpoint{3.762993in}{3.297777in}}%
\pgfpathlineto{\pgfqpoint{3.764727in}{3.514577in}}%
\pgfpathlineto{\pgfqpoint{3.765594in}{2.891671in}}%
\pgfpathlineto{\pgfqpoint{3.766462in}{3.978832in}}%
\pgfpathlineto{\pgfqpoint{3.767329in}{3.035797in}}%
\pgfpathlineto{\pgfqpoint{3.768196in}{3.553044in}}%
\pgfpathlineto{\pgfqpoint{3.769063in}{3.296366in}}%
\pgfpathlineto{\pgfqpoint{3.769930in}{3.680571in}}%
\pgfpathlineto{\pgfqpoint{3.770797in}{3.434074in}}%
\pgfpathlineto{\pgfqpoint{3.771664in}{3.509282in}}%
\pgfpathlineto{\pgfqpoint{3.772531in}{3.384517in}}%
\pgfpathlineto{\pgfqpoint{3.773399in}{3.436103in}}%
\pgfpathlineto{\pgfqpoint{3.774266in}{3.197136in}}%
\pgfpathlineto{\pgfqpoint{3.775133in}{3.637033in}}%
\pgfpathlineto{\pgfqpoint{3.776000in}{2.868797in}}%
\pgfpathlineto{\pgfqpoint{3.776867in}{3.535122in}}%
\pgfpathlineto{\pgfqpoint{3.778601in}{3.162056in}}%
\pgfpathlineto{\pgfqpoint{3.780336in}{3.562828in}}%
\pgfpathlineto{\pgfqpoint{3.781203in}{3.399735in}}%
\pgfpathlineto{\pgfqpoint{3.782070in}{2.926293in}}%
\pgfpathlineto{\pgfqpoint{3.782937in}{3.064950in}}%
\pgfpathlineto{\pgfqpoint{3.783804in}{3.398692in}}%
\pgfpathlineto{\pgfqpoint{3.784671in}{3.168871in}}%
\pgfpathlineto{\pgfqpoint{3.786406in}{3.554955in}}%
\pgfpathlineto{\pgfqpoint{3.787273in}{3.899024in}}%
\pgfpathlineto{\pgfqpoint{3.788140in}{3.528793in}}%
\pgfpathlineto{\pgfqpoint{3.789007in}{3.682486in}}%
\pgfpathlineto{\pgfqpoint{3.789874in}{3.523180in}}%
\pgfpathlineto{\pgfqpoint{3.790741in}{3.568712in}}%
\pgfpathlineto{\pgfqpoint{3.791608in}{3.406393in}}%
\pgfpathlineto{\pgfqpoint{3.792476in}{3.441521in}}%
\pgfpathlineto{\pgfqpoint{3.793343in}{3.328410in}}%
\pgfpathlineto{\pgfqpoint{3.794210in}{3.361918in}}%
\pgfpathlineto{\pgfqpoint{3.795077in}{3.442795in}}%
\pgfpathlineto{\pgfqpoint{3.796811in}{3.179006in}}%
\pgfpathlineto{\pgfqpoint{3.797678in}{3.635936in}}%
\pgfpathlineto{\pgfqpoint{3.798545in}{3.348328in}}%
\pgfpathlineto{\pgfqpoint{3.799413in}{3.587115in}}%
\pgfpathlineto{\pgfqpoint{3.800280in}{3.193209in}}%
\pgfpathlineto{\pgfqpoint{3.801147in}{3.426909in}}%
\pgfpathlineto{\pgfqpoint{3.802881in}{3.278859in}}%
\pgfpathlineto{\pgfqpoint{3.803748in}{3.758834in}}%
\pgfpathlineto{\pgfqpoint{3.804615in}{2.637200in}}%
\pgfpathlineto{\pgfqpoint{3.805483in}{3.818103in}}%
\pgfpathlineto{\pgfqpoint{3.807217in}{3.204766in}}%
\pgfpathlineto{\pgfqpoint{3.808084in}{3.690229in}}%
\pgfpathlineto{\pgfqpoint{3.808951in}{3.213679in}}%
\pgfpathlineto{\pgfqpoint{3.809818in}{3.484629in}}%
\pgfpathlineto{\pgfqpoint{3.810685in}{3.151448in}}%
\pgfpathlineto{\pgfqpoint{3.811552in}{3.575153in}}%
\pgfpathlineto{\pgfqpoint{3.812420in}{3.192243in}}%
\pgfpathlineto{\pgfqpoint{3.814154in}{3.657719in}}%
\pgfpathlineto{\pgfqpoint{3.815021in}{3.196567in}}%
\pgfpathlineto{\pgfqpoint{3.815888in}{3.710298in}}%
\pgfpathlineto{\pgfqpoint{3.816755in}{2.934742in}}%
\pgfpathlineto{\pgfqpoint{3.817622in}{4.030107in}}%
\pgfpathlineto{\pgfqpoint{3.818490in}{3.106026in}}%
\pgfpathlineto{\pgfqpoint{3.820224in}{3.466165in}}%
\pgfpathlineto{\pgfqpoint{3.821091in}{3.348094in}}%
\pgfpathlineto{\pgfqpoint{3.821958in}{4.003412in}}%
\pgfpathlineto{\pgfqpoint{3.822825in}{3.350387in}}%
\pgfpathlineto{\pgfqpoint{3.823692in}{3.392896in}}%
\pgfpathlineto{\pgfqpoint{3.825427in}{3.192095in}}%
\pgfpathlineto{\pgfqpoint{3.828028in}{3.566534in}}%
\pgfpathlineto{\pgfqpoint{3.828895in}{2.984492in}}%
\pgfpathlineto{\pgfqpoint{3.829762in}{3.880640in}}%
\pgfpathlineto{\pgfqpoint{3.830629in}{3.455546in}}%
\pgfpathlineto{\pgfqpoint{3.831497in}{3.822361in}}%
\pgfpathlineto{\pgfqpoint{3.833231in}{3.291258in}}%
\pgfpathlineto{\pgfqpoint{3.834098in}{3.591941in}}%
\pgfpathlineto{\pgfqpoint{3.834965in}{3.187865in}}%
\pgfpathlineto{\pgfqpoint{3.835832in}{3.965095in}}%
\pgfpathlineto{\pgfqpoint{3.836699in}{3.258970in}}%
\pgfpathlineto{\pgfqpoint{3.837566in}{3.519009in}}%
\pgfpathlineto{\pgfqpoint{3.838434in}{3.179526in}}%
\pgfpathlineto{\pgfqpoint{3.839301in}{3.888736in}}%
\pgfpathlineto{\pgfqpoint{3.841035in}{3.147134in}}%
\pgfpathlineto{\pgfqpoint{3.841902in}{3.331510in}}%
\pgfpathlineto{\pgfqpoint{3.842769in}{3.263161in}}%
\pgfpathlineto{\pgfqpoint{3.843636in}{3.593451in}}%
\pgfpathlineto{\pgfqpoint{3.844503in}{3.273638in}}%
\pgfpathlineto{\pgfqpoint{3.845371in}{3.738265in}}%
\pgfpathlineto{\pgfqpoint{3.846238in}{3.236800in}}%
\pgfpathlineto{\pgfqpoint{3.847105in}{3.262040in}}%
\pgfpathlineto{\pgfqpoint{3.848839in}{3.502532in}}%
\pgfpathlineto{\pgfqpoint{3.849706in}{3.410834in}}%
\pgfpathlineto{\pgfqpoint{3.850573in}{3.487483in}}%
\pgfpathlineto{\pgfqpoint{3.851441in}{3.432287in}}%
\pgfpathlineto{\pgfqpoint{3.852308in}{3.066106in}}%
\pgfpathlineto{\pgfqpoint{3.854042in}{3.838140in}}%
\pgfpathlineto{\pgfqpoint{3.855776in}{3.113980in}}%
\pgfpathlineto{\pgfqpoint{3.856643in}{3.436396in}}%
\pgfpathlineto{\pgfqpoint{3.857510in}{3.407086in}}%
\pgfpathlineto{\pgfqpoint{3.858378in}{3.071484in}}%
\pgfpathlineto{\pgfqpoint{3.859245in}{3.487941in}}%
\pgfpathlineto{\pgfqpoint{3.860112in}{3.216678in}}%
\pgfpathlineto{\pgfqpoint{3.861846in}{3.573010in}}%
\pgfpathlineto{\pgfqpoint{3.862713in}{3.446246in}}%
\pgfpathlineto{\pgfqpoint{3.863580in}{3.629264in}}%
\pgfpathlineto{\pgfqpoint{3.864448in}{3.238222in}}%
\pgfpathlineto{\pgfqpoint{3.865315in}{3.729659in}}%
\pgfpathlineto{\pgfqpoint{3.866182in}{3.220637in}}%
\pgfpathlineto{\pgfqpoint{3.867049in}{3.572199in}}%
\pgfpathlineto{\pgfqpoint{3.867916in}{3.521206in}}%
\pgfpathlineto{\pgfqpoint{3.868783in}{3.218842in}}%
\pgfpathlineto{\pgfqpoint{3.869650in}{3.632509in}}%
\pgfpathlineto{\pgfqpoint{3.870517in}{2.948982in}}%
\pgfpathlineto{\pgfqpoint{3.871385in}{3.362684in}}%
\pgfpathlineto{\pgfqpoint{3.872252in}{3.265205in}}%
\pgfpathlineto{\pgfqpoint{3.873119in}{3.497708in}}%
\pgfpathlineto{\pgfqpoint{3.873986in}{3.456185in}}%
\pgfpathlineto{\pgfqpoint{3.875720in}{3.803912in}}%
\pgfpathlineto{\pgfqpoint{3.876587in}{3.234152in}}%
\pgfpathlineto{\pgfqpoint{3.877455in}{3.360549in}}%
\pgfpathlineto{\pgfqpoint{3.879189in}{3.645124in}}%
\pgfpathlineto{\pgfqpoint{3.880056in}{3.558312in}}%
\pgfpathlineto{\pgfqpoint{3.880923in}{3.610518in}}%
\pgfpathlineto{\pgfqpoint{3.881790in}{3.526915in}}%
\pgfpathlineto{\pgfqpoint{3.882657in}{3.608942in}}%
\pgfpathlineto{\pgfqpoint{3.883524in}{3.175226in}}%
\pgfpathlineto{\pgfqpoint{3.884392in}{3.473352in}}%
\pgfpathlineto{\pgfqpoint{3.886126in}{3.293057in}}%
\pgfpathlineto{\pgfqpoint{3.886993in}{3.500313in}}%
\pgfpathlineto{\pgfqpoint{3.888727in}{3.262311in}}%
\pgfpathlineto{\pgfqpoint{3.889594in}{3.220189in}}%
\pgfpathlineto{\pgfqpoint{3.891329in}{3.582171in}}%
\pgfpathlineto{\pgfqpoint{3.892196in}{3.418298in}}%
\pgfpathlineto{\pgfqpoint{3.893063in}{3.661408in}}%
\pgfpathlineto{\pgfqpoint{3.895664in}{3.066233in}}%
\pgfpathlineto{\pgfqpoint{3.897399in}{3.446100in}}%
\pgfpathlineto{\pgfqpoint{3.899133in}{3.587441in}}%
\pgfpathlineto{\pgfqpoint{3.901734in}{3.116359in}}%
\pgfpathlineto{\pgfqpoint{3.902601in}{3.732766in}}%
\pgfpathlineto{\pgfqpoint{3.903469in}{3.175855in}}%
\pgfpathlineto{\pgfqpoint{3.905203in}{3.417653in}}%
\pgfpathlineto{\pgfqpoint{3.906070in}{3.238757in}}%
\pgfpathlineto{\pgfqpoint{3.906937in}{3.614705in}}%
\pgfpathlineto{\pgfqpoint{3.908671in}{3.133879in}}%
\pgfpathlineto{\pgfqpoint{3.910406in}{3.574407in}}%
\pgfpathlineto{\pgfqpoint{3.911273in}{3.308957in}}%
\pgfpathlineto{\pgfqpoint{3.913007in}{3.591399in}}%
\pgfpathlineto{\pgfqpoint{3.913874in}{3.474645in}}%
\pgfpathlineto{\pgfqpoint{3.914741in}{3.782823in}}%
\pgfpathlineto{\pgfqpoint{3.915608in}{3.487098in}}%
\pgfpathlineto{\pgfqpoint{3.916476in}{3.514348in}}%
\pgfpathlineto{\pgfqpoint{3.917343in}{3.320610in}}%
\pgfpathlineto{\pgfqpoint{3.918210in}{3.495621in}}%
\pgfpathlineto{\pgfqpoint{3.919077in}{3.412121in}}%
\pgfpathlineto{\pgfqpoint{3.919944in}{3.919776in}}%
\pgfpathlineto{\pgfqpoint{3.920811in}{3.261579in}}%
\pgfpathlineto{\pgfqpoint{3.921678in}{3.486970in}}%
\pgfpathlineto{\pgfqpoint{3.922545in}{3.168021in}}%
\pgfpathlineto{\pgfqpoint{3.925147in}{3.582901in}}%
\pgfpathlineto{\pgfqpoint{3.926014in}{3.423663in}}%
\pgfpathlineto{\pgfqpoint{3.926881in}{3.522771in}}%
\pgfpathlineto{\pgfqpoint{3.927748in}{3.320992in}}%
\pgfpathlineto{\pgfqpoint{3.929483in}{3.530703in}}%
\pgfpathlineto{\pgfqpoint{3.930350in}{3.212477in}}%
\pgfpathlineto{\pgfqpoint{3.931217in}{3.263257in}}%
\pgfpathlineto{\pgfqpoint{3.933818in}{3.582606in}}%
\pgfpathlineto{\pgfqpoint{3.934685in}{3.376894in}}%
\pgfpathlineto{\pgfqpoint{3.935552in}{3.462590in}}%
\pgfpathlineto{\pgfqpoint{3.937287in}{3.376123in}}%
\pgfpathlineto{\pgfqpoint{3.939021in}{3.500354in}}%
\pgfpathlineto{\pgfqpoint{3.939888in}{3.110438in}}%
\pgfpathlineto{\pgfqpoint{3.940755in}{3.378142in}}%
\pgfpathlineto{\pgfqpoint{3.941622in}{3.101146in}}%
\pgfpathlineto{\pgfqpoint{3.943357in}{3.498507in}}%
\pgfpathlineto{\pgfqpoint{3.944224in}{3.230879in}}%
\pgfpathlineto{\pgfqpoint{3.945091in}{3.243198in}}%
\pgfpathlineto{\pgfqpoint{3.945958in}{4.036625in}}%
\pgfpathlineto{\pgfqpoint{3.946825in}{3.656928in}}%
\pgfpathlineto{\pgfqpoint{3.947692in}{3.845630in}}%
\pgfpathlineto{\pgfqpoint{3.948559in}{3.565487in}}%
\pgfpathlineto{\pgfqpoint{3.949427in}{3.686687in}}%
\pgfpathlineto{\pgfqpoint{3.950294in}{3.673554in}}%
\pgfpathlineto{\pgfqpoint{3.951161in}{2.688384in}}%
\pgfpathlineto{\pgfqpoint{3.952895in}{3.147124in}}%
\pgfpathlineto{\pgfqpoint{3.953762in}{3.154046in}}%
\pgfpathlineto{\pgfqpoint{3.954629in}{3.552751in}}%
\pgfpathlineto{\pgfqpoint{3.955497in}{2.972388in}}%
\pgfpathlineto{\pgfqpoint{3.957231in}{3.407238in}}%
\pgfpathlineto{\pgfqpoint{3.958098in}{3.409061in}}%
\pgfpathlineto{\pgfqpoint{3.958965in}{3.255733in}}%
\pgfpathlineto{\pgfqpoint{3.959832in}{3.709278in}}%
\pgfpathlineto{\pgfqpoint{3.960699in}{3.343843in}}%
\pgfpathlineto{\pgfqpoint{3.961566in}{3.648968in}}%
\pgfpathlineto{\pgfqpoint{3.963301in}{3.272833in}}%
\pgfpathlineto{\pgfqpoint{3.965035in}{3.482894in}}%
\pgfpathlineto{\pgfqpoint{3.967636in}{2.850030in}}%
\pgfpathlineto{\pgfqpoint{3.968503in}{3.405447in}}%
\pgfpathlineto{\pgfqpoint{3.969371in}{3.142703in}}%
\pgfpathlineto{\pgfqpoint{3.970238in}{3.332816in}}%
\pgfpathlineto{\pgfqpoint{3.971105in}{3.263054in}}%
\pgfpathlineto{\pgfqpoint{3.971972in}{3.546639in}}%
\pgfpathlineto{\pgfqpoint{3.972839in}{3.311966in}}%
\pgfpathlineto{\pgfqpoint{3.973706in}{3.869545in}}%
\pgfpathlineto{\pgfqpoint{3.974573in}{3.470686in}}%
\pgfpathlineto{\pgfqpoint{3.975441in}{3.645525in}}%
\pgfpathlineto{\pgfqpoint{3.976308in}{3.573311in}}%
\pgfpathlineto{\pgfqpoint{3.977175in}{3.716810in}}%
\pgfpathlineto{\pgfqpoint{3.979776in}{2.993953in}}%
\pgfpathlineto{\pgfqpoint{3.982378in}{3.455632in}}%
\pgfpathlineto{\pgfqpoint{3.983245in}{3.211591in}}%
\pgfpathlineto{\pgfqpoint{3.984112in}{3.436341in}}%
\pgfpathlineto{\pgfqpoint{3.984979in}{3.254970in}}%
\pgfpathlineto{\pgfqpoint{3.985846in}{3.866838in}}%
\pgfpathlineto{\pgfqpoint{3.986713in}{3.079308in}}%
\pgfpathlineto{\pgfqpoint{3.988448in}{3.520466in}}%
\pgfpathlineto{\pgfqpoint{3.989315in}{3.488034in}}%
\pgfpathlineto{\pgfqpoint{3.991049in}{3.637805in}}%
\pgfpathlineto{\pgfqpoint{3.991916in}{3.176582in}}%
\pgfpathlineto{\pgfqpoint{3.992783in}{3.305748in}}%
\pgfpathlineto{\pgfqpoint{3.995385in}{2.861822in}}%
\pgfpathlineto{\pgfqpoint{3.996252in}{3.727636in}}%
\pgfpathlineto{\pgfqpoint{3.997119in}{3.272881in}}%
\pgfpathlineto{\pgfqpoint{3.997986in}{3.316027in}}%
\pgfpathlineto{\pgfqpoint{3.998853in}{3.219361in}}%
\pgfpathlineto{\pgfqpoint{3.999720in}{3.742392in}}%
\pgfpathlineto{\pgfqpoint{4.000587in}{3.450901in}}%
\pgfpathlineto{\pgfqpoint{4.001455in}{3.579893in}}%
\pgfpathlineto{\pgfqpoint{4.002322in}{3.037204in}}%
\pgfpathlineto{\pgfqpoint{4.004056in}{3.291116in}}%
\pgfpathlineto{\pgfqpoint{4.004923in}{3.119445in}}%
\pgfpathlineto{\pgfqpoint{4.007524in}{3.655722in}}%
\pgfpathlineto{\pgfqpoint{4.008392in}{3.674728in}}%
\pgfpathlineto{\pgfqpoint{4.009259in}{3.826390in}}%
\pgfpathlineto{\pgfqpoint{4.010126in}{3.321509in}}%
\pgfpathlineto{\pgfqpoint{4.010993in}{3.577651in}}%
\pgfpathlineto{\pgfqpoint{4.011860in}{3.562615in}}%
\pgfpathlineto{\pgfqpoint{4.012727in}{2.938773in}}%
\pgfpathlineto{\pgfqpoint{4.013594in}{3.095469in}}%
\pgfpathlineto{\pgfqpoint{4.014462in}{3.543224in}}%
\pgfpathlineto{\pgfqpoint{4.015329in}{3.260112in}}%
\pgfpathlineto{\pgfqpoint{4.016196in}{3.477707in}}%
\pgfpathlineto{\pgfqpoint{4.017063in}{3.268439in}}%
\pgfpathlineto{\pgfqpoint{4.017930in}{3.762380in}}%
\pgfpathlineto{\pgfqpoint{4.018797in}{3.625840in}}%
\pgfpathlineto{\pgfqpoint{4.019664in}{3.317847in}}%
\pgfpathlineto{\pgfqpoint{4.021399in}{3.753748in}}%
\pgfpathlineto{\pgfqpoint{4.024867in}{3.002035in}}%
\pgfpathlineto{\pgfqpoint{4.027469in}{3.228113in}}%
\pgfpathlineto{\pgfqpoint{4.028336in}{3.240143in}}%
\pgfpathlineto{\pgfqpoint{4.029203in}{3.342931in}}%
\pgfpathlineto{\pgfqpoint{4.030070in}{3.619322in}}%
\pgfpathlineto{\pgfqpoint{4.030937in}{3.298001in}}%
\pgfpathlineto{\pgfqpoint{4.031804in}{3.714779in}}%
\pgfpathlineto{\pgfqpoint{4.034406in}{3.070113in}}%
\pgfpathlineto{\pgfqpoint{4.035273in}{3.132245in}}%
\pgfpathlineto{\pgfqpoint{4.036140in}{3.307543in}}%
\pgfpathlineto{\pgfqpoint{4.037007in}{3.245581in}}%
\pgfpathlineto{\pgfqpoint{4.039608in}{3.704718in}}%
\pgfpathlineto{\pgfqpoint{4.040476in}{3.103173in}}%
\pgfpathlineto{\pgfqpoint{4.042210in}{3.560662in}}%
\pgfpathlineto{\pgfqpoint{4.043077in}{3.160492in}}%
\pgfpathlineto{\pgfqpoint{4.043944in}{3.326349in}}%
\pgfpathlineto{\pgfqpoint{4.044811in}{3.017051in}}%
\pgfpathlineto{\pgfqpoint{4.048280in}{3.612434in}}%
\pgfpathlineto{\pgfqpoint{4.049147in}{3.587178in}}%
\pgfpathlineto{\pgfqpoint{4.050014in}{3.692109in}}%
\pgfpathlineto{\pgfqpoint{4.052615in}{3.004099in}}%
\pgfpathlineto{\pgfqpoint{4.056084in}{3.240223in}}%
\pgfpathlineto{\pgfqpoint{4.056951in}{3.773706in}}%
\pgfpathlineto{\pgfqpoint{4.057818in}{3.753055in}}%
\pgfpathlineto{\pgfqpoint{4.058685in}{3.825451in}}%
\pgfpathlineto{\pgfqpoint{4.061287in}{3.359315in}}%
\pgfpathlineto{\pgfqpoint{4.062154in}{3.341561in}}%
\pgfpathlineto{\pgfqpoint{4.063888in}{3.182446in}}%
\pgfpathlineto{\pgfqpoint{4.064755in}{3.209035in}}%
\pgfpathlineto{\pgfqpoint{4.066490in}{3.386941in}}%
\pgfpathlineto{\pgfqpoint{4.067357in}{3.255315in}}%
\pgfpathlineto{\pgfqpoint{4.068224in}{3.580220in}}%
\pgfpathlineto{\pgfqpoint{4.069091in}{3.529608in}}%
\pgfpathlineto{\pgfqpoint{4.069958in}{3.280895in}}%
\pgfpathlineto{\pgfqpoint{4.071692in}{3.442912in}}%
\pgfpathlineto{\pgfqpoint{4.073427in}{2.952069in}}%
\pgfpathlineto{\pgfqpoint{4.076895in}{3.667509in}}%
\pgfpathlineto{\pgfqpoint{4.079497in}{3.151880in}}%
\pgfpathlineto{\pgfqpoint{4.080364in}{3.269117in}}%
\pgfpathlineto{\pgfqpoint{4.081231in}{3.186445in}}%
\pgfpathlineto{\pgfqpoint{4.082098in}{3.514766in}}%
\pgfpathlineto{\pgfqpoint{4.085566in}{2.976972in}}%
\pgfpathlineto{\pgfqpoint{4.087301in}{3.648145in}}%
\pgfpathlineto{\pgfqpoint{4.089035in}{3.077191in}}%
\pgfpathlineto{\pgfqpoint{4.089902in}{3.407933in}}%
\pgfpathlineto{\pgfqpoint{4.090769in}{3.383950in}}%
\pgfpathlineto{\pgfqpoint{4.091636in}{3.592800in}}%
\pgfpathlineto{\pgfqpoint{4.092503in}{2.985522in}}%
\pgfpathlineto{\pgfqpoint{4.093371in}{3.597676in}}%
\pgfpathlineto{\pgfqpoint{4.094238in}{3.532121in}}%
\pgfpathlineto{\pgfqpoint{4.095105in}{3.646933in}}%
\pgfpathlineto{\pgfqpoint{4.096839in}{3.014463in}}%
\pgfpathlineto{\pgfqpoint{4.097706in}{2.950548in}}%
\pgfpathlineto{\pgfqpoint{4.099441in}{3.633390in}}%
\pgfpathlineto{\pgfqpoint{4.100308in}{3.299318in}}%
\pgfpathlineto{\pgfqpoint{4.101175in}{3.542167in}}%
\pgfpathlineto{\pgfqpoint{4.102042in}{3.357665in}}%
\pgfpathlineto{\pgfqpoint{4.102909in}{3.521447in}}%
\pgfpathlineto{\pgfqpoint{4.105510in}{3.007055in}}%
\pgfpathlineto{\pgfqpoint{4.107245in}{3.473852in}}%
\pgfpathlineto{\pgfqpoint{4.108112in}{3.559891in}}%
\pgfpathlineto{\pgfqpoint{4.108979in}{3.483646in}}%
\pgfpathlineto{\pgfqpoint{4.109846in}{3.760625in}}%
\pgfpathlineto{\pgfqpoint{4.110713in}{3.285794in}}%
\pgfpathlineto{\pgfqpoint{4.111580in}{3.291551in}}%
\pgfpathlineto{\pgfqpoint{4.112448in}{3.248183in}}%
\pgfpathlineto{\pgfqpoint{4.113315in}{3.269769in}}%
\pgfpathlineto{\pgfqpoint{4.114182in}{3.174483in}}%
\pgfpathlineto{\pgfqpoint{4.115916in}{3.656360in}}%
\pgfpathlineto{\pgfqpoint{4.116783in}{3.409908in}}%
\pgfpathlineto{\pgfqpoint{4.118517in}{3.821176in}}%
\pgfpathlineto{\pgfqpoint{4.120252in}{3.216248in}}%
\pgfpathlineto{\pgfqpoint{4.121986in}{3.361802in}}%
\pgfpathlineto{\pgfqpoint{4.122853in}{3.844739in}}%
\pgfpathlineto{\pgfqpoint{4.123720in}{3.402478in}}%
\pgfpathlineto{\pgfqpoint{4.124587in}{3.731460in}}%
\pgfpathlineto{\pgfqpoint{4.125455in}{3.423345in}}%
\pgfpathlineto{\pgfqpoint{4.126322in}{3.569001in}}%
\pgfpathlineto{\pgfqpoint{4.128056in}{2.970950in}}%
\pgfpathlineto{\pgfqpoint{4.129790in}{3.510803in}}%
\pgfpathlineto{\pgfqpoint{4.130657in}{3.552563in}}%
\pgfpathlineto{\pgfqpoint{4.131524in}{3.505135in}}%
\pgfpathlineto{\pgfqpoint{4.132392in}{3.540657in}}%
\pgfpathlineto{\pgfqpoint{4.133259in}{3.070375in}}%
\pgfpathlineto{\pgfqpoint{4.134126in}{3.201423in}}%
\pgfpathlineto{\pgfqpoint{4.134993in}{3.566971in}}%
\pgfpathlineto{\pgfqpoint{4.136727in}{3.189070in}}%
\pgfpathlineto{\pgfqpoint{4.138462in}{3.671725in}}%
\pgfpathlineto{\pgfqpoint{4.139329in}{3.659143in}}%
\pgfpathlineto{\pgfqpoint{4.140196in}{3.670232in}}%
\pgfpathlineto{\pgfqpoint{4.141063in}{3.085356in}}%
\pgfpathlineto{\pgfqpoint{4.141930in}{3.176326in}}%
\pgfpathlineto{\pgfqpoint{4.144531in}{3.782884in}}%
\pgfpathlineto{\pgfqpoint{4.145399in}{3.607031in}}%
\pgfpathlineto{\pgfqpoint{4.146266in}{3.131602in}}%
\pgfpathlineto{\pgfqpoint{4.147133in}{3.350737in}}%
\pgfpathlineto{\pgfqpoint{4.148867in}{3.020378in}}%
\pgfpathlineto{\pgfqpoint{4.150601in}{3.643751in}}%
\pgfpathlineto{\pgfqpoint{4.151469in}{3.195697in}}%
\pgfpathlineto{\pgfqpoint{4.152336in}{3.288154in}}%
\pgfpathlineto{\pgfqpoint{4.153203in}{3.238438in}}%
\pgfpathlineto{\pgfqpoint{4.154070in}{3.301236in}}%
\pgfpathlineto{\pgfqpoint{4.155804in}{2.933514in}}%
\pgfpathlineto{\pgfqpoint{4.157538in}{3.554607in}}%
\pgfpathlineto{\pgfqpoint{4.158406in}{3.616272in}}%
\pgfpathlineto{\pgfqpoint{4.160140in}{3.494336in}}%
\pgfpathlineto{\pgfqpoint{4.161007in}{3.128054in}}%
\pgfpathlineto{\pgfqpoint{4.161874in}{3.456558in}}%
\pgfpathlineto{\pgfqpoint{4.162741in}{3.337713in}}%
\pgfpathlineto{\pgfqpoint{4.163608in}{3.589986in}}%
\pgfpathlineto{\pgfqpoint{4.164476in}{3.196605in}}%
\pgfpathlineto{\pgfqpoint{4.165343in}{3.748892in}}%
\pgfpathlineto{\pgfqpoint{4.166210in}{3.643475in}}%
\pgfpathlineto{\pgfqpoint{4.167077in}{3.298656in}}%
\pgfpathlineto{\pgfqpoint{4.167944in}{3.377762in}}%
\pgfpathlineto{\pgfqpoint{4.168811in}{3.381912in}}%
\pgfpathlineto{\pgfqpoint{4.169678in}{3.313858in}}%
\pgfpathlineto{\pgfqpoint{4.170545in}{3.465793in}}%
\pgfpathlineto{\pgfqpoint{4.171413in}{3.208079in}}%
\pgfpathlineto{\pgfqpoint{4.172280in}{3.654583in}}%
\pgfpathlineto{\pgfqpoint{4.174014in}{3.004741in}}%
\pgfpathlineto{\pgfqpoint{4.174881in}{3.088942in}}%
\pgfpathlineto{\pgfqpoint{4.175748in}{3.540627in}}%
\pgfpathlineto{\pgfqpoint{4.178350in}{3.016470in}}%
\pgfpathlineto{\pgfqpoint{4.179217in}{3.321708in}}%
\pgfpathlineto{\pgfqpoint{4.180084in}{3.208307in}}%
\pgfpathlineto{\pgfqpoint{4.180951in}{3.376335in}}%
\pgfpathlineto{\pgfqpoint{4.181818in}{3.211431in}}%
\pgfpathlineto{\pgfqpoint{4.184420in}{3.997854in}}%
\pgfpathlineto{\pgfqpoint{4.187021in}{2.983226in}}%
\pgfpathlineto{\pgfqpoint{4.187888in}{3.399186in}}%
\pgfpathlineto{\pgfqpoint{4.188755in}{3.310767in}}%
\pgfpathlineto{\pgfqpoint{4.189622in}{3.368361in}}%
\pgfpathlineto{\pgfqpoint{4.190490in}{3.581870in}}%
\pgfpathlineto{\pgfqpoint{4.191357in}{3.221459in}}%
\pgfpathlineto{\pgfqpoint{4.192224in}{3.341233in}}%
\pgfpathlineto{\pgfqpoint{4.193958in}{3.162006in}}%
\pgfpathlineto{\pgfqpoint{4.194825in}{3.228338in}}%
\pgfpathlineto{\pgfqpoint{4.195692in}{3.699263in}}%
\pgfpathlineto{\pgfqpoint{4.196559in}{3.672920in}}%
\pgfpathlineto{\pgfqpoint{4.198294in}{3.254853in}}%
\pgfpathlineto{\pgfqpoint{4.199161in}{3.081273in}}%
\pgfpathlineto{\pgfqpoint{4.200028in}{3.174115in}}%
\pgfpathlineto{\pgfqpoint{4.200895in}{3.564895in}}%
\pgfpathlineto{\pgfqpoint{4.201762in}{3.311067in}}%
\pgfpathlineto{\pgfqpoint{4.202629in}{3.357304in}}%
\pgfpathlineto{\pgfqpoint{4.203497in}{3.548701in}}%
\pgfpathlineto{\pgfqpoint{4.204364in}{2.895758in}}%
\pgfpathlineto{\pgfqpoint{4.206098in}{3.492239in}}%
\pgfpathlineto{\pgfqpoint{4.206965in}{3.436143in}}%
\pgfpathlineto{\pgfqpoint{4.208699in}{3.640388in}}%
\pgfpathlineto{\pgfqpoint{4.209566in}{3.383574in}}%
\pgfpathlineto{\pgfqpoint{4.210434in}{3.455105in}}%
\pgfpathlineto{\pgfqpoint{4.211301in}{3.508970in}}%
\pgfpathlineto{\pgfqpoint{4.212168in}{3.212178in}}%
\pgfpathlineto{\pgfqpoint{4.213035in}{3.457778in}}%
\pgfpathlineto{\pgfqpoint{4.213902in}{3.353270in}}%
\pgfpathlineto{\pgfqpoint{4.214769in}{3.386586in}}%
\pgfpathlineto{\pgfqpoint{4.215636in}{3.662842in}}%
\pgfpathlineto{\pgfqpoint{4.217371in}{2.856137in}}%
\pgfpathlineto{\pgfqpoint{4.218238in}{3.720668in}}%
\pgfpathlineto{\pgfqpoint{4.219972in}{3.401521in}}%
\pgfpathlineto{\pgfqpoint{4.220839in}{3.313444in}}%
\pgfpathlineto{\pgfqpoint{4.221706in}{3.340700in}}%
\pgfpathlineto{\pgfqpoint{4.223441in}{3.091968in}}%
\pgfpathlineto{\pgfqpoint{4.224308in}{3.029116in}}%
\pgfpathlineto{\pgfqpoint{4.225175in}{3.293850in}}%
\pgfpathlineto{\pgfqpoint{4.226042in}{3.123259in}}%
\pgfpathlineto{\pgfqpoint{4.227776in}{3.542608in}}%
\pgfpathlineto{\pgfqpoint{4.228643in}{3.183064in}}%
\pgfpathlineto{\pgfqpoint{4.229510in}{3.279421in}}%
\pgfpathlineto{\pgfqpoint{4.231245in}{3.688025in}}%
\pgfpathlineto{\pgfqpoint{4.232979in}{3.580984in}}%
\pgfpathlineto{\pgfqpoint{4.233846in}{3.245305in}}%
\pgfpathlineto{\pgfqpoint{4.235580in}{3.456284in}}%
\pgfpathlineto{\pgfqpoint{4.237315in}{3.266002in}}%
\pgfpathlineto{\pgfqpoint{4.238182in}{3.520610in}}%
\pgfpathlineto{\pgfqpoint{4.239049in}{3.221263in}}%
\pgfpathlineto{\pgfqpoint{4.239916in}{3.269511in}}%
\pgfpathlineto{\pgfqpoint{4.240783in}{3.485490in}}%
\pgfpathlineto{\pgfqpoint{4.241650in}{3.288295in}}%
\pgfpathlineto{\pgfqpoint{4.242517in}{3.329679in}}%
\pgfpathlineto{\pgfqpoint{4.243385in}{3.526030in}}%
\pgfpathlineto{\pgfqpoint{4.244252in}{3.186111in}}%
\pgfpathlineto{\pgfqpoint{4.245119in}{3.391992in}}%
\pgfpathlineto{\pgfqpoint{4.245986in}{2.986556in}}%
\pgfpathlineto{\pgfqpoint{4.247720in}{3.463108in}}%
\pgfpathlineto{\pgfqpoint{4.248587in}{3.537835in}}%
\pgfpathlineto{\pgfqpoint{4.249455in}{3.484158in}}%
\pgfpathlineto{\pgfqpoint{4.250322in}{3.503100in}}%
\pgfpathlineto{\pgfqpoint{4.252056in}{3.176689in}}%
\pgfpathlineto{\pgfqpoint{4.252923in}{3.729551in}}%
\pgfpathlineto{\pgfqpoint{4.253790in}{3.724857in}}%
\pgfpathlineto{\pgfqpoint{4.256392in}{3.136798in}}%
\pgfpathlineto{\pgfqpoint{4.258993in}{3.744633in}}%
\pgfpathlineto{\pgfqpoint{4.261594in}{3.014505in}}%
\pgfpathlineto{\pgfqpoint{4.262462in}{3.756946in}}%
\pgfpathlineto{\pgfqpoint{4.263329in}{3.358201in}}%
\pgfpathlineto{\pgfqpoint{4.264196in}{3.412228in}}%
\pgfpathlineto{\pgfqpoint{4.265063in}{3.474775in}}%
\pgfpathlineto{\pgfqpoint{4.265930in}{3.387943in}}%
\pgfpathlineto{\pgfqpoint{4.266797in}{3.736680in}}%
\pgfpathlineto{\pgfqpoint{4.267664in}{3.280656in}}%
\pgfpathlineto{\pgfqpoint{4.268531in}{3.750615in}}%
\pgfpathlineto{\pgfqpoint{4.269399in}{3.345466in}}%
\pgfpathlineto{\pgfqpoint{4.270266in}{3.519388in}}%
\pgfpathlineto{\pgfqpoint{4.271133in}{3.229474in}}%
\pgfpathlineto{\pgfqpoint{4.273734in}{3.816794in}}%
\pgfpathlineto{\pgfqpoint{4.275469in}{3.247571in}}%
\pgfpathlineto{\pgfqpoint{4.278070in}{3.399501in}}%
\pgfpathlineto{\pgfqpoint{4.278937in}{3.539614in}}%
\pgfpathlineto{\pgfqpoint{4.280671in}{3.497995in}}%
\pgfpathlineto{\pgfqpoint{4.281538in}{3.004308in}}%
\pgfpathlineto{\pgfqpoint{4.284140in}{3.692477in}}%
\pgfpathlineto{\pgfqpoint{4.285007in}{3.692242in}}%
\pgfpathlineto{\pgfqpoint{4.285874in}{3.640897in}}%
\pgfpathlineto{\pgfqpoint{4.287608in}{3.126588in}}%
\pgfpathlineto{\pgfqpoint{4.289343in}{3.430441in}}%
\pgfpathlineto{\pgfqpoint{4.291077in}{3.234474in}}%
\pgfpathlineto{\pgfqpoint{4.291944in}{3.346587in}}%
\pgfpathlineto{\pgfqpoint{4.292811in}{3.215929in}}%
\pgfpathlineto{\pgfqpoint{4.294545in}{3.553266in}}%
\pgfpathlineto{\pgfqpoint{4.295413in}{3.204580in}}%
\pgfpathlineto{\pgfqpoint{4.296280in}{3.231262in}}%
\pgfpathlineto{\pgfqpoint{4.298881in}{3.604904in}}%
\pgfpathlineto{\pgfqpoint{4.299748in}{3.680920in}}%
\pgfpathlineto{\pgfqpoint{4.300615in}{3.520611in}}%
\pgfpathlineto{\pgfqpoint{4.301483in}{3.659947in}}%
\pgfpathlineto{\pgfqpoint{4.302350in}{3.417217in}}%
\pgfpathlineto{\pgfqpoint{4.303217in}{3.837291in}}%
\pgfpathlineto{\pgfqpoint{4.304084in}{3.763939in}}%
\pgfpathlineto{\pgfqpoint{4.305818in}{3.008947in}}%
\pgfpathlineto{\pgfqpoint{4.307552in}{3.519159in}}%
\pgfpathlineto{\pgfqpoint{4.308420in}{3.240390in}}%
\pgfpathlineto{\pgfqpoint{4.309287in}{3.769191in}}%
\pgfpathlineto{\pgfqpoint{4.311021in}{3.127359in}}%
\pgfpathlineto{\pgfqpoint{4.312755in}{3.356909in}}%
\pgfpathlineto{\pgfqpoint{4.313622in}{3.134740in}}%
\pgfpathlineto{\pgfqpoint{4.314490in}{3.483640in}}%
\pgfpathlineto{\pgfqpoint{4.316224in}{3.139115in}}%
\pgfpathlineto{\pgfqpoint{4.317091in}{3.663158in}}%
\pgfpathlineto{\pgfqpoint{4.318825in}{3.249832in}}%
\pgfpathlineto{\pgfqpoint{4.319692in}{3.305216in}}%
\pgfpathlineto{\pgfqpoint{4.320559in}{3.292490in}}%
\pgfpathlineto{\pgfqpoint{4.321427in}{3.597651in}}%
\pgfpathlineto{\pgfqpoint{4.322294in}{3.317288in}}%
\pgfpathlineto{\pgfqpoint{4.323161in}{3.441211in}}%
\pgfpathlineto{\pgfqpoint{4.324028in}{3.177174in}}%
\pgfpathlineto{\pgfqpoint{4.324895in}{3.213411in}}%
\pgfpathlineto{\pgfqpoint{4.326629in}{3.465028in}}%
\pgfpathlineto{\pgfqpoint{4.327497in}{3.880823in}}%
\pgfpathlineto{\pgfqpoint{4.328364in}{3.457412in}}%
\pgfpathlineto{\pgfqpoint{4.329231in}{3.732513in}}%
\pgfpathlineto{\pgfqpoint{4.330965in}{3.451695in}}%
\pgfpathlineto{\pgfqpoint{4.332699in}{3.493716in}}%
\pgfpathlineto{\pgfqpoint{4.334434in}{3.190743in}}%
\pgfpathlineto{\pgfqpoint{4.335301in}{3.416112in}}%
\pgfpathlineto{\pgfqpoint{4.336168in}{3.320790in}}%
\pgfpathlineto{\pgfqpoint{4.337902in}{3.745020in}}%
\pgfpathlineto{\pgfqpoint{4.339636in}{3.168415in}}%
\pgfpathlineto{\pgfqpoint{4.342238in}{3.728912in}}%
\pgfpathlineto{\pgfqpoint{4.343972in}{3.232961in}}%
\pgfpathlineto{\pgfqpoint{4.344839in}{3.580374in}}%
\pgfpathlineto{\pgfqpoint{4.345706in}{3.057310in}}%
\pgfpathlineto{\pgfqpoint{4.346573in}{3.554472in}}%
\pgfpathlineto{\pgfqpoint{4.347441in}{3.391901in}}%
\pgfpathlineto{\pgfqpoint{4.348308in}{3.616606in}}%
\pgfpathlineto{\pgfqpoint{4.349175in}{3.570339in}}%
\pgfpathlineto{\pgfqpoint{4.350042in}{3.515909in}}%
\pgfpathlineto{\pgfqpoint{4.350909in}{2.949463in}}%
\pgfpathlineto{\pgfqpoint{4.352643in}{3.320445in}}%
\pgfpathlineto{\pgfqpoint{4.353510in}{3.289587in}}%
\pgfpathlineto{\pgfqpoint{4.354378in}{3.395790in}}%
\pgfpathlineto{\pgfqpoint{4.355245in}{3.332788in}}%
\pgfpathlineto{\pgfqpoint{4.356112in}{3.404596in}}%
\pgfpathlineto{\pgfqpoint{4.356979in}{3.396784in}}%
\pgfpathlineto{\pgfqpoint{4.358713in}{3.239327in}}%
\pgfpathlineto{\pgfqpoint{4.359580in}{3.699204in}}%
\pgfpathlineto{\pgfqpoint{4.360448in}{3.331369in}}%
\pgfpathlineto{\pgfqpoint{4.361315in}{3.498386in}}%
\pgfpathlineto{\pgfqpoint{4.362182in}{3.356723in}}%
\pgfpathlineto{\pgfqpoint{4.363049in}{3.541568in}}%
\pgfpathlineto{\pgfqpoint{4.363916in}{3.357427in}}%
\pgfpathlineto{\pgfqpoint{4.364783in}{3.464362in}}%
\pgfpathlineto{\pgfqpoint{4.365650in}{3.126464in}}%
\pgfpathlineto{\pgfqpoint{4.367385in}{3.747492in}}%
\pgfpathlineto{\pgfqpoint{4.369986in}{3.230591in}}%
\pgfpathlineto{\pgfqpoint{4.370853in}{3.786290in}}%
\pgfpathlineto{\pgfqpoint{4.373455in}{3.095000in}}%
\pgfpathlineto{\pgfqpoint{4.374322in}{3.048531in}}%
\pgfpathlineto{\pgfqpoint{4.375189in}{3.626358in}}%
\pgfpathlineto{\pgfqpoint{4.376056in}{3.229838in}}%
\pgfpathlineto{\pgfqpoint{4.376923in}{3.500584in}}%
\pgfpathlineto{\pgfqpoint{4.377790in}{3.167372in}}%
\pgfpathlineto{\pgfqpoint{4.379524in}{3.769297in}}%
\pgfpathlineto{\pgfqpoint{4.380392in}{3.284843in}}%
\pgfpathlineto{\pgfqpoint{4.381259in}{3.641757in}}%
\pgfpathlineto{\pgfqpoint{4.382993in}{3.030369in}}%
\pgfpathlineto{\pgfqpoint{4.384727in}{3.784733in}}%
\pgfpathlineto{\pgfqpoint{4.385594in}{3.334312in}}%
\pgfpathlineto{\pgfqpoint{4.387329in}{3.619312in}}%
\pgfpathlineto{\pgfqpoint{4.388196in}{3.449980in}}%
\pgfpathlineto{\pgfqpoint{4.389930in}{3.587466in}}%
\pgfpathlineto{\pgfqpoint{4.391664in}{3.285808in}}%
\pgfpathlineto{\pgfqpoint{4.394266in}{3.789943in}}%
\pgfpathlineto{\pgfqpoint{4.395133in}{3.339405in}}%
\pgfpathlineto{\pgfqpoint{4.396000in}{3.380781in}}%
\pgfpathlineto{\pgfqpoint{4.396867in}{3.386172in}}%
\pgfpathlineto{\pgfqpoint{4.398601in}{3.763745in}}%
\pgfpathlineto{\pgfqpoint{4.399469in}{3.294161in}}%
\pgfpathlineto{\pgfqpoint{4.400336in}{3.483024in}}%
\pgfpathlineto{\pgfqpoint{4.401203in}{3.480971in}}%
\pgfpathlineto{\pgfqpoint{4.402070in}{3.226331in}}%
\pgfpathlineto{\pgfqpoint{4.402937in}{3.598229in}}%
\pgfpathlineto{\pgfqpoint{4.403804in}{3.350551in}}%
\pgfpathlineto{\pgfqpoint{4.405538in}{3.567803in}}%
\pgfpathlineto{\pgfqpoint{4.406406in}{3.523484in}}%
\pgfpathlineto{\pgfqpoint{4.407273in}{3.427946in}}%
\pgfpathlineto{\pgfqpoint{4.408140in}{3.164699in}}%
\pgfpathlineto{\pgfqpoint{4.409007in}{3.497728in}}%
\pgfpathlineto{\pgfqpoint{4.409874in}{3.363082in}}%
\pgfpathlineto{\pgfqpoint{4.410741in}{3.475126in}}%
\pgfpathlineto{\pgfqpoint{4.411608in}{3.168487in}}%
\pgfpathlineto{\pgfqpoint{4.412476in}{3.627832in}}%
\pgfpathlineto{\pgfqpoint{4.413343in}{3.522829in}}%
\pgfpathlineto{\pgfqpoint{4.414210in}{3.474365in}}%
\pgfpathlineto{\pgfqpoint{4.415077in}{3.060825in}}%
\pgfpathlineto{\pgfqpoint{4.416811in}{3.578011in}}%
\pgfpathlineto{\pgfqpoint{4.417678in}{3.553727in}}%
\pgfpathlineto{\pgfqpoint{4.418545in}{3.066434in}}%
\pgfpathlineto{\pgfqpoint{4.419413in}{3.186127in}}%
\pgfpathlineto{\pgfqpoint{4.420280in}{3.918424in}}%
\pgfpathlineto{\pgfqpoint{4.422881in}{3.021041in}}%
\pgfpathlineto{\pgfqpoint{4.424615in}{3.652260in}}%
\pgfpathlineto{\pgfqpoint{4.426350in}{3.269738in}}%
\pgfpathlineto{\pgfqpoint{4.427217in}{3.207146in}}%
\pgfpathlineto{\pgfqpoint{4.428084in}{3.640055in}}%
\pgfpathlineto{\pgfqpoint{4.428951in}{3.579182in}}%
\pgfpathlineto{\pgfqpoint{4.429818in}{3.593198in}}%
\pgfpathlineto{\pgfqpoint{4.431552in}{3.259200in}}%
\pgfpathlineto{\pgfqpoint{4.432420in}{3.263339in}}%
\pgfpathlineto{\pgfqpoint{4.433287in}{3.650542in}}%
\pgfpathlineto{\pgfqpoint{4.434154in}{3.102486in}}%
\pgfpathlineto{\pgfqpoint{4.436755in}{3.596919in}}%
\pgfpathlineto{\pgfqpoint{4.438490in}{3.131754in}}%
\pgfpathlineto{\pgfqpoint{4.440224in}{3.664897in}}%
\pgfpathlineto{\pgfqpoint{4.441958in}{3.179760in}}%
\pgfpathlineto{\pgfqpoint{4.444559in}{3.435415in}}%
\pgfpathlineto{\pgfqpoint{4.445427in}{3.442095in}}%
\pgfpathlineto{\pgfqpoint{4.446294in}{3.486250in}}%
\pgfpathlineto{\pgfqpoint{4.447161in}{3.379947in}}%
\pgfpathlineto{\pgfqpoint{4.448895in}{3.743616in}}%
\pgfpathlineto{\pgfqpoint{4.449762in}{3.086808in}}%
\pgfpathlineto{\pgfqpoint{4.451497in}{3.694806in}}%
\pgfpathlineto{\pgfqpoint{4.452364in}{3.225223in}}%
\pgfpathlineto{\pgfqpoint{4.453231in}{3.338524in}}%
\pgfpathlineto{\pgfqpoint{4.454098in}{3.708052in}}%
\pgfpathlineto{\pgfqpoint{4.454965in}{3.270070in}}%
\pgfpathlineto{\pgfqpoint{4.456699in}{3.604717in}}%
\pgfpathlineto{\pgfqpoint{4.457566in}{3.217710in}}%
\pgfpathlineto{\pgfqpoint{4.458434in}{3.441441in}}%
\pgfpathlineto{\pgfqpoint{4.461035in}{2.986706in}}%
\pgfpathlineto{\pgfqpoint{4.461902in}{3.123468in}}%
\pgfpathlineto{\pgfqpoint{4.462769in}{3.474099in}}%
\pgfpathlineto{\pgfqpoint{4.463636in}{3.177406in}}%
\pgfpathlineto{\pgfqpoint{4.464503in}{3.515422in}}%
\pgfpathlineto{\pgfqpoint{4.465371in}{3.360893in}}%
\pgfpathlineto{\pgfqpoint{4.466238in}{3.779025in}}%
\pgfpathlineto{\pgfqpoint{4.467972in}{3.266160in}}%
\pgfpathlineto{\pgfqpoint{4.468839in}{3.646897in}}%
\pgfpathlineto{\pgfqpoint{4.471441in}{3.188552in}}%
\pgfpathlineto{\pgfqpoint{4.474042in}{3.691350in}}%
\pgfpathlineto{\pgfqpoint{4.476643in}{3.117187in}}%
\pgfpathlineto{\pgfqpoint{4.477510in}{3.609590in}}%
\pgfpathlineto{\pgfqpoint{4.478378in}{3.193360in}}%
\pgfpathlineto{\pgfqpoint{4.480979in}{3.650431in}}%
\pgfpathlineto{\pgfqpoint{4.481846in}{3.470763in}}%
\pgfpathlineto{\pgfqpoint{4.482713in}{3.477275in}}%
\pgfpathlineto{\pgfqpoint{4.484448in}{3.437963in}}%
\pgfpathlineto{\pgfqpoint{4.485315in}{3.527884in}}%
\pgfpathlineto{\pgfqpoint{4.486182in}{3.272323in}}%
\pgfpathlineto{\pgfqpoint{4.487049in}{3.516586in}}%
\pgfpathlineto{\pgfqpoint{4.487916in}{3.103977in}}%
\pgfpathlineto{\pgfqpoint{4.488783in}{3.918568in}}%
\pgfpathlineto{\pgfqpoint{4.489650in}{2.938929in}}%
\pgfpathlineto{\pgfqpoint{4.491385in}{3.372187in}}%
\pgfpathlineto{\pgfqpoint{4.492252in}{3.401876in}}%
\pgfpathlineto{\pgfqpoint{4.493119in}{3.110923in}}%
\pgfpathlineto{\pgfqpoint{4.493986in}{3.570705in}}%
\pgfpathlineto{\pgfqpoint{4.494853in}{3.568000in}}%
\pgfpathlineto{\pgfqpoint{4.495720in}{3.852147in}}%
\pgfpathlineto{\pgfqpoint{4.496587in}{3.176237in}}%
\pgfpathlineto{\pgfqpoint{4.497455in}{3.380537in}}%
\pgfpathlineto{\pgfqpoint{4.498322in}{3.100272in}}%
\pgfpathlineto{\pgfqpoint{4.500056in}{3.579768in}}%
\pgfpathlineto{\pgfqpoint{4.500923in}{3.462794in}}%
\pgfpathlineto{\pgfqpoint{4.501790in}{3.178997in}}%
\pgfpathlineto{\pgfqpoint{4.502657in}{3.760363in}}%
\pgfpathlineto{\pgfqpoint{4.504392in}{2.914590in}}%
\pgfpathlineto{\pgfqpoint{4.506126in}{3.753992in}}%
\pgfpathlineto{\pgfqpoint{4.506993in}{3.427819in}}%
\pgfpathlineto{\pgfqpoint{4.507860in}{3.517617in}}%
\pgfpathlineto{\pgfqpoint{4.508727in}{3.276506in}}%
\pgfpathlineto{\pgfqpoint{4.510462in}{3.624785in}}%
\pgfpathlineto{\pgfqpoint{4.511329in}{3.174607in}}%
\pgfpathlineto{\pgfqpoint{4.512196in}{3.333185in}}%
\pgfpathlineto{\pgfqpoint{4.513930in}{3.180000in}}%
\pgfpathlineto{\pgfqpoint{4.514797in}{3.203115in}}%
\pgfpathlineto{\pgfqpoint{4.516531in}{3.087239in}}%
\pgfpathlineto{\pgfqpoint{4.518266in}{3.736009in}}%
\pgfpathlineto{\pgfqpoint{4.519133in}{3.437500in}}%
\pgfpathlineto{\pgfqpoint{4.520000in}{3.510884in}}%
\pgfpathlineto{\pgfqpoint{4.520867in}{3.300904in}}%
\pgfpathlineto{\pgfqpoint{4.521734in}{3.319525in}}%
\pgfpathlineto{\pgfqpoint{4.522601in}{3.382984in}}%
\pgfpathlineto{\pgfqpoint{4.523469in}{3.684612in}}%
\pgfpathlineto{\pgfqpoint{4.525203in}{3.465702in}}%
\pgfpathlineto{\pgfqpoint{4.526070in}{3.363898in}}%
\pgfpathlineto{\pgfqpoint{4.526937in}{3.596349in}}%
\pgfpathlineto{\pgfqpoint{4.528671in}{3.258784in}}%
\pgfpathlineto{\pgfqpoint{4.529538in}{3.719205in}}%
\pgfpathlineto{\pgfqpoint{4.532140in}{3.160345in}}%
\pgfpathlineto{\pgfqpoint{4.533007in}{3.513446in}}%
\pgfpathlineto{\pgfqpoint{4.533874in}{3.460742in}}%
\pgfpathlineto{\pgfqpoint{4.535608in}{3.186049in}}%
\pgfpathlineto{\pgfqpoint{4.536476in}{3.660465in}}%
\pgfpathlineto{\pgfqpoint{4.538210in}{2.898985in}}%
\pgfpathlineto{\pgfqpoint{4.539944in}{3.688128in}}%
\pgfpathlineto{\pgfqpoint{4.542545in}{2.980077in}}%
\pgfpathlineto{\pgfqpoint{4.543413in}{3.664891in}}%
\pgfpathlineto{\pgfqpoint{4.544280in}{3.148120in}}%
\pgfpathlineto{\pgfqpoint{4.545147in}{3.321750in}}%
\pgfpathlineto{\pgfqpoint{4.546014in}{3.196639in}}%
\pgfpathlineto{\pgfqpoint{4.546881in}{3.483918in}}%
\pgfpathlineto{\pgfqpoint{4.547748in}{3.221106in}}%
\pgfpathlineto{\pgfqpoint{4.548615in}{3.641107in}}%
\pgfpathlineto{\pgfqpoint{4.549483in}{3.365733in}}%
\pgfpathlineto{\pgfqpoint{4.550350in}{3.379997in}}%
\pgfpathlineto{\pgfqpoint{4.551217in}{3.339498in}}%
\pgfpathlineto{\pgfqpoint{4.552084in}{3.513501in}}%
\pgfpathlineto{\pgfqpoint{4.553818in}{3.205867in}}%
\pgfpathlineto{\pgfqpoint{4.554685in}{3.118951in}}%
\pgfpathlineto{\pgfqpoint{4.555552in}{3.233944in}}%
\pgfpathlineto{\pgfqpoint{4.556420in}{3.624753in}}%
\pgfpathlineto{\pgfqpoint{4.558154in}{3.175215in}}%
\pgfpathlineto{\pgfqpoint{4.559021in}{3.287062in}}%
\pgfpathlineto{\pgfqpoint{4.559888in}{3.482791in}}%
\pgfpathlineto{\pgfqpoint{4.561622in}{3.130672in}}%
\pgfpathlineto{\pgfqpoint{4.563357in}{3.491099in}}%
\pgfpathlineto{\pgfqpoint{4.564224in}{3.270849in}}%
\pgfpathlineto{\pgfqpoint{4.565091in}{3.273250in}}%
\pgfpathlineto{\pgfqpoint{4.565958in}{3.576188in}}%
\pgfpathlineto{\pgfqpoint{4.566825in}{3.473109in}}%
\pgfpathlineto{\pgfqpoint{4.567692in}{3.527340in}}%
\pgfpathlineto{\pgfqpoint{4.568559in}{3.488730in}}%
\pgfpathlineto{\pgfqpoint{4.570294in}{3.872518in}}%
\pgfpathlineto{\pgfqpoint{4.571161in}{3.191931in}}%
\pgfpathlineto{\pgfqpoint{4.572028in}{3.661079in}}%
\pgfpathlineto{\pgfqpoint{4.572895in}{3.633679in}}%
\pgfpathlineto{\pgfqpoint{4.573762in}{3.503536in}}%
\pgfpathlineto{\pgfqpoint{4.574629in}{3.084639in}}%
\pgfpathlineto{\pgfqpoint{4.576364in}{3.571863in}}%
\pgfpathlineto{\pgfqpoint{4.577231in}{3.234256in}}%
\pgfpathlineto{\pgfqpoint{4.578965in}{3.728555in}}%
\pgfpathlineto{\pgfqpoint{4.580699in}{3.275356in}}%
\pgfpathlineto{\pgfqpoint{4.581566in}{3.490995in}}%
\pgfpathlineto{\pgfqpoint{4.582434in}{3.397289in}}%
\pgfpathlineto{\pgfqpoint{4.584168in}{2.927214in}}%
\pgfpathlineto{\pgfqpoint{4.585035in}{3.556206in}}%
\pgfpathlineto{\pgfqpoint{4.585902in}{3.475566in}}%
\pgfpathlineto{\pgfqpoint{4.587636in}{2.992710in}}%
\pgfpathlineto{\pgfqpoint{4.588503in}{3.444007in}}%
\pgfpathlineto{\pgfqpoint{4.589371in}{3.170479in}}%
\pgfpathlineto{\pgfqpoint{4.590238in}{3.199877in}}%
\pgfpathlineto{\pgfqpoint{4.591105in}{3.191211in}}%
\pgfpathlineto{\pgfqpoint{4.591972in}{3.386753in}}%
\pgfpathlineto{\pgfqpoint{4.592839in}{3.098377in}}%
\pgfpathlineto{\pgfqpoint{4.593706in}{3.443016in}}%
\pgfpathlineto{\pgfqpoint{4.594573in}{3.302459in}}%
\pgfpathlineto{\pgfqpoint{4.595441in}{3.521155in}}%
\pgfpathlineto{\pgfqpoint{4.597175in}{3.312109in}}%
\pgfpathlineto{\pgfqpoint{4.598909in}{3.526731in}}%
\pgfpathlineto{\pgfqpoint{4.599776in}{3.323530in}}%
\pgfpathlineto{\pgfqpoint{4.600643in}{3.640513in}}%
\pgfpathlineto{\pgfqpoint{4.602378in}{3.464669in}}%
\pgfpathlineto{\pgfqpoint{4.604979in}{3.198795in}}%
\pgfpathlineto{\pgfqpoint{4.605846in}{3.700397in}}%
\pgfpathlineto{\pgfqpoint{4.606713in}{3.590271in}}%
\pgfpathlineto{\pgfqpoint{4.608448in}{3.084865in}}%
\pgfpathlineto{\pgfqpoint{4.610182in}{3.758047in}}%
\pgfpathlineto{\pgfqpoint{4.611049in}{2.953615in}}%
\pgfpathlineto{\pgfqpoint{4.611916in}{3.085510in}}%
\pgfpathlineto{\pgfqpoint{4.612783in}{3.593034in}}%
\pgfpathlineto{\pgfqpoint{4.614517in}{3.123886in}}%
\pgfpathlineto{\pgfqpoint{4.615385in}{3.208860in}}%
\pgfpathlineto{\pgfqpoint{4.616252in}{3.633025in}}%
\pgfpathlineto{\pgfqpoint{4.617119in}{3.275157in}}%
\pgfpathlineto{\pgfqpoint{4.618853in}{3.583082in}}%
\pgfpathlineto{\pgfqpoint{4.620587in}{3.141597in}}%
\pgfpathlineto{\pgfqpoint{4.622322in}{3.559511in}}%
\pgfpathlineto{\pgfqpoint{4.624056in}{3.134696in}}%
\pgfpathlineto{\pgfqpoint{4.624923in}{3.203189in}}%
\pgfpathlineto{\pgfqpoint{4.625790in}{3.376393in}}%
\pgfpathlineto{\pgfqpoint{4.626657in}{2.811545in}}%
\pgfpathlineto{\pgfqpoint{4.627524in}{3.215222in}}%
\pgfpathlineto{\pgfqpoint{4.628392in}{2.993431in}}%
\pgfpathlineto{\pgfqpoint{4.629259in}{3.419700in}}%
\pgfpathlineto{\pgfqpoint{4.630126in}{3.218692in}}%
\pgfpathlineto{\pgfqpoint{4.631860in}{3.722333in}}%
\pgfpathlineto{\pgfqpoint{4.632727in}{3.034514in}}%
\pgfpathlineto{\pgfqpoint{4.634462in}{3.330745in}}%
\pgfpathlineto{\pgfqpoint{4.636196in}{2.945173in}}%
\pgfpathlineto{\pgfqpoint{4.637063in}{3.570501in}}%
\pgfpathlineto{\pgfqpoint{4.637930in}{3.328804in}}%
\pgfpathlineto{\pgfqpoint{4.640531in}{3.616013in}}%
\pgfpathlineto{\pgfqpoint{4.641399in}{3.127176in}}%
\pgfpathlineto{\pgfqpoint{4.643133in}{3.621274in}}%
\pgfpathlineto{\pgfqpoint{4.644000in}{2.956631in}}%
\pgfpathlineto{\pgfqpoint{4.644867in}{3.422806in}}%
\pgfpathlineto{\pgfqpoint{4.645734in}{3.239146in}}%
\pgfpathlineto{\pgfqpoint{4.647469in}{3.499387in}}%
\pgfpathlineto{\pgfqpoint{4.648336in}{3.381406in}}%
\pgfpathlineto{\pgfqpoint{4.649203in}{3.449317in}}%
\pgfpathlineto{\pgfqpoint{4.650070in}{3.372063in}}%
\pgfpathlineto{\pgfqpoint{4.650937in}{3.932434in}}%
\pgfpathlineto{\pgfqpoint{4.652671in}{3.120693in}}%
\pgfpathlineto{\pgfqpoint{4.655273in}{3.346240in}}%
\pgfpathlineto{\pgfqpoint{4.656140in}{3.153184in}}%
\pgfpathlineto{\pgfqpoint{4.657007in}{3.591906in}}%
\pgfpathlineto{\pgfqpoint{4.658741in}{3.193842in}}%
\pgfpathlineto{\pgfqpoint{4.660476in}{3.485183in}}%
\pgfpathlineto{\pgfqpoint{4.661343in}{3.355634in}}%
\pgfpathlineto{\pgfqpoint{4.662210in}{2.959093in}}%
\pgfpathlineto{\pgfqpoint{4.663077in}{3.560243in}}%
\pgfpathlineto{\pgfqpoint{4.663944in}{3.454876in}}%
\pgfpathlineto{\pgfqpoint{4.664811in}{3.563797in}}%
\pgfpathlineto{\pgfqpoint{4.665678in}{3.497875in}}%
\pgfpathlineto{\pgfqpoint{4.667413in}{3.155301in}}%
\pgfpathlineto{\pgfqpoint{4.668280in}{3.621805in}}%
\pgfpathlineto{\pgfqpoint{4.670881in}{2.924341in}}%
\pgfpathlineto{\pgfqpoint{4.671748in}{3.570978in}}%
\pgfpathlineto{\pgfqpoint{4.672615in}{3.458829in}}%
\pgfpathlineto{\pgfqpoint{4.673483in}{3.369561in}}%
\pgfpathlineto{\pgfqpoint{4.674350in}{3.601182in}}%
\pgfpathlineto{\pgfqpoint{4.676084in}{3.209325in}}%
\pgfpathlineto{\pgfqpoint{4.676951in}{3.294588in}}%
\pgfpathlineto{\pgfqpoint{4.678685in}{3.651747in}}%
\pgfpathlineto{\pgfqpoint{4.679552in}{3.568130in}}%
\pgfpathlineto{\pgfqpoint{4.680420in}{3.696629in}}%
\pgfpathlineto{\pgfqpoint{4.681287in}{2.996302in}}%
\pgfpathlineto{\pgfqpoint{4.683888in}{3.611403in}}%
\pgfpathlineto{\pgfqpoint{4.684755in}{3.008864in}}%
\pgfpathlineto{\pgfqpoint{4.686490in}{3.627152in}}%
\pgfpathlineto{\pgfqpoint{4.687357in}{3.286986in}}%
\pgfpathlineto{\pgfqpoint{4.689091in}{3.446842in}}%
\pgfpathlineto{\pgfqpoint{4.689958in}{3.014416in}}%
\pgfpathlineto{\pgfqpoint{4.690825in}{3.029203in}}%
\pgfpathlineto{\pgfqpoint{4.691692in}{3.573402in}}%
\pgfpathlineto{\pgfqpoint{4.693427in}{3.209599in}}%
\pgfpathlineto{\pgfqpoint{4.694294in}{3.390398in}}%
\pgfpathlineto{\pgfqpoint{4.695161in}{3.134312in}}%
\pgfpathlineto{\pgfqpoint{4.696028in}{3.542272in}}%
\pgfpathlineto{\pgfqpoint{4.696895in}{3.331104in}}%
\pgfpathlineto{\pgfqpoint{4.697762in}{3.584564in}}%
\pgfpathlineto{\pgfqpoint{4.698629in}{3.329110in}}%
\pgfpathlineto{\pgfqpoint{4.699497in}{3.660092in}}%
\pgfpathlineto{\pgfqpoint{4.700364in}{3.177204in}}%
\pgfpathlineto{\pgfqpoint{4.701231in}{3.299674in}}%
\pgfpathlineto{\pgfqpoint{4.702098in}{3.655635in}}%
\pgfpathlineto{\pgfqpoint{4.703832in}{3.155321in}}%
\pgfpathlineto{\pgfqpoint{4.705566in}{3.502057in}}%
\pgfpathlineto{\pgfqpoint{4.707301in}{3.535463in}}%
\pgfpathlineto{\pgfqpoint{4.708168in}{3.243556in}}%
\pgfpathlineto{\pgfqpoint{4.709035in}{3.529792in}}%
\pgfpathlineto{\pgfqpoint{4.710769in}{3.324761in}}%
\pgfpathlineto{\pgfqpoint{4.711636in}{3.395040in}}%
\pgfpathlineto{\pgfqpoint{4.712503in}{3.208502in}}%
\pgfpathlineto{\pgfqpoint{4.713371in}{3.595861in}}%
\pgfpathlineto{\pgfqpoint{4.714238in}{3.025161in}}%
\pgfpathlineto{\pgfqpoint{4.715972in}{3.433793in}}%
\pgfpathlineto{\pgfqpoint{4.716839in}{3.180557in}}%
\pgfpathlineto{\pgfqpoint{4.718573in}{3.575826in}}%
\pgfpathlineto{\pgfqpoint{4.720308in}{3.230911in}}%
\pgfpathlineto{\pgfqpoint{4.721175in}{3.844268in}}%
\pgfpathlineto{\pgfqpoint{4.722042in}{3.210930in}}%
\pgfpathlineto{\pgfqpoint{4.722909in}{3.261543in}}%
\pgfpathlineto{\pgfqpoint{4.723776in}{3.470629in}}%
\pgfpathlineto{\pgfqpoint{4.724643in}{3.383464in}}%
\pgfpathlineto{\pgfqpoint{4.725510in}{2.988830in}}%
\pgfpathlineto{\pgfqpoint{4.726378in}{2.996618in}}%
\pgfpathlineto{\pgfqpoint{4.728979in}{3.629670in}}%
\pgfpathlineto{\pgfqpoint{4.729846in}{3.402986in}}%
\pgfpathlineto{\pgfqpoint{4.730713in}{3.444880in}}%
\pgfpathlineto{\pgfqpoint{4.731580in}{3.478115in}}%
\pgfpathlineto{\pgfqpoint{4.733315in}{3.218714in}}%
\pgfpathlineto{\pgfqpoint{4.734182in}{3.283939in}}%
\pgfpathlineto{\pgfqpoint{4.735049in}{3.706040in}}%
\pgfpathlineto{\pgfqpoint{4.735916in}{3.611262in}}%
\pgfpathlineto{\pgfqpoint{4.736783in}{3.225558in}}%
\pgfpathlineto{\pgfqpoint{4.737650in}{3.496224in}}%
\pgfpathlineto{\pgfqpoint{4.738517in}{3.279585in}}%
\pgfpathlineto{\pgfqpoint{4.739385in}{3.451077in}}%
\pgfpathlineto{\pgfqpoint{4.740252in}{3.898479in}}%
\pgfpathlineto{\pgfqpoint{4.741986in}{3.665770in}}%
\pgfpathlineto{\pgfqpoint{4.742853in}{3.626316in}}%
\pgfpathlineto{\pgfqpoint{4.743720in}{3.384951in}}%
\pgfpathlineto{\pgfqpoint{4.744587in}{3.385554in}}%
\pgfpathlineto{\pgfqpoint{4.745455in}{3.660059in}}%
\pgfpathlineto{\pgfqpoint{4.746322in}{3.399810in}}%
\pgfpathlineto{\pgfqpoint{4.747189in}{3.420106in}}%
\pgfpathlineto{\pgfqpoint{4.748923in}{3.055132in}}%
\pgfpathlineto{\pgfqpoint{4.750657in}{3.449616in}}%
\pgfpathlineto{\pgfqpoint{4.751524in}{3.406946in}}%
\pgfpathlineto{\pgfqpoint{4.752392in}{3.106395in}}%
\pgfpathlineto{\pgfqpoint{4.753259in}{3.507553in}}%
\pgfpathlineto{\pgfqpoint{4.754126in}{3.096238in}}%
\pgfpathlineto{\pgfqpoint{4.755860in}{3.690144in}}%
\pgfpathlineto{\pgfqpoint{4.756727in}{3.200978in}}%
\pgfpathlineto{\pgfqpoint{4.757594in}{3.594802in}}%
\pgfpathlineto{\pgfqpoint{4.760196in}{3.137699in}}%
\pgfpathlineto{\pgfqpoint{4.761063in}{3.795242in}}%
\pgfpathlineto{\pgfqpoint{4.762797in}{3.204193in}}%
\pgfpathlineto{\pgfqpoint{4.763664in}{3.636840in}}%
\pgfpathlineto{\pgfqpoint{4.764531in}{3.329647in}}%
\pgfpathlineto{\pgfqpoint{4.765399in}{3.476192in}}%
\pgfpathlineto{\pgfqpoint{4.766266in}{3.421449in}}%
\pgfpathlineto{\pgfqpoint{4.767133in}{3.651787in}}%
\pgfpathlineto{\pgfqpoint{4.768000in}{3.516161in}}%
\pgfpathlineto{\pgfqpoint{4.768867in}{3.639491in}}%
\pgfpathlineto{\pgfqpoint{4.769734in}{3.307935in}}%
\pgfpathlineto{\pgfqpoint{4.770601in}{3.388548in}}%
\pgfpathlineto{\pgfqpoint{4.771469in}{3.747823in}}%
\pgfpathlineto{\pgfqpoint{4.773203in}{3.370656in}}%
\pgfpathlineto{\pgfqpoint{4.774070in}{3.659583in}}%
\pgfpathlineto{\pgfqpoint{4.774937in}{3.242561in}}%
\pgfpathlineto{\pgfqpoint{4.775804in}{3.526483in}}%
\pgfpathlineto{\pgfqpoint{4.777538in}{3.303857in}}%
\pgfpathlineto{\pgfqpoint{4.778406in}{3.710660in}}%
\pgfpathlineto{\pgfqpoint{4.779273in}{3.296485in}}%
\pgfpathlineto{\pgfqpoint{4.780140in}{3.389191in}}%
\pgfpathlineto{\pgfqpoint{4.781007in}{3.643521in}}%
\pgfpathlineto{\pgfqpoint{4.782741in}{3.209519in}}%
\pgfpathlineto{\pgfqpoint{4.783608in}{3.311431in}}%
\pgfpathlineto{\pgfqpoint{4.784476in}{3.636988in}}%
\pgfpathlineto{\pgfqpoint{4.785343in}{3.210003in}}%
\pgfpathlineto{\pgfqpoint{4.787077in}{3.486500in}}%
\pgfpathlineto{\pgfqpoint{4.787944in}{3.233898in}}%
\pgfpathlineto{\pgfqpoint{4.788811in}{3.458227in}}%
\pgfpathlineto{\pgfqpoint{4.789678in}{3.361126in}}%
\pgfpathlineto{\pgfqpoint{4.790545in}{3.501776in}}%
\pgfpathlineto{\pgfqpoint{4.792280in}{3.857600in}}%
\pgfpathlineto{\pgfqpoint{4.794014in}{3.336477in}}%
\pgfpathlineto{\pgfqpoint{4.794881in}{3.209738in}}%
\pgfpathlineto{\pgfqpoint{4.795748in}{3.350471in}}%
\pgfpathlineto{\pgfqpoint{4.796615in}{3.754631in}}%
\pgfpathlineto{\pgfqpoint{4.799217in}{3.219665in}}%
\pgfpathlineto{\pgfqpoint{4.801818in}{3.735458in}}%
\pgfpathlineto{\pgfqpoint{4.802685in}{3.709358in}}%
\pgfpathlineto{\pgfqpoint{4.803552in}{3.314479in}}%
\pgfpathlineto{\pgfqpoint{4.804420in}{3.435195in}}%
\pgfpathlineto{\pgfqpoint{4.805287in}{2.836078in}}%
\pgfpathlineto{\pgfqpoint{4.806154in}{3.634734in}}%
\pgfpathlineto{\pgfqpoint{4.807021in}{3.424766in}}%
\pgfpathlineto{\pgfqpoint{4.807888in}{3.176496in}}%
\pgfpathlineto{\pgfqpoint{4.808755in}{3.864664in}}%
\pgfpathlineto{\pgfqpoint{4.810490in}{3.297690in}}%
\pgfpathlineto{\pgfqpoint{4.812224in}{3.451001in}}%
\pgfpathlineto{\pgfqpoint{4.813091in}{2.936090in}}%
\pgfpathlineto{\pgfqpoint{4.813958in}{3.470079in}}%
\pgfpathlineto{\pgfqpoint{4.814825in}{3.064582in}}%
\pgfpathlineto{\pgfqpoint{4.816559in}{3.626036in}}%
\pgfpathlineto{\pgfqpoint{4.817427in}{3.152004in}}%
\pgfpathlineto{\pgfqpoint{4.818294in}{3.236557in}}%
\pgfpathlineto{\pgfqpoint{4.819161in}{3.594234in}}%
\pgfpathlineto{\pgfqpoint{4.820028in}{3.242560in}}%
\pgfpathlineto{\pgfqpoint{4.820895in}{3.251278in}}%
\pgfpathlineto{\pgfqpoint{4.823497in}{3.480115in}}%
\pgfpathlineto{\pgfqpoint{4.824364in}{3.305770in}}%
\pgfpathlineto{\pgfqpoint{4.825231in}{3.357408in}}%
\pgfpathlineto{\pgfqpoint{4.826098in}{3.353665in}}%
\pgfpathlineto{\pgfqpoint{4.826965in}{3.204404in}}%
\pgfpathlineto{\pgfqpoint{4.828699in}{3.823259in}}%
\pgfpathlineto{\pgfqpoint{4.830434in}{3.198530in}}%
\pgfpathlineto{\pgfqpoint{4.831301in}{3.601866in}}%
\pgfpathlineto{\pgfqpoint{4.832168in}{3.428454in}}%
\pgfpathlineto{\pgfqpoint{4.833035in}{3.668676in}}%
\pgfpathlineto{\pgfqpoint{4.833902in}{3.045732in}}%
\pgfpathlineto{\pgfqpoint{4.834769in}{3.071020in}}%
\pgfpathlineto{\pgfqpoint{4.835636in}{3.402235in}}%
\pgfpathlineto{\pgfqpoint{4.836503in}{3.252046in}}%
\pgfpathlineto{\pgfqpoint{4.838238in}{3.566203in}}%
\pgfpathlineto{\pgfqpoint{4.839105in}{2.993772in}}%
\pgfpathlineto{\pgfqpoint{4.840839in}{3.662557in}}%
\pgfpathlineto{\pgfqpoint{4.841706in}{3.251311in}}%
\pgfpathlineto{\pgfqpoint{4.842573in}{3.534069in}}%
\pgfpathlineto{\pgfqpoint{4.845175in}{3.256179in}}%
\pgfpathlineto{\pgfqpoint{4.846042in}{3.467849in}}%
\pgfpathlineto{\pgfqpoint{4.846909in}{3.253506in}}%
\pgfpathlineto{\pgfqpoint{4.848643in}{3.555279in}}%
\pgfpathlineto{\pgfqpoint{4.849510in}{3.486878in}}%
\pgfpathlineto{\pgfqpoint{4.851245in}{3.209985in}}%
\pgfpathlineto{\pgfqpoint{4.852112in}{3.487282in}}%
\pgfpathlineto{\pgfqpoint{4.852979in}{3.092682in}}%
\pgfpathlineto{\pgfqpoint{4.854713in}{3.477299in}}%
\pgfpathlineto{\pgfqpoint{4.855580in}{2.842136in}}%
\pgfpathlineto{\pgfqpoint{4.857315in}{3.611509in}}%
\pgfpathlineto{\pgfqpoint{4.858182in}{3.206866in}}%
\pgfpathlineto{\pgfqpoint{4.859049in}{3.703013in}}%
\pgfpathlineto{\pgfqpoint{4.860783in}{3.252342in}}%
\pgfpathlineto{\pgfqpoint{4.861650in}{3.520694in}}%
\pgfpathlineto{\pgfqpoint{4.862517in}{3.262906in}}%
\pgfpathlineto{\pgfqpoint{4.863385in}{3.317440in}}%
\pgfpathlineto{\pgfqpoint{4.864252in}{3.335949in}}%
\pgfpathlineto{\pgfqpoint{4.865986in}{3.010097in}}%
\pgfpathlineto{\pgfqpoint{4.866853in}{3.401478in}}%
\pgfpathlineto{\pgfqpoint{4.867720in}{2.910526in}}%
\pgfpathlineto{\pgfqpoint{4.868587in}{3.598207in}}%
\pgfpathlineto{\pgfqpoint{4.869455in}{3.205937in}}%
\pgfpathlineto{\pgfqpoint{4.870322in}{3.416528in}}%
\pgfpathlineto{\pgfqpoint{4.871189in}{3.290009in}}%
\pgfpathlineto{\pgfqpoint{4.873790in}{3.483450in}}%
\pgfpathlineto{\pgfqpoint{4.874657in}{3.146661in}}%
\pgfpathlineto{\pgfqpoint{4.877259in}{3.506780in}}%
\pgfpathlineto{\pgfqpoint{4.878126in}{3.528611in}}%
\pgfpathlineto{\pgfqpoint{4.878993in}{3.201515in}}%
\pgfpathlineto{\pgfqpoint{4.879860in}{3.462242in}}%
\pgfpathlineto{\pgfqpoint{4.880727in}{3.178002in}}%
\pgfpathlineto{\pgfqpoint{4.881594in}{3.197816in}}%
\pgfpathlineto{\pgfqpoint{4.883329in}{3.500413in}}%
\pgfpathlineto{\pgfqpoint{4.884196in}{3.096617in}}%
\pgfpathlineto{\pgfqpoint{4.885063in}{3.466408in}}%
\pgfpathlineto{\pgfqpoint{4.885930in}{3.016013in}}%
\pgfpathlineto{\pgfqpoint{4.886797in}{3.073686in}}%
\pgfpathlineto{\pgfqpoint{4.887664in}{3.359656in}}%
\pgfpathlineto{\pgfqpoint{4.888531in}{3.155922in}}%
\pgfpathlineto{\pgfqpoint{4.889399in}{3.578167in}}%
\pgfpathlineto{\pgfqpoint{4.890266in}{3.384940in}}%
\pgfpathlineto{\pgfqpoint{4.891133in}{3.462508in}}%
\pgfpathlineto{\pgfqpoint{4.892867in}{3.180437in}}%
\pgfpathlineto{\pgfqpoint{4.893734in}{3.503410in}}%
\pgfpathlineto{\pgfqpoint{4.894601in}{3.360500in}}%
\pgfpathlineto{\pgfqpoint{4.896336in}{3.584333in}}%
\pgfpathlineto{\pgfqpoint{4.898070in}{3.024383in}}%
\pgfpathlineto{\pgfqpoint{4.901538in}{3.605963in}}%
\pgfpathlineto{\pgfqpoint{4.903273in}{3.286671in}}%
\pgfpathlineto{\pgfqpoint{4.904140in}{3.479186in}}%
\pgfpathlineto{\pgfqpoint{4.905007in}{3.428910in}}%
\pgfpathlineto{\pgfqpoint{4.905874in}{3.135867in}}%
\pgfpathlineto{\pgfqpoint{4.908476in}{3.468364in}}%
\pgfpathlineto{\pgfqpoint{4.909343in}{3.510091in}}%
\pgfpathlineto{\pgfqpoint{4.910210in}{3.606440in}}%
\pgfpathlineto{\pgfqpoint{4.911077in}{2.969209in}}%
\pgfpathlineto{\pgfqpoint{4.911944in}{3.560955in}}%
\pgfpathlineto{\pgfqpoint{4.912811in}{3.363135in}}%
\pgfpathlineto{\pgfqpoint{4.914545in}{3.674140in}}%
\pgfpathlineto{\pgfqpoint{4.915413in}{2.952766in}}%
\pgfpathlineto{\pgfqpoint{4.916280in}{3.446449in}}%
\pgfpathlineto{\pgfqpoint{4.917147in}{3.345124in}}%
\pgfpathlineto{\pgfqpoint{4.918014in}{3.166323in}}%
\pgfpathlineto{\pgfqpoint{4.920615in}{3.624797in}}%
\pgfpathlineto{\pgfqpoint{4.921483in}{3.501076in}}%
\pgfpathlineto{\pgfqpoint{4.922350in}{3.114308in}}%
\pgfpathlineto{\pgfqpoint{4.924084in}{3.414983in}}%
\pgfpathlineto{\pgfqpoint{4.925818in}{3.325097in}}%
\pgfpathlineto{\pgfqpoint{4.926685in}{3.458085in}}%
\pgfpathlineto{\pgfqpoint{4.927552in}{3.353667in}}%
\pgfpathlineto{\pgfqpoint{4.929287in}{3.705992in}}%
\pgfpathlineto{\pgfqpoint{4.931021in}{3.410077in}}%
\pgfpathlineto{\pgfqpoint{4.931888in}{3.622117in}}%
\pgfpathlineto{\pgfqpoint{4.933622in}{3.216591in}}%
\pgfpathlineto{\pgfqpoint{4.934490in}{3.658928in}}%
\pgfpathlineto{\pgfqpoint{4.935357in}{3.349616in}}%
\pgfpathlineto{\pgfqpoint{4.936224in}{3.981207in}}%
\pgfpathlineto{\pgfqpoint{4.937958in}{2.934436in}}%
\pgfpathlineto{\pgfqpoint{4.938825in}{3.022584in}}%
\pgfpathlineto{\pgfqpoint{4.940559in}{3.453726in}}%
\pgfpathlineto{\pgfqpoint{4.942294in}{3.204276in}}%
\pgfpathlineto{\pgfqpoint{4.943161in}{3.695382in}}%
\pgfpathlineto{\pgfqpoint{4.944028in}{3.073160in}}%
\pgfpathlineto{\pgfqpoint{4.944895in}{3.659281in}}%
\pgfpathlineto{\pgfqpoint{4.945762in}{3.507460in}}%
\pgfpathlineto{\pgfqpoint{4.947497in}{3.282279in}}%
\pgfpathlineto{\pgfqpoint{4.948364in}{3.371579in}}%
\pgfpathlineto{\pgfqpoint{4.949231in}{3.733831in}}%
\pgfpathlineto{\pgfqpoint{4.950098in}{3.389338in}}%
\pgfpathlineto{\pgfqpoint{4.950965in}{3.414998in}}%
\pgfpathlineto{\pgfqpoint{4.951832in}{3.429147in}}%
\pgfpathlineto{\pgfqpoint{4.952699in}{3.471204in}}%
\pgfpathlineto{\pgfqpoint{4.953566in}{3.590509in}}%
\pgfpathlineto{\pgfqpoint{4.954434in}{3.254128in}}%
\pgfpathlineto{\pgfqpoint{4.955301in}{3.445385in}}%
\pgfpathlineto{\pgfqpoint{4.956168in}{3.234705in}}%
\pgfpathlineto{\pgfqpoint{4.957035in}{3.285974in}}%
\pgfpathlineto{\pgfqpoint{4.957902in}{3.358720in}}%
\pgfpathlineto{\pgfqpoint{4.958769in}{3.687363in}}%
\pgfpathlineto{\pgfqpoint{4.960503in}{3.459411in}}%
\pgfpathlineto{\pgfqpoint{4.961371in}{3.428643in}}%
\pgfpathlineto{\pgfqpoint{4.962238in}{3.333597in}}%
\pgfpathlineto{\pgfqpoint{4.963105in}{3.590069in}}%
\pgfpathlineto{\pgfqpoint{4.963972in}{3.261170in}}%
\pgfpathlineto{\pgfqpoint{4.965706in}{3.430928in}}%
\pgfpathlineto{\pgfqpoint{4.966573in}{3.350656in}}%
\pgfpathlineto{\pgfqpoint{4.968308in}{3.632939in}}%
\pgfpathlineto{\pgfqpoint{4.970042in}{3.337808in}}%
\pgfpathlineto{\pgfqpoint{4.970909in}{3.426110in}}%
\pgfpathlineto{\pgfqpoint{4.971776in}{3.050690in}}%
\pgfpathlineto{\pgfqpoint{4.975245in}{3.722799in}}%
\pgfpathlineto{\pgfqpoint{4.976112in}{3.402425in}}%
\pgfpathlineto{\pgfqpoint{4.976979in}{3.431578in}}%
\pgfpathlineto{\pgfqpoint{4.978713in}{3.327484in}}%
\pgfpathlineto{\pgfqpoint{4.979580in}{3.618334in}}%
\pgfpathlineto{\pgfqpoint{4.980448in}{3.602401in}}%
\pgfpathlineto{\pgfqpoint{4.981315in}{3.669121in}}%
\pgfpathlineto{\pgfqpoint{4.982182in}{3.589103in}}%
\pgfpathlineto{\pgfqpoint{4.983916in}{3.258559in}}%
\pgfpathlineto{\pgfqpoint{4.985650in}{3.305010in}}%
\pgfpathlineto{\pgfqpoint{4.986517in}{3.216504in}}%
\pgfpathlineto{\pgfqpoint{4.987385in}{3.472134in}}%
\pgfpathlineto{\pgfqpoint{4.989119in}{3.082947in}}%
\pgfpathlineto{\pgfqpoint{4.989986in}{3.672200in}}%
\pgfpathlineto{\pgfqpoint{4.990853in}{3.132795in}}%
\pgfpathlineto{\pgfqpoint{4.991720in}{3.716564in}}%
\pgfpathlineto{\pgfqpoint{4.993455in}{3.072201in}}%
\pgfpathlineto{\pgfqpoint{4.994322in}{3.603170in}}%
\pgfpathlineto{\pgfqpoint{4.995189in}{3.475045in}}%
\pgfpathlineto{\pgfqpoint{4.996056in}{3.423937in}}%
\pgfpathlineto{\pgfqpoint{4.996923in}{3.273443in}}%
\pgfpathlineto{\pgfqpoint{4.997790in}{3.303259in}}%
\pgfpathlineto{\pgfqpoint{4.998657in}{3.288807in}}%
\pgfpathlineto{\pgfqpoint{4.999524in}{3.239170in}}%
\pgfpathlineto{\pgfqpoint{5.001259in}{3.764564in}}%
\pgfpathlineto{\pgfqpoint{5.002993in}{3.313738in}}%
\pgfpathlineto{\pgfqpoint{5.003860in}{3.342583in}}%
\pgfpathlineto{\pgfqpoint{5.004727in}{3.741053in}}%
\pgfpathlineto{\pgfqpoint{5.005594in}{3.352908in}}%
\pgfpathlineto{\pgfqpoint{5.006462in}{3.648602in}}%
\pgfpathlineto{\pgfqpoint{5.008196in}{3.184929in}}%
\pgfpathlineto{\pgfqpoint{5.009063in}{3.545431in}}%
\pgfpathlineto{\pgfqpoint{5.009930in}{3.234287in}}%
\pgfpathlineto{\pgfqpoint{5.010797in}{3.417950in}}%
\pgfpathlineto{\pgfqpoint{5.011664in}{3.122993in}}%
\pgfpathlineto{\pgfqpoint{5.013399in}{3.420728in}}%
\pgfpathlineto{\pgfqpoint{5.014266in}{3.281219in}}%
\pgfpathlineto{\pgfqpoint{5.016867in}{3.563321in}}%
\pgfpathlineto{\pgfqpoint{5.018601in}{3.191235in}}%
\pgfpathlineto{\pgfqpoint{5.019469in}{3.793298in}}%
\pgfpathlineto{\pgfqpoint{5.020336in}{3.274915in}}%
\pgfpathlineto{\pgfqpoint{5.021203in}{3.667951in}}%
\pgfpathlineto{\pgfqpoint{5.022070in}{3.144106in}}%
\pgfpathlineto{\pgfqpoint{5.022937in}{3.469703in}}%
\pgfpathlineto{\pgfqpoint{5.023804in}{3.109268in}}%
\pgfpathlineto{\pgfqpoint{5.024671in}{3.713472in}}%
\pgfpathlineto{\pgfqpoint{5.026406in}{3.418849in}}%
\pgfpathlineto{\pgfqpoint{5.027273in}{3.391623in}}%
\pgfpathlineto{\pgfqpoint{5.028140in}{3.083539in}}%
\pgfpathlineto{\pgfqpoint{5.029007in}{3.785651in}}%
\pgfpathlineto{\pgfqpoint{5.029874in}{3.250737in}}%
\pgfpathlineto{\pgfqpoint{5.030741in}{3.322709in}}%
\pgfpathlineto{\pgfqpoint{5.031608in}{3.641127in}}%
\pgfpathlineto{\pgfqpoint{5.032476in}{3.111095in}}%
\pgfpathlineto{\pgfqpoint{5.033343in}{3.609046in}}%
\pgfpathlineto{\pgfqpoint{5.034210in}{2.988757in}}%
\pgfpathlineto{\pgfqpoint{5.035077in}{3.756353in}}%
\pgfpathlineto{\pgfqpoint{5.035944in}{3.078890in}}%
\pgfpathlineto{\pgfqpoint{5.036811in}{3.675992in}}%
\pgfpathlineto{\pgfqpoint{5.037678in}{3.326309in}}%
\pgfpathlineto{\pgfqpoint{5.038545in}{3.430850in}}%
\pgfpathlineto{\pgfqpoint{5.040280in}{3.234103in}}%
\pgfpathlineto{\pgfqpoint{5.043748in}{3.591146in}}%
\pgfpathlineto{\pgfqpoint{5.044615in}{3.412488in}}%
\pgfpathlineto{\pgfqpoint{5.045483in}{3.507392in}}%
\pgfpathlineto{\pgfqpoint{5.046350in}{2.990480in}}%
\pgfpathlineto{\pgfqpoint{5.047217in}{3.487291in}}%
\pgfpathlineto{\pgfqpoint{5.048084in}{3.434552in}}%
\pgfpathlineto{\pgfqpoint{5.048951in}{3.430925in}}%
\pgfpathlineto{\pgfqpoint{5.049818in}{3.585944in}}%
\pgfpathlineto{\pgfqpoint{5.052420in}{3.010930in}}%
\pgfpathlineto{\pgfqpoint{5.053287in}{3.627158in}}%
\pgfpathlineto{\pgfqpoint{5.054154in}{3.409295in}}%
\pgfpathlineto{\pgfqpoint{5.055021in}{3.662885in}}%
\pgfpathlineto{\pgfqpoint{5.055888in}{3.045139in}}%
\pgfpathlineto{\pgfqpoint{5.056755in}{3.651839in}}%
\pgfpathlineto{\pgfqpoint{5.058490in}{3.020356in}}%
\pgfpathlineto{\pgfqpoint{5.059357in}{3.418019in}}%
\pgfpathlineto{\pgfqpoint{5.060224in}{3.187119in}}%
\pgfpathlineto{\pgfqpoint{5.061091in}{3.233118in}}%
\pgfpathlineto{\pgfqpoint{5.062825in}{3.200850in}}%
\pgfpathlineto{\pgfqpoint{5.064559in}{3.498198in}}%
\pgfpathlineto{\pgfqpoint{5.065427in}{3.437756in}}%
\pgfpathlineto{\pgfqpoint{5.066294in}{3.109030in}}%
\pgfpathlineto{\pgfqpoint{5.067161in}{3.310191in}}%
\pgfpathlineto{\pgfqpoint{5.068895in}{3.017692in}}%
\pgfpathlineto{\pgfqpoint{5.069762in}{3.115034in}}%
\pgfpathlineto{\pgfqpoint{5.070629in}{3.537899in}}%
\pgfpathlineto{\pgfqpoint{5.072364in}{3.002586in}}%
\pgfpathlineto{\pgfqpoint{5.074098in}{3.581078in}}%
\pgfpathlineto{\pgfqpoint{5.074965in}{3.490703in}}%
\pgfpathlineto{\pgfqpoint{5.075832in}{3.613109in}}%
\pgfpathlineto{\pgfqpoint{5.076699in}{3.141674in}}%
\pgfpathlineto{\pgfqpoint{5.077566in}{3.468447in}}%
\pgfpathlineto{\pgfqpoint{5.078434in}{3.466166in}}%
\pgfpathlineto{\pgfqpoint{5.079301in}{3.337034in}}%
\pgfpathlineto{\pgfqpoint{5.080168in}{3.031601in}}%
\pgfpathlineto{\pgfqpoint{5.082769in}{3.728477in}}%
\pgfpathlineto{\pgfqpoint{5.084503in}{3.173411in}}%
\pgfpathlineto{\pgfqpoint{5.085371in}{3.541709in}}%
\pgfpathlineto{\pgfqpoint{5.086238in}{3.474335in}}%
\pgfpathlineto{\pgfqpoint{5.087105in}{3.132390in}}%
\pgfpathlineto{\pgfqpoint{5.087972in}{3.690804in}}%
\pgfpathlineto{\pgfqpoint{5.088839in}{3.041627in}}%
\pgfpathlineto{\pgfqpoint{5.090573in}{3.382535in}}%
\pgfpathlineto{\pgfqpoint{5.091441in}{3.371626in}}%
\pgfpathlineto{\pgfqpoint{5.092308in}{3.315176in}}%
\pgfpathlineto{\pgfqpoint{5.093175in}{3.571249in}}%
\pgfpathlineto{\pgfqpoint{5.094042in}{3.425418in}}%
\pgfpathlineto{\pgfqpoint{5.095776in}{3.831567in}}%
\pgfpathlineto{\pgfqpoint{5.097510in}{3.260773in}}%
\pgfpathlineto{\pgfqpoint{5.099245in}{3.564925in}}%
\pgfpathlineto{\pgfqpoint{5.100112in}{3.002588in}}%
\pgfpathlineto{\pgfqpoint{5.101846in}{3.567122in}}%
\pgfpathlineto{\pgfqpoint{5.102713in}{3.216595in}}%
\pgfpathlineto{\pgfqpoint{5.103580in}{3.380546in}}%
\pgfpathlineto{\pgfqpoint{5.104448in}{3.141375in}}%
\pgfpathlineto{\pgfqpoint{5.105315in}{3.672464in}}%
\pgfpathlineto{\pgfqpoint{5.107049in}{3.315067in}}%
\pgfpathlineto{\pgfqpoint{5.107916in}{3.473693in}}%
\pgfpathlineto{\pgfqpoint{5.108783in}{3.381214in}}%
\pgfpathlineto{\pgfqpoint{5.109650in}{3.385098in}}%
\pgfpathlineto{\pgfqpoint{5.110517in}{3.460249in}}%
\pgfpathlineto{\pgfqpoint{5.111385in}{3.126948in}}%
\pgfpathlineto{\pgfqpoint{5.113119in}{3.634629in}}%
\pgfpathlineto{\pgfqpoint{5.113986in}{3.155995in}}%
\pgfpathlineto{\pgfqpoint{5.114853in}{3.238121in}}%
\pgfpathlineto{\pgfqpoint{5.115720in}{2.881477in}}%
\pgfpathlineto{\pgfqpoint{5.116587in}{3.533438in}}%
\pgfpathlineto{\pgfqpoint{5.117455in}{3.399233in}}%
\pgfpathlineto{\pgfqpoint{5.118322in}{3.309130in}}%
\pgfpathlineto{\pgfqpoint{5.119189in}{3.711523in}}%
\pgfpathlineto{\pgfqpoint{5.120923in}{3.162164in}}%
\pgfpathlineto{\pgfqpoint{5.121790in}{3.515963in}}%
\pgfpathlineto{\pgfqpoint{5.122657in}{3.210090in}}%
\pgfpathlineto{\pgfqpoint{5.124392in}{3.469983in}}%
\pgfpathlineto{\pgfqpoint{5.125259in}{3.347035in}}%
\pgfpathlineto{\pgfqpoint{5.126126in}{3.406869in}}%
\pgfpathlineto{\pgfqpoint{5.126993in}{3.388581in}}%
\pgfpathlineto{\pgfqpoint{5.128727in}{3.489828in}}%
\pgfpathlineto{\pgfqpoint{5.129594in}{3.320115in}}%
\pgfpathlineto{\pgfqpoint{5.130462in}{3.399509in}}%
\pgfpathlineto{\pgfqpoint{5.131329in}{3.634321in}}%
\pgfpathlineto{\pgfqpoint{5.132196in}{3.223881in}}%
\pgfpathlineto{\pgfqpoint{5.133063in}{3.351755in}}%
\pgfpathlineto{\pgfqpoint{5.133930in}{3.192755in}}%
\pgfpathlineto{\pgfqpoint{5.134797in}{3.702766in}}%
\pgfpathlineto{\pgfqpoint{5.135664in}{3.476722in}}%
\pgfpathlineto{\pgfqpoint{5.136531in}{3.545385in}}%
\pgfpathlineto{\pgfqpoint{5.137399in}{3.510465in}}%
\pgfpathlineto{\pgfqpoint{5.139133in}{3.263255in}}%
\pgfpathlineto{\pgfqpoint{5.140867in}{3.677859in}}%
\pgfpathlineto{\pgfqpoint{5.142601in}{3.251430in}}%
\pgfpathlineto{\pgfqpoint{5.143469in}{3.295196in}}%
\pgfpathlineto{\pgfqpoint{5.146070in}{3.709795in}}%
\pgfpathlineto{\pgfqpoint{5.148671in}{2.882610in}}%
\pgfpathlineto{\pgfqpoint{5.149538in}{3.517987in}}%
\pgfpathlineto{\pgfqpoint{5.150406in}{3.450310in}}%
\pgfpathlineto{\pgfqpoint{5.151273in}{3.330016in}}%
\pgfpathlineto{\pgfqpoint{5.152140in}{3.417331in}}%
\pgfpathlineto{\pgfqpoint{5.153874in}{3.098735in}}%
\pgfpathlineto{\pgfqpoint{5.154741in}{3.565964in}}%
\pgfpathlineto{\pgfqpoint{5.155608in}{3.271897in}}%
\pgfpathlineto{\pgfqpoint{5.156476in}{3.687725in}}%
\pgfpathlineto{\pgfqpoint{5.157343in}{3.336665in}}%
\pgfpathlineto{\pgfqpoint{5.158210in}{3.428230in}}%
\pgfpathlineto{\pgfqpoint{5.159944in}{3.069538in}}%
\pgfpathlineto{\pgfqpoint{5.160811in}{3.659309in}}%
\pgfpathlineto{\pgfqpoint{5.161678in}{3.325948in}}%
\pgfpathlineto{\pgfqpoint{5.162545in}{3.634463in}}%
\pgfpathlineto{\pgfqpoint{5.163413in}{3.343864in}}%
\pgfpathlineto{\pgfqpoint{5.164280in}{3.367811in}}%
\pgfpathlineto{\pgfqpoint{5.165147in}{3.423668in}}%
\pgfpathlineto{\pgfqpoint{5.166014in}{3.373589in}}%
\pgfpathlineto{\pgfqpoint{5.167748in}{3.073113in}}%
\pgfpathlineto{\pgfqpoint{5.169483in}{3.472369in}}%
\pgfpathlineto{\pgfqpoint{5.170350in}{3.519280in}}%
\pgfpathlineto{\pgfqpoint{5.171217in}{3.726955in}}%
\pgfpathlineto{\pgfqpoint{5.172951in}{3.412821in}}%
\pgfpathlineto{\pgfqpoint{5.173818in}{3.529751in}}%
\pgfpathlineto{\pgfqpoint{5.176420in}{3.009721in}}%
\pgfpathlineto{\pgfqpoint{5.177287in}{3.360054in}}%
\pgfpathlineto{\pgfqpoint{5.178154in}{3.299295in}}%
\pgfpathlineto{\pgfqpoint{5.179021in}{3.249958in}}%
\pgfpathlineto{\pgfqpoint{5.179888in}{3.267289in}}%
\pgfpathlineto{\pgfqpoint{5.180755in}{3.321174in}}%
\pgfpathlineto{\pgfqpoint{5.181622in}{3.484505in}}%
\pgfpathlineto{\pgfqpoint{5.182490in}{3.234970in}}%
\pgfpathlineto{\pgfqpoint{5.183357in}{3.417262in}}%
\pgfpathlineto{\pgfqpoint{5.184224in}{3.385858in}}%
\pgfpathlineto{\pgfqpoint{5.185091in}{3.439212in}}%
\pgfpathlineto{\pgfqpoint{5.185958in}{3.253538in}}%
\pgfpathlineto{\pgfqpoint{5.188559in}{3.576811in}}%
\pgfpathlineto{\pgfqpoint{5.189427in}{3.629266in}}%
\pgfpathlineto{\pgfqpoint{5.190294in}{3.533162in}}%
\pgfpathlineto{\pgfqpoint{5.191161in}{3.300159in}}%
\pgfpathlineto{\pgfqpoint{5.192028in}{3.336996in}}%
\pgfpathlineto{\pgfqpoint{5.192895in}{3.581032in}}%
\pgfpathlineto{\pgfqpoint{5.193762in}{3.406402in}}%
\pgfpathlineto{\pgfqpoint{5.194629in}{3.483400in}}%
\pgfpathlineto{\pgfqpoint{5.197231in}{3.210099in}}%
\pgfpathlineto{\pgfqpoint{5.198965in}{3.633402in}}%
\pgfpathlineto{\pgfqpoint{5.200699in}{3.353582in}}%
\pgfpathlineto{\pgfqpoint{5.201566in}{3.376709in}}%
\pgfpathlineto{\pgfqpoint{5.202434in}{3.434618in}}%
\pgfpathlineto{\pgfqpoint{5.203301in}{3.174605in}}%
\pgfpathlineto{\pgfqpoint{5.205035in}{3.604947in}}%
\pgfpathlineto{\pgfqpoint{5.205902in}{3.555914in}}%
\pgfpathlineto{\pgfqpoint{5.206769in}{3.236369in}}%
\pgfpathlineto{\pgfqpoint{5.207636in}{3.315509in}}%
\pgfpathlineto{\pgfqpoint{5.208503in}{3.619738in}}%
\pgfpathlineto{\pgfqpoint{5.210238in}{3.199967in}}%
\pgfpathlineto{\pgfqpoint{5.211105in}{3.293941in}}%
\pgfpathlineto{\pgfqpoint{5.211972in}{3.188893in}}%
\pgfpathlineto{\pgfqpoint{5.212839in}{3.248960in}}%
\pgfpathlineto{\pgfqpoint{5.213706in}{3.461047in}}%
\pgfpathlineto{\pgfqpoint{5.214573in}{3.406592in}}%
\pgfpathlineto{\pgfqpoint{5.216308in}{3.299879in}}%
\pgfpathlineto{\pgfqpoint{5.217175in}{3.554378in}}%
\pgfpathlineto{\pgfqpoint{5.218042in}{3.501350in}}%
\pgfpathlineto{\pgfqpoint{5.218909in}{3.671239in}}%
\pgfpathlineto{\pgfqpoint{5.219776in}{3.171698in}}%
\pgfpathlineto{\pgfqpoint{5.220643in}{3.438052in}}%
\pgfpathlineto{\pgfqpoint{5.221510in}{3.324314in}}%
\pgfpathlineto{\pgfqpoint{5.223245in}{3.697858in}}%
\pgfpathlineto{\pgfqpoint{5.224979in}{3.109670in}}%
\pgfpathlineto{\pgfqpoint{5.225846in}{3.277121in}}%
\pgfpathlineto{\pgfqpoint{5.227580in}{3.477291in}}%
\pgfpathlineto{\pgfqpoint{5.228448in}{3.437397in}}%
\pgfpathlineto{\pgfqpoint{5.230182in}{3.142348in}}%
\pgfpathlineto{\pgfqpoint{5.231049in}{3.554415in}}%
\pgfpathlineto{\pgfqpoint{5.231916in}{3.109708in}}%
\pgfpathlineto{\pgfqpoint{5.232783in}{3.529604in}}%
\pgfpathlineto{\pgfqpoint{5.234517in}{3.353122in}}%
\pgfpathlineto{\pgfqpoint{5.235385in}{3.331366in}}%
\pgfpathlineto{\pgfqpoint{5.236252in}{3.224413in}}%
\pgfpathlineto{\pgfqpoint{5.237119in}{3.760847in}}%
\pgfpathlineto{\pgfqpoint{5.237986in}{3.110970in}}%
\pgfpathlineto{\pgfqpoint{5.239720in}{3.637818in}}%
\pgfpathlineto{\pgfqpoint{5.240587in}{3.536963in}}%
\pgfpathlineto{\pgfqpoint{5.241455in}{3.542073in}}%
\pgfpathlineto{\pgfqpoint{5.242322in}{3.650882in}}%
\pgfpathlineto{\pgfqpoint{5.243189in}{3.256128in}}%
\pgfpathlineto{\pgfqpoint{5.245790in}{3.497396in}}%
\pgfpathlineto{\pgfqpoint{5.246657in}{3.117507in}}%
\pgfpathlineto{\pgfqpoint{5.248392in}{3.610728in}}%
\pgfpathlineto{\pgfqpoint{5.249259in}{3.333517in}}%
\pgfpathlineto{\pgfqpoint{5.250126in}{3.413306in}}%
\pgfpathlineto{\pgfqpoint{5.250993in}{3.196875in}}%
\pgfpathlineto{\pgfqpoint{5.251860in}{3.630990in}}%
\pgfpathlineto{\pgfqpoint{5.253594in}{3.104480in}}%
\pgfpathlineto{\pgfqpoint{5.254462in}{3.471852in}}%
\pgfpathlineto{\pgfqpoint{5.256196in}{3.197717in}}%
\pgfpathlineto{\pgfqpoint{5.257063in}{3.560406in}}%
\pgfpathlineto{\pgfqpoint{5.257930in}{3.398214in}}%
\pgfpathlineto{\pgfqpoint{5.258797in}{3.575283in}}%
\pgfpathlineto{\pgfqpoint{5.260531in}{3.350355in}}%
\pgfpathlineto{\pgfqpoint{5.262266in}{3.625630in}}%
\pgfpathlineto{\pgfqpoint{5.263133in}{3.399402in}}%
\pgfpathlineto{\pgfqpoint{5.264867in}{3.506280in}}%
\pgfpathlineto{\pgfqpoint{5.265734in}{3.315668in}}%
\pgfpathlineto{\pgfqpoint{5.266601in}{3.336230in}}%
\pgfpathlineto{\pgfqpoint{5.267469in}{3.288759in}}%
\pgfpathlineto{\pgfqpoint{5.268336in}{3.681288in}}%
\pgfpathlineto{\pgfqpoint{5.270070in}{3.300287in}}%
\pgfpathlineto{\pgfqpoint{5.270937in}{3.362040in}}%
\pgfpathlineto{\pgfqpoint{5.271804in}{3.593799in}}%
\pgfpathlineto{\pgfqpoint{5.272671in}{2.763978in}}%
\pgfpathlineto{\pgfqpoint{5.273538in}{3.233002in}}%
\pgfpathlineto{\pgfqpoint{5.274406in}{3.205445in}}%
\pgfpathlineto{\pgfqpoint{5.275273in}{3.312659in}}%
\pgfpathlineto{\pgfqpoint{5.277007in}{3.686027in}}%
\pgfpathlineto{\pgfqpoint{5.277874in}{3.417578in}}%
\pgfpathlineto{\pgfqpoint{5.278741in}{3.475302in}}%
\pgfpathlineto{\pgfqpoint{5.279608in}{3.560716in}}%
\pgfpathlineto{\pgfqpoint{5.280476in}{3.338031in}}%
\pgfpathlineto{\pgfqpoint{5.281343in}{3.374188in}}%
\pgfpathlineto{\pgfqpoint{5.282210in}{3.533061in}}%
\pgfpathlineto{\pgfqpoint{5.283944in}{3.182334in}}%
\pgfpathlineto{\pgfqpoint{5.285678in}{3.570347in}}%
\pgfpathlineto{\pgfqpoint{5.286545in}{3.590373in}}%
\pgfpathlineto{\pgfqpoint{5.287413in}{3.800883in}}%
\pgfpathlineto{\pgfqpoint{5.288280in}{3.138876in}}%
\pgfpathlineto{\pgfqpoint{5.289147in}{3.245242in}}%
\pgfpathlineto{\pgfqpoint{5.290014in}{3.266790in}}%
\pgfpathlineto{\pgfqpoint{5.290881in}{3.516468in}}%
\pgfpathlineto{\pgfqpoint{5.291748in}{3.257134in}}%
\pgfpathlineto{\pgfqpoint{5.292615in}{3.291511in}}%
\pgfpathlineto{\pgfqpoint{5.294350in}{3.749023in}}%
\pgfpathlineto{\pgfqpoint{5.295217in}{3.465229in}}%
\pgfpathlineto{\pgfqpoint{5.296084in}{3.553853in}}%
\pgfpathlineto{\pgfqpoint{5.297818in}{3.411364in}}%
\pgfpathlineto{\pgfqpoint{5.298685in}{3.574053in}}%
\pgfpathlineto{\pgfqpoint{5.299552in}{3.323609in}}%
\pgfpathlineto{\pgfqpoint{5.301287in}{3.728901in}}%
\pgfpathlineto{\pgfqpoint{5.303021in}{3.712899in}}%
\pgfpathlineto{\pgfqpoint{5.303888in}{3.158407in}}%
\pgfpathlineto{\pgfqpoint{5.304755in}{3.179462in}}%
\pgfpathlineto{\pgfqpoint{5.305622in}{3.373927in}}%
\pgfpathlineto{\pgfqpoint{5.306490in}{3.113203in}}%
\pgfpathlineto{\pgfqpoint{5.309091in}{3.884777in}}%
\pgfpathlineto{\pgfqpoint{5.309958in}{3.514686in}}%
\pgfpathlineto{\pgfqpoint{5.310825in}{3.600524in}}%
\pgfpathlineto{\pgfqpoint{5.312559in}{3.255531in}}%
\pgfpathlineto{\pgfqpoint{5.313427in}{3.581232in}}%
\pgfpathlineto{\pgfqpoint{5.314294in}{3.482683in}}%
\pgfpathlineto{\pgfqpoint{5.315161in}{3.536180in}}%
\pgfpathlineto{\pgfqpoint{5.316028in}{3.661293in}}%
\pgfpathlineto{\pgfqpoint{5.317762in}{3.200302in}}%
\pgfpathlineto{\pgfqpoint{5.318629in}{3.381284in}}%
\pgfpathlineto{\pgfqpoint{5.319497in}{3.275181in}}%
\pgfpathlineto{\pgfqpoint{5.320364in}{3.005387in}}%
\pgfpathlineto{\pgfqpoint{5.321231in}{3.463584in}}%
\pgfpathlineto{\pgfqpoint{5.322965in}{3.247716in}}%
\pgfpathlineto{\pgfqpoint{5.323832in}{3.502676in}}%
\pgfpathlineto{\pgfqpoint{5.324699in}{3.200347in}}%
\pgfpathlineto{\pgfqpoint{5.326434in}{3.665859in}}%
\pgfpathlineto{\pgfqpoint{5.327301in}{3.149327in}}%
\pgfpathlineto{\pgfqpoint{5.328168in}{3.505790in}}%
\pgfpathlineto{\pgfqpoint{5.329035in}{3.400410in}}%
\pgfpathlineto{\pgfqpoint{5.329902in}{3.540213in}}%
\pgfpathlineto{\pgfqpoint{5.332503in}{3.144807in}}%
\pgfpathlineto{\pgfqpoint{5.333371in}{3.543965in}}%
\pgfpathlineto{\pgfqpoint{5.334238in}{3.302077in}}%
\pgfpathlineto{\pgfqpoint{5.335105in}{3.664661in}}%
\pgfpathlineto{\pgfqpoint{5.337706in}{3.268533in}}%
\pgfpathlineto{\pgfqpoint{5.338573in}{3.517807in}}%
\pgfpathlineto{\pgfqpoint{5.340308in}{3.308002in}}%
\pgfpathlineto{\pgfqpoint{5.341175in}{3.531937in}}%
\pgfpathlineto{\pgfqpoint{5.342042in}{3.369847in}}%
\pgfpathlineto{\pgfqpoint{5.342909in}{3.540341in}}%
\pgfpathlineto{\pgfqpoint{5.343776in}{3.476653in}}%
\pgfpathlineto{\pgfqpoint{5.344643in}{3.291043in}}%
\pgfpathlineto{\pgfqpoint{5.345510in}{3.520305in}}%
\pgfpathlineto{\pgfqpoint{5.346378in}{3.304641in}}%
\pgfpathlineto{\pgfqpoint{5.347245in}{3.401094in}}%
\pgfpathlineto{\pgfqpoint{5.348112in}{3.278524in}}%
\pgfpathlineto{\pgfqpoint{5.348979in}{3.500024in}}%
\pgfpathlineto{\pgfqpoint{5.349846in}{3.227748in}}%
\pgfpathlineto{\pgfqpoint{5.350713in}{3.350510in}}%
\pgfpathlineto{\pgfqpoint{5.351580in}{3.341138in}}%
\pgfpathlineto{\pgfqpoint{5.352448in}{3.157937in}}%
\pgfpathlineto{\pgfqpoint{5.355916in}{3.519583in}}%
\pgfpathlineto{\pgfqpoint{5.356783in}{3.119795in}}%
\pgfpathlineto{\pgfqpoint{5.357650in}{3.483914in}}%
\pgfpathlineto{\pgfqpoint{5.359385in}{3.220994in}}%
\pgfpathlineto{\pgfqpoint{5.361119in}{3.626181in}}%
\pgfpathlineto{\pgfqpoint{5.361986in}{3.289904in}}%
\pgfpathlineto{\pgfqpoint{5.363720in}{3.529055in}}%
\pgfpathlineto{\pgfqpoint{5.364587in}{3.402926in}}%
\pgfpathlineto{\pgfqpoint{5.366322in}{3.530803in}}%
\pgfpathlineto{\pgfqpoint{5.367189in}{3.295057in}}%
\pgfpathlineto{\pgfqpoint{5.368923in}{3.450202in}}%
\pgfpathlineto{\pgfqpoint{5.369790in}{3.462411in}}%
\pgfpathlineto{\pgfqpoint{5.370657in}{3.209652in}}%
\pgfpathlineto{\pgfqpoint{5.371524in}{3.499830in}}%
\pgfpathlineto{\pgfqpoint{5.372392in}{3.262997in}}%
\pgfpathlineto{\pgfqpoint{5.373259in}{3.664439in}}%
\pgfpathlineto{\pgfqpoint{5.374126in}{2.870465in}}%
\pgfpathlineto{\pgfqpoint{5.376727in}{3.538877in}}%
\pgfpathlineto{\pgfqpoint{5.377594in}{3.503250in}}%
\pgfpathlineto{\pgfqpoint{5.378462in}{3.219735in}}%
\pgfpathlineto{\pgfqpoint{5.380196in}{3.521945in}}%
\pgfpathlineto{\pgfqpoint{5.381063in}{3.247550in}}%
\pgfpathlineto{\pgfqpoint{5.381930in}{3.591377in}}%
\pgfpathlineto{\pgfqpoint{5.382797in}{3.469866in}}%
\pgfpathlineto{\pgfqpoint{5.383664in}{3.579762in}}%
\pgfpathlineto{\pgfqpoint{5.384531in}{3.492754in}}%
\pgfpathlineto{\pgfqpoint{5.385399in}{3.544066in}}%
\pgfpathlineto{\pgfqpoint{5.386266in}{3.142465in}}%
\pgfpathlineto{\pgfqpoint{5.387133in}{3.544930in}}%
\pgfpathlineto{\pgfqpoint{5.389734in}{2.970375in}}%
\pgfpathlineto{\pgfqpoint{5.390601in}{3.115012in}}%
\pgfpathlineto{\pgfqpoint{5.391469in}{3.665276in}}%
\pgfpathlineto{\pgfqpoint{5.392336in}{3.581657in}}%
\pgfpathlineto{\pgfqpoint{5.394070in}{3.435230in}}%
\pgfpathlineto{\pgfqpoint{5.394937in}{3.592022in}}%
\pgfpathlineto{\pgfqpoint{5.395804in}{3.004962in}}%
\pgfpathlineto{\pgfqpoint{5.398406in}{3.578578in}}%
\pgfpathlineto{\pgfqpoint{5.399273in}{3.579530in}}%
\pgfpathlineto{\pgfqpoint{5.400140in}{3.534382in}}%
\pgfpathlineto{\pgfqpoint{5.401007in}{3.359983in}}%
\pgfpathlineto{\pgfqpoint{5.402741in}{3.584009in}}%
\pgfpathlineto{\pgfqpoint{5.405343in}{3.129062in}}%
\pgfpathlineto{\pgfqpoint{5.406210in}{3.194921in}}%
\pgfpathlineto{\pgfqpoint{5.407077in}{3.159273in}}%
\pgfpathlineto{\pgfqpoint{5.407944in}{3.520011in}}%
\pgfpathlineto{\pgfqpoint{5.408811in}{3.436372in}}%
\pgfpathlineto{\pgfqpoint{5.409678in}{3.474501in}}%
\pgfpathlineto{\pgfqpoint{5.410545in}{3.238873in}}%
\pgfpathlineto{\pgfqpoint{5.411413in}{3.710714in}}%
\pgfpathlineto{\pgfqpoint{5.412280in}{3.292874in}}%
\pgfpathlineto{\pgfqpoint{5.414881in}{3.730535in}}%
\pgfpathlineto{\pgfqpoint{5.415748in}{3.449846in}}%
\pgfpathlineto{\pgfqpoint{5.416615in}{3.836819in}}%
\pgfpathlineto{\pgfqpoint{5.417483in}{3.619989in}}%
\pgfpathlineto{\pgfqpoint{5.418350in}{3.044880in}}%
\pgfpathlineto{\pgfqpoint{5.420951in}{3.515360in}}%
\pgfpathlineto{\pgfqpoint{5.421818in}{3.572658in}}%
\pgfpathlineto{\pgfqpoint{5.423552in}{3.498890in}}%
\pgfpathlineto{\pgfqpoint{5.424420in}{3.586198in}}%
\pgfpathlineto{\pgfqpoint{5.427021in}{3.191112in}}%
\pgfpathlineto{\pgfqpoint{5.427888in}{3.514643in}}%
\pgfpathlineto{\pgfqpoint{5.428755in}{3.376001in}}%
\pgfpathlineto{\pgfqpoint{5.429622in}{3.609952in}}%
\pgfpathlineto{\pgfqpoint{5.430490in}{3.495524in}}%
\pgfpathlineto{\pgfqpoint{5.431357in}{3.619386in}}%
\pgfpathlineto{\pgfqpoint{5.432224in}{3.330079in}}%
\pgfpathlineto{\pgfqpoint{5.433091in}{3.358337in}}%
\pgfpathlineto{\pgfqpoint{5.433958in}{3.626849in}}%
\pgfpathlineto{\pgfqpoint{5.434825in}{3.443880in}}%
\pgfpathlineto{\pgfqpoint{5.435692in}{3.594404in}}%
\pgfpathlineto{\pgfqpoint{5.436559in}{3.483678in}}%
\pgfpathlineto{\pgfqpoint{5.437427in}{3.585272in}}%
\pgfpathlineto{\pgfqpoint{5.439161in}{3.140546in}}%
\pgfpathlineto{\pgfqpoint{5.440895in}{3.649335in}}%
\pgfpathlineto{\pgfqpoint{5.442629in}{3.296375in}}%
\pgfpathlineto{\pgfqpoint{5.444364in}{3.783705in}}%
\pgfpathlineto{\pgfqpoint{5.445231in}{3.269331in}}%
\pgfpathlineto{\pgfqpoint{5.446098in}{3.742296in}}%
\pgfpathlineto{\pgfqpoint{5.448699in}{3.218526in}}%
\pgfpathlineto{\pgfqpoint{5.449566in}{3.408040in}}%
\pgfpathlineto{\pgfqpoint{5.450434in}{3.053999in}}%
\pgfpathlineto{\pgfqpoint{5.452168in}{3.577144in}}%
\pgfpathlineto{\pgfqpoint{5.453035in}{3.207284in}}%
\pgfpathlineto{\pgfqpoint{5.454769in}{3.526677in}}%
\pgfpathlineto{\pgfqpoint{5.455636in}{3.355600in}}%
\pgfpathlineto{\pgfqpoint{5.456503in}{3.456748in}}%
\pgfpathlineto{\pgfqpoint{5.457371in}{3.115681in}}%
\pgfpathlineto{\pgfqpoint{5.458238in}{3.792980in}}%
\pgfpathlineto{\pgfqpoint{5.459972in}{3.262142in}}%
\pgfpathlineto{\pgfqpoint{5.462573in}{3.599015in}}%
\pgfpathlineto{\pgfqpoint{5.463441in}{3.417454in}}%
\pgfpathlineto{\pgfqpoint{5.464308in}{3.648224in}}%
\pgfpathlineto{\pgfqpoint{5.466042in}{3.232327in}}%
\pgfpathlineto{\pgfqpoint{5.467776in}{3.686191in}}%
\pgfpathlineto{\pgfqpoint{5.468643in}{3.277947in}}%
\pgfpathlineto{\pgfqpoint{5.469510in}{3.834889in}}%
\pgfpathlineto{\pgfqpoint{5.470378in}{3.514235in}}%
\pgfpathlineto{\pgfqpoint{5.471245in}{3.669972in}}%
\pgfpathlineto{\pgfqpoint{5.472112in}{3.280921in}}%
\pgfpathlineto{\pgfqpoint{5.472979in}{3.622367in}}%
\pgfpathlineto{\pgfqpoint{5.474713in}{3.064923in}}%
\pgfpathlineto{\pgfqpoint{5.475580in}{3.354396in}}%
\pgfpathlineto{\pgfqpoint{5.476448in}{3.249731in}}%
\pgfpathlineto{\pgfqpoint{5.477315in}{3.389611in}}%
\pgfpathlineto{\pgfqpoint{5.478182in}{3.155612in}}%
\pgfpathlineto{\pgfqpoint{5.479049in}{3.219297in}}%
\pgfpathlineto{\pgfqpoint{5.479916in}{3.436699in}}%
\pgfpathlineto{\pgfqpoint{5.480783in}{3.352704in}}%
\pgfpathlineto{\pgfqpoint{5.483385in}{3.738117in}}%
\pgfpathlineto{\pgfqpoint{5.484252in}{3.327818in}}%
\pgfpathlineto{\pgfqpoint{5.485119in}{3.535419in}}%
\pgfpathlineto{\pgfqpoint{5.485986in}{3.186178in}}%
\pgfpathlineto{\pgfqpoint{5.486853in}{3.246247in}}%
\pgfpathlineto{\pgfqpoint{5.487720in}{3.653981in}}%
\pgfpathlineto{\pgfqpoint{5.488587in}{3.357130in}}%
\pgfpathlineto{\pgfqpoint{5.489455in}{3.731198in}}%
\pgfpathlineto{\pgfqpoint{5.491189in}{3.273345in}}%
\pgfpathlineto{\pgfqpoint{5.492056in}{3.341571in}}%
\pgfpathlineto{\pgfqpoint{5.492923in}{3.528630in}}%
\pgfpathlineto{\pgfqpoint{5.493790in}{3.268641in}}%
\pgfpathlineto{\pgfqpoint{5.496392in}{3.689852in}}%
\pgfpathlineto{\pgfqpoint{5.497259in}{3.599341in}}%
\pgfpathlineto{\pgfqpoint{5.498126in}{3.365473in}}%
\pgfpathlineto{\pgfqpoint{5.498993in}{3.387554in}}%
\pgfpathlineto{\pgfqpoint{5.499860in}{3.516180in}}%
\pgfpathlineto{\pgfqpoint{5.500727in}{3.222150in}}%
\pgfpathlineto{\pgfqpoint{5.501594in}{3.599054in}}%
\pgfpathlineto{\pgfqpoint{5.503329in}{3.077128in}}%
\pgfpathlineto{\pgfqpoint{5.506797in}{3.675973in}}%
\pgfpathlineto{\pgfqpoint{5.507664in}{3.368492in}}%
\pgfpathlineto{\pgfqpoint{5.508531in}{3.569425in}}%
\pgfpathlineto{\pgfqpoint{5.509399in}{3.163680in}}%
\pgfpathlineto{\pgfqpoint{5.510266in}{3.276776in}}%
\pgfpathlineto{\pgfqpoint{5.511133in}{2.998535in}}%
\pgfpathlineto{\pgfqpoint{5.512000in}{3.449181in}}%
\pgfpathlineto{\pgfqpoint{5.512867in}{3.435416in}}%
\pgfpathlineto{\pgfqpoint{5.513734in}{3.594474in}}%
\pgfpathlineto{\pgfqpoint{5.516336in}{3.251942in}}%
\pgfpathlineto{\pgfqpoint{5.518070in}{3.488705in}}%
\pgfpathlineto{\pgfqpoint{5.518937in}{3.276231in}}%
\pgfpathlineto{\pgfqpoint{5.519804in}{3.653775in}}%
\pgfpathlineto{\pgfqpoint{5.521538in}{3.139831in}}%
\pgfpathlineto{\pgfqpoint{5.522406in}{3.676876in}}%
\pgfpathlineto{\pgfqpoint{5.523273in}{3.230723in}}%
\pgfpathlineto{\pgfqpoint{5.524140in}{3.694097in}}%
\pgfpathlineto{\pgfqpoint{5.525007in}{3.552739in}}%
\pgfpathlineto{\pgfqpoint{5.525874in}{3.596005in}}%
\pgfpathlineto{\pgfqpoint{5.527608in}{3.007975in}}%
\pgfpathlineto{\pgfqpoint{5.528476in}{3.277956in}}%
\pgfpathlineto{\pgfqpoint{5.529343in}{3.066777in}}%
\pgfpathlineto{\pgfqpoint{5.530210in}{3.220264in}}%
\pgfpathlineto{\pgfqpoint{5.531077in}{3.598177in}}%
\pgfpathlineto{\pgfqpoint{5.531944in}{3.265069in}}%
\pgfpathlineto{\pgfqpoint{5.532811in}{3.673420in}}%
\pgfpathlineto{\pgfqpoint{5.534545in}{3.061857in}}%
\pgfpathlineto{\pgfqpoint{5.534545in}{3.061857in}}%
\pgfusepath{stroke}%
\end{pgfscope}%
\begin{pgfscope}%
\pgfsetrectcap%
\pgfsetmiterjoin%
\pgfsetlinewidth{0.803000pt}%
\definecolor{currentstroke}{rgb}{0.000000,0.000000,0.000000}%
\pgfsetstrokecolor{currentstroke}%
\pgfsetdash{}{0pt}%
\pgfpathmoveto{\pgfqpoint{0.800000in}{2.544000in}}%
\pgfpathlineto{\pgfqpoint{0.800000in}{4.224000in}}%
\pgfusepath{stroke}%
\end{pgfscope}%
\begin{pgfscope}%
\pgfsetrectcap%
\pgfsetmiterjoin%
\pgfsetlinewidth{0.803000pt}%
\definecolor{currentstroke}{rgb}{0.000000,0.000000,0.000000}%
\pgfsetstrokecolor{currentstroke}%
\pgfsetdash{}{0pt}%
\pgfpathmoveto{\pgfqpoint{5.760000in}{2.544000in}}%
\pgfpathlineto{\pgfqpoint{5.760000in}{4.224000in}}%
\pgfusepath{stroke}%
\end{pgfscope}%
\begin{pgfscope}%
\pgfsetrectcap%
\pgfsetmiterjoin%
\pgfsetlinewidth{0.803000pt}%
\definecolor{currentstroke}{rgb}{0.000000,0.000000,0.000000}%
\pgfsetstrokecolor{currentstroke}%
\pgfsetdash{}{0pt}%
\pgfpathmoveto{\pgfqpoint{0.800000in}{2.544000in}}%
\pgfpathlineto{\pgfqpoint{5.760000in}{2.544000in}}%
\pgfusepath{stroke}%
\end{pgfscope}%
\begin{pgfscope}%
\pgfsetrectcap%
\pgfsetmiterjoin%
\pgfsetlinewidth{0.803000pt}%
\definecolor{currentstroke}{rgb}{0.000000,0.000000,0.000000}%
\pgfsetstrokecolor{currentstroke}%
\pgfsetdash{}{0pt}%
\pgfpathmoveto{\pgfqpoint{0.800000in}{4.224000in}}%
\pgfpathlineto{\pgfqpoint{5.760000in}{4.224000in}}%
\pgfusepath{stroke}%
\end{pgfscope}%
\begin{pgfscope}%
\pgfsetbuttcap%
\pgfsetmiterjoin%
\definecolor{currentfill}{rgb}{1.000000,1.000000,1.000000}%
\pgfsetfillcolor{currentfill}%
\pgfsetlinewidth{0.000000pt}%
\definecolor{currentstroke}{rgb}{0.000000,0.000000,0.000000}%
\pgfsetstrokecolor{currentstroke}%
\pgfsetstrokeopacity{0.000000}%
\pgfsetdash{}{0pt}%
\pgfpathmoveto{\pgfqpoint{0.800000in}{0.528000in}}%
\pgfpathlineto{\pgfqpoint{5.760000in}{0.528000in}}%
\pgfpathlineto{\pgfqpoint{5.760000in}{2.208000in}}%
\pgfpathlineto{\pgfqpoint{0.800000in}{2.208000in}}%
\pgfpathclose%
\pgfusepath{fill}%
\end{pgfscope}%
\begin{pgfscope}%
\pgfpathrectangle{\pgfqpoint{0.800000in}{0.528000in}}{\pgfqpoint{4.960000in}{1.680000in}} %
\pgfusepath{clip}%
\pgfsetrectcap%
\pgfsetroundjoin%
\pgfsetlinewidth{0.803000pt}%
\definecolor{currentstroke}{rgb}{0.690196,0.690196,0.690196}%
\pgfsetstrokecolor{currentstroke}%
\pgfsetdash{}{0pt}%
\pgfpathmoveto{\pgfqpoint{0.967635in}{0.528000in}}%
\pgfpathlineto{\pgfqpoint{0.967635in}{2.208000in}}%
\pgfusepath{stroke}%
\end{pgfscope}%
\begin{pgfscope}%
\pgfsetbuttcap%
\pgfsetroundjoin%
\definecolor{currentfill}{rgb}{0.000000,0.000000,0.000000}%
\pgfsetfillcolor{currentfill}%
\pgfsetlinewidth{0.803000pt}%
\definecolor{currentstroke}{rgb}{0.000000,0.000000,0.000000}%
\pgfsetstrokecolor{currentstroke}%
\pgfsetdash{}{0pt}%
\pgfsys@defobject{currentmarker}{\pgfqpoint{0.000000in}{-0.048611in}}{\pgfqpoint{0.000000in}{0.000000in}}{%
\pgfpathmoveto{\pgfqpoint{0.000000in}{0.000000in}}%
\pgfpathlineto{\pgfqpoint{0.000000in}{-0.048611in}}%
\pgfusepath{stroke,fill}%
}%
\begin{pgfscope}%
\pgfsys@transformshift{0.967635in}{0.528000in}%
\pgfsys@useobject{currentmarker}{}%
\end{pgfscope}%
\end{pgfscope}%
\begin{pgfscope}%
\pgftext[x=0.967635in,y=0.430778in,,top]{\sffamily\fontsize{10.000000}{12.000000}\selectfont -100}%
\end{pgfscope}%
\begin{pgfscope}%
\pgfpathrectangle{\pgfqpoint{0.800000in}{0.528000in}}{\pgfqpoint{4.960000in}{1.680000in}} %
\pgfusepath{clip}%
\pgfsetrectcap%
\pgfsetroundjoin%
\pgfsetlinewidth{0.803000pt}%
\definecolor{currentstroke}{rgb}{0.690196,0.690196,0.690196}%
\pgfsetstrokecolor{currentstroke}%
\pgfsetdash{}{0pt}%
\pgfpathmoveto{\pgfqpoint{1.545834in}{0.528000in}}%
\pgfpathlineto{\pgfqpoint{1.545834in}{2.208000in}}%
\pgfusepath{stroke}%
\end{pgfscope}%
\begin{pgfscope}%
\pgfsetbuttcap%
\pgfsetroundjoin%
\definecolor{currentfill}{rgb}{0.000000,0.000000,0.000000}%
\pgfsetfillcolor{currentfill}%
\pgfsetlinewidth{0.803000pt}%
\definecolor{currentstroke}{rgb}{0.000000,0.000000,0.000000}%
\pgfsetstrokecolor{currentstroke}%
\pgfsetdash{}{0pt}%
\pgfsys@defobject{currentmarker}{\pgfqpoint{0.000000in}{-0.048611in}}{\pgfqpoint{0.000000in}{0.000000in}}{%
\pgfpathmoveto{\pgfqpoint{0.000000in}{0.000000in}}%
\pgfpathlineto{\pgfqpoint{0.000000in}{-0.048611in}}%
\pgfusepath{stroke,fill}%
}%
\begin{pgfscope}%
\pgfsys@transformshift{1.545834in}{0.528000in}%
\pgfsys@useobject{currentmarker}{}%
\end{pgfscope}%
\end{pgfscope}%
\begin{pgfscope}%
\pgftext[x=1.545834in,y=0.430778in,,top]{\sffamily\fontsize{10.000000}{12.000000}\selectfont -75}%
\end{pgfscope}%
\begin{pgfscope}%
\pgfpathrectangle{\pgfqpoint{0.800000in}{0.528000in}}{\pgfqpoint{4.960000in}{1.680000in}} %
\pgfusepath{clip}%
\pgfsetrectcap%
\pgfsetroundjoin%
\pgfsetlinewidth{0.803000pt}%
\definecolor{currentstroke}{rgb}{0.690196,0.690196,0.690196}%
\pgfsetstrokecolor{currentstroke}%
\pgfsetdash{}{0pt}%
\pgfpathmoveto{\pgfqpoint{2.124034in}{0.528000in}}%
\pgfpathlineto{\pgfqpoint{2.124034in}{2.208000in}}%
\pgfusepath{stroke}%
\end{pgfscope}%
\begin{pgfscope}%
\pgfsetbuttcap%
\pgfsetroundjoin%
\definecolor{currentfill}{rgb}{0.000000,0.000000,0.000000}%
\pgfsetfillcolor{currentfill}%
\pgfsetlinewidth{0.803000pt}%
\definecolor{currentstroke}{rgb}{0.000000,0.000000,0.000000}%
\pgfsetstrokecolor{currentstroke}%
\pgfsetdash{}{0pt}%
\pgfsys@defobject{currentmarker}{\pgfqpoint{0.000000in}{-0.048611in}}{\pgfqpoint{0.000000in}{0.000000in}}{%
\pgfpathmoveto{\pgfqpoint{0.000000in}{0.000000in}}%
\pgfpathlineto{\pgfqpoint{0.000000in}{-0.048611in}}%
\pgfusepath{stroke,fill}%
}%
\begin{pgfscope}%
\pgfsys@transformshift{2.124034in}{0.528000in}%
\pgfsys@useobject{currentmarker}{}%
\end{pgfscope}%
\end{pgfscope}%
\begin{pgfscope}%
\pgftext[x=2.124034in,y=0.430778in,,top]{\sffamily\fontsize{10.000000}{12.000000}\selectfont -50}%
\end{pgfscope}%
\begin{pgfscope}%
\pgfpathrectangle{\pgfqpoint{0.800000in}{0.528000in}}{\pgfqpoint{4.960000in}{1.680000in}} %
\pgfusepath{clip}%
\pgfsetrectcap%
\pgfsetroundjoin%
\pgfsetlinewidth{0.803000pt}%
\definecolor{currentstroke}{rgb}{0.690196,0.690196,0.690196}%
\pgfsetstrokecolor{currentstroke}%
\pgfsetdash{}{0pt}%
\pgfpathmoveto{\pgfqpoint{2.702234in}{0.528000in}}%
\pgfpathlineto{\pgfqpoint{2.702234in}{2.208000in}}%
\pgfusepath{stroke}%
\end{pgfscope}%
\begin{pgfscope}%
\pgfsetbuttcap%
\pgfsetroundjoin%
\definecolor{currentfill}{rgb}{0.000000,0.000000,0.000000}%
\pgfsetfillcolor{currentfill}%
\pgfsetlinewidth{0.803000pt}%
\definecolor{currentstroke}{rgb}{0.000000,0.000000,0.000000}%
\pgfsetstrokecolor{currentstroke}%
\pgfsetdash{}{0pt}%
\pgfsys@defobject{currentmarker}{\pgfqpoint{0.000000in}{-0.048611in}}{\pgfqpoint{0.000000in}{0.000000in}}{%
\pgfpathmoveto{\pgfqpoint{0.000000in}{0.000000in}}%
\pgfpathlineto{\pgfqpoint{0.000000in}{-0.048611in}}%
\pgfusepath{stroke,fill}%
}%
\begin{pgfscope}%
\pgfsys@transformshift{2.702234in}{0.528000in}%
\pgfsys@useobject{currentmarker}{}%
\end{pgfscope}%
\end{pgfscope}%
\begin{pgfscope}%
\pgftext[x=2.702234in,y=0.430778in,,top]{\sffamily\fontsize{10.000000}{12.000000}\selectfont -25}%
\end{pgfscope}%
\begin{pgfscope}%
\pgfpathrectangle{\pgfqpoint{0.800000in}{0.528000in}}{\pgfqpoint{4.960000in}{1.680000in}} %
\pgfusepath{clip}%
\pgfsetrectcap%
\pgfsetroundjoin%
\pgfsetlinewidth{0.803000pt}%
\definecolor{currentstroke}{rgb}{0.690196,0.690196,0.690196}%
\pgfsetstrokecolor{currentstroke}%
\pgfsetdash{}{0pt}%
\pgfpathmoveto{\pgfqpoint{3.280434in}{0.528000in}}%
\pgfpathlineto{\pgfqpoint{3.280434in}{2.208000in}}%
\pgfusepath{stroke}%
\end{pgfscope}%
\begin{pgfscope}%
\pgfsetbuttcap%
\pgfsetroundjoin%
\definecolor{currentfill}{rgb}{0.000000,0.000000,0.000000}%
\pgfsetfillcolor{currentfill}%
\pgfsetlinewidth{0.803000pt}%
\definecolor{currentstroke}{rgb}{0.000000,0.000000,0.000000}%
\pgfsetstrokecolor{currentstroke}%
\pgfsetdash{}{0pt}%
\pgfsys@defobject{currentmarker}{\pgfqpoint{0.000000in}{-0.048611in}}{\pgfqpoint{0.000000in}{0.000000in}}{%
\pgfpathmoveto{\pgfqpoint{0.000000in}{0.000000in}}%
\pgfpathlineto{\pgfqpoint{0.000000in}{-0.048611in}}%
\pgfusepath{stroke,fill}%
}%
\begin{pgfscope}%
\pgfsys@transformshift{3.280434in}{0.528000in}%
\pgfsys@useobject{currentmarker}{}%
\end{pgfscope}%
\end{pgfscope}%
\begin{pgfscope}%
\pgftext[x=3.280434in,y=0.430778in,,top]{\sffamily\fontsize{10.000000}{12.000000}\selectfont 0}%
\end{pgfscope}%
\begin{pgfscope}%
\pgfpathrectangle{\pgfqpoint{0.800000in}{0.528000in}}{\pgfqpoint{4.960000in}{1.680000in}} %
\pgfusepath{clip}%
\pgfsetrectcap%
\pgfsetroundjoin%
\pgfsetlinewidth{0.803000pt}%
\definecolor{currentstroke}{rgb}{0.690196,0.690196,0.690196}%
\pgfsetstrokecolor{currentstroke}%
\pgfsetdash{}{0pt}%
\pgfpathmoveto{\pgfqpoint{3.858633in}{0.528000in}}%
\pgfpathlineto{\pgfqpoint{3.858633in}{2.208000in}}%
\pgfusepath{stroke}%
\end{pgfscope}%
\begin{pgfscope}%
\pgfsetbuttcap%
\pgfsetroundjoin%
\definecolor{currentfill}{rgb}{0.000000,0.000000,0.000000}%
\pgfsetfillcolor{currentfill}%
\pgfsetlinewidth{0.803000pt}%
\definecolor{currentstroke}{rgb}{0.000000,0.000000,0.000000}%
\pgfsetstrokecolor{currentstroke}%
\pgfsetdash{}{0pt}%
\pgfsys@defobject{currentmarker}{\pgfqpoint{0.000000in}{-0.048611in}}{\pgfqpoint{0.000000in}{0.000000in}}{%
\pgfpathmoveto{\pgfqpoint{0.000000in}{0.000000in}}%
\pgfpathlineto{\pgfqpoint{0.000000in}{-0.048611in}}%
\pgfusepath{stroke,fill}%
}%
\begin{pgfscope}%
\pgfsys@transformshift{3.858633in}{0.528000in}%
\pgfsys@useobject{currentmarker}{}%
\end{pgfscope}%
\end{pgfscope}%
\begin{pgfscope}%
\pgftext[x=3.858633in,y=0.430778in,,top]{\sffamily\fontsize{10.000000}{12.000000}\selectfont 25}%
\end{pgfscope}%
\begin{pgfscope}%
\pgfpathrectangle{\pgfqpoint{0.800000in}{0.528000in}}{\pgfqpoint{4.960000in}{1.680000in}} %
\pgfusepath{clip}%
\pgfsetrectcap%
\pgfsetroundjoin%
\pgfsetlinewidth{0.803000pt}%
\definecolor{currentstroke}{rgb}{0.690196,0.690196,0.690196}%
\pgfsetstrokecolor{currentstroke}%
\pgfsetdash{}{0pt}%
\pgfpathmoveto{\pgfqpoint{4.436833in}{0.528000in}}%
\pgfpathlineto{\pgfqpoint{4.436833in}{2.208000in}}%
\pgfusepath{stroke}%
\end{pgfscope}%
\begin{pgfscope}%
\pgfsetbuttcap%
\pgfsetroundjoin%
\definecolor{currentfill}{rgb}{0.000000,0.000000,0.000000}%
\pgfsetfillcolor{currentfill}%
\pgfsetlinewidth{0.803000pt}%
\definecolor{currentstroke}{rgb}{0.000000,0.000000,0.000000}%
\pgfsetstrokecolor{currentstroke}%
\pgfsetdash{}{0pt}%
\pgfsys@defobject{currentmarker}{\pgfqpoint{0.000000in}{-0.048611in}}{\pgfqpoint{0.000000in}{0.000000in}}{%
\pgfpathmoveto{\pgfqpoint{0.000000in}{0.000000in}}%
\pgfpathlineto{\pgfqpoint{0.000000in}{-0.048611in}}%
\pgfusepath{stroke,fill}%
}%
\begin{pgfscope}%
\pgfsys@transformshift{4.436833in}{0.528000in}%
\pgfsys@useobject{currentmarker}{}%
\end{pgfscope}%
\end{pgfscope}%
\begin{pgfscope}%
\pgftext[x=4.436833in,y=0.430778in,,top]{\sffamily\fontsize{10.000000}{12.000000}\selectfont 50}%
\end{pgfscope}%
\begin{pgfscope}%
\pgfpathrectangle{\pgfqpoint{0.800000in}{0.528000in}}{\pgfqpoint{4.960000in}{1.680000in}} %
\pgfusepath{clip}%
\pgfsetrectcap%
\pgfsetroundjoin%
\pgfsetlinewidth{0.803000pt}%
\definecolor{currentstroke}{rgb}{0.690196,0.690196,0.690196}%
\pgfsetstrokecolor{currentstroke}%
\pgfsetdash{}{0pt}%
\pgfpathmoveto{\pgfqpoint{5.015033in}{0.528000in}}%
\pgfpathlineto{\pgfqpoint{5.015033in}{2.208000in}}%
\pgfusepath{stroke}%
\end{pgfscope}%
\begin{pgfscope}%
\pgfsetbuttcap%
\pgfsetroundjoin%
\definecolor{currentfill}{rgb}{0.000000,0.000000,0.000000}%
\pgfsetfillcolor{currentfill}%
\pgfsetlinewidth{0.803000pt}%
\definecolor{currentstroke}{rgb}{0.000000,0.000000,0.000000}%
\pgfsetstrokecolor{currentstroke}%
\pgfsetdash{}{0pt}%
\pgfsys@defobject{currentmarker}{\pgfqpoint{0.000000in}{-0.048611in}}{\pgfqpoint{0.000000in}{0.000000in}}{%
\pgfpathmoveto{\pgfqpoint{0.000000in}{0.000000in}}%
\pgfpathlineto{\pgfqpoint{0.000000in}{-0.048611in}}%
\pgfusepath{stroke,fill}%
}%
\begin{pgfscope}%
\pgfsys@transformshift{5.015033in}{0.528000in}%
\pgfsys@useobject{currentmarker}{}%
\end{pgfscope}%
\end{pgfscope}%
\begin{pgfscope}%
\pgftext[x=5.015033in,y=0.430778in,,top]{\sffamily\fontsize{10.000000}{12.000000}\selectfont 75}%
\end{pgfscope}%
\begin{pgfscope}%
\pgfpathrectangle{\pgfqpoint{0.800000in}{0.528000in}}{\pgfqpoint{4.960000in}{1.680000in}} %
\pgfusepath{clip}%
\pgfsetrectcap%
\pgfsetroundjoin%
\pgfsetlinewidth{0.803000pt}%
\definecolor{currentstroke}{rgb}{0.690196,0.690196,0.690196}%
\pgfsetstrokecolor{currentstroke}%
\pgfsetdash{}{0pt}%
\pgfpathmoveto{\pgfqpoint{5.593233in}{0.528000in}}%
\pgfpathlineto{\pgfqpoint{5.593233in}{2.208000in}}%
\pgfusepath{stroke}%
\end{pgfscope}%
\begin{pgfscope}%
\pgfsetbuttcap%
\pgfsetroundjoin%
\definecolor{currentfill}{rgb}{0.000000,0.000000,0.000000}%
\pgfsetfillcolor{currentfill}%
\pgfsetlinewidth{0.803000pt}%
\definecolor{currentstroke}{rgb}{0.000000,0.000000,0.000000}%
\pgfsetstrokecolor{currentstroke}%
\pgfsetdash{}{0pt}%
\pgfsys@defobject{currentmarker}{\pgfqpoint{0.000000in}{-0.048611in}}{\pgfqpoint{0.000000in}{0.000000in}}{%
\pgfpathmoveto{\pgfqpoint{0.000000in}{0.000000in}}%
\pgfpathlineto{\pgfqpoint{0.000000in}{-0.048611in}}%
\pgfusepath{stroke,fill}%
}%
\begin{pgfscope}%
\pgfsys@transformshift{5.593233in}{0.528000in}%
\pgfsys@useobject{currentmarker}{}%
\end{pgfscope}%
\end{pgfscope}%
\begin{pgfscope}%
\pgftext[x=5.593233in,y=0.430778in,,top]{\sffamily\fontsize{10.000000}{12.000000}\selectfont 100}%
\end{pgfscope}%
\begin{pgfscope}%
\pgftext[x=3.280000in,y=0.240809in,,top]{\sffamily\fontsize{10.000000}{12.000000}\selectfont Frequency [MHz]}%
\end{pgfscope}%
\begin{pgfscope}%
\pgfpathrectangle{\pgfqpoint{0.800000in}{0.528000in}}{\pgfqpoint{4.960000in}{1.680000in}} %
\pgfusepath{clip}%
\pgfsetrectcap%
\pgfsetroundjoin%
\pgfsetlinewidth{0.803000pt}%
\definecolor{currentstroke}{rgb}{0.690196,0.690196,0.690196}%
\pgfsetstrokecolor{currentstroke}%
\pgfsetdash{}{0pt}%
\pgfpathmoveto{\pgfqpoint{0.800000in}{0.600050in}}%
\pgfpathlineto{\pgfqpoint{5.760000in}{0.600050in}}%
\pgfusepath{stroke}%
\end{pgfscope}%
\begin{pgfscope}%
\pgfsetbuttcap%
\pgfsetroundjoin%
\definecolor{currentfill}{rgb}{0.000000,0.000000,0.000000}%
\pgfsetfillcolor{currentfill}%
\pgfsetlinewidth{0.803000pt}%
\definecolor{currentstroke}{rgb}{0.000000,0.000000,0.000000}%
\pgfsetstrokecolor{currentstroke}%
\pgfsetdash{}{0pt}%
\pgfsys@defobject{currentmarker}{\pgfqpoint{-0.048611in}{0.000000in}}{\pgfqpoint{0.000000in}{0.000000in}}{%
\pgfpathmoveto{\pgfqpoint{0.000000in}{0.000000in}}%
\pgfpathlineto{\pgfqpoint{-0.048611in}{0.000000in}}%
\pgfusepath{stroke,fill}%
}%
\begin{pgfscope}%
\pgfsys@transformshift{0.800000in}{0.600050in}%
\pgfsys@useobject{currentmarker}{}%
\end{pgfscope}%
\end{pgfscope}%
\begin{pgfscope}%
\pgftext[x=0.481898in,y=0.547289in,left,base]{\sffamily\fontsize{10.000000}{12.000000}\selectfont 0.0}%
\end{pgfscope}%
\begin{pgfscope}%
\pgfpathrectangle{\pgfqpoint{0.800000in}{0.528000in}}{\pgfqpoint{4.960000in}{1.680000in}} %
\pgfusepath{clip}%
\pgfsetrectcap%
\pgfsetroundjoin%
\pgfsetlinewidth{0.803000pt}%
\definecolor{currentstroke}{rgb}{0.690196,0.690196,0.690196}%
\pgfsetstrokecolor{currentstroke}%
\pgfsetdash{}{0pt}%
\pgfpathmoveto{\pgfqpoint{0.800000in}{0.906367in}}%
\pgfpathlineto{\pgfqpoint{5.760000in}{0.906367in}}%
\pgfusepath{stroke}%
\end{pgfscope}%
\begin{pgfscope}%
\pgfsetbuttcap%
\pgfsetroundjoin%
\definecolor{currentfill}{rgb}{0.000000,0.000000,0.000000}%
\pgfsetfillcolor{currentfill}%
\pgfsetlinewidth{0.803000pt}%
\definecolor{currentstroke}{rgb}{0.000000,0.000000,0.000000}%
\pgfsetstrokecolor{currentstroke}%
\pgfsetdash{}{0pt}%
\pgfsys@defobject{currentmarker}{\pgfqpoint{-0.048611in}{0.000000in}}{\pgfqpoint{0.000000in}{0.000000in}}{%
\pgfpathmoveto{\pgfqpoint{0.000000in}{0.000000in}}%
\pgfpathlineto{\pgfqpoint{-0.048611in}{0.000000in}}%
\pgfusepath{stroke,fill}%
}%
\begin{pgfscope}%
\pgfsys@transformshift{0.800000in}{0.906367in}%
\pgfsys@useobject{currentmarker}{}%
\end{pgfscope}%
\end{pgfscope}%
\begin{pgfscope}%
\pgftext[x=0.481898in,y=0.853606in,left,base]{\sffamily\fontsize{10.000000}{12.000000}\selectfont 0.2}%
\end{pgfscope}%
\begin{pgfscope}%
\pgfpathrectangle{\pgfqpoint{0.800000in}{0.528000in}}{\pgfqpoint{4.960000in}{1.680000in}} %
\pgfusepath{clip}%
\pgfsetrectcap%
\pgfsetroundjoin%
\pgfsetlinewidth{0.803000pt}%
\definecolor{currentstroke}{rgb}{0.690196,0.690196,0.690196}%
\pgfsetstrokecolor{currentstroke}%
\pgfsetdash{}{0pt}%
\pgfpathmoveto{\pgfqpoint{0.800000in}{1.212685in}}%
\pgfpathlineto{\pgfqpoint{5.760000in}{1.212685in}}%
\pgfusepath{stroke}%
\end{pgfscope}%
\begin{pgfscope}%
\pgfsetbuttcap%
\pgfsetroundjoin%
\definecolor{currentfill}{rgb}{0.000000,0.000000,0.000000}%
\pgfsetfillcolor{currentfill}%
\pgfsetlinewidth{0.803000pt}%
\definecolor{currentstroke}{rgb}{0.000000,0.000000,0.000000}%
\pgfsetstrokecolor{currentstroke}%
\pgfsetdash{}{0pt}%
\pgfsys@defobject{currentmarker}{\pgfqpoint{-0.048611in}{0.000000in}}{\pgfqpoint{0.000000in}{0.000000in}}{%
\pgfpathmoveto{\pgfqpoint{0.000000in}{0.000000in}}%
\pgfpathlineto{\pgfqpoint{-0.048611in}{0.000000in}}%
\pgfusepath{stroke,fill}%
}%
\begin{pgfscope}%
\pgfsys@transformshift{0.800000in}{1.212685in}%
\pgfsys@useobject{currentmarker}{}%
\end{pgfscope}%
\end{pgfscope}%
\begin{pgfscope}%
\pgftext[x=0.481898in,y=1.159923in,left,base]{\sffamily\fontsize{10.000000}{12.000000}\selectfont 0.4}%
\end{pgfscope}%
\begin{pgfscope}%
\pgfpathrectangle{\pgfqpoint{0.800000in}{0.528000in}}{\pgfqpoint{4.960000in}{1.680000in}} %
\pgfusepath{clip}%
\pgfsetrectcap%
\pgfsetroundjoin%
\pgfsetlinewidth{0.803000pt}%
\definecolor{currentstroke}{rgb}{0.690196,0.690196,0.690196}%
\pgfsetstrokecolor{currentstroke}%
\pgfsetdash{}{0pt}%
\pgfpathmoveto{\pgfqpoint{0.800000in}{1.519002in}}%
\pgfpathlineto{\pgfqpoint{5.760000in}{1.519002in}}%
\pgfusepath{stroke}%
\end{pgfscope}%
\begin{pgfscope}%
\pgfsetbuttcap%
\pgfsetroundjoin%
\definecolor{currentfill}{rgb}{0.000000,0.000000,0.000000}%
\pgfsetfillcolor{currentfill}%
\pgfsetlinewidth{0.803000pt}%
\definecolor{currentstroke}{rgb}{0.000000,0.000000,0.000000}%
\pgfsetstrokecolor{currentstroke}%
\pgfsetdash{}{0pt}%
\pgfsys@defobject{currentmarker}{\pgfqpoint{-0.048611in}{0.000000in}}{\pgfqpoint{0.000000in}{0.000000in}}{%
\pgfpathmoveto{\pgfqpoint{0.000000in}{0.000000in}}%
\pgfpathlineto{\pgfqpoint{-0.048611in}{0.000000in}}%
\pgfusepath{stroke,fill}%
}%
\begin{pgfscope}%
\pgfsys@transformshift{0.800000in}{1.519002in}%
\pgfsys@useobject{currentmarker}{}%
\end{pgfscope}%
\end{pgfscope}%
\begin{pgfscope}%
\pgftext[x=0.481898in,y=1.466240in,left,base]{\sffamily\fontsize{10.000000}{12.000000}\selectfont 0.6}%
\end{pgfscope}%
\begin{pgfscope}%
\pgfpathrectangle{\pgfqpoint{0.800000in}{0.528000in}}{\pgfqpoint{4.960000in}{1.680000in}} %
\pgfusepath{clip}%
\pgfsetrectcap%
\pgfsetroundjoin%
\pgfsetlinewidth{0.803000pt}%
\definecolor{currentstroke}{rgb}{0.690196,0.690196,0.690196}%
\pgfsetstrokecolor{currentstroke}%
\pgfsetdash{}{0pt}%
\pgfpathmoveto{\pgfqpoint{0.800000in}{1.825319in}}%
\pgfpathlineto{\pgfqpoint{5.760000in}{1.825319in}}%
\pgfusepath{stroke}%
\end{pgfscope}%
\begin{pgfscope}%
\pgfsetbuttcap%
\pgfsetroundjoin%
\definecolor{currentfill}{rgb}{0.000000,0.000000,0.000000}%
\pgfsetfillcolor{currentfill}%
\pgfsetlinewidth{0.803000pt}%
\definecolor{currentstroke}{rgb}{0.000000,0.000000,0.000000}%
\pgfsetstrokecolor{currentstroke}%
\pgfsetdash{}{0pt}%
\pgfsys@defobject{currentmarker}{\pgfqpoint{-0.048611in}{0.000000in}}{\pgfqpoint{0.000000in}{0.000000in}}{%
\pgfpathmoveto{\pgfqpoint{0.000000in}{0.000000in}}%
\pgfpathlineto{\pgfqpoint{-0.048611in}{0.000000in}}%
\pgfusepath{stroke,fill}%
}%
\begin{pgfscope}%
\pgfsys@transformshift{0.800000in}{1.825319in}%
\pgfsys@useobject{currentmarker}{}%
\end{pgfscope}%
\end{pgfscope}%
\begin{pgfscope}%
\pgftext[x=0.481898in,y=1.772558in,left,base]{\sffamily\fontsize{10.000000}{12.000000}\selectfont 0.8}%
\end{pgfscope}%
\begin{pgfscope}%
\pgfpathrectangle{\pgfqpoint{0.800000in}{0.528000in}}{\pgfqpoint{4.960000in}{1.680000in}} %
\pgfusepath{clip}%
\pgfsetrectcap%
\pgfsetroundjoin%
\pgfsetlinewidth{0.803000pt}%
\definecolor{currentstroke}{rgb}{0.690196,0.690196,0.690196}%
\pgfsetstrokecolor{currentstroke}%
\pgfsetdash{}{0pt}%
\pgfpathmoveto{\pgfqpoint{0.800000in}{2.131636in}}%
\pgfpathlineto{\pgfqpoint{5.760000in}{2.131636in}}%
\pgfusepath{stroke}%
\end{pgfscope}%
\begin{pgfscope}%
\pgfsetbuttcap%
\pgfsetroundjoin%
\definecolor{currentfill}{rgb}{0.000000,0.000000,0.000000}%
\pgfsetfillcolor{currentfill}%
\pgfsetlinewidth{0.803000pt}%
\definecolor{currentstroke}{rgb}{0.000000,0.000000,0.000000}%
\pgfsetstrokecolor{currentstroke}%
\pgfsetdash{}{0pt}%
\pgfsys@defobject{currentmarker}{\pgfqpoint{-0.048611in}{0.000000in}}{\pgfqpoint{0.000000in}{0.000000in}}{%
\pgfpathmoveto{\pgfqpoint{0.000000in}{0.000000in}}%
\pgfpathlineto{\pgfqpoint{-0.048611in}{0.000000in}}%
\pgfusepath{stroke,fill}%
}%
\begin{pgfscope}%
\pgfsys@transformshift{0.800000in}{2.131636in}%
\pgfsys@useobject{currentmarker}{}%
\end{pgfscope}%
\end{pgfscope}%
\begin{pgfscope}%
\pgftext[x=0.481898in,y=2.078875in,left,base]{\sffamily\fontsize{10.000000}{12.000000}\selectfont 1.0}%
\end{pgfscope}%
\begin{pgfscope}%
\pgftext[x=0.426343in,y=1.368000in,,bottom,rotate=90.000000]{\sffamily\fontsize{10.000000}{12.000000}\selectfont Magnitude of the Fourier transform.}%
\end{pgfscope}%
\begin{pgfscope}%
\pgfpathrectangle{\pgfqpoint{0.800000in}{0.528000in}}{\pgfqpoint{4.960000in}{1.680000in}} %
\pgfusepath{clip}%
\pgfsetrectcap%
\pgfsetroundjoin%
\pgfsetlinewidth{1.505625pt}%
\definecolor{currentstroke}{rgb}{0.121569,0.466667,0.705882}%
\pgfsetstrokecolor{currentstroke}%
\pgfsetdash{}{0pt}%
\pgfpathmoveto{\pgfqpoint{1.025455in}{1.041043in}}%
\pgfpathlineto{\pgfqpoint{1.026322in}{0.950947in}}%
\pgfpathlineto{\pgfqpoint{1.027189in}{0.944405in}}%
\pgfpathlineto{\pgfqpoint{1.028923in}{1.212427in}}%
\pgfpathlineto{\pgfqpoint{1.029790in}{0.899455in}}%
\pgfpathlineto{\pgfqpoint{1.030657in}{1.300207in}}%
\pgfpathlineto{\pgfqpoint{1.031524in}{0.886146in}}%
\pgfpathlineto{\pgfqpoint{1.033259in}{1.630134in}}%
\pgfpathlineto{\pgfqpoint{1.034993in}{0.789860in}}%
\pgfpathlineto{\pgfqpoint{1.035860in}{1.386129in}}%
\pgfpathlineto{\pgfqpoint{1.037594in}{0.961971in}}%
\pgfpathlineto{\pgfqpoint{1.038462in}{1.002391in}}%
\pgfpathlineto{\pgfqpoint{1.039329in}{1.144614in}}%
\pgfpathlineto{\pgfqpoint{1.041063in}{0.755967in}}%
\pgfpathlineto{\pgfqpoint{1.041930in}{0.711727in}}%
\pgfpathlineto{\pgfqpoint{1.042797in}{1.347520in}}%
\pgfpathlineto{\pgfqpoint{1.043664in}{0.772167in}}%
\pgfpathlineto{\pgfqpoint{1.045399in}{1.150314in}}%
\pgfpathlineto{\pgfqpoint{1.046266in}{1.278582in}}%
\pgfpathlineto{\pgfqpoint{1.047133in}{1.803990in}}%
\pgfpathlineto{\pgfqpoint{1.048000in}{0.991578in}}%
\pgfpathlineto{\pgfqpoint{1.048867in}{1.063559in}}%
\pgfpathlineto{\pgfqpoint{1.049734in}{0.750557in}}%
\pgfpathlineto{\pgfqpoint{1.050601in}{1.479783in}}%
\pgfpathlineto{\pgfqpoint{1.052336in}{0.749795in}}%
\pgfpathlineto{\pgfqpoint{1.053203in}{0.912919in}}%
\pgfpathlineto{\pgfqpoint{1.054070in}{0.779888in}}%
\pgfpathlineto{\pgfqpoint{1.054937in}{1.322009in}}%
\pgfpathlineto{\pgfqpoint{1.055804in}{0.936826in}}%
\pgfpathlineto{\pgfqpoint{1.057538in}{1.159057in}}%
\pgfpathlineto{\pgfqpoint{1.060140in}{0.808448in}}%
\pgfpathlineto{\pgfqpoint{1.061007in}{0.919147in}}%
\pgfpathlineto{\pgfqpoint{1.062741in}{0.787335in}}%
\pgfpathlineto{\pgfqpoint{1.063608in}{0.880587in}}%
\pgfpathlineto{\pgfqpoint{1.064476in}{1.829226in}}%
\pgfpathlineto{\pgfqpoint{1.065343in}{1.113686in}}%
\pgfpathlineto{\pgfqpoint{1.066210in}{1.116630in}}%
\pgfpathlineto{\pgfqpoint{1.067944in}{0.816795in}}%
\pgfpathlineto{\pgfqpoint{1.068811in}{0.877179in}}%
\pgfpathlineto{\pgfqpoint{1.069678in}{1.565074in}}%
\pgfpathlineto{\pgfqpoint{1.071413in}{0.964462in}}%
\pgfpathlineto{\pgfqpoint{1.072280in}{0.693894in}}%
\pgfpathlineto{\pgfqpoint{1.073147in}{1.376838in}}%
\pgfpathlineto{\pgfqpoint{1.074014in}{0.952315in}}%
\pgfpathlineto{\pgfqpoint{1.074881in}{1.102667in}}%
\pgfpathlineto{\pgfqpoint{1.075748in}{0.842865in}}%
\pgfpathlineto{\pgfqpoint{1.078350in}{1.651365in}}%
\pgfpathlineto{\pgfqpoint{1.080951in}{0.753325in}}%
\pgfpathlineto{\pgfqpoint{1.082685in}{1.226766in}}%
\pgfpathlineto{\pgfqpoint{1.085287in}{0.699483in}}%
\pgfpathlineto{\pgfqpoint{1.087021in}{1.097641in}}%
\pgfpathlineto{\pgfqpoint{1.087888in}{1.063710in}}%
\pgfpathlineto{\pgfqpoint{1.088755in}{1.255258in}}%
\pgfpathlineto{\pgfqpoint{1.089622in}{1.113839in}}%
\pgfpathlineto{\pgfqpoint{1.090490in}{0.754361in}}%
\pgfpathlineto{\pgfqpoint{1.092224in}{1.870994in}}%
\pgfpathlineto{\pgfqpoint{1.093091in}{0.895020in}}%
\pgfpathlineto{\pgfqpoint{1.093958in}{0.999849in}}%
\pgfpathlineto{\pgfqpoint{1.094825in}{0.863888in}}%
\pgfpathlineto{\pgfqpoint{1.095692in}{0.922919in}}%
\pgfpathlineto{\pgfqpoint{1.096559in}{0.710054in}}%
\pgfpathlineto{\pgfqpoint{1.097427in}{0.858994in}}%
\pgfpathlineto{\pgfqpoint{1.098294in}{0.668202in}}%
\pgfpathlineto{\pgfqpoint{1.100028in}{1.029811in}}%
\pgfpathlineto{\pgfqpoint{1.100895in}{0.738236in}}%
\pgfpathlineto{\pgfqpoint{1.101762in}{0.783589in}}%
\pgfpathlineto{\pgfqpoint{1.102629in}{0.747950in}}%
\pgfpathlineto{\pgfqpoint{1.103497in}{1.084187in}}%
\pgfpathlineto{\pgfqpoint{1.104364in}{1.028105in}}%
\pgfpathlineto{\pgfqpoint{1.105231in}{0.796075in}}%
\pgfpathlineto{\pgfqpoint{1.106965in}{0.995130in}}%
\pgfpathlineto{\pgfqpoint{1.107832in}{0.748389in}}%
\pgfpathlineto{\pgfqpoint{1.108699in}{0.800594in}}%
\pgfpathlineto{\pgfqpoint{1.109566in}{1.308110in}}%
\pgfpathlineto{\pgfqpoint{1.113035in}{0.702116in}}%
\pgfpathlineto{\pgfqpoint{1.113902in}{1.553421in}}%
\pgfpathlineto{\pgfqpoint{1.115636in}{0.863681in}}%
\pgfpathlineto{\pgfqpoint{1.117371in}{1.298871in}}%
\pgfpathlineto{\pgfqpoint{1.118238in}{0.854943in}}%
\pgfpathlineto{\pgfqpoint{1.119105in}{0.924864in}}%
\pgfpathlineto{\pgfqpoint{1.119972in}{1.099244in}}%
\pgfpathlineto{\pgfqpoint{1.120839in}{0.724924in}}%
\pgfpathlineto{\pgfqpoint{1.121706in}{0.961344in}}%
\pgfpathlineto{\pgfqpoint{1.124308in}{0.794350in}}%
\pgfpathlineto{\pgfqpoint{1.125175in}{0.915592in}}%
\pgfpathlineto{\pgfqpoint{1.126042in}{1.243200in}}%
\pgfpathlineto{\pgfqpoint{1.126909in}{1.184017in}}%
\pgfpathlineto{\pgfqpoint{1.127776in}{1.383996in}}%
\pgfpathlineto{\pgfqpoint{1.129510in}{0.790002in}}%
\pgfpathlineto{\pgfqpoint{1.130378in}{0.964329in}}%
\pgfpathlineto{\pgfqpoint{1.131245in}{0.797252in}}%
\pgfpathlineto{\pgfqpoint{1.132112in}{1.448262in}}%
\pgfpathlineto{\pgfqpoint{1.132979in}{0.896466in}}%
\pgfpathlineto{\pgfqpoint{1.133846in}{1.100145in}}%
\pgfpathlineto{\pgfqpoint{1.134713in}{0.653246in}}%
\pgfpathlineto{\pgfqpoint{1.136448in}{1.277628in}}%
\pgfpathlineto{\pgfqpoint{1.139049in}{0.733472in}}%
\pgfpathlineto{\pgfqpoint{1.139916in}{0.991655in}}%
\pgfpathlineto{\pgfqpoint{1.141650in}{0.738158in}}%
\pgfpathlineto{\pgfqpoint{1.142517in}{1.110756in}}%
\pgfpathlineto{\pgfqpoint{1.145119in}{0.664349in}}%
\pgfpathlineto{\pgfqpoint{1.145986in}{1.256592in}}%
\pgfpathlineto{\pgfqpoint{1.146853in}{1.190877in}}%
\pgfpathlineto{\pgfqpoint{1.148587in}{0.917050in}}%
\pgfpathlineto{\pgfqpoint{1.149455in}{1.255833in}}%
\pgfpathlineto{\pgfqpoint{1.150322in}{1.151925in}}%
\pgfpathlineto{\pgfqpoint{1.151189in}{1.189249in}}%
\pgfpathlineto{\pgfqpoint{1.152056in}{0.876743in}}%
\pgfpathlineto{\pgfqpoint{1.152923in}{1.361634in}}%
\pgfpathlineto{\pgfqpoint{1.153790in}{0.833620in}}%
\pgfpathlineto{\pgfqpoint{1.154657in}{1.088658in}}%
\pgfpathlineto{\pgfqpoint{1.156392in}{0.625968in}}%
\pgfpathlineto{\pgfqpoint{1.158126in}{1.162924in}}%
\pgfpathlineto{\pgfqpoint{1.158993in}{0.963650in}}%
\pgfpathlineto{\pgfqpoint{1.159860in}{0.780162in}}%
\pgfpathlineto{\pgfqpoint{1.161594in}{1.053376in}}%
\pgfpathlineto{\pgfqpoint{1.162462in}{0.859862in}}%
\pgfpathlineto{\pgfqpoint{1.164196in}{1.063214in}}%
\pgfpathlineto{\pgfqpoint{1.165063in}{0.932272in}}%
\pgfpathlineto{\pgfqpoint{1.166797in}{1.097900in}}%
\pgfpathlineto{\pgfqpoint{1.167664in}{0.730525in}}%
\pgfpathlineto{\pgfqpoint{1.168531in}{1.670342in}}%
\pgfpathlineto{\pgfqpoint{1.171133in}{0.671970in}}%
\pgfpathlineto{\pgfqpoint{1.173734in}{1.466945in}}%
\pgfpathlineto{\pgfqpoint{1.175469in}{0.946719in}}%
\pgfpathlineto{\pgfqpoint{1.176336in}{1.682291in}}%
\pgfpathlineto{\pgfqpoint{1.177203in}{0.847113in}}%
\pgfpathlineto{\pgfqpoint{1.178070in}{1.132926in}}%
\pgfpathlineto{\pgfqpoint{1.178937in}{0.638634in}}%
\pgfpathlineto{\pgfqpoint{1.179804in}{1.048596in}}%
\pgfpathlineto{\pgfqpoint{1.180671in}{0.881782in}}%
\pgfpathlineto{\pgfqpoint{1.181538in}{1.232004in}}%
\pgfpathlineto{\pgfqpoint{1.182406in}{1.166184in}}%
\pgfpathlineto{\pgfqpoint{1.183273in}{0.809723in}}%
\pgfpathlineto{\pgfqpoint{1.184140in}{1.369685in}}%
\pgfpathlineto{\pgfqpoint{1.185007in}{1.282709in}}%
\pgfpathlineto{\pgfqpoint{1.185874in}{1.272767in}}%
\pgfpathlineto{\pgfqpoint{1.186741in}{1.014203in}}%
\pgfpathlineto{\pgfqpoint{1.187608in}{1.017178in}}%
\pgfpathlineto{\pgfqpoint{1.188476in}{0.970490in}}%
\pgfpathlineto{\pgfqpoint{1.189343in}{1.354412in}}%
\pgfpathlineto{\pgfqpoint{1.190210in}{0.793440in}}%
\pgfpathlineto{\pgfqpoint{1.191077in}{1.051358in}}%
\pgfpathlineto{\pgfqpoint{1.191944in}{0.852797in}}%
\pgfpathlineto{\pgfqpoint{1.192811in}{1.162247in}}%
\pgfpathlineto{\pgfqpoint{1.193678in}{0.634088in}}%
\pgfpathlineto{\pgfqpoint{1.194545in}{1.015709in}}%
\pgfpathlineto{\pgfqpoint{1.195413in}{0.823882in}}%
\pgfpathlineto{\pgfqpoint{1.197147in}{1.144890in}}%
\pgfpathlineto{\pgfqpoint{1.198881in}{0.680414in}}%
\pgfpathlineto{\pgfqpoint{1.199748in}{1.354002in}}%
\pgfpathlineto{\pgfqpoint{1.200615in}{0.848823in}}%
\pgfpathlineto{\pgfqpoint{1.201483in}{0.957370in}}%
\pgfpathlineto{\pgfqpoint{1.202350in}{0.756418in}}%
\pgfpathlineto{\pgfqpoint{1.203217in}{1.113629in}}%
\pgfpathlineto{\pgfqpoint{1.204084in}{0.899496in}}%
\pgfpathlineto{\pgfqpoint{1.204951in}{1.402778in}}%
\pgfpathlineto{\pgfqpoint{1.206685in}{1.006849in}}%
\pgfpathlineto{\pgfqpoint{1.207552in}{1.222123in}}%
\pgfpathlineto{\pgfqpoint{1.208420in}{1.185498in}}%
\pgfpathlineto{\pgfqpoint{1.209287in}{1.112534in}}%
\pgfpathlineto{\pgfqpoint{1.210154in}{0.749268in}}%
\pgfpathlineto{\pgfqpoint{1.211021in}{0.881942in}}%
\pgfpathlineto{\pgfqpoint{1.211888in}{1.223523in}}%
\pgfpathlineto{\pgfqpoint{1.213622in}{1.027926in}}%
\pgfpathlineto{\pgfqpoint{1.214490in}{0.994316in}}%
\pgfpathlineto{\pgfqpoint{1.216224in}{1.092912in}}%
\pgfpathlineto{\pgfqpoint{1.217091in}{0.932485in}}%
\pgfpathlineto{\pgfqpoint{1.217958in}{1.060272in}}%
\pgfpathlineto{\pgfqpoint{1.219692in}{0.858848in}}%
\pgfpathlineto{\pgfqpoint{1.221427in}{1.191803in}}%
\pgfpathlineto{\pgfqpoint{1.222294in}{0.766429in}}%
\pgfpathlineto{\pgfqpoint{1.224028in}{1.586156in}}%
\pgfpathlineto{\pgfqpoint{1.224895in}{0.813477in}}%
\pgfpathlineto{\pgfqpoint{1.225762in}{1.158213in}}%
\pgfpathlineto{\pgfqpoint{1.226629in}{0.837604in}}%
\pgfpathlineto{\pgfqpoint{1.227497in}{0.983041in}}%
\pgfpathlineto{\pgfqpoint{1.228364in}{1.370761in}}%
\pgfpathlineto{\pgfqpoint{1.229231in}{0.966735in}}%
\pgfpathlineto{\pgfqpoint{1.230098in}{1.060171in}}%
\pgfpathlineto{\pgfqpoint{1.230965in}{0.870829in}}%
\pgfpathlineto{\pgfqpoint{1.231832in}{1.367258in}}%
\pgfpathlineto{\pgfqpoint{1.232699in}{0.962520in}}%
\pgfpathlineto{\pgfqpoint{1.233566in}{1.153557in}}%
\pgfpathlineto{\pgfqpoint{1.234434in}{0.811179in}}%
\pgfpathlineto{\pgfqpoint{1.235301in}{1.190650in}}%
\pgfpathlineto{\pgfqpoint{1.237035in}{0.817321in}}%
\pgfpathlineto{\pgfqpoint{1.240503in}{1.136786in}}%
\pgfpathlineto{\pgfqpoint{1.241371in}{0.799478in}}%
\pgfpathlineto{\pgfqpoint{1.243972in}{1.191087in}}%
\pgfpathlineto{\pgfqpoint{1.244839in}{0.948205in}}%
\pgfpathlineto{\pgfqpoint{1.245706in}{1.203627in}}%
\pgfpathlineto{\pgfqpoint{1.246573in}{0.893190in}}%
\pgfpathlineto{\pgfqpoint{1.247441in}{1.034061in}}%
\pgfpathlineto{\pgfqpoint{1.248308in}{0.828324in}}%
\pgfpathlineto{\pgfqpoint{1.250909in}{1.283095in}}%
\pgfpathlineto{\pgfqpoint{1.251776in}{1.249212in}}%
\pgfpathlineto{\pgfqpoint{1.252643in}{0.841597in}}%
\pgfpathlineto{\pgfqpoint{1.253510in}{1.230236in}}%
\pgfpathlineto{\pgfqpoint{1.254378in}{0.921341in}}%
\pgfpathlineto{\pgfqpoint{1.255245in}{1.061847in}}%
\pgfpathlineto{\pgfqpoint{1.256112in}{0.749874in}}%
\pgfpathlineto{\pgfqpoint{1.256979in}{1.507940in}}%
\pgfpathlineto{\pgfqpoint{1.257846in}{0.859794in}}%
\pgfpathlineto{\pgfqpoint{1.258713in}{1.371259in}}%
\pgfpathlineto{\pgfqpoint{1.260448in}{0.923400in}}%
\pgfpathlineto{\pgfqpoint{1.261315in}{1.106662in}}%
\pgfpathlineto{\pgfqpoint{1.263049in}{0.760328in}}%
\pgfpathlineto{\pgfqpoint{1.263916in}{1.348506in}}%
\pgfpathlineto{\pgfqpoint{1.264783in}{1.291488in}}%
\pgfpathlineto{\pgfqpoint{1.265650in}{1.247865in}}%
\pgfpathlineto{\pgfqpoint{1.267385in}{1.322432in}}%
\pgfpathlineto{\pgfqpoint{1.269119in}{0.746148in}}%
\pgfpathlineto{\pgfqpoint{1.270853in}{1.412248in}}%
\pgfpathlineto{\pgfqpoint{1.271720in}{0.953195in}}%
\pgfpathlineto{\pgfqpoint{1.273455in}{1.428233in}}%
\pgfpathlineto{\pgfqpoint{1.274322in}{1.131649in}}%
\pgfpathlineto{\pgfqpoint{1.275189in}{1.296767in}}%
\pgfpathlineto{\pgfqpoint{1.276056in}{1.248876in}}%
\pgfpathlineto{\pgfqpoint{1.276923in}{0.827428in}}%
\pgfpathlineto{\pgfqpoint{1.278657in}{1.111037in}}%
\pgfpathlineto{\pgfqpoint{1.279524in}{0.788379in}}%
\pgfpathlineto{\pgfqpoint{1.280392in}{1.318267in}}%
\pgfpathlineto{\pgfqpoint{1.282993in}{0.845609in}}%
\pgfpathlineto{\pgfqpoint{1.283860in}{1.389041in}}%
\pgfpathlineto{\pgfqpoint{1.284727in}{1.299515in}}%
\pgfpathlineto{\pgfqpoint{1.286462in}{0.751434in}}%
\pgfpathlineto{\pgfqpoint{1.287329in}{1.241612in}}%
\pgfpathlineto{\pgfqpoint{1.289063in}{0.793580in}}%
\pgfpathlineto{\pgfqpoint{1.289930in}{1.126688in}}%
\pgfpathlineto{\pgfqpoint{1.291664in}{0.980545in}}%
\pgfpathlineto{\pgfqpoint{1.292531in}{0.982613in}}%
\pgfpathlineto{\pgfqpoint{1.293399in}{1.482349in}}%
\pgfpathlineto{\pgfqpoint{1.295133in}{1.108377in}}%
\pgfpathlineto{\pgfqpoint{1.296000in}{1.156391in}}%
\pgfpathlineto{\pgfqpoint{1.296867in}{1.383864in}}%
\pgfpathlineto{\pgfqpoint{1.297734in}{0.689413in}}%
\pgfpathlineto{\pgfqpoint{1.299469in}{1.283618in}}%
\pgfpathlineto{\pgfqpoint{1.300336in}{1.271753in}}%
\pgfpathlineto{\pgfqpoint{1.301203in}{1.164951in}}%
\pgfpathlineto{\pgfqpoint{1.302070in}{1.372379in}}%
\pgfpathlineto{\pgfqpoint{1.302937in}{1.343309in}}%
\pgfpathlineto{\pgfqpoint{1.303804in}{0.881657in}}%
\pgfpathlineto{\pgfqpoint{1.304671in}{0.980258in}}%
\pgfpathlineto{\pgfqpoint{1.306406in}{1.399817in}}%
\pgfpathlineto{\pgfqpoint{1.307273in}{1.280610in}}%
\pgfpathlineto{\pgfqpoint{1.308140in}{1.309314in}}%
\pgfpathlineto{\pgfqpoint{1.309007in}{1.479446in}}%
\pgfpathlineto{\pgfqpoint{1.311608in}{0.942472in}}%
\pgfpathlineto{\pgfqpoint{1.312476in}{1.278759in}}%
\pgfpathlineto{\pgfqpoint{1.313343in}{0.845476in}}%
\pgfpathlineto{\pgfqpoint{1.315077in}{1.327330in}}%
\pgfpathlineto{\pgfqpoint{1.315944in}{0.663559in}}%
\pgfpathlineto{\pgfqpoint{1.316811in}{0.791561in}}%
\pgfpathlineto{\pgfqpoint{1.318545in}{0.977383in}}%
\pgfpathlineto{\pgfqpoint{1.319413in}{0.670119in}}%
\pgfpathlineto{\pgfqpoint{1.321147in}{1.494048in}}%
\pgfpathlineto{\pgfqpoint{1.322014in}{1.157852in}}%
\pgfpathlineto{\pgfqpoint{1.322881in}{1.489716in}}%
\pgfpathlineto{\pgfqpoint{1.324615in}{1.032334in}}%
\pgfpathlineto{\pgfqpoint{1.325483in}{0.772661in}}%
\pgfpathlineto{\pgfqpoint{1.326350in}{1.868531in}}%
\pgfpathlineto{\pgfqpoint{1.328084in}{1.129579in}}%
\pgfpathlineto{\pgfqpoint{1.328951in}{1.075880in}}%
\pgfpathlineto{\pgfqpoint{1.329818in}{1.901663in}}%
\pgfpathlineto{\pgfqpoint{1.331552in}{0.971002in}}%
\pgfpathlineto{\pgfqpoint{1.333287in}{1.461074in}}%
\pgfpathlineto{\pgfqpoint{1.335021in}{0.955793in}}%
\pgfpathlineto{\pgfqpoint{1.335888in}{1.575326in}}%
\pgfpathlineto{\pgfqpoint{1.336755in}{1.398884in}}%
\pgfpathlineto{\pgfqpoint{1.337622in}{0.759701in}}%
\pgfpathlineto{\pgfqpoint{1.338490in}{0.978606in}}%
\pgfpathlineto{\pgfqpoint{1.339357in}{0.730740in}}%
\pgfpathlineto{\pgfqpoint{1.340224in}{0.934764in}}%
\pgfpathlineto{\pgfqpoint{1.341091in}{0.862809in}}%
\pgfpathlineto{\pgfqpoint{1.342825in}{1.264243in}}%
\pgfpathlineto{\pgfqpoint{1.343692in}{0.617704in}}%
\pgfpathlineto{\pgfqpoint{1.346294in}{1.458704in}}%
\pgfpathlineto{\pgfqpoint{1.348028in}{0.999970in}}%
\pgfpathlineto{\pgfqpoint{1.348895in}{1.146818in}}%
\pgfpathlineto{\pgfqpoint{1.349762in}{0.706683in}}%
\pgfpathlineto{\pgfqpoint{1.350629in}{1.005561in}}%
\pgfpathlineto{\pgfqpoint{1.351497in}{0.772850in}}%
\pgfpathlineto{\pgfqpoint{1.352364in}{1.158655in}}%
\pgfpathlineto{\pgfqpoint{1.353231in}{0.832061in}}%
\pgfpathlineto{\pgfqpoint{1.354098in}{0.925702in}}%
\pgfpathlineto{\pgfqpoint{1.354965in}{1.162060in}}%
\pgfpathlineto{\pgfqpoint{1.355832in}{0.699515in}}%
\pgfpathlineto{\pgfqpoint{1.356699in}{1.092847in}}%
\pgfpathlineto{\pgfqpoint{1.357566in}{0.748952in}}%
\pgfpathlineto{\pgfqpoint{1.358434in}{1.059072in}}%
\pgfpathlineto{\pgfqpoint{1.359301in}{1.037198in}}%
\pgfpathlineto{\pgfqpoint{1.360168in}{0.773627in}}%
\pgfpathlineto{\pgfqpoint{1.361902in}{1.383373in}}%
\pgfpathlineto{\pgfqpoint{1.362769in}{1.009376in}}%
\pgfpathlineto{\pgfqpoint{1.364503in}{1.516803in}}%
\pgfpathlineto{\pgfqpoint{1.365371in}{0.961529in}}%
\pgfpathlineto{\pgfqpoint{1.366238in}{1.658815in}}%
\pgfpathlineto{\pgfqpoint{1.367105in}{0.842003in}}%
\pgfpathlineto{\pgfqpoint{1.367972in}{1.001294in}}%
\pgfpathlineto{\pgfqpoint{1.368839in}{1.145433in}}%
\pgfpathlineto{\pgfqpoint{1.369706in}{0.864973in}}%
\pgfpathlineto{\pgfqpoint{1.370573in}{0.897807in}}%
\pgfpathlineto{\pgfqpoint{1.371441in}{1.545660in}}%
\pgfpathlineto{\pgfqpoint{1.372308in}{0.934602in}}%
\pgfpathlineto{\pgfqpoint{1.373175in}{1.279117in}}%
\pgfpathlineto{\pgfqpoint{1.374042in}{0.734647in}}%
\pgfpathlineto{\pgfqpoint{1.374909in}{0.827547in}}%
\pgfpathlineto{\pgfqpoint{1.375776in}{1.446017in}}%
\pgfpathlineto{\pgfqpoint{1.376643in}{0.689373in}}%
\pgfpathlineto{\pgfqpoint{1.378378in}{2.025812in}}%
\pgfpathlineto{\pgfqpoint{1.380979in}{0.872740in}}%
\pgfpathlineto{\pgfqpoint{1.381846in}{1.989043in}}%
\pgfpathlineto{\pgfqpoint{1.382713in}{0.772530in}}%
\pgfpathlineto{\pgfqpoint{1.383580in}{1.006137in}}%
\pgfpathlineto{\pgfqpoint{1.384448in}{1.368991in}}%
\pgfpathlineto{\pgfqpoint{1.385315in}{0.877438in}}%
\pgfpathlineto{\pgfqpoint{1.387916in}{1.398180in}}%
\pgfpathlineto{\pgfqpoint{1.388783in}{1.342972in}}%
\pgfpathlineto{\pgfqpoint{1.389650in}{1.307576in}}%
\pgfpathlineto{\pgfqpoint{1.390517in}{1.223275in}}%
\pgfpathlineto{\pgfqpoint{1.391385in}{1.505635in}}%
\pgfpathlineto{\pgfqpoint{1.392252in}{1.471170in}}%
\pgfpathlineto{\pgfqpoint{1.393119in}{1.485836in}}%
\pgfpathlineto{\pgfqpoint{1.394853in}{0.711346in}}%
\pgfpathlineto{\pgfqpoint{1.396587in}{1.513143in}}%
\pgfpathlineto{\pgfqpoint{1.397455in}{0.974073in}}%
\pgfpathlineto{\pgfqpoint{1.398322in}{1.156077in}}%
\pgfpathlineto{\pgfqpoint{1.399189in}{0.788434in}}%
\pgfpathlineto{\pgfqpoint{1.400923in}{1.308504in}}%
\pgfpathlineto{\pgfqpoint{1.402657in}{0.841224in}}%
\pgfpathlineto{\pgfqpoint{1.403524in}{1.139833in}}%
\pgfpathlineto{\pgfqpoint{1.404392in}{1.063371in}}%
\pgfpathlineto{\pgfqpoint{1.405259in}{0.850572in}}%
\pgfpathlineto{\pgfqpoint{1.407860in}{1.213562in}}%
\pgfpathlineto{\pgfqpoint{1.408727in}{0.918500in}}%
\pgfpathlineto{\pgfqpoint{1.409594in}{1.429883in}}%
\pgfpathlineto{\pgfqpoint{1.411329in}{0.748300in}}%
\pgfpathlineto{\pgfqpoint{1.412196in}{1.024716in}}%
\pgfpathlineto{\pgfqpoint{1.413063in}{0.725397in}}%
\pgfpathlineto{\pgfqpoint{1.413930in}{1.230156in}}%
\pgfpathlineto{\pgfqpoint{1.414797in}{0.915950in}}%
\pgfpathlineto{\pgfqpoint{1.415664in}{1.262095in}}%
\pgfpathlineto{\pgfqpoint{1.416531in}{0.839035in}}%
\pgfpathlineto{\pgfqpoint{1.417399in}{1.032383in}}%
\pgfpathlineto{\pgfqpoint{1.418266in}{0.868826in}}%
\pgfpathlineto{\pgfqpoint{1.419133in}{1.232183in}}%
\pgfpathlineto{\pgfqpoint{1.420000in}{1.089195in}}%
\pgfpathlineto{\pgfqpoint{1.420867in}{1.432812in}}%
\pgfpathlineto{\pgfqpoint{1.422601in}{1.199034in}}%
\pgfpathlineto{\pgfqpoint{1.423469in}{1.159367in}}%
\pgfpathlineto{\pgfqpoint{1.424336in}{1.007085in}}%
\pgfpathlineto{\pgfqpoint{1.425203in}{1.313243in}}%
\pgfpathlineto{\pgfqpoint{1.426070in}{0.908879in}}%
\pgfpathlineto{\pgfqpoint{1.427804in}{1.342002in}}%
\pgfpathlineto{\pgfqpoint{1.430406in}{0.935446in}}%
\pgfpathlineto{\pgfqpoint{1.431273in}{0.921529in}}%
\pgfpathlineto{\pgfqpoint{1.432140in}{0.850646in}}%
\pgfpathlineto{\pgfqpoint{1.433007in}{1.474904in}}%
\pgfpathlineto{\pgfqpoint{1.433874in}{0.936318in}}%
\pgfpathlineto{\pgfqpoint{1.434741in}{1.155165in}}%
\pgfpathlineto{\pgfqpoint{1.435608in}{0.900854in}}%
\pgfpathlineto{\pgfqpoint{1.436476in}{1.218445in}}%
\pgfpathlineto{\pgfqpoint{1.437343in}{0.631851in}}%
\pgfpathlineto{\pgfqpoint{1.439077in}{1.312619in}}%
\pgfpathlineto{\pgfqpoint{1.440811in}{0.781786in}}%
\pgfpathlineto{\pgfqpoint{1.444280in}{1.230920in}}%
\pgfpathlineto{\pgfqpoint{1.445147in}{1.146286in}}%
\pgfpathlineto{\pgfqpoint{1.446014in}{1.275535in}}%
\pgfpathlineto{\pgfqpoint{1.446881in}{0.782962in}}%
\pgfpathlineto{\pgfqpoint{1.447748in}{1.517820in}}%
\pgfpathlineto{\pgfqpoint{1.448615in}{1.191682in}}%
\pgfpathlineto{\pgfqpoint{1.449483in}{1.450195in}}%
\pgfpathlineto{\pgfqpoint{1.451217in}{0.912982in}}%
\pgfpathlineto{\pgfqpoint{1.452951in}{1.449270in}}%
\pgfpathlineto{\pgfqpoint{1.453818in}{0.735398in}}%
\pgfpathlineto{\pgfqpoint{1.455552in}{1.303886in}}%
\pgfpathlineto{\pgfqpoint{1.456420in}{0.639977in}}%
\pgfpathlineto{\pgfqpoint{1.457287in}{1.277561in}}%
\pgfpathlineto{\pgfqpoint{1.458154in}{1.079582in}}%
\pgfpathlineto{\pgfqpoint{1.459021in}{1.274087in}}%
\pgfpathlineto{\pgfqpoint{1.459888in}{1.036135in}}%
\pgfpathlineto{\pgfqpoint{1.461622in}{1.467844in}}%
\pgfpathlineto{\pgfqpoint{1.462490in}{1.481975in}}%
\pgfpathlineto{\pgfqpoint{1.465091in}{0.688797in}}%
\pgfpathlineto{\pgfqpoint{1.465958in}{0.783031in}}%
\pgfpathlineto{\pgfqpoint{1.466825in}{1.462488in}}%
\pgfpathlineto{\pgfqpoint{1.467692in}{1.087653in}}%
\pgfpathlineto{\pgfqpoint{1.468559in}{1.097490in}}%
\pgfpathlineto{\pgfqpoint{1.469427in}{0.873455in}}%
\pgfpathlineto{\pgfqpoint{1.471161in}{1.161638in}}%
\pgfpathlineto{\pgfqpoint{1.472028in}{0.866339in}}%
\pgfpathlineto{\pgfqpoint{1.472895in}{0.878540in}}%
\pgfpathlineto{\pgfqpoint{1.473762in}{1.003167in}}%
\pgfpathlineto{\pgfqpoint{1.474629in}{0.927404in}}%
\pgfpathlineto{\pgfqpoint{1.475497in}{0.963875in}}%
\pgfpathlineto{\pgfqpoint{1.477231in}{1.150082in}}%
\pgfpathlineto{\pgfqpoint{1.478098in}{0.948908in}}%
\pgfpathlineto{\pgfqpoint{1.478965in}{1.306705in}}%
\pgfpathlineto{\pgfqpoint{1.479832in}{0.831781in}}%
\pgfpathlineto{\pgfqpoint{1.480699in}{0.898298in}}%
\pgfpathlineto{\pgfqpoint{1.481566in}{0.676987in}}%
\pgfpathlineto{\pgfqpoint{1.483301in}{1.507284in}}%
\pgfpathlineto{\pgfqpoint{1.485035in}{0.884803in}}%
\pgfpathlineto{\pgfqpoint{1.485902in}{1.244498in}}%
\pgfpathlineto{\pgfqpoint{1.486769in}{0.990123in}}%
\pgfpathlineto{\pgfqpoint{1.487636in}{1.302077in}}%
\pgfpathlineto{\pgfqpoint{1.489371in}{0.745405in}}%
\pgfpathlineto{\pgfqpoint{1.491105in}{1.249600in}}%
\pgfpathlineto{\pgfqpoint{1.491972in}{0.725796in}}%
\pgfpathlineto{\pgfqpoint{1.492839in}{1.547362in}}%
\pgfpathlineto{\pgfqpoint{1.493706in}{1.162760in}}%
\pgfpathlineto{\pgfqpoint{1.494573in}{1.219982in}}%
\pgfpathlineto{\pgfqpoint{1.495441in}{1.176339in}}%
\pgfpathlineto{\pgfqpoint{1.496308in}{1.492792in}}%
\pgfpathlineto{\pgfqpoint{1.497175in}{1.261281in}}%
\pgfpathlineto{\pgfqpoint{1.498042in}{1.801383in}}%
\pgfpathlineto{\pgfqpoint{1.498909in}{0.915393in}}%
\pgfpathlineto{\pgfqpoint{1.499776in}{1.030839in}}%
\pgfpathlineto{\pgfqpoint{1.500643in}{1.117506in}}%
\pgfpathlineto{\pgfqpoint{1.501510in}{0.751235in}}%
\pgfpathlineto{\pgfqpoint{1.502378in}{1.412808in}}%
\pgfpathlineto{\pgfqpoint{1.503245in}{1.020961in}}%
\pgfpathlineto{\pgfqpoint{1.504112in}{1.071220in}}%
\pgfpathlineto{\pgfqpoint{1.504979in}{1.130714in}}%
\pgfpathlineto{\pgfqpoint{1.505846in}{0.800806in}}%
\pgfpathlineto{\pgfqpoint{1.506713in}{0.876685in}}%
\pgfpathlineto{\pgfqpoint{1.507580in}{1.171484in}}%
\pgfpathlineto{\pgfqpoint{1.508448in}{1.102261in}}%
\pgfpathlineto{\pgfqpoint{1.509315in}{0.764133in}}%
\pgfpathlineto{\pgfqpoint{1.510182in}{0.967719in}}%
\pgfpathlineto{\pgfqpoint{1.511049in}{0.826482in}}%
\pgfpathlineto{\pgfqpoint{1.512783in}{1.273185in}}%
\pgfpathlineto{\pgfqpoint{1.513650in}{1.142896in}}%
\pgfpathlineto{\pgfqpoint{1.514517in}{1.073272in}}%
\pgfpathlineto{\pgfqpoint{1.515385in}{0.731549in}}%
\pgfpathlineto{\pgfqpoint{1.516252in}{0.772127in}}%
\pgfpathlineto{\pgfqpoint{1.517119in}{1.224549in}}%
\pgfpathlineto{\pgfqpoint{1.517986in}{1.171634in}}%
\pgfpathlineto{\pgfqpoint{1.518853in}{0.908443in}}%
\pgfpathlineto{\pgfqpoint{1.520587in}{1.359432in}}%
\pgfpathlineto{\pgfqpoint{1.521455in}{0.867263in}}%
\pgfpathlineto{\pgfqpoint{1.523189in}{1.606481in}}%
\pgfpathlineto{\pgfqpoint{1.524923in}{1.066090in}}%
\pgfpathlineto{\pgfqpoint{1.525790in}{0.772948in}}%
\pgfpathlineto{\pgfqpoint{1.526657in}{1.043675in}}%
\pgfpathlineto{\pgfqpoint{1.527524in}{1.013061in}}%
\pgfpathlineto{\pgfqpoint{1.528392in}{1.086323in}}%
\pgfpathlineto{\pgfqpoint{1.529259in}{0.825622in}}%
\pgfpathlineto{\pgfqpoint{1.530126in}{1.246216in}}%
\pgfpathlineto{\pgfqpoint{1.531860in}{0.740866in}}%
\pgfpathlineto{\pgfqpoint{1.532727in}{1.434356in}}%
\pgfpathlineto{\pgfqpoint{1.533594in}{0.847123in}}%
\pgfpathlineto{\pgfqpoint{1.534462in}{1.240103in}}%
\pgfpathlineto{\pgfqpoint{1.535329in}{0.842141in}}%
\pgfpathlineto{\pgfqpoint{1.536196in}{0.926631in}}%
\pgfpathlineto{\pgfqpoint{1.537063in}{1.156440in}}%
\pgfpathlineto{\pgfqpoint{1.537930in}{0.759540in}}%
\pgfpathlineto{\pgfqpoint{1.538797in}{1.279800in}}%
\pgfpathlineto{\pgfqpoint{1.539664in}{0.961267in}}%
\pgfpathlineto{\pgfqpoint{1.540531in}{1.167624in}}%
\pgfpathlineto{\pgfqpoint{1.542266in}{1.034015in}}%
\pgfpathlineto{\pgfqpoint{1.544000in}{1.260868in}}%
\pgfpathlineto{\pgfqpoint{1.544867in}{0.860449in}}%
\pgfpathlineto{\pgfqpoint{1.547469in}{1.204606in}}%
\pgfpathlineto{\pgfqpoint{1.548336in}{0.917534in}}%
\pgfpathlineto{\pgfqpoint{1.549203in}{1.229094in}}%
\pgfpathlineto{\pgfqpoint{1.550070in}{0.917949in}}%
\pgfpathlineto{\pgfqpoint{1.550937in}{0.965313in}}%
\pgfpathlineto{\pgfqpoint{1.551804in}{0.706880in}}%
\pgfpathlineto{\pgfqpoint{1.553538in}{1.076419in}}%
\pgfpathlineto{\pgfqpoint{1.554406in}{0.679374in}}%
\pgfpathlineto{\pgfqpoint{1.556140in}{1.191402in}}%
\pgfpathlineto{\pgfqpoint{1.557007in}{0.887306in}}%
\pgfpathlineto{\pgfqpoint{1.557874in}{1.191481in}}%
\pgfpathlineto{\pgfqpoint{1.558741in}{0.972540in}}%
\pgfpathlineto{\pgfqpoint{1.559608in}{1.045300in}}%
\pgfpathlineto{\pgfqpoint{1.560476in}{0.994580in}}%
\pgfpathlineto{\pgfqpoint{1.561343in}{1.177789in}}%
\pgfpathlineto{\pgfqpoint{1.563077in}{0.942667in}}%
\pgfpathlineto{\pgfqpoint{1.563944in}{1.271852in}}%
\pgfpathlineto{\pgfqpoint{1.564811in}{1.017457in}}%
\pgfpathlineto{\pgfqpoint{1.565678in}{1.591137in}}%
\pgfpathlineto{\pgfqpoint{1.566545in}{1.540979in}}%
\pgfpathlineto{\pgfqpoint{1.567413in}{1.015126in}}%
\pgfpathlineto{\pgfqpoint{1.568280in}{1.622074in}}%
\pgfpathlineto{\pgfqpoint{1.569147in}{0.877396in}}%
\pgfpathlineto{\pgfqpoint{1.570014in}{1.233050in}}%
\pgfpathlineto{\pgfqpoint{1.570881in}{0.983748in}}%
\pgfpathlineto{\pgfqpoint{1.571748in}{1.203667in}}%
\pgfpathlineto{\pgfqpoint{1.572615in}{1.101723in}}%
\pgfpathlineto{\pgfqpoint{1.573483in}{0.765850in}}%
\pgfpathlineto{\pgfqpoint{1.574350in}{0.821850in}}%
\pgfpathlineto{\pgfqpoint{1.576951in}{1.494469in}}%
\pgfpathlineto{\pgfqpoint{1.578685in}{0.898963in}}%
\pgfpathlineto{\pgfqpoint{1.579552in}{0.908069in}}%
\pgfpathlineto{\pgfqpoint{1.580420in}{1.132871in}}%
\pgfpathlineto{\pgfqpoint{1.581287in}{1.059308in}}%
\pgfpathlineto{\pgfqpoint{1.582154in}{1.225707in}}%
\pgfpathlineto{\pgfqpoint{1.583021in}{0.874832in}}%
\pgfpathlineto{\pgfqpoint{1.583888in}{1.310336in}}%
\pgfpathlineto{\pgfqpoint{1.584755in}{0.717962in}}%
\pgfpathlineto{\pgfqpoint{1.585622in}{1.021245in}}%
\pgfpathlineto{\pgfqpoint{1.586490in}{0.879147in}}%
\pgfpathlineto{\pgfqpoint{1.588224in}{1.035167in}}%
\pgfpathlineto{\pgfqpoint{1.589091in}{1.542106in}}%
\pgfpathlineto{\pgfqpoint{1.589958in}{0.900232in}}%
\pgfpathlineto{\pgfqpoint{1.590825in}{1.030631in}}%
\pgfpathlineto{\pgfqpoint{1.591692in}{0.896792in}}%
\pgfpathlineto{\pgfqpoint{1.592559in}{1.063382in}}%
\pgfpathlineto{\pgfqpoint{1.594294in}{0.791106in}}%
\pgfpathlineto{\pgfqpoint{1.595161in}{1.795135in}}%
\pgfpathlineto{\pgfqpoint{1.596895in}{1.029386in}}%
\pgfpathlineto{\pgfqpoint{1.597762in}{1.266302in}}%
\pgfpathlineto{\pgfqpoint{1.598629in}{0.848031in}}%
\pgfpathlineto{\pgfqpoint{1.599497in}{0.883408in}}%
\pgfpathlineto{\pgfqpoint{1.600364in}{0.790351in}}%
\pgfpathlineto{\pgfqpoint{1.601231in}{1.532203in}}%
\pgfpathlineto{\pgfqpoint{1.602965in}{0.809919in}}%
\pgfpathlineto{\pgfqpoint{1.603832in}{1.030029in}}%
\pgfpathlineto{\pgfqpoint{1.604699in}{0.917481in}}%
\pgfpathlineto{\pgfqpoint{1.606434in}{1.147120in}}%
\pgfpathlineto{\pgfqpoint{1.607301in}{1.101495in}}%
\pgfpathlineto{\pgfqpoint{1.608168in}{1.155358in}}%
\pgfpathlineto{\pgfqpoint{1.609035in}{0.718689in}}%
\pgfpathlineto{\pgfqpoint{1.610769in}{1.337810in}}%
\pgfpathlineto{\pgfqpoint{1.611636in}{0.736894in}}%
\pgfpathlineto{\pgfqpoint{1.612503in}{1.018493in}}%
\pgfpathlineto{\pgfqpoint{1.615105in}{0.727271in}}%
\pgfpathlineto{\pgfqpoint{1.616839in}{1.259331in}}%
\pgfpathlineto{\pgfqpoint{1.617706in}{0.979932in}}%
\pgfpathlineto{\pgfqpoint{1.618573in}{1.122442in}}%
\pgfpathlineto{\pgfqpoint{1.619441in}{0.934555in}}%
\pgfpathlineto{\pgfqpoint{1.620308in}{1.125697in}}%
\pgfpathlineto{\pgfqpoint{1.621175in}{0.955114in}}%
\pgfpathlineto{\pgfqpoint{1.622042in}{1.006865in}}%
\pgfpathlineto{\pgfqpoint{1.623776in}{1.271780in}}%
\pgfpathlineto{\pgfqpoint{1.624643in}{0.853629in}}%
\pgfpathlineto{\pgfqpoint{1.625510in}{1.036073in}}%
\pgfpathlineto{\pgfqpoint{1.626378in}{0.658856in}}%
\pgfpathlineto{\pgfqpoint{1.627245in}{1.352948in}}%
\pgfpathlineto{\pgfqpoint{1.628112in}{1.200369in}}%
\pgfpathlineto{\pgfqpoint{1.629846in}{1.061514in}}%
\pgfpathlineto{\pgfqpoint{1.630713in}{1.305037in}}%
\pgfpathlineto{\pgfqpoint{1.631580in}{1.962164in}}%
\pgfpathlineto{\pgfqpoint{1.632448in}{0.699384in}}%
\pgfpathlineto{\pgfqpoint{1.633315in}{1.578223in}}%
\pgfpathlineto{\pgfqpoint{1.634182in}{0.843484in}}%
\pgfpathlineto{\pgfqpoint{1.635049in}{1.327709in}}%
\pgfpathlineto{\pgfqpoint{1.636783in}{0.848190in}}%
\pgfpathlineto{\pgfqpoint{1.637650in}{0.886616in}}%
\pgfpathlineto{\pgfqpoint{1.638517in}{1.510753in}}%
\pgfpathlineto{\pgfqpoint{1.640252in}{0.671784in}}%
\pgfpathlineto{\pgfqpoint{1.641119in}{1.560284in}}%
\pgfpathlineto{\pgfqpoint{1.642853in}{0.801129in}}%
\pgfpathlineto{\pgfqpoint{1.643720in}{1.015004in}}%
\pgfpathlineto{\pgfqpoint{1.644587in}{1.385087in}}%
\pgfpathlineto{\pgfqpoint{1.647189in}{1.005590in}}%
\pgfpathlineto{\pgfqpoint{1.648056in}{0.753459in}}%
\pgfpathlineto{\pgfqpoint{1.648923in}{0.836436in}}%
\pgfpathlineto{\pgfqpoint{1.649790in}{0.791826in}}%
\pgfpathlineto{\pgfqpoint{1.651524in}{1.212634in}}%
\pgfpathlineto{\pgfqpoint{1.652392in}{0.979971in}}%
\pgfpathlineto{\pgfqpoint{1.653259in}{1.129238in}}%
\pgfpathlineto{\pgfqpoint{1.654126in}{2.131636in}}%
\pgfpathlineto{\pgfqpoint{1.655860in}{1.188185in}}%
\pgfpathlineto{\pgfqpoint{1.657594in}{1.505683in}}%
\pgfpathlineto{\pgfqpoint{1.658462in}{0.902957in}}%
\pgfpathlineto{\pgfqpoint{1.659329in}{1.084694in}}%
\pgfpathlineto{\pgfqpoint{1.660196in}{0.773722in}}%
\pgfpathlineto{\pgfqpoint{1.661063in}{0.803777in}}%
\pgfpathlineto{\pgfqpoint{1.662797in}{1.141019in}}%
\pgfpathlineto{\pgfqpoint{1.663664in}{1.091787in}}%
\pgfpathlineto{\pgfqpoint{1.664531in}{0.804554in}}%
\pgfpathlineto{\pgfqpoint{1.665399in}{0.844598in}}%
\pgfpathlineto{\pgfqpoint{1.667133in}{1.288763in}}%
\pgfpathlineto{\pgfqpoint{1.668000in}{1.314333in}}%
\pgfpathlineto{\pgfqpoint{1.668867in}{1.643581in}}%
\pgfpathlineto{\pgfqpoint{1.669734in}{1.115673in}}%
\pgfpathlineto{\pgfqpoint{1.670601in}{1.206279in}}%
\pgfpathlineto{\pgfqpoint{1.671469in}{1.717745in}}%
\pgfpathlineto{\pgfqpoint{1.673203in}{0.938496in}}%
\pgfpathlineto{\pgfqpoint{1.674070in}{1.322296in}}%
\pgfpathlineto{\pgfqpoint{1.675804in}{1.188780in}}%
\pgfpathlineto{\pgfqpoint{1.676671in}{1.331989in}}%
\pgfpathlineto{\pgfqpoint{1.677538in}{1.307775in}}%
\pgfpathlineto{\pgfqpoint{1.678406in}{0.892504in}}%
\pgfpathlineto{\pgfqpoint{1.679273in}{1.004168in}}%
\pgfpathlineto{\pgfqpoint{1.681007in}{1.582329in}}%
\pgfpathlineto{\pgfqpoint{1.681874in}{1.072580in}}%
\pgfpathlineto{\pgfqpoint{1.682741in}{1.181156in}}%
\pgfpathlineto{\pgfqpoint{1.683608in}{1.780671in}}%
\pgfpathlineto{\pgfqpoint{1.684476in}{1.270036in}}%
\pgfpathlineto{\pgfqpoint{1.685343in}{1.382530in}}%
\pgfpathlineto{\pgfqpoint{1.686210in}{1.492970in}}%
\pgfpathlineto{\pgfqpoint{1.687077in}{0.799305in}}%
\pgfpathlineto{\pgfqpoint{1.687944in}{1.480141in}}%
\pgfpathlineto{\pgfqpoint{1.688811in}{0.867473in}}%
\pgfpathlineto{\pgfqpoint{1.690545in}{1.445753in}}%
\pgfpathlineto{\pgfqpoint{1.691413in}{1.086348in}}%
\pgfpathlineto{\pgfqpoint{1.692280in}{1.695263in}}%
\pgfpathlineto{\pgfqpoint{1.693147in}{1.626649in}}%
\pgfpathlineto{\pgfqpoint{1.694014in}{0.744388in}}%
\pgfpathlineto{\pgfqpoint{1.694881in}{1.473520in}}%
\pgfpathlineto{\pgfqpoint{1.696615in}{1.040213in}}%
\pgfpathlineto{\pgfqpoint{1.697483in}{1.116850in}}%
\pgfpathlineto{\pgfqpoint{1.699217in}{1.426233in}}%
\pgfpathlineto{\pgfqpoint{1.700084in}{0.821603in}}%
\pgfpathlineto{\pgfqpoint{1.701818in}{1.204940in}}%
\pgfpathlineto{\pgfqpoint{1.703552in}{1.021469in}}%
\pgfpathlineto{\pgfqpoint{1.704420in}{1.051302in}}%
\pgfpathlineto{\pgfqpoint{1.706154in}{1.690896in}}%
\pgfpathlineto{\pgfqpoint{1.707888in}{0.891914in}}%
\pgfpathlineto{\pgfqpoint{1.708755in}{0.722596in}}%
\pgfpathlineto{\pgfqpoint{1.709622in}{1.172818in}}%
\pgfpathlineto{\pgfqpoint{1.710490in}{0.830343in}}%
\pgfpathlineto{\pgfqpoint{1.711357in}{1.023755in}}%
\pgfpathlineto{\pgfqpoint{1.712224in}{1.544364in}}%
\pgfpathlineto{\pgfqpoint{1.713958in}{0.974606in}}%
\pgfpathlineto{\pgfqpoint{1.714825in}{0.940130in}}%
\pgfpathlineto{\pgfqpoint{1.715692in}{1.562224in}}%
\pgfpathlineto{\pgfqpoint{1.717427in}{0.893926in}}%
\pgfpathlineto{\pgfqpoint{1.718294in}{1.891343in}}%
\pgfpathlineto{\pgfqpoint{1.720028in}{0.860142in}}%
\pgfpathlineto{\pgfqpoint{1.720895in}{1.137540in}}%
\pgfpathlineto{\pgfqpoint{1.721762in}{1.395370in}}%
\pgfpathlineto{\pgfqpoint{1.722629in}{1.091899in}}%
\pgfpathlineto{\pgfqpoint{1.723497in}{1.145519in}}%
\pgfpathlineto{\pgfqpoint{1.724364in}{1.846416in}}%
\pgfpathlineto{\pgfqpoint{1.725231in}{1.068925in}}%
\pgfpathlineto{\pgfqpoint{1.726098in}{1.206587in}}%
\pgfpathlineto{\pgfqpoint{1.727832in}{0.983658in}}%
\pgfpathlineto{\pgfqpoint{1.728699in}{1.168076in}}%
\pgfpathlineto{\pgfqpoint{1.729566in}{1.145589in}}%
\pgfpathlineto{\pgfqpoint{1.730434in}{1.810961in}}%
\pgfpathlineto{\pgfqpoint{1.731301in}{1.078924in}}%
\pgfpathlineto{\pgfqpoint{1.732168in}{1.486955in}}%
\pgfpathlineto{\pgfqpoint{1.733035in}{0.918648in}}%
\pgfpathlineto{\pgfqpoint{1.734769in}{1.206802in}}%
\pgfpathlineto{\pgfqpoint{1.735636in}{0.861378in}}%
\pgfpathlineto{\pgfqpoint{1.737371in}{1.923261in}}%
\pgfpathlineto{\pgfqpoint{1.738238in}{1.615849in}}%
\pgfpathlineto{\pgfqpoint{1.739105in}{0.844349in}}%
\pgfpathlineto{\pgfqpoint{1.739972in}{1.022340in}}%
\pgfpathlineto{\pgfqpoint{1.740839in}{1.239523in}}%
\pgfpathlineto{\pgfqpoint{1.741706in}{1.192192in}}%
\pgfpathlineto{\pgfqpoint{1.742573in}{1.198645in}}%
\pgfpathlineto{\pgfqpoint{1.744308in}{0.883309in}}%
\pgfpathlineto{\pgfqpoint{1.745175in}{0.989817in}}%
\pgfpathlineto{\pgfqpoint{1.746042in}{1.363217in}}%
\pgfpathlineto{\pgfqpoint{1.748643in}{0.780776in}}%
\pgfpathlineto{\pgfqpoint{1.749510in}{0.830343in}}%
\pgfpathlineto{\pgfqpoint{1.750378in}{1.707428in}}%
\pgfpathlineto{\pgfqpoint{1.752112in}{0.950958in}}%
\pgfpathlineto{\pgfqpoint{1.752979in}{1.104111in}}%
\pgfpathlineto{\pgfqpoint{1.753846in}{1.175357in}}%
\pgfpathlineto{\pgfqpoint{1.755580in}{0.828797in}}%
\pgfpathlineto{\pgfqpoint{1.756448in}{1.506289in}}%
\pgfpathlineto{\pgfqpoint{1.757315in}{1.089105in}}%
\pgfpathlineto{\pgfqpoint{1.758182in}{1.215844in}}%
\pgfpathlineto{\pgfqpoint{1.760783in}{0.882866in}}%
\pgfpathlineto{\pgfqpoint{1.761650in}{0.835354in}}%
\pgfpathlineto{\pgfqpoint{1.763385in}{1.453325in}}%
\pgfpathlineto{\pgfqpoint{1.765986in}{1.070710in}}%
\pgfpathlineto{\pgfqpoint{1.766853in}{1.639682in}}%
\pgfpathlineto{\pgfqpoint{1.767720in}{1.456627in}}%
\pgfpathlineto{\pgfqpoint{1.768587in}{0.826757in}}%
\pgfpathlineto{\pgfqpoint{1.769455in}{0.989732in}}%
\pgfpathlineto{\pgfqpoint{1.770322in}{1.001271in}}%
\pgfpathlineto{\pgfqpoint{1.771189in}{0.895094in}}%
\pgfpathlineto{\pgfqpoint{1.772056in}{1.051159in}}%
\pgfpathlineto{\pgfqpoint{1.772923in}{1.618472in}}%
\pgfpathlineto{\pgfqpoint{1.773790in}{0.956155in}}%
\pgfpathlineto{\pgfqpoint{1.774657in}{1.002146in}}%
\pgfpathlineto{\pgfqpoint{1.775524in}{1.325634in}}%
\pgfpathlineto{\pgfqpoint{1.777259in}{0.820306in}}%
\pgfpathlineto{\pgfqpoint{1.778126in}{1.293198in}}%
\pgfpathlineto{\pgfqpoint{1.778993in}{1.225479in}}%
\pgfpathlineto{\pgfqpoint{1.779860in}{1.352170in}}%
\pgfpathlineto{\pgfqpoint{1.780727in}{0.977830in}}%
\pgfpathlineto{\pgfqpoint{1.781594in}{1.552439in}}%
\pgfpathlineto{\pgfqpoint{1.782462in}{1.424825in}}%
\pgfpathlineto{\pgfqpoint{1.783329in}{1.657789in}}%
\pgfpathlineto{\pgfqpoint{1.784196in}{1.086970in}}%
\pgfpathlineto{\pgfqpoint{1.785063in}{1.177885in}}%
\pgfpathlineto{\pgfqpoint{1.785930in}{1.097002in}}%
\pgfpathlineto{\pgfqpoint{1.786797in}{1.194388in}}%
\pgfpathlineto{\pgfqpoint{1.787664in}{1.163127in}}%
\pgfpathlineto{\pgfqpoint{1.788531in}{1.067205in}}%
\pgfpathlineto{\pgfqpoint{1.789399in}{1.403998in}}%
\pgfpathlineto{\pgfqpoint{1.790266in}{0.785634in}}%
\pgfpathlineto{\pgfqpoint{1.791133in}{1.102716in}}%
\pgfpathlineto{\pgfqpoint{1.792000in}{1.096415in}}%
\pgfpathlineto{\pgfqpoint{1.793734in}{0.654163in}}%
\pgfpathlineto{\pgfqpoint{1.794601in}{0.995928in}}%
\pgfpathlineto{\pgfqpoint{1.795469in}{0.740753in}}%
\pgfpathlineto{\pgfqpoint{1.796336in}{1.074816in}}%
\pgfpathlineto{\pgfqpoint{1.797203in}{0.700575in}}%
\pgfpathlineto{\pgfqpoint{1.798937in}{1.499480in}}%
\pgfpathlineto{\pgfqpoint{1.799804in}{1.037826in}}%
\pgfpathlineto{\pgfqpoint{1.800671in}{1.197225in}}%
\pgfpathlineto{\pgfqpoint{1.801538in}{1.176546in}}%
\pgfpathlineto{\pgfqpoint{1.802406in}{1.253227in}}%
\pgfpathlineto{\pgfqpoint{1.803273in}{1.092877in}}%
\pgfpathlineto{\pgfqpoint{1.804140in}{0.659467in}}%
\pgfpathlineto{\pgfqpoint{1.806741in}{1.362232in}}%
\pgfpathlineto{\pgfqpoint{1.807608in}{1.247944in}}%
\pgfpathlineto{\pgfqpoint{1.808476in}{0.910609in}}%
\pgfpathlineto{\pgfqpoint{1.810210in}{1.783946in}}%
\pgfpathlineto{\pgfqpoint{1.811077in}{1.730489in}}%
\pgfpathlineto{\pgfqpoint{1.811944in}{1.199812in}}%
\pgfpathlineto{\pgfqpoint{1.812811in}{1.753133in}}%
\pgfpathlineto{\pgfqpoint{1.813678in}{1.224687in}}%
\pgfpathlineto{\pgfqpoint{1.814545in}{1.516985in}}%
\pgfpathlineto{\pgfqpoint{1.815413in}{1.115260in}}%
\pgfpathlineto{\pgfqpoint{1.816280in}{1.131123in}}%
\pgfpathlineto{\pgfqpoint{1.817147in}{1.073099in}}%
\pgfpathlineto{\pgfqpoint{1.818881in}{0.751287in}}%
\pgfpathlineto{\pgfqpoint{1.819748in}{1.177357in}}%
\pgfpathlineto{\pgfqpoint{1.820615in}{1.054160in}}%
\pgfpathlineto{\pgfqpoint{1.821483in}{0.704320in}}%
\pgfpathlineto{\pgfqpoint{1.822350in}{1.467005in}}%
\pgfpathlineto{\pgfqpoint{1.824951in}{0.745704in}}%
\pgfpathlineto{\pgfqpoint{1.825818in}{1.113563in}}%
\pgfpathlineto{\pgfqpoint{1.827552in}{0.889659in}}%
\pgfpathlineto{\pgfqpoint{1.830154in}{1.516347in}}%
\pgfpathlineto{\pgfqpoint{1.831888in}{0.743438in}}%
\pgfpathlineto{\pgfqpoint{1.832755in}{0.698052in}}%
\pgfpathlineto{\pgfqpoint{1.833622in}{1.272348in}}%
\pgfpathlineto{\pgfqpoint{1.834490in}{1.200554in}}%
\pgfpathlineto{\pgfqpoint{1.835357in}{0.890740in}}%
\pgfpathlineto{\pgfqpoint{1.836224in}{1.065433in}}%
\pgfpathlineto{\pgfqpoint{1.837091in}{1.056306in}}%
\pgfpathlineto{\pgfqpoint{1.837958in}{0.949871in}}%
\pgfpathlineto{\pgfqpoint{1.838825in}{1.185269in}}%
\pgfpathlineto{\pgfqpoint{1.839692in}{0.688330in}}%
\pgfpathlineto{\pgfqpoint{1.840559in}{0.764688in}}%
\pgfpathlineto{\pgfqpoint{1.841427in}{1.057343in}}%
\pgfpathlineto{\pgfqpoint{1.842294in}{1.016111in}}%
\pgfpathlineto{\pgfqpoint{1.843161in}{0.985393in}}%
\pgfpathlineto{\pgfqpoint{1.844895in}{0.744693in}}%
\pgfpathlineto{\pgfqpoint{1.845762in}{1.568531in}}%
\pgfpathlineto{\pgfqpoint{1.846629in}{1.060887in}}%
\pgfpathlineto{\pgfqpoint{1.847497in}{1.293227in}}%
\pgfpathlineto{\pgfqpoint{1.848364in}{0.866664in}}%
\pgfpathlineto{\pgfqpoint{1.849231in}{1.127152in}}%
\pgfpathlineto{\pgfqpoint{1.850098in}{0.832139in}}%
\pgfpathlineto{\pgfqpoint{1.850965in}{1.428985in}}%
\pgfpathlineto{\pgfqpoint{1.851832in}{1.273346in}}%
\pgfpathlineto{\pgfqpoint{1.852699in}{1.514320in}}%
\pgfpathlineto{\pgfqpoint{1.853566in}{1.008200in}}%
\pgfpathlineto{\pgfqpoint{1.854434in}{1.046287in}}%
\pgfpathlineto{\pgfqpoint{1.855301in}{1.061109in}}%
\pgfpathlineto{\pgfqpoint{1.856168in}{1.035227in}}%
\pgfpathlineto{\pgfqpoint{1.857035in}{1.302848in}}%
\pgfpathlineto{\pgfqpoint{1.857902in}{0.684992in}}%
\pgfpathlineto{\pgfqpoint{1.859636in}{1.132435in}}%
\pgfpathlineto{\pgfqpoint{1.860503in}{1.056430in}}%
\pgfpathlineto{\pgfqpoint{1.861371in}{1.234629in}}%
\pgfpathlineto{\pgfqpoint{1.862238in}{1.151014in}}%
\pgfpathlineto{\pgfqpoint{1.863105in}{0.752554in}}%
\pgfpathlineto{\pgfqpoint{1.864839in}{1.070523in}}%
\pgfpathlineto{\pgfqpoint{1.865706in}{0.725794in}}%
\pgfpathlineto{\pgfqpoint{1.866573in}{1.258187in}}%
\pgfpathlineto{\pgfqpoint{1.867441in}{1.037649in}}%
\pgfpathlineto{\pgfqpoint{1.869175in}{1.640692in}}%
\pgfpathlineto{\pgfqpoint{1.870909in}{1.212740in}}%
\pgfpathlineto{\pgfqpoint{1.872643in}{0.828190in}}%
\pgfpathlineto{\pgfqpoint{1.873510in}{1.326621in}}%
\pgfpathlineto{\pgfqpoint{1.875245in}{0.829116in}}%
\pgfpathlineto{\pgfqpoint{1.876112in}{1.449326in}}%
\pgfpathlineto{\pgfqpoint{1.876979in}{0.899741in}}%
\pgfpathlineto{\pgfqpoint{1.877846in}{1.254536in}}%
\pgfpathlineto{\pgfqpoint{1.878713in}{0.659459in}}%
\pgfpathlineto{\pgfqpoint{1.879580in}{1.444385in}}%
\pgfpathlineto{\pgfqpoint{1.880448in}{0.896361in}}%
\pgfpathlineto{\pgfqpoint{1.881315in}{1.262170in}}%
\pgfpathlineto{\pgfqpoint{1.882182in}{0.871994in}}%
\pgfpathlineto{\pgfqpoint{1.883049in}{0.884511in}}%
\pgfpathlineto{\pgfqpoint{1.883916in}{1.461573in}}%
\pgfpathlineto{\pgfqpoint{1.885650in}{0.973064in}}%
\pgfpathlineto{\pgfqpoint{1.888252in}{1.163362in}}%
\pgfpathlineto{\pgfqpoint{1.889119in}{1.539465in}}%
\pgfpathlineto{\pgfqpoint{1.889986in}{0.935107in}}%
\pgfpathlineto{\pgfqpoint{1.890853in}{1.212905in}}%
\pgfpathlineto{\pgfqpoint{1.891720in}{1.102300in}}%
\pgfpathlineto{\pgfqpoint{1.892587in}{1.551370in}}%
\pgfpathlineto{\pgfqpoint{1.894322in}{0.985957in}}%
\pgfpathlineto{\pgfqpoint{1.896056in}{0.788865in}}%
\pgfpathlineto{\pgfqpoint{1.896923in}{1.005116in}}%
\pgfpathlineto{\pgfqpoint{1.897790in}{0.865014in}}%
\pgfpathlineto{\pgfqpoint{1.899524in}{1.325034in}}%
\pgfpathlineto{\pgfqpoint{1.902126in}{1.130133in}}%
\pgfpathlineto{\pgfqpoint{1.903860in}{0.925735in}}%
\pgfpathlineto{\pgfqpoint{1.904727in}{1.285183in}}%
\pgfpathlineto{\pgfqpoint{1.905594in}{1.248783in}}%
\pgfpathlineto{\pgfqpoint{1.908196in}{0.707315in}}%
\pgfpathlineto{\pgfqpoint{1.909063in}{1.461637in}}%
\pgfpathlineto{\pgfqpoint{1.909930in}{1.060417in}}%
\pgfpathlineto{\pgfqpoint{1.910797in}{1.095769in}}%
\pgfpathlineto{\pgfqpoint{1.912531in}{0.836701in}}%
\pgfpathlineto{\pgfqpoint{1.914266in}{1.099283in}}%
\pgfpathlineto{\pgfqpoint{1.915133in}{1.082161in}}%
\pgfpathlineto{\pgfqpoint{1.916000in}{0.829164in}}%
\pgfpathlineto{\pgfqpoint{1.918601in}{1.658963in}}%
\pgfpathlineto{\pgfqpoint{1.919469in}{1.741520in}}%
\pgfpathlineto{\pgfqpoint{1.920336in}{0.721870in}}%
\pgfpathlineto{\pgfqpoint{1.921203in}{1.257388in}}%
\pgfpathlineto{\pgfqpoint{1.922070in}{0.954195in}}%
\pgfpathlineto{\pgfqpoint{1.922937in}{1.029883in}}%
\pgfpathlineto{\pgfqpoint{1.923804in}{1.233097in}}%
\pgfpathlineto{\pgfqpoint{1.924671in}{0.659036in}}%
\pgfpathlineto{\pgfqpoint{1.925538in}{0.732396in}}%
\pgfpathlineto{\pgfqpoint{1.928140in}{1.063886in}}%
\pgfpathlineto{\pgfqpoint{1.929007in}{1.630334in}}%
\pgfpathlineto{\pgfqpoint{1.930741in}{1.092544in}}%
\pgfpathlineto{\pgfqpoint{1.932476in}{1.430610in}}%
\pgfpathlineto{\pgfqpoint{1.933343in}{0.926202in}}%
\pgfpathlineto{\pgfqpoint{1.934210in}{0.969664in}}%
\pgfpathlineto{\pgfqpoint{1.935944in}{1.364109in}}%
\pgfpathlineto{\pgfqpoint{1.938545in}{1.016729in}}%
\pgfpathlineto{\pgfqpoint{1.939413in}{1.081063in}}%
\pgfpathlineto{\pgfqpoint{1.940280in}{0.886561in}}%
\pgfpathlineto{\pgfqpoint{1.942014in}{1.337033in}}%
\pgfpathlineto{\pgfqpoint{1.942881in}{1.277656in}}%
\pgfpathlineto{\pgfqpoint{1.944615in}{0.891633in}}%
\pgfpathlineto{\pgfqpoint{1.945483in}{0.930094in}}%
\pgfpathlineto{\pgfqpoint{1.946350in}{1.200194in}}%
\pgfpathlineto{\pgfqpoint{1.947217in}{1.147110in}}%
\pgfpathlineto{\pgfqpoint{1.948084in}{1.111032in}}%
\pgfpathlineto{\pgfqpoint{1.948951in}{0.782043in}}%
\pgfpathlineto{\pgfqpoint{1.949818in}{1.113214in}}%
\pgfpathlineto{\pgfqpoint{1.950685in}{0.982548in}}%
\pgfpathlineto{\pgfqpoint{1.952420in}{1.368389in}}%
\pgfpathlineto{\pgfqpoint{1.955021in}{0.785891in}}%
\pgfpathlineto{\pgfqpoint{1.955888in}{1.553947in}}%
\pgfpathlineto{\pgfqpoint{1.956755in}{1.001852in}}%
\pgfpathlineto{\pgfqpoint{1.958490in}{1.477207in}}%
\pgfpathlineto{\pgfqpoint{1.959357in}{0.701078in}}%
\pgfpathlineto{\pgfqpoint{1.960224in}{0.790654in}}%
\pgfpathlineto{\pgfqpoint{1.961958in}{1.165842in}}%
\pgfpathlineto{\pgfqpoint{1.963692in}{0.735878in}}%
\pgfpathlineto{\pgfqpoint{1.965427in}{1.255670in}}%
\pgfpathlineto{\pgfqpoint{1.966294in}{1.692305in}}%
\pgfpathlineto{\pgfqpoint{1.968028in}{0.934719in}}%
\pgfpathlineto{\pgfqpoint{1.968895in}{0.742716in}}%
\pgfpathlineto{\pgfqpoint{1.969762in}{1.327989in}}%
\pgfpathlineto{\pgfqpoint{1.970629in}{1.014040in}}%
\pgfpathlineto{\pgfqpoint{1.971497in}{1.412376in}}%
\pgfpathlineto{\pgfqpoint{1.974098in}{0.818352in}}%
\pgfpathlineto{\pgfqpoint{1.975832in}{0.981500in}}%
\pgfpathlineto{\pgfqpoint{1.976699in}{0.971707in}}%
\pgfpathlineto{\pgfqpoint{1.977566in}{1.050007in}}%
\pgfpathlineto{\pgfqpoint{1.978434in}{0.663698in}}%
\pgfpathlineto{\pgfqpoint{1.979301in}{0.730619in}}%
\pgfpathlineto{\pgfqpoint{1.980168in}{1.002993in}}%
\pgfpathlineto{\pgfqpoint{1.981035in}{0.937713in}}%
\pgfpathlineto{\pgfqpoint{1.981902in}{1.065217in}}%
\pgfpathlineto{\pgfqpoint{1.982769in}{1.357314in}}%
\pgfpathlineto{\pgfqpoint{1.983636in}{0.750672in}}%
\pgfpathlineto{\pgfqpoint{1.984503in}{1.167358in}}%
\pgfpathlineto{\pgfqpoint{1.985371in}{1.165239in}}%
\pgfpathlineto{\pgfqpoint{1.986238in}{0.942433in}}%
\pgfpathlineto{\pgfqpoint{1.987105in}{1.247122in}}%
\pgfpathlineto{\pgfqpoint{1.988839in}{0.788765in}}%
\pgfpathlineto{\pgfqpoint{1.989706in}{0.790519in}}%
\pgfpathlineto{\pgfqpoint{1.990573in}{1.520235in}}%
\pgfpathlineto{\pgfqpoint{1.991441in}{0.880071in}}%
\pgfpathlineto{\pgfqpoint{1.992308in}{1.255438in}}%
\pgfpathlineto{\pgfqpoint{1.993175in}{1.208149in}}%
\pgfpathlineto{\pgfqpoint{1.994042in}{1.363252in}}%
\pgfpathlineto{\pgfqpoint{1.994909in}{1.298711in}}%
\pgfpathlineto{\pgfqpoint{1.995776in}{0.860560in}}%
\pgfpathlineto{\pgfqpoint{1.997510in}{1.297287in}}%
\pgfpathlineto{\pgfqpoint{1.998378in}{0.808915in}}%
\pgfpathlineto{\pgfqpoint{2.000112in}{1.418158in}}%
\pgfpathlineto{\pgfqpoint{2.000979in}{0.991666in}}%
\pgfpathlineto{\pgfqpoint{2.001846in}{1.032791in}}%
\pgfpathlineto{\pgfqpoint{2.002713in}{1.503933in}}%
\pgfpathlineto{\pgfqpoint{2.003580in}{1.362868in}}%
\pgfpathlineto{\pgfqpoint{2.004448in}{1.674629in}}%
\pgfpathlineto{\pgfqpoint{2.007049in}{0.726340in}}%
\pgfpathlineto{\pgfqpoint{2.007916in}{1.315426in}}%
\pgfpathlineto{\pgfqpoint{2.008783in}{1.071003in}}%
\pgfpathlineto{\pgfqpoint{2.010517in}{1.243809in}}%
\pgfpathlineto{\pgfqpoint{2.011385in}{1.218744in}}%
\pgfpathlineto{\pgfqpoint{2.012252in}{1.021267in}}%
\pgfpathlineto{\pgfqpoint{2.013119in}{1.450947in}}%
\pgfpathlineto{\pgfqpoint{2.013986in}{0.780156in}}%
\pgfpathlineto{\pgfqpoint{2.014853in}{0.806113in}}%
\pgfpathlineto{\pgfqpoint{2.015720in}{1.289396in}}%
\pgfpathlineto{\pgfqpoint{2.016587in}{0.992197in}}%
\pgfpathlineto{\pgfqpoint{2.017455in}{1.326484in}}%
\pgfpathlineto{\pgfqpoint{2.018322in}{0.759116in}}%
\pgfpathlineto{\pgfqpoint{2.019189in}{0.850052in}}%
\pgfpathlineto{\pgfqpoint{2.020056in}{1.868555in}}%
\pgfpathlineto{\pgfqpoint{2.020923in}{1.019295in}}%
\pgfpathlineto{\pgfqpoint{2.021790in}{1.129546in}}%
\pgfpathlineto{\pgfqpoint{2.022657in}{0.789974in}}%
\pgfpathlineto{\pgfqpoint{2.024392in}{1.285250in}}%
\pgfpathlineto{\pgfqpoint{2.025259in}{0.686967in}}%
\pgfpathlineto{\pgfqpoint{2.026993in}{1.496210in}}%
\pgfpathlineto{\pgfqpoint{2.027860in}{1.424123in}}%
\pgfpathlineto{\pgfqpoint{2.028727in}{1.101917in}}%
\pgfpathlineto{\pgfqpoint{2.029594in}{1.598280in}}%
\pgfpathlineto{\pgfqpoint{2.031329in}{0.700855in}}%
\pgfpathlineto{\pgfqpoint{2.032196in}{0.984193in}}%
\pgfpathlineto{\pgfqpoint{2.033063in}{0.889880in}}%
\pgfpathlineto{\pgfqpoint{2.033930in}{1.265912in}}%
\pgfpathlineto{\pgfqpoint{2.036531in}{0.760743in}}%
\pgfpathlineto{\pgfqpoint{2.038266in}{0.977106in}}%
\pgfpathlineto{\pgfqpoint{2.039133in}{1.738067in}}%
\pgfpathlineto{\pgfqpoint{2.040867in}{0.881785in}}%
\pgfpathlineto{\pgfqpoint{2.042601in}{0.679935in}}%
\pgfpathlineto{\pgfqpoint{2.043469in}{1.312308in}}%
\pgfpathlineto{\pgfqpoint{2.044336in}{1.052951in}}%
\pgfpathlineto{\pgfqpoint{2.045203in}{1.086937in}}%
\pgfpathlineto{\pgfqpoint{2.046070in}{1.541903in}}%
\pgfpathlineto{\pgfqpoint{2.046937in}{1.045405in}}%
\pgfpathlineto{\pgfqpoint{2.047804in}{1.567666in}}%
\pgfpathlineto{\pgfqpoint{2.048671in}{0.619331in}}%
\pgfpathlineto{\pgfqpoint{2.049538in}{1.410655in}}%
\pgfpathlineto{\pgfqpoint{2.050406in}{1.193034in}}%
\pgfpathlineto{\pgfqpoint{2.051273in}{1.129804in}}%
\pgfpathlineto{\pgfqpoint{2.052140in}{1.164172in}}%
\pgfpathlineto{\pgfqpoint{2.053007in}{0.866044in}}%
\pgfpathlineto{\pgfqpoint{2.053874in}{0.921432in}}%
\pgfpathlineto{\pgfqpoint{2.055608in}{1.273798in}}%
\pgfpathlineto{\pgfqpoint{2.056476in}{0.790101in}}%
\pgfpathlineto{\pgfqpoint{2.059077in}{1.395660in}}%
\pgfpathlineto{\pgfqpoint{2.059944in}{1.016821in}}%
\pgfpathlineto{\pgfqpoint{2.060811in}{1.152481in}}%
\pgfpathlineto{\pgfqpoint{2.061678in}{1.112609in}}%
\pgfpathlineto{\pgfqpoint{2.062545in}{0.820482in}}%
\pgfpathlineto{\pgfqpoint{2.063413in}{1.275700in}}%
\pgfpathlineto{\pgfqpoint{2.064280in}{0.903352in}}%
\pgfpathlineto{\pgfqpoint{2.066014in}{1.363840in}}%
\pgfpathlineto{\pgfqpoint{2.068615in}{0.765051in}}%
\pgfpathlineto{\pgfqpoint{2.069483in}{1.193436in}}%
\pgfpathlineto{\pgfqpoint{2.071217in}{0.820403in}}%
\pgfpathlineto{\pgfqpoint{2.072084in}{0.832275in}}%
\pgfpathlineto{\pgfqpoint{2.072951in}{1.133048in}}%
\pgfpathlineto{\pgfqpoint{2.073818in}{0.862630in}}%
\pgfpathlineto{\pgfqpoint{2.074685in}{1.380560in}}%
\pgfpathlineto{\pgfqpoint{2.075552in}{1.363355in}}%
\pgfpathlineto{\pgfqpoint{2.076420in}{1.152892in}}%
\pgfpathlineto{\pgfqpoint{2.077287in}{1.572328in}}%
\pgfpathlineto{\pgfqpoint{2.079021in}{0.712360in}}%
\pgfpathlineto{\pgfqpoint{2.081622in}{1.775408in}}%
\pgfpathlineto{\pgfqpoint{2.083357in}{0.815848in}}%
\pgfpathlineto{\pgfqpoint{2.084224in}{0.784953in}}%
\pgfpathlineto{\pgfqpoint{2.085091in}{1.325761in}}%
\pgfpathlineto{\pgfqpoint{2.085958in}{0.876176in}}%
\pgfpathlineto{\pgfqpoint{2.086825in}{1.308629in}}%
\pgfpathlineto{\pgfqpoint{2.087692in}{0.738440in}}%
\pgfpathlineto{\pgfqpoint{2.089427in}{1.084756in}}%
\pgfpathlineto{\pgfqpoint{2.090294in}{1.052640in}}%
\pgfpathlineto{\pgfqpoint{2.091161in}{0.721040in}}%
\pgfpathlineto{\pgfqpoint{2.094629in}{1.308794in}}%
\pgfpathlineto{\pgfqpoint{2.095497in}{0.816681in}}%
\pgfpathlineto{\pgfqpoint{2.097231in}{1.235176in}}%
\pgfpathlineto{\pgfqpoint{2.098098in}{0.625049in}}%
\pgfpathlineto{\pgfqpoint{2.098965in}{1.685366in}}%
\pgfpathlineto{\pgfqpoint{2.099832in}{1.524516in}}%
\pgfpathlineto{\pgfqpoint{2.100699in}{1.596784in}}%
\pgfpathlineto{\pgfqpoint{2.101566in}{0.922098in}}%
\pgfpathlineto{\pgfqpoint{2.102434in}{0.965919in}}%
\pgfpathlineto{\pgfqpoint{2.103301in}{1.553127in}}%
\pgfpathlineto{\pgfqpoint{2.104168in}{1.206354in}}%
\pgfpathlineto{\pgfqpoint{2.105035in}{1.233299in}}%
\pgfpathlineto{\pgfqpoint{2.105902in}{1.391146in}}%
\pgfpathlineto{\pgfqpoint{2.106769in}{0.722741in}}%
\pgfpathlineto{\pgfqpoint{2.109371in}{1.869750in}}%
\pgfpathlineto{\pgfqpoint{2.110238in}{0.949948in}}%
\pgfpathlineto{\pgfqpoint{2.111105in}{1.063993in}}%
\pgfpathlineto{\pgfqpoint{2.111972in}{1.063762in}}%
\pgfpathlineto{\pgfqpoint{2.112839in}{0.992540in}}%
\pgfpathlineto{\pgfqpoint{2.113706in}{1.428573in}}%
\pgfpathlineto{\pgfqpoint{2.114573in}{0.840279in}}%
\pgfpathlineto{\pgfqpoint{2.115441in}{0.972598in}}%
\pgfpathlineto{\pgfqpoint{2.116308in}{1.091194in}}%
\pgfpathlineto{\pgfqpoint{2.118042in}{0.793655in}}%
\pgfpathlineto{\pgfqpoint{2.118909in}{1.385002in}}%
\pgfpathlineto{\pgfqpoint{2.120643in}{0.850296in}}%
\pgfpathlineto{\pgfqpoint{2.121510in}{1.443509in}}%
\pgfpathlineto{\pgfqpoint{2.122378in}{1.327959in}}%
\pgfpathlineto{\pgfqpoint{2.123245in}{0.907967in}}%
\pgfpathlineto{\pgfqpoint{2.124112in}{1.555515in}}%
\pgfpathlineto{\pgfqpoint{2.124979in}{0.891063in}}%
\pgfpathlineto{\pgfqpoint{2.125846in}{1.229383in}}%
\pgfpathlineto{\pgfqpoint{2.126713in}{0.705709in}}%
\pgfpathlineto{\pgfqpoint{2.127580in}{1.047358in}}%
\pgfpathlineto{\pgfqpoint{2.128448in}{0.967004in}}%
\pgfpathlineto{\pgfqpoint{2.129315in}{1.009790in}}%
\pgfpathlineto{\pgfqpoint{2.131049in}{1.466160in}}%
\pgfpathlineto{\pgfqpoint{2.132783in}{0.789598in}}%
\pgfpathlineto{\pgfqpoint{2.133650in}{0.918068in}}%
\pgfpathlineto{\pgfqpoint{2.134517in}{1.392108in}}%
\pgfpathlineto{\pgfqpoint{2.136252in}{0.903447in}}%
\pgfpathlineto{\pgfqpoint{2.137119in}{0.824654in}}%
\pgfpathlineto{\pgfqpoint{2.138853in}{1.870627in}}%
\pgfpathlineto{\pgfqpoint{2.139720in}{0.733306in}}%
\pgfpathlineto{\pgfqpoint{2.140587in}{1.387402in}}%
\pgfpathlineto{\pgfqpoint{2.141455in}{1.382411in}}%
\pgfpathlineto{\pgfqpoint{2.142322in}{1.432408in}}%
\pgfpathlineto{\pgfqpoint{2.143189in}{1.316821in}}%
\pgfpathlineto{\pgfqpoint{2.144923in}{0.830159in}}%
\pgfpathlineto{\pgfqpoint{2.145790in}{1.223125in}}%
\pgfpathlineto{\pgfqpoint{2.146657in}{0.771211in}}%
\pgfpathlineto{\pgfqpoint{2.147524in}{1.625116in}}%
\pgfpathlineto{\pgfqpoint{2.148392in}{1.118757in}}%
\pgfpathlineto{\pgfqpoint{2.149259in}{1.405004in}}%
\pgfpathlineto{\pgfqpoint{2.150126in}{1.117271in}}%
\pgfpathlineto{\pgfqpoint{2.150993in}{1.900061in}}%
\pgfpathlineto{\pgfqpoint{2.152727in}{0.930342in}}%
\pgfpathlineto{\pgfqpoint{2.153594in}{1.410709in}}%
\pgfpathlineto{\pgfqpoint{2.154462in}{1.370119in}}%
\pgfpathlineto{\pgfqpoint{2.155329in}{1.451114in}}%
\pgfpathlineto{\pgfqpoint{2.156196in}{1.322019in}}%
\pgfpathlineto{\pgfqpoint{2.157063in}{0.819900in}}%
\pgfpathlineto{\pgfqpoint{2.157930in}{1.449595in}}%
\pgfpathlineto{\pgfqpoint{2.159664in}{0.974580in}}%
\pgfpathlineto{\pgfqpoint{2.160531in}{1.161982in}}%
\pgfpathlineto{\pgfqpoint{2.161399in}{0.947257in}}%
\pgfpathlineto{\pgfqpoint{2.162266in}{1.388201in}}%
\pgfpathlineto{\pgfqpoint{2.163133in}{1.233804in}}%
\pgfpathlineto{\pgfqpoint{2.164867in}{0.674138in}}%
\pgfpathlineto{\pgfqpoint{2.165734in}{1.089184in}}%
\pgfpathlineto{\pgfqpoint{2.166601in}{0.969400in}}%
\pgfpathlineto{\pgfqpoint{2.169203in}{1.426078in}}%
\pgfpathlineto{\pgfqpoint{2.170070in}{1.378133in}}%
\pgfpathlineto{\pgfqpoint{2.170937in}{1.473390in}}%
\pgfpathlineto{\pgfqpoint{2.171804in}{0.817896in}}%
\pgfpathlineto{\pgfqpoint{2.172671in}{1.408481in}}%
\pgfpathlineto{\pgfqpoint{2.174406in}{0.847567in}}%
\pgfpathlineto{\pgfqpoint{2.176140in}{1.239113in}}%
\pgfpathlineto{\pgfqpoint{2.177874in}{0.985679in}}%
\pgfpathlineto{\pgfqpoint{2.178741in}{0.849518in}}%
\pgfpathlineto{\pgfqpoint{2.179608in}{1.487406in}}%
\pgfpathlineto{\pgfqpoint{2.180476in}{0.827747in}}%
\pgfpathlineto{\pgfqpoint{2.181343in}{1.243657in}}%
\pgfpathlineto{\pgfqpoint{2.182210in}{1.062300in}}%
\pgfpathlineto{\pgfqpoint{2.183077in}{1.270166in}}%
\pgfpathlineto{\pgfqpoint{2.183944in}{1.191288in}}%
\pgfpathlineto{\pgfqpoint{2.184811in}{0.762471in}}%
\pgfpathlineto{\pgfqpoint{2.185678in}{1.142062in}}%
\pgfpathlineto{\pgfqpoint{2.187413in}{0.780651in}}%
\pgfpathlineto{\pgfqpoint{2.189147in}{1.801775in}}%
\pgfpathlineto{\pgfqpoint{2.190014in}{1.205550in}}%
\pgfpathlineto{\pgfqpoint{2.190881in}{1.297825in}}%
\pgfpathlineto{\pgfqpoint{2.191748in}{0.848570in}}%
\pgfpathlineto{\pgfqpoint{2.192615in}{1.111629in}}%
\pgfpathlineto{\pgfqpoint{2.193483in}{0.695669in}}%
\pgfpathlineto{\pgfqpoint{2.194350in}{1.331885in}}%
\pgfpathlineto{\pgfqpoint{2.195217in}{0.983403in}}%
\pgfpathlineto{\pgfqpoint{2.196951in}{1.615119in}}%
\pgfpathlineto{\pgfqpoint{2.197818in}{1.152532in}}%
\pgfpathlineto{\pgfqpoint{2.198685in}{1.209848in}}%
\pgfpathlineto{\pgfqpoint{2.200420in}{1.276033in}}%
\pgfpathlineto{\pgfqpoint{2.201287in}{0.894332in}}%
\pgfpathlineto{\pgfqpoint{2.202154in}{1.217039in}}%
\pgfpathlineto{\pgfqpoint{2.203021in}{0.714238in}}%
\pgfpathlineto{\pgfqpoint{2.203888in}{1.159942in}}%
\pgfpathlineto{\pgfqpoint{2.204755in}{1.133121in}}%
\pgfpathlineto{\pgfqpoint{2.205622in}{0.780502in}}%
\pgfpathlineto{\pgfqpoint{2.206490in}{1.106682in}}%
\pgfpathlineto{\pgfqpoint{2.207357in}{0.653376in}}%
\pgfpathlineto{\pgfqpoint{2.209091in}{1.229349in}}%
\pgfpathlineto{\pgfqpoint{2.209958in}{1.149953in}}%
\pgfpathlineto{\pgfqpoint{2.210825in}{1.270888in}}%
\pgfpathlineto{\pgfqpoint{2.211692in}{0.940000in}}%
\pgfpathlineto{\pgfqpoint{2.212559in}{1.144974in}}%
\pgfpathlineto{\pgfqpoint{2.213427in}{0.913140in}}%
\pgfpathlineto{\pgfqpoint{2.215161in}{1.245198in}}%
\pgfpathlineto{\pgfqpoint{2.216028in}{1.192082in}}%
\pgfpathlineto{\pgfqpoint{2.217762in}{1.429697in}}%
\pgfpathlineto{\pgfqpoint{2.218629in}{0.789449in}}%
\pgfpathlineto{\pgfqpoint{2.219497in}{0.798489in}}%
\pgfpathlineto{\pgfqpoint{2.220364in}{1.504606in}}%
\pgfpathlineto{\pgfqpoint{2.221231in}{0.722140in}}%
\pgfpathlineto{\pgfqpoint{2.222098in}{0.854415in}}%
\pgfpathlineto{\pgfqpoint{2.223832in}{1.676342in}}%
\pgfpathlineto{\pgfqpoint{2.224699in}{1.132867in}}%
\pgfpathlineto{\pgfqpoint{2.225566in}{1.183888in}}%
\pgfpathlineto{\pgfqpoint{2.226434in}{1.148449in}}%
\pgfpathlineto{\pgfqpoint{2.227301in}{1.427169in}}%
\pgfpathlineto{\pgfqpoint{2.229035in}{0.789614in}}%
\pgfpathlineto{\pgfqpoint{2.229902in}{0.780176in}}%
\pgfpathlineto{\pgfqpoint{2.230769in}{1.030637in}}%
\pgfpathlineto{\pgfqpoint{2.232503in}{0.875727in}}%
\pgfpathlineto{\pgfqpoint{2.233371in}{1.035920in}}%
\pgfpathlineto{\pgfqpoint{2.234238in}{0.936007in}}%
\pgfpathlineto{\pgfqpoint{2.235105in}{0.988359in}}%
\pgfpathlineto{\pgfqpoint{2.235972in}{1.924142in}}%
\pgfpathlineto{\pgfqpoint{2.236839in}{1.140414in}}%
\pgfpathlineto{\pgfqpoint{2.237706in}{1.267032in}}%
\pgfpathlineto{\pgfqpoint{2.238573in}{0.913207in}}%
\pgfpathlineto{\pgfqpoint{2.240308in}{1.355660in}}%
\pgfpathlineto{\pgfqpoint{2.242042in}{0.898717in}}%
\pgfpathlineto{\pgfqpoint{2.242909in}{1.196355in}}%
\pgfpathlineto{\pgfqpoint{2.244643in}{0.724281in}}%
\pgfpathlineto{\pgfqpoint{2.246378in}{1.410301in}}%
\pgfpathlineto{\pgfqpoint{2.248112in}{0.843175in}}%
\pgfpathlineto{\pgfqpoint{2.249846in}{1.360440in}}%
\pgfpathlineto{\pgfqpoint{2.250713in}{1.546596in}}%
\pgfpathlineto{\pgfqpoint{2.252448in}{0.627675in}}%
\pgfpathlineto{\pgfqpoint{2.253315in}{0.917721in}}%
\pgfpathlineto{\pgfqpoint{2.254182in}{0.817335in}}%
\pgfpathlineto{\pgfqpoint{2.255049in}{1.092130in}}%
\pgfpathlineto{\pgfqpoint{2.255916in}{0.955184in}}%
\pgfpathlineto{\pgfqpoint{2.256783in}{1.514778in}}%
\pgfpathlineto{\pgfqpoint{2.258517in}{1.044604in}}%
\pgfpathlineto{\pgfqpoint{2.259385in}{1.364178in}}%
\pgfpathlineto{\pgfqpoint{2.261119in}{0.904536in}}%
\pgfpathlineto{\pgfqpoint{2.261986in}{1.082052in}}%
\pgfpathlineto{\pgfqpoint{2.262853in}{1.625162in}}%
\pgfpathlineto{\pgfqpoint{2.263720in}{1.128530in}}%
\pgfpathlineto{\pgfqpoint{2.264587in}{1.570708in}}%
\pgfpathlineto{\pgfqpoint{2.266322in}{1.165265in}}%
\pgfpathlineto{\pgfqpoint{2.267189in}{1.130552in}}%
\pgfpathlineto{\pgfqpoint{2.268056in}{0.907798in}}%
\pgfpathlineto{\pgfqpoint{2.269790in}{1.365676in}}%
\pgfpathlineto{\pgfqpoint{2.270657in}{1.051011in}}%
\pgfpathlineto{\pgfqpoint{2.271524in}{1.279389in}}%
\pgfpathlineto{\pgfqpoint{2.272392in}{1.256213in}}%
\pgfpathlineto{\pgfqpoint{2.273259in}{1.327151in}}%
\pgfpathlineto{\pgfqpoint{2.274993in}{0.721424in}}%
\pgfpathlineto{\pgfqpoint{2.276727in}{1.639382in}}%
\pgfpathlineto{\pgfqpoint{2.277594in}{0.816580in}}%
\pgfpathlineto{\pgfqpoint{2.278462in}{0.826240in}}%
\pgfpathlineto{\pgfqpoint{2.279329in}{1.198421in}}%
\pgfpathlineto{\pgfqpoint{2.280196in}{0.866593in}}%
\pgfpathlineto{\pgfqpoint{2.282797in}{1.147496in}}%
\pgfpathlineto{\pgfqpoint{2.283664in}{1.061846in}}%
\pgfpathlineto{\pgfqpoint{2.284531in}{0.824987in}}%
\pgfpathlineto{\pgfqpoint{2.285399in}{1.085714in}}%
\pgfpathlineto{\pgfqpoint{2.286266in}{0.842079in}}%
\pgfpathlineto{\pgfqpoint{2.287133in}{1.299340in}}%
\pgfpathlineto{\pgfqpoint{2.288000in}{0.880320in}}%
\pgfpathlineto{\pgfqpoint{2.289734in}{1.287208in}}%
\pgfpathlineto{\pgfqpoint{2.290601in}{1.009201in}}%
\pgfpathlineto{\pgfqpoint{2.291469in}{1.668036in}}%
\pgfpathlineto{\pgfqpoint{2.293203in}{0.836854in}}%
\pgfpathlineto{\pgfqpoint{2.294070in}{1.330230in}}%
\pgfpathlineto{\pgfqpoint{2.294937in}{0.866496in}}%
\pgfpathlineto{\pgfqpoint{2.295804in}{0.876133in}}%
\pgfpathlineto{\pgfqpoint{2.296671in}{1.181598in}}%
\pgfpathlineto{\pgfqpoint{2.297538in}{0.826327in}}%
\pgfpathlineto{\pgfqpoint{2.299273in}{1.313633in}}%
\pgfpathlineto{\pgfqpoint{2.301007in}{0.771143in}}%
\pgfpathlineto{\pgfqpoint{2.301874in}{1.255820in}}%
\pgfpathlineto{\pgfqpoint{2.302741in}{1.075720in}}%
\pgfpathlineto{\pgfqpoint{2.303608in}{1.378629in}}%
\pgfpathlineto{\pgfqpoint{2.304476in}{1.144777in}}%
\pgfpathlineto{\pgfqpoint{2.305343in}{1.246241in}}%
\pgfpathlineto{\pgfqpoint{2.306210in}{0.993052in}}%
\pgfpathlineto{\pgfqpoint{2.307077in}{1.142011in}}%
\pgfpathlineto{\pgfqpoint{2.307944in}{1.074622in}}%
\pgfpathlineto{\pgfqpoint{2.308811in}{1.141293in}}%
\pgfpathlineto{\pgfqpoint{2.309678in}{1.456943in}}%
\pgfpathlineto{\pgfqpoint{2.312280in}{0.804198in}}%
\pgfpathlineto{\pgfqpoint{2.314014in}{0.707300in}}%
\pgfpathlineto{\pgfqpoint{2.314881in}{1.065684in}}%
\pgfpathlineto{\pgfqpoint{2.315748in}{1.004276in}}%
\pgfpathlineto{\pgfqpoint{2.316615in}{1.054522in}}%
\pgfpathlineto{\pgfqpoint{2.318350in}{0.716589in}}%
\pgfpathlineto{\pgfqpoint{2.320951in}{1.015129in}}%
\pgfpathlineto{\pgfqpoint{2.321818in}{1.175724in}}%
\pgfpathlineto{\pgfqpoint{2.322685in}{0.958341in}}%
\pgfpathlineto{\pgfqpoint{2.323552in}{1.439614in}}%
\pgfpathlineto{\pgfqpoint{2.324420in}{1.064230in}}%
\pgfpathlineto{\pgfqpoint{2.325287in}{1.423794in}}%
\pgfpathlineto{\pgfqpoint{2.326154in}{1.294570in}}%
\pgfpathlineto{\pgfqpoint{2.327021in}{0.867876in}}%
\pgfpathlineto{\pgfqpoint{2.327888in}{1.223653in}}%
\pgfpathlineto{\pgfqpoint{2.328755in}{0.921817in}}%
\pgfpathlineto{\pgfqpoint{2.329622in}{1.307848in}}%
\pgfpathlineto{\pgfqpoint{2.330490in}{1.028633in}}%
\pgfpathlineto{\pgfqpoint{2.331357in}{1.216313in}}%
\pgfpathlineto{\pgfqpoint{2.333091in}{0.825619in}}%
\pgfpathlineto{\pgfqpoint{2.333958in}{1.419813in}}%
\pgfpathlineto{\pgfqpoint{2.334825in}{1.399790in}}%
\pgfpathlineto{\pgfqpoint{2.335692in}{1.320168in}}%
\pgfpathlineto{\pgfqpoint{2.337427in}{0.710864in}}%
\pgfpathlineto{\pgfqpoint{2.338294in}{1.429165in}}%
\pgfpathlineto{\pgfqpoint{2.339161in}{1.354923in}}%
\pgfpathlineto{\pgfqpoint{2.340028in}{0.828112in}}%
\pgfpathlineto{\pgfqpoint{2.341762in}{1.344131in}}%
\pgfpathlineto{\pgfqpoint{2.343497in}{0.786220in}}%
\pgfpathlineto{\pgfqpoint{2.346098in}{1.720135in}}%
\pgfpathlineto{\pgfqpoint{2.346965in}{0.952071in}}%
\pgfpathlineto{\pgfqpoint{2.347832in}{1.241036in}}%
\pgfpathlineto{\pgfqpoint{2.348699in}{0.848998in}}%
\pgfpathlineto{\pgfqpoint{2.350434in}{1.028270in}}%
\pgfpathlineto{\pgfqpoint{2.351301in}{0.760966in}}%
\pgfpathlineto{\pgfqpoint{2.352168in}{0.948817in}}%
\pgfpathlineto{\pgfqpoint{2.353035in}{0.781482in}}%
\pgfpathlineto{\pgfqpoint{2.354769in}{0.979453in}}%
\pgfpathlineto{\pgfqpoint{2.355636in}{1.022564in}}%
\pgfpathlineto{\pgfqpoint{2.356503in}{0.694622in}}%
\pgfpathlineto{\pgfqpoint{2.357371in}{1.051221in}}%
\pgfpathlineto{\pgfqpoint{2.359105in}{0.720897in}}%
\pgfpathlineto{\pgfqpoint{2.359972in}{1.087044in}}%
\pgfpathlineto{\pgfqpoint{2.360839in}{0.909643in}}%
\pgfpathlineto{\pgfqpoint{2.361706in}{1.573078in}}%
\pgfpathlineto{\pgfqpoint{2.362573in}{1.475326in}}%
\pgfpathlineto{\pgfqpoint{2.363441in}{0.888245in}}%
\pgfpathlineto{\pgfqpoint{2.364308in}{0.949517in}}%
\pgfpathlineto{\pgfqpoint{2.366042in}{0.994778in}}%
\pgfpathlineto{\pgfqpoint{2.366909in}{1.388820in}}%
\pgfpathlineto{\pgfqpoint{2.367776in}{1.331013in}}%
\pgfpathlineto{\pgfqpoint{2.368643in}{0.975006in}}%
\pgfpathlineto{\pgfqpoint{2.369510in}{1.071932in}}%
\pgfpathlineto{\pgfqpoint{2.370378in}{1.016834in}}%
\pgfpathlineto{\pgfqpoint{2.371245in}{0.729415in}}%
\pgfpathlineto{\pgfqpoint{2.372112in}{0.914598in}}%
\pgfpathlineto{\pgfqpoint{2.372979in}{0.841849in}}%
\pgfpathlineto{\pgfqpoint{2.373846in}{1.468007in}}%
\pgfpathlineto{\pgfqpoint{2.374713in}{1.048564in}}%
\pgfpathlineto{\pgfqpoint{2.376448in}{1.320466in}}%
\pgfpathlineto{\pgfqpoint{2.377315in}{0.876344in}}%
\pgfpathlineto{\pgfqpoint{2.378182in}{0.933477in}}%
\pgfpathlineto{\pgfqpoint{2.379049in}{1.262437in}}%
\pgfpathlineto{\pgfqpoint{2.379916in}{0.884929in}}%
\pgfpathlineto{\pgfqpoint{2.380783in}{1.023036in}}%
\pgfpathlineto{\pgfqpoint{2.381650in}{0.675508in}}%
\pgfpathlineto{\pgfqpoint{2.383385in}{1.142762in}}%
\pgfpathlineto{\pgfqpoint{2.384252in}{1.026563in}}%
\pgfpathlineto{\pgfqpoint{2.385119in}{1.236526in}}%
\pgfpathlineto{\pgfqpoint{2.385986in}{1.196393in}}%
\pgfpathlineto{\pgfqpoint{2.386853in}{1.200603in}}%
\pgfpathlineto{\pgfqpoint{2.388587in}{1.315658in}}%
\pgfpathlineto{\pgfqpoint{2.389455in}{1.497900in}}%
\pgfpathlineto{\pgfqpoint{2.390322in}{0.806827in}}%
\pgfpathlineto{\pgfqpoint{2.392923in}{1.409313in}}%
\pgfpathlineto{\pgfqpoint{2.394657in}{0.948945in}}%
\pgfpathlineto{\pgfqpoint{2.395524in}{1.003790in}}%
\pgfpathlineto{\pgfqpoint{2.396392in}{0.716490in}}%
\pgfpathlineto{\pgfqpoint{2.397259in}{0.812985in}}%
\pgfpathlineto{\pgfqpoint{2.398126in}{1.208211in}}%
\pgfpathlineto{\pgfqpoint{2.398993in}{0.818346in}}%
\pgfpathlineto{\pgfqpoint{2.400727in}{1.325784in}}%
\pgfpathlineto{\pgfqpoint{2.401594in}{1.048725in}}%
\pgfpathlineto{\pgfqpoint{2.403329in}{1.207410in}}%
\pgfpathlineto{\pgfqpoint{2.404196in}{0.792958in}}%
\pgfpathlineto{\pgfqpoint{2.405063in}{1.269538in}}%
\pgfpathlineto{\pgfqpoint{2.406797in}{0.833573in}}%
\pgfpathlineto{\pgfqpoint{2.407664in}{1.046918in}}%
\pgfpathlineto{\pgfqpoint{2.408531in}{0.874558in}}%
\pgfpathlineto{\pgfqpoint{2.409399in}{1.195466in}}%
\pgfpathlineto{\pgfqpoint{2.410266in}{0.891416in}}%
\pgfpathlineto{\pgfqpoint{2.411133in}{1.246970in}}%
\pgfpathlineto{\pgfqpoint{2.412000in}{1.083070in}}%
\pgfpathlineto{\pgfqpoint{2.412867in}{1.152918in}}%
\pgfpathlineto{\pgfqpoint{2.413734in}{1.364631in}}%
\pgfpathlineto{\pgfqpoint{2.414601in}{1.001459in}}%
\pgfpathlineto{\pgfqpoint{2.415469in}{1.593863in}}%
\pgfpathlineto{\pgfqpoint{2.417203in}{0.736419in}}%
\pgfpathlineto{\pgfqpoint{2.418070in}{1.231766in}}%
\pgfpathlineto{\pgfqpoint{2.418937in}{0.784726in}}%
\pgfpathlineto{\pgfqpoint{2.419804in}{0.796762in}}%
\pgfpathlineto{\pgfqpoint{2.420671in}{1.548124in}}%
\pgfpathlineto{\pgfqpoint{2.421538in}{1.115720in}}%
\pgfpathlineto{\pgfqpoint{2.422406in}{1.203179in}}%
\pgfpathlineto{\pgfqpoint{2.423273in}{1.057176in}}%
\pgfpathlineto{\pgfqpoint{2.424140in}{1.271691in}}%
\pgfpathlineto{\pgfqpoint{2.425007in}{0.770792in}}%
\pgfpathlineto{\pgfqpoint{2.425874in}{1.368887in}}%
\pgfpathlineto{\pgfqpoint{2.426741in}{1.297680in}}%
\pgfpathlineto{\pgfqpoint{2.427608in}{1.084275in}}%
\pgfpathlineto{\pgfqpoint{2.428476in}{1.232764in}}%
\pgfpathlineto{\pgfqpoint{2.429343in}{1.023866in}}%
\pgfpathlineto{\pgfqpoint{2.430210in}{1.044441in}}%
\pgfpathlineto{\pgfqpoint{2.431077in}{1.555316in}}%
\pgfpathlineto{\pgfqpoint{2.431944in}{0.980889in}}%
\pgfpathlineto{\pgfqpoint{2.432811in}{1.371551in}}%
\pgfpathlineto{\pgfqpoint{2.434545in}{0.903393in}}%
\pgfpathlineto{\pgfqpoint{2.435413in}{1.288933in}}%
\pgfpathlineto{\pgfqpoint{2.436280in}{1.133891in}}%
\pgfpathlineto{\pgfqpoint{2.437147in}{1.304145in}}%
\pgfpathlineto{\pgfqpoint{2.439748in}{0.888623in}}%
\pgfpathlineto{\pgfqpoint{2.440615in}{0.849366in}}%
\pgfpathlineto{\pgfqpoint{2.441483in}{1.206238in}}%
\pgfpathlineto{\pgfqpoint{2.442350in}{0.870708in}}%
\pgfpathlineto{\pgfqpoint{2.444084in}{1.541314in}}%
\pgfpathlineto{\pgfqpoint{2.445818in}{0.990430in}}%
\pgfpathlineto{\pgfqpoint{2.446685in}{1.115430in}}%
\pgfpathlineto{\pgfqpoint{2.447552in}{0.904816in}}%
\pgfpathlineto{\pgfqpoint{2.448420in}{1.294954in}}%
\pgfpathlineto{\pgfqpoint{2.449287in}{1.084618in}}%
\pgfpathlineto{\pgfqpoint{2.450154in}{1.222650in}}%
\pgfpathlineto{\pgfqpoint{2.451021in}{0.623935in}}%
\pgfpathlineto{\pgfqpoint{2.454490in}{1.707198in}}%
\pgfpathlineto{\pgfqpoint{2.456224in}{0.832584in}}%
\pgfpathlineto{\pgfqpoint{2.457091in}{0.751867in}}%
\pgfpathlineto{\pgfqpoint{2.457958in}{0.898382in}}%
\pgfpathlineto{\pgfqpoint{2.458825in}{0.812226in}}%
\pgfpathlineto{\pgfqpoint{2.460559in}{1.202308in}}%
\pgfpathlineto{\pgfqpoint{2.461427in}{1.262778in}}%
\pgfpathlineto{\pgfqpoint{2.463161in}{0.921630in}}%
\pgfpathlineto{\pgfqpoint{2.465762in}{1.467136in}}%
\pgfpathlineto{\pgfqpoint{2.466629in}{1.123756in}}%
\pgfpathlineto{\pgfqpoint{2.467497in}{1.187969in}}%
\pgfpathlineto{\pgfqpoint{2.468364in}{1.765679in}}%
\pgfpathlineto{\pgfqpoint{2.469231in}{0.819223in}}%
\pgfpathlineto{\pgfqpoint{2.470098in}{1.056486in}}%
\pgfpathlineto{\pgfqpoint{2.470965in}{1.095410in}}%
\pgfpathlineto{\pgfqpoint{2.471832in}{0.935486in}}%
\pgfpathlineto{\pgfqpoint{2.472699in}{0.968100in}}%
\pgfpathlineto{\pgfqpoint{2.473566in}{1.355643in}}%
\pgfpathlineto{\pgfqpoint{2.474434in}{1.135028in}}%
\pgfpathlineto{\pgfqpoint{2.476168in}{1.388182in}}%
\pgfpathlineto{\pgfqpoint{2.477035in}{0.981116in}}%
\pgfpathlineto{\pgfqpoint{2.477902in}{1.485315in}}%
\pgfpathlineto{\pgfqpoint{2.479636in}{0.694003in}}%
\pgfpathlineto{\pgfqpoint{2.480503in}{0.809839in}}%
\pgfpathlineto{\pgfqpoint{2.481371in}{1.335127in}}%
\pgfpathlineto{\pgfqpoint{2.483105in}{0.873577in}}%
\pgfpathlineto{\pgfqpoint{2.483972in}{0.858023in}}%
\pgfpathlineto{\pgfqpoint{2.484839in}{1.217468in}}%
\pgfpathlineto{\pgfqpoint{2.486573in}{0.891211in}}%
\pgfpathlineto{\pgfqpoint{2.488308in}{1.072341in}}%
\pgfpathlineto{\pgfqpoint{2.489175in}{0.786410in}}%
\pgfpathlineto{\pgfqpoint{2.490042in}{1.774170in}}%
\pgfpathlineto{\pgfqpoint{2.490909in}{1.508159in}}%
\pgfpathlineto{\pgfqpoint{2.492643in}{1.340638in}}%
\pgfpathlineto{\pgfqpoint{2.493510in}{1.488442in}}%
\pgfpathlineto{\pgfqpoint{2.496112in}{0.935371in}}%
\pgfpathlineto{\pgfqpoint{2.496979in}{1.005530in}}%
\pgfpathlineto{\pgfqpoint{2.497846in}{1.213593in}}%
\pgfpathlineto{\pgfqpoint{2.498713in}{0.993691in}}%
\pgfpathlineto{\pgfqpoint{2.499580in}{1.235563in}}%
\pgfpathlineto{\pgfqpoint{2.502182in}{0.734659in}}%
\pgfpathlineto{\pgfqpoint{2.503049in}{0.884544in}}%
\pgfpathlineto{\pgfqpoint{2.503916in}{1.601120in}}%
\pgfpathlineto{\pgfqpoint{2.505650in}{0.976666in}}%
\pgfpathlineto{\pgfqpoint{2.506517in}{1.003785in}}%
\pgfpathlineto{\pgfqpoint{2.507385in}{0.875230in}}%
\pgfpathlineto{\pgfqpoint{2.508252in}{1.374789in}}%
\pgfpathlineto{\pgfqpoint{2.509119in}{0.939023in}}%
\pgfpathlineto{\pgfqpoint{2.509986in}{1.653043in}}%
\pgfpathlineto{\pgfqpoint{2.510853in}{0.761468in}}%
\pgfpathlineto{\pgfqpoint{2.511720in}{0.842150in}}%
\pgfpathlineto{\pgfqpoint{2.512587in}{1.734177in}}%
\pgfpathlineto{\pgfqpoint{2.514322in}{1.085893in}}%
\pgfpathlineto{\pgfqpoint{2.516056in}{1.000292in}}%
\pgfpathlineto{\pgfqpoint{2.516923in}{1.513027in}}%
\pgfpathlineto{\pgfqpoint{2.517790in}{0.831071in}}%
\pgfpathlineto{\pgfqpoint{2.519524in}{1.599635in}}%
\pgfpathlineto{\pgfqpoint{2.520392in}{1.229020in}}%
\pgfpathlineto{\pgfqpoint{2.521259in}{1.477810in}}%
\pgfpathlineto{\pgfqpoint{2.522126in}{0.890501in}}%
\pgfpathlineto{\pgfqpoint{2.522993in}{1.437506in}}%
\pgfpathlineto{\pgfqpoint{2.523860in}{1.334553in}}%
\pgfpathlineto{\pgfqpoint{2.524727in}{1.075937in}}%
\pgfpathlineto{\pgfqpoint{2.525594in}{1.417135in}}%
\pgfpathlineto{\pgfqpoint{2.526462in}{1.203315in}}%
\pgfpathlineto{\pgfqpoint{2.527329in}{1.230189in}}%
\pgfpathlineto{\pgfqpoint{2.528196in}{1.235366in}}%
\pgfpathlineto{\pgfqpoint{2.529063in}{1.182009in}}%
\pgfpathlineto{\pgfqpoint{2.529930in}{1.228522in}}%
\pgfpathlineto{\pgfqpoint{2.531664in}{0.919068in}}%
\pgfpathlineto{\pgfqpoint{2.532531in}{1.570968in}}%
\pgfpathlineto{\pgfqpoint{2.535133in}{0.880295in}}%
\pgfpathlineto{\pgfqpoint{2.536000in}{1.319868in}}%
\pgfpathlineto{\pgfqpoint{2.536867in}{0.790535in}}%
\pgfpathlineto{\pgfqpoint{2.538601in}{1.387738in}}%
\pgfpathlineto{\pgfqpoint{2.539469in}{0.851919in}}%
\pgfpathlineto{\pgfqpoint{2.540336in}{1.336631in}}%
\pgfpathlineto{\pgfqpoint{2.541203in}{1.291968in}}%
\pgfpathlineto{\pgfqpoint{2.542070in}{1.450950in}}%
\pgfpathlineto{\pgfqpoint{2.545538in}{0.837451in}}%
\pgfpathlineto{\pgfqpoint{2.547273in}{1.523089in}}%
\pgfpathlineto{\pgfqpoint{2.548140in}{1.618575in}}%
\pgfpathlineto{\pgfqpoint{2.549874in}{0.718798in}}%
\pgfpathlineto{\pgfqpoint{2.550741in}{1.281143in}}%
\pgfpathlineto{\pgfqpoint{2.551608in}{1.236686in}}%
\pgfpathlineto{\pgfqpoint{2.552476in}{1.471536in}}%
\pgfpathlineto{\pgfqpoint{2.553343in}{0.857166in}}%
\pgfpathlineto{\pgfqpoint{2.554210in}{1.392773in}}%
\pgfpathlineto{\pgfqpoint{2.555077in}{1.243024in}}%
\pgfpathlineto{\pgfqpoint{2.555944in}{1.635641in}}%
\pgfpathlineto{\pgfqpoint{2.557678in}{0.738754in}}%
\pgfpathlineto{\pgfqpoint{2.558545in}{1.029463in}}%
\pgfpathlineto{\pgfqpoint{2.559413in}{0.756469in}}%
\pgfpathlineto{\pgfqpoint{2.560280in}{1.488386in}}%
\pgfpathlineto{\pgfqpoint{2.561147in}{0.711736in}}%
\pgfpathlineto{\pgfqpoint{2.562014in}{0.954235in}}%
\pgfpathlineto{\pgfqpoint{2.562881in}{1.602695in}}%
\pgfpathlineto{\pgfqpoint{2.563748in}{1.558011in}}%
\pgfpathlineto{\pgfqpoint{2.565483in}{1.068675in}}%
\pgfpathlineto{\pgfqpoint{2.567217in}{0.855439in}}%
\pgfpathlineto{\pgfqpoint{2.568084in}{1.272247in}}%
\pgfpathlineto{\pgfqpoint{2.569818in}{0.698783in}}%
\pgfpathlineto{\pgfqpoint{2.571552in}{1.307816in}}%
\pgfpathlineto{\pgfqpoint{2.572420in}{0.860612in}}%
\pgfpathlineto{\pgfqpoint{2.573287in}{1.060836in}}%
\pgfpathlineto{\pgfqpoint{2.574154in}{1.659158in}}%
\pgfpathlineto{\pgfqpoint{2.575888in}{1.135638in}}%
\pgfpathlineto{\pgfqpoint{2.576755in}{1.107869in}}%
\pgfpathlineto{\pgfqpoint{2.577622in}{0.999643in}}%
\pgfpathlineto{\pgfqpoint{2.578490in}{1.732732in}}%
\pgfpathlineto{\pgfqpoint{2.579357in}{0.821927in}}%
\pgfpathlineto{\pgfqpoint{2.581091in}{1.372177in}}%
\pgfpathlineto{\pgfqpoint{2.581958in}{1.307777in}}%
\pgfpathlineto{\pgfqpoint{2.582825in}{1.412360in}}%
\pgfpathlineto{\pgfqpoint{2.584559in}{0.786085in}}%
\pgfpathlineto{\pgfqpoint{2.585427in}{1.343255in}}%
\pgfpathlineto{\pgfqpoint{2.586294in}{1.277478in}}%
\pgfpathlineto{\pgfqpoint{2.587161in}{1.627374in}}%
\pgfpathlineto{\pgfqpoint{2.588895in}{0.811184in}}%
\pgfpathlineto{\pgfqpoint{2.590629in}{1.281367in}}%
\pgfpathlineto{\pgfqpoint{2.591497in}{1.065167in}}%
\pgfpathlineto{\pgfqpoint{2.592364in}{1.302124in}}%
\pgfpathlineto{\pgfqpoint{2.593231in}{1.126739in}}%
\pgfpathlineto{\pgfqpoint{2.594965in}{1.260177in}}%
\pgfpathlineto{\pgfqpoint{2.595832in}{0.939287in}}%
\pgfpathlineto{\pgfqpoint{2.596699in}{1.351674in}}%
\pgfpathlineto{\pgfqpoint{2.597566in}{0.962163in}}%
\pgfpathlineto{\pgfqpoint{2.598434in}{1.431116in}}%
\pgfpathlineto{\pgfqpoint{2.599301in}{0.807950in}}%
\pgfpathlineto{\pgfqpoint{2.600168in}{0.809533in}}%
\pgfpathlineto{\pgfqpoint{2.603636in}{1.256592in}}%
\pgfpathlineto{\pgfqpoint{2.604503in}{1.492981in}}%
\pgfpathlineto{\pgfqpoint{2.606238in}{0.978305in}}%
\pgfpathlineto{\pgfqpoint{2.607105in}{1.532782in}}%
\pgfpathlineto{\pgfqpoint{2.608839in}{0.864305in}}%
\pgfpathlineto{\pgfqpoint{2.609706in}{0.952507in}}%
\pgfpathlineto{\pgfqpoint{2.610573in}{1.457637in}}%
\pgfpathlineto{\pgfqpoint{2.612308in}{0.714697in}}%
\pgfpathlineto{\pgfqpoint{2.613175in}{1.448891in}}%
\pgfpathlineto{\pgfqpoint{2.614042in}{0.933420in}}%
\pgfpathlineto{\pgfqpoint{2.614909in}{1.021372in}}%
\pgfpathlineto{\pgfqpoint{2.616643in}{1.511666in}}%
\pgfpathlineto{\pgfqpoint{2.618378in}{0.843421in}}%
\pgfpathlineto{\pgfqpoint{2.619245in}{0.832308in}}%
\pgfpathlineto{\pgfqpoint{2.620979in}{1.679675in}}%
\pgfpathlineto{\pgfqpoint{2.621846in}{1.462971in}}%
\pgfpathlineto{\pgfqpoint{2.622713in}{0.993879in}}%
\pgfpathlineto{\pgfqpoint{2.623580in}{1.282885in}}%
\pgfpathlineto{\pgfqpoint{2.624448in}{1.181246in}}%
\pgfpathlineto{\pgfqpoint{2.625315in}{1.232659in}}%
\pgfpathlineto{\pgfqpoint{2.627049in}{0.797629in}}%
\pgfpathlineto{\pgfqpoint{2.627916in}{1.581731in}}%
\pgfpathlineto{\pgfqpoint{2.629650in}{0.918413in}}%
\pgfpathlineto{\pgfqpoint{2.630517in}{1.093685in}}%
\pgfpathlineto{\pgfqpoint{2.631385in}{0.838361in}}%
\pgfpathlineto{\pgfqpoint{2.632252in}{1.053750in}}%
\pgfpathlineto{\pgfqpoint{2.633119in}{1.012807in}}%
\pgfpathlineto{\pgfqpoint{2.633986in}{0.831987in}}%
\pgfpathlineto{\pgfqpoint{2.634853in}{0.907971in}}%
\pgfpathlineto{\pgfqpoint{2.635720in}{1.350399in}}%
\pgfpathlineto{\pgfqpoint{2.636587in}{0.876017in}}%
\pgfpathlineto{\pgfqpoint{2.637455in}{1.020345in}}%
\pgfpathlineto{\pgfqpoint{2.639189in}{0.756930in}}%
\pgfpathlineto{\pgfqpoint{2.640056in}{0.868151in}}%
\pgfpathlineto{\pgfqpoint{2.640923in}{0.749134in}}%
\pgfpathlineto{\pgfqpoint{2.641790in}{1.271033in}}%
\pgfpathlineto{\pgfqpoint{2.642657in}{1.152824in}}%
\pgfpathlineto{\pgfqpoint{2.643524in}{0.852839in}}%
\pgfpathlineto{\pgfqpoint{2.646126in}{1.660999in}}%
\pgfpathlineto{\pgfqpoint{2.646993in}{0.788442in}}%
\pgfpathlineto{\pgfqpoint{2.647860in}{1.885964in}}%
\pgfpathlineto{\pgfqpoint{2.649594in}{0.996609in}}%
\pgfpathlineto{\pgfqpoint{2.650462in}{1.051009in}}%
\pgfpathlineto{\pgfqpoint{2.651329in}{0.821224in}}%
\pgfpathlineto{\pgfqpoint{2.652196in}{0.830817in}}%
\pgfpathlineto{\pgfqpoint{2.653063in}{0.703752in}}%
\pgfpathlineto{\pgfqpoint{2.654797in}{0.993355in}}%
\pgfpathlineto{\pgfqpoint{2.655664in}{1.080179in}}%
\pgfpathlineto{\pgfqpoint{2.656531in}{1.351169in}}%
\pgfpathlineto{\pgfqpoint{2.657399in}{1.035767in}}%
\pgfpathlineto{\pgfqpoint{2.658266in}{1.239850in}}%
\pgfpathlineto{\pgfqpoint{2.659133in}{1.177583in}}%
\pgfpathlineto{\pgfqpoint{2.660000in}{1.561795in}}%
\pgfpathlineto{\pgfqpoint{2.661734in}{1.009341in}}%
\pgfpathlineto{\pgfqpoint{2.662601in}{1.156308in}}%
\pgfpathlineto{\pgfqpoint{2.663469in}{1.529289in}}%
\pgfpathlineto{\pgfqpoint{2.665203in}{0.992977in}}%
\pgfpathlineto{\pgfqpoint{2.666070in}{1.490890in}}%
\pgfpathlineto{\pgfqpoint{2.667804in}{0.971220in}}%
\pgfpathlineto{\pgfqpoint{2.668671in}{0.889406in}}%
\pgfpathlineto{\pgfqpoint{2.669538in}{1.168265in}}%
\pgfpathlineto{\pgfqpoint{2.671273in}{0.736301in}}%
\pgfpathlineto{\pgfqpoint{2.672140in}{1.597939in}}%
\pgfpathlineto{\pgfqpoint{2.673007in}{0.910009in}}%
\pgfpathlineto{\pgfqpoint{2.673874in}{1.282268in}}%
\pgfpathlineto{\pgfqpoint{2.674741in}{1.240259in}}%
\pgfpathlineto{\pgfqpoint{2.677343in}{0.794735in}}%
\pgfpathlineto{\pgfqpoint{2.678210in}{1.085595in}}%
\pgfpathlineto{\pgfqpoint{2.679944in}{0.830720in}}%
\pgfpathlineto{\pgfqpoint{2.682545in}{1.230876in}}%
\pgfpathlineto{\pgfqpoint{2.683413in}{1.030295in}}%
\pgfpathlineto{\pgfqpoint{2.684280in}{1.460513in}}%
\pgfpathlineto{\pgfqpoint{2.685147in}{1.341461in}}%
\pgfpathlineto{\pgfqpoint{2.686014in}{1.391155in}}%
\pgfpathlineto{\pgfqpoint{2.686881in}{1.162672in}}%
\pgfpathlineto{\pgfqpoint{2.687748in}{1.320735in}}%
\pgfpathlineto{\pgfqpoint{2.688615in}{1.258448in}}%
\pgfpathlineto{\pgfqpoint{2.689483in}{0.881708in}}%
\pgfpathlineto{\pgfqpoint{2.690350in}{1.242293in}}%
\pgfpathlineto{\pgfqpoint{2.692951in}{0.909065in}}%
\pgfpathlineto{\pgfqpoint{2.693818in}{1.288921in}}%
\pgfpathlineto{\pgfqpoint{2.694685in}{0.837065in}}%
\pgfpathlineto{\pgfqpoint{2.695552in}{0.954953in}}%
\pgfpathlineto{\pgfqpoint{2.696420in}{1.442923in}}%
\pgfpathlineto{\pgfqpoint{2.697287in}{0.775632in}}%
\pgfpathlineto{\pgfqpoint{2.698154in}{1.523755in}}%
\pgfpathlineto{\pgfqpoint{2.699888in}{1.076344in}}%
\pgfpathlineto{\pgfqpoint{2.700755in}{0.803525in}}%
\pgfpathlineto{\pgfqpoint{2.702490in}{1.338651in}}%
\pgfpathlineto{\pgfqpoint{2.703357in}{1.207438in}}%
\pgfpathlineto{\pgfqpoint{2.704224in}{0.855394in}}%
\pgfpathlineto{\pgfqpoint{2.705091in}{1.739932in}}%
\pgfpathlineto{\pgfqpoint{2.706825in}{0.960717in}}%
\pgfpathlineto{\pgfqpoint{2.708559in}{1.380527in}}%
\pgfpathlineto{\pgfqpoint{2.710294in}{0.864968in}}%
\pgfpathlineto{\pgfqpoint{2.711161in}{1.042965in}}%
\pgfpathlineto{\pgfqpoint{2.712028in}{0.652267in}}%
\pgfpathlineto{\pgfqpoint{2.713762in}{1.019561in}}%
\pgfpathlineto{\pgfqpoint{2.716364in}{0.848884in}}%
\pgfpathlineto{\pgfqpoint{2.717231in}{1.728699in}}%
\pgfpathlineto{\pgfqpoint{2.719832in}{0.810783in}}%
\pgfpathlineto{\pgfqpoint{2.720699in}{0.840619in}}%
\pgfpathlineto{\pgfqpoint{2.721566in}{1.228880in}}%
\pgfpathlineto{\pgfqpoint{2.723301in}{0.716874in}}%
\pgfpathlineto{\pgfqpoint{2.724168in}{1.342019in}}%
\pgfpathlineto{\pgfqpoint{2.725035in}{1.236623in}}%
\pgfpathlineto{\pgfqpoint{2.725902in}{1.333440in}}%
\pgfpathlineto{\pgfqpoint{2.726769in}{0.860329in}}%
\pgfpathlineto{\pgfqpoint{2.727636in}{1.172691in}}%
\pgfpathlineto{\pgfqpoint{2.729371in}{0.771082in}}%
\pgfpathlineto{\pgfqpoint{2.730238in}{0.818883in}}%
\pgfpathlineto{\pgfqpoint{2.731105in}{1.506289in}}%
\pgfpathlineto{\pgfqpoint{2.731972in}{0.703500in}}%
\pgfpathlineto{\pgfqpoint{2.732839in}{1.673570in}}%
\pgfpathlineto{\pgfqpoint{2.733706in}{0.681406in}}%
\pgfpathlineto{\pgfqpoint{2.734573in}{0.935514in}}%
\pgfpathlineto{\pgfqpoint{2.735441in}{0.797095in}}%
\pgfpathlineto{\pgfqpoint{2.736308in}{0.799550in}}%
\pgfpathlineto{\pgfqpoint{2.737175in}{0.850022in}}%
\pgfpathlineto{\pgfqpoint{2.738042in}{1.615652in}}%
\pgfpathlineto{\pgfqpoint{2.738909in}{1.170411in}}%
\pgfpathlineto{\pgfqpoint{2.739776in}{1.414714in}}%
\pgfpathlineto{\pgfqpoint{2.740643in}{1.039414in}}%
\pgfpathlineto{\pgfqpoint{2.741510in}{1.587696in}}%
\pgfpathlineto{\pgfqpoint{2.743245in}{1.012902in}}%
\pgfpathlineto{\pgfqpoint{2.744112in}{1.127760in}}%
\pgfpathlineto{\pgfqpoint{2.745846in}{0.982919in}}%
\pgfpathlineto{\pgfqpoint{2.747580in}{1.451587in}}%
\pgfpathlineto{\pgfqpoint{2.748448in}{0.849393in}}%
\pgfpathlineto{\pgfqpoint{2.749315in}{0.889907in}}%
\pgfpathlineto{\pgfqpoint{2.750182in}{0.814759in}}%
\pgfpathlineto{\pgfqpoint{2.751049in}{1.154023in}}%
\pgfpathlineto{\pgfqpoint{2.751916in}{0.741342in}}%
\pgfpathlineto{\pgfqpoint{2.752783in}{0.932709in}}%
\pgfpathlineto{\pgfqpoint{2.754517in}{0.741792in}}%
\pgfpathlineto{\pgfqpoint{2.755385in}{0.881557in}}%
\pgfpathlineto{\pgfqpoint{2.757119in}{1.584003in}}%
\pgfpathlineto{\pgfqpoint{2.757986in}{0.991876in}}%
\pgfpathlineto{\pgfqpoint{2.758853in}{1.085054in}}%
\pgfpathlineto{\pgfqpoint{2.759720in}{0.947707in}}%
\pgfpathlineto{\pgfqpoint{2.760587in}{1.043987in}}%
\pgfpathlineto{\pgfqpoint{2.761455in}{0.735364in}}%
\pgfpathlineto{\pgfqpoint{2.762322in}{1.246656in}}%
\pgfpathlineto{\pgfqpoint{2.764056in}{1.023498in}}%
\pgfpathlineto{\pgfqpoint{2.765790in}{1.664109in}}%
\pgfpathlineto{\pgfqpoint{2.766657in}{0.699750in}}%
\pgfpathlineto{\pgfqpoint{2.769259in}{1.444548in}}%
\pgfpathlineto{\pgfqpoint{2.770993in}{0.679112in}}%
\pgfpathlineto{\pgfqpoint{2.771860in}{1.171299in}}%
\pgfpathlineto{\pgfqpoint{2.773594in}{0.663748in}}%
\pgfpathlineto{\pgfqpoint{2.775329in}{1.647215in}}%
\pgfpathlineto{\pgfqpoint{2.776196in}{0.782549in}}%
\pgfpathlineto{\pgfqpoint{2.777063in}{1.128828in}}%
\pgfpathlineto{\pgfqpoint{2.777930in}{0.653343in}}%
\pgfpathlineto{\pgfqpoint{2.779664in}{0.995931in}}%
\pgfpathlineto{\pgfqpoint{2.780531in}{0.953846in}}%
\pgfpathlineto{\pgfqpoint{2.781399in}{0.971983in}}%
\pgfpathlineto{\pgfqpoint{2.782266in}{1.247326in}}%
\pgfpathlineto{\pgfqpoint{2.783133in}{0.831884in}}%
\pgfpathlineto{\pgfqpoint{2.784000in}{1.079396in}}%
\pgfpathlineto{\pgfqpoint{2.784867in}{0.811897in}}%
\pgfpathlineto{\pgfqpoint{2.785734in}{1.040843in}}%
\pgfpathlineto{\pgfqpoint{2.786601in}{0.851487in}}%
\pgfpathlineto{\pgfqpoint{2.787469in}{1.317246in}}%
\pgfpathlineto{\pgfqpoint{2.788336in}{1.204299in}}%
\pgfpathlineto{\pgfqpoint{2.789203in}{1.289555in}}%
\pgfpathlineto{\pgfqpoint{2.790937in}{1.017083in}}%
\pgfpathlineto{\pgfqpoint{2.791804in}{1.078665in}}%
\pgfpathlineto{\pgfqpoint{2.792671in}{1.376207in}}%
\pgfpathlineto{\pgfqpoint{2.793538in}{0.724006in}}%
\pgfpathlineto{\pgfqpoint{2.795273in}{1.386727in}}%
\pgfpathlineto{\pgfqpoint{2.796140in}{0.749628in}}%
\pgfpathlineto{\pgfqpoint{2.797007in}{1.623026in}}%
\pgfpathlineto{\pgfqpoint{2.797874in}{1.007141in}}%
\pgfpathlineto{\pgfqpoint{2.798741in}{1.509317in}}%
\pgfpathlineto{\pgfqpoint{2.799608in}{0.799431in}}%
\pgfpathlineto{\pgfqpoint{2.800476in}{1.370206in}}%
\pgfpathlineto{\pgfqpoint{2.801343in}{0.907221in}}%
\pgfpathlineto{\pgfqpoint{2.802210in}{1.762379in}}%
\pgfpathlineto{\pgfqpoint{2.803944in}{0.814862in}}%
\pgfpathlineto{\pgfqpoint{2.804811in}{1.096848in}}%
\pgfpathlineto{\pgfqpoint{2.806545in}{0.790651in}}%
\pgfpathlineto{\pgfqpoint{2.807413in}{1.213729in}}%
\pgfpathlineto{\pgfqpoint{2.808280in}{0.625952in}}%
\pgfpathlineto{\pgfqpoint{2.810881in}{1.535875in}}%
\pgfpathlineto{\pgfqpoint{2.811748in}{0.910602in}}%
\pgfpathlineto{\pgfqpoint{2.812615in}{1.028160in}}%
\pgfpathlineto{\pgfqpoint{2.814350in}{1.461484in}}%
\pgfpathlineto{\pgfqpoint{2.815217in}{1.048742in}}%
\pgfpathlineto{\pgfqpoint{2.816084in}{1.419103in}}%
\pgfpathlineto{\pgfqpoint{2.816951in}{0.797238in}}%
\pgfpathlineto{\pgfqpoint{2.817818in}{1.664886in}}%
\pgfpathlineto{\pgfqpoint{2.819552in}{0.766914in}}%
\pgfpathlineto{\pgfqpoint{2.820420in}{1.684500in}}%
\pgfpathlineto{\pgfqpoint{2.821287in}{1.069385in}}%
\pgfpathlineto{\pgfqpoint{2.822154in}{1.310225in}}%
\pgfpathlineto{\pgfqpoint{2.823021in}{0.958898in}}%
\pgfpathlineto{\pgfqpoint{2.823888in}{1.016045in}}%
\pgfpathlineto{\pgfqpoint{2.824755in}{1.129645in}}%
\pgfpathlineto{\pgfqpoint{2.825622in}{1.044951in}}%
\pgfpathlineto{\pgfqpoint{2.826490in}{1.116379in}}%
\pgfpathlineto{\pgfqpoint{2.828224in}{0.680247in}}%
\pgfpathlineto{\pgfqpoint{2.829958in}{1.089354in}}%
\pgfpathlineto{\pgfqpoint{2.830825in}{1.699586in}}%
\pgfpathlineto{\pgfqpoint{2.832559in}{0.792703in}}%
\pgfpathlineto{\pgfqpoint{2.833427in}{1.185495in}}%
\pgfpathlineto{\pgfqpoint{2.834294in}{1.024153in}}%
\pgfpathlineto{\pgfqpoint{2.835161in}{1.070640in}}%
\pgfpathlineto{\pgfqpoint{2.836028in}{1.493093in}}%
\pgfpathlineto{\pgfqpoint{2.836895in}{0.715160in}}%
\pgfpathlineto{\pgfqpoint{2.837762in}{1.206860in}}%
\pgfpathlineto{\pgfqpoint{2.838629in}{1.022519in}}%
\pgfpathlineto{\pgfqpoint{2.839497in}{1.124548in}}%
\pgfpathlineto{\pgfqpoint{2.841231in}{0.815298in}}%
\pgfpathlineto{\pgfqpoint{2.842098in}{1.326730in}}%
\pgfpathlineto{\pgfqpoint{2.842965in}{0.784912in}}%
\pgfpathlineto{\pgfqpoint{2.843832in}{0.915906in}}%
\pgfpathlineto{\pgfqpoint{2.844699in}{1.121034in}}%
\pgfpathlineto{\pgfqpoint{2.845566in}{0.908392in}}%
\pgfpathlineto{\pgfqpoint{2.846434in}{1.374147in}}%
\pgfpathlineto{\pgfqpoint{2.847301in}{1.318065in}}%
\pgfpathlineto{\pgfqpoint{2.848168in}{1.001973in}}%
\pgfpathlineto{\pgfqpoint{2.849035in}{1.288236in}}%
\pgfpathlineto{\pgfqpoint{2.850769in}{0.957171in}}%
\pgfpathlineto{\pgfqpoint{2.852503in}{1.430483in}}%
\pgfpathlineto{\pgfqpoint{2.854238in}{0.644915in}}%
\pgfpathlineto{\pgfqpoint{2.855105in}{0.783208in}}%
\pgfpathlineto{\pgfqpoint{2.855972in}{1.254037in}}%
\pgfpathlineto{\pgfqpoint{2.856839in}{0.641355in}}%
\pgfpathlineto{\pgfqpoint{2.858573in}{1.362923in}}%
\pgfpathlineto{\pgfqpoint{2.859441in}{1.121982in}}%
\pgfpathlineto{\pgfqpoint{2.860308in}{1.089698in}}%
\pgfpathlineto{\pgfqpoint{2.861175in}{0.888343in}}%
\pgfpathlineto{\pgfqpoint{2.862042in}{0.891316in}}%
\pgfpathlineto{\pgfqpoint{2.862909in}{1.793618in}}%
\pgfpathlineto{\pgfqpoint{2.863776in}{1.574768in}}%
\pgfpathlineto{\pgfqpoint{2.864643in}{0.871608in}}%
\pgfpathlineto{\pgfqpoint{2.865510in}{1.325202in}}%
\pgfpathlineto{\pgfqpoint{2.867245in}{0.748160in}}%
\pgfpathlineto{\pgfqpoint{2.868979in}{1.253728in}}%
\pgfpathlineto{\pgfqpoint{2.869846in}{1.210505in}}%
\pgfpathlineto{\pgfqpoint{2.870713in}{0.879695in}}%
\pgfpathlineto{\pgfqpoint{2.871580in}{0.970467in}}%
\pgfpathlineto{\pgfqpoint{2.872448in}{1.098190in}}%
\pgfpathlineto{\pgfqpoint{2.873315in}{0.930861in}}%
\pgfpathlineto{\pgfqpoint{2.874182in}{1.167213in}}%
\pgfpathlineto{\pgfqpoint{2.875049in}{0.906226in}}%
\pgfpathlineto{\pgfqpoint{2.875916in}{1.651813in}}%
\pgfpathlineto{\pgfqpoint{2.876783in}{0.731421in}}%
\pgfpathlineto{\pgfqpoint{2.877650in}{1.140925in}}%
\pgfpathlineto{\pgfqpoint{2.878517in}{1.053823in}}%
\pgfpathlineto{\pgfqpoint{2.879385in}{1.282813in}}%
\pgfpathlineto{\pgfqpoint{2.880252in}{1.227439in}}%
\pgfpathlineto{\pgfqpoint{2.881119in}{1.159065in}}%
\pgfpathlineto{\pgfqpoint{2.881986in}{0.886219in}}%
\pgfpathlineto{\pgfqpoint{2.882853in}{1.088144in}}%
\pgfpathlineto{\pgfqpoint{2.883720in}{0.980901in}}%
\pgfpathlineto{\pgfqpoint{2.884587in}{1.526659in}}%
\pgfpathlineto{\pgfqpoint{2.886322in}{0.832853in}}%
\pgfpathlineto{\pgfqpoint{2.888923in}{1.606777in}}%
\pgfpathlineto{\pgfqpoint{2.889790in}{1.344640in}}%
\pgfpathlineto{\pgfqpoint{2.890657in}{1.563172in}}%
\pgfpathlineto{\pgfqpoint{2.892392in}{0.853794in}}%
\pgfpathlineto{\pgfqpoint{2.893259in}{1.201378in}}%
\pgfpathlineto{\pgfqpoint{2.894126in}{0.967269in}}%
\pgfpathlineto{\pgfqpoint{2.894993in}{1.878746in}}%
\pgfpathlineto{\pgfqpoint{2.896727in}{0.895270in}}%
\pgfpathlineto{\pgfqpoint{2.897594in}{0.718219in}}%
\pgfpathlineto{\pgfqpoint{2.898462in}{1.310618in}}%
\pgfpathlineto{\pgfqpoint{2.899329in}{1.243708in}}%
\pgfpathlineto{\pgfqpoint{2.900196in}{0.971909in}}%
\pgfpathlineto{\pgfqpoint{2.901063in}{1.146293in}}%
\pgfpathlineto{\pgfqpoint{2.901930in}{1.092459in}}%
\pgfpathlineto{\pgfqpoint{2.902797in}{1.174031in}}%
\pgfpathlineto{\pgfqpoint{2.904531in}{0.977372in}}%
\pgfpathlineto{\pgfqpoint{2.905399in}{1.560667in}}%
\pgfpathlineto{\pgfqpoint{2.906266in}{0.910459in}}%
\pgfpathlineto{\pgfqpoint{2.907133in}{1.617843in}}%
\pgfpathlineto{\pgfqpoint{2.908000in}{1.099114in}}%
\pgfpathlineto{\pgfqpoint{2.908867in}{1.908471in}}%
\pgfpathlineto{\pgfqpoint{2.909734in}{0.931207in}}%
\pgfpathlineto{\pgfqpoint{2.910601in}{1.237768in}}%
\pgfpathlineto{\pgfqpoint{2.911469in}{0.905821in}}%
\pgfpathlineto{\pgfqpoint{2.912336in}{1.420603in}}%
\pgfpathlineto{\pgfqpoint{2.913203in}{1.052204in}}%
\pgfpathlineto{\pgfqpoint{2.914070in}{1.184978in}}%
\pgfpathlineto{\pgfqpoint{2.914937in}{1.550697in}}%
\pgfpathlineto{\pgfqpoint{2.916671in}{0.926672in}}%
\pgfpathlineto{\pgfqpoint{2.917538in}{1.117263in}}%
\pgfpathlineto{\pgfqpoint{2.918406in}{1.043424in}}%
\pgfpathlineto{\pgfqpoint{2.919273in}{1.320493in}}%
\pgfpathlineto{\pgfqpoint{2.920140in}{1.161072in}}%
\pgfpathlineto{\pgfqpoint{2.921007in}{1.338777in}}%
\pgfpathlineto{\pgfqpoint{2.921874in}{0.868995in}}%
\pgfpathlineto{\pgfqpoint{2.922741in}{1.017431in}}%
\pgfpathlineto{\pgfqpoint{2.923608in}{1.369237in}}%
\pgfpathlineto{\pgfqpoint{2.925343in}{0.896078in}}%
\pgfpathlineto{\pgfqpoint{2.926210in}{1.545030in}}%
\pgfpathlineto{\pgfqpoint{2.927077in}{0.845853in}}%
\pgfpathlineto{\pgfqpoint{2.928811in}{1.317624in}}%
\pgfpathlineto{\pgfqpoint{2.929678in}{1.179589in}}%
\pgfpathlineto{\pgfqpoint{2.931413in}{0.671020in}}%
\pgfpathlineto{\pgfqpoint{2.932280in}{1.069934in}}%
\pgfpathlineto{\pgfqpoint{2.933147in}{0.861983in}}%
\pgfpathlineto{\pgfqpoint{2.934881in}{1.392465in}}%
\pgfpathlineto{\pgfqpoint{2.935748in}{1.040449in}}%
\pgfpathlineto{\pgfqpoint{2.937483in}{1.373013in}}%
\pgfpathlineto{\pgfqpoint{2.939217in}{0.968239in}}%
\pgfpathlineto{\pgfqpoint{2.940084in}{1.769595in}}%
\pgfpathlineto{\pgfqpoint{2.941818in}{0.778775in}}%
\pgfpathlineto{\pgfqpoint{2.942685in}{0.974680in}}%
\pgfpathlineto{\pgfqpoint{2.943552in}{0.676318in}}%
\pgfpathlineto{\pgfqpoint{2.944420in}{0.971059in}}%
\pgfpathlineto{\pgfqpoint{2.945287in}{1.883928in}}%
\pgfpathlineto{\pgfqpoint{2.946154in}{1.747114in}}%
\pgfpathlineto{\pgfqpoint{2.947021in}{0.855300in}}%
\pgfpathlineto{\pgfqpoint{2.948755in}{1.384307in}}%
\pgfpathlineto{\pgfqpoint{2.949622in}{1.340188in}}%
\pgfpathlineto{\pgfqpoint{2.951357in}{0.919170in}}%
\pgfpathlineto{\pgfqpoint{2.952224in}{0.866464in}}%
\pgfpathlineto{\pgfqpoint{2.953091in}{1.218408in}}%
\pgfpathlineto{\pgfqpoint{2.954825in}{0.932750in}}%
\pgfpathlineto{\pgfqpoint{2.955692in}{1.585770in}}%
\pgfpathlineto{\pgfqpoint{2.957427in}{0.742973in}}%
\pgfpathlineto{\pgfqpoint{2.959161in}{1.181439in}}%
\pgfpathlineto{\pgfqpoint{2.960028in}{0.840608in}}%
\pgfpathlineto{\pgfqpoint{2.960895in}{1.496099in}}%
\pgfpathlineto{\pgfqpoint{2.961762in}{0.895786in}}%
\pgfpathlineto{\pgfqpoint{2.962629in}{0.939582in}}%
\pgfpathlineto{\pgfqpoint{2.963497in}{1.054366in}}%
\pgfpathlineto{\pgfqpoint{2.964364in}{2.003050in}}%
\pgfpathlineto{\pgfqpoint{2.966098in}{1.101886in}}%
\pgfpathlineto{\pgfqpoint{2.966965in}{1.146825in}}%
\pgfpathlineto{\pgfqpoint{2.967832in}{1.749018in}}%
\pgfpathlineto{\pgfqpoint{2.968699in}{1.623435in}}%
\pgfpathlineto{\pgfqpoint{2.969566in}{0.891294in}}%
\pgfpathlineto{\pgfqpoint{2.970434in}{1.300688in}}%
\pgfpathlineto{\pgfqpoint{2.971301in}{1.255945in}}%
\pgfpathlineto{\pgfqpoint{2.972168in}{0.898104in}}%
\pgfpathlineto{\pgfqpoint{2.973035in}{1.224507in}}%
\pgfpathlineto{\pgfqpoint{2.974769in}{0.874763in}}%
\pgfpathlineto{\pgfqpoint{2.975636in}{1.403236in}}%
\pgfpathlineto{\pgfqpoint{2.976503in}{0.752260in}}%
\pgfpathlineto{\pgfqpoint{2.977371in}{0.983749in}}%
\pgfpathlineto{\pgfqpoint{2.978238in}{0.696899in}}%
\pgfpathlineto{\pgfqpoint{2.979105in}{1.473071in}}%
\pgfpathlineto{\pgfqpoint{2.981706in}{0.702912in}}%
\pgfpathlineto{\pgfqpoint{2.983441in}{1.529578in}}%
\pgfpathlineto{\pgfqpoint{2.984308in}{1.310170in}}%
\pgfpathlineto{\pgfqpoint{2.985175in}{0.956590in}}%
\pgfpathlineto{\pgfqpoint{2.986042in}{1.183184in}}%
\pgfpathlineto{\pgfqpoint{2.986909in}{1.173582in}}%
\pgfpathlineto{\pgfqpoint{2.987776in}{0.755733in}}%
\pgfpathlineto{\pgfqpoint{2.989510in}{1.285548in}}%
\pgfpathlineto{\pgfqpoint{2.990378in}{1.130939in}}%
\pgfpathlineto{\pgfqpoint{2.991245in}{1.203945in}}%
\pgfpathlineto{\pgfqpoint{2.992979in}{0.995901in}}%
\pgfpathlineto{\pgfqpoint{2.993846in}{1.231220in}}%
\pgfpathlineto{\pgfqpoint{2.994713in}{0.801344in}}%
\pgfpathlineto{\pgfqpoint{2.996448in}{1.256700in}}%
\pgfpathlineto{\pgfqpoint{2.998182in}{0.953413in}}%
\pgfpathlineto{\pgfqpoint{2.999049in}{1.167014in}}%
\pgfpathlineto{\pgfqpoint{2.999916in}{2.059720in}}%
\pgfpathlineto{\pgfqpoint{3.000783in}{1.137357in}}%
\pgfpathlineto{\pgfqpoint{3.001650in}{1.761341in}}%
\pgfpathlineto{\pgfqpoint{3.002517in}{0.999139in}}%
\pgfpathlineto{\pgfqpoint{3.003385in}{1.040650in}}%
\pgfpathlineto{\pgfqpoint{3.004252in}{1.105467in}}%
\pgfpathlineto{\pgfqpoint{3.005119in}{1.071220in}}%
\pgfpathlineto{\pgfqpoint{3.005986in}{1.333837in}}%
\pgfpathlineto{\pgfqpoint{3.006853in}{1.292886in}}%
\pgfpathlineto{\pgfqpoint{3.008587in}{0.877475in}}%
\pgfpathlineto{\pgfqpoint{3.009455in}{0.846700in}}%
\pgfpathlineto{\pgfqpoint{3.010322in}{1.608606in}}%
\pgfpathlineto{\pgfqpoint{3.011189in}{0.787995in}}%
\pgfpathlineto{\pgfqpoint{3.012923in}{1.233222in}}%
\pgfpathlineto{\pgfqpoint{3.013790in}{1.322522in}}%
\pgfpathlineto{\pgfqpoint{3.014657in}{0.762972in}}%
\pgfpathlineto{\pgfqpoint{3.015524in}{1.425568in}}%
\pgfpathlineto{\pgfqpoint{3.017259in}{0.917499in}}%
\pgfpathlineto{\pgfqpoint{3.018126in}{0.938411in}}%
\pgfpathlineto{\pgfqpoint{3.018993in}{0.823997in}}%
\pgfpathlineto{\pgfqpoint{3.020727in}{1.251946in}}%
\pgfpathlineto{\pgfqpoint{3.021594in}{1.014170in}}%
\pgfpathlineto{\pgfqpoint{3.022462in}{1.063795in}}%
\pgfpathlineto{\pgfqpoint{3.023329in}{0.883797in}}%
\pgfpathlineto{\pgfqpoint{3.024196in}{0.901251in}}%
\pgfpathlineto{\pgfqpoint{3.025063in}{1.301290in}}%
\pgfpathlineto{\pgfqpoint{3.025930in}{1.002572in}}%
\pgfpathlineto{\pgfqpoint{3.026797in}{1.034430in}}%
\pgfpathlineto{\pgfqpoint{3.027664in}{0.957673in}}%
\pgfpathlineto{\pgfqpoint{3.028531in}{1.661861in}}%
\pgfpathlineto{\pgfqpoint{3.030266in}{0.714229in}}%
\pgfpathlineto{\pgfqpoint{3.031133in}{1.216973in}}%
\pgfpathlineto{\pgfqpoint{3.032867in}{0.875763in}}%
\pgfpathlineto{\pgfqpoint{3.033734in}{0.890467in}}%
\pgfpathlineto{\pgfqpoint{3.035469in}{1.122596in}}%
\pgfpathlineto{\pgfqpoint{3.036336in}{1.402086in}}%
\pgfpathlineto{\pgfqpoint{3.037203in}{1.214228in}}%
\pgfpathlineto{\pgfqpoint{3.038070in}{1.342721in}}%
\pgfpathlineto{\pgfqpoint{3.038937in}{0.706011in}}%
\pgfpathlineto{\pgfqpoint{3.039804in}{1.384294in}}%
\pgfpathlineto{\pgfqpoint{3.040671in}{0.712581in}}%
\pgfpathlineto{\pgfqpoint{3.041538in}{0.981238in}}%
\pgfpathlineto{\pgfqpoint{3.042406in}{0.974628in}}%
\pgfpathlineto{\pgfqpoint{3.043273in}{0.782024in}}%
\pgfpathlineto{\pgfqpoint{3.044140in}{1.329865in}}%
\pgfpathlineto{\pgfqpoint{3.045874in}{0.860210in}}%
\pgfpathlineto{\pgfqpoint{3.046741in}{0.870959in}}%
\pgfpathlineto{\pgfqpoint{3.047608in}{1.317938in}}%
\pgfpathlineto{\pgfqpoint{3.048476in}{0.869442in}}%
\pgfpathlineto{\pgfqpoint{3.049343in}{1.235590in}}%
\pgfpathlineto{\pgfqpoint{3.051077in}{0.767813in}}%
\pgfpathlineto{\pgfqpoint{3.052811in}{1.538180in}}%
\pgfpathlineto{\pgfqpoint{3.053678in}{0.931284in}}%
\pgfpathlineto{\pgfqpoint{3.055413in}{1.346382in}}%
\pgfpathlineto{\pgfqpoint{3.056280in}{1.431869in}}%
\pgfpathlineto{\pgfqpoint{3.058014in}{0.909620in}}%
\pgfpathlineto{\pgfqpoint{3.058881in}{1.251272in}}%
\pgfpathlineto{\pgfqpoint{3.059748in}{0.966755in}}%
\pgfpathlineto{\pgfqpoint{3.060615in}{0.975964in}}%
\pgfpathlineto{\pgfqpoint{3.061483in}{1.492210in}}%
\pgfpathlineto{\pgfqpoint{3.062350in}{1.430945in}}%
\pgfpathlineto{\pgfqpoint{3.063217in}{0.858980in}}%
\pgfpathlineto{\pgfqpoint{3.064084in}{1.288908in}}%
\pgfpathlineto{\pgfqpoint{3.064951in}{0.855067in}}%
\pgfpathlineto{\pgfqpoint{3.065818in}{0.961367in}}%
\pgfpathlineto{\pgfqpoint{3.067552in}{1.345289in}}%
\pgfpathlineto{\pgfqpoint{3.069287in}{0.788688in}}%
\pgfpathlineto{\pgfqpoint{3.070154in}{1.055712in}}%
\pgfpathlineto{\pgfqpoint{3.071888in}{0.679956in}}%
\pgfpathlineto{\pgfqpoint{3.073622in}{1.128853in}}%
\pgfpathlineto{\pgfqpoint{3.074490in}{1.066047in}}%
\pgfpathlineto{\pgfqpoint{3.075357in}{1.180972in}}%
\pgfpathlineto{\pgfqpoint{3.076224in}{0.891323in}}%
\pgfpathlineto{\pgfqpoint{3.077091in}{1.110503in}}%
\pgfpathlineto{\pgfqpoint{3.077958in}{0.961899in}}%
\pgfpathlineto{\pgfqpoint{3.078825in}{1.208714in}}%
\pgfpathlineto{\pgfqpoint{3.079692in}{0.961691in}}%
\pgfpathlineto{\pgfqpoint{3.080559in}{0.961933in}}%
\pgfpathlineto{\pgfqpoint{3.081427in}{0.857336in}}%
\pgfpathlineto{\pgfqpoint{3.082294in}{1.122793in}}%
\pgfpathlineto{\pgfqpoint{3.083161in}{0.806055in}}%
\pgfpathlineto{\pgfqpoint{3.084028in}{0.891927in}}%
\pgfpathlineto{\pgfqpoint{3.084895in}{1.179871in}}%
\pgfpathlineto{\pgfqpoint{3.085762in}{0.807884in}}%
\pgfpathlineto{\pgfqpoint{3.086629in}{0.836708in}}%
\pgfpathlineto{\pgfqpoint{3.088364in}{1.180000in}}%
\pgfpathlineto{\pgfqpoint{3.089231in}{0.861016in}}%
\pgfpathlineto{\pgfqpoint{3.090098in}{1.266158in}}%
\pgfpathlineto{\pgfqpoint{3.090965in}{0.879126in}}%
\pgfpathlineto{\pgfqpoint{3.091832in}{1.323502in}}%
\pgfpathlineto{\pgfqpoint{3.092699in}{0.665431in}}%
\pgfpathlineto{\pgfqpoint{3.093566in}{0.983069in}}%
\pgfpathlineto{\pgfqpoint{3.094434in}{0.926328in}}%
\pgfpathlineto{\pgfqpoint{3.096168in}{1.043436in}}%
\pgfpathlineto{\pgfqpoint{3.097035in}{1.328170in}}%
\pgfpathlineto{\pgfqpoint{3.097902in}{0.901552in}}%
\pgfpathlineto{\pgfqpoint{3.098769in}{1.069728in}}%
\pgfpathlineto{\pgfqpoint{3.099636in}{0.706518in}}%
\pgfpathlineto{\pgfqpoint{3.100503in}{1.032177in}}%
\pgfpathlineto{\pgfqpoint{3.101371in}{0.890677in}}%
\pgfpathlineto{\pgfqpoint{3.102238in}{1.449773in}}%
\pgfpathlineto{\pgfqpoint{3.103972in}{1.060346in}}%
\pgfpathlineto{\pgfqpoint{3.104839in}{1.253120in}}%
\pgfpathlineto{\pgfqpoint{3.105706in}{1.241867in}}%
\pgfpathlineto{\pgfqpoint{3.107441in}{0.836006in}}%
\pgfpathlineto{\pgfqpoint{3.108308in}{1.054475in}}%
\pgfpathlineto{\pgfqpoint{3.110042in}{0.710637in}}%
\pgfpathlineto{\pgfqpoint{3.111776in}{1.009885in}}%
\pgfpathlineto{\pgfqpoint{3.112643in}{1.046599in}}%
\pgfpathlineto{\pgfqpoint{3.114378in}{1.223144in}}%
\pgfpathlineto{\pgfqpoint{3.115245in}{1.592849in}}%
\pgfpathlineto{\pgfqpoint{3.116979in}{1.030586in}}%
\pgfpathlineto{\pgfqpoint{3.117846in}{1.623645in}}%
\pgfpathlineto{\pgfqpoint{3.119580in}{0.760715in}}%
\pgfpathlineto{\pgfqpoint{3.120448in}{1.607196in}}%
\pgfpathlineto{\pgfqpoint{3.121315in}{0.969043in}}%
\pgfpathlineto{\pgfqpoint{3.122182in}{1.102704in}}%
\pgfpathlineto{\pgfqpoint{3.123916in}{0.779115in}}%
\pgfpathlineto{\pgfqpoint{3.124783in}{0.695295in}}%
\pgfpathlineto{\pgfqpoint{3.126517in}{1.102664in}}%
\pgfpathlineto{\pgfqpoint{3.127385in}{1.148060in}}%
\pgfpathlineto{\pgfqpoint{3.128252in}{0.826268in}}%
\pgfpathlineto{\pgfqpoint{3.129986in}{1.492683in}}%
\pgfpathlineto{\pgfqpoint{3.130853in}{0.897719in}}%
\pgfpathlineto{\pgfqpoint{3.131720in}{1.267141in}}%
\pgfpathlineto{\pgfqpoint{3.132587in}{0.838209in}}%
\pgfpathlineto{\pgfqpoint{3.135189in}{1.329359in}}%
\pgfpathlineto{\pgfqpoint{3.136056in}{0.925601in}}%
\pgfpathlineto{\pgfqpoint{3.137790in}{1.266175in}}%
\pgfpathlineto{\pgfqpoint{3.138657in}{0.836263in}}%
\pgfpathlineto{\pgfqpoint{3.139524in}{1.055170in}}%
\pgfpathlineto{\pgfqpoint{3.140392in}{0.752970in}}%
\pgfpathlineto{\pgfqpoint{3.141259in}{1.211222in}}%
\pgfpathlineto{\pgfqpoint{3.142126in}{0.889475in}}%
\pgfpathlineto{\pgfqpoint{3.142993in}{0.965711in}}%
\pgfpathlineto{\pgfqpoint{3.143860in}{0.795686in}}%
\pgfpathlineto{\pgfqpoint{3.144727in}{1.447525in}}%
\pgfpathlineto{\pgfqpoint{3.146462in}{0.714526in}}%
\pgfpathlineto{\pgfqpoint{3.149063in}{1.223875in}}%
\pgfpathlineto{\pgfqpoint{3.149930in}{0.872018in}}%
\pgfpathlineto{\pgfqpoint{3.150797in}{1.290840in}}%
\pgfpathlineto{\pgfqpoint{3.151664in}{1.224982in}}%
\pgfpathlineto{\pgfqpoint{3.152531in}{0.707035in}}%
\pgfpathlineto{\pgfqpoint{3.156000in}{1.351834in}}%
\pgfpathlineto{\pgfqpoint{3.156867in}{0.939636in}}%
\pgfpathlineto{\pgfqpoint{3.157734in}{1.437371in}}%
\pgfpathlineto{\pgfqpoint{3.158601in}{0.689077in}}%
\pgfpathlineto{\pgfqpoint{3.159469in}{0.811873in}}%
\pgfpathlineto{\pgfqpoint{3.160336in}{1.091179in}}%
\pgfpathlineto{\pgfqpoint{3.161203in}{1.047485in}}%
\pgfpathlineto{\pgfqpoint{3.162070in}{1.029680in}}%
\pgfpathlineto{\pgfqpoint{3.162937in}{0.656897in}}%
\pgfpathlineto{\pgfqpoint{3.164671in}{1.257223in}}%
\pgfpathlineto{\pgfqpoint{3.166406in}{0.807624in}}%
\pgfpathlineto{\pgfqpoint{3.167273in}{1.583159in}}%
\pgfpathlineto{\pgfqpoint{3.169007in}{0.930033in}}%
\pgfpathlineto{\pgfqpoint{3.169874in}{0.945696in}}%
\pgfpathlineto{\pgfqpoint{3.170741in}{0.704068in}}%
\pgfpathlineto{\pgfqpoint{3.174210in}{1.166580in}}%
\pgfpathlineto{\pgfqpoint{3.175077in}{0.944694in}}%
\pgfpathlineto{\pgfqpoint{3.175944in}{0.975153in}}%
\pgfpathlineto{\pgfqpoint{3.176811in}{0.875458in}}%
\pgfpathlineto{\pgfqpoint{3.178545in}{1.110153in}}%
\pgfpathlineto{\pgfqpoint{3.179413in}{1.108087in}}%
\pgfpathlineto{\pgfqpoint{3.182014in}{0.823867in}}%
\pgfpathlineto{\pgfqpoint{3.184615in}{1.312191in}}%
\pgfpathlineto{\pgfqpoint{3.185483in}{0.756166in}}%
\pgfpathlineto{\pgfqpoint{3.186350in}{1.158704in}}%
\pgfpathlineto{\pgfqpoint{3.187217in}{0.789151in}}%
\pgfpathlineto{\pgfqpoint{3.188951in}{1.369799in}}%
\pgfpathlineto{\pgfqpoint{3.189818in}{0.799175in}}%
\pgfpathlineto{\pgfqpoint{3.190685in}{1.297580in}}%
\pgfpathlineto{\pgfqpoint{3.191552in}{0.862583in}}%
\pgfpathlineto{\pgfqpoint{3.192420in}{0.977087in}}%
\pgfpathlineto{\pgfqpoint{3.193287in}{1.365437in}}%
\pgfpathlineto{\pgfqpoint{3.194154in}{0.752316in}}%
\pgfpathlineto{\pgfqpoint{3.195021in}{0.887119in}}%
\pgfpathlineto{\pgfqpoint{3.195888in}{1.387681in}}%
\pgfpathlineto{\pgfqpoint{3.197622in}{0.662056in}}%
\pgfpathlineto{\pgfqpoint{3.198490in}{0.721567in}}%
\pgfpathlineto{\pgfqpoint{3.199357in}{0.900929in}}%
\pgfpathlineto{\pgfqpoint{3.200224in}{1.536964in}}%
\pgfpathlineto{\pgfqpoint{3.201091in}{1.051476in}}%
\pgfpathlineto{\pgfqpoint{3.201958in}{1.108588in}}%
\pgfpathlineto{\pgfqpoint{3.203692in}{1.342325in}}%
\pgfpathlineto{\pgfqpoint{3.204559in}{0.773005in}}%
\pgfpathlineto{\pgfqpoint{3.205427in}{0.782430in}}%
\pgfpathlineto{\pgfqpoint{3.206294in}{0.753709in}}%
\pgfpathlineto{\pgfqpoint{3.208895in}{1.069673in}}%
\pgfpathlineto{\pgfqpoint{3.209762in}{0.736068in}}%
\pgfpathlineto{\pgfqpoint{3.211497in}{1.437666in}}%
\pgfpathlineto{\pgfqpoint{3.212364in}{0.719357in}}%
\pgfpathlineto{\pgfqpoint{3.213231in}{0.871634in}}%
\pgfpathlineto{\pgfqpoint{3.214098in}{0.928625in}}%
\pgfpathlineto{\pgfqpoint{3.214965in}{1.088162in}}%
\pgfpathlineto{\pgfqpoint{3.217566in}{0.763638in}}%
\pgfpathlineto{\pgfqpoint{3.218434in}{0.820291in}}%
\pgfpathlineto{\pgfqpoint{3.219301in}{1.108166in}}%
\pgfpathlineto{\pgfqpoint{3.220168in}{1.077015in}}%
\pgfpathlineto{\pgfqpoint{3.221035in}{1.070256in}}%
\pgfpathlineto{\pgfqpoint{3.221902in}{1.250192in}}%
\pgfpathlineto{\pgfqpoint{3.224503in}{0.895551in}}%
\pgfpathlineto{\pgfqpoint{3.225371in}{0.936889in}}%
\pgfpathlineto{\pgfqpoint{3.226238in}{0.846539in}}%
\pgfpathlineto{\pgfqpoint{3.227105in}{0.849926in}}%
\pgfpathlineto{\pgfqpoint{3.227972in}{1.240800in}}%
\pgfpathlineto{\pgfqpoint{3.228839in}{1.189909in}}%
\pgfpathlineto{\pgfqpoint{3.230573in}{0.781217in}}%
\pgfpathlineto{\pgfqpoint{3.231441in}{1.127077in}}%
\pgfpathlineto{\pgfqpoint{3.234042in}{0.703873in}}%
\pgfpathlineto{\pgfqpoint{3.234909in}{0.719837in}}%
\pgfpathlineto{\pgfqpoint{3.235776in}{0.747912in}}%
\pgfpathlineto{\pgfqpoint{3.236643in}{1.484168in}}%
\pgfpathlineto{\pgfqpoint{3.237510in}{0.918220in}}%
\pgfpathlineto{\pgfqpoint{3.239245in}{1.168886in}}%
\pgfpathlineto{\pgfqpoint{3.240112in}{0.824490in}}%
\pgfpathlineto{\pgfqpoint{3.243580in}{1.414902in}}%
\pgfpathlineto{\pgfqpoint{3.245315in}{0.912512in}}%
\pgfpathlineto{\pgfqpoint{3.247049in}{1.264738in}}%
\pgfpathlineto{\pgfqpoint{3.248783in}{0.979687in}}%
\pgfpathlineto{\pgfqpoint{3.249650in}{1.151603in}}%
\pgfpathlineto{\pgfqpoint{3.251385in}{0.790604in}}%
\pgfpathlineto{\pgfqpoint{3.252252in}{0.855834in}}%
\pgfpathlineto{\pgfqpoint{3.253119in}{1.166256in}}%
\pgfpathlineto{\pgfqpoint{3.253986in}{0.754774in}}%
\pgfpathlineto{\pgfqpoint{3.255720in}{1.260181in}}%
\pgfpathlineto{\pgfqpoint{3.256587in}{1.851889in}}%
\pgfpathlineto{\pgfqpoint{3.258322in}{0.635726in}}%
\pgfpathlineto{\pgfqpoint{3.260056in}{1.244300in}}%
\pgfpathlineto{\pgfqpoint{3.260923in}{0.711481in}}%
\pgfpathlineto{\pgfqpoint{3.261790in}{0.927722in}}%
\pgfpathlineto{\pgfqpoint{3.262657in}{1.741243in}}%
\pgfpathlineto{\pgfqpoint{3.264392in}{1.165318in}}%
\pgfpathlineto{\pgfqpoint{3.265259in}{1.185848in}}%
\pgfpathlineto{\pgfqpoint{3.268727in}{0.739958in}}%
\pgfpathlineto{\pgfqpoint{3.269594in}{1.156612in}}%
\pgfpathlineto{\pgfqpoint{3.270462in}{0.796529in}}%
\pgfpathlineto{\pgfqpoint{3.271329in}{1.019651in}}%
\pgfpathlineto{\pgfqpoint{3.272196in}{1.827460in}}%
\pgfpathlineto{\pgfqpoint{3.273930in}{0.892503in}}%
\pgfpathlineto{\pgfqpoint{3.275664in}{1.066361in}}%
\pgfpathlineto{\pgfqpoint{3.276531in}{1.225920in}}%
\pgfpathlineto{\pgfqpoint{3.277399in}{1.637457in}}%
\pgfpathlineto{\pgfqpoint{3.279133in}{0.604364in}}%
\pgfpathlineto{\pgfqpoint{3.280000in}{0.958763in}}%
\pgfpathlineto{\pgfqpoint{3.280867in}{0.604364in}}%
\pgfpathlineto{\pgfqpoint{3.282601in}{1.637457in}}%
\pgfpathlineto{\pgfqpoint{3.284336in}{1.066361in}}%
\pgfpathlineto{\pgfqpoint{3.286070in}{0.892503in}}%
\pgfpathlineto{\pgfqpoint{3.286937in}{1.112533in}}%
\pgfpathlineto{\pgfqpoint{3.287804in}{1.827460in}}%
\pgfpathlineto{\pgfqpoint{3.289538in}{0.796529in}}%
\pgfpathlineto{\pgfqpoint{3.290406in}{1.156612in}}%
\pgfpathlineto{\pgfqpoint{3.291273in}{0.739958in}}%
\pgfpathlineto{\pgfqpoint{3.293007in}{1.008419in}}%
\pgfpathlineto{\pgfqpoint{3.293874in}{1.058881in}}%
\pgfpathlineto{\pgfqpoint{3.294741in}{1.185848in}}%
\pgfpathlineto{\pgfqpoint{3.295608in}{1.165318in}}%
\pgfpathlineto{\pgfqpoint{3.296476in}{1.169985in}}%
\pgfpathlineto{\pgfqpoint{3.297343in}{1.741243in}}%
\pgfpathlineto{\pgfqpoint{3.299077in}{0.711481in}}%
\pgfpathlineto{\pgfqpoint{3.299944in}{1.244300in}}%
\pgfpathlineto{\pgfqpoint{3.300811in}{1.153279in}}%
\pgfpathlineto{\pgfqpoint{3.301678in}{0.635726in}}%
\pgfpathlineto{\pgfqpoint{3.302545in}{0.684016in}}%
\pgfpathlineto{\pgfqpoint{3.303413in}{1.851889in}}%
\pgfpathlineto{\pgfqpoint{3.306014in}{0.754774in}}%
\pgfpathlineto{\pgfqpoint{3.306881in}{1.166256in}}%
\pgfpathlineto{\pgfqpoint{3.308615in}{0.790604in}}%
\pgfpathlineto{\pgfqpoint{3.310350in}{1.151603in}}%
\pgfpathlineto{\pgfqpoint{3.311217in}{0.979687in}}%
\pgfpathlineto{\pgfqpoint{3.312084in}{1.004324in}}%
\pgfpathlineto{\pgfqpoint{3.312951in}{1.264738in}}%
\pgfpathlineto{\pgfqpoint{3.314685in}{0.912512in}}%
\pgfpathlineto{\pgfqpoint{3.315552in}{1.058080in}}%
\pgfpathlineto{\pgfqpoint{3.316420in}{1.414902in}}%
\pgfpathlineto{\pgfqpoint{3.318154in}{1.088158in}}%
\pgfpathlineto{\pgfqpoint{3.319888in}{0.824490in}}%
\pgfpathlineto{\pgfqpoint{3.320755in}{1.168886in}}%
\pgfpathlineto{\pgfqpoint{3.321622in}{1.082731in}}%
\pgfpathlineto{\pgfqpoint{3.322490in}{0.918220in}}%
\pgfpathlineto{\pgfqpoint{3.323357in}{1.484168in}}%
\pgfpathlineto{\pgfqpoint{3.325091in}{0.719837in}}%
\pgfpathlineto{\pgfqpoint{3.325958in}{0.703873in}}%
\pgfpathlineto{\pgfqpoint{3.327692in}{0.989405in}}%
\pgfpathlineto{\pgfqpoint{3.328559in}{1.127077in}}%
\pgfpathlineto{\pgfqpoint{3.329427in}{0.781217in}}%
\pgfpathlineto{\pgfqpoint{3.330294in}{0.899385in}}%
\pgfpathlineto{\pgfqpoint{3.332028in}{1.240800in}}%
\pgfpathlineto{\pgfqpoint{3.333762in}{0.846539in}}%
\pgfpathlineto{\pgfqpoint{3.334629in}{0.936889in}}%
\pgfpathlineto{\pgfqpoint{3.335497in}{0.895551in}}%
\pgfpathlineto{\pgfqpoint{3.337231in}{1.046032in}}%
\pgfpathlineto{\pgfqpoint{3.338098in}{1.250192in}}%
\pgfpathlineto{\pgfqpoint{3.338965in}{1.070256in}}%
\pgfpathlineto{\pgfqpoint{3.339832in}{1.077015in}}%
\pgfpathlineto{\pgfqpoint{3.340699in}{1.108166in}}%
\pgfpathlineto{\pgfqpoint{3.342434in}{0.763638in}}%
\pgfpathlineto{\pgfqpoint{3.345035in}{1.088162in}}%
\pgfpathlineto{\pgfqpoint{3.347636in}{0.719357in}}%
\pgfpathlineto{\pgfqpoint{3.348503in}{1.437666in}}%
\pgfpathlineto{\pgfqpoint{3.350238in}{0.736068in}}%
\pgfpathlineto{\pgfqpoint{3.351105in}{1.069673in}}%
\pgfpathlineto{\pgfqpoint{3.351972in}{0.978324in}}%
\pgfpathlineto{\pgfqpoint{3.352839in}{0.912051in}}%
\pgfpathlineto{\pgfqpoint{3.353706in}{0.753709in}}%
\pgfpathlineto{\pgfqpoint{3.354573in}{0.782430in}}%
\pgfpathlineto{\pgfqpoint{3.355441in}{0.773005in}}%
\pgfpathlineto{\pgfqpoint{3.356308in}{1.342325in}}%
\pgfpathlineto{\pgfqpoint{3.357175in}{1.295880in}}%
\pgfpathlineto{\pgfqpoint{3.358909in}{1.051476in}}%
\pgfpathlineto{\pgfqpoint{3.359776in}{1.536964in}}%
\pgfpathlineto{\pgfqpoint{3.361510in}{0.721567in}}%
\pgfpathlineto{\pgfqpoint{3.362378in}{0.662056in}}%
\pgfpathlineto{\pgfqpoint{3.363245in}{0.692608in}}%
\pgfpathlineto{\pgfqpoint{3.364112in}{1.387681in}}%
\pgfpathlineto{\pgfqpoint{3.365846in}{0.752316in}}%
\pgfpathlineto{\pgfqpoint{3.366713in}{1.365437in}}%
\pgfpathlineto{\pgfqpoint{3.368448in}{0.862583in}}%
\pgfpathlineto{\pgfqpoint{3.369315in}{1.297580in}}%
\pgfpathlineto{\pgfqpoint{3.370182in}{0.799175in}}%
\pgfpathlineto{\pgfqpoint{3.371049in}{1.369799in}}%
\pgfpathlineto{\pgfqpoint{3.371916in}{1.260942in}}%
\pgfpathlineto{\pgfqpoint{3.372783in}{0.789151in}}%
\pgfpathlineto{\pgfqpoint{3.373650in}{1.158704in}}%
\pgfpathlineto{\pgfqpoint{3.374517in}{0.756166in}}%
\pgfpathlineto{\pgfqpoint{3.375385in}{1.312191in}}%
\pgfpathlineto{\pgfqpoint{3.377119in}{0.959158in}}%
\pgfpathlineto{\pgfqpoint{3.377986in}{0.823867in}}%
\pgfpathlineto{\pgfqpoint{3.380587in}{1.108087in}}%
\pgfpathlineto{\pgfqpoint{3.381455in}{1.110153in}}%
\pgfpathlineto{\pgfqpoint{3.382322in}{1.071761in}}%
\pgfpathlineto{\pgfqpoint{3.383189in}{0.875458in}}%
\pgfpathlineto{\pgfqpoint{3.384056in}{0.975153in}}%
\pgfpathlineto{\pgfqpoint{3.384923in}{0.944694in}}%
\pgfpathlineto{\pgfqpoint{3.385790in}{1.166580in}}%
\pgfpathlineto{\pgfqpoint{3.386657in}{0.909501in}}%
\pgfpathlineto{\pgfqpoint{3.387524in}{0.923601in}}%
\pgfpathlineto{\pgfqpoint{3.389259in}{0.704068in}}%
\pgfpathlineto{\pgfqpoint{3.390126in}{0.945696in}}%
\pgfpathlineto{\pgfqpoint{3.390993in}{0.930033in}}%
\pgfpathlineto{\pgfqpoint{3.391860in}{0.948815in}}%
\pgfpathlineto{\pgfqpoint{3.392727in}{1.583159in}}%
\pgfpathlineto{\pgfqpoint{3.393594in}{0.807624in}}%
\pgfpathlineto{\pgfqpoint{3.394462in}{0.912340in}}%
\pgfpathlineto{\pgfqpoint{3.395329in}{1.257223in}}%
\pgfpathlineto{\pgfqpoint{3.396196in}{1.149687in}}%
\pgfpathlineto{\pgfqpoint{3.397063in}{0.656897in}}%
\pgfpathlineto{\pgfqpoint{3.398797in}{1.047485in}}%
\pgfpathlineto{\pgfqpoint{3.399664in}{1.091179in}}%
\pgfpathlineto{\pgfqpoint{3.401399in}{0.689077in}}%
\pgfpathlineto{\pgfqpoint{3.402266in}{1.437371in}}%
\pgfpathlineto{\pgfqpoint{3.403133in}{0.939636in}}%
\pgfpathlineto{\pgfqpoint{3.404000in}{1.351834in}}%
\pgfpathlineto{\pgfqpoint{3.407469in}{0.707035in}}%
\pgfpathlineto{\pgfqpoint{3.409203in}{1.290840in}}%
\pgfpathlineto{\pgfqpoint{3.410070in}{0.872018in}}%
\pgfpathlineto{\pgfqpoint{3.410937in}{1.223875in}}%
\pgfpathlineto{\pgfqpoint{3.411804in}{0.907196in}}%
\pgfpathlineto{\pgfqpoint{3.412671in}{0.981045in}}%
\pgfpathlineto{\pgfqpoint{3.413538in}{0.714526in}}%
\pgfpathlineto{\pgfqpoint{3.414406in}{0.858983in}}%
\pgfpathlineto{\pgfqpoint{3.415273in}{1.447525in}}%
\pgfpathlineto{\pgfqpoint{3.416140in}{0.795686in}}%
\pgfpathlineto{\pgfqpoint{3.417007in}{0.965711in}}%
\pgfpathlineto{\pgfqpoint{3.417874in}{0.889475in}}%
\pgfpathlineto{\pgfqpoint{3.418741in}{1.211222in}}%
\pgfpathlineto{\pgfqpoint{3.419608in}{0.752970in}}%
\pgfpathlineto{\pgfqpoint{3.420476in}{1.055170in}}%
\pgfpathlineto{\pgfqpoint{3.421343in}{0.836263in}}%
\pgfpathlineto{\pgfqpoint{3.422210in}{1.266175in}}%
\pgfpathlineto{\pgfqpoint{3.423077in}{1.230758in}}%
\pgfpathlineto{\pgfqpoint{3.423944in}{0.925601in}}%
\pgfpathlineto{\pgfqpoint{3.424811in}{1.329359in}}%
\pgfpathlineto{\pgfqpoint{3.427413in}{0.838209in}}%
\pgfpathlineto{\pgfqpoint{3.428280in}{1.267141in}}%
\pgfpathlineto{\pgfqpoint{3.429147in}{0.897719in}}%
\pgfpathlineto{\pgfqpoint{3.430014in}{1.492683in}}%
\pgfpathlineto{\pgfqpoint{3.431748in}{0.826268in}}%
\pgfpathlineto{\pgfqpoint{3.432615in}{1.148060in}}%
\pgfpathlineto{\pgfqpoint{3.433483in}{1.102664in}}%
\pgfpathlineto{\pgfqpoint{3.435217in}{0.695295in}}%
\pgfpathlineto{\pgfqpoint{3.436084in}{0.779115in}}%
\pgfpathlineto{\pgfqpoint{3.436951in}{0.796889in}}%
\pgfpathlineto{\pgfqpoint{3.437818in}{1.102704in}}%
\pgfpathlineto{\pgfqpoint{3.438685in}{0.969043in}}%
\pgfpathlineto{\pgfqpoint{3.439552in}{1.607196in}}%
\pgfpathlineto{\pgfqpoint{3.440420in}{0.760715in}}%
\pgfpathlineto{\pgfqpoint{3.442154in}{1.623645in}}%
\pgfpathlineto{\pgfqpoint{3.443021in}{1.030586in}}%
\pgfpathlineto{\pgfqpoint{3.444755in}{1.592849in}}%
\pgfpathlineto{\pgfqpoint{3.446490in}{1.154409in}}%
\pgfpathlineto{\pgfqpoint{3.448224in}{1.009885in}}%
\pgfpathlineto{\pgfqpoint{3.449091in}{0.996914in}}%
\pgfpathlineto{\pgfqpoint{3.449958in}{0.710637in}}%
\pgfpathlineto{\pgfqpoint{3.450825in}{0.807311in}}%
\pgfpathlineto{\pgfqpoint{3.451692in}{1.054475in}}%
\pgfpathlineto{\pgfqpoint{3.452559in}{0.836006in}}%
\pgfpathlineto{\pgfqpoint{3.453427in}{0.907748in}}%
\pgfpathlineto{\pgfqpoint{3.455161in}{1.253120in}}%
\pgfpathlineto{\pgfqpoint{3.456028in}{1.060346in}}%
\pgfpathlineto{\pgfqpoint{3.456895in}{1.063634in}}%
\pgfpathlineto{\pgfqpoint{3.457762in}{1.449773in}}%
\pgfpathlineto{\pgfqpoint{3.458629in}{0.890677in}}%
\pgfpathlineto{\pgfqpoint{3.459497in}{1.032177in}}%
\pgfpathlineto{\pgfqpoint{3.460364in}{0.706518in}}%
\pgfpathlineto{\pgfqpoint{3.461231in}{1.069728in}}%
\pgfpathlineto{\pgfqpoint{3.462098in}{0.901552in}}%
\pgfpathlineto{\pgfqpoint{3.462965in}{1.328170in}}%
\pgfpathlineto{\pgfqpoint{3.464699in}{0.963398in}}%
\pgfpathlineto{\pgfqpoint{3.465566in}{0.926328in}}%
\pgfpathlineto{\pgfqpoint{3.466434in}{0.983069in}}%
\pgfpathlineto{\pgfqpoint{3.467301in}{0.665431in}}%
\pgfpathlineto{\pgfqpoint{3.468168in}{1.323502in}}%
\pgfpathlineto{\pgfqpoint{3.469035in}{0.879126in}}%
\pgfpathlineto{\pgfqpoint{3.469902in}{1.266158in}}%
\pgfpathlineto{\pgfqpoint{3.470769in}{0.861016in}}%
\pgfpathlineto{\pgfqpoint{3.471636in}{1.180000in}}%
\pgfpathlineto{\pgfqpoint{3.473371in}{0.836708in}}%
\pgfpathlineto{\pgfqpoint{3.474238in}{0.807884in}}%
\pgfpathlineto{\pgfqpoint{3.475105in}{1.179871in}}%
\pgfpathlineto{\pgfqpoint{3.476839in}{0.806055in}}%
\pgfpathlineto{\pgfqpoint{3.477706in}{1.122793in}}%
\pgfpathlineto{\pgfqpoint{3.478573in}{0.857336in}}%
\pgfpathlineto{\pgfqpoint{3.481175in}{1.208714in}}%
\pgfpathlineto{\pgfqpoint{3.482042in}{0.961899in}}%
\pgfpathlineto{\pgfqpoint{3.482909in}{1.110503in}}%
\pgfpathlineto{\pgfqpoint{3.483776in}{0.891323in}}%
\pgfpathlineto{\pgfqpoint{3.484643in}{1.180972in}}%
\pgfpathlineto{\pgfqpoint{3.485510in}{1.066047in}}%
\pgfpathlineto{\pgfqpoint{3.486378in}{1.128853in}}%
\pgfpathlineto{\pgfqpoint{3.487245in}{1.113064in}}%
\pgfpathlineto{\pgfqpoint{3.488112in}{0.679956in}}%
\pgfpathlineto{\pgfqpoint{3.489846in}{1.055712in}}%
\pgfpathlineto{\pgfqpoint{3.490713in}{0.788688in}}%
\pgfpathlineto{\pgfqpoint{3.492448in}{1.345289in}}%
\pgfpathlineto{\pgfqpoint{3.493315in}{1.253442in}}%
\pgfpathlineto{\pgfqpoint{3.495049in}{0.855067in}}%
\pgfpathlineto{\pgfqpoint{3.495916in}{1.288908in}}%
\pgfpathlineto{\pgfqpoint{3.496783in}{0.858980in}}%
\pgfpathlineto{\pgfqpoint{3.498517in}{1.492210in}}%
\pgfpathlineto{\pgfqpoint{3.500252in}{0.966755in}}%
\pgfpathlineto{\pgfqpoint{3.501119in}{1.251272in}}%
\pgfpathlineto{\pgfqpoint{3.501986in}{0.909620in}}%
\pgfpathlineto{\pgfqpoint{3.502853in}{0.971360in}}%
\pgfpathlineto{\pgfqpoint{3.503720in}{1.431869in}}%
\pgfpathlineto{\pgfqpoint{3.504587in}{1.346382in}}%
\pgfpathlineto{\pgfqpoint{3.505455in}{1.235952in}}%
\pgfpathlineto{\pgfqpoint{3.506322in}{0.931284in}}%
\pgfpathlineto{\pgfqpoint{3.507189in}{1.538180in}}%
\pgfpathlineto{\pgfqpoint{3.508923in}{0.767813in}}%
\pgfpathlineto{\pgfqpoint{3.510657in}{1.235590in}}%
\pgfpathlineto{\pgfqpoint{3.511524in}{0.869442in}}%
\pgfpathlineto{\pgfqpoint{3.512392in}{1.317938in}}%
\pgfpathlineto{\pgfqpoint{3.514126in}{0.860210in}}%
\pgfpathlineto{\pgfqpoint{3.514993in}{0.897514in}}%
\pgfpathlineto{\pgfqpoint{3.515860in}{1.329865in}}%
\pgfpathlineto{\pgfqpoint{3.516727in}{0.782024in}}%
\pgfpathlineto{\pgfqpoint{3.518462in}{0.981238in}}%
\pgfpathlineto{\pgfqpoint{3.519329in}{0.712581in}}%
\pgfpathlineto{\pgfqpoint{3.520196in}{1.384294in}}%
\pgfpathlineto{\pgfqpoint{3.521063in}{0.706011in}}%
\pgfpathlineto{\pgfqpoint{3.521930in}{1.342721in}}%
\pgfpathlineto{\pgfqpoint{3.522797in}{1.214228in}}%
\pgfpathlineto{\pgfqpoint{3.523664in}{1.402086in}}%
\pgfpathlineto{\pgfqpoint{3.526266in}{0.890467in}}%
\pgfpathlineto{\pgfqpoint{3.527133in}{0.875763in}}%
\pgfpathlineto{\pgfqpoint{3.528867in}{1.216973in}}%
\pgfpathlineto{\pgfqpoint{3.529734in}{0.714229in}}%
\pgfpathlineto{\pgfqpoint{3.530601in}{0.981685in}}%
\pgfpathlineto{\pgfqpoint{3.531469in}{1.661861in}}%
\pgfpathlineto{\pgfqpoint{3.532336in}{0.957673in}}%
\pgfpathlineto{\pgfqpoint{3.533203in}{1.034430in}}%
\pgfpathlineto{\pgfqpoint{3.534070in}{1.002572in}}%
\pgfpathlineto{\pgfqpoint{3.534937in}{1.301290in}}%
\pgfpathlineto{\pgfqpoint{3.536671in}{0.883797in}}%
\pgfpathlineto{\pgfqpoint{3.539273in}{1.251946in}}%
\pgfpathlineto{\pgfqpoint{3.540140in}{1.208996in}}%
\pgfpathlineto{\pgfqpoint{3.541007in}{0.823997in}}%
\pgfpathlineto{\pgfqpoint{3.541874in}{0.938411in}}%
\pgfpathlineto{\pgfqpoint{3.542741in}{0.917499in}}%
\pgfpathlineto{\pgfqpoint{3.543608in}{0.982269in}}%
\pgfpathlineto{\pgfqpoint{3.544476in}{1.425568in}}%
\pgfpathlineto{\pgfqpoint{3.545343in}{0.762972in}}%
\pgfpathlineto{\pgfqpoint{3.546210in}{1.322522in}}%
\pgfpathlineto{\pgfqpoint{3.547077in}{1.233222in}}%
\pgfpathlineto{\pgfqpoint{3.547944in}{1.225069in}}%
\pgfpathlineto{\pgfqpoint{3.548811in}{0.787995in}}%
\pgfpathlineto{\pgfqpoint{3.549678in}{1.608606in}}%
\pgfpathlineto{\pgfqpoint{3.550545in}{0.846700in}}%
\pgfpathlineto{\pgfqpoint{3.551413in}{0.877475in}}%
\pgfpathlineto{\pgfqpoint{3.552280in}{0.964704in}}%
\pgfpathlineto{\pgfqpoint{3.554014in}{1.333837in}}%
\pgfpathlineto{\pgfqpoint{3.554881in}{1.071220in}}%
\pgfpathlineto{\pgfqpoint{3.555748in}{1.105467in}}%
\pgfpathlineto{\pgfqpoint{3.557483in}{0.999139in}}%
\pgfpathlineto{\pgfqpoint{3.558350in}{1.761341in}}%
\pgfpathlineto{\pgfqpoint{3.559217in}{1.137357in}}%
\pgfpathlineto{\pgfqpoint{3.560084in}{2.059720in}}%
\pgfpathlineto{\pgfqpoint{3.561818in}{0.953413in}}%
\pgfpathlineto{\pgfqpoint{3.562685in}{0.989939in}}%
\pgfpathlineto{\pgfqpoint{3.563552in}{1.256700in}}%
\pgfpathlineto{\pgfqpoint{3.564420in}{1.125606in}}%
\pgfpathlineto{\pgfqpoint{3.565287in}{0.801344in}}%
\pgfpathlineto{\pgfqpoint{3.566154in}{1.231220in}}%
\pgfpathlineto{\pgfqpoint{3.567021in}{0.995901in}}%
\pgfpathlineto{\pgfqpoint{3.568755in}{1.203945in}}%
\pgfpathlineto{\pgfqpoint{3.569622in}{1.130939in}}%
\pgfpathlineto{\pgfqpoint{3.570490in}{1.285548in}}%
\pgfpathlineto{\pgfqpoint{3.572224in}{0.755733in}}%
\pgfpathlineto{\pgfqpoint{3.573958in}{1.183184in}}%
\pgfpathlineto{\pgfqpoint{3.574825in}{0.956590in}}%
\pgfpathlineto{\pgfqpoint{3.576559in}{1.529578in}}%
\pgfpathlineto{\pgfqpoint{3.578294in}{0.702912in}}%
\pgfpathlineto{\pgfqpoint{3.580895in}{1.473071in}}%
\pgfpathlineto{\pgfqpoint{3.581762in}{0.696899in}}%
\pgfpathlineto{\pgfqpoint{3.582629in}{0.983749in}}%
\pgfpathlineto{\pgfqpoint{3.583497in}{0.752260in}}%
\pgfpathlineto{\pgfqpoint{3.584364in}{1.403236in}}%
\pgfpathlineto{\pgfqpoint{3.585231in}{0.874763in}}%
\pgfpathlineto{\pgfqpoint{3.586965in}{1.224507in}}%
\pgfpathlineto{\pgfqpoint{3.587832in}{0.898104in}}%
\pgfpathlineto{\pgfqpoint{3.589566in}{1.300688in}}%
\pgfpathlineto{\pgfqpoint{3.590434in}{0.891294in}}%
\pgfpathlineto{\pgfqpoint{3.592168in}{1.749018in}}%
\pgfpathlineto{\pgfqpoint{3.593902in}{1.101886in}}%
\pgfpathlineto{\pgfqpoint{3.594769in}{1.218347in}}%
\pgfpathlineto{\pgfqpoint{3.595636in}{2.003050in}}%
\pgfpathlineto{\pgfqpoint{3.597371in}{0.939582in}}%
\pgfpathlineto{\pgfqpoint{3.598238in}{0.895786in}}%
\pgfpathlineto{\pgfqpoint{3.599105in}{1.496099in}}%
\pgfpathlineto{\pgfqpoint{3.599972in}{0.840608in}}%
\pgfpathlineto{\pgfqpoint{3.600839in}{1.181439in}}%
\pgfpathlineto{\pgfqpoint{3.602573in}{0.742973in}}%
\pgfpathlineto{\pgfqpoint{3.603441in}{0.903473in}}%
\pgfpathlineto{\pgfqpoint{3.604308in}{1.585770in}}%
\pgfpathlineto{\pgfqpoint{3.605175in}{0.932750in}}%
\pgfpathlineto{\pgfqpoint{3.606042in}{1.036844in}}%
\pgfpathlineto{\pgfqpoint{3.606909in}{1.218408in}}%
\pgfpathlineto{\pgfqpoint{3.607776in}{0.866464in}}%
\pgfpathlineto{\pgfqpoint{3.608643in}{0.919170in}}%
\pgfpathlineto{\pgfqpoint{3.609510in}{0.968412in}}%
\pgfpathlineto{\pgfqpoint{3.611245in}{1.384307in}}%
\pgfpathlineto{\pgfqpoint{3.612112in}{1.349156in}}%
\pgfpathlineto{\pgfqpoint{3.612979in}{0.855300in}}%
\pgfpathlineto{\pgfqpoint{3.614713in}{1.883928in}}%
\pgfpathlineto{\pgfqpoint{3.616448in}{0.676318in}}%
\pgfpathlineto{\pgfqpoint{3.617315in}{0.974680in}}%
\pgfpathlineto{\pgfqpoint{3.618182in}{0.778775in}}%
\pgfpathlineto{\pgfqpoint{3.619916in}{1.769595in}}%
\pgfpathlineto{\pgfqpoint{3.620783in}{0.968239in}}%
\pgfpathlineto{\pgfqpoint{3.621650in}{1.128093in}}%
\pgfpathlineto{\pgfqpoint{3.622517in}{1.373013in}}%
\pgfpathlineto{\pgfqpoint{3.623385in}{1.354886in}}%
\pgfpathlineto{\pgfqpoint{3.624252in}{1.040449in}}%
\pgfpathlineto{\pgfqpoint{3.625119in}{1.392465in}}%
\pgfpathlineto{\pgfqpoint{3.625986in}{1.264480in}}%
\pgfpathlineto{\pgfqpoint{3.626853in}{0.861983in}}%
\pgfpathlineto{\pgfqpoint{3.627720in}{1.069934in}}%
\pgfpathlineto{\pgfqpoint{3.628587in}{0.671020in}}%
\pgfpathlineto{\pgfqpoint{3.629455in}{0.718814in}}%
\pgfpathlineto{\pgfqpoint{3.631189in}{1.317624in}}%
\pgfpathlineto{\pgfqpoint{3.632923in}{0.845853in}}%
\pgfpathlineto{\pgfqpoint{3.633790in}{1.545030in}}%
\pgfpathlineto{\pgfqpoint{3.634657in}{0.896078in}}%
\pgfpathlineto{\pgfqpoint{3.635524in}{0.929894in}}%
\pgfpathlineto{\pgfqpoint{3.636392in}{1.369237in}}%
\pgfpathlineto{\pgfqpoint{3.638126in}{0.868995in}}%
\pgfpathlineto{\pgfqpoint{3.638993in}{1.338777in}}%
\pgfpathlineto{\pgfqpoint{3.639860in}{1.161072in}}%
\pgfpathlineto{\pgfqpoint{3.640727in}{1.320493in}}%
\pgfpathlineto{\pgfqpoint{3.641594in}{1.043424in}}%
\pgfpathlineto{\pgfqpoint{3.642462in}{1.117263in}}%
\pgfpathlineto{\pgfqpoint{3.643329in}{0.926672in}}%
\pgfpathlineto{\pgfqpoint{3.645063in}{1.550697in}}%
\pgfpathlineto{\pgfqpoint{3.646797in}{1.052204in}}%
\pgfpathlineto{\pgfqpoint{3.647664in}{1.420603in}}%
\pgfpathlineto{\pgfqpoint{3.648531in}{0.905821in}}%
\pgfpathlineto{\pgfqpoint{3.649399in}{1.237768in}}%
\pgfpathlineto{\pgfqpoint{3.650266in}{0.931207in}}%
\pgfpathlineto{\pgfqpoint{3.651133in}{1.908471in}}%
\pgfpathlineto{\pgfqpoint{3.652000in}{1.099114in}}%
\pgfpathlineto{\pgfqpoint{3.652867in}{1.617843in}}%
\pgfpathlineto{\pgfqpoint{3.653734in}{0.910459in}}%
\pgfpathlineto{\pgfqpoint{3.654601in}{1.560667in}}%
\pgfpathlineto{\pgfqpoint{3.655469in}{0.977372in}}%
\pgfpathlineto{\pgfqpoint{3.656336in}{1.078820in}}%
\pgfpathlineto{\pgfqpoint{3.657203in}{1.174031in}}%
\pgfpathlineto{\pgfqpoint{3.658070in}{1.092459in}}%
\pgfpathlineto{\pgfqpoint{3.658937in}{1.146293in}}%
\pgfpathlineto{\pgfqpoint{3.659804in}{0.971909in}}%
\pgfpathlineto{\pgfqpoint{3.661538in}{1.310618in}}%
\pgfpathlineto{\pgfqpoint{3.662406in}{0.718219in}}%
\pgfpathlineto{\pgfqpoint{3.664140in}{1.007325in}}%
\pgfpathlineto{\pgfqpoint{3.665007in}{1.878746in}}%
\pgfpathlineto{\pgfqpoint{3.665874in}{0.967269in}}%
\pgfpathlineto{\pgfqpoint{3.666741in}{1.201378in}}%
\pgfpathlineto{\pgfqpoint{3.667608in}{0.853794in}}%
\pgfpathlineto{\pgfqpoint{3.668476in}{1.033707in}}%
\pgfpathlineto{\pgfqpoint{3.669343in}{1.563172in}}%
\pgfpathlineto{\pgfqpoint{3.670210in}{1.344640in}}%
\pgfpathlineto{\pgfqpoint{3.671077in}{1.606777in}}%
\pgfpathlineto{\pgfqpoint{3.673678in}{0.832853in}}%
\pgfpathlineto{\pgfqpoint{3.675413in}{1.526659in}}%
\pgfpathlineto{\pgfqpoint{3.676280in}{0.980901in}}%
\pgfpathlineto{\pgfqpoint{3.677147in}{1.088144in}}%
\pgfpathlineto{\pgfqpoint{3.678014in}{0.886219in}}%
\pgfpathlineto{\pgfqpoint{3.679748in}{1.227439in}}%
\pgfpathlineto{\pgfqpoint{3.680615in}{1.282813in}}%
\pgfpathlineto{\pgfqpoint{3.681483in}{1.053823in}}%
\pgfpathlineto{\pgfqpoint{3.682350in}{1.140925in}}%
\pgfpathlineto{\pgfqpoint{3.683217in}{0.731421in}}%
\pgfpathlineto{\pgfqpoint{3.684084in}{1.651813in}}%
\pgfpathlineto{\pgfqpoint{3.684951in}{0.906226in}}%
\pgfpathlineto{\pgfqpoint{3.685818in}{1.167213in}}%
\pgfpathlineto{\pgfqpoint{3.686685in}{0.930861in}}%
\pgfpathlineto{\pgfqpoint{3.687552in}{1.098190in}}%
\pgfpathlineto{\pgfqpoint{3.689287in}{0.879695in}}%
\pgfpathlineto{\pgfqpoint{3.691021in}{1.253728in}}%
\pgfpathlineto{\pgfqpoint{3.692755in}{0.748160in}}%
\pgfpathlineto{\pgfqpoint{3.694490in}{1.325202in}}%
\pgfpathlineto{\pgfqpoint{3.695357in}{0.871608in}}%
\pgfpathlineto{\pgfqpoint{3.697091in}{1.793618in}}%
\pgfpathlineto{\pgfqpoint{3.698825in}{0.888343in}}%
\pgfpathlineto{\pgfqpoint{3.701427in}{1.362923in}}%
\pgfpathlineto{\pgfqpoint{3.703161in}{0.641355in}}%
\pgfpathlineto{\pgfqpoint{3.704028in}{1.254037in}}%
\pgfpathlineto{\pgfqpoint{3.705762in}{0.644915in}}%
\pgfpathlineto{\pgfqpoint{3.707497in}{1.430483in}}%
\pgfpathlineto{\pgfqpoint{3.709231in}{0.957171in}}%
\pgfpathlineto{\pgfqpoint{3.710965in}{1.288236in}}%
\pgfpathlineto{\pgfqpoint{3.711832in}{1.001973in}}%
\pgfpathlineto{\pgfqpoint{3.713566in}{1.374147in}}%
\pgfpathlineto{\pgfqpoint{3.714434in}{0.908392in}}%
\pgfpathlineto{\pgfqpoint{3.715301in}{1.121034in}}%
\pgfpathlineto{\pgfqpoint{3.717035in}{0.784912in}}%
\pgfpathlineto{\pgfqpoint{3.717902in}{1.326730in}}%
\pgfpathlineto{\pgfqpoint{3.718769in}{0.815298in}}%
\pgfpathlineto{\pgfqpoint{3.719636in}{0.930883in}}%
\pgfpathlineto{\pgfqpoint{3.720503in}{1.124548in}}%
\pgfpathlineto{\pgfqpoint{3.721371in}{1.022519in}}%
\pgfpathlineto{\pgfqpoint{3.722238in}{1.206860in}}%
\pgfpathlineto{\pgfqpoint{3.723105in}{0.715160in}}%
\pgfpathlineto{\pgfqpoint{3.723972in}{1.493093in}}%
\pgfpathlineto{\pgfqpoint{3.725706in}{1.024153in}}%
\pgfpathlineto{\pgfqpoint{3.726573in}{1.185495in}}%
\pgfpathlineto{\pgfqpoint{3.727441in}{0.792703in}}%
\pgfpathlineto{\pgfqpoint{3.729175in}{1.699586in}}%
\pgfpathlineto{\pgfqpoint{3.730909in}{0.843424in}}%
\pgfpathlineto{\pgfqpoint{3.731776in}{0.680247in}}%
\pgfpathlineto{\pgfqpoint{3.732643in}{0.756007in}}%
\pgfpathlineto{\pgfqpoint{3.733510in}{1.116379in}}%
\pgfpathlineto{\pgfqpoint{3.734378in}{1.044951in}}%
\pgfpathlineto{\pgfqpoint{3.735245in}{1.129645in}}%
\pgfpathlineto{\pgfqpoint{3.736979in}{0.958898in}}%
\pgfpathlineto{\pgfqpoint{3.737846in}{1.310225in}}%
\pgfpathlineto{\pgfqpoint{3.738713in}{1.069385in}}%
\pgfpathlineto{\pgfqpoint{3.739580in}{1.684500in}}%
\pgfpathlineto{\pgfqpoint{3.740448in}{0.766914in}}%
\pgfpathlineto{\pgfqpoint{3.742182in}{1.664886in}}%
\pgfpathlineto{\pgfqpoint{3.743049in}{0.797238in}}%
\pgfpathlineto{\pgfqpoint{3.743916in}{1.419103in}}%
\pgfpathlineto{\pgfqpoint{3.744783in}{1.048742in}}%
\pgfpathlineto{\pgfqpoint{3.745650in}{1.461484in}}%
\pgfpathlineto{\pgfqpoint{3.748252in}{0.910602in}}%
\pgfpathlineto{\pgfqpoint{3.749119in}{1.535875in}}%
\pgfpathlineto{\pgfqpoint{3.751720in}{0.625952in}}%
\pgfpathlineto{\pgfqpoint{3.752587in}{1.213729in}}%
\pgfpathlineto{\pgfqpoint{3.753455in}{0.790651in}}%
\pgfpathlineto{\pgfqpoint{3.754322in}{0.812674in}}%
\pgfpathlineto{\pgfqpoint{3.755189in}{1.096848in}}%
\pgfpathlineto{\pgfqpoint{3.756056in}{0.814862in}}%
\pgfpathlineto{\pgfqpoint{3.756923in}{1.033537in}}%
\pgfpathlineto{\pgfqpoint{3.757790in}{1.762379in}}%
\pgfpathlineto{\pgfqpoint{3.758657in}{0.907221in}}%
\pgfpathlineto{\pgfqpoint{3.759524in}{1.370206in}}%
\pgfpathlineto{\pgfqpoint{3.760392in}{0.799431in}}%
\pgfpathlineto{\pgfqpoint{3.761259in}{1.509317in}}%
\pgfpathlineto{\pgfqpoint{3.762126in}{1.007141in}}%
\pgfpathlineto{\pgfqpoint{3.762993in}{1.623026in}}%
\pgfpathlineto{\pgfqpoint{3.763860in}{0.749628in}}%
\pgfpathlineto{\pgfqpoint{3.764727in}{1.386727in}}%
\pgfpathlineto{\pgfqpoint{3.766462in}{0.724006in}}%
\pgfpathlineto{\pgfqpoint{3.767329in}{1.376207in}}%
\pgfpathlineto{\pgfqpoint{3.769063in}{1.017083in}}%
\pgfpathlineto{\pgfqpoint{3.770797in}{1.289555in}}%
\pgfpathlineto{\pgfqpoint{3.771664in}{1.204299in}}%
\pgfpathlineto{\pgfqpoint{3.772531in}{1.317246in}}%
\pgfpathlineto{\pgfqpoint{3.773399in}{0.851487in}}%
\pgfpathlineto{\pgfqpoint{3.774266in}{1.040843in}}%
\pgfpathlineto{\pgfqpoint{3.775133in}{0.811897in}}%
\pgfpathlineto{\pgfqpoint{3.776000in}{1.079396in}}%
\pgfpathlineto{\pgfqpoint{3.776867in}{0.831884in}}%
\pgfpathlineto{\pgfqpoint{3.777734in}{1.247326in}}%
\pgfpathlineto{\pgfqpoint{3.779469in}{0.953846in}}%
\pgfpathlineto{\pgfqpoint{3.780336in}{0.995931in}}%
\pgfpathlineto{\pgfqpoint{3.782070in}{0.653343in}}%
\pgfpathlineto{\pgfqpoint{3.782937in}{1.128828in}}%
\pgfpathlineto{\pgfqpoint{3.783804in}{0.782549in}}%
\pgfpathlineto{\pgfqpoint{3.784671in}{1.647215in}}%
\pgfpathlineto{\pgfqpoint{3.786406in}{0.663748in}}%
\pgfpathlineto{\pgfqpoint{3.787273in}{0.699329in}}%
\pgfpathlineto{\pgfqpoint{3.788140in}{1.171299in}}%
\pgfpathlineto{\pgfqpoint{3.789007in}{0.679112in}}%
\pgfpathlineto{\pgfqpoint{3.790741in}{1.444548in}}%
\pgfpathlineto{\pgfqpoint{3.793343in}{0.699750in}}%
\pgfpathlineto{\pgfqpoint{3.794210in}{1.664109in}}%
\pgfpathlineto{\pgfqpoint{3.795077in}{1.608802in}}%
\pgfpathlineto{\pgfqpoint{3.795944in}{1.023498in}}%
\pgfpathlineto{\pgfqpoint{3.796811in}{1.090892in}}%
\pgfpathlineto{\pgfqpoint{3.797678in}{1.246656in}}%
\pgfpathlineto{\pgfqpoint{3.798545in}{0.735364in}}%
\pgfpathlineto{\pgfqpoint{3.799413in}{1.043987in}}%
\pgfpathlineto{\pgfqpoint{3.800280in}{0.947707in}}%
\pgfpathlineto{\pgfqpoint{3.801147in}{1.085054in}}%
\pgfpathlineto{\pgfqpoint{3.802014in}{0.991876in}}%
\pgfpathlineto{\pgfqpoint{3.802881in}{1.584003in}}%
\pgfpathlineto{\pgfqpoint{3.803748in}{1.453052in}}%
\pgfpathlineto{\pgfqpoint{3.805483in}{0.741792in}}%
\pgfpathlineto{\pgfqpoint{3.806350in}{0.747088in}}%
\pgfpathlineto{\pgfqpoint{3.807217in}{0.932709in}}%
\pgfpathlineto{\pgfqpoint{3.808084in}{0.741342in}}%
\pgfpathlineto{\pgfqpoint{3.808951in}{1.154023in}}%
\pgfpathlineto{\pgfqpoint{3.809818in}{0.814759in}}%
\pgfpathlineto{\pgfqpoint{3.810685in}{0.889907in}}%
\pgfpathlineto{\pgfqpoint{3.811552in}{0.849393in}}%
\pgfpathlineto{\pgfqpoint{3.812420in}{1.451587in}}%
\pgfpathlineto{\pgfqpoint{3.813287in}{1.315450in}}%
\pgfpathlineto{\pgfqpoint{3.814154in}{0.982919in}}%
\pgfpathlineto{\pgfqpoint{3.815888in}{1.127760in}}%
\pgfpathlineto{\pgfqpoint{3.816755in}{1.012902in}}%
\pgfpathlineto{\pgfqpoint{3.817622in}{1.087331in}}%
\pgfpathlineto{\pgfqpoint{3.818490in}{1.587696in}}%
\pgfpathlineto{\pgfqpoint{3.819357in}{1.039414in}}%
\pgfpathlineto{\pgfqpoint{3.820224in}{1.414714in}}%
\pgfpathlineto{\pgfqpoint{3.821091in}{1.170411in}}%
\pgfpathlineto{\pgfqpoint{3.821958in}{1.615652in}}%
\pgfpathlineto{\pgfqpoint{3.823692in}{0.799550in}}%
\pgfpathlineto{\pgfqpoint{3.824559in}{0.797095in}}%
\pgfpathlineto{\pgfqpoint{3.825427in}{0.935514in}}%
\pgfpathlineto{\pgfqpoint{3.826294in}{0.681406in}}%
\pgfpathlineto{\pgfqpoint{3.827161in}{1.673570in}}%
\pgfpathlineto{\pgfqpoint{3.828028in}{0.703500in}}%
\pgfpathlineto{\pgfqpoint{3.828895in}{1.506289in}}%
\pgfpathlineto{\pgfqpoint{3.830629in}{0.771082in}}%
\pgfpathlineto{\pgfqpoint{3.831497in}{0.867949in}}%
\pgfpathlineto{\pgfqpoint{3.832364in}{1.172691in}}%
\pgfpathlineto{\pgfqpoint{3.833231in}{0.860329in}}%
\pgfpathlineto{\pgfqpoint{3.834098in}{1.333440in}}%
\pgfpathlineto{\pgfqpoint{3.834965in}{1.236623in}}%
\pgfpathlineto{\pgfqpoint{3.835832in}{1.342019in}}%
\pgfpathlineto{\pgfqpoint{3.836699in}{0.716874in}}%
\pgfpathlineto{\pgfqpoint{3.837566in}{0.867307in}}%
\pgfpathlineto{\pgfqpoint{3.838434in}{1.228880in}}%
\pgfpathlineto{\pgfqpoint{3.840168in}{0.810783in}}%
\pgfpathlineto{\pgfqpoint{3.842769in}{1.728699in}}%
\pgfpathlineto{\pgfqpoint{3.843636in}{0.848884in}}%
\pgfpathlineto{\pgfqpoint{3.844503in}{0.893773in}}%
\pgfpathlineto{\pgfqpoint{3.846238in}{1.019561in}}%
\pgfpathlineto{\pgfqpoint{3.847105in}{0.922212in}}%
\pgfpathlineto{\pgfqpoint{3.847972in}{0.652267in}}%
\pgfpathlineto{\pgfqpoint{3.848839in}{1.042965in}}%
\pgfpathlineto{\pgfqpoint{3.849706in}{0.864968in}}%
\pgfpathlineto{\pgfqpoint{3.850573in}{0.958238in}}%
\pgfpathlineto{\pgfqpoint{3.851441in}{1.380527in}}%
\pgfpathlineto{\pgfqpoint{3.853175in}{0.960717in}}%
\pgfpathlineto{\pgfqpoint{3.854909in}{1.739932in}}%
\pgfpathlineto{\pgfqpoint{3.855776in}{0.855394in}}%
\pgfpathlineto{\pgfqpoint{3.857510in}{1.338651in}}%
\pgfpathlineto{\pgfqpoint{3.858378in}{1.221558in}}%
\pgfpathlineto{\pgfqpoint{3.859245in}{0.803525in}}%
\pgfpathlineto{\pgfqpoint{3.861846in}{1.523755in}}%
\pgfpathlineto{\pgfqpoint{3.862713in}{0.775632in}}%
\pgfpathlineto{\pgfqpoint{3.863580in}{1.442923in}}%
\pgfpathlineto{\pgfqpoint{3.865315in}{0.837065in}}%
\pgfpathlineto{\pgfqpoint{3.866182in}{1.288921in}}%
\pgfpathlineto{\pgfqpoint{3.867049in}{0.909065in}}%
\pgfpathlineto{\pgfqpoint{3.867916in}{0.973125in}}%
\pgfpathlineto{\pgfqpoint{3.869650in}{1.242293in}}%
\pgfpathlineto{\pgfqpoint{3.870517in}{0.881708in}}%
\pgfpathlineto{\pgfqpoint{3.872252in}{1.320735in}}%
\pgfpathlineto{\pgfqpoint{3.873119in}{1.162672in}}%
\pgfpathlineto{\pgfqpoint{3.873986in}{1.391155in}}%
\pgfpathlineto{\pgfqpoint{3.874853in}{1.341461in}}%
\pgfpathlineto{\pgfqpoint{3.875720in}{1.460513in}}%
\pgfpathlineto{\pgfqpoint{3.876587in}{1.030295in}}%
\pgfpathlineto{\pgfqpoint{3.877455in}{1.230876in}}%
\pgfpathlineto{\pgfqpoint{3.879189in}{1.054440in}}%
\pgfpathlineto{\pgfqpoint{3.880056in}{0.830720in}}%
\pgfpathlineto{\pgfqpoint{3.880923in}{0.838509in}}%
\pgfpathlineto{\pgfqpoint{3.881790in}{1.085595in}}%
\pgfpathlineto{\pgfqpoint{3.882657in}{0.794735in}}%
\pgfpathlineto{\pgfqpoint{3.886126in}{1.282268in}}%
\pgfpathlineto{\pgfqpoint{3.886993in}{0.910009in}}%
\pgfpathlineto{\pgfqpoint{3.887860in}{1.597939in}}%
\pgfpathlineto{\pgfqpoint{3.888727in}{0.736301in}}%
\pgfpathlineto{\pgfqpoint{3.890462in}{1.168265in}}%
\pgfpathlineto{\pgfqpoint{3.891329in}{0.889406in}}%
\pgfpathlineto{\pgfqpoint{3.893063in}{1.005525in}}%
\pgfpathlineto{\pgfqpoint{3.893930in}{1.490890in}}%
\pgfpathlineto{\pgfqpoint{3.894797in}{0.992977in}}%
\pgfpathlineto{\pgfqpoint{3.895664in}{1.100849in}}%
\pgfpathlineto{\pgfqpoint{3.896531in}{1.529289in}}%
\pgfpathlineto{\pgfqpoint{3.898266in}{1.009341in}}%
\pgfpathlineto{\pgfqpoint{3.900000in}{1.561795in}}%
\pgfpathlineto{\pgfqpoint{3.900867in}{1.177583in}}%
\pgfpathlineto{\pgfqpoint{3.901734in}{1.239850in}}%
\pgfpathlineto{\pgfqpoint{3.902601in}{1.035767in}}%
\pgfpathlineto{\pgfqpoint{3.903469in}{1.351169in}}%
\pgfpathlineto{\pgfqpoint{3.905203in}{0.993355in}}%
\pgfpathlineto{\pgfqpoint{3.906070in}{0.952881in}}%
\pgfpathlineto{\pgfqpoint{3.906937in}{0.703752in}}%
\pgfpathlineto{\pgfqpoint{3.909538in}{1.051009in}}%
\pgfpathlineto{\pgfqpoint{3.910406in}{0.996609in}}%
\pgfpathlineto{\pgfqpoint{3.911273in}{1.047040in}}%
\pgfpathlineto{\pgfqpoint{3.912140in}{1.885964in}}%
\pgfpathlineto{\pgfqpoint{3.913007in}{0.788442in}}%
\pgfpathlineto{\pgfqpoint{3.913874in}{1.660999in}}%
\pgfpathlineto{\pgfqpoint{3.915608in}{1.102631in}}%
\pgfpathlineto{\pgfqpoint{3.916476in}{0.852839in}}%
\pgfpathlineto{\pgfqpoint{3.918210in}{1.271033in}}%
\pgfpathlineto{\pgfqpoint{3.919077in}{0.749134in}}%
\pgfpathlineto{\pgfqpoint{3.919944in}{0.868151in}}%
\pgfpathlineto{\pgfqpoint{3.920811in}{0.756930in}}%
\pgfpathlineto{\pgfqpoint{3.921678in}{0.787704in}}%
\pgfpathlineto{\pgfqpoint{3.922545in}{1.020345in}}%
\pgfpathlineto{\pgfqpoint{3.923413in}{0.876017in}}%
\pgfpathlineto{\pgfqpoint{3.924280in}{1.350399in}}%
\pgfpathlineto{\pgfqpoint{3.926014in}{0.831987in}}%
\pgfpathlineto{\pgfqpoint{3.927748in}{1.053750in}}%
\pgfpathlineto{\pgfqpoint{3.928615in}{0.838361in}}%
\pgfpathlineto{\pgfqpoint{3.929483in}{1.093685in}}%
\pgfpathlineto{\pgfqpoint{3.930350in}{0.918413in}}%
\pgfpathlineto{\pgfqpoint{3.931217in}{1.004405in}}%
\pgfpathlineto{\pgfqpoint{3.932084in}{1.581731in}}%
\pgfpathlineto{\pgfqpoint{3.932951in}{0.797629in}}%
\pgfpathlineto{\pgfqpoint{3.933818in}{0.918307in}}%
\pgfpathlineto{\pgfqpoint{3.934685in}{1.232659in}}%
\pgfpathlineto{\pgfqpoint{3.935552in}{1.181246in}}%
\pgfpathlineto{\pgfqpoint{3.936420in}{1.282885in}}%
\pgfpathlineto{\pgfqpoint{3.937287in}{0.993879in}}%
\pgfpathlineto{\pgfqpoint{3.939021in}{1.679675in}}%
\pgfpathlineto{\pgfqpoint{3.940755in}{0.832308in}}%
\pgfpathlineto{\pgfqpoint{3.941622in}{0.843421in}}%
\pgfpathlineto{\pgfqpoint{3.942490in}{0.963646in}}%
\pgfpathlineto{\pgfqpoint{3.943357in}{1.511666in}}%
\pgfpathlineto{\pgfqpoint{3.945958in}{0.933420in}}%
\pgfpathlineto{\pgfqpoint{3.946825in}{1.448891in}}%
\pgfpathlineto{\pgfqpoint{3.947692in}{0.714697in}}%
\pgfpathlineto{\pgfqpoint{3.949427in}{1.457637in}}%
\pgfpathlineto{\pgfqpoint{3.951161in}{0.864305in}}%
\pgfpathlineto{\pgfqpoint{3.952895in}{1.532782in}}%
\pgfpathlineto{\pgfqpoint{3.953762in}{0.978305in}}%
\pgfpathlineto{\pgfqpoint{3.955497in}{1.492981in}}%
\pgfpathlineto{\pgfqpoint{3.958098in}{1.059781in}}%
\pgfpathlineto{\pgfqpoint{3.959832in}{0.809533in}}%
\pgfpathlineto{\pgfqpoint{3.960699in}{0.807950in}}%
\pgfpathlineto{\pgfqpoint{3.961566in}{1.431116in}}%
\pgfpathlineto{\pgfqpoint{3.962434in}{0.962163in}}%
\pgfpathlineto{\pgfqpoint{3.963301in}{1.351674in}}%
\pgfpathlineto{\pgfqpoint{3.964168in}{0.939287in}}%
\pgfpathlineto{\pgfqpoint{3.965035in}{1.260177in}}%
\pgfpathlineto{\pgfqpoint{3.965902in}{1.209674in}}%
\pgfpathlineto{\pgfqpoint{3.966769in}{1.126739in}}%
\pgfpathlineto{\pgfqpoint{3.967636in}{1.302124in}}%
\pgfpathlineto{\pgfqpoint{3.968503in}{1.065167in}}%
\pgfpathlineto{\pgfqpoint{3.969371in}{1.281367in}}%
\pgfpathlineto{\pgfqpoint{3.971105in}{0.811184in}}%
\pgfpathlineto{\pgfqpoint{3.972839in}{1.627374in}}%
\pgfpathlineto{\pgfqpoint{3.975441in}{0.786085in}}%
\pgfpathlineto{\pgfqpoint{3.977175in}{1.412360in}}%
\pgfpathlineto{\pgfqpoint{3.978042in}{1.307777in}}%
\pgfpathlineto{\pgfqpoint{3.978909in}{1.372177in}}%
\pgfpathlineto{\pgfqpoint{3.980643in}{0.821927in}}%
\pgfpathlineto{\pgfqpoint{3.981510in}{1.732732in}}%
\pgfpathlineto{\pgfqpoint{3.982378in}{0.999643in}}%
\pgfpathlineto{\pgfqpoint{3.983245in}{1.107869in}}%
\pgfpathlineto{\pgfqpoint{3.984979in}{1.149546in}}%
\pgfpathlineto{\pgfqpoint{3.985846in}{1.659158in}}%
\pgfpathlineto{\pgfqpoint{3.987580in}{0.860612in}}%
\pgfpathlineto{\pgfqpoint{3.988448in}{1.307816in}}%
\pgfpathlineto{\pgfqpoint{3.989315in}{1.148914in}}%
\pgfpathlineto{\pgfqpoint{3.990182in}{0.698783in}}%
\pgfpathlineto{\pgfqpoint{3.991916in}{1.272247in}}%
\pgfpathlineto{\pgfqpoint{3.992783in}{0.855439in}}%
\pgfpathlineto{\pgfqpoint{3.993650in}{0.868574in}}%
\pgfpathlineto{\pgfqpoint{3.997119in}{1.602695in}}%
\pgfpathlineto{\pgfqpoint{3.998853in}{0.711736in}}%
\pgfpathlineto{\pgfqpoint{3.999720in}{1.488386in}}%
\pgfpathlineto{\pgfqpoint{4.000587in}{0.756469in}}%
\pgfpathlineto{\pgfqpoint{4.001455in}{1.029463in}}%
\pgfpathlineto{\pgfqpoint{4.002322in}{0.738754in}}%
\pgfpathlineto{\pgfqpoint{4.003189in}{0.932991in}}%
\pgfpathlineto{\pgfqpoint{4.004056in}{1.635641in}}%
\pgfpathlineto{\pgfqpoint{4.004923in}{1.243024in}}%
\pgfpathlineto{\pgfqpoint{4.005790in}{1.392773in}}%
\pgfpathlineto{\pgfqpoint{4.006657in}{0.857166in}}%
\pgfpathlineto{\pgfqpoint{4.007524in}{1.471536in}}%
\pgfpathlineto{\pgfqpoint{4.010126in}{0.718798in}}%
\pgfpathlineto{\pgfqpoint{4.011860in}{1.618575in}}%
\pgfpathlineto{\pgfqpoint{4.012727in}{1.523089in}}%
\pgfpathlineto{\pgfqpoint{4.013594in}{1.422311in}}%
\pgfpathlineto{\pgfqpoint{4.014462in}{0.837451in}}%
\pgfpathlineto{\pgfqpoint{4.016196in}{1.268194in}}%
\pgfpathlineto{\pgfqpoint{4.017063in}{1.264494in}}%
\pgfpathlineto{\pgfqpoint{4.017930in}{1.450950in}}%
\pgfpathlineto{\pgfqpoint{4.020531in}{0.851919in}}%
\pgfpathlineto{\pgfqpoint{4.021399in}{1.387738in}}%
\pgfpathlineto{\pgfqpoint{4.022266in}{1.223098in}}%
\pgfpathlineto{\pgfqpoint{4.023133in}{0.790535in}}%
\pgfpathlineto{\pgfqpoint{4.024000in}{1.319868in}}%
\pgfpathlineto{\pgfqpoint{4.024867in}{0.880295in}}%
\pgfpathlineto{\pgfqpoint{4.025734in}{0.968663in}}%
\pgfpathlineto{\pgfqpoint{4.027469in}{1.570968in}}%
\pgfpathlineto{\pgfqpoint{4.028336in}{0.919068in}}%
\pgfpathlineto{\pgfqpoint{4.029203in}{0.950180in}}%
\pgfpathlineto{\pgfqpoint{4.030070in}{1.228522in}}%
\pgfpathlineto{\pgfqpoint{4.030937in}{1.182009in}}%
\pgfpathlineto{\pgfqpoint{4.031804in}{1.235366in}}%
\pgfpathlineto{\pgfqpoint{4.032671in}{1.230189in}}%
\pgfpathlineto{\pgfqpoint{4.033538in}{1.203315in}}%
\pgfpathlineto{\pgfqpoint{4.034406in}{1.417135in}}%
\pgfpathlineto{\pgfqpoint{4.035273in}{1.075937in}}%
\pgfpathlineto{\pgfqpoint{4.037007in}{1.437506in}}%
\pgfpathlineto{\pgfqpoint{4.037874in}{0.890501in}}%
\pgfpathlineto{\pgfqpoint{4.038741in}{1.477810in}}%
\pgfpathlineto{\pgfqpoint{4.039608in}{1.229020in}}%
\pgfpathlineto{\pgfqpoint{4.040476in}{1.599635in}}%
\pgfpathlineto{\pgfqpoint{4.042210in}{0.831071in}}%
\pgfpathlineto{\pgfqpoint{4.043077in}{1.513027in}}%
\pgfpathlineto{\pgfqpoint{4.043944in}{1.000292in}}%
\pgfpathlineto{\pgfqpoint{4.044811in}{1.048534in}}%
\pgfpathlineto{\pgfqpoint{4.046545in}{1.140977in}}%
\pgfpathlineto{\pgfqpoint{4.047413in}{1.734177in}}%
\pgfpathlineto{\pgfqpoint{4.049147in}{0.761468in}}%
\pgfpathlineto{\pgfqpoint{4.050014in}{1.653043in}}%
\pgfpathlineto{\pgfqpoint{4.050881in}{0.939023in}}%
\pgfpathlineto{\pgfqpoint{4.051748in}{1.374789in}}%
\pgfpathlineto{\pgfqpoint{4.052615in}{0.875230in}}%
\pgfpathlineto{\pgfqpoint{4.053483in}{1.003785in}}%
\pgfpathlineto{\pgfqpoint{4.054350in}{0.976666in}}%
\pgfpathlineto{\pgfqpoint{4.056084in}{1.601120in}}%
\pgfpathlineto{\pgfqpoint{4.057818in}{0.734659in}}%
\pgfpathlineto{\pgfqpoint{4.058685in}{0.763006in}}%
\pgfpathlineto{\pgfqpoint{4.060420in}{1.235563in}}%
\pgfpathlineto{\pgfqpoint{4.061287in}{0.993691in}}%
\pgfpathlineto{\pgfqpoint{4.062154in}{1.213593in}}%
\pgfpathlineto{\pgfqpoint{4.063888in}{0.935371in}}%
\pgfpathlineto{\pgfqpoint{4.064755in}{0.999771in}}%
\pgfpathlineto{\pgfqpoint{4.066490in}{1.488442in}}%
\pgfpathlineto{\pgfqpoint{4.067357in}{1.340638in}}%
\pgfpathlineto{\pgfqpoint{4.069091in}{1.508159in}}%
\pgfpathlineto{\pgfqpoint{4.069958in}{1.774170in}}%
\pgfpathlineto{\pgfqpoint{4.070825in}{0.786410in}}%
\pgfpathlineto{\pgfqpoint{4.071692in}{1.072341in}}%
\pgfpathlineto{\pgfqpoint{4.072559in}{1.000311in}}%
\pgfpathlineto{\pgfqpoint{4.073427in}{0.891211in}}%
\pgfpathlineto{\pgfqpoint{4.074294in}{0.988371in}}%
\pgfpathlineto{\pgfqpoint{4.075161in}{1.217468in}}%
\pgfpathlineto{\pgfqpoint{4.076028in}{0.858023in}}%
\pgfpathlineto{\pgfqpoint{4.076895in}{0.873577in}}%
\pgfpathlineto{\pgfqpoint{4.077762in}{0.915421in}}%
\pgfpathlineto{\pgfqpoint{4.078629in}{1.335127in}}%
\pgfpathlineto{\pgfqpoint{4.080364in}{0.694003in}}%
\pgfpathlineto{\pgfqpoint{4.082098in}{1.485315in}}%
\pgfpathlineto{\pgfqpoint{4.082965in}{0.981116in}}%
\pgfpathlineto{\pgfqpoint{4.083832in}{1.388182in}}%
\pgfpathlineto{\pgfqpoint{4.085566in}{1.135028in}}%
\pgfpathlineto{\pgfqpoint{4.086434in}{1.355643in}}%
\pgfpathlineto{\pgfqpoint{4.088168in}{0.935486in}}%
\pgfpathlineto{\pgfqpoint{4.089035in}{1.095410in}}%
\pgfpathlineto{\pgfqpoint{4.089902in}{1.056486in}}%
\pgfpathlineto{\pgfqpoint{4.090769in}{0.819223in}}%
\pgfpathlineto{\pgfqpoint{4.091636in}{1.765679in}}%
\pgfpathlineto{\pgfqpoint{4.093371in}{1.123756in}}%
\pgfpathlineto{\pgfqpoint{4.094238in}{1.467136in}}%
\pgfpathlineto{\pgfqpoint{4.096839in}{0.921630in}}%
\pgfpathlineto{\pgfqpoint{4.097706in}{0.929488in}}%
\pgfpathlineto{\pgfqpoint{4.098573in}{1.262778in}}%
\pgfpathlineto{\pgfqpoint{4.099441in}{1.202308in}}%
\pgfpathlineto{\pgfqpoint{4.100308in}{1.117621in}}%
\pgfpathlineto{\pgfqpoint{4.101175in}{0.812226in}}%
\pgfpathlineto{\pgfqpoint{4.102042in}{0.898382in}}%
\pgfpathlineto{\pgfqpoint{4.102909in}{0.751867in}}%
\pgfpathlineto{\pgfqpoint{4.104643in}{0.900953in}}%
\pgfpathlineto{\pgfqpoint{4.105510in}{1.707198in}}%
\pgfpathlineto{\pgfqpoint{4.107245in}{0.908766in}}%
\pgfpathlineto{\pgfqpoint{4.108112in}{0.856962in}}%
\pgfpathlineto{\pgfqpoint{4.108979in}{0.623935in}}%
\pgfpathlineto{\pgfqpoint{4.109846in}{1.222650in}}%
\pgfpathlineto{\pgfqpoint{4.110713in}{1.084618in}}%
\pgfpathlineto{\pgfqpoint{4.111580in}{1.294954in}}%
\pgfpathlineto{\pgfqpoint{4.112448in}{0.904816in}}%
\pgfpathlineto{\pgfqpoint{4.113315in}{1.115430in}}%
\pgfpathlineto{\pgfqpoint{4.114182in}{0.990430in}}%
\pgfpathlineto{\pgfqpoint{4.115049in}{1.044617in}}%
\pgfpathlineto{\pgfqpoint{4.115916in}{1.541314in}}%
\pgfpathlineto{\pgfqpoint{4.116783in}{1.409123in}}%
\pgfpathlineto{\pgfqpoint{4.117650in}{0.870708in}}%
\pgfpathlineto{\pgfqpoint{4.118517in}{1.206238in}}%
\pgfpathlineto{\pgfqpoint{4.119385in}{0.849366in}}%
\pgfpathlineto{\pgfqpoint{4.120252in}{0.888623in}}%
\pgfpathlineto{\pgfqpoint{4.121986in}{1.067516in}}%
\pgfpathlineto{\pgfqpoint{4.122853in}{1.304145in}}%
\pgfpathlineto{\pgfqpoint{4.123720in}{1.133891in}}%
\pgfpathlineto{\pgfqpoint{4.124587in}{1.288933in}}%
\pgfpathlineto{\pgfqpoint{4.125455in}{0.903393in}}%
\pgfpathlineto{\pgfqpoint{4.127189in}{1.371551in}}%
\pgfpathlineto{\pgfqpoint{4.128056in}{0.980889in}}%
\pgfpathlineto{\pgfqpoint{4.128923in}{1.555316in}}%
\pgfpathlineto{\pgfqpoint{4.130657in}{1.023866in}}%
\pgfpathlineto{\pgfqpoint{4.131524in}{1.232764in}}%
\pgfpathlineto{\pgfqpoint{4.132392in}{1.084275in}}%
\pgfpathlineto{\pgfqpoint{4.134126in}{1.368887in}}%
\pgfpathlineto{\pgfqpoint{4.134993in}{0.770792in}}%
\pgfpathlineto{\pgfqpoint{4.135860in}{1.271691in}}%
\pgfpathlineto{\pgfqpoint{4.136727in}{1.057176in}}%
\pgfpathlineto{\pgfqpoint{4.137594in}{1.203179in}}%
\pgfpathlineto{\pgfqpoint{4.138462in}{1.115720in}}%
\pgfpathlineto{\pgfqpoint{4.139329in}{1.548124in}}%
\pgfpathlineto{\pgfqpoint{4.141063in}{0.784726in}}%
\pgfpathlineto{\pgfqpoint{4.141930in}{1.231766in}}%
\pgfpathlineto{\pgfqpoint{4.142797in}{0.736419in}}%
\pgfpathlineto{\pgfqpoint{4.144531in}{1.593863in}}%
\pgfpathlineto{\pgfqpoint{4.145399in}{1.001459in}}%
\pgfpathlineto{\pgfqpoint{4.146266in}{1.364631in}}%
\pgfpathlineto{\pgfqpoint{4.148000in}{1.083070in}}%
\pgfpathlineto{\pgfqpoint{4.148867in}{1.246970in}}%
\pgfpathlineto{\pgfqpoint{4.149734in}{0.891416in}}%
\pgfpathlineto{\pgfqpoint{4.150601in}{1.195466in}}%
\pgfpathlineto{\pgfqpoint{4.151469in}{0.874558in}}%
\pgfpathlineto{\pgfqpoint{4.152336in}{1.046918in}}%
\pgfpathlineto{\pgfqpoint{4.153203in}{0.833573in}}%
\pgfpathlineto{\pgfqpoint{4.154070in}{0.951495in}}%
\pgfpathlineto{\pgfqpoint{4.154937in}{1.269538in}}%
\pgfpathlineto{\pgfqpoint{4.155804in}{0.792958in}}%
\pgfpathlineto{\pgfqpoint{4.156671in}{1.207410in}}%
\pgfpathlineto{\pgfqpoint{4.157538in}{1.146295in}}%
\pgfpathlineto{\pgfqpoint{4.158406in}{1.048725in}}%
\pgfpathlineto{\pgfqpoint{4.159273in}{1.325784in}}%
\pgfpathlineto{\pgfqpoint{4.161007in}{0.818346in}}%
\pgfpathlineto{\pgfqpoint{4.161874in}{1.208211in}}%
\pgfpathlineto{\pgfqpoint{4.163608in}{0.716490in}}%
\pgfpathlineto{\pgfqpoint{4.164476in}{1.003790in}}%
\pgfpathlineto{\pgfqpoint{4.165343in}{0.948945in}}%
\pgfpathlineto{\pgfqpoint{4.166210in}{1.079545in}}%
\pgfpathlineto{\pgfqpoint{4.167077in}{1.409313in}}%
\pgfpathlineto{\pgfqpoint{4.167944in}{1.032045in}}%
\pgfpathlineto{\pgfqpoint{4.168811in}{1.113197in}}%
\pgfpathlineto{\pgfqpoint{4.169678in}{0.806827in}}%
\pgfpathlineto{\pgfqpoint{4.170545in}{1.497900in}}%
\pgfpathlineto{\pgfqpoint{4.171413in}{1.315658in}}%
\pgfpathlineto{\pgfqpoint{4.173147in}{1.200603in}}%
\pgfpathlineto{\pgfqpoint{4.174014in}{1.196393in}}%
\pgfpathlineto{\pgfqpoint{4.174881in}{1.236526in}}%
\pgfpathlineto{\pgfqpoint{4.175748in}{1.026563in}}%
\pgfpathlineto{\pgfqpoint{4.176615in}{1.142762in}}%
\pgfpathlineto{\pgfqpoint{4.178350in}{0.675508in}}%
\pgfpathlineto{\pgfqpoint{4.179217in}{1.023036in}}%
\pgfpathlineto{\pgfqpoint{4.180084in}{0.884929in}}%
\pgfpathlineto{\pgfqpoint{4.180951in}{1.262437in}}%
\pgfpathlineto{\pgfqpoint{4.182685in}{0.876344in}}%
\pgfpathlineto{\pgfqpoint{4.183552in}{1.320466in}}%
\pgfpathlineto{\pgfqpoint{4.184420in}{1.304547in}}%
\pgfpathlineto{\pgfqpoint{4.185287in}{1.048564in}}%
\pgfpathlineto{\pgfqpoint{4.186154in}{1.468007in}}%
\pgfpathlineto{\pgfqpoint{4.187021in}{0.841849in}}%
\pgfpathlineto{\pgfqpoint{4.187888in}{0.914598in}}%
\pgfpathlineto{\pgfqpoint{4.188755in}{0.729415in}}%
\pgfpathlineto{\pgfqpoint{4.190490in}{1.071932in}}%
\pgfpathlineto{\pgfqpoint{4.191357in}{0.975006in}}%
\pgfpathlineto{\pgfqpoint{4.193091in}{1.388820in}}%
\pgfpathlineto{\pgfqpoint{4.194825in}{0.986333in}}%
\pgfpathlineto{\pgfqpoint{4.196559in}{0.888245in}}%
\pgfpathlineto{\pgfqpoint{4.198294in}{1.573078in}}%
\pgfpathlineto{\pgfqpoint{4.199161in}{0.909643in}}%
\pgfpathlineto{\pgfqpoint{4.200028in}{1.087044in}}%
\pgfpathlineto{\pgfqpoint{4.200895in}{0.720897in}}%
\pgfpathlineto{\pgfqpoint{4.201762in}{0.785252in}}%
\pgfpathlineto{\pgfqpoint{4.202629in}{1.051221in}}%
\pgfpathlineto{\pgfqpoint{4.203497in}{0.694622in}}%
\pgfpathlineto{\pgfqpoint{4.204364in}{1.022564in}}%
\pgfpathlineto{\pgfqpoint{4.205231in}{0.979453in}}%
\pgfpathlineto{\pgfqpoint{4.206098in}{0.935777in}}%
\pgfpathlineto{\pgfqpoint{4.206965in}{0.781482in}}%
\pgfpathlineto{\pgfqpoint{4.207832in}{0.948817in}}%
\pgfpathlineto{\pgfqpoint{4.208699in}{0.760966in}}%
\pgfpathlineto{\pgfqpoint{4.209566in}{1.028270in}}%
\pgfpathlineto{\pgfqpoint{4.210434in}{0.961855in}}%
\pgfpathlineto{\pgfqpoint{4.211301in}{0.848998in}}%
\pgfpathlineto{\pgfqpoint{4.212168in}{1.241036in}}%
\pgfpathlineto{\pgfqpoint{4.213035in}{0.952071in}}%
\pgfpathlineto{\pgfqpoint{4.213902in}{1.720135in}}%
\pgfpathlineto{\pgfqpoint{4.214769in}{1.598582in}}%
\pgfpathlineto{\pgfqpoint{4.216503in}{0.786220in}}%
\pgfpathlineto{\pgfqpoint{4.218238in}{1.344131in}}%
\pgfpathlineto{\pgfqpoint{4.219105in}{1.163773in}}%
\pgfpathlineto{\pgfqpoint{4.219972in}{0.828112in}}%
\pgfpathlineto{\pgfqpoint{4.221706in}{1.429165in}}%
\pgfpathlineto{\pgfqpoint{4.222573in}{0.710864in}}%
\pgfpathlineto{\pgfqpoint{4.225175in}{1.399790in}}%
\pgfpathlineto{\pgfqpoint{4.226042in}{1.419813in}}%
\pgfpathlineto{\pgfqpoint{4.226909in}{0.825619in}}%
\pgfpathlineto{\pgfqpoint{4.228643in}{1.216313in}}%
\pgfpathlineto{\pgfqpoint{4.229510in}{1.028633in}}%
\pgfpathlineto{\pgfqpoint{4.230378in}{1.307848in}}%
\pgfpathlineto{\pgfqpoint{4.231245in}{0.921817in}}%
\pgfpathlineto{\pgfqpoint{4.232112in}{1.223653in}}%
\pgfpathlineto{\pgfqpoint{4.232979in}{0.867876in}}%
\pgfpathlineto{\pgfqpoint{4.234713in}{1.423794in}}%
\pgfpathlineto{\pgfqpoint{4.235580in}{1.064230in}}%
\pgfpathlineto{\pgfqpoint{4.236448in}{1.439614in}}%
\pgfpathlineto{\pgfqpoint{4.237315in}{0.958341in}}%
\pgfpathlineto{\pgfqpoint{4.238182in}{1.175724in}}%
\pgfpathlineto{\pgfqpoint{4.241650in}{0.716589in}}%
\pgfpathlineto{\pgfqpoint{4.243385in}{1.054522in}}%
\pgfpathlineto{\pgfqpoint{4.244252in}{1.004276in}}%
\pgfpathlineto{\pgfqpoint{4.245119in}{1.065684in}}%
\pgfpathlineto{\pgfqpoint{4.245986in}{0.707300in}}%
\pgfpathlineto{\pgfqpoint{4.246853in}{0.750295in}}%
\pgfpathlineto{\pgfqpoint{4.247720in}{0.804198in}}%
\pgfpathlineto{\pgfqpoint{4.250322in}{1.456943in}}%
\pgfpathlineto{\pgfqpoint{4.252056in}{1.074622in}}%
\pgfpathlineto{\pgfqpoint{4.252923in}{1.142011in}}%
\pgfpathlineto{\pgfqpoint{4.253790in}{0.993052in}}%
\pgfpathlineto{\pgfqpoint{4.254657in}{1.246241in}}%
\pgfpathlineto{\pgfqpoint{4.255524in}{1.144777in}}%
\pgfpathlineto{\pgfqpoint{4.256392in}{1.378629in}}%
\pgfpathlineto{\pgfqpoint{4.257259in}{1.075720in}}%
\pgfpathlineto{\pgfqpoint{4.258126in}{1.255820in}}%
\pgfpathlineto{\pgfqpoint{4.258993in}{0.771143in}}%
\pgfpathlineto{\pgfqpoint{4.260727in}{1.313633in}}%
\pgfpathlineto{\pgfqpoint{4.261594in}{1.180559in}}%
\pgfpathlineto{\pgfqpoint{4.262462in}{0.826327in}}%
\pgfpathlineto{\pgfqpoint{4.263329in}{1.181598in}}%
\pgfpathlineto{\pgfqpoint{4.265063in}{0.866496in}}%
\pgfpathlineto{\pgfqpoint{4.265930in}{1.330230in}}%
\pgfpathlineto{\pgfqpoint{4.266797in}{0.836854in}}%
\pgfpathlineto{\pgfqpoint{4.267664in}{0.869582in}}%
\pgfpathlineto{\pgfqpoint{4.268531in}{1.668036in}}%
\pgfpathlineto{\pgfqpoint{4.269399in}{1.009201in}}%
\pgfpathlineto{\pgfqpoint{4.270266in}{1.287208in}}%
\pgfpathlineto{\pgfqpoint{4.272000in}{0.880320in}}%
\pgfpathlineto{\pgfqpoint{4.272867in}{1.299340in}}%
\pgfpathlineto{\pgfqpoint{4.273734in}{0.842079in}}%
\pgfpathlineto{\pgfqpoint{4.274601in}{1.085714in}}%
\pgfpathlineto{\pgfqpoint{4.275469in}{0.824987in}}%
\pgfpathlineto{\pgfqpoint{4.277203in}{1.147496in}}%
\pgfpathlineto{\pgfqpoint{4.278070in}{1.146228in}}%
\pgfpathlineto{\pgfqpoint{4.279804in}{0.866593in}}%
\pgfpathlineto{\pgfqpoint{4.280671in}{1.198421in}}%
\pgfpathlineto{\pgfqpoint{4.282406in}{0.816580in}}%
\pgfpathlineto{\pgfqpoint{4.283273in}{1.639382in}}%
\pgfpathlineto{\pgfqpoint{4.285007in}{0.721424in}}%
\pgfpathlineto{\pgfqpoint{4.286741in}{1.327151in}}%
\pgfpathlineto{\pgfqpoint{4.287608in}{1.256213in}}%
\pgfpathlineto{\pgfqpoint{4.288476in}{1.279389in}}%
\pgfpathlineto{\pgfqpoint{4.289343in}{1.051011in}}%
\pgfpathlineto{\pgfqpoint{4.290210in}{1.365676in}}%
\pgfpathlineto{\pgfqpoint{4.291944in}{0.907798in}}%
\pgfpathlineto{\pgfqpoint{4.293678in}{1.165265in}}%
\pgfpathlineto{\pgfqpoint{4.294545in}{1.238658in}}%
\pgfpathlineto{\pgfqpoint{4.295413in}{1.570708in}}%
\pgfpathlineto{\pgfqpoint{4.296280in}{1.128530in}}%
\pgfpathlineto{\pgfqpoint{4.297147in}{1.625162in}}%
\pgfpathlineto{\pgfqpoint{4.298881in}{0.904536in}}%
\pgfpathlineto{\pgfqpoint{4.300615in}{1.364178in}}%
\pgfpathlineto{\pgfqpoint{4.301483in}{1.044604in}}%
\pgfpathlineto{\pgfqpoint{4.302350in}{1.047467in}}%
\pgfpathlineto{\pgfqpoint{4.303217in}{1.514778in}}%
\pgfpathlineto{\pgfqpoint{4.304084in}{0.955184in}}%
\pgfpathlineto{\pgfqpoint{4.304951in}{1.092130in}}%
\pgfpathlineto{\pgfqpoint{4.305818in}{0.817335in}}%
\pgfpathlineto{\pgfqpoint{4.306685in}{0.917721in}}%
\pgfpathlineto{\pgfqpoint{4.307552in}{0.627675in}}%
\pgfpathlineto{\pgfqpoint{4.309287in}{1.546596in}}%
\pgfpathlineto{\pgfqpoint{4.310154in}{1.360440in}}%
\pgfpathlineto{\pgfqpoint{4.311021in}{1.349483in}}%
\pgfpathlineto{\pgfqpoint{4.311888in}{0.843175in}}%
\pgfpathlineto{\pgfqpoint{4.313622in}{1.410301in}}%
\pgfpathlineto{\pgfqpoint{4.315357in}{0.724281in}}%
\pgfpathlineto{\pgfqpoint{4.317091in}{1.196355in}}%
\pgfpathlineto{\pgfqpoint{4.317958in}{0.898717in}}%
\pgfpathlineto{\pgfqpoint{4.319692in}{1.355660in}}%
\pgfpathlineto{\pgfqpoint{4.320559in}{1.329216in}}%
\pgfpathlineto{\pgfqpoint{4.321427in}{0.913207in}}%
\pgfpathlineto{\pgfqpoint{4.322294in}{1.267032in}}%
\pgfpathlineto{\pgfqpoint{4.323161in}{1.140414in}}%
\pgfpathlineto{\pgfqpoint{4.324028in}{1.924142in}}%
\pgfpathlineto{\pgfqpoint{4.325762in}{0.936007in}}%
\pgfpathlineto{\pgfqpoint{4.326629in}{1.035920in}}%
\pgfpathlineto{\pgfqpoint{4.327497in}{0.875727in}}%
\pgfpathlineto{\pgfqpoint{4.328364in}{0.900833in}}%
\pgfpathlineto{\pgfqpoint{4.329231in}{1.030637in}}%
\pgfpathlineto{\pgfqpoint{4.330098in}{0.780176in}}%
\pgfpathlineto{\pgfqpoint{4.330965in}{0.789614in}}%
\pgfpathlineto{\pgfqpoint{4.331832in}{0.925842in}}%
\pgfpathlineto{\pgfqpoint{4.332699in}{1.427169in}}%
\pgfpathlineto{\pgfqpoint{4.333566in}{1.148449in}}%
\pgfpathlineto{\pgfqpoint{4.334434in}{1.183888in}}%
\pgfpathlineto{\pgfqpoint{4.335301in}{1.132867in}}%
\pgfpathlineto{\pgfqpoint{4.336168in}{1.676342in}}%
\pgfpathlineto{\pgfqpoint{4.337902in}{0.854415in}}%
\pgfpathlineto{\pgfqpoint{4.338769in}{0.722140in}}%
\pgfpathlineto{\pgfqpoint{4.339636in}{1.504606in}}%
\pgfpathlineto{\pgfqpoint{4.341371in}{0.789449in}}%
\pgfpathlineto{\pgfqpoint{4.342238in}{1.429697in}}%
\pgfpathlineto{\pgfqpoint{4.343105in}{1.310977in}}%
\pgfpathlineto{\pgfqpoint{4.343972in}{1.192082in}}%
\pgfpathlineto{\pgfqpoint{4.344839in}{1.245198in}}%
\pgfpathlineto{\pgfqpoint{4.346573in}{0.913140in}}%
\pgfpathlineto{\pgfqpoint{4.347441in}{1.144974in}}%
\pgfpathlineto{\pgfqpoint{4.348308in}{0.940000in}}%
\pgfpathlineto{\pgfqpoint{4.349175in}{1.270888in}}%
\pgfpathlineto{\pgfqpoint{4.350042in}{1.149953in}}%
\pgfpathlineto{\pgfqpoint{4.350909in}{1.229349in}}%
\pgfpathlineto{\pgfqpoint{4.352643in}{0.653376in}}%
\pgfpathlineto{\pgfqpoint{4.353510in}{1.106682in}}%
\pgfpathlineto{\pgfqpoint{4.354378in}{0.780502in}}%
\pgfpathlineto{\pgfqpoint{4.356112in}{1.159942in}}%
\pgfpathlineto{\pgfqpoint{4.356979in}{0.714238in}}%
\pgfpathlineto{\pgfqpoint{4.357846in}{1.217039in}}%
\pgfpathlineto{\pgfqpoint{4.358713in}{0.894332in}}%
\pgfpathlineto{\pgfqpoint{4.359580in}{1.276033in}}%
\pgfpathlineto{\pgfqpoint{4.360448in}{1.254244in}}%
\pgfpathlineto{\pgfqpoint{4.362182in}{1.152532in}}%
\pgfpathlineto{\pgfqpoint{4.363049in}{1.615119in}}%
\pgfpathlineto{\pgfqpoint{4.364783in}{0.983403in}}%
\pgfpathlineto{\pgfqpoint{4.365650in}{1.331885in}}%
\pgfpathlineto{\pgfqpoint{4.366517in}{0.695669in}}%
\pgfpathlineto{\pgfqpoint{4.367385in}{1.111629in}}%
\pgfpathlineto{\pgfqpoint{4.368252in}{0.848570in}}%
\pgfpathlineto{\pgfqpoint{4.370853in}{1.801775in}}%
\pgfpathlineto{\pgfqpoint{4.372587in}{0.780651in}}%
\pgfpathlineto{\pgfqpoint{4.374322in}{1.142062in}}%
\pgfpathlineto{\pgfqpoint{4.375189in}{0.762471in}}%
\pgfpathlineto{\pgfqpoint{4.376923in}{1.270166in}}%
\pgfpathlineto{\pgfqpoint{4.377790in}{1.062300in}}%
\pgfpathlineto{\pgfqpoint{4.378657in}{1.243657in}}%
\pgfpathlineto{\pgfqpoint{4.379524in}{0.827747in}}%
\pgfpathlineto{\pgfqpoint{4.380392in}{1.487406in}}%
\pgfpathlineto{\pgfqpoint{4.381259in}{0.849518in}}%
\pgfpathlineto{\pgfqpoint{4.382126in}{0.985679in}}%
\pgfpathlineto{\pgfqpoint{4.382993in}{1.001596in}}%
\pgfpathlineto{\pgfqpoint{4.383860in}{1.239113in}}%
\pgfpathlineto{\pgfqpoint{4.385594in}{0.847567in}}%
\pgfpathlineto{\pgfqpoint{4.387329in}{1.408481in}}%
\pgfpathlineto{\pgfqpoint{4.388196in}{0.817896in}}%
\pgfpathlineto{\pgfqpoint{4.389063in}{1.473390in}}%
\pgfpathlineto{\pgfqpoint{4.389930in}{1.378133in}}%
\pgfpathlineto{\pgfqpoint{4.390797in}{1.426078in}}%
\pgfpathlineto{\pgfqpoint{4.391664in}{1.368722in}}%
\pgfpathlineto{\pgfqpoint{4.395133in}{0.674138in}}%
\pgfpathlineto{\pgfqpoint{4.397734in}{1.388201in}}%
\pgfpathlineto{\pgfqpoint{4.398601in}{0.947257in}}%
\pgfpathlineto{\pgfqpoint{4.399469in}{1.161982in}}%
\pgfpathlineto{\pgfqpoint{4.400336in}{0.974580in}}%
\pgfpathlineto{\pgfqpoint{4.401203in}{1.117026in}}%
\pgfpathlineto{\pgfqpoint{4.402070in}{1.449595in}}%
\pgfpathlineto{\pgfqpoint{4.402937in}{0.819900in}}%
\pgfpathlineto{\pgfqpoint{4.404671in}{1.451114in}}%
\pgfpathlineto{\pgfqpoint{4.405538in}{1.370119in}}%
\pgfpathlineto{\pgfqpoint{4.406406in}{1.410709in}}%
\pgfpathlineto{\pgfqpoint{4.407273in}{0.930342in}}%
\pgfpathlineto{\pgfqpoint{4.408140in}{0.963113in}}%
\pgfpathlineto{\pgfqpoint{4.409007in}{1.900061in}}%
\pgfpathlineto{\pgfqpoint{4.409874in}{1.117271in}}%
\pgfpathlineto{\pgfqpoint{4.410741in}{1.405004in}}%
\pgfpathlineto{\pgfqpoint{4.411608in}{1.118757in}}%
\pgfpathlineto{\pgfqpoint{4.412476in}{1.625116in}}%
\pgfpathlineto{\pgfqpoint{4.413343in}{0.771211in}}%
\pgfpathlineto{\pgfqpoint{4.414210in}{1.223125in}}%
\pgfpathlineto{\pgfqpoint{4.415077in}{0.830159in}}%
\pgfpathlineto{\pgfqpoint{4.415944in}{0.974998in}}%
\pgfpathlineto{\pgfqpoint{4.417678in}{1.432408in}}%
\pgfpathlineto{\pgfqpoint{4.418545in}{1.382411in}}%
\pgfpathlineto{\pgfqpoint{4.419413in}{1.387402in}}%
\pgfpathlineto{\pgfqpoint{4.420280in}{0.733306in}}%
\pgfpathlineto{\pgfqpoint{4.421147in}{1.870627in}}%
\pgfpathlineto{\pgfqpoint{4.422881in}{0.824654in}}%
\pgfpathlineto{\pgfqpoint{4.423748in}{0.903447in}}%
\pgfpathlineto{\pgfqpoint{4.424615in}{0.915058in}}%
\pgfpathlineto{\pgfqpoint{4.425483in}{1.392108in}}%
\pgfpathlineto{\pgfqpoint{4.427217in}{0.789598in}}%
\pgfpathlineto{\pgfqpoint{4.428084in}{0.880859in}}%
\pgfpathlineto{\pgfqpoint{4.428951in}{1.466160in}}%
\pgfpathlineto{\pgfqpoint{4.429818in}{1.429793in}}%
\pgfpathlineto{\pgfqpoint{4.431552in}{0.967004in}}%
\pgfpathlineto{\pgfqpoint{4.432420in}{1.047358in}}%
\pgfpathlineto{\pgfqpoint{4.433287in}{0.705709in}}%
\pgfpathlineto{\pgfqpoint{4.434154in}{1.229383in}}%
\pgfpathlineto{\pgfqpoint{4.435021in}{0.891063in}}%
\pgfpathlineto{\pgfqpoint{4.435888in}{1.555515in}}%
\pgfpathlineto{\pgfqpoint{4.436755in}{0.907967in}}%
\pgfpathlineto{\pgfqpoint{4.438490in}{1.443509in}}%
\pgfpathlineto{\pgfqpoint{4.439357in}{0.850296in}}%
\pgfpathlineto{\pgfqpoint{4.441091in}{1.385002in}}%
\pgfpathlineto{\pgfqpoint{4.441958in}{0.793655in}}%
\pgfpathlineto{\pgfqpoint{4.442825in}{0.838698in}}%
\pgfpathlineto{\pgfqpoint{4.443692in}{1.091194in}}%
\pgfpathlineto{\pgfqpoint{4.445427in}{0.840279in}}%
\pgfpathlineto{\pgfqpoint{4.446294in}{1.428573in}}%
\pgfpathlineto{\pgfqpoint{4.447161in}{0.992540in}}%
\pgfpathlineto{\pgfqpoint{4.448028in}{1.063762in}}%
\pgfpathlineto{\pgfqpoint{4.448895in}{1.063993in}}%
\pgfpathlineto{\pgfqpoint{4.449762in}{0.949948in}}%
\pgfpathlineto{\pgfqpoint{4.450629in}{1.869750in}}%
\pgfpathlineto{\pgfqpoint{4.453231in}{0.722741in}}%
\pgfpathlineto{\pgfqpoint{4.454098in}{1.391146in}}%
\pgfpathlineto{\pgfqpoint{4.454965in}{1.233299in}}%
\pgfpathlineto{\pgfqpoint{4.455832in}{1.206354in}}%
\pgfpathlineto{\pgfqpoint{4.456699in}{1.553127in}}%
\pgfpathlineto{\pgfqpoint{4.458434in}{0.922098in}}%
\pgfpathlineto{\pgfqpoint{4.459301in}{1.596784in}}%
\pgfpathlineto{\pgfqpoint{4.460168in}{1.524516in}}%
\pgfpathlineto{\pgfqpoint{4.461035in}{1.685366in}}%
\pgfpathlineto{\pgfqpoint{4.461902in}{0.625049in}}%
\pgfpathlineto{\pgfqpoint{4.462769in}{1.235176in}}%
\pgfpathlineto{\pgfqpoint{4.463636in}{1.095014in}}%
\pgfpathlineto{\pgfqpoint{4.464503in}{0.816681in}}%
\pgfpathlineto{\pgfqpoint{4.465371in}{1.308794in}}%
\pgfpathlineto{\pgfqpoint{4.468839in}{0.721040in}}%
\pgfpathlineto{\pgfqpoint{4.470573in}{1.084756in}}%
\pgfpathlineto{\pgfqpoint{4.471441in}{1.007558in}}%
\pgfpathlineto{\pgfqpoint{4.472308in}{0.738440in}}%
\pgfpathlineto{\pgfqpoint{4.473175in}{1.308629in}}%
\pgfpathlineto{\pgfqpoint{4.474042in}{0.876176in}}%
\pgfpathlineto{\pgfqpoint{4.474909in}{1.325761in}}%
\pgfpathlineto{\pgfqpoint{4.475776in}{0.784953in}}%
\pgfpathlineto{\pgfqpoint{4.476643in}{0.815848in}}%
\pgfpathlineto{\pgfqpoint{4.477510in}{0.976350in}}%
\pgfpathlineto{\pgfqpoint{4.478378in}{1.775408in}}%
\pgfpathlineto{\pgfqpoint{4.480979in}{0.712360in}}%
\pgfpathlineto{\pgfqpoint{4.481846in}{0.936568in}}%
\pgfpathlineto{\pgfqpoint{4.482713in}{1.572328in}}%
\pgfpathlineto{\pgfqpoint{4.483580in}{1.152892in}}%
\pgfpathlineto{\pgfqpoint{4.485315in}{1.380560in}}%
\pgfpathlineto{\pgfqpoint{4.486182in}{0.862630in}}%
\pgfpathlineto{\pgfqpoint{4.487049in}{1.133048in}}%
\pgfpathlineto{\pgfqpoint{4.488783in}{0.820403in}}%
\pgfpathlineto{\pgfqpoint{4.489650in}{0.826944in}}%
\pgfpathlineto{\pgfqpoint{4.490517in}{1.193436in}}%
\pgfpathlineto{\pgfqpoint{4.491385in}{0.765051in}}%
\pgfpathlineto{\pgfqpoint{4.493986in}{1.363840in}}%
\pgfpathlineto{\pgfqpoint{4.494853in}{1.290868in}}%
\pgfpathlineto{\pgfqpoint{4.495720in}{0.903352in}}%
\pgfpathlineto{\pgfqpoint{4.496587in}{1.275700in}}%
\pgfpathlineto{\pgfqpoint{4.497455in}{0.820482in}}%
\pgfpathlineto{\pgfqpoint{4.499189in}{1.152481in}}%
\pgfpathlineto{\pgfqpoint{4.500056in}{1.016821in}}%
\pgfpathlineto{\pgfqpoint{4.500923in}{1.395660in}}%
\pgfpathlineto{\pgfqpoint{4.501790in}{1.038361in}}%
\pgfpathlineto{\pgfqpoint{4.502657in}{1.064972in}}%
\pgfpathlineto{\pgfqpoint{4.503524in}{0.790101in}}%
\pgfpathlineto{\pgfqpoint{4.504392in}{1.273798in}}%
\pgfpathlineto{\pgfqpoint{4.505259in}{1.209714in}}%
\pgfpathlineto{\pgfqpoint{4.506993in}{0.866044in}}%
\pgfpathlineto{\pgfqpoint{4.507860in}{1.164172in}}%
\pgfpathlineto{\pgfqpoint{4.508727in}{1.129804in}}%
\pgfpathlineto{\pgfqpoint{4.509594in}{1.193034in}}%
\pgfpathlineto{\pgfqpoint{4.510462in}{1.410655in}}%
\pgfpathlineto{\pgfqpoint{4.511329in}{0.619331in}}%
\pgfpathlineto{\pgfqpoint{4.512196in}{1.567666in}}%
\pgfpathlineto{\pgfqpoint{4.513063in}{1.045405in}}%
\pgfpathlineto{\pgfqpoint{4.513930in}{1.541903in}}%
\pgfpathlineto{\pgfqpoint{4.515664in}{1.052951in}}%
\pgfpathlineto{\pgfqpoint{4.516531in}{1.312308in}}%
\pgfpathlineto{\pgfqpoint{4.517399in}{0.679935in}}%
\pgfpathlineto{\pgfqpoint{4.518266in}{0.756720in}}%
\pgfpathlineto{\pgfqpoint{4.520000in}{0.886687in}}%
\pgfpathlineto{\pgfqpoint{4.520867in}{1.738067in}}%
\pgfpathlineto{\pgfqpoint{4.522601in}{0.911433in}}%
\pgfpathlineto{\pgfqpoint{4.523469in}{0.760743in}}%
\pgfpathlineto{\pgfqpoint{4.526070in}{1.265912in}}%
\pgfpathlineto{\pgfqpoint{4.526937in}{0.889880in}}%
\pgfpathlineto{\pgfqpoint{4.527804in}{0.984193in}}%
\pgfpathlineto{\pgfqpoint{4.528671in}{0.700855in}}%
\pgfpathlineto{\pgfqpoint{4.530406in}{1.598280in}}%
\pgfpathlineto{\pgfqpoint{4.531273in}{1.101917in}}%
\pgfpathlineto{\pgfqpoint{4.533007in}{1.496210in}}%
\pgfpathlineto{\pgfqpoint{4.533874in}{1.395205in}}%
\pgfpathlineto{\pgfqpoint{4.534741in}{0.686967in}}%
\pgfpathlineto{\pgfqpoint{4.535608in}{1.285250in}}%
\pgfpathlineto{\pgfqpoint{4.536476in}{1.166163in}}%
\pgfpathlineto{\pgfqpoint{4.537343in}{0.789974in}}%
\pgfpathlineto{\pgfqpoint{4.538210in}{1.129546in}}%
\pgfpathlineto{\pgfqpoint{4.539077in}{1.019295in}}%
\pgfpathlineto{\pgfqpoint{4.539944in}{1.868555in}}%
\pgfpathlineto{\pgfqpoint{4.541678in}{0.759116in}}%
\pgfpathlineto{\pgfqpoint{4.542545in}{1.326484in}}%
\pgfpathlineto{\pgfqpoint{4.543413in}{0.992197in}}%
\pgfpathlineto{\pgfqpoint{4.544280in}{1.289396in}}%
\pgfpathlineto{\pgfqpoint{4.546014in}{0.780156in}}%
\pgfpathlineto{\pgfqpoint{4.546881in}{1.450947in}}%
\pgfpathlineto{\pgfqpoint{4.547748in}{1.021267in}}%
\pgfpathlineto{\pgfqpoint{4.549483in}{1.243809in}}%
\pgfpathlineto{\pgfqpoint{4.551217in}{1.071003in}}%
\pgfpathlineto{\pgfqpoint{4.552084in}{1.315426in}}%
\pgfpathlineto{\pgfqpoint{4.552951in}{0.726340in}}%
\pgfpathlineto{\pgfqpoint{4.553818in}{1.299614in}}%
\pgfpathlineto{\pgfqpoint{4.554685in}{1.156558in}}%
\pgfpathlineto{\pgfqpoint{4.555552in}{1.674629in}}%
\pgfpathlineto{\pgfqpoint{4.556420in}{1.362868in}}%
\pgfpathlineto{\pgfqpoint{4.557287in}{1.503933in}}%
\pgfpathlineto{\pgfqpoint{4.559021in}{0.991666in}}%
\pgfpathlineto{\pgfqpoint{4.559888in}{1.418158in}}%
\pgfpathlineto{\pgfqpoint{4.560755in}{1.416037in}}%
\pgfpathlineto{\pgfqpoint{4.561622in}{0.808915in}}%
\pgfpathlineto{\pgfqpoint{4.562490in}{1.297287in}}%
\pgfpathlineto{\pgfqpoint{4.564224in}{0.860560in}}%
\pgfpathlineto{\pgfqpoint{4.565958in}{1.363252in}}%
\pgfpathlineto{\pgfqpoint{4.566825in}{1.208149in}}%
\pgfpathlineto{\pgfqpoint{4.567692in}{1.255438in}}%
\pgfpathlineto{\pgfqpoint{4.568559in}{0.880071in}}%
\pgfpathlineto{\pgfqpoint{4.569427in}{1.520235in}}%
\pgfpathlineto{\pgfqpoint{4.571161in}{0.788765in}}%
\pgfpathlineto{\pgfqpoint{4.572028in}{0.819960in}}%
\pgfpathlineto{\pgfqpoint{4.572895in}{1.247122in}}%
\pgfpathlineto{\pgfqpoint{4.573762in}{0.942433in}}%
\pgfpathlineto{\pgfqpoint{4.575497in}{1.167358in}}%
\pgfpathlineto{\pgfqpoint{4.576364in}{0.750672in}}%
\pgfpathlineto{\pgfqpoint{4.577231in}{1.357314in}}%
\pgfpathlineto{\pgfqpoint{4.578965in}{0.937713in}}%
\pgfpathlineto{\pgfqpoint{4.579832in}{1.002993in}}%
\pgfpathlineto{\pgfqpoint{4.581566in}{0.663698in}}%
\pgfpathlineto{\pgfqpoint{4.582434in}{1.050007in}}%
\pgfpathlineto{\pgfqpoint{4.583301in}{0.971707in}}%
\pgfpathlineto{\pgfqpoint{4.584168in}{0.981500in}}%
\pgfpathlineto{\pgfqpoint{4.585902in}{0.818352in}}%
\pgfpathlineto{\pgfqpoint{4.588503in}{1.412376in}}%
\pgfpathlineto{\pgfqpoint{4.589371in}{1.014040in}}%
\pgfpathlineto{\pgfqpoint{4.590238in}{1.327989in}}%
\pgfpathlineto{\pgfqpoint{4.591105in}{0.742716in}}%
\pgfpathlineto{\pgfqpoint{4.592839in}{0.994664in}}%
\pgfpathlineto{\pgfqpoint{4.593706in}{1.692305in}}%
\pgfpathlineto{\pgfqpoint{4.596308in}{0.735878in}}%
\pgfpathlineto{\pgfqpoint{4.598042in}{1.165842in}}%
\pgfpathlineto{\pgfqpoint{4.598909in}{1.055505in}}%
\pgfpathlineto{\pgfqpoint{4.600643in}{0.701078in}}%
\pgfpathlineto{\pgfqpoint{4.601510in}{1.477207in}}%
\pgfpathlineto{\pgfqpoint{4.603245in}{1.001852in}}%
\pgfpathlineto{\pgfqpoint{4.604112in}{1.553947in}}%
\pgfpathlineto{\pgfqpoint{4.604979in}{0.785891in}}%
\pgfpathlineto{\pgfqpoint{4.605846in}{0.976741in}}%
\pgfpathlineto{\pgfqpoint{4.607580in}{1.368389in}}%
\pgfpathlineto{\pgfqpoint{4.609315in}{0.982548in}}%
\pgfpathlineto{\pgfqpoint{4.610182in}{1.113214in}}%
\pgfpathlineto{\pgfqpoint{4.611049in}{0.782043in}}%
\pgfpathlineto{\pgfqpoint{4.612783in}{1.147110in}}%
\pgfpathlineto{\pgfqpoint{4.613650in}{1.200194in}}%
\pgfpathlineto{\pgfqpoint{4.615385in}{0.891633in}}%
\pgfpathlineto{\pgfqpoint{4.617986in}{1.337033in}}%
\pgfpathlineto{\pgfqpoint{4.618853in}{1.308218in}}%
\pgfpathlineto{\pgfqpoint{4.619720in}{0.886561in}}%
\pgfpathlineto{\pgfqpoint{4.620587in}{1.081063in}}%
\pgfpathlineto{\pgfqpoint{4.621455in}{1.016729in}}%
\pgfpathlineto{\pgfqpoint{4.624056in}{1.364109in}}%
\pgfpathlineto{\pgfqpoint{4.626657in}{0.926202in}}%
\pgfpathlineto{\pgfqpoint{4.627524in}{1.430610in}}%
\pgfpathlineto{\pgfqpoint{4.628392in}{1.378117in}}%
\pgfpathlineto{\pgfqpoint{4.629259in}{1.092544in}}%
\pgfpathlineto{\pgfqpoint{4.630993in}{1.630334in}}%
\pgfpathlineto{\pgfqpoint{4.632727in}{0.921660in}}%
\pgfpathlineto{\pgfqpoint{4.635329in}{0.659036in}}%
\pgfpathlineto{\pgfqpoint{4.636196in}{1.233097in}}%
\pgfpathlineto{\pgfqpoint{4.637930in}{0.954195in}}%
\pgfpathlineto{\pgfqpoint{4.638797in}{1.257388in}}%
\pgfpathlineto{\pgfqpoint{4.639664in}{0.721870in}}%
\pgfpathlineto{\pgfqpoint{4.640531in}{1.741520in}}%
\pgfpathlineto{\pgfqpoint{4.641399in}{1.658963in}}%
\pgfpathlineto{\pgfqpoint{4.642266in}{1.126722in}}%
\pgfpathlineto{\pgfqpoint{4.643133in}{1.187518in}}%
\pgfpathlineto{\pgfqpoint{4.644000in}{0.829164in}}%
\pgfpathlineto{\pgfqpoint{4.645734in}{1.099283in}}%
\pgfpathlineto{\pgfqpoint{4.647469in}{0.836701in}}%
\pgfpathlineto{\pgfqpoint{4.648336in}{0.878198in}}%
\pgfpathlineto{\pgfqpoint{4.650937in}{1.461637in}}%
\pgfpathlineto{\pgfqpoint{4.651804in}{0.707315in}}%
\pgfpathlineto{\pgfqpoint{4.653538in}{1.092701in}}%
\pgfpathlineto{\pgfqpoint{4.655273in}{1.285183in}}%
\pgfpathlineto{\pgfqpoint{4.656140in}{0.925735in}}%
\pgfpathlineto{\pgfqpoint{4.657007in}{0.953898in}}%
\pgfpathlineto{\pgfqpoint{4.658741in}{1.141982in}}%
\pgfpathlineto{\pgfqpoint{4.660476in}{1.325034in}}%
\pgfpathlineto{\pgfqpoint{4.662210in}{0.865014in}}%
\pgfpathlineto{\pgfqpoint{4.663077in}{1.005116in}}%
\pgfpathlineto{\pgfqpoint{4.663944in}{0.788865in}}%
\pgfpathlineto{\pgfqpoint{4.664811in}{0.831829in}}%
\pgfpathlineto{\pgfqpoint{4.666545in}{1.017033in}}%
\pgfpathlineto{\pgfqpoint{4.667413in}{1.551370in}}%
\pgfpathlineto{\pgfqpoint{4.668280in}{1.102300in}}%
\pgfpathlineto{\pgfqpoint{4.669147in}{1.212905in}}%
\pgfpathlineto{\pgfqpoint{4.670014in}{0.935107in}}%
\pgfpathlineto{\pgfqpoint{4.670881in}{1.539465in}}%
\pgfpathlineto{\pgfqpoint{4.672615in}{1.102309in}}%
\pgfpathlineto{\pgfqpoint{4.673483in}{1.073998in}}%
\pgfpathlineto{\pgfqpoint{4.674350in}{0.973064in}}%
\pgfpathlineto{\pgfqpoint{4.676084in}{1.461573in}}%
\pgfpathlineto{\pgfqpoint{4.677818in}{0.871994in}}%
\pgfpathlineto{\pgfqpoint{4.678685in}{1.262170in}}%
\pgfpathlineto{\pgfqpoint{4.679552in}{0.896361in}}%
\pgfpathlineto{\pgfqpoint{4.680420in}{1.444385in}}%
\pgfpathlineto{\pgfqpoint{4.681287in}{0.659459in}}%
\pgfpathlineto{\pgfqpoint{4.682154in}{1.254536in}}%
\pgfpathlineto{\pgfqpoint{4.683021in}{0.899741in}}%
\pgfpathlineto{\pgfqpoint{4.683888in}{1.449326in}}%
\pgfpathlineto{\pgfqpoint{4.684755in}{0.829116in}}%
\pgfpathlineto{\pgfqpoint{4.686490in}{1.326621in}}%
\pgfpathlineto{\pgfqpoint{4.687357in}{0.828190in}}%
\pgfpathlineto{\pgfqpoint{4.688224in}{0.961717in}}%
\pgfpathlineto{\pgfqpoint{4.690825in}{1.640692in}}%
\pgfpathlineto{\pgfqpoint{4.692559in}{1.037649in}}%
\pgfpathlineto{\pgfqpoint{4.693427in}{1.258187in}}%
\pgfpathlineto{\pgfqpoint{4.694294in}{0.725794in}}%
\pgfpathlineto{\pgfqpoint{4.695161in}{1.070523in}}%
\pgfpathlineto{\pgfqpoint{4.696895in}{0.752554in}}%
\pgfpathlineto{\pgfqpoint{4.698629in}{1.234629in}}%
\pgfpathlineto{\pgfqpoint{4.699497in}{1.056430in}}%
\pgfpathlineto{\pgfqpoint{4.700364in}{1.132435in}}%
\pgfpathlineto{\pgfqpoint{4.701231in}{1.016980in}}%
\pgfpathlineto{\pgfqpoint{4.702098in}{0.684992in}}%
\pgfpathlineto{\pgfqpoint{4.702965in}{1.302848in}}%
\pgfpathlineto{\pgfqpoint{4.703832in}{1.035227in}}%
\pgfpathlineto{\pgfqpoint{4.704699in}{1.061109in}}%
\pgfpathlineto{\pgfqpoint{4.705566in}{1.046287in}}%
\pgfpathlineto{\pgfqpoint{4.706434in}{1.008200in}}%
\pgfpathlineto{\pgfqpoint{4.707301in}{1.514320in}}%
\pgfpathlineto{\pgfqpoint{4.708168in}{1.273346in}}%
\pgfpathlineto{\pgfqpoint{4.709035in}{1.428985in}}%
\pgfpathlineto{\pgfqpoint{4.709902in}{0.832139in}}%
\pgfpathlineto{\pgfqpoint{4.710769in}{1.127152in}}%
\pgfpathlineto{\pgfqpoint{4.711636in}{0.866664in}}%
\pgfpathlineto{\pgfqpoint{4.712503in}{1.293227in}}%
\pgfpathlineto{\pgfqpoint{4.713371in}{1.060887in}}%
\pgfpathlineto{\pgfqpoint{4.714238in}{1.568531in}}%
\pgfpathlineto{\pgfqpoint{4.715105in}{0.744693in}}%
\pgfpathlineto{\pgfqpoint{4.715972in}{0.820314in}}%
\pgfpathlineto{\pgfqpoint{4.717706in}{1.016111in}}%
\pgfpathlineto{\pgfqpoint{4.718573in}{1.057343in}}%
\pgfpathlineto{\pgfqpoint{4.720308in}{0.688330in}}%
\pgfpathlineto{\pgfqpoint{4.721175in}{1.185269in}}%
\pgfpathlineto{\pgfqpoint{4.722042in}{0.949871in}}%
\pgfpathlineto{\pgfqpoint{4.723776in}{1.065433in}}%
\pgfpathlineto{\pgfqpoint{4.724643in}{0.890740in}}%
\pgfpathlineto{\pgfqpoint{4.726378in}{1.272348in}}%
\pgfpathlineto{\pgfqpoint{4.727245in}{0.698052in}}%
\pgfpathlineto{\pgfqpoint{4.728112in}{0.743438in}}%
\pgfpathlineto{\pgfqpoint{4.728979in}{0.920338in}}%
\pgfpathlineto{\pgfqpoint{4.729846in}{1.516347in}}%
\pgfpathlineto{\pgfqpoint{4.730713in}{1.137141in}}%
\pgfpathlineto{\pgfqpoint{4.731580in}{1.205922in}}%
\pgfpathlineto{\pgfqpoint{4.732448in}{0.889659in}}%
\pgfpathlineto{\pgfqpoint{4.733315in}{0.894998in}}%
\pgfpathlineto{\pgfqpoint{4.734182in}{1.113563in}}%
\pgfpathlineto{\pgfqpoint{4.735049in}{0.745704in}}%
\pgfpathlineto{\pgfqpoint{4.735916in}{0.866529in}}%
\pgfpathlineto{\pgfqpoint{4.737650in}{1.467005in}}%
\pgfpathlineto{\pgfqpoint{4.738517in}{0.704320in}}%
\pgfpathlineto{\pgfqpoint{4.740252in}{1.177357in}}%
\pgfpathlineto{\pgfqpoint{4.741119in}{0.751287in}}%
\pgfpathlineto{\pgfqpoint{4.741986in}{0.863647in}}%
\pgfpathlineto{\pgfqpoint{4.743720in}{1.131123in}}%
\pgfpathlineto{\pgfqpoint{4.744587in}{1.115260in}}%
\pgfpathlineto{\pgfqpoint{4.745455in}{1.516985in}}%
\pgfpathlineto{\pgfqpoint{4.746322in}{1.224687in}}%
\pgfpathlineto{\pgfqpoint{4.747189in}{1.753133in}}%
\pgfpathlineto{\pgfqpoint{4.748056in}{1.199812in}}%
\pgfpathlineto{\pgfqpoint{4.749790in}{1.783946in}}%
\pgfpathlineto{\pgfqpoint{4.751524in}{0.910609in}}%
\pgfpathlineto{\pgfqpoint{4.753259in}{1.362232in}}%
\pgfpathlineto{\pgfqpoint{4.754126in}{0.935949in}}%
\pgfpathlineto{\pgfqpoint{4.754993in}{0.946784in}}%
\pgfpathlineto{\pgfqpoint{4.755860in}{0.659467in}}%
\pgfpathlineto{\pgfqpoint{4.757594in}{1.253227in}}%
\pgfpathlineto{\pgfqpoint{4.760196in}{1.037826in}}%
\pgfpathlineto{\pgfqpoint{4.761063in}{1.499480in}}%
\pgfpathlineto{\pgfqpoint{4.762797in}{0.700575in}}%
\pgfpathlineto{\pgfqpoint{4.763664in}{1.074816in}}%
\pgfpathlineto{\pgfqpoint{4.764531in}{0.740753in}}%
\pgfpathlineto{\pgfqpoint{4.765399in}{0.995928in}}%
\pgfpathlineto{\pgfqpoint{4.766266in}{0.654163in}}%
\pgfpathlineto{\pgfqpoint{4.768867in}{1.102716in}}%
\pgfpathlineto{\pgfqpoint{4.769734in}{0.785634in}}%
\pgfpathlineto{\pgfqpoint{4.770601in}{1.403998in}}%
\pgfpathlineto{\pgfqpoint{4.771469in}{1.067205in}}%
\pgfpathlineto{\pgfqpoint{4.773203in}{1.194388in}}%
\pgfpathlineto{\pgfqpoint{4.774070in}{1.097002in}}%
\pgfpathlineto{\pgfqpoint{4.774937in}{1.177885in}}%
\pgfpathlineto{\pgfqpoint{4.775804in}{1.086970in}}%
\pgfpathlineto{\pgfqpoint{4.776671in}{1.657789in}}%
\pgfpathlineto{\pgfqpoint{4.777538in}{1.424825in}}%
\pgfpathlineto{\pgfqpoint{4.778406in}{1.552439in}}%
\pgfpathlineto{\pgfqpoint{4.779273in}{0.977830in}}%
\pgfpathlineto{\pgfqpoint{4.780140in}{1.352170in}}%
\pgfpathlineto{\pgfqpoint{4.781007in}{1.225479in}}%
\pgfpathlineto{\pgfqpoint{4.781874in}{1.293198in}}%
\pgfpathlineto{\pgfqpoint{4.782741in}{0.820306in}}%
\pgfpathlineto{\pgfqpoint{4.784476in}{1.325634in}}%
\pgfpathlineto{\pgfqpoint{4.786210in}{0.956155in}}%
\pgfpathlineto{\pgfqpoint{4.787077in}{1.618472in}}%
\pgfpathlineto{\pgfqpoint{4.788811in}{0.895094in}}%
\pgfpathlineto{\pgfqpoint{4.789678in}{1.001271in}}%
\pgfpathlineto{\pgfqpoint{4.790545in}{0.989732in}}%
\pgfpathlineto{\pgfqpoint{4.791413in}{0.826757in}}%
\pgfpathlineto{\pgfqpoint{4.793147in}{1.639682in}}%
\pgfpathlineto{\pgfqpoint{4.794014in}{1.070710in}}%
\pgfpathlineto{\pgfqpoint{4.794881in}{1.138441in}}%
\pgfpathlineto{\pgfqpoint{4.796615in}{1.453325in}}%
\pgfpathlineto{\pgfqpoint{4.798350in}{0.835354in}}%
\pgfpathlineto{\pgfqpoint{4.799217in}{0.882866in}}%
\pgfpathlineto{\pgfqpoint{4.801818in}{1.215844in}}%
\pgfpathlineto{\pgfqpoint{4.802685in}{1.089105in}}%
\pgfpathlineto{\pgfqpoint{4.803552in}{1.506289in}}%
\pgfpathlineto{\pgfqpoint{4.804420in}{0.828797in}}%
\pgfpathlineto{\pgfqpoint{4.805287in}{1.005823in}}%
\pgfpathlineto{\pgfqpoint{4.806154in}{1.175357in}}%
\pgfpathlineto{\pgfqpoint{4.807888in}{0.950958in}}%
\pgfpathlineto{\pgfqpoint{4.809622in}{1.707428in}}%
\pgfpathlineto{\pgfqpoint{4.811357in}{0.780776in}}%
\pgfpathlineto{\pgfqpoint{4.813958in}{1.363217in}}%
\pgfpathlineto{\pgfqpoint{4.815692in}{0.883309in}}%
\pgfpathlineto{\pgfqpoint{4.817427in}{1.198645in}}%
\pgfpathlineto{\pgfqpoint{4.818294in}{1.192192in}}%
\pgfpathlineto{\pgfqpoint{4.819161in}{1.239523in}}%
\pgfpathlineto{\pgfqpoint{4.820895in}{0.844349in}}%
\pgfpathlineto{\pgfqpoint{4.822629in}{1.923261in}}%
\pgfpathlineto{\pgfqpoint{4.824364in}{0.861378in}}%
\pgfpathlineto{\pgfqpoint{4.825231in}{1.206802in}}%
\pgfpathlineto{\pgfqpoint{4.826098in}{1.184197in}}%
\pgfpathlineto{\pgfqpoint{4.826965in}{0.918648in}}%
\pgfpathlineto{\pgfqpoint{4.827832in}{1.486955in}}%
\pgfpathlineto{\pgfqpoint{4.828699in}{1.078924in}}%
\pgfpathlineto{\pgfqpoint{4.829566in}{1.810961in}}%
\pgfpathlineto{\pgfqpoint{4.830434in}{1.145589in}}%
\pgfpathlineto{\pgfqpoint{4.831301in}{1.168076in}}%
\pgfpathlineto{\pgfqpoint{4.832168in}{0.983658in}}%
\pgfpathlineto{\pgfqpoint{4.833035in}{1.001385in}}%
\pgfpathlineto{\pgfqpoint{4.833902in}{1.206587in}}%
\pgfpathlineto{\pgfqpoint{4.834769in}{1.068925in}}%
\pgfpathlineto{\pgfqpoint{4.835636in}{1.846416in}}%
\pgfpathlineto{\pgfqpoint{4.837371in}{1.091899in}}%
\pgfpathlineto{\pgfqpoint{4.838238in}{1.395370in}}%
\pgfpathlineto{\pgfqpoint{4.839972in}{0.860142in}}%
\pgfpathlineto{\pgfqpoint{4.841706in}{1.891343in}}%
\pgfpathlineto{\pgfqpoint{4.842573in}{0.893926in}}%
\pgfpathlineto{\pgfqpoint{4.843441in}{0.966522in}}%
\pgfpathlineto{\pgfqpoint{4.844308in}{1.562224in}}%
\pgfpathlineto{\pgfqpoint{4.845175in}{0.940130in}}%
\pgfpathlineto{\pgfqpoint{4.846042in}{0.974606in}}%
\pgfpathlineto{\pgfqpoint{4.846909in}{1.075909in}}%
\pgfpathlineto{\pgfqpoint{4.847776in}{1.544364in}}%
\pgfpathlineto{\pgfqpoint{4.849510in}{0.830343in}}%
\pgfpathlineto{\pgfqpoint{4.850378in}{1.172818in}}%
\pgfpathlineto{\pgfqpoint{4.851245in}{0.722596in}}%
\pgfpathlineto{\pgfqpoint{4.852979in}{1.028040in}}%
\pgfpathlineto{\pgfqpoint{4.853846in}{1.690896in}}%
\pgfpathlineto{\pgfqpoint{4.856448in}{1.021469in}}%
\pgfpathlineto{\pgfqpoint{4.857315in}{1.053042in}}%
\pgfpathlineto{\pgfqpoint{4.858182in}{1.204940in}}%
\pgfpathlineto{\pgfqpoint{4.859049in}{1.171077in}}%
\pgfpathlineto{\pgfqpoint{4.859916in}{0.821603in}}%
\pgfpathlineto{\pgfqpoint{4.860783in}{1.426233in}}%
\pgfpathlineto{\pgfqpoint{4.861650in}{1.378032in}}%
\pgfpathlineto{\pgfqpoint{4.863385in}{1.040213in}}%
\pgfpathlineto{\pgfqpoint{4.864252in}{1.058084in}}%
\pgfpathlineto{\pgfqpoint{4.865119in}{1.473520in}}%
\pgfpathlineto{\pgfqpoint{4.865986in}{0.744388in}}%
\pgfpathlineto{\pgfqpoint{4.867720in}{1.695263in}}%
\pgfpathlineto{\pgfqpoint{4.868587in}{1.086348in}}%
\pgfpathlineto{\pgfqpoint{4.869455in}{1.445753in}}%
\pgfpathlineto{\pgfqpoint{4.871189in}{0.867473in}}%
\pgfpathlineto{\pgfqpoint{4.872056in}{1.480141in}}%
\pgfpathlineto{\pgfqpoint{4.872923in}{0.799305in}}%
\pgfpathlineto{\pgfqpoint{4.873790in}{1.492970in}}%
\pgfpathlineto{\pgfqpoint{4.874657in}{1.382530in}}%
\pgfpathlineto{\pgfqpoint{4.875524in}{1.270036in}}%
\pgfpathlineto{\pgfqpoint{4.876392in}{1.780671in}}%
\pgfpathlineto{\pgfqpoint{4.878126in}{1.072580in}}%
\pgfpathlineto{\pgfqpoint{4.878993in}{1.582329in}}%
\pgfpathlineto{\pgfqpoint{4.881594in}{0.892504in}}%
\pgfpathlineto{\pgfqpoint{4.883329in}{1.331989in}}%
\pgfpathlineto{\pgfqpoint{4.884196in}{1.188780in}}%
\pgfpathlineto{\pgfqpoint{4.885063in}{1.212684in}}%
\pgfpathlineto{\pgfqpoint{4.885930in}{1.322296in}}%
\pgfpathlineto{\pgfqpoint{4.886797in}{0.938496in}}%
\pgfpathlineto{\pgfqpoint{4.887664in}{1.121320in}}%
\pgfpathlineto{\pgfqpoint{4.888531in}{1.717745in}}%
\pgfpathlineto{\pgfqpoint{4.890266in}{1.115673in}}%
\pgfpathlineto{\pgfqpoint{4.891133in}{1.643581in}}%
\pgfpathlineto{\pgfqpoint{4.892867in}{1.288763in}}%
\pgfpathlineto{\pgfqpoint{4.895469in}{0.804554in}}%
\pgfpathlineto{\pgfqpoint{4.897203in}{1.141019in}}%
\pgfpathlineto{\pgfqpoint{4.899804in}{0.773722in}}%
\pgfpathlineto{\pgfqpoint{4.900671in}{1.084694in}}%
\pgfpathlineto{\pgfqpoint{4.901538in}{0.902957in}}%
\pgfpathlineto{\pgfqpoint{4.902406in}{1.505683in}}%
\pgfpathlineto{\pgfqpoint{4.904140in}{1.188185in}}%
\pgfpathlineto{\pgfqpoint{4.905007in}{1.280099in}}%
\pgfpathlineto{\pgfqpoint{4.905874in}{2.131636in}}%
\pgfpathlineto{\pgfqpoint{4.907608in}{0.979971in}}%
\pgfpathlineto{\pgfqpoint{4.908476in}{1.212634in}}%
\pgfpathlineto{\pgfqpoint{4.909343in}{1.087580in}}%
\pgfpathlineto{\pgfqpoint{4.910210in}{0.791826in}}%
\pgfpathlineto{\pgfqpoint{4.911077in}{0.836436in}}%
\pgfpathlineto{\pgfqpoint{4.911944in}{0.753459in}}%
\pgfpathlineto{\pgfqpoint{4.914545in}{1.272458in}}%
\pgfpathlineto{\pgfqpoint{4.915413in}{1.385087in}}%
\pgfpathlineto{\pgfqpoint{4.917147in}{0.801129in}}%
\pgfpathlineto{\pgfqpoint{4.918881in}{1.560284in}}%
\pgfpathlineto{\pgfqpoint{4.919748in}{0.671784in}}%
\pgfpathlineto{\pgfqpoint{4.920615in}{0.903895in}}%
\pgfpathlineto{\pgfqpoint{4.921483in}{1.510753in}}%
\pgfpathlineto{\pgfqpoint{4.923217in}{0.848190in}}%
\pgfpathlineto{\pgfqpoint{4.924951in}{1.327709in}}%
\pgfpathlineto{\pgfqpoint{4.925818in}{0.843484in}}%
\pgfpathlineto{\pgfqpoint{4.926685in}{1.578223in}}%
\pgfpathlineto{\pgfqpoint{4.927552in}{0.699384in}}%
\pgfpathlineto{\pgfqpoint{4.928420in}{1.962164in}}%
\pgfpathlineto{\pgfqpoint{4.930154in}{1.061514in}}%
\pgfpathlineto{\pgfqpoint{4.931021in}{1.065604in}}%
\pgfpathlineto{\pgfqpoint{4.932755in}{1.352948in}}%
\pgfpathlineto{\pgfqpoint{4.933622in}{0.658856in}}%
\pgfpathlineto{\pgfqpoint{4.934490in}{1.036073in}}%
\pgfpathlineto{\pgfqpoint{4.935357in}{0.853629in}}%
\pgfpathlineto{\pgfqpoint{4.936224in}{1.271780in}}%
\pgfpathlineto{\pgfqpoint{4.938825in}{0.955114in}}%
\pgfpathlineto{\pgfqpoint{4.939692in}{1.125697in}}%
\pgfpathlineto{\pgfqpoint{4.940559in}{0.934555in}}%
\pgfpathlineto{\pgfqpoint{4.941427in}{1.122442in}}%
\pgfpathlineto{\pgfqpoint{4.942294in}{0.979932in}}%
\pgfpathlineto{\pgfqpoint{4.943161in}{1.259331in}}%
\pgfpathlineto{\pgfqpoint{4.944028in}{1.241293in}}%
\pgfpathlineto{\pgfqpoint{4.944895in}{0.727271in}}%
\pgfpathlineto{\pgfqpoint{4.945762in}{0.767458in}}%
\pgfpathlineto{\pgfqpoint{4.947497in}{1.018493in}}%
\pgfpathlineto{\pgfqpoint{4.948364in}{0.736894in}}%
\pgfpathlineto{\pgfqpoint{4.949231in}{1.337810in}}%
\pgfpathlineto{\pgfqpoint{4.950098in}{1.294351in}}%
\pgfpathlineto{\pgfqpoint{4.950965in}{0.718689in}}%
\pgfpathlineto{\pgfqpoint{4.951832in}{1.155358in}}%
\pgfpathlineto{\pgfqpoint{4.952699in}{1.101495in}}%
\pgfpathlineto{\pgfqpoint{4.953566in}{1.147120in}}%
\pgfpathlineto{\pgfqpoint{4.954434in}{1.096090in}}%
\pgfpathlineto{\pgfqpoint{4.955301in}{0.917481in}}%
\pgfpathlineto{\pgfqpoint{4.956168in}{1.030029in}}%
\pgfpathlineto{\pgfqpoint{4.957035in}{0.809919in}}%
\pgfpathlineto{\pgfqpoint{4.957902in}{0.920466in}}%
\pgfpathlineto{\pgfqpoint{4.958769in}{1.532203in}}%
\pgfpathlineto{\pgfqpoint{4.959636in}{0.790351in}}%
\pgfpathlineto{\pgfqpoint{4.960503in}{0.883408in}}%
\pgfpathlineto{\pgfqpoint{4.961371in}{0.848031in}}%
\pgfpathlineto{\pgfqpoint{4.962238in}{1.266302in}}%
\pgfpathlineto{\pgfqpoint{4.963105in}{1.029386in}}%
\pgfpathlineto{\pgfqpoint{4.963972in}{1.031490in}}%
\pgfpathlineto{\pgfqpoint{4.964839in}{1.795135in}}%
\pgfpathlineto{\pgfqpoint{4.965706in}{0.791106in}}%
\pgfpathlineto{\pgfqpoint{4.966573in}{0.905194in}}%
\pgfpathlineto{\pgfqpoint{4.967441in}{1.063382in}}%
\pgfpathlineto{\pgfqpoint{4.968308in}{0.896792in}}%
\pgfpathlineto{\pgfqpoint{4.969175in}{1.030631in}}%
\pgfpathlineto{\pgfqpoint{4.970042in}{0.900232in}}%
\pgfpathlineto{\pgfqpoint{4.970909in}{1.542106in}}%
\pgfpathlineto{\pgfqpoint{4.972643in}{1.025579in}}%
\pgfpathlineto{\pgfqpoint{4.973510in}{0.879147in}}%
\pgfpathlineto{\pgfqpoint{4.974378in}{1.021245in}}%
\pgfpathlineto{\pgfqpoint{4.975245in}{0.717962in}}%
\pgfpathlineto{\pgfqpoint{4.976112in}{1.310336in}}%
\pgfpathlineto{\pgfqpoint{4.976979in}{0.874832in}}%
\pgfpathlineto{\pgfqpoint{4.977846in}{1.225707in}}%
\pgfpathlineto{\pgfqpoint{4.978713in}{1.059308in}}%
\pgfpathlineto{\pgfqpoint{4.979580in}{1.132871in}}%
\pgfpathlineto{\pgfqpoint{4.981315in}{0.898963in}}%
\pgfpathlineto{\pgfqpoint{4.982182in}{0.932197in}}%
\pgfpathlineto{\pgfqpoint{4.983049in}{1.494469in}}%
\pgfpathlineto{\pgfqpoint{4.986517in}{0.765850in}}%
\pgfpathlineto{\pgfqpoint{4.988252in}{1.203667in}}%
\pgfpathlineto{\pgfqpoint{4.989119in}{0.983748in}}%
\pgfpathlineto{\pgfqpoint{4.989986in}{1.233050in}}%
\pgfpathlineto{\pgfqpoint{4.990853in}{0.877396in}}%
\pgfpathlineto{\pgfqpoint{4.991720in}{1.622074in}}%
\pgfpathlineto{\pgfqpoint{4.992587in}{1.015126in}}%
\pgfpathlineto{\pgfqpoint{4.994322in}{1.591137in}}%
\pgfpathlineto{\pgfqpoint{4.995189in}{1.017457in}}%
\pgfpathlineto{\pgfqpoint{4.996056in}{1.271852in}}%
\pgfpathlineto{\pgfqpoint{4.996923in}{0.942667in}}%
\pgfpathlineto{\pgfqpoint{4.998657in}{1.177789in}}%
\pgfpathlineto{\pgfqpoint{4.999524in}{0.994580in}}%
\pgfpathlineto{\pgfqpoint{5.000392in}{1.045300in}}%
\pgfpathlineto{\pgfqpoint{5.001259in}{0.972540in}}%
\pgfpathlineto{\pgfqpoint{5.002126in}{1.191481in}}%
\pgfpathlineto{\pgfqpoint{5.002993in}{0.887306in}}%
\pgfpathlineto{\pgfqpoint{5.003860in}{1.191402in}}%
\pgfpathlineto{\pgfqpoint{5.005594in}{0.679374in}}%
\pgfpathlineto{\pgfqpoint{5.006462in}{1.076419in}}%
\pgfpathlineto{\pgfqpoint{5.007329in}{0.987057in}}%
\pgfpathlineto{\pgfqpoint{5.008196in}{0.706880in}}%
\pgfpathlineto{\pgfqpoint{5.010797in}{1.229094in}}%
\pgfpathlineto{\pgfqpoint{5.011664in}{0.917534in}}%
\pgfpathlineto{\pgfqpoint{5.012531in}{1.204606in}}%
\pgfpathlineto{\pgfqpoint{5.015133in}{0.860449in}}%
\pgfpathlineto{\pgfqpoint{5.016000in}{1.260868in}}%
\pgfpathlineto{\pgfqpoint{5.017734in}{1.034015in}}%
\pgfpathlineto{\pgfqpoint{5.018601in}{1.063041in}}%
\pgfpathlineto{\pgfqpoint{5.019469in}{1.167624in}}%
\pgfpathlineto{\pgfqpoint{5.020336in}{0.961267in}}%
\pgfpathlineto{\pgfqpoint{5.021203in}{1.279800in}}%
\pgfpathlineto{\pgfqpoint{5.022070in}{0.759540in}}%
\pgfpathlineto{\pgfqpoint{5.022937in}{1.156440in}}%
\pgfpathlineto{\pgfqpoint{5.024671in}{0.842141in}}%
\pgfpathlineto{\pgfqpoint{5.025538in}{1.240103in}}%
\pgfpathlineto{\pgfqpoint{5.026406in}{0.847123in}}%
\pgfpathlineto{\pgfqpoint{5.027273in}{1.434356in}}%
\pgfpathlineto{\pgfqpoint{5.028140in}{0.740866in}}%
\pgfpathlineto{\pgfqpoint{5.029007in}{0.814388in}}%
\pgfpathlineto{\pgfqpoint{5.029874in}{1.246216in}}%
\pgfpathlineto{\pgfqpoint{5.030741in}{0.825622in}}%
\pgfpathlineto{\pgfqpoint{5.031608in}{1.086323in}}%
\pgfpathlineto{\pgfqpoint{5.032476in}{1.013061in}}%
\pgfpathlineto{\pgfqpoint{5.033343in}{1.043675in}}%
\pgfpathlineto{\pgfqpoint{5.034210in}{0.772948in}}%
\pgfpathlineto{\pgfqpoint{5.036811in}{1.606481in}}%
\pgfpathlineto{\pgfqpoint{5.038545in}{0.867263in}}%
\pgfpathlineto{\pgfqpoint{5.039413in}{1.359432in}}%
\pgfpathlineto{\pgfqpoint{5.041147in}{0.908443in}}%
\pgfpathlineto{\pgfqpoint{5.042881in}{1.224549in}}%
\pgfpathlineto{\pgfqpoint{5.044615in}{0.731549in}}%
\pgfpathlineto{\pgfqpoint{5.046350in}{1.142896in}}%
\pgfpathlineto{\pgfqpoint{5.047217in}{1.273185in}}%
\pgfpathlineto{\pgfqpoint{5.048951in}{0.826482in}}%
\pgfpathlineto{\pgfqpoint{5.049818in}{0.967719in}}%
\pgfpathlineto{\pgfqpoint{5.050685in}{0.764133in}}%
\pgfpathlineto{\pgfqpoint{5.052420in}{1.171484in}}%
\pgfpathlineto{\pgfqpoint{5.054154in}{0.800806in}}%
\pgfpathlineto{\pgfqpoint{5.055021in}{1.130714in}}%
\pgfpathlineto{\pgfqpoint{5.055888in}{1.071220in}}%
\pgfpathlineto{\pgfqpoint{5.056755in}{1.020961in}}%
\pgfpathlineto{\pgfqpoint{5.057622in}{1.412808in}}%
\pgfpathlineto{\pgfqpoint{5.058490in}{0.751235in}}%
\pgfpathlineto{\pgfqpoint{5.059357in}{1.117506in}}%
\pgfpathlineto{\pgfqpoint{5.060224in}{1.030839in}}%
\pgfpathlineto{\pgfqpoint{5.061091in}{0.915393in}}%
\pgfpathlineto{\pgfqpoint{5.061958in}{1.801383in}}%
\pgfpathlineto{\pgfqpoint{5.062825in}{1.261281in}}%
\pgfpathlineto{\pgfqpoint{5.063692in}{1.492792in}}%
\pgfpathlineto{\pgfqpoint{5.064559in}{1.176339in}}%
\pgfpathlineto{\pgfqpoint{5.065427in}{1.219982in}}%
\pgfpathlineto{\pgfqpoint{5.066294in}{1.162760in}}%
\pgfpathlineto{\pgfqpoint{5.067161in}{1.547362in}}%
\pgfpathlineto{\pgfqpoint{5.068028in}{0.725796in}}%
\pgfpathlineto{\pgfqpoint{5.068895in}{1.249600in}}%
\pgfpathlineto{\pgfqpoint{5.070629in}{0.745405in}}%
\pgfpathlineto{\pgfqpoint{5.072364in}{1.302077in}}%
\pgfpathlineto{\pgfqpoint{5.073231in}{0.990123in}}%
\pgfpathlineto{\pgfqpoint{5.074098in}{1.244498in}}%
\pgfpathlineto{\pgfqpoint{5.074965in}{0.884803in}}%
\pgfpathlineto{\pgfqpoint{5.076699in}{1.507284in}}%
\pgfpathlineto{\pgfqpoint{5.078434in}{0.676987in}}%
\pgfpathlineto{\pgfqpoint{5.079301in}{0.898298in}}%
\pgfpathlineto{\pgfqpoint{5.080168in}{0.831781in}}%
\pgfpathlineto{\pgfqpoint{5.081035in}{1.306705in}}%
\pgfpathlineto{\pgfqpoint{5.081902in}{0.948908in}}%
\pgfpathlineto{\pgfqpoint{5.082769in}{1.150082in}}%
\pgfpathlineto{\pgfqpoint{5.083636in}{1.138746in}}%
\pgfpathlineto{\pgfqpoint{5.085371in}{0.927404in}}%
\pgfpathlineto{\pgfqpoint{5.086238in}{1.003167in}}%
\pgfpathlineto{\pgfqpoint{5.087972in}{0.866339in}}%
\pgfpathlineto{\pgfqpoint{5.088839in}{1.161638in}}%
\pgfpathlineto{\pgfqpoint{5.090573in}{0.873455in}}%
\pgfpathlineto{\pgfqpoint{5.093175in}{1.462488in}}%
\pgfpathlineto{\pgfqpoint{5.094909in}{0.688797in}}%
\pgfpathlineto{\pgfqpoint{5.095776in}{0.794069in}}%
\pgfpathlineto{\pgfqpoint{5.097510in}{1.481975in}}%
\pgfpathlineto{\pgfqpoint{5.098378in}{1.467844in}}%
\pgfpathlineto{\pgfqpoint{5.100112in}{1.036135in}}%
\pgfpathlineto{\pgfqpoint{5.100979in}{1.274087in}}%
\pgfpathlineto{\pgfqpoint{5.101846in}{1.079582in}}%
\pgfpathlineto{\pgfqpoint{5.102713in}{1.277561in}}%
\pgfpathlineto{\pgfqpoint{5.103580in}{0.639977in}}%
\pgfpathlineto{\pgfqpoint{5.104448in}{1.303886in}}%
\pgfpathlineto{\pgfqpoint{5.106182in}{0.735398in}}%
\pgfpathlineto{\pgfqpoint{5.107049in}{1.449270in}}%
\pgfpathlineto{\pgfqpoint{5.108783in}{0.912982in}}%
\pgfpathlineto{\pgfqpoint{5.109650in}{0.981567in}}%
\pgfpathlineto{\pgfqpoint{5.110517in}{1.450195in}}%
\pgfpathlineto{\pgfqpoint{5.111385in}{1.191682in}}%
\pgfpathlineto{\pgfqpoint{5.112252in}{1.517820in}}%
\pgfpathlineto{\pgfqpoint{5.113119in}{0.782962in}}%
\pgfpathlineto{\pgfqpoint{5.113986in}{1.275535in}}%
\pgfpathlineto{\pgfqpoint{5.114853in}{1.146286in}}%
\pgfpathlineto{\pgfqpoint{5.115720in}{1.230920in}}%
\pgfpathlineto{\pgfqpoint{5.117455in}{0.949116in}}%
\pgfpathlineto{\pgfqpoint{5.119189in}{0.781786in}}%
\pgfpathlineto{\pgfqpoint{5.120923in}{1.312619in}}%
\pgfpathlineto{\pgfqpoint{5.121790in}{1.293242in}}%
\pgfpathlineto{\pgfqpoint{5.122657in}{0.631851in}}%
\pgfpathlineto{\pgfqpoint{5.123524in}{1.218445in}}%
\pgfpathlineto{\pgfqpoint{5.124392in}{0.900854in}}%
\pgfpathlineto{\pgfqpoint{5.125259in}{1.155165in}}%
\pgfpathlineto{\pgfqpoint{5.126126in}{0.936318in}}%
\pgfpathlineto{\pgfqpoint{5.126993in}{1.474904in}}%
\pgfpathlineto{\pgfqpoint{5.127860in}{0.850646in}}%
\pgfpathlineto{\pgfqpoint{5.128727in}{0.921529in}}%
\pgfpathlineto{\pgfqpoint{5.129594in}{0.935446in}}%
\pgfpathlineto{\pgfqpoint{5.132196in}{1.342002in}}%
\pgfpathlineto{\pgfqpoint{5.133930in}{0.908879in}}%
\pgfpathlineto{\pgfqpoint{5.134797in}{1.313243in}}%
\pgfpathlineto{\pgfqpoint{5.135664in}{1.007085in}}%
\pgfpathlineto{\pgfqpoint{5.137399in}{1.199034in}}%
\pgfpathlineto{\pgfqpoint{5.138266in}{1.245087in}}%
\pgfpathlineto{\pgfqpoint{5.139133in}{1.432812in}}%
\pgfpathlineto{\pgfqpoint{5.140000in}{1.089195in}}%
\pgfpathlineto{\pgfqpoint{5.140867in}{1.232183in}}%
\pgfpathlineto{\pgfqpoint{5.141734in}{0.868826in}}%
\pgfpathlineto{\pgfqpoint{5.142601in}{1.032383in}}%
\pgfpathlineto{\pgfqpoint{5.143469in}{0.839035in}}%
\pgfpathlineto{\pgfqpoint{5.144336in}{1.262095in}}%
\pgfpathlineto{\pgfqpoint{5.145203in}{0.915950in}}%
\pgfpathlineto{\pgfqpoint{5.146070in}{1.230156in}}%
\pgfpathlineto{\pgfqpoint{5.146937in}{0.725397in}}%
\pgfpathlineto{\pgfqpoint{5.147804in}{1.024716in}}%
\pgfpathlineto{\pgfqpoint{5.148671in}{0.748300in}}%
\pgfpathlineto{\pgfqpoint{5.149538in}{0.825949in}}%
\pgfpathlineto{\pgfqpoint{5.150406in}{1.429883in}}%
\pgfpathlineto{\pgfqpoint{5.151273in}{0.918500in}}%
\pgfpathlineto{\pgfqpoint{5.152140in}{1.213562in}}%
\pgfpathlineto{\pgfqpoint{5.154741in}{0.850572in}}%
\pgfpathlineto{\pgfqpoint{5.156476in}{1.139833in}}%
\pgfpathlineto{\pgfqpoint{5.157343in}{0.841224in}}%
\pgfpathlineto{\pgfqpoint{5.159077in}{1.308504in}}%
\pgfpathlineto{\pgfqpoint{5.160811in}{0.788434in}}%
\pgfpathlineto{\pgfqpoint{5.161678in}{1.156077in}}%
\pgfpathlineto{\pgfqpoint{5.162545in}{0.974073in}}%
\pgfpathlineto{\pgfqpoint{5.163413in}{1.513143in}}%
\pgfpathlineto{\pgfqpoint{5.164280in}{1.394307in}}%
\pgfpathlineto{\pgfqpoint{5.165147in}{0.711346in}}%
\pgfpathlineto{\pgfqpoint{5.166014in}{0.810769in}}%
\pgfpathlineto{\pgfqpoint{5.166881in}{1.485836in}}%
\pgfpathlineto{\pgfqpoint{5.167748in}{1.471170in}}%
\pgfpathlineto{\pgfqpoint{5.168615in}{1.505635in}}%
\pgfpathlineto{\pgfqpoint{5.169483in}{1.223275in}}%
\pgfpathlineto{\pgfqpoint{5.172084in}{1.398180in}}%
\pgfpathlineto{\pgfqpoint{5.174685in}{0.877438in}}%
\pgfpathlineto{\pgfqpoint{5.175552in}{1.368991in}}%
\pgfpathlineto{\pgfqpoint{5.177287in}{0.772530in}}%
\pgfpathlineto{\pgfqpoint{5.178154in}{1.989043in}}%
\pgfpathlineto{\pgfqpoint{5.179021in}{0.872740in}}%
\pgfpathlineto{\pgfqpoint{5.179888in}{1.186267in}}%
\pgfpathlineto{\pgfqpoint{5.180755in}{1.398074in}}%
\pgfpathlineto{\pgfqpoint{5.181622in}{2.025812in}}%
\pgfpathlineto{\pgfqpoint{5.183357in}{0.689373in}}%
\pgfpathlineto{\pgfqpoint{5.184224in}{1.446017in}}%
\pgfpathlineto{\pgfqpoint{5.185958in}{0.734647in}}%
\pgfpathlineto{\pgfqpoint{5.186825in}{1.279117in}}%
\pgfpathlineto{\pgfqpoint{5.187692in}{0.934602in}}%
\pgfpathlineto{\pgfqpoint{5.188559in}{1.545660in}}%
\pgfpathlineto{\pgfqpoint{5.190294in}{0.864973in}}%
\pgfpathlineto{\pgfqpoint{5.191161in}{1.145433in}}%
\pgfpathlineto{\pgfqpoint{5.192895in}{0.842003in}}%
\pgfpathlineto{\pgfqpoint{5.193762in}{1.658815in}}%
\pgfpathlineto{\pgfqpoint{5.194629in}{0.961529in}}%
\pgfpathlineto{\pgfqpoint{5.195497in}{1.516803in}}%
\pgfpathlineto{\pgfqpoint{5.197231in}{1.009376in}}%
\pgfpathlineto{\pgfqpoint{5.198098in}{1.383373in}}%
\pgfpathlineto{\pgfqpoint{5.198965in}{1.376778in}}%
\pgfpathlineto{\pgfqpoint{5.199832in}{0.773627in}}%
\pgfpathlineto{\pgfqpoint{5.201566in}{1.059072in}}%
\pgfpathlineto{\pgfqpoint{5.202434in}{0.748952in}}%
\pgfpathlineto{\pgfqpoint{5.203301in}{1.092847in}}%
\pgfpathlineto{\pgfqpoint{5.204168in}{0.699515in}}%
\pgfpathlineto{\pgfqpoint{5.205035in}{1.162060in}}%
\pgfpathlineto{\pgfqpoint{5.206769in}{0.832061in}}%
\pgfpathlineto{\pgfqpoint{5.207636in}{1.158655in}}%
\pgfpathlineto{\pgfqpoint{5.208503in}{0.772850in}}%
\pgfpathlineto{\pgfqpoint{5.209371in}{1.005561in}}%
\pgfpathlineto{\pgfqpoint{5.210238in}{0.706683in}}%
\pgfpathlineto{\pgfqpoint{5.211105in}{1.146818in}}%
\pgfpathlineto{\pgfqpoint{5.211972in}{0.999970in}}%
\pgfpathlineto{\pgfqpoint{5.213706in}{1.458704in}}%
\pgfpathlineto{\pgfqpoint{5.216308in}{0.617704in}}%
\pgfpathlineto{\pgfqpoint{5.217175in}{1.264243in}}%
\pgfpathlineto{\pgfqpoint{5.218042in}{1.178804in}}%
\pgfpathlineto{\pgfqpoint{5.218909in}{0.862809in}}%
\pgfpathlineto{\pgfqpoint{5.219776in}{0.934764in}}%
\pgfpathlineto{\pgfqpoint{5.220643in}{0.730740in}}%
\pgfpathlineto{\pgfqpoint{5.221510in}{0.978606in}}%
\pgfpathlineto{\pgfqpoint{5.222378in}{0.759701in}}%
\pgfpathlineto{\pgfqpoint{5.224112in}{1.575326in}}%
\pgfpathlineto{\pgfqpoint{5.224979in}{0.955793in}}%
\pgfpathlineto{\pgfqpoint{5.226713in}{1.461074in}}%
\pgfpathlineto{\pgfqpoint{5.228448in}{0.971002in}}%
\pgfpathlineto{\pgfqpoint{5.229315in}{1.126909in}}%
\pgfpathlineto{\pgfqpoint{5.230182in}{1.901663in}}%
\pgfpathlineto{\pgfqpoint{5.231049in}{1.075880in}}%
\pgfpathlineto{\pgfqpoint{5.231916in}{1.129579in}}%
\pgfpathlineto{\pgfqpoint{5.232783in}{1.258484in}}%
\pgfpathlineto{\pgfqpoint{5.233650in}{1.868531in}}%
\pgfpathlineto{\pgfqpoint{5.234517in}{0.772661in}}%
\pgfpathlineto{\pgfqpoint{5.235385in}{1.032334in}}%
\pgfpathlineto{\pgfqpoint{5.236252in}{1.034210in}}%
\pgfpathlineto{\pgfqpoint{5.237119in}{1.489716in}}%
\pgfpathlineto{\pgfqpoint{5.237986in}{1.157852in}}%
\pgfpathlineto{\pgfqpoint{5.238853in}{1.494048in}}%
\pgfpathlineto{\pgfqpoint{5.239720in}{1.446995in}}%
\pgfpathlineto{\pgfqpoint{5.240587in}{0.670119in}}%
\pgfpathlineto{\pgfqpoint{5.241455in}{0.977383in}}%
\pgfpathlineto{\pgfqpoint{5.242322in}{0.955702in}}%
\pgfpathlineto{\pgfqpoint{5.244056in}{0.663559in}}%
\pgfpathlineto{\pgfqpoint{5.244923in}{1.327330in}}%
\pgfpathlineto{\pgfqpoint{5.245790in}{1.292968in}}%
\pgfpathlineto{\pgfqpoint{5.246657in}{0.845476in}}%
\pgfpathlineto{\pgfqpoint{5.247524in}{1.278759in}}%
\pgfpathlineto{\pgfqpoint{5.248392in}{0.942472in}}%
\pgfpathlineto{\pgfqpoint{5.250993in}{1.479446in}}%
\pgfpathlineto{\pgfqpoint{5.252727in}{1.280610in}}%
\pgfpathlineto{\pgfqpoint{5.253594in}{1.399817in}}%
\pgfpathlineto{\pgfqpoint{5.256196in}{0.881657in}}%
\pgfpathlineto{\pgfqpoint{5.257930in}{1.372379in}}%
\pgfpathlineto{\pgfqpoint{5.258797in}{1.164951in}}%
\pgfpathlineto{\pgfqpoint{5.260531in}{1.283618in}}%
\pgfpathlineto{\pgfqpoint{5.262266in}{0.689413in}}%
\pgfpathlineto{\pgfqpoint{5.263133in}{1.383864in}}%
\pgfpathlineto{\pgfqpoint{5.264867in}{1.108377in}}%
\pgfpathlineto{\pgfqpoint{5.266601in}{1.482349in}}%
\pgfpathlineto{\pgfqpoint{5.268336in}{0.980545in}}%
\pgfpathlineto{\pgfqpoint{5.269203in}{0.994212in}}%
\pgfpathlineto{\pgfqpoint{5.270070in}{1.126688in}}%
\pgfpathlineto{\pgfqpoint{5.270937in}{0.793580in}}%
\pgfpathlineto{\pgfqpoint{5.272671in}{1.241612in}}%
\pgfpathlineto{\pgfqpoint{5.273538in}{0.751434in}}%
\pgfpathlineto{\pgfqpoint{5.276140in}{1.389041in}}%
\pgfpathlineto{\pgfqpoint{5.277007in}{0.845609in}}%
\pgfpathlineto{\pgfqpoint{5.277874in}{0.867892in}}%
\pgfpathlineto{\pgfqpoint{5.279608in}{1.318267in}}%
\pgfpathlineto{\pgfqpoint{5.280476in}{0.788379in}}%
\pgfpathlineto{\pgfqpoint{5.281343in}{1.111037in}}%
\pgfpathlineto{\pgfqpoint{5.282210in}{1.104567in}}%
\pgfpathlineto{\pgfqpoint{5.283077in}{0.827428in}}%
\pgfpathlineto{\pgfqpoint{5.284811in}{1.296767in}}%
\pgfpathlineto{\pgfqpoint{5.285678in}{1.131649in}}%
\pgfpathlineto{\pgfqpoint{5.286545in}{1.428233in}}%
\pgfpathlineto{\pgfqpoint{5.288280in}{0.953195in}}%
\pgfpathlineto{\pgfqpoint{5.289147in}{1.412248in}}%
\pgfpathlineto{\pgfqpoint{5.290014in}{1.390047in}}%
\pgfpathlineto{\pgfqpoint{5.290881in}{0.746148in}}%
\pgfpathlineto{\pgfqpoint{5.292615in}{1.322432in}}%
\pgfpathlineto{\pgfqpoint{5.293483in}{1.313185in}}%
\pgfpathlineto{\pgfqpoint{5.294350in}{1.247865in}}%
\pgfpathlineto{\pgfqpoint{5.296084in}{1.348506in}}%
\pgfpathlineto{\pgfqpoint{5.296951in}{0.760328in}}%
\pgfpathlineto{\pgfqpoint{5.297818in}{0.923119in}}%
\pgfpathlineto{\pgfqpoint{5.298685in}{1.106662in}}%
\pgfpathlineto{\pgfqpoint{5.299552in}{0.923400in}}%
\pgfpathlineto{\pgfqpoint{5.301287in}{1.371259in}}%
\pgfpathlineto{\pgfqpoint{5.302154in}{0.859794in}}%
\pgfpathlineto{\pgfqpoint{5.303021in}{1.507940in}}%
\pgfpathlineto{\pgfqpoint{5.303888in}{0.749874in}}%
\pgfpathlineto{\pgfqpoint{5.304755in}{1.061847in}}%
\pgfpathlineto{\pgfqpoint{5.305622in}{0.921341in}}%
\pgfpathlineto{\pgfqpoint{5.306490in}{1.230236in}}%
\pgfpathlineto{\pgfqpoint{5.307357in}{0.841597in}}%
\pgfpathlineto{\pgfqpoint{5.309091in}{1.283095in}}%
\pgfpathlineto{\pgfqpoint{5.309958in}{0.996544in}}%
\pgfpathlineto{\pgfqpoint{5.310825in}{1.051931in}}%
\pgfpathlineto{\pgfqpoint{5.311692in}{0.828324in}}%
\pgfpathlineto{\pgfqpoint{5.312559in}{1.034061in}}%
\pgfpathlineto{\pgfqpoint{5.313427in}{0.893190in}}%
\pgfpathlineto{\pgfqpoint{5.314294in}{1.203627in}}%
\pgfpathlineto{\pgfqpoint{5.315161in}{0.948205in}}%
\pgfpathlineto{\pgfqpoint{5.316028in}{1.191087in}}%
\pgfpathlineto{\pgfqpoint{5.316895in}{1.116197in}}%
\pgfpathlineto{\pgfqpoint{5.318629in}{0.799478in}}%
\pgfpathlineto{\pgfqpoint{5.319497in}{1.136786in}}%
\pgfpathlineto{\pgfqpoint{5.320364in}{1.006449in}}%
\pgfpathlineto{\pgfqpoint{5.321231in}{1.073932in}}%
\pgfpathlineto{\pgfqpoint{5.322965in}{0.817321in}}%
\pgfpathlineto{\pgfqpoint{5.324699in}{1.190650in}}%
\pgfpathlineto{\pgfqpoint{5.325566in}{0.811179in}}%
\pgfpathlineto{\pgfqpoint{5.326434in}{1.153557in}}%
\pgfpathlineto{\pgfqpoint{5.327301in}{0.962520in}}%
\pgfpathlineto{\pgfqpoint{5.328168in}{1.367258in}}%
\pgfpathlineto{\pgfqpoint{5.329035in}{0.870829in}}%
\pgfpathlineto{\pgfqpoint{5.329902in}{1.060171in}}%
\pgfpathlineto{\pgfqpoint{5.330769in}{0.966735in}}%
\pgfpathlineto{\pgfqpoint{5.331636in}{1.370761in}}%
\pgfpathlineto{\pgfqpoint{5.333371in}{0.837604in}}%
\pgfpathlineto{\pgfqpoint{5.334238in}{1.158213in}}%
\pgfpathlineto{\pgfqpoint{5.335105in}{0.813477in}}%
\pgfpathlineto{\pgfqpoint{5.335972in}{1.586156in}}%
\pgfpathlineto{\pgfqpoint{5.337706in}{0.766429in}}%
\pgfpathlineto{\pgfqpoint{5.338573in}{1.191803in}}%
\pgfpathlineto{\pgfqpoint{5.340308in}{0.858848in}}%
\pgfpathlineto{\pgfqpoint{5.342042in}{1.060272in}}%
\pgfpathlineto{\pgfqpoint{5.342909in}{0.932485in}}%
\pgfpathlineto{\pgfqpoint{5.343776in}{1.092912in}}%
\pgfpathlineto{\pgfqpoint{5.344643in}{1.086009in}}%
\pgfpathlineto{\pgfqpoint{5.345510in}{0.994316in}}%
\pgfpathlineto{\pgfqpoint{5.347245in}{1.083616in}}%
\pgfpathlineto{\pgfqpoint{5.348112in}{1.223523in}}%
\pgfpathlineto{\pgfqpoint{5.349846in}{0.749268in}}%
\pgfpathlineto{\pgfqpoint{5.351580in}{1.185498in}}%
\pgfpathlineto{\pgfqpoint{5.352448in}{1.222123in}}%
\pgfpathlineto{\pgfqpoint{5.353315in}{1.006849in}}%
\pgfpathlineto{\pgfqpoint{5.355049in}{1.402778in}}%
\pgfpathlineto{\pgfqpoint{5.355916in}{0.899496in}}%
\pgfpathlineto{\pgfqpoint{5.356783in}{1.113629in}}%
\pgfpathlineto{\pgfqpoint{5.357650in}{0.756418in}}%
\pgfpathlineto{\pgfqpoint{5.358517in}{0.957370in}}%
\pgfpathlineto{\pgfqpoint{5.359385in}{0.848823in}}%
\pgfpathlineto{\pgfqpoint{5.360252in}{1.354002in}}%
\pgfpathlineto{\pgfqpoint{5.361119in}{0.680414in}}%
\pgfpathlineto{\pgfqpoint{5.362853in}{1.144890in}}%
\pgfpathlineto{\pgfqpoint{5.363720in}{1.114912in}}%
\pgfpathlineto{\pgfqpoint{5.364587in}{0.823882in}}%
\pgfpathlineto{\pgfqpoint{5.365455in}{1.015709in}}%
\pgfpathlineto{\pgfqpoint{5.366322in}{0.634088in}}%
\pgfpathlineto{\pgfqpoint{5.367189in}{1.162247in}}%
\pgfpathlineto{\pgfqpoint{5.368056in}{0.852797in}}%
\pgfpathlineto{\pgfqpoint{5.368923in}{1.051358in}}%
\pgfpathlineto{\pgfqpoint{5.369790in}{0.793440in}}%
\pgfpathlineto{\pgfqpoint{5.370657in}{1.354412in}}%
\pgfpathlineto{\pgfqpoint{5.371524in}{0.970490in}}%
\pgfpathlineto{\pgfqpoint{5.372392in}{1.017178in}}%
\pgfpathlineto{\pgfqpoint{5.373259in}{1.014203in}}%
\pgfpathlineto{\pgfqpoint{5.374993in}{1.282709in}}%
\pgfpathlineto{\pgfqpoint{5.375860in}{1.369685in}}%
\pgfpathlineto{\pgfqpoint{5.376727in}{0.809723in}}%
\pgfpathlineto{\pgfqpoint{5.378462in}{1.232004in}}%
\pgfpathlineto{\pgfqpoint{5.379329in}{0.881782in}}%
\pgfpathlineto{\pgfqpoint{5.380196in}{1.048596in}}%
\pgfpathlineto{\pgfqpoint{5.381063in}{0.638634in}}%
\pgfpathlineto{\pgfqpoint{5.381930in}{1.132926in}}%
\pgfpathlineto{\pgfqpoint{5.382797in}{0.847113in}}%
\pgfpathlineto{\pgfqpoint{5.383664in}{1.682291in}}%
\pgfpathlineto{\pgfqpoint{5.384531in}{0.946719in}}%
\pgfpathlineto{\pgfqpoint{5.385399in}{1.099162in}}%
\pgfpathlineto{\pgfqpoint{5.386266in}{1.466945in}}%
\pgfpathlineto{\pgfqpoint{5.388000in}{0.883153in}}%
\pgfpathlineto{\pgfqpoint{5.388867in}{0.671970in}}%
\pgfpathlineto{\pgfqpoint{5.389734in}{0.731929in}}%
\pgfpathlineto{\pgfqpoint{5.391469in}{1.670342in}}%
\pgfpathlineto{\pgfqpoint{5.392336in}{0.730525in}}%
\pgfpathlineto{\pgfqpoint{5.393203in}{1.097900in}}%
\pgfpathlineto{\pgfqpoint{5.394070in}{1.092160in}}%
\pgfpathlineto{\pgfqpoint{5.394937in}{0.932272in}}%
\pgfpathlineto{\pgfqpoint{5.395804in}{1.063214in}}%
\pgfpathlineto{\pgfqpoint{5.397538in}{0.859862in}}%
\pgfpathlineto{\pgfqpoint{5.398406in}{1.053376in}}%
\pgfpathlineto{\pgfqpoint{5.399273in}{1.051695in}}%
\pgfpathlineto{\pgfqpoint{5.400140in}{0.780162in}}%
\pgfpathlineto{\pgfqpoint{5.401874in}{1.162924in}}%
\pgfpathlineto{\pgfqpoint{5.403608in}{0.625968in}}%
\pgfpathlineto{\pgfqpoint{5.405343in}{1.088658in}}%
\pgfpathlineto{\pgfqpoint{5.406210in}{0.833620in}}%
\pgfpathlineto{\pgfqpoint{5.407077in}{1.361634in}}%
\pgfpathlineto{\pgfqpoint{5.407944in}{0.876743in}}%
\pgfpathlineto{\pgfqpoint{5.408811in}{1.189249in}}%
\pgfpathlineto{\pgfqpoint{5.409678in}{1.151925in}}%
\pgfpathlineto{\pgfqpoint{5.410545in}{1.255833in}}%
\pgfpathlineto{\pgfqpoint{5.411413in}{0.917050in}}%
\pgfpathlineto{\pgfqpoint{5.412280in}{0.921026in}}%
\pgfpathlineto{\pgfqpoint{5.414014in}{1.256592in}}%
\pgfpathlineto{\pgfqpoint{5.414881in}{0.664349in}}%
\pgfpathlineto{\pgfqpoint{5.415748in}{0.697294in}}%
\pgfpathlineto{\pgfqpoint{5.417483in}{1.110756in}}%
\pgfpathlineto{\pgfqpoint{5.418350in}{0.738158in}}%
\pgfpathlineto{\pgfqpoint{5.420084in}{0.991655in}}%
\pgfpathlineto{\pgfqpoint{5.420951in}{0.733472in}}%
\pgfpathlineto{\pgfqpoint{5.423552in}{1.277628in}}%
\pgfpathlineto{\pgfqpoint{5.425287in}{0.653246in}}%
\pgfpathlineto{\pgfqpoint{5.426154in}{1.100145in}}%
\pgfpathlineto{\pgfqpoint{5.427021in}{0.896466in}}%
\pgfpathlineto{\pgfqpoint{5.427888in}{1.448262in}}%
\pgfpathlineto{\pgfqpoint{5.428755in}{0.797252in}}%
\pgfpathlineto{\pgfqpoint{5.429622in}{0.964329in}}%
\pgfpathlineto{\pgfqpoint{5.430490in}{0.790002in}}%
\pgfpathlineto{\pgfqpoint{5.432224in}{1.383996in}}%
\pgfpathlineto{\pgfqpoint{5.433091in}{1.184017in}}%
\pgfpathlineto{\pgfqpoint{5.433958in}{1.243200in}}%
\pgfpathlineto{\pgfqpoint{5.435692in}{0.794350in}}%
\pgfpathlineto{\pgfqpoint{5.438294in}{0.961344in}}%
\pgfpathlineto{\pgfqpoint{5.439161in}{0.724924in}}%
\pgfpathlineto{\pgfqpoint{5.440028in}{1.099244in}}%
\pgfpathlineto{\pgfqpoint{5.441762in}{0.854943in}}%
\pgfpathlineto{\pgfqpoint{5.442629in}{1.298871in}}%
\pgfpathlineto{\pgfqpoint{5.444364in}{0.863681in}}%
\pgfpathlineto{\pgfqpoint{5.445231in}{0.967667in}}%
\pgfpathlineto{\pgfqpoint{5.446098in}{1.553421in}}%
\pgfpathlineto{\pgfqpoint{5.446965in}{0.702116in}}%
\pgfpathlineto{\pgfqpoint{5.447832in}{0.766544in}}%
\pgfpathlineto{\pgfqpoint{5.450434in}{1.308110in}}%
\pgfpathlineto{\pgfqpoint{5.452168in}{0.748389in}}%
\pgfpathlineto{\pgfqpoint{5.453035in}{0.995130in}}%
\pgfpathlineto{\pgfqpoint{5.454769in}{0.796075in}}%
\pgfpathlineto{\pgfqpoint{5.456503in}{1.084187in}}%
\pgfpathlineto{\pgfqpoint{5.457371in}{0.747950in}}%
\pgfpathlineto{\pgfqpoint{5.458238in}{0.783589in}}%
\pgfpathlineto{\pgfqpoint{5.459105in}{0.738236in}}%
\pgfpathlineto{\pgfqpoint{5.459972in}{1.029811in}}%
\pgfpathlineto{\pgfqpoint{5.460839in}{1.008122in}}%
\pgfpathlineto{\pgfqpoint{5.461706in}{0.668202in}}%
\pgfpathlineto{\pgfqpoint{5.462573in}{0.858994in}}%
\pgfpathlineto{\pgfqpoint{5.463441in}{0.710054in}}%
\pgfpathlineto{\pgfqpoint{5.464308in}{0.922919in}}%
\pgfpathlineto{\pgfqpoint{5.465175in}{0.863888in}}%
\pgfpathlineto{\pgfqpoint{5.466042in}{0.999849in}}%
\pgfpathlineto{\pgfqpoint{5.466909in}{0.895020in}}%
\pgfpathlineto{\pgfqpoint{5.467776in}{1.870994in}}%
\pgfpathlineto{\pgfqpoint{5.469510in}{0.754361in}}%
\pgfpathlineto{\pgfqpoint{5.471245in}{1.255258in}}%
\pgfpathlineto{\pgfqpoint{5.472112in}{1.063710in}}%
\pgfpathlineto{\pgfqpoint{5.472979in}{1.097641in}}%
\pgfpathlineto{\pgfqpoint{5.473846in}{0.996583in}}%
\pgfpathlineto{\pgfqpoint{5.474713in}{0.699483in}}%
\pgfpathlineto{\pgfqpoint{5.476448in}{1.030295in}}%
\pgfpathlineto{\pgfqpoint{5.477315in}{1.226766in}}%
\pgfpathlineto{\pgfqpoint{5.479049in}{0.753325in}}%
\pgfpathlineto{\pgfqpoint{5.481650in}{1.651365in}}%
\pgfpathlineto{\pgfqpoint{5.482517in}{1.149006in}}%
\pgfpathlineto{\pgfqpoint{5.483385in}{1.262310in}}%
\pgfpathlineto{\pgfqpoint{5.484252in}{0.842865in}}%
\pgfpathlineto{\pgfqpoint{5.485119in}{1.102667in}}%
\pgfpathlineto{\pgfqpoint{5.485986in}{0.952315in}}%
\pgfpathlineto{\pgfqpoint{5.486853in}{1.376838in}}%
\pgfpathlineto{\pgfqpoint{5.487720in}{0.693894in}}%
\pgfpathlineto{\pgfqpoint{5.490322in}{1.565074in}}%
\pgfpathlineto{\pgfqpoint{5.492056in}{0.816795in}}%
\pgfpathlineto{\pgfqpoint{5.493790in}{1.116630in}}%
\pgfpathlineto{\pgfqpoint{5.494657in}{1.113686in}}%
\pgfpathlineto{\pgfqpoint{5.495524in}{1.829226in}}%
\pgfpathlineto{\pgfqpoint{5.497259in}{0.787335in}}%
\pgfpathlineto{\pgfqpoint{5.498993in}{0.919147in}}%
\pgfpathlineto{\pgfqpoint{5.499860in}{0.808448in}}%
\pgfpathlineto{\pgfqpoint{5.500727in}{0.874107in}}%
\pgfpathlineto{\pgfqpoint{5.502462in}{1.159057in}}%
\pgfpathlineto{\pgfqpoint{5.504196in}{0.936826in}}%
\pgfpathlineto{\pgfqpoint{5.505063in}{1.322009in}}%
\pgfpathlineto{\pgfqpoint{5.505930in}{0.779888in}}%
\pgfpathlineto{\pgfqpoint{5.506797in}{0.912919in}}%
\pgfpathlineto{\pgfqpoint{5.507664in}{0.749795in}}%
\pgfpathlineto{\pgfqpoint{5.508531in}{0.798090in}}%
\pgfpathlineto{\pgfqpoint{5.509399in}{1.479783in}}%
\pgfpathlineto{\pgfqpoint{5.510266in}{0.750557in}}%
\pgfpathlineto{\pgfqpoint{5.512867in}{1.803990in}}%
\pgfpathlineto{\pgfqpoint{5.514601in}{1.150314in}}%
\pgfpathlineto{\pgfqpoint{5.515469in}{1.091376in}}%
\pgfpathlineto{\pgfqpoint{5.516336in}{0.772167in}}%
\pgfpathlineto{\pgfqpoint{5.517203in}{1.347520in}}%
\pgfpathlineto{\pgfqpoint{5.518070in}{0.711727in}}%
\pgfpathlineto{\pgfqpoint{5.518937in}{0.755967in}}%
\pgfpathlineto{\pgfqpoint{5.519804in}{0.871324in}}%
\pgfpathlineto{\pgfqpoint{5.520671in}{1.144614in}}%
\pgfpathlineto{\pgfqpoint{5.522406in}{0.961971in}}%
\pgfpathlineto{\pgfqpoint{5.524140in}{1.386129in}}%
\pgfpathlineto{\pgfqpoint{5.525007in}{0.789860in}}%
\pgfpathlineto{\pgfqpoint{5.526741in}{1.630134in}}%
\pgfpathlineto{\pgfqpoint{5.527608in}{1.558716in}}%
\pgfpathlineto{\pgfqpoint{5.528476in}{0.886146in}}%
\pgfpathlineto{\pgfqpoint{5.529343in}{1.300207in}}%
\pgfpathlineto{\pgfqpoint{5.530210in}{0.899455in}}%
\pgfpathlineto{\pgfqpoint{5.531077in}{1.212427in}}%
\pgfpathlineto{\pgfqpoint{5.531944in}{1.197450in}}%
\pgfpathlineto{\pgfqpoint{5.532811in}{0.944405in}}%
\pgfpathlineto{\pgfqpoint{5.533678in}{0.950947in}}%
\pgfpathlineto{\pgfqpoint{5.534545in}{1.041043in}}%
\pgfpathlineto{\pgfqpoint{5.534545in}{1.041043in}}%
\pgfusepath{stroke}%
\end{pgfscope}%
\begin{pgfscope}%
\pgfsetrectcap%
\pgfsetmiterjoin%
\pgfsetlinewidth{0.803000pt}%
\definecolor{currentstroke}{rgb}{0.000000,0.000000,0.000000}%
\pgfsetstrokecolor{currentstroke}%
\pgfsetdash{}{0pt}%
\pgfpathmoveto{\pgfqpoint{0.800000in}{0.528000in}}%
\pgfpathlineto{\pgfqpoint{0.800000in}{2.208000in}}%
\pgfusepath{stroke}%
\end{pgfscope}%
\begin{pgfscope}%
\pgfsetrectcap%
\pgfsetmiterjoin%
\pgfsetlinewidth{0.803000pt}%
\definecolor{currentstroke}{rgb}{0.000000,0.000000,0.000000}%
\pgfsetstrokecolor{currentstroke}%
\pgfsetdash{}{0pt}%
\pgfpathmoveto{\pgfqpoint{5.760000in}{0.528000in}}%
\pgfpathlineto{\pgfqpoint{5.760000in}{2.208000in}}%
\pgfusepath{stroke}%
\end{pgfscope}%
\begin{pgfscope}%
\pgfsetrectcap%
\pgfsetmiterjoin%
\pgfsetlinewidth{0.803000pt}%
\definecolor{currentstroke}{rgb}{0.000000,0.000000,0.000000}%
\pgfsetstrokecolor{currentstroke}%
\pgfsetdash{}{0pt}%
\pgfpathmoveto{\pgfqpoint{0.800000in}{0.528000in}}%
\pgfpathlineto{\pgfqpoint{5.760000in}{0.528000in}}%
\pgfusepath{stroke}%
\end{pgfscope}%
\begin{pgfscope}%
\pgfsetrectcap%
\pgfsetmiterjoin%
\pgfsetlinewidth{0.803000pt}%
\definecolor{currentstroke}{rgb}{0.000000,0.000000,0.000000}%
\pgfsetstrokecolor{currentstroke}%
\pgfsetdash{}{0pt}%
\pgfpathmoveto{\pgfqpoint{0.800000in}{2.208000in}}%
\pgfpathlineto{\pgfqpoint{5.760000in}{2.208000in}}%
\pgfusepath{stroke}%
\end{pgfscope}%
\end{pgfpicture}%
\makeatother%
\endgroup%
}}
  \caption{Received signal.}
  \label{fig:task1_sr}
\end{figure}

\section{Task 2}\label{sec:2}
The pulse-compressed signal can be seen in Figure~\ref{fig:task2}, where three targets can be seen, located $\SI{7.4}{\kilo\meter}$, $\SI{8.4}{\kilo\meter}$ and $\SI{8.4}{\kilo\meter}$ away.
\begin{figure}[h]
  \centering
  \noindent\makebox[\textwidth]{\scalebox{0.90}{%% Creator: Matplotlib, PGF backend
%%
%% To include the figure in your LaTeX document, write
%%   \input{<filename>.pgf}
%%
%% Make sure the required packages are loaded in your preamble
%%   \usepackage{pgf}
%%
%% Figures using additional raster images can only be included by \input if
%% they are in the same directory as the main LaTeX file. For loading figures
%% from other directories you can use the `import` package
%%   \usepackage{import}
%% and then include the figures with
%%   \import{<path to file>}{<filename>.pgf}
%%
%% Matplotlib used the following preamble
%%   \usepackage{fontspec}
%%   \setmainfont{DejaVu Serif}
%%   \setsansfont{DejaVu Sans}
%%   \setmonofont{DejaVu Sans Mono}
%%
\begingroup%
\makeatletter%
\begin{pgfpicture}%
\pgfpathrectangle{\pgfpointorigin}{\pgfqpoint{6.400000in}{4.800000in}}%
\pgfusepath{use as bounding box, clip}%
\begin{pgfscope}%
\pgfsetbuttcap%
\pgfsetmiterjoin%
\definecolor{currentfill}{rgb}{1.000000,1.000000,1.000000}%
\pgfsetfillcolor{currentfill}%
\pgfsetlinewidth{0.000000pt}%
\definecolor{currentstroke}{rgb}{1.000000,1.000000,1.000000}%
\pgfsetstrokecolor{currentstroke}%
\pgfsetdash{}{0pt}%
\pgfpathmoveto{\pgfqpoint{0.000000in}{0.000000in}}%
\pgfpathlineto{\pgfqpoint{6.400000in}{0.000000in}}%
\pgfpathlineto{\pgfqpoint{6.400000in}{4.800000in}}%
\pgfpathlineto{\pgfqpoint{0.000000in}{4.800000in}}%
\pgfpathclose%
\pgfusepath{fill}%
\end{pgfscope}%
\begin{pgfscope}%
\pgfsetbuttcap%
\pgfsetmiterjoin%
\definecolor{currentfill}{rgb}{1.000000,1.000000,1.000000}%
\pgfsetfillcolor{currentfill}%
\pgfsetlinewidth{0.000000pt}%
\definecolor{currentstroke}{rgb}{0.000000,0.000000,0.000000}%
\pgfsetstrokecolor{currentstroke}%
\pgfsetstrokeopacity{0.000000}%
\pgfsetdash{}{0pt}%
\pgfpathmoveto{\pgfqpoint{0.800000in}{0.528000in}}%
\pgfpathlineto{\pgfqpoint{5.760000in}{0.528000in}}%
\pgfpathlineto{\pgfqpoint{5.760000in}{4.224000in}}%
\pgfpathlineto{\pgfqpoint{0.800000in}{4.224000in}}%
\pgfpathclose%
\pgfusepath{fill}%
\end{pgfscope}%
\begin{pgfscope}%
\pgfsetbuttcap%
\pgfsetroundjoin%
\definecolor{currentfill}{rgb}{0.000000,0.000000,0.000000}%
\pgfsetfillcolor{currentfill}%
\pgfsetlinewidth{0.803000pt}%
\definecolor{currentstroke}{rgb}{0.000000,0.000000,0.000000}%
\pgfsetstrokecolor{currentstroke}%
\pgfsetdash{}{0pt}%
\pgfsys@defobject{currentmarker}{\pgfqpoint{0.000000in}{-0.048611in}}{\pgfqpoint{0.000000in}{0.000000in}}{%
\pgfpathmoveto{\pgfqpoint{0.000000in}{0.000000in}}%
\pgfpathlineto{\pgfqpoint{0.000000in}{-0.048611in}}%
\pgfusepath{stroke,fill}%
}%
\begin{pgfscope}%
\pgfsys@transformshift{1.030917in}{0.528000in}%
\pgfsys@useobject{currentmarker}{}%
\end{pgfscope}%
\end{pgfscope}%
\begin{pgfscope}%
\pgftext[x=1.030917in,y=0.430778in,,top]{\sffamily\fontsize{10.000000}{12.000000}\selectfont 7.0}%
\end{pgfscope}%
\begin{pgfscope}%
\pgfsetbuttcap%
\pgfsetroundjoin%
\definecolor{currentfill}{rgb}{0.000000,0.000000,0.000000}%
\pgfsetfillcolor{currentfill}%
\pgfsetlinewidth{0.803000pt}%
\definecolor{currentstroke}{rgb}{0.000000,0.000000,0.000000}%
\pgfsetstrokecolor{currentstroke}%
\pgfsetdash{}{0pt}%
\pgfsys@defobject{currentmarker}{\pgfqpoint{0.000000in}{-0.048611in}}{\pgfqpoint{0.000000in}{0.000000in}}{%
\pgfpathmoveto{\pgfqpoint{0.000000in}{0.000000in}}%
\pgfpathlineto{\pgfqpoint{0.000000in}{-0.048611in}}%
\pgfusepath{stroke,fill}%
}%
\begin{pgfscope}%
\pgfsys@transformshift{1.594944in}{0.528000in}%
\pgfsys@useobject{currentmarker}{}%
\end{pgfscope}%
\end{pgfscope}%
\begin{pgfscope}%
\pgftext[x=1.594944in,y=0.430778in,,top]{\sffamily\fontsize{10.000000}{12.000000}\selectfont 7.5}%
\end{pgfscope}%
\begin{pgfscope}%
\pgfsetbuttcap%
\pgfsetroundjoin%
\definecolor{currentfill}{rgb}{0.000000,0.000000,0.000000}%
\pgfsetfillcolor{currentfill}%
\pgfsetlinewidth{0.803000pt}%
\definecolor{currentstroke}{rgb}{0.000000,0.000000,0.000000}%
\pgfsetstrokecolor{currentstroke}%
\pgfsetdash{}{0pt}%
\pgfsys@defobject{currentmarker}{\pgfqpoint{0.000000in}{-0.048611in}}{\pgfqpoint{0.000000in}{0.000000in}}{%
\pgfpathmoveto{\pgfqpoint{0.000000in}{0.000000in}}%
\pgfpathlineto{\pgfqpoint{0.000000in}{-0.048611in}}%
\pgfusepath{stroke,fill}%
}%
\begin{pgfscope}%
\pgfsys@transformshift{2.158970in}{0.528000in}%
\pgfsys@useobject{currentmarker}{}%
\end{pgfscope}%
\end{pgfscope}%
\begin{pgfscope}%
\pgftext[x=2.158970in,y=0.430778in,,top]{\sffamily\fontsize{10.000000}{12.000000}\selectfont 8.0}%
\end{pgfscope}%
\begin{pgfscope}%
\pgfsetbuttcap%
\pgfsetroundjoin%
\definecolor{currentfill}{rgb}{0.000000,0.000000,0.000000}%
\pgfsetfillcolor{currentfill}%
\pgfsetlinewidth{0.803000pt}%
\definecolor{currentstroke}{rgb}{0.000000,0.000000,0.000000}%
\pgfsetstrokecolor{currentstroke}%
\pgfsetdash{}{0pt}%
\pgfsys@defobject{currentmarker}{\pgfqpoint{0.000000in}{-0.048611in}}{\pgfqpoint{0.000000in}{0.000000in}}{%
\pgfpathmoveto{\pgfqpoint{0.000000in}{0.000000in}}%
\pgfpathlineto{\pgfqpoint{0.000000in}{-0.048611in}}%
\pgfusepath{stroke,fill}%
}%
\begin{pgfscope}%
\pgfsys@transformshift{2.722997in}{0.528000in}%
\pgfsys@useobject{currentmarker}{}%
\end{pgfscope}%
\end{pgfscope}%
\begin{pgfscope}%
\pgftext[x=2.722997in,y=0.430778in,,top]{\sffamily\fontsize{10.000000}{12.000000}\selectfont 8.5}%
\end{pgfscope}%
\begin{pgfscope}%
\pgfsetbuttcap%
\pgfsetroundjoin%
\definecolor{currentfill}{rgb}{0.000000,0.000000,0.000000}%
\pgfsetfillcolor{currentfill}%
\pgfsetlinewidth{0.803000pt}%
\definecolor{currentstroke}{rgb}{0.000000,0.000000,0.000000}%
\pgfsetstrokecolor{currentstroke}%
\pgfsetdash{}{0pt}%
\pgfsys@defobject{currentmarker}{\pgfqpoint{0.000000in}{-0.048611in}}{\pgfqpoint{0.000000in}{0.000000in}}{%
\pgfpathmoveto{\pgfqpoint{0.000000in}{0.000000in}}%
\pgfpathlineto{\pgfqpoint{0.000000in}{-0.048611in}}%
\pgfusepath{stroke,fill}%
}%
\begin{pgfscope}%
\pgfsys@transformshift{3.287024in}{0.528000in}%
\pgfsys@useobject{currentmarker}{}%
\end{pgfscope}%
\end{pgfscope}%
\begin{pgfscope}%
\pgftext[x=3.287024in,y=0.430778in,,top]{\sffamily\fontsize{10.000000}{12.000000}\selectfont 9.0}%
\end{pgfscope}%
\begin{pgfscope}%
\pgfsetbuttcap%
\pgfsetroundjoin%
\definecolor{currentfill}{rgb}{0.000000,0.000000,0.000000}%
\pgfsetfillcolor{currentfill}%
\pgfsetlinewidth{0.803000pt}%
\definecolor{currentstroke}{rgb}{0.000000,0.000000,0.000000}%
\pgfsetstrokecolor{currentstroke}%
\pgfsetdash{}{0pt}%
\pgfsys@defobject{currentmarker}{\pgfqpoint{0.000000in}{-0.048611in}}{\pgfqpoint{0.000000in}{0.000000in}}{%
\pgfpathmoveto{\pgfqpoint{0.000000in}{0.000000in}}%
\pgfpathlineto{\pgfqpoint{0.000000in}{-0.048611in}}%
\pgfusepath{stroke,fill}%
}%
\begin{pgfscope}%
\pgfsys@transformshift{3.851050in}{0.528000in}%
\pgfsys@useobject{currentmarker}{}%
\end{pgfscope}%
\end{pgfscope}%
\begin{pgfscope}%
\pgftext[x=3.851050in,y=0.430778in,,top]{\sffamily\fontsize{10.000000}{12.000000}\selectfont 9.5}%
\end{pgfscope}%
\begin{pgfscope}%
\pgfsetbuttcap%
\pgfsetroundjoin%
\definecolor{currentfill}{rgb}{0.000000,0.000000,0.000000}%
\pgfsetfillcolor{currentfill}%
\pgfsetlinewidth{0.803000pt}%
\definecolor{currentstroke}{rgb}{0.000000,0.000000,0.000000}%
\pgfsetstrokecolor{currentstroke}%
\pgfsetdash{}{0pt}%
\pgfsys@defobject{currentmarker}{\pgfqpoint{0.000000in}{-0.048611in}}{\pgfqpoint{0.000000in}{0.000000in}}{%
\pgfpathmoveto{\pgfqpoint{0.000000in}{0.000000in}}%
\pgfpathlineto{\pgfqpoint{0.000000in}{-0.048611in}}%
\pgfusepath{stroke,fill}%
}%
\begin{pgfscope}%
\pgfsys@transformshift{4.415077in}{0.528000in}%
\pgfsys@useobject{currentmarker}{}%
\end{pgfscope}%
\end{pgfscope}%
\begin{pgfscope}%
\pgftext[x=4.415077in,y=0.430778in,,top]{\sffamily\fontsize{10.000000}{12.000000}\selectfont 10.0}%
\end{pgfscope}%
\begin{pgfscope}%
\pgfsetbuttcap%
\pgfsetroundjoin%
\definecolor{currentfill}{rgb}{0.000000,0.000000,0.000000}%
\pgfsetfillcolor{currentfill}%
\pgfsetlinewidth{0.803000pt}%
\definecolor{currentstroke}{rgb}{0.000000,0.000000,0.000000}%
\pgfsetstrokecolor{currentstroke}%
\pgfsetdash{}{0pt}%
\pgfsys@defobject{currentmarker}{\pgfqpoint{0.000000in}{-0.048611in}}{\pgfqpoint{0.000000in}{0.000000in}}{%
\pgfpathmoveto{\pgfqpoint{0.000000in}{0.000000in}}%
\pgfpathlineto{\pgfqpoint{0.000000in}{-0.048611in}}%
\pgfusepath{stroke,fill}%
}%
\begin{pgfscope}%
\pgfsys@transformshift{4.979103in}{0.528000in}%
\pgfsys@useobject{currentmarker}{}%
\end{pgfscope}%
\end{pgfscope}%
\begin{pgfscope}%
\pgftext[x=4.979103in,y=0.430778in,,top]{\sffamily\fontsize{10.000000}{12.000000}\selectfont 10.5}%
\end{pgfscope}%
\begin{pgfscope}%
\pgfsetbuttcap%
\pgfsetroundjoin%
\definecolor{currentfill}{rgb}{0.000000,0.000000,0.000000}%
\pgfsetfillcolor{currentfill}%
\pgfsetlinewidth{0.803000pt}%
\definecolor{currentstroke}{rgb}{0.000000,0.000000,0.000000}%
\pgfsetstrokecolor{currentstroke}%
\pgfsetdash{}{0pt}%
\pgfsys@defobject{currentmarker}{\pgfqpoint{0.000000in}{-0.048611in}}{\pgfqpoint{0.000000in}{0.000000in}}{%
\pgfpathmoveto{\pgfqpoint{0.000000in}{0.000000in}}%
\pgfpathlineto{\pgfqpoint{0.000000in}{-0.048611in}}%
\pgfusepath{stroke,fill}%
}%
\begin{pgfscope}%
\pgfsys@transformshift{5.543130in}{0.528000in}%
\pgfsys@useobject{currentmarker}{}%
\end{pgfscope}%
\end{pgfscope}%
\begin{pgfscope}%
\pgftext[x=5.543130in,y=0.430778in,,top]{\sffamily\fontsize{10.000000}{12.000000}\selectfont 11.0}%
\end{pgfscope}%
\begin{pgfscope}%
\pgftext[x=3.280000in,y=0.240809in,,top]{\sffamily\fontsize{10.000000}{12.000000}\selectfont Distance [km]}%
\end{pgfscope}%
\begin{pgfscope}%
\pgfsetbuttcap%
\pgfsetroundjoin%
\definecolor{currentfill}{rgb}{0.000000,0.000000,0.000000}%
\pgfsetfillcolor{currentfill}%
\pgfsetlinewidth{0.803000pt}%
\definecolor{currentstroke}{rgb}{0.000000,0.000000,0.000000}%
\pgfsetstrokecolor{currentstroke}%
\pgfsetdash{}{0pt}%
\pgfsys@defobject{currentmarker}{\pgfqpoint{-0.048611in}{0.000000in}}{\pgfqpoint{0.000000in}{0.000000in}}{%
\pgfpathmoveto{\pgfqpoint{0.000000in}{0.000000in}}%
\pgfpathlineto{\pgfqpoint{-0.048611in}{0.000000in}}%
\pgfusepath{stroke,fill}%
}%
\begin{pgfscope}%
\pgfsys@transformshift{0.800000in}{0.695998in}%
\pgfsys@useobject{currentmarker}{}%
\end{pgfscope}%
\end{pgfscope}%
\begin{pgfscope}%
\pgftext[x=0.481898in,y=0.643236in,left,base]{\sffamily\fontsize{10.000000}{12.000000}\selectfont 0.0}%
\end{pgfscope}%
\begin{pgfscope}%
\pgfsetbuttcap%
\pgfsetroundjoin%
\definecolor{currentfill}{rgb}{0.000000,0.000000,0.000000}%
\pgfsetfillcolor{currentfill}%
\pgfsetlinewidth{0.803000pt}%
\definecolor{currentstroke}{rgb}{0.000000,0.000000,0.000000}%
\pgfsetstrokecolor{currentstroke}%
\pgfsetdash{}{0pt}%
\pgfsys@defobject{currentmarker}{\pgfqpoint{-0.048611in}{0.000000in}}{\pgfqpoint{0.000000in}{0.000000in}}{%
\pgfpathmoveto{\pgfqpoint{0.000000in}{0.000000in}}%
\pgfpathlineto{\pgfqpoint{-0.048611in}{0.000000in}}%
\pgfusepath{stroke,fill}%
}%
\begin{pgfscope}%
\pgfsys@transformshift{0.800000in}{1.367998in}%
\pgfsys@useobject{currentmarker}{}%
\end{pgfscope}%
\end{pgfscope}%
\begin{pgfscope}%
\pgftext[x=0.481898in,y=1.315237in,left,base]{\sffamily\fontsize{10.000000}{12.000000}\selectfont 0.2}%
\end{pgfscope}%
\begin{pgfscope}%
\pgfsetbuttcap%
\pgfsetroundjoin%
\definecolor{currentfill}{rgb}{0.000000,0.000000,0.000000}%
\pgfsetfillcolor{currentfill}%
\pgfsetlinewidth{0.803000pt}%
\definecolor{currentstroke}{rgb}{0.000000,0.000000,0.000000}%
\pgfsetstrokecolor{currentstroke}%
\pgfsetdash{}{0pt}%
\pgfsys@defobject{currentmarker}{\pgfqpoint{-0.048611in}{0.000000in}}{\pgfqpoint{0.000000in}{0.000000in}}{%
\pgfpathmoveto{\pgfqpoint{0.000000in}{0.000000in}}%
\pgfpathlineto{\pgfqpoint{-0.048611in}{0.000000in}}%
\pgfusepath{stroke,fill}%
}%
\begin{pgfscope}%
\pgfsys@transformshift{0.800000in}{2.039999in}%
\pgfsys@useobject{currentmarker}{}%
\end{pgfscope}%
\end{pgfscope}%
\begin{pgfscope}%
\pgftext[x=0.481898in,y=1.987237in,left,base]{\sffamily\fontsize{10.000000}{12.000000}\selectfont 0.4}%
\end{pgfscope}%
\begin{pgfscope}%
\pgfsetbuttcap%
\pgfsetroundjoin%
\definecolor{currentfill}{rgb}{0.000000,0.000000,0.000000}%
\pgfsetfillcolor{currentfill}%
\pgfsetlinewidth{0.803000pt}%
\definecolor{currentstroke}{rgb}{0.000000,0.000000,0.000000}%
\pgfsetstrokecolor{currentstroke}%
\pgfsetdash{}{0pt}%
\pgfsys@defobject{currentmarker}{\pgfqpoint{-0.048611in}{0.000000in}}{\pgfqpoint{0.000000in}{0.000000in}}{%
\pgfpathmoveto{\pgfqpoint{0.000000in}{0.000000in}}%
\pgfpathlineto{\pgfqpoint{-0.048611in}{0.000000in}}%
\pgfusepath{stroke,fill}%
}%
\begin{pgfscope}%
\pgfsys@transformshift{0.800000in}{2.711999in}%
\pgfsys@useobject{currentmarker}{}%
\end{pgfscope}%
\end{pgfscope}%
\begin{pgfscope}%
\pgftext[x=0.481898in,y=2.659238in,left,base]{\sffamily\fontsize{10.000000}{12.000000}\selectfont 0.6}%
\end{pgfscope}%
\begin{pgfscope}%
\pgfsetbuttcap%
\pgfsetroundjoin%
\definecolor{currentfill}{rgb}{0.000000,0.000000,0.000000}%
\pgfsetfillcolor{currentfill}%
\pgfsetlinewidth{0.803000pt}%
\definecolor{currentstroke}{rgb}{0.000000,0.000000,0.000000}%
\pgfsetstrokecolor{currentstroke}%
\pgfsetdash{}{0pt}%
\pgfsys@defobject{currentmarker}{\pgfqpoint{-0.048611in}{0.000000in}}{\pgfqpoint{0.000000in}{0.000000in}}{%
\pgfpathmoveto{\pgfqpoint{0.000000in}{0.000000in}}%
\pgfpathlineto{\pgfqpoint{-0.048611in}{0.000000in}}%
\pgfusepath{stroke,fill}%
}%
\begin{pgfscope}%
\pgfsys@transformshift{0.800000in}{3.384000in}%
\pgfsys@useobject{currentmarker}{}%
\end{pgfscope}%
\end{pgfscope}%
\begin{pgfscope}%
\pgftext[x=0.481898in,y=3.331238in,left,base]{\sffamily\fontsize{10.000000}{12.000000}\selectfont 0.8}%
\end{pgfscope}%
\begin{pgfscope}%
\pgfsetbuttcap%
\pgfsetroundjoin%
\definecolor{currentfill}{rgb}{0.000000,0.000000,0.000000}%
\pgfsetfillcolor{currentfill}%
\pgfsetlinewidth{0.803000pt}%
\definecolor{currentstroke}{rgb}{0.000000,0.000000,0.000000}%
\pgfsetstrokecolor{currentstroke}%
\pgfsetdash{}{0pt}%
\pgfsys@defobject{currentmarker}{\pgfqpoint{-0.048611in}{0.000000in}}{\pgfqpoint{0.000000in}{0.000000in}}{%
\pgfpathmoveto{\pgfqpoint{0.000000in}{0.000000in}}%
\pgfpathlineto{\pgfqpoint{-0.048611in}{0.000000in}}%
\pgfusepath{stroke,fill}%
}%
\begin{pgfscope}%
\pgfsys@transformshift{0.800000in}{4.056000in}%
\pgfsys@useobject{currentmarker}{}%
\end{pgfscope}%
\end{pgfscope}%
\begin{pgfscope}%
\pgftext[x=0.481898in,y=4.003238in,left,base]{\sffamily\fontsize{10.000000}{12.000000}\selectfont 1.0}%
\end{pgfscope}%
\begin{pgfscope}%
\pgftext[x=0.426343in,y=2.376000in,,bottom,rotate=90.000000]{\sffamily\fontsize{10.000000}{12.000000}\selectfont Relative (squared) magnitude}%
\end{pgfscope}%
\begin{pgfscope}%
\pgfpathrectangle{\pgfqpoint{0.800000in}{0.528000in}}{\pgfqpoint{4.960000in}{3.696000in}} %
\pgfusepath{clip}%
\pgfsetrectcap%
\pgfsetroundjoin%
\pgfsetlinewidth{1.505625pt}%
\definecolor{currentstroke}{rgb}{0.121569,0.466667,0.705882}%
\pgfsetstrokecolor{currentstroke}%
\pgfsetdash{}{0pt}%
\pgfpathmoveto{\pgfqpoint{1.025455in}{0.696552in}}%
\pgfpathlineto{\pgfqpoint{1.028056in}{0.696245in}}%
\pgfpathlineto{\pgfqpoint{1.044531in}{0.696103in}}%
\pgfpathlineto{\pgfqpoint{1.050601in}{0.696148in}}%
\pgfpathlineto{\pgfqpoint{1.074014in}{0.696249in}}%
\pgfpathlineto{\pgfqpoint{1.080084in}{0.696525in}}%
\pgfpathlineto{\pgfqpoint{1.081818in}{0.697070in}}%
\pgfpathlineto{\pgfqpoint{1.085287in}{0.698729in}}%
\pgfpathlineto{\pgfqpoint{1.086154in}{0.696771in}}%
\pgfpathlineto{\pgfqpoint{1.087021in}{0.699121in}}%
\pgfpathlineto{\pgfqpoint{1.087888in}{0.696677in}}%
\pgfpathlineto{\pgfqpoint{1.088755in}{0.700674in}}%
\pgfpathlineto{\pgfqpoint{1.089622in}{0.696240in}}%
\pgfpathlineto{\pgfqpoint{1.090490in}{0.700547in}}%
\pgfpathlineto{\pgfqpoint{1.091357in}{0.696280in}}%
\pgfpathlineto{\pgfqpoint{1.092224in}{0.701709in}}%
\pgfpathlineto{\pgfqpoint{1.093091in}{0.696217in}}%
\pgfpathlineto{\pgfqpoint{1.093958in}{0.703272in}}%
\pgfpathlineto{\pgfqpoint{1.094825in}{0.697191in}}%
\pgfpathlineto{\pgfqpoint{1.095692in}{0.704281in}}%
\pgfpathlineto{\pgfqpoint{1.096559in}{0.699450in}}%
\pgfpathlineto{\pgfqpoint{1.097427in}{0.702918in}}%
\pgfpathlineto{\pgfqpoint{1.099161in}{0.701501in}}%
\pgfpathlineto{\pgfqpoint{1.100028in}{0.705399in}}%
\pgfpathlineto{\pgfqpoint{1.100895in}{0.699994in}}%
\pgfpathlineto{\pgfqpoint{1.101762in}{0.709423in}}%
\pgfpathlineto{\pgfqpoint{1.102629in}{0.698335in}}%
\pgfpathlineto{\pgfqpoint{1.103497in}{0.712587in}}%
\pgfpathlineto{\pgfqpoint{1.104364in}{0.698230in}}%
\pgfpathlineto{\pgfqpoint{1.105231in}{0.716999in}}%
\pgfpathlineto{\pgfqpoint{1.106098in}{0.699850in}}%
\pgfpathlineto{\pgfqpoint{1.106965in}{0.718571in}}%
\pgfpathlineto{\pgfqpoint{1.107832in}{0.704653in}}%
\pgfpathlineto{\pgfqpoint{1.108699in}{0.718594in}}%
\pgfpathlineto{\pgfqpoint{1.109566in}{0.710878in}}%
\pgfpathlineto{\pgfqpoint{1.111301in}{0.719139in}}%
\pgfpathlineto{\pgfqpoint{1.112168in}{0.715796in}}%
\pgfpathlineto{\pgfqpoint{1.113035in}{0.725913in}}%
\pgfpathlineto{\pgfqpoint{1.113902in}{0.715170in}}%
\pgfpathlineto{\pgfqpoint{1.114769in}{0.730324in}}%
\pgfpathlineto{\pgfqpoint{1.115636in}{0.718908in}}%
\pgfpathlineto{\pgfqpoint{1.116503in}{0.734567in}}%
\pgfpathlineto{\pgfqpoint{1.117371in}{0.726593in}}%
\pgfpathlineto{\pgfqpoint{1.119105in}{0.736192in}}%
\pgfpathlineto{\pgfqpoint{1.119972in}{0.732604in}}%
\pgfpathlineto{\pgfqpoint{1.120839in}{0.742623in}}%
\pgfpathlineto{\pgfqpoint{1.121706in}{0.735360in}}%
\pgfpathlineto{\pgfqpoint{1.122573in}{0.746186in}}%
\pgfpathlineto{\pgfqpoint{1.123441in}{0.742236in}}%
\pgfpathlineto{\pgfqpoint{1.124308in}{0.744830in}}%
\pgfpathlineto{\pgfqpoint{1.125175in}{0.751637in}}%
\pgfpathlineto{\pgfqpoint{1.126042in}{0.739973in}}%
\pgfpathlineto{\pgfqpoint{1.126909in}{0.764356in}}%
\pgfpathlineto{\pgfqpoint{1.127776in}{0.735116in}}%
\pgfpathlineto{\pgfqpoint{1.128643in}{0.771882in}}%
\pgfpathlineto{\pgfqpoint{1.129510in}{0.729318in}}%
\pgfpathlineto{\pgfqpoint{1.130378in}{0.769507in}}%
\pgfpathlineto{\pgfqpoint{1.131245in}{0.734994in}}%
\pgfpathlineto{\pgfqpoint{1.132112in}{0.760459in}}%
\pgfpathlineto{\pgfqpoint{1.133846in}{0.739827in}}%
\pgfpathlineto{\pgfqpoint{1.134713in}{0.755397in}}%
\pgfpathlineto{\pgfqpoint{1.135580in}{0.727128in}}%
\pgfpathlineto{\pgfqpoint{1.136448in}{0.757143in}}%
\pgfpathlineto{\pgfqpoint{1.137315in}{0.718178in}}%
\pgfpathlineto{\pgfqpoint{1.138182in}{0.745979in}}%
\pgfpathlineto{\pgfqpoint{1.139049in}{0.722047in}}%
\pgfpathlineto{\pgfqpoint{1.139916in}{0.731782in}}%
\pgfpathlineto{\pgfqpoint{1.141650in}{0.715545in}}%
\pgfpathlineto{\pgfqpoint{1.142517in}{0.732495in}}%
\pgfpathlineto{\pgfqpoint{1.143385in}{0.703532in}}%
\pgfpathlineto{\pgfqpoint{1.144252in}{0.729284in}}%
\pgfpathlineto{\pgfqpoint{1.145119in}{0.704892in}}%
\pgfpathlineto{\pgfqpoint{1.145986in}{0.719068in}}%
\pgfpathlineto{\pgfqpoint{1.146853in}{0.718696in}}%
\pgfpathlineto{\pgfqpoint{1.147720in}{0.703054in}}%
\pgfpathlineto{\pgfqpoint{1.148587in}{0.730859in}}%
\pgfpathlineto{\pgfqpoint{1.149455in}{0.696099in}}%
\pgfpathlineto{\pgfqpoint{1.150322in}{0.733911in}}%
\pgfpathlineto{\pgfqpoint{1.151189in}{0.704040in}}%
\pgfpathlineto{\pgfqpoint{1.152923in}{0.728086in}}%
\pgfpathlineto{\pgfqpoint{1.153790in}{0.707155in}}%
\pgfpathlineto{\pgfqpoint{1.154657in}{0.746932in}}%
\pgfpathlineto{\pgfqpoint{1.155524in}{0.699970in}}%
\pgfpathlineto{\pgfqpoint{1.156392in}{0.752036in}}%
\pgfpathlineto{\pgfqpoint{1.157259in}{0.708932in}}%
\pgfpathlineto{\pgfqpoint{1.158126in}{0.737264in}}%
\pgfpathlineto{\pgfqpoint{1.159860in}{0.716892in}}%
\pgfpathlineto{\pgfqpoint{1.160727in}{0.744343in}}%
\pgfpathlineto{\pgfqpoint{1.161594in}{0.701073in}}%
\pgfpathlineto{\pgfqpoint{1.162462in}{0.753847in}}%
\pgfpathlineto{\pgfqpoint{1.163329in}{0.696172in}}%
\pgfpathlineto{\pgfqpoint{1.164196in}{0.744170in}}%
\pgfpathlineto{\pgfqpoint{1.165063in}{0.701040in}}%
\pgfpathlineto{\pgfqpoint{1.165930in}{0.733985in}}%
\pgfpathlineto{\pgfqpoint{1.166797in}{0.711270in}}%
\pgfpathlineto{\pgfqpoint{1.167664in}{0.719159in}}%
\pgfpathlineto{\pgfqpoint{1.168531in}{0.711534in}}%
\pgfpathlineto{\pgfqpoint{1.169399in}{0.718719in}}%
\pgfpathlineto{\pgfqpoint{1.170266in}{0.711720in}}%
\pgfpathlineto{\pgfqpoint{1.171133in}{0.724971in}}%
\pgfpathlineto{\pgfqpoint{1.172000in}{0.718316in}}%
\pgfpathlineto{\pgfqpoint{1.173734in}{0.733607in}}%
\pgfpathlineto{\pgfqpoint{1.174601in}{0.719093in}}%
\pgfpathlineto{\pgfqpoint{1.175469in}{0.756158in}}%
\pgfpathlineto{\pgfqpoint{1.176336in}{0.706643in}}%
\pgfpathlineto{\pgfqpoint{1.177203in}{0.788045in}}%
\pgfpathlineto{\pgfqpoint{1.178070in}{0.698069in}}%
\pgfpathlineto{\pgfqpoint{1.178937in}{0.792781in}}%
\pgfpathlineto{\pgfqpoint{1.179804in}{0.706027in}}%
\pgfpathlineto{\pgfqpoint{1.180671in}{0.777488in}}%
\pgfpathlineto{\pgfqpoint{1.181538in}{0.732872in}}%
\pgfpathlineto{\pgfqpoint{1.182406in}{0.742353in}}%
\pgfpathlineto{\pgfqpoint{1.183273in}{0.757955in}}%
\pgfpathlineto{\pgfqpoint{1.184140in}{0.711005in}}%
\pgfpathlineto{\pgfqpoint{1.185007in}{0.766441in}}%
\pgfpathlineto{\pgfqpoint{1.185874in}{0.697623in}}%
\pgfpathlineto{\pgfqpoint{1.186741in}{0.752254in}}%
\pgfpathlineto{\pgfqpoint{1.187608in}{0.700055in}}%
\pgfpathlineto{\pgfqpoint{1.188476in}{0.731839in}}%
\pgfpathlineto{\pgfqpoint{1.190210in}{0.707356in}}%
\pgfpathlineto{\pgfqpoint{1.191077in}{0.712597in}}%
\pgfpathlineto{\pgfqpoint{1.191944in}{0.697592in}}%
\pgfpathlineto{\pgfqpoint{1.192811in}{0.706304in}}%
\pgfpathlineto{\pgfqpoint{1.193678in}{0.696596in}}%
\pgfpathlineto{\pgfqpoint{1.194545in}{0.702412in}}%
\pgfpathlineto{\pgfqpoint{1.195413in}{0.697743in}}%
\pgfpathlineto{\pgfqpoint{1.196280in}{0.701197in}}%
\pgfpathlineto{\pgfqpoint{1.197147in}{0.699057in}}%
\pgfpathlineto{\pgfqpoint{1.198881in}{0.704486in}}%
\pgfpathlineto{\pgfqpoint{1.199748in}{0.702422in}}%
\pgfpathlineto{\pgfqpoint{1.200615in}{0.711520in}}%
\pgfpathlineto{\pgfqpoint{1.201483in}{0.709791in}}%
\pgfpathlineto{\pgfqpoint{1.203217in}{0.720878in}}%
\pgfpathlineto{\pgfqpoint{1.204084in}{0.712492in}}%
\pgfpathlineto{\pgfqpoint{1.204951in}{0.728852in}}%
\pgfpathlineto{\pgfqpoint{1.205818in}{0.725352in}}%
\pgfpathlineto{\pgfqpoint{1.206685in}{0.715946in}}%
\pgfpathlineto{\pgfqpoint{1.207552in}{0.743489in}}%
\pgfpathlineto{\pgfqpoint{1.208420in}{0.701728in}}%
\pgfpathlineto{\pgfqpoint{1.209287in}{0.752952in}}%
\pgfpathlineto{\pgfqpoint{1.210154in}{0.701355in}}%
\pgfpathlineto{\pgfqpoint{1.211021in}{0.741600in}}%
\pgfpathlineto{\pgfqpoint{1.212755in}{0.714115in}}%
\pgfpathlineto{\pgfqpoint{1.213622in}{0.736408in}}%
\pgfpathlineto{\pgfqpoint{1.214490in}{0.697967in}}%
\pgfpathlineto{\pgfqpoint{1.215357in}{0.730807in}}%
\pgfpathlineto{\pgfqpoint{1.216224in}{0.698646in}}%
\pgfpathlineto{\pgfqpoint{1.217091in}{0.714552in}}%
\pgfpathlineto{\pgfqpoint{1.217958in}{0.702599in}}%
\pgfpathlineto{\pgfqpoint{1.218825in}{0.702951in}}%
\pgfpathlineto{\pgfqpoint{1.219692in}{0.700485in}}%
\pgfpathlineto{\pgfqpoint{1.220559in}{0.701915in}}%
\pgfpathlineto{\pgfqpoint{1.221427in}{0.699392in}}%
\pgfpathlineto{\pgfqpoint{1.222294in}{0.704629in}}%
\pgfpathlineto{\pgfqpoint{1.223161in}{0.699954in}}%
\pgfpathlineto{\pgfqpoint{1.224028in}{0.710197in}}%
\pgfpathlineto{\pgfqpoint{1.224895in}{0.702502in}}%
\pgfpathlineto{\pgfqpoint{1.226629in}{0.722145in}}%
\pgfpathlineto{\pgfqpoint{1.227497in}{0.708354in}}%
\pgfpathlineto{\pgfqpoint{1.228364in}{0.756365in}}%
\pgfpathlineto{\pgfqpoint{1.229231in}{0.696257in}}%
\pgfpathlineto{\pgfqpoint{1.230098in}{0.776348in}}%
\pgfpathlineto{\pgfqpoint{1.230965in}{0.709886in}}%
\pgfpathlineto{\pgfqpoint{1.231832in}{0.758620in}}%
\pgfpathlineto{\pgfqpoint{1.232699in}{0.757283in}}%
\pgfpathlineto{\pgfqpoint{1.233566in}{0.715356in}}%
\pgfpathlineto{\pgfqpoint{1.234434in}{0.774459in}}%
\pgfpathlineto{\pgfqpoint{1.235301in}{0.704281in}}%
\pgfpathlineto{\pgfqpoint{1.236168in}{0.764596in}}%
\pgfpathlineto{\pgfqpoint{1.237902in}{0.719372in}}%
\pgfpathlineto{\pgfqpoint{1.238769in}{0.732944in}}%
\pgfpathlineto{\pgfqpoint{1.239636in}{0.700700in}}%
\pgfpathlineto{\pgfqpoint{1.240503in}{0.714178in}}%
\pgfpathlineto{\pgfqpoint{1.242238in}{0.698422in}}%
\pgfpathlineto{\pgfqpoint{1.243105in}{0.700030in}}%
\pgfpathlineto{\pgfqpoint{1.244839in}{0.696265in}}%
\pgfpathlineto{\pgfqpoint{1.245706in}{0.698725in}}%
\pgfpathlineto{\pgfqpoint{1.246573in}{0.704766in}}%
\pgfpathlineto{\pgfqpoint{1.247441in}{0.702105in}}%
\pgfpathlineto{\pgfqpoint{1.249175in}{0.714380in}}%
\pgfpathlineto{\pgfqpoint{1.250042in}{0.706430in}}%
\pgfpathlineto{\pgfqpoint{1.250909in}{0.727073in}}%
\pgfpathlineto{\pgfqpoint{1.251776in}{0.703661in}}%
\pgfpathlineto{\pgfqpoint{1.252643in}{0.729230in}}%
\pgfpathlineto{\pgfqpoint{1.253510in}{0.711478in}}%
\pgfpathlineto{\pgfqpoint{1.254378in}{0.712700in}}%
\pgfpathlineto{\pgfqpoint{1.255245in}{0.716136in}}%
\pgfpathlineto{\pgfqpoint{1.256112in}{0.702165in}}%
\pgfpathlineto{\pgfqpoint{1.256979in}{0.720932in}}%
\pgfpathlineto{\pgfqpoint{1.257846in}{0.696267in}}%
\pgfpathlineto{\pgfqpoint{1.258713in}{0.710050in}}%
\pgfpathlineto{\pgfqpoint{1.260448in}{0.700506in}}%
\pgfpathlineto{\pgfqpoint{1.261315in}{0.707659in}}%
\pgfpathlineto{\pgfqpoint{1.262182in}{0.696902in}}%
\pgfpathlineto{\pgfqpoint{1.263049in}{0.708263in}}%
\pgfpathlineto{\pgfqpoint{1.263916in}{0.705060in}}%
\pgfpathlineto{\pgfqpoint{1.264783in}{0.699207in}}%
\pgfpathlineto{\pgfqpoint{1.265650in}{0.708224in}}%
\pgfpathlineto{\pgfqpoint{1.266517in}{0.696899in}}%
\pgfpathlineto{\pgfqpoint{1.267385in}{0.702995in}}%
\pgfpathlineto{\pgfqpoint{1.268252in}{0.697803in}}%
\pgfpathlineto{\pgfqpoint{1.269119in}{0.699578in}}%
\pgfpathlineto{\pgfqpoint{1.269986in}{0.696876in}}%
\pgfpathlineto{\pgfqpoint{1.271720in}{0.711200in}}%
\pgfpathlineto{\pgfqpoint{1.272587in}{0.703300in}}%
\pgfpathlineto{\pgfqpoint{1.274322in}{0.729578in}}%
\pgfpathlineto{\pgfqpoint{1.275189in}{0.715604in}}%
\pgfpathlineto{\pgfqpoint{1.276056in}{0.782779in}}%
\pgfpathlineto{\pgfqpoint{1.276923in}{0.696622in}}%
\pgfpathlineto{\pgfqpoint{1.277790in}{0.796692in}}%
\pgfpathlineto{\pgfqpoint{1.278657in}{0.730501in}}%
\pgfpathlineto{\pgfqpoint{1.279524in}{0.743612in}}%
\pgfpathlineto{\pgfqpoint{1.280392in}{0.781471in}}%
\pgfpathlineto{\pgfqpoint{1.281259in}{0.703359in}}%
\pgfpathlineto{\pgfqpoint{1.282126in}{0.768294in}}%
\pgfpathlineto{\pgfqpoint{1.282993in}{0.717640in}}%
\pgfpathlineto{\pgfqpoint{1.283860in}{0.721604in}}%
\pgfpathlineto{\pgfqpoint{1.284727in}{0.731043in}}%
\pgfpathlineto{\pgfqpoint{1.285594in}{0.705583in}}%
\pgfpathlineto{\pgfqpoint{1.286462in}{0.716763in}}%
\pgfpathlineto{\pgfqpoint{1.288196in}{0.702248in}}%
\pgfpathlineto{\pgfqpoint{1.289063in}{0.702206in}}%
\pgfpathlineto{\pgfqpoint{1.289930in}{0.697145in}}%
\pgfpathlineto{\pgfqpoint{1.290797in}{0.698367in}}%
\pgfpathlineto{\pgfqpoint{1.291664in}{0.698192in}}%
\pgfpathlineto{\pgfqpoint{1.292531in}{0.702007in}}%
\pgfpathlineto{\pgfqpoint{1.293399in}{0.714889in}}%
\pgfpathlineto{\pgfqpoint{1.294266in}{0.700319in}}%
\pgfpathlineto{\pgfqpoint{1.295133in}{0.733733in}}%
\pgfpathlineto{\pgfqpoint{1.296000in}{0.713320in}}%
\pgfpathlineto{\pgfqpoint{1.297734in}{0.758800in}}%
\pgfpathlineto{\pgfqpoint{1.298601in}{0.697102in}}%
\pgfpathlineto{\pgfqpoint{1.299469in}{0.760014in}}%
\pgfpathlineto{\pgfqpoint{1.300336in}{0.710061in}}%
\pgfpathlineto{\pgfqpoint{1.301203in}{0.717360in}}%
\pgfpathlineto{\pgfqpoint{1.302070in}{0.723702in}}%
\pgfpathlineto{\pgfqpoint{1.303804in}{0.698300in}}%
\pgfpathlineto{\pgfqpoint{1.304671in}{0.708058in}}%
\pgfpathlineto{\pgfqpoint{1.305538in}{0.701906in}}%
\pgfpathlineto{\pgfqpoint{1.306406in}{0.707184in}}%
\pgfpathlineto{\pgfqpoint{1.307273in}{0.698309in}}%
\pgfpathlineto{\pgfqpoint{1.308140in}{0.729850in}}%
\pgfpathlineto{\pgfqpoint{1.309874in}{0.709106in}}%
\pgfpathlineto{\pgfqpoint{1.310741in}{0.757941in}}%
\pgfpathlineto{\pgfqpoint{1.311608in}{0.716062in}}%
\pgfpathlineto{\pgfqpoint{1.312476in}{0.727034in}}%
\pgfpathlineto{\pgfqpoint{1.313343in}{0.803540in}}%
\pgfpathlineto{\pgfqpoint{1.314210in}{0.707397in}}%
\pgfpathlineto{\pgfqpoint{1.315077in}{0.797997in}}%
\pgfpathlineto{\pgfqpoint{1.315944in}{0.786843in}}%
\pgfpathlineto{\pgfqpoint{1.316811in}{0.753780in}}%
\pgfpathlineto{\pgfqpoint{1.317678in}{0.797637in}}%
\pgfpathlineto{\pgfqpoint{1.320280in}{0.741964in}}%
\pgfpathlineto{\pgfqpoint{1.321147in}{0.769855in}}%
\pgfpathlineto{\pgfqpoint{1.322014in}{0.702069in}}%
\pgfpathlineto{\pgfqpoint{1.322881in}{0.724183in}}%
\pgfpathlineto{\pgfqpoint{1.325483in}{0.701543in}}%
\pgfpathlineto{\pgfqpoint{1.326350in}{0.722741in}}%
\pgfpathlineto{\pgfqpoint{1.328084in}{0.701503in}}%
\pgfpathlineto{\pgfqpoint{1.328951in}{0.732873in}}%
\pgfpathlineto{\pgfqpoint{1.329818in}{0.702210in}}%
\pgfpathlineto{\pgfqpoint{1.330685in}{0.704912in}}%
\pgfpathlineto{\pgfqpoint{1.331552in}{0.711705in}}%
\pgfpathlineto{\pgfqpoint{1.333287in}{0.699039in}}%
\pgfpathlineto{\pgfqpoint{1.335021in}{0.731995in}}%
\pgfpathlineto{\pgfqpoint{1.335888in}{0.698818in}}%
\pgfpathlineto{\pgfqpoint{1.336755in}{0.774044in}}%
\pgfpathlineto{\pgfqpoint{1.338490in}{0.715314in}}%
\pgfpathlineto{\pgfqpoint{1.339357in}{0.781647in}}%
\pgfpathlineto{\pgfqpoint{1.340224in}{0.708428in}}%
\pgfpathlineto{\pgfqpoint{1.341091in}{0.727790in}}%
\pgfpathlineto{\pgfqpoint{1.341958in}{0.757279in}}%
\pgfpathlineto{\pgfqpoint{1.342825in}{0.707629in}}%
\pgfpathlineto{\pgfqpoint{1.344559in}{0.733525in}}%
\pgfpathlineto{\pgfqpoint{1.346294in}{0.712644in}}%
\pgfpathlineto{\pgfqpoint{1.347161in}{0.717801in}}%
\pgfpathlineto{\pgfqpoint{1.348028in}{0.700897in}}%
\pgfpathlineto{\pgfqpoint{1.348895in}{0.702955in}}%
\pgfpathlineto{\pgfqpoint{1.351497in}{0.696884in}}%
\pgfpathlineto{\pgfqpoint{1.353231in}{0.742450in}}%
\pgfpathlineto{\pgfqpoint{1.354098in}{0.696507in}}%
\pgfpathlineto{\pgfqpoint{1.354965in}{0.807921in}}%
\pgfpathlineto{\pgfqpoint{1.355832in}{0.798689in}}%
\pgfpathlineto{\pgfqpoint{1.356699in}{0.707943in}}%
\pgfpathlineto{\pgfqpoint{1.357566in}{0.949909in}}%
\pgfpathlineto{\pgfqpoint{1.359301in}{0.768036in}}%
\pgfpathlineto{\pgfqpoint{1.360168in}{1.055632in}}%
\pgfpathlineto{\pgfqpoint{1.361035in}{0.773039in}}%
\pgfpathlineto{\pgfqpoint{1.362769in}{0.951300in}}%
\pgfpathlineto{\pgfqpoint{1.363636in}{0.729700in}}%
\pgfpathlineto{\pgfqpoint{1.364503in}{0.832977in}}%
\pgfpathlineto{\pgfqpoint{1.366238in}{0.729156in}}%
\pgfpathlineto{\pgfqpoint{1.367105in}{0.732830in}}%
\pgfpathlineto{\pgfqpoint{1.367972in}{0.696236in}}%
\pgfpathlineto{\pgfqpoint{1.369706in}{0.738642in}}%
\pgfpathlineto{\pgfqpoint{1.370573in}{0.698359in}}%
\pgfpathlineto{\pgfqpoint{1.371441in}{0.749320in}}%
\pgfpathlineto{\pgfqpoint{1.372308in}{0.741237in}}%
\pgfpathlineto{\pgfqpoint{1.373175in}{0.722172in}}%
\pgfpathlineto{\pgfqpoint{1.374042in}{0.722296in}}%
\pgfpathlineto{\pgfqpoint{1.374909in}{0.738231in}}%
\pgfpathlineto{\pgfqpoint{1.376643in}{0.705360in}}%
\pgfpathlineto{\pgfqpoint{1.378378in}{0.737112in}}%
\pgfpathlineto{\pgfqpoint{1.379245in}{0.721345in}}%
\pgfpathlineto{\pgfqpoint{1.380112in}{0.722302in}}%
\pgfpathlineto{\pgfqpoint{1.380979in}{0.807504in}}%
\pgfpathlineto{\pgfqpoint{1.381846in}{0.721134in}}%
\pgfpathlineto{\pgfqpoint{1.383580in}{0.802613in}}%
\pgfpathlineto{\pgfqpoint{1.385315in}{0.745517in}}%
\pgfpathlineto{\pgfqpoint{1.386182in}{0.788641in}}%
\pgfpathlineto{\pgfqpoint{1.387049in}{0.702073in}}%
\pgfpathlineto{\pgfqpoint{1.387916in}{0.755780in}}%
\pgfpathlineto{\pgfqpoint{1.388783in}{0.705458in}}%
\pgfpathlineto{\pgfqpoint{1.389650in}{0.771028in}}%
\pgfpathlineto{\pgfqpoint{1.390517in}{0.759865in}}%
\pgfpathlineto{\pgfqpoint{1.391385in}{0.747804in}}%
\pgfpathlineto{\pgfqpoint{1.392252in}{0.870894in}}%
\pgfpathlineto{\pgfqpoint{1.393119in}{0.726468in}}%
\pgfpathlineto{\pgfqpoint{1.393986in}{0.794538in}}%
\pgfpathlineto{\pgfqpoint{1.394853in}{0.777926in}}%
\pgfpathlineto{\pgfqpoint{1.395720in}{0.696228in}}%
\pgfpathlineto{\pgfqpoint{1.396587in}{0.723565in}}%
\pgfpathlineto{\pgfqpoint{1.397455in}{0.700682in}}%
\pgfpathlineto{\pgfqpoint{1.398322in}{0.776885in}}%
\pgfpathlineto{\pgfqpoint{1.399189in}{0.742573in}}%
\pgfpathlineto{\pgfqpoint{1.400056in}{0.758002in}}%
\pgfpathlineto{\pgfqpoint{1.400923in}{0.882054in}}%
\pgfpathlineto{\pgfqpoint{1.401790in}{0.743403in}}%
\pgfpathlineto{\pgfqpoint{1.402657in}{0.802816in}}%
\pgfpathlineto{\pgfqpoint{1.405259in}{0.727239in}}%
\pgfpathlineto{\pgfqpoint{1.406126in}{0.752455in}}%
\pgfpathlineto{\pgfqpoint{1.407860in}{0.726586in}}%
\pgfpathlineto{\pgfqpoint{1.408727in}{0.806586in}}%
\pgfpathlineto{\pgfqpoint{1.409594in}{0.717260in}}%
\pgfpathlineto{\pgfqpoint{1.411329in}{0.772338in}}%
\pgfpathlineto{\pgfqpoint{1.412196in}{0.756580in}}%
\pgfpathlineto{\pgfqpoint{1.413063in}{0.891767in}}%
\pgfpathlineto{\pgfqpoint{1.413930in}{0.697814in}}%
\pgfpathlineto{\pgfqpoint{1.414797in}{1.007606in}}%
\pgfpathlineto{\pgfqpoint{1.416531in}{0.755648in}}%
\pgfpathlineto{\pgfqpoint{1.417399in}{0.945482in}}%
\pgfpathlineto{\pgfqpoint{1.418266in}{0.771771in}}%
\pgfpathlineto{\pgfqpoint{1.419133in}{0.775617in}}%
\pgfpathlineto{\pgfqpoint{1.420000in}{0.711430in}}%
\pgfpathlineto{\pgfqpoint{1.422601in}{0.746299in}}%
\pgfpathlineto{\pgfqpoint{1.423469in}{0.735243in}}%
\pgfpathlineto{\pgfqpoint{1.424336in}{0.764788in}}%
\pgfpathlineto{\pgfqpoint{1.425203in}{0.936446in}}%
\pgfpathlineto{\pgfqpoint{1.426070in}{0.791335in}}%
\pgfpathlineto{\pgfqpoint{1.427804in}{1.170775in}}%
\pgfpathlineto{\pgfqpoint{1.428671in}{0.820219in}}%
\pgfpathlineto{\pgfqpoint{1.429538in}{1.175162in}}%
\pgfpathlineto{\pgfqpoint{1.430406in}{1.136527in}}%
\pgfpathlineto{\pgfqpoint{1.431273in}{0.779922in}}%
\pgfpathlineto{\pgfqpoint{1.433007in}{1.021724in}}%
\pgfpathlineto{\pgfqpoint{1.434741in}{0.742853in}}%
\pgfpathlineto{\pgfqpoint{1.435608in}{1.015233in}}%
\pgfpathlineto{\pgfqpoint{1.437343in}{0.731667in}}%
\pgfpathlineto{\pgfqpoint{1.438210in}{0.953895in}}%
\pgfpathlineto{\pgfqpoint{1.439944in}{0.766519in}}%
\pgfpathlineto{\pgfqpoint{1.440811in}{0.780768in}}%
\pgfpathlineto{\pgfqpoint{1.442545in}{0.721853in}}%
\pgfpathlineto{\pgfqpoint{1.443413in}{0.744715in}}%
\pgfpathlineto{\pgfqpoint{1.444280in}{0.701996in}}%
\pgfpathlineto{\pgfqpoint{1.445147in}{0.767181in}}%
\pgfpathlineto{\pgfqpoint{1.446014in}{0.710662in}}%
\pgfpathlineto{\pgfqpoint{1.447748in}{0.836981in}}%
\pgfpathlineto{\pgfqpoint{1.448615in}{0.701442in}}%
\pgfpathlineto{\pgfqpoint{1.449483in}{0.875637in}}%
\pgfpathlineto{\pgfqpoint{1.451217in}{0.697073in}}%
\pgfpathlineto{\pgfqpoint{1.452084in}{0.723670in}}%
\pgfpathlineto{\pgfqpoint{1.453818in}{0.823260in}}%
\pgfpathlineto{\pgfqpoint{1.454685in}{0.709028in}}%
\pgfpathlineto{\pgfqpoint{1.456420in}{0.952005in}}%
\pgfpathlineto{\pgfqpoint{1.457287in}{0.712108in}}%
\pgfpathlineto{\pgfqpoint{1.458154in}{0.810193in}}%
\pgfpathlineto{\pgfqpoint{1.459888in}{0.725726in}}%
\pgfpathlineto{\pgfqpoint{1.460755in}{0.747784in}}%
\pgfpathlineto{\pgfqpoint{1.461622in}{0.699126in}}%
\pgfpathlineto{\pgfqpoint{1.462490in}{0.756471in}}%
\pgfpathlineto{\pgfqpoint{1.463357in}{0.704226in}}%
\pgfpathlineto{\pgfqpoint{1.466825in}{0.824172in}}%
\pgfpathlineto{\pgfqpoint{1.467692in}{0.733186in}}%
\pgfpathlineto{\pgfqpoint{1.468559in}{0.917277in}}%
\pgfpathlineto{\pgfqpoint{1.470294in}{0.752039in}}%
\pgfpathlineto{\pgfqpoint{1.471161in}{0.789769in}}%
\pgfpathlineto{\pgfqpoint{1.472028in}{0.700533in}}%
\pgfpathlineto{\pgfqpoint{1.472895in}{0.704224in}}%
\pgfpathlineto{\pgfqpoint{1.474629in}{0.726012in}}%
\pgfpathlineto{\pgfqpoint{1.475497in}{0.721818in}}%
\pgfpathlineto{\pgfqpoint{1.476364in}{0.753689in}}%
\pgfpathlineto{\pgfqpoint{1.477231in}{0.709223in}}%
\pgfpathlineto{\pgfqpoint{1.478098in}{0.758182in}}%
\pgfpathlineto{\pgfqpoint{1.478965in}{0.706197in}}%
\pgfpathlineto{\pgfqpoint{1.480699in}{0.743539in}}%
\pgfpathlineto{\pgfqpoint{1.481566in}{0.733462in}}%
\pgfpathlineto{\pgfqpoint{1.482434in}{0.759703in}}%
\pgfpathlineto{\pgfqpoint{1.483301in}{0.743107in}}%
\pgfpathlineto{\pgfqpoint{1.484168in}{0.754329in}}%
\pgfpathlineto{\pgfqpoint{1.485035in}{0.704852in}}%
\pgfpathlineto{\pgfqpoint{1.485902in}{0.718114in}}%
\pgfpathlineto{\pgfqpoint{1.487636in}{0.757242in}}%
\pgfpathlineto{\pgfqpoint{1.488503in}{0.737360in}}%
\pgfpathlineto{\pgfqpoint{1.490238in}{0.817808in}}%
\pgfpathlineto{\pgfqpoint{1.491105in}{0.718924in}}%
\pgfpathlineto{\pgfqpoint{1.491972in}{0.754957in}}%
\pgfpathlineto{\pgfqpoint{1.492839in}{0.745359in}}%
\pgfpathlineto{\pgfqpoint{1.493706in}{0.703990in}}%
\pgfpathlineto{\pgfqpoint{1.494573in}{0.777861in}}%
\pgfpathlineto{\pgfqpoint{1.495441in}{0.700638in}}%
\pgfpathlineto{\pgfqpoint{1.496308in}{0.784008in}}%
\pgfpathlineto{\pgfqpoint{1.497175in}{0.763392in}}%
\pgfpathlineto{\pgfqpoint{1.498042in}{0.741255in}}%
\pgfpathlineto{\pgfqpoint{1.498909in}{0.779023in}}%
\pgfpathlineto{\pgfqpoint{1.499776in}{0.749725in}}%
\pgfpathlineto{\pgfqpoint{1.500643in}{0.824632in}}%
\pgfpathlineto{\pgfqpoint{1.501510in}{0.707469in}}%
\pgfpathlineto{\pgfqpoint{1.502378in}{0.924179in}}%
\pgfpathlineto{\pgfqpoint{1.503245in}{0.708574in}}%
\pgfpathlineto{\pgfqpoint{1.504112in}{0.944104in}}%
\pgfpathlineto{\pgfqpoint{1.505846in}{0.762420in}}%
\pgfpathlineto{\pgfqpoint{1.506713in}{0.975973in}}%
\pgfpathlineto{\pgfqpoint{1.507580in}{0.725474in}}%
\pgfpathlineto{\pgfqpoint{1.509315in}{0.803569in}}%
\pgfpathlineto{\pgfqpoint{1.510182in}{0.721283in}}%
\pgfpathlineto{\pgfqpoint{1.511049in}{0.731699in}}%
\pgfpathlineto{\pgfqpoint{1.511916in}{0.858652in}}%
\pgfpathlineto{\pgfqpoint{1.512783in}{0.746745in}}%
\pgfpathlineto{\pgfqpoint{1.514517in}{0.946632in}}%
\pgfpathlineto{\pgfqpoint{1.515385in}{0.786573in}}%
\pgfpathlineto{\pgfqpoint{1.516252in}{0.801383in}}%
\pgfpathlineto{\pgfqpoint{1.517119in}{0.780447in}}%
\pgfpathlineto{\pgfqpoint{1.518853in}{0.699310in}}%
\pgfpathlineto{\pgfqpoint{1.520587in}{0.739279in}}%
\pgfpathlineto{\pgfqpoint{1.521455in}{0.703450in}}%
\pgfpathlineto{\pgfqpoint{1.522322in}{0.793992in}}%
\pgfpathlineto{\pgfqpoint{1.524923in}{0.696000in}}%
\pgfpathlineto{\pgfqpoint{1.526657in}{0.746646in}}%
\pgfpathlineto{\pgfqpoint{1.527524in}{0.743869in}}%
\pgfpathlineto{\pgfqpoint{1.529259in}{0.698314in}}%
\pgfpathlineto{\pgfqpoint{1.530993in}{0.741906in}}%
\pgfpathlineto{\pgfqpoint{1.531860in}{0.748196in}}%
\pgfpathlineto{\pgfqpoint{1.532727in}{0.796401in}}%
\pgfpathlineto{\pgfqpoint{1.533594in}{0.697878in}}%
\pgfpathlineto{\pgfqpoint{1.534462in}{0.756690in}}%
\pgfpathlineto{\pgfqpoint{1.535329in}{0.701216in}}%
\pgfpathlineto{\pgfqpoint{1.537063in}{0.816659in}}%
\pgfpathlineto{\pgfqpoint{1.537930in}{0.787872in}}%
\pgfpathlineto{\pgfqpoint{1.538797in}{0.858602in}}%
\pgfpathlineto{\pgfqpoint{1.539664in}{0.705864in}}%
\pgfpathlineto{\pgfqpoint{1.540531in}{0.766469in}}%
\pgfpathlineto{\pgfqpoint{1.541399in}{0.732052in}}%
\pgfpathlineto{\pgfqpoint{1.542266in}{0.759775in}}%
\pgfpathlineto{\pgfqpoint{1.544000in}{0.701437in}}%
\pgfpathlineto{\pgfqpoint{1.544867in}{0.751629in}}%
\pgfpathlineto{\pgfqpoint{1.545734in}{0.725296in}}%
\pgfpathlineto{\pgfqpoint{1.546601in}{0.747538in}}%
\pgfpathlineto{\pgfqpoint{1.547469in}{0.858112in}}%
\pgfpathlineto{\pgfqpoint{1.548336in}{0.823564in}}%
\pgfpathlineto{\pgfqpoint{1.549203in}{0.848784in}}%
\pgfpathlineto{\pgfqpoint{1.550937in}{0.696614in}}%
\pgfpathlineto{\pgfqpoint{1.552671in}{0.892973in}}%
\pgfpathlineto{\pgfqpoint{1.553538in}{0.832985in}}%
\pgfpathlineto{\pgfqpoint{1.554406in}{0.697585in}}%
\pgfpathlineto{\pgfqpoint{1.555273in}{0.758679in}}%
\pgfpathlineto{\pgfqpoint{1.556140in}{0.740412in}}%
\pgfpathlineto{\pgfqpoint{1.557007in}{0.698306in}}%
\pgfpathlineto{\pgfqpoint{1.558741in}{0.885839in}}%
\pgfpathlineto{\pgfqpoint{1.559608in}{0.707825in}}%
\pgfpathlineto{\pgfqpoint{1.560476in}{0.960189in}}%
\pgfpathlineto{\pgfqpoint{1.562210in}{0.742609in}}%
\pgfpathlineto{\pgfqpoint{1.563077in}{0.837847in}}%
\pgfpathlineto{\pgfqpoint{1.563944in}{0.816567in}}%
\pgfpathlineto{\pgfqpoint{1.564811in}{0.873621in}}%
\pgfpathlineto{\pgfqpoint{1.565678in}{0.840219in}}%
\pgfpathlineto{\pgfqpoint{1.566545in}{1.144328in}}%
\pgfpathlineto{\pgfqpoint{1.568280in}{0.781031in}}%
\pgfpathlineto{\pgfqpoint{1.569147in}{0.969178in}}%
\pgfpathlineto{\pgfqpoint{1.570881in}{0.719008in}}%
\pgfpathlineto{\pgfqpoint{1.571748in}{0.777483in}}%
\pgfpathlineto{\pgfqpoint{1.572615in}{0.725083in}}%
\pgfpathlineto{\pgfqpoint{1.573483in}{0.954092in}}%
\pgfpathlineto{\pgfqpoint{1.574350in}{0.801780in}}%
\pgfpathlineto{\pgfqpoint{1.575217in}{1.037520in}}%
\pgfpathlineto{\pgfqpoint{1.576951in}{0.869350in}}%
\pgfpathlineto{\pgfqpoint{1.577818in}{0.914579in}}%
\pgfpathlineto{\pgfqpoint{1.578685in}{0.703888in}}%
\pgfpathlineto{\pgfqpoint{1.579552in}{0.808711in}}%
\pgfpathlineto{\pgfqpoint{1.580420in}{0.756704in}}%
\pgfpathlineto{\pgfqpoint{1.581287in}{1.014376in}}%
\pgfpathlineto{\pgfqpoint{1.582154in}{0.949535in}}%
\pgfpathlineto{\pgfqpoint{1.583021in}{0.947609in}}%
\pgfpathlineto{\pgfqpoint{1.583888in}{0.844253in}}%
\pgfpathlineto{\pgfqpoint{1.584755in}{0.876837in}}%
\pgfpathlineto{\pgfqpoint{1.585622in}{0.863414in}}%
\pgfpathlineto{\pgfqpoint{1.586490in}{0.725767in}}%
\pgfpathlineto{\pgfqpoint{1.587357in}{0.748422in}}%
\pgfpathlineto{\pgfqpoint{1.588224in}{0.839459in}}%
\pgfpathlineto{\pgfqpoint{1.589958in}{0.702611in}}%
\pgfpathlineto{\pgfqpoint{1.590825in}{0.701525in}}%
\pgfpathlineto{\pgfqpoint{1.591692in}{0.776004in}}%
\pgfpathlineto{\pgfqpoint{1.592559in}{0.733473in}}%
\pgfpathlineto{\pgfqpoint{1.594294in}{0.845355in}}%
\pgfpathlineto{\pgfqpoint{1.595161in}{0.734085in}}%
\pgfpathlineto{\pgfqpoint{1.596028in}{0.734424in}}%
\pgfpathlineto{\pgfqpoint{1.596895in}{0.806935in}}%
\pgfpathlineto{\pgfqpoint{1.597762in}{0.700475in}}%
\pgfpathlineto{\pgfqpoint{1.598629in}{0.840345in}}%
\pgfpathlineto{\pgfqpoint{1.599497in}{0.800394in}}%
\pgfpathlineto{\pgfqpoint{1.600364in}{0.700996in}}%
\pgfpathlineto{\pgfqpoint{1.601231in}{0.720650in}}%
\pgfpathlineto{\pgfqpoint{1.602965in}{0.706992in}}%
\pgfpathlineto{\pgfqpoint{1.603832in}{0.735125in}}%
\pgfpathlineto{\pgfqpoint{1.605566in}{0.894846in}}%
\pgfpathlineto{\pgfqpoint{1.606434in}{0.711536in}}%
\pgfpathlineto{\pgfqpoint{1.608168in}{0.880462in}}%
\pgfpathlineto{\pgfqpoint{1.609035in}{0.717713in}}%
\pgfpathlineto{\pgfqpoint{1.610769in}{0.883590in}}%
\pgfpathlineto{\pgfqpoint{1.611636in}{0.705454in}}%
\pgfpathlineto{\pgfqpoint{1.613371in}{0.888896in}}%
\pgfpathlineto{\pgfqpoint{1.614238in}{0.711311in}}%
\pgfpathlineto{\pgfqpoint{1.615105in}{0.861589in}}%
\pgfpathlineto{\pgfqpoint{1.616839in}{0.748179in}}%
\pgfpathlineto{\pgfqpoint{1.618573in}{0.949724in}}%
\pgfpathlineto{\pgfqpoint{1.619441in}{0.894304in}}%
\pgfpathlineto{\pgfqpoint{1.620308in}{0.710601in}}%
\pgfpathlineto{\pgfqpoint{1.622042in}{1.090066in}}%
\pgfpathlineto{\pgfqpoint{1.623776in}{0.734882in}}%
\pgfpathlineto{\pgfqpoint{1.625510in}{0.819974in}}%
\pgfpathlineto{\pgfqpoint{1.627245in}{0.736628in}}%
\pgfpathlineto{\pgfqpoint{1.628979in}{0.707939in}}%
\pgfpathlineto{\pgfqpoint{1.630713in}{0.809617in}}%
\pgfpathlineto{\pgfqpoint{1.632448in}{0.707401in}}%
\pgfpathlineto{\pgfqpoint{1.633315in}{0.810117in}}%
\pgfpathlineto{\pgfqpoint{1.634182in}{0.722355in}}%
\pgfpathlineto{\pgfqpoint{1.635916in}{0.817335in}}%
\pgfpathlineto{\pgfqpoint{1.636783in}{0.699512in}}%
\pgfpathlineto{\pgfqpoint{1.637650in}{0.762899in}}%
\pgfpathlineto{\pgfqpoint{1.638517in}{0.702705in}}%
\pgfpathlineto{\pgfqpoint{1.639385in}{0.784675in}}%
\pgfpathlineto{\pgfqpoint{1.640252in}{0.736068in}}%
\pgfpathlineto{\pgfqpoint{1.641119in}{0.758942in}}%
\pgfpathlineto{\pgfqpoint{1.641986in}{0.815769in}}%
\pgfpathlineto{\pgfqpoint{1.642853in}{0.703290in}}%
\pgfpathlineto{\pgfqpoint{1.644587in}{0.854957in}}%
\pgfpathlineto{\pgfqpoint{1.645455in}{0.737639in}}%
\pgfpathlineto{\pgfqpoint{1.647189in}{0.837110in}}%
\pgfpathlineto{\pgfqpoint{1.648923in}{0.743263in}}%
\pgfpathlineto{\pgfqpoint{1.649790in}{0.757433in}}%
\pgfpathlineto{\pgfqpoint{1.650657in}{0.716534in}}%
\pgfpathlineto{\pgfqpoint{1.651524in}{0.717392in}}%
\pgfpathlineto{\pgfqpoint{1.652392in}{0.704821in}}%
\pgfpathlineto{\pgfqpoint{1.653259in}{0.747268in}}%
\pgfpathlineto{\pgfqpoint{1.654126in}{0.738590in}}%
\pgfpathlineto{\pgfqpoint{1.655860in}{0.704790in}}%
\pgfpathlineto{\pgfqpoint{1.656727in}{0.710672in}}%
\pgfpathlineto{\pgfqpoint{1.657594in}{0.746372in}}%
\pgfpathlineto{\pgfqpoint{1.658462in}{0.731485in}}%
\pgfpathlineto{\pgfqpoint{1.659329in}{0.825176in}}%
\pgfpathlineto{\pgfqpoint{1.661063in}{0.700431in}}%
\pgfpathlineto{\pgfqpoint{1.661930in}{0.800124in}}%
\pgfpathlineto{\pgfqpoint{1.662797in}{0.798439in}}%
\pgfpathlineto{\pgfqpoint{1.664531in}{0.700455in}}%
\pgfpathlineto{\pgfqpoint{1.665399in}{0.761843in}}%
\pgfpathlineto{\pgfqpoint{1.666266in}{0.704635in}}%
\pgfpathlineto{\pgfqpoint{1.668000in}{0.773794in}}%
\pgfpathlineto{\pgfqpoint{1.668867in}{0.715951in}}%
\pgfpathlineto{\pgfqpoint{1.669734in}{0.877101in}}%
\pgfpathlineto{\pgfqpoint{1.671469in}{0.725308in}}%
\pgfpathlineto{\pgfqpoint{1.673203in}{0.701319in}}%
\pgfpathlineto{\pgfqpoint{1.674937in}{0.767463in}}%
\pgfpathlineto{\pgfqpoint{1.675804in}{0.752353in}}%
\pgfpathlineto{\pgfqpoint{1.677538in}{0.708471in}}%
\pgfpathlineto{\pgfqpoint{1.678406in}{0.743901in}}%
\pgfpathlineto{\pgfqpoint{1.680140in}{0.908597in}}%
\pgfpathlineto{\pgfqpoint{1.681007in}{0.729228in}}%
\pgfpathlineto{\pgfqpoint{1.682741in}{1.242964in}}%
\pgfpathlineto{\pgfqpoint{1.684476in}{0.774730in}}%
\pgfpathlineto{\pgfqpoint{1.685343in}{0.855040in}}%
\pgfpathlineto{\pgfqpoint{1.687077in}{0.714107in}}%
\pgfpathlineto{\pgfqpoint{1.687944in}{0.712066in}}%
\pgfpathlineto{\pgfqpoint{1.688811in}{0.731879in}}%
\pgfpathlineto{\pgfqpoint{1.690545in}{1.043775in}}%
\pgfpathlineto{\pgfqpoint{1.691413in}{0.704662in}}%
\pgfpathlineto{\pgfqpoint{1.693147in}{1.348214in}}%
\pgfpathlineto{\pgfqpoint{1.694881in}{0.769252in}}%
\pgfpathlineto{\pgfqpoint{1.695748in}{0.935102in}}%
\pgfpathlineto{\pgfqpoint{1.698350in}{0.697734in}}%
\pgfpathlineto{\pgfqpoint{1.699217in}{0.704314in}}%
\pgfpathlineto{\pgfqpoint{1.700084in}{0.751453in}}%
\pgfpathlineto{\pgfqpoint{1.700951in}{0.730631in}}%
\pgfpathlineto{\pgfqpoint{1.702685in}{0.881067in}}%
\pgfpathlineto{\pgfqpoint{1.703552in}{0.717454in}}%
\pgfpathlineto{\pgfqpoint{1.704420in}{0.729695in}}%
\pgfpathlineto{\pgfqpoint{1.705287in}{0.756376in}}%
\pgfpathlineto{\pgfqpoint{1.707021in}{0.699043in}}%
\pgfpathlineto{\pgfqpoint{1.708755in}{0.873297in}}%
\pgfpathlineto{\pgfqpoint{1.709622in}{0.736594in}}%
\pgfpathlineto{\pgfqpoint{1.710490in}{0.778020in}}%
\pgfpathlineto{\pgfqpoint{1.711357in}{0.971593in}}%
\pgfpathlineto{\pgfqpoint{1.712224in}{0.744234in}}%
\pgfpathlineto{\pgfqpoint{1.713091in}{0.761087in}}%
\pgfpathlineto{\pgfqpoint{1.713958in}{0.788234in}}%
\pgfpathlineto{\pgfqpoint{1.715692in}{0.744479in}}%
\pgfpathlineto{\pgfqpoint{1.716559in}{0.782139in}}%
\pgfpathlineto{\pgfqpoint{1.717427in}{0.954769in}}%
\pgfpathlineto{\pgfqpoint{1.719161in}{0.710697in}}%
\pgfpathlineto{\pgfqpoint{1.720028in}{0.816761in}}%
\pgfpathlineto{\pgfqpoint{1.720895in}{0.814930in}}%
\pgfpathlineto{\pgfqpoint{1.721762in}{0.733450in}}%
\pgfpathlineto{\pgfqpoint{1.722629in}{0.763442in}}%
\pgfpathlineto{\pgfqpoint{1.723497in}{0.841735in}}%
\pgfpathlineto{\pgfqpoint{1.725231in}{0.700224in}}%
\pgfpathlineto{\pgfqpoint{1.726098in}{0.709617in}}%
\pgfpathlineto{\pgfqpoint{1.726965in}{0.719818in}}%
\pgfpathlineto{\pgfqpoint{1.728699in}{0.774405in}}%
\pgfpathlineto{\pgfqpoint{1.729566in}{0.729169in}}%
\pgfpathlineto{\pgfqpoint{1.730434in}{0.790396in}}%
\pgfpathlineto{\pgfqpoint{1.732168in}{0.706950in}}%
\pgfpathlineto{\pgfqpoint{1.733035in}{0.816843in}}%
\pgfpathlineto{\pgfqpoint{1.733902in}{0.798831in}}%
\pgfpathlineto{\pgfqpoint{1.734769in}{0.701432in}}%
\pgfpathlineto{\pgfqpoint{1.735636in}{0.749976in}}%
\pgfpathlineto{\pgfqpoint{1.736503in}{0.740249in}}%
\pgfpathlineto{\pgfqpoint{1.737371in}{0.746327in}}%
\pgfpathlineto{\pgfqpoint{1.738238in}{0.882831in}}%
\pgfpathlineto{\pgfqpoint{1.739972in}{0.730806in}}%
\pgfpathlineto{\pgfqpoint{1.740839in}{0.803502in}}%
\pgfpathlineto{\pgfqpoint{1.741706in}{0.760779in}}%
\pgfpathlineto{\pgfqpoint{1.742573in}{0.816301in}}%
\pgfpathlineto{\pgfqpoint{1.743441in}{0.787615in}}%
\pgfpathlineto{\pgfqpoint{1.744308in}{0.717665in}}%
\pgfpathlineto{\pgfqpoint{1.745175in}{0.908391in}}%
\pgfpathlineto{\pgfqpoint{1.746909in}{0.717412in}}%
\pgfpathlineto{\pgfqpoint{1.747776in}{0.849957in}}%
\pgfpathlineto{\pgfqpoint{1.749510in}{0.754821in}}%
\pgfpathlineto{\pgfqpoint{1.750378in}{0.756586in}}%
\pgfpathlineto{\pgfqpoint{1.752112in}{0.700284in}}%
\pgfpathlineto{\pgfqpoint{1.753846in}{0.714862in}}%
\pgfpathlineto{\pgfqpoint{1.754713in}{0.833921in}}%
\pgfpathlineto{\pgfqpoint{1.755580in}{0.735946in}}%
\pgfpathlineto{\pgfqpoint{1.756448in}{0.736886in}}%
\pgfpathlineto{\pgfqpoint{1.757315in}{0.760970in}}%
\pgfpathlineto{\pgfqpoint{1.758182in}{0.705102in}}%
\pgfpathlineto{\pgfqpoint{1.759049in}{0.706452in}}%
\pgfpathlineto{\pgfqpoint{1.759916in}{0.721486in}}%
\pgfpathlineto{\pgfqpoint{1.760783in}{0.790130in}}%
\pgfpathlineto{\pgfqpoint{1.762517in}{0.706982in}}%
\pgfpathlineto{\pgfqpoint{1.763385in}{0.702240in}}%
\pgfpathlineto{\pgfqpoint{1.764252in}{0.736668in}}%
\pgfpathlineto{\pgfqpoint{1.765119in}{0.870511in}}%
\pgfpathlineto{\pgfqpoint{1.765986in}{0.868826in}}%
\pgfpathlineto{\pgfqpoint{1.766853in}{0.698873in}}%
\pgfpathlineto{\pgfqpoint{1.768587in}{1.090041in}}%
\pgfpathlineto{\pgfqpoint{1.769455in}{0.765442in}}%
\pgfpathlineto{\pgfqpoint{1.770322in}{1.063309in}}%
\pgfpathlineto{\pgfqpoint{1.772056in}{0.779266in}}%
\pgfpathlineto{\pgfqpoint{1.772923in}{0.821591in}}%
\pgfpathlineto{\pgfqpoint{1.773790in}{0.723621in}}%
\pgfpathlineto{\pgfqpoint{1.774657in}{0.725740in}}%
\pgfpathlineto{\pgfqpoint{1.776392in}{1.004638in}}%
\pgfpathlineto{\pgfqpoint{1.777259in}{0.711657in}}%
\pgfpathlineto{\pgfqpoint{1.778126in}{1.057386in}}%
\pgfpathlineto{\pgfqpoint{1.779860in}{0.791731in}}%
\pgfpathlineto{\pgfqpoint{1.780727in}{0.712458in}}%
\pgfpathlineto{\pgfqpoint{1.781594in}{0.793909in}}%
\pgfpathlineto{\pgfqpoint{1.782462in}{0.753353in}}%
\pgfpathlineto{\pgfqpoint{1.783329in}{0.753420in}}%
\pgfpathlineto{\pgfqpoint{1.784196in}{0.759028in}}%
\pgfpathlineto{\pgfqpoint{1.785063in}{0.713613in}}%
\pgfpathlineto{\pgfqpoint{1.785930in}{0.722658in}}%
\pgfpathlineto{\pgfqpoint{1.786797in}{0.790525in}}%
\pgfpathlineto{\pgfqpoint{1.787664in}{0.761631in}}%
\pgfpathlineto{\pgfqpoint{1.789399in}{0.799618in}}%
\pgfpathlineto{\pgfqpoint{1.791133in}{0.703032in}}%
\pgfpathlineto{\pgfqpoint{1.792867in}{0.733658in}}%
\pgfpathlineto{\pgfqpoint{1.793734in}{0.724924in}}%
\pgfpathlineto{\pgfqpoint{1.794601in}{0.760751in}}%
\pgfpathlineto{\pgfqpoint{1.797203in}{0.701118in}}%
\pgfpathlineto{\pgfqpoint{1.798070in}{0.713605in}}%
\pgfpathlineto{\pgfqpoint{1.799804in}{0.744094in}}%
\pgfpathlineto{\pgfqpoint{1.800671in}{0.754628in}}%
\pgfpathlineto{\pgfqpoint{1.802406in}{0.706815in}}%
\pgfpathlineto{\pgfqpoint{1.803273in}{0.733093in}}%
\pgfpathlineto{\pgfqpoint{1.805007in}{0.697136in}}%
\pgfpathlineto{\pgfqpoint{1.806741in}{0.748913in}}%
\pgfpathlineto{\pgfqpoint{1.807608in}{0.763189in}}%
\pgfpathlineto{\pgfqpoint{1.808476in}{0.983202in}}%
\pgfpathlineto{\pgfqpoint{1.810210in}{0.717878in}}%
\pgfpathlineto{\pgfqpoint{1.811077in}{0.837855in}}%
\pgfpathlineto{\pgfqpoint{1.811944in}{0.743672in}}%
\pgfpathlineto{\pgfqpoint{1.812811in}{0.759426in}}%
\pgfpathlineto{\pgfqpoint{1.813678in}{0.866005in}}%
\pgfpathlineto{\pgfqpoint{1.815413in}{0.712296in}}%
\pgfpathlineto{\pgfqpoint{1.816280in}{0.786053in}}%
\pgfpathlineto{\pgfqpoint{1.818014in}{0.714551in}}%
\pgfpathlineto{\pgfqpoint{1.819748in}{0.721174in}}%
\pgfpathlineto{\pgfqpoint{1.820615in}{0.717756in}}%
\pgfpathlineto{\pgfqpoint{1.822350in}{0.813969in}}%
\pgfpathlineto{\pgfqpoint{1.824084in}{0.708514in}}%
\pgfpathlineto{\pgfqpoint{1.825818in}{0.886079in}}%
\pgfpathlineto{\pgfqpoint{1.827552in}{0.738501in}}%
\pgfpathlineto{\pgfqpoint{1.828420in}{0.870245in}}%
\pgfpathlineto{\pgfqpoint{1.830154in}{0.708492in}}%
\pgfpathlineto{\pgfqpoint{1.831021in}{0.733552in}}%
\pgfpathlineto{\pgfqpoint{1.834490in}{0.702409in}}%
\pgfpathlineto{\pgfqpoint{1.835357in}{0.711705in}}%
\pgfpathlineto{\pgfqpoint{1.836224in}{0.711236in}}%
\pgfpathlineto{\pgfqpoint{1.837091in}{0.790713in}}%
\pgfpathlineto{\pgfqpoint{1.837958in}{0.756911in}}%
\pgfpathlineto{\pgfqpoint{1.838825in}{0.789767in}}%
\pgfpathlineto{\pgfqpoint{1.839692in}{0.774315in}}%
\pgfpathlineto{\pgfqpoint{1.840559in}{0.720974in}}%
\pgfpathlineto{\pgfqpoint{1.841427in}{0.748536in}}%
\pgfpathlineto{\pgfqpoint{1.842294in}{0.704744in}}%
\pgfpathlineto{\pgfqpoint{1.844028in}{0.773976in}}%
\pgfpathlineto{\pgfqpoint{1.844895in}{0.716120in}}%
\pgfpathlineto{\pgfqpoint{1.845762in}{0.725627in}}%
\pgfpathlineto{\pgfqpoint{1.846629in}{0.786200in}}%
\pgfpathlineto{\pgfqpoint{1.847497in}{0.717550in}}%
\pgfpathlineto{\pgfqpoint{1.848364in}{0.794134in}}%
\pgfpathlineto{\pgfqpoint{1.849231in}{0.730209in}}%
\pgfpathlineto{\pgfqpoint{1.850965in}{0.972175in}}%
\pgfpathlineto{\pgfqpoint{1.852699in}{0.747787in}}%
\pgfpathlineto{\pgfqpoint{1.854434in}{0.823129in}}%
\pgfpathlineto{\pgfqpoint{1.855301in}{0.807824in}}%
\pgfpathlineto{\pgfqpoint{1.856168in}{0.711938in}}%
\pgfpathlineto{\pgfqpoint{1.857902in}{0.830389in}}%
\pgfpathlineto{\pgfqpoint{1.858769in}{0.731960in}}%
\pgfpathlineto{\pgfqpoint{1.859636in}{0.780020in}}%
\pgfpathlineto{\pgfqpoint{1.861371in}{0.714727in}}%
\pgfpathlineto{\pgfqpoint{1.863105in}{0.760675in}}%
\pgfpathlineto{\pgfqpoint{1.864839in}{0.702548in}}%
\pgfpathlineto{\pgfqpoint{1.866573in}{0.729106in}}%
\pgfpathlineto{\pgfqpoint{1.867441in}{0.696444in}}%
\pgfpathlineto{\pgfqpoint{1.869175in}{0.723706in}}%
\pgfpathlineto{\pgfqpoint{1.870042in}{0.937218in}}%
\pgfpathlineto{\pgfqpoint{1.871776in}{0.742817in}}%
\pgfpathlineto{\pgfqpoint{1.872643in}{0.844072in}}%
\pgfpathlineto{\pgfqpoint{1.873510in}{0.715227in}}%
\pgfpathlineto{\pgfqpoint{1.874378in}{0.724937in}}%
\pgfpathlineto{\pgfqpoint{1.876112in}{0.889488in}}%
\pgfpathlineto{\pgfqpoint{1.876979in}{0.706936in}}%
\pgfpathlineto{\pgfqpoint{1.878713in}{0.922796in}}%
\pgfpathlineto{\pgfqpoint{1.880448in}{0.727622in}}%
\pgfpathlineto{\pgfqpoint{1.881315in}{0.741932in}}%
\pgfpathlineto{\pgfqpoint{1.882182in}{0.703865in}}%
\pgfpathlineto{\pgfqpoint{1.883049in}{0.762521in}}%
\pgfpathlineto{\pgfqpoint{1.883916in}{0.932121in}}%
\pgfpathlineto{\pgfqpoint{1.884783in}{0.910814in}}%
\pgfpathlineto{\pgfqpoint{1.886517in}{0.757119in}}%
\pgfpathlineto{\pgfqpoint{1.887385in}{0.840578in}}%
\pgfpathlineto{\pgfqpoint{1.889119in}{0.697017in}}%
\pgfpathlineto{\pgfqpoint{1.892587in}{0.738393in}}%
\pgfpathlineto{\pgfqpoint{1.893455in}{0.812863in}}%
\pgfpathlineto{\pgfqpoint{1.895189in}{0.705687in}}%
\pgfpathlineto{\pgfqpoint{1.896056in}{0.762869in}}%
\pgfpathlineto{\pgfqpoint{1.897790in}{0.711178in}}%
\pgfpathlineto{\pgfqpoint{1.898657in}{0.751807in}}%
\pgfpathlineto{\pgfqpoint{1.899524in}{0.711081in}}%
\pgfpathlineto{\pgfqpoint{1.901259in}{0.883001in}}%
\pgfpathlineto{\pgfqpoint{1.902126in}{0.714651in}}%
\pgfpathlineto{\pgfqpoint{1.903860in}{1.011267in}}%
\pgfpathlineto{\pgfqpoint{1.904727in}{0.731404in}}%
\pgfpathlineto{\pgfqpoint{1.905594in}{0.811257in}}%
\pgfpathlineto{\pgfqpoint{1.906462in}{0.765198in}}%
\pgfpathlineto{\pgfqpoint{1.907329in}{0.772328in}}%
\pgfpathlineto{\pgfqpoint{1.909063in}{0.740095in}}%
\pgfpathlineto{\pgfqpoint{1.909930in}{0.880896in}}%
\pgfpathlineto{\pgfqpoint{1.910797in}{0.738737in}}%
\pgfpathlineto{\pgfqpoint{1.911664in}{0.774328in}}%
\pgfpathlineto{\pgfqpoint{1.914266in}{0.931699in}}%
\pgfpathlineto{\pgfqpoint{1.916000in}{0.750836in}}%
\pgfpathlineto{\pgfqpoint{1.916867in}{0.806570in}}%
\pgfpathlineto{\pgfqpoint{1.918601in}{0.716217in}}%
\pgfpathlineto{\pgfqpoint{1.919469in}{0.739017in}}%
\pgfpathlineto{\pgfqpoint{1.921203in}{0.707572in}}%
\pgfpathlineto{\pgfqpoint{1.922070in}{0.726308in}}%
\pgfpathlineto{\pgfqpoint{1.922937in}{0.704245in}}%
\pgfpathlineto{\pgfqpoint{1.923804in}{0.791227in}}%
\pgfpathlineto{\pgfqpoint{1.924671in}{0.742543in}}%
\pgfpathlineto{\pgfqpoint{1.925538in}{0.827145in}}%
\pgfpathlineto{\pgfqpoint{1.926406in}{1.032364in}}%
\pgfpathlineto{\pgfqpoint{1.927273in}{0.842617in}}%
\pgfpathlineto{\pgfqpoint{1.928140in}{0.866296in}}%
\pgfpathlineto{\pgfqpoint{1.929007in}{0.885722in}}%
\pgfpathlineto{\pgfqpoint{1.931608in}{0.784784in}}%
\pgfpathlineto{\pgfqpoint{1.932476in}{1.098662in}}%
\pgfpathlineto{\pgfqpoint{1.933343in}{0.818406in}}%
\pgfpathlineto{\pgfqpoint{1.935077in}{1.009989in}}%
\pgfpathlineto{\pgfqpoint{1.935944in}{0.720411in}}%
\pgfpathlineto{\pgfqpoint{1.936811in}{0.895153in}}%
\pgfpathlineto{\pgfqpoint{1.937678in}{0.720166in}}%
\pgfpathlineto{\pgfqpoint{1.939413in}{0.815510in}}%
\pgfpathlineto{\pgfqpoint{1.940280in}{0.702026in}}%
\pgfpathlineto{\pgfqpoint{1.942014in}{0.874036in}}%
\pgfpathlineto{\pgfqpoint{1.942881in}{0.718014in}}%
\pgfpathlineto{\pgfqpoint{1.944615in}{0.836573in}}%
\pgfpathlineto{\pgfqpoint{1.945483in}{0.729281in}}%
\pgfpathlineto{\pgfqpoint{1.947217in}{0.979108in}}%
\pgfpathlineto{\pgfqpoint{1.948951in}{0.878644in}}%
\pgfpathlineto{\pgfqpoint{1.949818in}{1.090093in}}%
\pgfpathlineto{\pgfqpoint{1.951552in}{0.701938in}}%
\pgfpathlineto{\pgfqpoint{1.952420in}{0.807937in}}%
\pgfpathlineto{\pgfqpoint{1.954154in}{0.696942in}}%
\pgfpathlineto{\pgfqpoint{1.955888in}{0.723560in}}%
\pgfpathlineto{\pgfqpoint{1.956755in}{0.788447in}}%
\pgfpathlineto{\pgfqpoint{1.957622in}{0.704801in}}%
\pgfpathlineto{\pgfqpoint{1.959357in}{0.817031in}}%
\pgfpathlineto{\pgfqpoint{1.960224in}{0.717763in}}%
\pgfpathlineto{\pgfqpoint{1.962825in}{0.899000in}}%
\pgfpathlineto{\pgfqpoint{1.965427in}{0.825413in}}%
\pgfpathlineto{\pgfqpoint{1.966294in}{0.850214in}}%
\pgfpathlineto{\pgfqpoint{1.968028in}{0.726210in}}%
\pgfpathlineto{\pgfqpoint{1.969762in}{0.827372in}}%
\pgfpathlineto{\pgfqpoint{1.971497in}{0.715984in}}%
\pgfpathlineto{\pgfqpoint{1.972364in}{0.737472in}}%
\pgfpathlineto{\pgfqpoint{1.973231in}{0.737346in}}%
\pgfpathlineto{\pgfqpoint{1.974965in}{0.703355in}}%
\pgfpathlineto{\pgfqpoint{1.975832in}{0.721580in}}%
\pgfpathlineto{\pgfqpoint{1.977566in}{0.880991in}}%
\pgfpathlineto{\pgfqpoint{1.979301in}{0.709107in}}%
\pgfpathlineto{\pgfqpoint{1.981035in}{0.950305in}}%
\pgfpathlineto{\pgfqpoint{1.981902in}{1.086534in}}%
\pgfpathlineto{\pgfqpoint{1.982769in}{1.000586in}}%
\pgfpathlineto{\pgfqpoint{1.983636in}{0.716856in}}%
\pgfpathlineto{\pgfqpoint{1.985371in}{0.948380in}}%
\pgfpathlineto{\pgfqpoint{1.986238in}{0.790547in}}%
\pgfpathlineto{\pgfqpoint{1.987105in}{0.814153in}}%
\pgfpathlineto{\pgfqpoint{1.988839in}{0.722763in}}%
\pgfpathlineto{\pgfqpoint{1.989706in}{0.949927in}}%
\pgfpathlineto{\pgfqpoint{1.991441in}{0.713944in}}%
\pgfpathlineto{\pgfqpoint{1.992308in}{0.820134in}}%
\pgfpathlineto{\pgfqpoint{1.994909in}{0.705303in}}%
\pgfpathlineto{\pgfqpoint{1.996643in}{0.868510in}}%
\pgfpathlineto{\pgfqpoint{1.998378in}{0.698576in}}%
\pgfpathlineto{\pgfqpoint{2.000112in}{1.020448in}}%
\pgfpathlineto{\pgfqpoint{2.000979in}{0.757414in}}%
\pgfpathlineto{\pgfqpoint{2.001846in}{0.793079in}}%
\pgfpathlineto{\pgfqpoint{2.002713in}{0.953759in}}%
\pgfpathlineto{\pgfqpoint{2.004448in}{0.729622in}}%
\pgfpathlineto{\pgfqpoint{2.005315in}{0.709859in}}%
\pgfpathlineto{\pgfqpoint{2.006182in}{0.714214in}}%
\pgfpathlineto{\pgfqpoint{2.007049in}{0.723856in}}%
\pgfpathlineto{\pgfqpoint{2.007916in}{0.721118in}}%
\pgfpathlineto{\pgfqpoint{2.008783in}{0.814347in}}%
\pgfpathlineto{\pgfqpoint{2.009650in}{0.796918in}}%
\pgfpathlineto{\pgfqpoint{2.010517in}{0.700543in}}%
\pgfpathlineto{\pgfqpoint{2.011385in}{0.776674in}}%
\pgfpathlineto{\pgfqpoint{2.012252in}{0.956734in}}%
\pgfpathlineto{\pgfqpoint{2.013986in}{0.726212in}}%
\pgfpathlineto{\pgfqpoint{2.014853in}{0.739250in}}%
\pgfpathlineto{\pgfqpoint{2.015720in}{0.750677in}}%
\pgfpathlineto{\pgfqpoint{2.016587in}{0.860122in}}%
\pgfpathlineto{\pgfqpoint{2.018322in}{0.717730in}}%
\pgfpathlineto{\pgfqpoint{2.019189in}{0.827990in}}%
\pgfpathlineto{\pgfqpoint{2.020056in}{0.808405in}}%
\pgfpathlineto{\pgfqpoint{2.020923in}{0.713343in}}%
\pgfpathlineto{\pgfqpoint{2.022657in}{0.842311in}}%
\pgfpathlineto{\pgfqpoint{2.024392in}{0.708214in}}%
\pgfpathlineto{\pgfqpoint{2.025259in}{0.786155in}}%
\pgfpathlineto{\pgfqpoint{2.026993in}{0.707552in}}%
\pgfpathlineto{\pgfqpoint{2.027860in}{0.699605in}}%
\pgfpathlineto{\pgfqpoint{2.029594in}{0.831593in}}%
\pgfpathlineto{\pgfqpoint{2.030462in}{0.706472in}}%
\pgfpathlineto{\pgfqpoint{2.032196in}{0.982183in}}%
\pgfpathlineto{\pgfqpoint{2.033063in}{0.898279in}}%
\pgfpathlineto{\pgfqpoint{2.033930in}{0.824853in}}%
\pgfpathlineto{\pgfqpoint{2.035664in}{0.941046in}}%
\pgfpathlineto{\pgfqpoint{2.037399in}{0.716196in}}%
\pgfpathlineto{\pgfqpoint{2.038266in}{0.714542in}}%
\pgfpathlineto{\pgfqpoint{2.040000in}{0.967411in}}%
\pgfpathlineto{\pgfqpoint{2.040867in}{0.720281in}}%
\pgfpathlineto{\pgfqpoint{2.042601in}{0.970087in}}%
\pgfpathlineto{\pgfqpoint{2.044336in}{0.704618in}}%
\pgfpathlineto{\pgfqpoint{2.046070in}{0.939490in}}%
\pgfpathlineto{\pgfqpoint{2.046937in}{0.742724in}}%
\pgfpathlineto{\pgfqpoint{2.049538in}{1.002789in}}%
\pgfpathlineto{\pgfqpoint{2.050406in}{0.929582in}}%
\pgfpathlineto{\pgfqpoint{2.051273in}{0.701766in}}%
\pgfpathlineto{\pgfqpoint{2.052140in}{0.759666in}}%
\pgfpathlineto{\pgfqpoint{2.053874in}{0.806645in}}%
\pgfpathlineto{\pgfqpoint{2.055608in}{0.702074in}}%
\pgfpathlineto{\pgfqpoint{2.056476in}{0.727423in}}%
\pgfpathlineto{\pgfqpoint{2.057343in}{0.709317in}}%
\pgfpathlineto{\pgfqpoint{2.058210in}{0.751481in}}%
\pgfpathlineto{\pgfqpoint{2.059077in}{0.959202in}}%
\pgfpathlineto{\pgfqpoint{2.059944in}{0.947334in}}%
\pgfpathlineto{\pgfqpoint{2.061678in}{0.701021in}}%
\pgfpathlineto{\pgfqpoint{2.063413in}{0.978875in}}%
\pgfpathlineto{\pgfqpoint{2.064280in}{1.133918in}}%
\pgfpathlineto{\pgfqpoint{2.066014in}{0.716493in}}%
\pgfpathlineto{\pgfqpoint{2.066881in}{0.787956in}}%
\pgfpathlineto{\pgfqpoint{2.067748in}{0.719967in}}%
\pgfpathlineto{\pgfqpoint{2.068615in}{0.722496in}}%
\pgfpathlineto{\pgfqpoint{2.069483in}{0.702392in}}%
\pgfpathlineto{\pgfqpoint{2.070350in}{0.728453in}}%
\pgfpathlineto{\pgfqpoint{2.071217in}{0.708292in}}%
\pgfpathlineto{\pgfqpoint{2.073818in}{0.776359in}}%
\pgfpathlineto{\pgfqpoint{2.074685in}{0.928158in}}%
\pgfpathlineto{\pgfqpoint{2.075552in}{0.797746in}}%
\pgfpathlineto{\pgfqpoint{2.077287in}{1.141616in}}%
\pgfpathlineto{\pgfqpoint{2.078154in}{0.696173in}}%
\pgfpathlineto{\pgfqpoint{2.079021in}{1.159920in}}%
\pgfpathlineto{\pgfqpoint{2.080755in}{0.745765in}}%
\pgfpathlineto{\pgfqpoint{2.081622in}{0.913279in}}%
\pgfpathlineto{\pgfqpoint{2.082490in}{0.726353in}}%
\pgfpathlineto{\pgfqpoint{2.083357in}{0.935357in}}%
\pgfpathlineto{\pgfqpoint{2.085091in}{0.790602in}}%
\pgfpathlineto{\pgfqpoint{2.085958in}{0.853756in}}%
\pgfpathlineto{\pgfqpoint{2.088559in}{0.706066in}}%
\pgfpathlineto{\pgfqpoint{2.089427in}{0.717816in}}%
\pgfpathlineto{\pgfqpoint{2.090294in}{0.697989in}}%
\pgfpathlineto{\pgfqpoint{2.092028in}{0.795534in}}%
\pgfpathlineto{\pgfqpoint{2.092895in}{0.701203in}}%
\pgfpathlineto{\pgfqpoint{2.093762in}{0.786440in}}%
\pgfpathlineto{\pgfqpoint{2.096364in}{0.697181in}}%
\pgfpathlineto{\pgfqpoint{2.097231in}{0.717901in}}%
\pgfpathlineto{\pgfqpoint{2.098098in}{0.706579in}}%
\pgfpathlineto{\pgfqpoint{2.098965in}{0.711630in}}%
\pgfpathlineto{\pgfqpoint{2.099832in}{0.702856in}}%
\pgfpathlineto{\pgfqpoint{2.100699in}{0.801122in}}%
\pgfpathlineto{\pgfqpoint{2.101566in}{0.749024in}}%
\pgfpathlineto{\pgfqpoint{2.102434in}{0.927865in}}%
\pgfpathlineto{\pgfqpoint{2.103301in}{0.896749in}}%
\pgfpathlineto{\pgfqpoint{2.104168in}{0.706903in}}%
\pgfpathlineto{\pgfqpoint{2.105035in}{0.772408in}}%
\pgfpathlineto{\pgfqpoint{2.105902in}{0.733886in}}%
\pgfpathlineto{\pgfqpoint{2.106769in}{0.802253in}}%
\pgfpathlineto{\pgfqpoint{2.107636in}{0.704587in}}%
\pgfpathlineto{\pgfqpoint{2.108503in}{1.025360in}}%
\pgfpathlineto{\pgfqpoint{2.110238in}{0.712035in}}%
\pgfpathlineto{\pgfqpoint{2.111105in}{0.820413in}}%
\pgfpathlineto{\pgfqpoint{2.111972in}{0.778031in}}%
\pgfpathlineto{\pgfqpoint{2.112839in}{0.871325in}}%
\pgfpathlineto{\pgfqpoint{2.114573in}{0.703142in}}%
\pgfpathlineto{\pgfqpoint{2.115441in}{0.842557in}}%
\pgfpathlineto{\pgfqpoint{2.116308in}{0.801914in}}%
\pgfpathlineto{\pgfqpoint{2.117175in}{0.708050in}}%
\pgfpathlineto{\pgfqpoint{2.118042in}{0.717503in}}%
\pgfpathlineto{\pgfqpoint{2.118909in}{0.739178in}}%
\pgfpathlineto{\pgfqpoint{2.119776in}{0.704353in}}%
\pgfpathlineto{\pgfqpoint{2.121510in}{0.885054in}}%
\pgfpathlineto{\pgfqpoint{2.123245in}{0.707921in}}%
\pgfpathlineto{\pgfqpoint{2.124112in}{0.700073in}}%
\pgfpathlineto{\pgfqpoint{2.124979in}{0.745146in}}%
\pgfpathlineto{\pgfqpoint{2.125846in}{0.926521in}}%
\pgfpathlineto{\pgfqpoint{2.126713in}{0.909381in}}%
\pgfpathlineto{\pgfqpoint{2.127580in}{0.717618in}}%
\pgfpathlineto{\pgfqpoint{2.129315in}{1.010812in}}%
\pgfpathlineto{\pgfqpoint{2.130182in}{0.780473in}}%
\pgfpathlineto{\pgfqpoint{2.131049in}{0.856444in}}%
\pgfpathlineto{\pgfqpoint{2.131916in}{1.276523in}}%
\pgfpathlineto{\pgfqpoint{2.133650in}{0.860035in}}%
\pgfpathlineto{\pgfqpoint{2.134517in}{0.862388in}}%
\pgfpathlineto{\pgfqpoint{2.135385in}{1.164234in}}%
\pgfpathlineto{\pgfqpoint{2.137119in}{0.708827in}}%
\pgfpathlineto{\pgfqpoint{2.137986in}{0.971084in}}%
\pgfpathlineto{\pgfqpoint{2.138853in}{0.921807in}}%
\pgfpathlineto{\pgfqpoint{2.140587in}{0.740546in}}%
\pgfpathlineto{\pgfqpoint{2.141455in}{0.722460in}}%
\pgfpathlineto{\pgfqpoint{2.142322in}{0.966751in}}%
\pgfpathlineto{\pgfqpoint{2.143189in}{0.961516in}}%
\pgfpathlineto{\pgfqpoint{2.144056in}{0.723046in}}%
\pgfpathlineto{\pgfqpoint{2.144923in}{0.784137in}}%
\pgfpathlineto{\pgfqpoint{2.145790in}{0.818247in}}%
\pgfpathlineto{\pgfqpoint{2.146657in}{0.713734in}}%
\pgfpathlineto{\pgfqpoint{2.147524in}{0.756049in}}%
\pgfpathlineto{\pgfqpoint{2.148392in}{0.738366in}}%
\pgfpathlineto{\pgfqpoint{2.149259in}{0.781250in}}%
\pgfpathlineto{\pgfqpoint{2.150126in}{0.991373in}}%
\pgfpathlineto{\pgfqpoint{2.150993in}{0.713437in}}%
\pgfpathlineto{\pgfqpoint{2.152727in}{0.899579in}}%
\pgfpathlineto{\pgfqpoint{2.153594in}{0.711224in}}%
\pgfpathlineto{\pgfqpoint{2.154462in}{0.711672in}}%
\pgfpathlineto{\pgfqpoint{2.156196in}{0.744318in}}%
\pgfpathlineto{\pgfqpoint{2.157063in}{0.707656in}}%
\pgfpathlineto{\pgfqpoint{2.158797in}{0.959584in}}%
\pgfpathlineto{\pgfqpoint{2.159664in}{0.724971in}}%
\pgfpathlineto{\pgfqpoint{2.161399in}{0.966160in}}%
\pgfpathlineto{\pgfqpoint{2.162266in}{0.908180in}}%
\pgfpathlineto{\pgfqpoint{2.163133in}{0.703466in}}%
\pgfpathlineto{\pgfqpoint{2.164867in}{1.111182in}}%
\pgfpathlineto{\pgfqpoint{2.165734in}{0.711114in}}%
\pgfpathlineto{\pgfqpoint{2.166601in}{0.789011in}}%
\pgfpathlineto{\pgfqpoint{2.167469in}{0.798508in}}%
\pgfpathlineto{\pgfqpoint{2.169203in}{0.718618in}}%
\pgfpathlineto{\pgfqpoint{2.170070in}{0.713938in}}%
\pgfpathlineto{\pgfqpoint{2.170937in}{0.730287in}}%
\pgfpathlineto{\pgfqpoint{2.171804in}{0.772012in}}%
\pgfpathlineto{\pgfqpoint{2.172671in}{0.701867in}}%
\pgfpathlineto{\pgfqpoint{2.174406in}{0.805064in}}%
\pgfpathlineto{\pgfqpoint{2.176140in}{0.722917in}}%
\pgfpathlineto{\pgfqpoint{2.177007in}{0.750967in}}%
\pgfpathlineto{\pgfqpoint{2.178741in}{0.699279in}}%
\pgfpathlineto{\pgfqpoint{2.179608in}{0.716578in}}%
\pgfpathlineto{\pgfqpoint{2.180476in}{0.801149in}}%
\pgfpathlineto{\pgfqpoint{2.182210in}{0.696850in}}%
\pgfpathlineto{\pgfqpoint{2.183077in}{0.741711in}}%
\pgfpathlineto{\pgfqpoint{2.184811in}{0.715040in}}%
\pgfpathlineto{\pgfqpoint{2.185678in}{0.764250in}}%
\pgfpathlineto{\pgfqpoint{2.187413in}{0.742127in}}%
\pgfpathlineto{\pgfqpoint{2.188280in}{0.908617in}}%
\pgfpathlineto{\pgfqpoint{2.190014in}{0.763721in}}%
\pgfpathlineto{\pgfqpoint{2.190881in}{0.727980in}}%
\pgfpathlineto{\pgfqpoint{2.191748in}{0.752735in}}%
\pgfpathlineto{\pgfqpoint{2.192615in}{0.826935in}}%
\pgfpathlineto{\pgfqpoint{2.193483in}{0.750496in}}%
\pgfpathlineto{\pgfqpoint{2.194350in}{0.885366in}}%
\pgfpathlineto{\pgfqpoint{2.195217in}{0.878280in}}%
\pgfpathlineto{\pgfqpoint{2.196951in}{0.697182in}}%
\pgfpathlineto{\pgfqpoint{2.197818in}{0.697452in}}%
\pgfpathlineto{\pgfqpoint{2.198685in}{0.703869in}}%
\pgfpathlineto{\pgfqpoint{2.200420in}{0.784484in}}%
\pgfpathlineto{\pgfqpoint{2.201287in}{0.716175in}}%
\pgfpathlineto{\pgfqpoint{2.202154in}{0.762653in}}%
\pgfpathlineto{\pgfqpoint{2.203021in}{0.739345in}}%
\pgfpathlineto{\pgfqpoint{2.203888in}{0.764189in}}%
\pgfpathlineto{\pgfqpoint{2.204755in}{0.699562in}}%
\pgfpathlineto{\pgfqpoint{2.206490in}{0.887929in}}%
\pgfpathlineto{\pgfqpoint{2.207357in}{0.701179in}}%
\pgfpathlineto{\pgfqpoint{2.208224in}{0.831291in}}%
\pgfpathlineto{\pgfqpoint{2.209091in}{0.809304in}}%
\pgfpathlineto{\pgfqpoint{2.209958in}{0.698473in}}%
\pgfpathlineto{\pgfqpoint{2.210825in}{0.709721in}}%
\pgfpathlineto{\pgfqpoint{2.211692in}{0.697513in}}%
\pgfpathlineto{\pgfqpoint{2.213427in}{0.718516in}}%
\pgfpathlineto{\pgfqpoint{2.215161in}{0.698404in}}%
\pgfpathlineto{\pgfqpoint{2.216028in}{0.734443in}}%
\pgfpathlineto{\pgfqpoint{2.216895in}{0.847113in}}%
\pgfpathlineto{\pgfqpoint{2.218629in}{0.735354in}}%
\pgfpathlineto{\pgfqpoint{2.219497in}{1.090245in}}%
\pgfpathlineto{\pgfqpoint{2.221231in}{0.706045in}}%
\pgfpathlineto{\pgfqpoint{2.222098in}{0.817766in}}%
\pgfpathlineto{\pgfqpoint{2.223832in}{0.722916in}}%
\pgfpathlineto{\pgfqpoint{2.224699in}{0.719379in}}%
\pgfpathlineto{\pgfqpoint{2.225566in}{0.799882in}}%
\pgfpathlineto{\pgfqpoint{2.226434in}{0.705524in}}%
\pgfpathlineto{\pgfqpoint{2.227301in}{0.793050in}}%
\pgfpathlineto{\pgfqpoint{2.229035in}{0.707350in}}%
\pgfpathlineto{\pgfqpoint{2.229902in}{0.712002in}}%
\pgfpathlineto{\pgfqpoint{2.230769in}{0.708879in}}%
\pgfpathlineto{\pgfqpoint{2.232503in}{0.730066in}}%
\pgfpathlineto{\pgfqpoint{2.233371in}{0.709625in}}%
\pgfpathlineto{\pgfqpoint{2.235105in}{1.253402in}}%
\pgfpathlineto{\pgfqpoint{2.235972in}{0.798431in}}%
\pgfpathlineto{\pgfqpoint{2.237706in}{1.586019in}}%
\pgfpathlineto{\pgfqpoint{2.239441in}{0.738081in}}%
\pgfpathlineto{\pgfqpoint{2.241175in}{0.944940in}}%
\pgfpathlineto{\pgfqpoint{2.242042in}{0.838194in}}%
\pgfpathlineto{\pgfqpoint{2.242909in}{0.944502in}}%
\pgfpathlineto{\pgfqpoint{2.244643in}{0.788533in}}%
\pgfpathlineto{\pgfqpoint{2.245510in}{0.791414in}}%
\pgfpathlineto{\pgfqpoint{2.247245in}{0.724296in}}%
\pgfpathlineto{\pgfqpoint{2.248112in}{0.826842in}}%
\pgfpathlineto{\pgfqpoint{2.248979in}{0.800330in}}%
\pgfpathlineto{\pgfqpoint{2.249846in}{0.778304in}}%
\pgfpathlineto{\pgfqpoint{2.250713in}{0.996296in}}%
\pgfpathlineto{\pgfqpoint{2.252448in}{0.753578in}}%
\pgfpathlineto{\pgfqpoint{2.253315in}{0.806174in}}%
\pgfpathlineto{\pgfqpoint{2.254182in}{0.726168in}}%
\pgfpathlineto{\pgfqpoint{2.255049in}{0.736731in}}%
\pgfpathlineto{\pgfqpoint{2.255916in}{0.698726in}}%
\pgfpathlineto{\pgfqpoint{2.257650in}{0.774854in}}%
\pgfpathlineto{\pgfqpoint{2.259385in}{0.709553in}}%
\pgfpathlineto{\pgfqpoint{2.260252in}{0.709183in}}%
\pgfpathlineto{\pgfqpoint{2.261119in}{0.714577in}}%
\pgfpathlineto{\pgfqpoint{2.261986in}{0.768881in}}%
\pgfpathlineto{\pgfqpoint{2.262853in}{0.755354in}}%
\pgfpathlineto{\pgfqpoint{2.265455in}{0.701079in}}%
\pgfpathlineto{\pgfqpoint{2.266322in}{0.772196in}}%
\pgfpathlineto{\pgfqpoint{2.268056in}{0.705558in}}%
\pgfpathlineto{\pgfqpoint{2.270657in}{0.830788in}}%
\pgfpathlineto{\pgfqpoint{2.272392in}{0.711033in}}%
\pgfpathlineto{\pgfqpoint{2.273259in}{0.891419in}}%
\pgfpathlineto{\pgfqpoint{2.274126in}{0.848582in}}%
\pgfpathlineto{\pgfqpoint{2.274993in}{0.698000in}}%
\pgfpathlineto{\pgfqpoint{2.275860in}{0.797723in}}%
\pgfpathlineto{\pgfqpoint{2.276727in}{0.797083in}}%
\pgfpathlineto{\pgfqpoint{2.278462in}{0.731827in}}%
\pgfpathlineto{\pgfqpoint{2.279329in}{0.757279in}}%
\pgfpathlineto{\pgfqpoint{2.280196in}{0.961022in}}%
\pgfpathlineto{\pgfqpoint{2.281930in}{0.745294in}}%
\pgfpathlineto{\pgfqpoint{2.283664in}{0.904626in}}%
\pgfpathlineto{\pgfqpoint{2.285399in}{0.759170in}}%
\pgfpathlineto{\pgfqpoint{2.286266in}{0.780483in}}%
\pgfpathlineto{\pgfqpoint{2.288000in}{1.213576in}}%
\pgfpathlineto{\pgfqpoint{2.289734in}{0.797132in}}%
\pgfpathlineto{\pgfqpoint{2.291469in}{1.153483in}}%
\pgfpathlineto{\pgfqpoint{2.292336in}{0.738298in}}%
\pgfpathlineto{\pgfqpoint{2.293203in}{0.742319in}}%
\pgfpathlineto{\pgfqpoint{2.294070in}{0.709634in}}%
\pgfpathlineto{\pgfqpoint{2.294937in}{0.718172in}}%
\pgfpathlineto{\pgfqpoint{2.295804in}{0.711080in}}%
\pgfpathlineto{\pgfqpoint{2.297538in}{0.759124in}}%
\pgfpathlineto{\pgfqpoint{2.299273in}{0.696602in}}%
\pgfpathlineto{\pgfqpoint{2.301007in}{0.720443in}}%
\pgfpathlineto{\pgfqpoint{2.301874in}{0.719869in}}%
\pgfpathlineto{\pgfqpoint{2.302741in}{0.701391in}}%
\pgfpathlineto{\pgfqpoint{2.304476in}{0.899002in}}%
\pgfpathlineto{\pgfqpoint{2.306210in}{0.710792in}}%
\pgfpathlineto{\pgfqpoint{2.307077in}{0.914649in}}%
\pgfpathlineto{\pgfqpoint{2.308811in}{0.725505in}}%
\pgfpathlineto{\pgfqpoint{2.309678in}{0.832956in}}%
\pgfpathlineto{\pgfqpoint{2.311413in}{0.703217in}}%
\pgfpathlineto{\pgfqpoint{2.312280in}{0.735847in}}%
\pgfpathlineto{\pgfqpoint{2.313147in}{0.709229in}}%
\pgfpathlineto{\pgfqpoint{2.314014in}{0.792614in}}%
\pgfpathlineto{\pgfqpoint{2.314881in}{0.785843in}}%
\pgfpathlineto{\pgfqpoint{2.315748in}{0.700016in}}%
\pgfpathlineto{\pgfqpoint{2.316615in}{0.828652in}}%
\pgfpathlineto{\pgfqpoint{2.317483in}{0.803299in}}%
\pgfpathlineto{\pgfqpoint{2.318350in}{0.762596in}}%
\pgfpathlineto{\pgfqpoint{2.320084in}{0.819009in}}%
\pgfpathlineto{\pgfqpoint{2.320951in}{0.814623in}}%
\pgfpathlineto{\pgfqpoint{2.322685in}{0.761896in}}%
\pgfpathlineto{\pgfqpoint{2.324420in}{0.728641in}}%
\pgfpathlineto{\pgfqpoint{2.325287in}{0.830626in}}%
\pgfpathlineto{\pgfqpoint{2.326154in}{0.821692in}}%
\pgfpathlineto{\pgfqpoint{2.327888in}{0.724234in}}%
\pgfpathlineto{\pgfqpoint{2.329622in}{0.834394in}}%
\pgfpathlineto{\pgfqpoint{2.330490in}{0.780208in}}%
\pgfpathlineto{\pgfqpoint{2.332224in}{0.943943in}}%
\pgfpathlineto{\pgfqpoint{2.334825in}{0.712837in}}%
\pgfpathlineto{\pgfqpoint{2.336559in}{0.853735in}}%
\pgfpathlineto{\pgfqpoint{2.337427in}{0.727135in}}%
\pgfpathlineto{\pgfqpoint{2.338294in}{0.946425in}}%
\pgfpathlineto{\pgfqpoint{2.339161in}{0.937127in}}%
\pgfpathlineto{\pgfqpoint{2.340028in}{0.968684in}}%
\pgfpathlineto{\pgfqpoint{2.340895in}{1.089999in}}%
\pgfpathlineto{\pgfqpoint{2.343497in}{0.698400in}}%
\pgfpathlineto{\pgfqpoint{2.344364in}{0.753671in}}%
\pgfpathlineto{\pgfqpoint{2.346098in}{1.165132in}}%
\pgfpathlineto{\pgfqpoint{2.347832in}{0.725494in}}%
\pgfpathlineto{\pgfqpoint{2.348699in}{0.746073in}}%
\pgfpathlineto{\pgfqpoint{2.349566in}{0.701488in}}%
\pgfpathlineto{\pgfqpoint{2.351301in}{0.987720in}}%
\pgfpathlineto{\pgfqpoint{2.353035in}{0.725968in}}%
\pgfpathlineto{\pgfqpoint{2.355636in}{0.950994in}}%
\pgfpathlineto{\pgfqpoint{2.357371in}{0.777997in}}%
\pgfpathlineto{\pgfqpoint{2.358238in}{1.094062in}}%
\pgfpathlineto{\pgfqpoint{2.359972in}{0.733686in}}%
\pgfpathlineto{\pgfqpoint{2.360839in}{0.773090in}}%
\pgfpathlineto{\pgfqpoint{2.361706in}{0.788588in}}%
\pgfpathlineto{\pgfqpoint{2.362573in}{0.776252in}}%
\pgfpathlineto{\pgfqpoint{2.363441in}{0.718534in}}%
\pgfpathlineto{\pgfqpoint{2.365175in}{1.021513in}}%
\pgfpathlineto{\pgfqpoint{2.366042in}{0.762015in}}%
\pgfpathlineto{\pgfqpoint{2.366909in}{0.875749in}}%
\pgfpathlineto{\pgfqpoint{2.369510in}{0.734356in}}%
\pgfpathlineto{\pgfqpoint{2.371245in}{0.962643in}}%
\pgfpathlineto{\pgfqpoint{2.372112in}{0.752092in}}%
\pgfpathlineto{\pgfqpoint{2.372979in}{1.012381in}}%
\pgfpathlineto{\pgfqpoint{2.373846in}{0.973798in}}%
\pgfpathlineto{\pgfqpoint{2.374713in}{0.699861in}}%
\pgfpathlineto{\pgfqpoint{2.375580in}{0.840348in}}%
\pgfpathlineto{\pgfqpoint{2.377315in}{0.697540in}}%
\pgfpathlineto{\pgfqpoint{2.378182in}{0.705796in}}%
\pgfpathlineto{\pgfqpoint{2.379049in}{0.698839in}}%
\pgfpathlineto{\pgfqpoint{2.380783in}{0.716783in}}%
\pgfpathlineto{\pgfqpoint{2.381650in}{0.706394in}}%
\pgfpathlineto{\pgfqpoint{2.382517in}{0.721396in}}%
\pgfpathlineto{\pgfqpoint{2.383385in}{0.825068in}}%
\pgfpathlineto{\pgfqpoint{2.384252in}{0.813378in}}%
\pgfpathlineto{\pgfqpoint{2.385119in}{0.698428in}}%
\pgfpathlineto{\pgfqpoint{2.386853in}{0.873464in}}%
\pgfpathlineto{\pgfqpoint{2.388587in}{0.723974in}}%
\pgfpathlineto{\pgfqpoint{2.389455in}{0.814940in}}%
\pgfpathlineto{\pgfqpoint{2.391189in}{0.715216in}}%
\pgfpathlineto{\pgfqpoint{2.392923in}{0.805557in}}%
\pgfpathlineto{\pgfqpoint{2.393790in}{0.730125in}}%
\pgfpathlineto{\pgfqpoint{2.394657in}{0.770621in}}%
\pgfpathlineto{\pgfqpoint{2.395524in}{0.975542in}}%
\pgfpathlineto{\pgfqpoint{2.397259in}{0.731389in}}%
\pgfpathlineto{\pgfqpoint{2.398993in}{0.894317in}}%
\pgfpathlineto{\pgfqpoint{2.399860in}{0.744493in}}%
\pgfpathlineto{\pgfqpoint{2.400727in}{0.757233in}}%
\pgfpathlineto{\pgfqpoint{2.402462in}{1.239869in}}%
\pgfpathlineto{\pgfqpoint{2.404196in}{0.772350in}}%
\pgfpathlineto{\pgfqpoint{2.405063in}{1.069361in}}%
\pgfpathlineto{\pgfqpoint{2.406797in}{0.706341in}}%
\pgfpathlineto{\pgfqpoint{2.407664in}{0.733902in}}%
\pgfpathlineto{\pgfqpoint{2.408531in}{0.744100in}}%
\pgfpathlineto{\pgfqpoint{2.410266in}{0.719815in}}%
\pgfpathlineto{\pgfqpoint{2.411133in}{0.725334in}}%
\pgfpathlineto{\pgfqpoint{2.412000in}{0.702377in}}%
\pgfpathlineto{\pgfqpoint{2.413734in}{1.122247in}}%
\pgfpathlineto{\pgfqpoint{2.415469in}{0.797574in}}%
\pgfpathlineto{\pgfqpoint{2.416336in}{1.029567in}}%
\pgfpathlineto{\pgfqpoint{2.418070in}{0.745680in}}%
\pgfpathlineto{\pgfqpoint{2.419804in}{0.903614in}}%
\pgfpathlineto{\pgfqpoint{2.420671in}{0.786645in}}%
\pgfpathlineto{\pgfqpoint{2.421538in}{0.876235in}}%
\pgfpathlineto{\pgfqpoint{2.422406in}{0.873809in}}%
\pgfpathlineto{\pgfqpoint{2.423273in}{0.864301in}}%
\pgfpathlineto{\pgfqpoint{2.424140in}{0.724310in}}%
\pgfpathlineto{\pgfqpoint{2.425007in}{0.769440in}}%
\pgfpathlineto{\pgfqpoint{2.425874in}{0.728264in}}%
\pgfpathlineto{\pgfqpoint{2.427608in}{0.862171in}}%
\pgfpathlineto{\pgfqpoint{2.429343in}{0.709638in}}%
\pgfpathlineto{\pgfqpoint{2.431944in}{0.755962in}}%
\pgfpathlineto{\pgfqpoint{2.433678in}{1.142441in}}%
\pgfpathlineto{\pgfqpoint{2.435413in}{0.703003in}}%
\pgfpathlineto{\pgfqpoint{2.436280in}{0.762173in}}%
\pgfpathlineto{\pgfqpoint{2.437147in}{0.945413in}}%
\pgfpathlineto{\pgfqpoint{2.438014in}{0.941893in}}%
\pgfpathlineto{\pgfqpoint{2.438881in}{0.732115in}}%
\pgfpathlineto{\pgfqpoint{2.439748in}{0.900793in}}%
\pgfpathlineto{\pgfqpoint{2.440615in}{0.854353in}}%
\pgfpathlineto{\pgfqpoint{2.441483in}{0.706903in}}%
\pgfpathlineto{\pgfqpoint{2.443217in}{0.878341in}}%
\pgfpathlineto{\pgfqpoint{2.444951in}{0.734937in}}%
\pgfpathlineto{\pgfqpoint{2.446685in}{0.925309in}}%
\pgfpathlineto{\pgfqpoint{2.448420in}{0.698209in}}%
\pgfpathlineto{\pgfqpoint{2.450154in}{0.893418in}}%
\pgfpathlineto{\pgfqpoint{2.451021in}{0.775445in}}%
\pgfpathlineto{\pgfqpoint{2.451888in}{1.169060in}}%
\pgfpathlineto{\pgfqpoint{2.452755in}{1.125200in}}%
\pgfpathlineto{\pgfqpoint{2.453622in}{0.709737in}}%
\pgfpathlineto{\pgfqpoint{2.455357in}{0.945534in}}%
\pgfpathlineto{\pgfqpoint{2.457091in}{0.706108in}}%
\pgfpathlineto{\pgfqpoint{2.458825in}{0.890904in}}%
\pgfpathlineto{\pgfqpoint{2.460559in}{0.721088in}}%
\pgfpathlineto{\pgfqpoint{2.462294in}{0.706787in}}%
\pgfpathlineto{\pgfqpoint{2.464028in}{0.916268in}}%
\pgfpathlineto{\pgfqpoint{2.465762in}{0.734862in}}%
\pgfpathlineto{\pgfqpoint{2.466629in}{0.773992in}}%
\pgfpathlineto{\pgfqpoint{2.468364in}{0.709750in}}%
\pgfpathlineto{\pgfqpoint{2.469231in}{0.720870in}}%
\pgfpathlineto{\pgfqpoint{2.470098in}{0.796869in}}%
\pgfpathlineto{\pgfqpoint{2.471832in}{0.709601in}}%
\pgfpathlineto{\pgfqpoint{2.472699in}{0.724440in}}%
\pgfpathlineto{\pgfqpoint{2.473566in}{0.700641in}}%
\pgfpathlineto{\pgfqpoint{2.475301in}{0.777312in}}%
\pgfpathlineto{\pgfqpoint{2.477035in}{0.868302in}}%
\pgfpathlineto{\pgfqpoint{2.477902in}{0.753943in}}%
\pgfpathlineto{\pgfqpoint{2.478769in}{1.217198in}}%
\pgfpathlineto{\pgfqpoint{2.480503in}{0.749786in}}%
\pgfpathlineto{\pgfqpoint{2.481371in}{0.845490in}}%
\pgfpathlineto{\pgfqpoint{2.482238in}{0.711695in}}%
\pgfpathlineto{\pgfqpoint{2.483105in}{0.781575in}}%
\pgfpathlineto{\pgfqpoint{2.483972in}{0.715318in}}%
\pgfpathlineto{\pgfqpoint{2.484839in}{0.737867in}}%
\pgfpathlineto{\pgfqpoint{2.486573in}{0.824175in}}%
\pgfpathlineto{\pgfqpoint{2.488308in}{0.731518in}}%
\pgfpathlineto{\pgfqpoint{2.489175in}{1.033760in}}%
\pgfpathlineto{\pgfqpoint{2.490042in}{0.956930in}}%
\pgfpathlineto{\pgfqpoint{2.490909in}{0.712422in}}%
\pgfpathlineto{\pgfqpoint{2.491776in}{0.747821in}}%
\pgfpathlineto{\pgfqpoint{2.493510in}{0.720624in}}%
\pgfpathlineto{\pgfqpoint{2.494378in}{0.917688in}}%
\pgfpathlineto{\pgfqpoint{2.496112in}{0.718823in}}%
\pgfpathlineto{\pgfqpoint{2.496979in}{0.877441in}}%
\pgfpathlineto{\pgfqpoint{2.497846in}{0.876976in}}%
\pgfpathlineto{\pgfqpoint{2.498713in}{0.731508in}}%
\pgfpathlineto{\pgfqpoint{2.499580in}{0.764989in}}%
\pgfpathlineto{\pgfqpoint{2.501315in}{0.864583in}}%
\pgfpathlineto{\pgfqpoint{2.502182in}{0.825813in}}%
\pgfpathlineto{\pgfqpoint{2.503049in}{0.753229in}}%
\pgfpathlineto{\pgfqpoint{2.503916in}{0.865062in}}%
\pgfpathlineto{\pgfqpoint{2.504783in}{0.702473in}}%
\pgfpathlineto{\pgfqpoint{2.505650in}{0.915424in}}%
\pgfpathlineto{\pgfqpoint{2.506517in}{0.885156in}}%
\pgfpathlineto{\pgfqpoint{2.508252in}{0.706303in}}%
\pgfpathlineto{\pgfqpoint{2.509119in}{0.752924in}}%
\pgfpathlineto{\pgfqpoint{2.509986in}{0.697313in}}%
\pgfpathlineto{\pgfqpoint{2.511720in}{0.897497in}}%
\pgfpathlineto{\pgfqpoint{2.513455in}{0.706949in}}%
\pgfpathlineto{\pgfqpoint{2.516056in}{1.013508in}}%
\pgfpathlineto{\pgfqpoint{2.516923in}{0.952511in}}%
\pgfpathlineto{\pgfqpoint{2.517790in}{0.772140in}}%
\pgfpathlineto{\pgfqpoint{2.518657in}{1.215761in}}%
\pgfpathlineto{\pgfqpoint{2.520392in}{0.702522in}}%
\pgfpathlineto{\pgfqpoint{2.521259in}{0.739295in}}%
\pgfpathlineto{\pgfqpoint{2.522126in}{0.724119in}}%
\pgfpathlineto{\pgfqpoint{2.522993in}{0.736255in}}%
\pgfpathlineto{\pgfqpoint{2.523860in}{0.821058in}}%
\pgfpathlineto{\pgfqpoint{2.524727in}{0.739879in}}%
\pgfpathlineto{\pgfqpoint{2.525594in}{0.802881in}}%
\pgfpathlineto{\pgfqpoint{2.526462in}{0.796345in}}%
\pgfpathlineto{\pgfqpoint{2.527329in}{0.698981in}}%
\pgfpathlineto{\pgfqpoint{2.528196in}{0.715834in}}%
\pgfpathlineto{\pgfqpoint{2.529063in}{0.772801in}}%
\pgfpathlineto{\pgfqpoint{2.529930in}{0.771359in}}%
\pgfpathlineto{\pgfqpoint{2.530797in}{0.715237in}}%
\pgfpathlineto{\pgfqpoint{2.531664in}{0.871895in}}%
\pgfpathlineto{\pgfqpoint{2.533399in}{0.732038in}}%
\pgfpathlineto{\pgfqpoint{2.534266in}{0.730240in}}%
\pgfpathlineto{\pgfqpoint{2.535133in}{0.784749in}}%
\pgfpathlineto{\pgfqpoint{2.536000in}{0.769189in}}%
\pgfpathlineto{\pgfqpoint{2.536867in}{0.703841in}}%
\pgfpathlineto{\pgfqpoint{2.538601in}{0.773905in}}%
\pgfpathlineto{\pgfqpoint{2.541203in}{0.706719in}}%
\pgfpathlineto{\pgfqpoint{2.542070in}{0.706283in}}%
\pgfpathlineto{\pgfqpoint{2.542937in}{0.719477in}}%
\pgfpathlineto{\pgfqpoint{2.543804in}{0.712447in}}%
\pgfpathlineto{\pgfqpoint{2.544671in}{0.737271in}}%
\pgfpathlineto{\pgfqpoint{2.545538in}{0.915497in}}%
\pgfpathlineto{\pgfqpoint{2.547273in}{0.710498in}}%
\pgfpathlineto{\pgfqpoint{2.548140in}{0.717429in}}%
\pgfpathlineto{\pgfqpoint{2.549007in}{0.701973in}}%
\pgfpathlineto{\pgfqpoint{2.549874in}{0.710178in}}%
\pgfpathlineto{\pgfqpoint{2.550741in}{0.778421in}}%
\pgfpathlineto{\pgfqpoint{2.551608in}{0.769974in}}%
\pgfpathlineto{\pgfqpoint{2.552476in}{0.717442in}}%
\pgfpathlineto{\pgfqpoint{2.554210in}{0.818635in}}%
\pgfpathlineto{\pgfqpoint{2.555077in}{0.716585in}}%
\pgfpathlineto{\pgfqpoint{2.556811in}{1.066103in}}%
\pgfpathlineto{\pgfqpoint{2.557678in}{0.741693in}}%
\pgfpathlineto{\pgfqpoint{2.558545in}{0.757874in}}%
\pgfpathlineto{\pgfqpoint{2.559413in}{0.726236in}}%
\pgfpathlineto{\pgfqpoint{2.561147in}{0.820605in}}%
\pgfpathlineto{\pgfqpoint{2.562014in}{0.698948in}}%
\pgfpathlineto{\pgfqpoint{2.562881in}{0.772453in}}%
\pgfpathlineto{\pgfqpoint{2.563748in}{0.714419in}}%
\pgfpathlineto{\pgfqpoint{2.564615in}{0.725090in}}%
\pgfpathlineto{\pgfqpoint{2.565483in}{0.699546in}}%
\pgfpathlineto{\pgfqpoint{2.567217in}{0.874660in}}%
\pgfpathlineto{\pgfqpoint{2.568951in}{0.716993in}}%
\pgfpathlineto{\pgfqpoint{2.570685in}{0.825060in}}%
\pgfpathlineto{\pgfqpoint{2.572420in}{0.790859in}}%
\pgfpathlineto{\pgfqpoint{2.574154in}{0.700421in}}%
\pgfpathlineto{\pgfqpoint{2.575021in}{0.709515in}}%
\pgfpathlineto{\pgfqpoint{2.575888in}{0.704621in}}%
\pgfpathlineto{\pgfqpoint{2.576755in}{0.760530in}}%
\pgfpathlineto{\pgfqpoint{2.578490in}{0.977320in}}%
\pgfpathlineto{\pgfqpoint{2.580224in}{0.741571in}}%
\pgfpathlineto{\pgfqpoint{2.581958in}{1.101192in}}%
\pgfpathlineto{\pgfqpoint{2.583692in}{0.711058in}}%
\pgfpathlineto{\pgfqpoint{2.584559in}{0.781979in}}%
\pgfpathlineto{\pgfqpoint{2.585427in}{1.064753in}}%
\pgfpathlineto{\pgfqpoint{2.586294in}{1.051228in}}%
\pgfpathlineto{\pgfqpoint{2.587161in}{0.901925in}}%
\pgfpathlineto{\pgfqpoint{2.588895in}{1.006070in}}%
\pgfpathlineto{\pgfqpoint{2.590629in}{0.743982in}}%
\pgfpathlineto{\pgfqpoint{2.591497in}{0.927407in}}%
\pgfpathlineto{\pgfqpoint{2.592364in}{0.751506in}}%
\pgfpathlineto{\pgfqpoint{2.594098in}{0.983322in}}%
\pgfpathlineto{\pgfqpoint{2.594965in}{0.717982in}}%
\pgfpathlineto{\pgfqpoint{2.595832in}{0.756986in}}%
\pgfpathlineto{\pgfqpoint{2.596699in}{0.720451in}}%
\pgfpathlineto{\pgfqpoint{2.598434in}{0.777060in}}%
\pgfpathlineto{\pgfqpoint{2.599301in}{0.749116in}}%
\pgfpathlineto{\pgfqpoint{2.601035in}{0.775952in}}%
\pgfpathlineto{\pgfqpoint{2.602769in}{0.708469in}}%
\pgfpathlineto{\pgfqpoint{2.604503in}{0.748692in}}%
\pgfpathlineto{\pgfqpoint{2.605371in}{0.715217in}}%
\pgfpathlineto{\pgfqpoint{2.606238in}{0.725037in}}%
\pgfpathlineto{\pgfqpoint{2.607105in}{0.810946in}}%
\pgfpathlineto{\pgfqpoint{2.608839in}{0.725170in}}%
\pgfpathlineto{\pgfqpoint{2.609706in}{0.776962in}}%
\pgfpathlineto{\pgfqpoint{2.610573in}{0.700447in}}%
\pgfpathlineto{\pgfqpoint{2.611441in}{0.806646in}}%
\pgfpathlineto{\pgfqpoint{2.612308in}{0.739669in}}%
\pgfpathlineto{\pgfqpoint{2.613175in}{0.784001in}}%
\pgfpathlineto{\pgfqpoint{2.614042in}{0.921059in}}%
\pgfpathlineto{\pgfqpoint{2.615776in}{0.703916in}}%
\pgfpathlineto{\pgfqpoint{2.617510in}{0.916201in}}%
\pgfpathlineto{\pgfqpoint{2.619245in}{0.699030in}}%
\pgfpathlineto{\pgfqpoint{2.620979in}{0.769633in}}%
\pgfpathlineto{\pgfqpoint{2.622713in}{0.737333in}}%
\pgfpathlineto{\pgfqpoint{2.623580in}{0.772296in}}%
\pgfpathlineto{\pgfqpoint{2.624448in}{0.760711in}}%
\pgfpathlineto{\pgfqpoint{2.625315in}{0.700659in}}%
\pgfpathlineto{\pgfqpoint{2.626182in}{0.921998in}}%
\pgfpathlineto{\pgfqpoint{2.627049in}{0.914838in}}%
\pgfpathlineto{\pgfqpoint{2.627916in}{0.730820in}}%
\pgfpathlineto{\pgfqpoint{2.628783in}{1.090482in}}%
\pgfpathlineto{\pgfqpoint{2.629650in}{1.000120in}}%
\pgfpathlineto{\pgfqpoint{2.630517in}{0.769597in}}%
\pgfpathlineto{\pgfqpoint{2.631385in}{0.804382in}}%
\pgfpathlineto{\pgfqpoint{2.632252in}{0.790952in}}%
\pgfpathlineto{\pgfqpoint{2.633119in}{0.980194in}}%
\pgfpathlineto{\pgfqpoint{2.633986in}{0.899939in}}%
\pgfpathlineto{\pgfqpoint{2.634853in}{0.700484in}}%
\pgfpathlineto{\pgfqpoint{2.635720in}{0.743496in}}%
\pgfpathlineto{\pgfqpoint{2.637455in}{0.755839in}}%
\pgfpathlineto{\pgfqpoint{2.638322in}{0.698328in}}%
\pgfpathlineto{\pgfqpoint{2.640056in}{0.944339in}}%
\pgfpathlineto{\pgfqpoint{2.641790in}{0.703455in}}%
\pgfpathlineto{\pgfqpoint{2.642657in}{0.786044in}}%
\pgfpathlineto{\pgfqpoint{2.643524in}{0.782518in}}%
\pgfpathlineto{\pgfqpoint{2.645259in}{0.700260in}}%
\pgfpathlineto{\pgfqpoint{2.646126in}{0.852822in}}%
\pgfpathlineto{\pgfqpoint{2.646993in}{0.815886in}}%
\pgfpathlineto{\pgfqpoint{2.647860in}{0.703451in}}%
\pgfpathlineto{\pgfqpoint{2.648727in}{0.823896in}}%
\pgfpathlineto{\pgfqpoint{2.649594in}{0.811613in}}%
\pgfpathlineto{\pgfqpoint{2.650462in}{0.906981in}}%
\pgfpathlineto{\pgfqpoint{2.651329in}{0.804430in}}%
\pgfpathlineto{\pgfqpoint{2.653063in}{0.968629in}}%
\pgfpathlineto{\pgfqpoint{2.653930in}{0.703773in}}%
\pgfpathlineto{\pgfqpoint{2.654797in}{0.880187in}}%
\pgfpathlineto{\pgfqpoint{2.655664in}{0.801821in}}%
\pgfpathlineto{\pgfqpoint{2.656531in}{0.816457in}}%
\pgfpathlineto{\pgfqpoint{2.657399in}{0.832217in}}%
\pgfpathlineto{\pgfqpoint{2.658266in}{0.758251in}}%
\pgfpathlineto{\pgfqpoint{2.660000in}{0.814600in}}%
\pgfpathlineto{\pgfqpoint{2.660867in}{0.760768in}}%
\pgfpathlineto{\pgfqpoint{2.661734in}{0.843552in}}%
\pgfpathlineto{\pgfqpoint{2.663469in}{0.706060in}}%
\pgfpathlineto{\pgfqpoint{2.665203in}{0.933689in}}%
\pgfpathlineto{\pgfqpoint{2.666070in}{0.843612in}}%
\pgfpathlineto{\pgfqpoint{2.666937in}{0.885714in}}%
\pgfpathlineto{\pgfqpoint{2.667804in}{0.864666in}}%
\pgfpathlineto{\pgfqpoint{2.668671in}{0.709582in}}%
\pgfpathlineto{\pgfqpoint{2.669538in}{0.745267in}}%
\pgfpathlineto{\pgfqpoint{2.671273in}{0.852791in}}%
\pgfpathlineto{\pgfqpoint{2.672140in}{0.820250in}}%
\pgfpathlineto{\pgfqpoint{2.673007in}{0.797394in}}%
\pgfpathlineto{\pgfqpoint{2.673874in}{0.850626in}}%
\pgfpathlineto{\pgfqpoint{2.674741in}{0.766797in}}%
\pgfpathlineto{\pgfqpoint{2.676476in}{0.990136in}}%
\pgfpathlineto{\pgfqpoint{2.678210in}{0.704284in}}%
\pgfpathlineto{\pgfqpoint{2.679077in}{0.743319in}}%
\pgfpathlineto{\pgfqpoint{2.679944in}{0.734068in}}%
\pgfpathlineto{\pgfqpoint{2.680811in}{0.729016in}}%
\pgfpathlineto{\pgfqpoint{2.681678in}{0.743518in}}%
\pgfpathlineto{\pgfqpoint{2.682545in}{0.739253in}}%
\pgfpathlineto{\pgfqpoint{2.683413in}{0.718165in}}%
\pgfpathlineto{\pgfqpoint{2.684280in}{0.746203in}}%
\pgfpathlineto{\pgfqpoint{2.685147in}{0.837014in}}%
\pgfpathlineto{\pgfqpoint{2.686014in}{0.816122in}}%
\pgfpathlineto{\pgfqpoint{2.686881in}{0.753887in}}%
\pgfpathlineto{\pgfqpoint{2.687748in}{0.775261in}}%
\pgfpathlineto{\pgfqpoint{2.688615in}{0.711870in}}%
\pgfpathlineto{\pgfqpoint{2.690350in}{0.902295in}}%
\pgfpathlineto{\pgfqpoint{2.692084in}{0.753830in}}%
\pgfpathlineto{\pgfqpoint{2.692951in}{0.981244in}}%
\pgfpathlineto{\pgfqpoint{2.694685in}{0.711171in}}%
\pgfpathlineto{\pgfqpoint{2.696420in}{0.864062in}}%
\pgfpathlineto{\pgfqpoint{2.698154in}{0.738089in}}%
\pgfpathlineto{\pgfqpoint{2.699021in}{0.777505in}}%
\pgfpathlineto{\pgfqpoint{2.699888in}{0.895354in}}%
\pgfpathlineto{\pgfqpoint{2.700755in}{0.873971in}}%
\pgfpathlineto{\pgfqpoint{2.701622in}{0.750105in}}%
\pgfpathlineto{\pgfqpoint{2.703357in}{1.116502in}}%
\pgfpathlineto{\pgfqpoint{2.705091in}{0.707416in}}%
\pgfpathlineto{\pgfqpoint{2.706825in}{0.837189in}}%
\pgfpathlineto{\pgfqpoint{2.707692in}{0.737401in}}%
\pgfpathlineto{\pgfqpoint{2.708559in}{0.751136in}}%
\pgfpathlineto{\pgfqpoint{2.709427in}{0.816186in}}%
\pgfpathlineto{\pgfqpoint{2.711161in}{0.734248in}}%
\pgfpathlineto{\pgfqpoint{2.712895in}{0.743494in}}%
\pgfpathlineto{\pgfqpoint{2.713762in}{0.739437in}}%
\pgfpathlineto{\pgfqpoint{2.714629in}{0.712996in}}%
\pgfpathlineto{\pgfqpoint{2.715497in}{0.721342in}}%
\pgfpathlineto{\pgfqpoint{2.716364in}{0.775357in}}%
\pgfpathlineto{\pgfqpoint{2.717231in}{0.755477in}}%
\pgfpathlineto{\pgfqpoint{2.718098in}{0.704858in}}%
\pgfpathlineto{\pgfqpoint{2.719832in}{0.757179in}}%
\pgfpathlineto{\pgfqpoint{2.720699in}{0.698380in}}%
\pgfpathlineto{\pgfqpoint{2.722434in}{0.921648in}}%
\pgfpathlineto{\pgfqpoint{2.723301in}{0.797419in}}%
\pgfpathlineto{\pgfqpoint{2.724168in}{0.829201in}}%
\pgfpathlineto{\pgfqpoint{2.725902in}{0.720281in}}%
\pgfpathlineto{\pgfqpoint{2.727636in}{1.094906in}}%
\pgfpathlineto{\pgfqpoint{2.729371in}{0.748684in}}%
\pgfpathlineto{\pgfqpoint{2.730238in}{0.796540in}}%
\pgfpathlineto{\pgfqpoint{2.731105in}{0.801599in}}%
\pgfpathlineto{\pgfqpoint{2.731972in}{0.727023in}}%
\pgfpathlineto{\pgfqpoint{2.732839in}{0.730847in}}%
\pgfpathlineto{\pgfqpoint{2.733706in}{0.891671in}}%
\pgfpathlineto{\pgfqpoint{2.735441in}{0.745465in}}%
\pgfpathlineto{\pgfqpoint{2.736308in}{0.952352in}}%
\pgfpathlineto{\pgfqpoint{2.738042in}{0.742410in}}%
\pgfpathlineto{\pgfqpoint{2.738909in}{0.848568in}}%
\pgfpathlineto{\pgfqpoint{2.739776in}{0.705993in}}%
\pgfpathlineto{\pgfqpoint{2.741510in}{0.795586in}}%
\pgfpathlineto{\pgfqpoint{2.743245in}{0.770355in}}%
\pgfpathlineto{\pgfqpoint{2.744112in}{0.723479in}}%
\pgfpathlineto{\pgfqpoint{2.745846in}{0.881448in}}%
\pgfpathlineto{\pgfqpoint{2.746713in}{0.806672in}}%
\pgfpathlineto{\pgfqpoint{2.747580in}{0.896903in}}%
\pgfpathlineto{\pgfqpoint{2.748448in}{0.863400in}}%
\pgfpathlineto{\pgfqpoint{2.749315in}{0.701138in}}%
\pgfpathlineto{\pgfqpoint{2.750182in}{0.735218in}}%
\pgfpathlineto{\pgfqpoint{2.751049in}{0.714598in}}%
\pgfpathlineto{\pgfqpoint{2.752783in}{0.773083in}}%
\pgfpathlineto{\pgfqpoint{2.754517in}{0.708699in}}%
\pgfpathlineto{\pgfqpoint{2.755385in}{0.762622in}}%
\pgfpathlineto{\pgfqpoint{2.756252in}{0.723193in}}%
\pgfpathlineto{\pgfqpoint{2.757986in}{0.892859in}}%
\pgfpathlineto{\pgfqpoint{2.759720in}{0.718131in}}%
\pgfpathlineto{\pgfqpoint{2.760587in}{0.706121in}}%
\pgfpathlineto{\pgfqpoint{2.761455in}{0.763058in}}%
\pgfpathlineto{\pgfqpoint{2.762322in}{0.730267in}}%
\pgfpathlineto{\pgfqpoint{2.764056in}{0.919878in}}%
\pgfpathlineto{\pgfqpoint{2.764923in}{0.722978in}}%
\pgfpathlineto{\pgfqpoint{2.765790in}{0.755513in}}%
\pgfpathlineto{\pgfqpoint{2.766657in}{0.758395in}}%
\pgfpathlineto{\pgfqpoint{2.767524in}{0.696522in}}%
\pgfpathlineto{\pgfqpoint{2.769259in}{0.751055in}}%
\pgfpathlineto{\pgfqpoint{2.770126in}{0.744153in}}%
\pgfpathlineto{\pgfqpoint{2.770993in}{0.698905in}}%
\pgfpathlineto{\pgfqpoint{2.771860in}{0.722456in}}%
\pgfpathlineto{\pgfqpoint{2.772727in}{0.790506in}}%
\pgfpathlineto{\pgfqpoint{2.774462in}{0.707713in}}%
\pgfpathlineto{\pgfqpoint{2.775329in}{0.762683in}}%
\pgfpathlineto{\pgfqpoint{2.776196in}{0.726120in}}%
\pgfpathlineto{\pgfqpoint{2.777930in}{0.827008in}}%
\pgfpathlineto{\pgfqpoint{2.779664in}{0.709647in}}%
\pgfpathlineto{\pgfqpoint{2.780531in}{0.812138in}}%
\pgfpathlineto{\pgfqpoint{2.781399in}{0.807448in}}%
\pgfpathlineto{\pgfqpoint{2.784000in}{0.712787in}}%
\pgfpathlineto{\pgfqpoint{2.784867in}{0.734745in}}%
\pgfpathlineto{\pgfqpoint{2.785734in}{0.727226in}}%
\pgfpathlineto{\pgfqpoint{2.786601in}{0.745269in}}%
\pgfpathlineto{\pgfqpoint{2.787469in}{0.729802in}}%
\pgfpathlineto{\pgfqpoint{2.789203in}{0.912842in}}%
\pgfpathlineto{\pgfqpoint{2.790070in}{0.740122in}}%
\pgfpathlineto{\pgfqpoint{2.790937in}{0.778965in}}%
\pgfpathlineto{\pgfqpoint{2.793538in}{0.701380in}}%
\pgfpathlineto{\pgfqpoint{2.795273in}{0.950342in}}%
\pgfpathlineto{\pgfqpoint{2.796140in}{0.737777in}}%
\pgfpathlineto{\pgfqpoint{2.797007in}{0.752282in}}%
\pgfpathlineto{\pgfqpoint{2.797874in}{0.825753in}}%
\pgfpathlineto{\pgfqpoint{2.798741in}{0.704214in}}%
\pgfpathlineto{\pgfqpoint{2.799608in}{0.758443in}}%
\pgfpathlineto{\pgfqpoint{2.800476in}{0.717114in}}%
\pgfpathlineto{\pgfqpoint{2.802210in}{0.805987in}}%
\pgfpathlineto{\pgfqpoint{2.803077in}{0.724854in}}%
\pgfpathlineto{\pgfqpoint{2.803944in}{0.749735in}}%
\pgfpathlineto{\pgfqpoint{2.804811in}{0.735264in}}%
\pgfpathlineto{\pgfqpoint{2.805678in}{0.701664in}}%
\pgfpathlineto{\pgfqpoint{2.806545in}{0.714855in}}%
\pgfpathlineto{\pgfqpoint{2.807413in}{0.705422in}}%
\pgfpathlineto{\pgfqpoint{2.808280in}{0.817273in}}%
\pgfpathlineto{\pgfqpoint{2.810014in}{0.719809in}}%
\pgfpathlineto{\pgfqpoint{2.810881in}{0.781644in}}%
\pgfpathlineto{\pgfqpoint{2.811748in}{0.720063in}}%
\pgfpathlineto{\pgfqpoint{2.812615in}{0.755592in}}%
\pgfpathlineto{\pgfqpoint{2.813483in}{0.932647in}}%
\pgfpathlineto{\pgfqpoint{2.815217in}{0.726846in}}%
\pgfpathlineto{\pgfqpoint{2.816084in}{0.741965in}}%
\pgfpathlineto{\pgfqpoint{2.816951in}{0.731720in}}%
\pgfpathlineto{\pgfqpoint{2.817818in}{0.704267in}}%
\pgfpathlineto{\pgfqpoint{2.819552in}{1.072382in}}%
\pgfpathlineto{\pgfqpoint{2.820420in}{0.755937in}}%
\pgfpathlineto{\pgfqpoint{2.821287in}{0.809049in}}%
\pgfpathlineto{\pgfqpoint{2.822154in}{0.780610in}}%
\pgfpathlineto{\pgfqpoint{2.823021in}{0.796146in}}%
\pgfpathlineto{\pgfqpoint{2.823888in}{0.700494in}}%
\pgfpathlineto{\pgfqpoint{2.824755in}{0.811225in}}%
\pgfpathlineto{\pgfqpoint{2.825622in}{0.718284in}}%
\pgfpathlineto{\pgfqpoint{2.827357in}{0.808237in}}%
\pgfpathlineto{\pgfqpoint{2.828224in}{0.784562in}}%
\pgfpathlineto{\pgfqpoint{2.829091in}{0.721994in}}%
\pgfpathlineto{\pgfqpoint{2.829958in}{0.822229in}}%
\pgfpathlineto{\pgfqpoint{2.830825in}{0.813046in}}%
\pgfpathlineto{\pgfqpoint{2.832559in}{0.735957in}}%
\pgfpathlineto{\pgfqpoint{2.833427in}{0.769312in}}%
\pgfpathlineto{\pgfqpoint{2.834294in}{0.901377in}}%
\pgfpathlineto{\pgfqpoint{2.835161in}{0.722948in}}%
\pgfpathlineto{\pgfqpoint{2.836028in}{0.771562in}}%
\pgfpathlineto{\pgfqpoint{2.836895in}{0.836515in}}%
\pgfpathlineto{\pgfqpoint{2.837762in}{0.702382in}}%
\pgfpathlineto{\pgfqpoint{2.839497in}{0.802629in}}%
\pgfpathlineto{\pgfqpoint{2.840364in}{0.699063in}}%
\pgfpathlineto{\pgfqpoint{2.842098in}{0.792076in}}%
\pgfpathlineto{\pgfqpoint{2.843832in}{0.696606in}}%
\pgfpathlineto{\pgfqpoint{2.845566in}{0.755824in}}%
\pgfpathlineto{\pgfqpoint{2.846434in}{0.696878in}}%
\pgfpathlineto{\pgfqpoint{2.847301in}{0.715413in}}%
\pgfpathlineto{\pgfqpoint{2.848168in}{0.702656in}}%
\pgfpathlineto{\pgfqpoint{2.849035in}{0.716284in}}%
\pgfpathlineto{\pgfqpoint{2.849902in}{0.701313in}}%
\pgfpathlineto{\pgfqpoint{2.850769in}{0.750963in}}%
\pgfpathlineto{\pgfqpoint{2.852503in}{0.720995in}}%
\pgfpathlineto{\pgfqpoint{2.853371in}{0.793949in}}%
\pgfpathlineto{\pgfqpoint{2.854238in}{0.700518in}}%
\pgfpathlineto{\pgfqpoint{2.855972in}{0.825629in}}%
\pgfpathlineto{\pgfqpoint{2.856839in}{0.746235in}}%
\pgfpathlineto{\pgfqpoint{2.857706in}{0.756138in}}%
\pgfpathlineto{\pgfqpoint{2.858573in}{0.772086in}}%
\pgfpathlineto{\pgfqpoint{2.859441in}{0.832084in}}%
\pgfpathlineto{\pgfqpoint{2.861175in}{0.703859in}}%
\pgfpathlineto{\pgfqpoint{2.862909in}{0.774721in}}%
\pgfpathlineto{\pgfqpoint{2.863776in}{0.731005in}}%
\pgfpathlineto{\pgfqpoint{2.865510in}{0.995148in}}%
\pgfpathlineto{\pgfqpoint{2.866378in}{0.848546in}}%
\pgfpathlineto{\pgfqpoint{2.867245in}{0.939136in}}%
\pgfpathlineto{\pgfqpoint{2.868979in}{0.788921in}}%
\pgfpathlineto{\pgfqpoint{2.870713in}{0.866654in}}%
\pgfpathlineto{\pgfqpoint{2.871580in}{0.728762in}}%
\pgfpathlineto{\pgfqpoint{2.872448in}{0.815241in}}%
\pgfpathlineto{\pgfqpoint{2.873315in}{1.024738in}}%
\pgfpathlineto{\pgfqpoint{2.875049in}{0.697898in}}%
\pgfpathlineto{\pgfqpoint{2.876783in}{0.768516in}}%
\pgfpathlineto{\pgfqpoint{2.878517in}{0.699618in}}%
\pgfpathlineto{\pgfqpoint{2.880252in}{0.728154in}}%
\pgfpathlineto{\pgfqpoint{2.881119in}{0.721574in}}%
\pgfpathlineto{\pgfqpoint{2.881986in}{0.938029in}}%
\pgfpathlineto{\pgfqpoint{2.882853in}{0.933034in}}%
\pgfpathlineto{\pgfqpoint{2.883720in}{0.752951in}}%
\pgfpathlineto{\pgfqpoint{2.885455in}{0.989704in}}%
\pgfpathlineto{\pgfqpoint{2.887189in}{0.707412in}}%
\pgfpathlineto{\pgfqpoint{2.888056in}{0.797275in}}%
\pgfpathlineto{\pgfqpoint{2.888923in}{0.713012in}}%
\pgfpathlineto{\pgfqpoint{2.889790in}{0.798122in}}%
\pgfpathlineto{\pgfqpoint{2.890657in}{0.797945in}}%
\pgfpathlineto{\pgfqpoint{2.891524in}{0.856882in}}%
\pgfpathlineto{\pgfqpoint{2.892392in}{0.851482in}}%
\pgfpathlineto{\pgfqpoint{2.893259in}{0.711934in}}%
\pgfpathlineto{\pgfqpoint{2.894126in}{0.946670in}}%
\pgfpathlineto{\pgfqpoint{2.894993in}{0.760700in}}%
\pgfpathlineto{\pgfqpoint{2.895860in}{0.761321in}}%
\pgfpathlineto{\pgfqpoint{2.896727in}{0.775366in}}%
\pgfpathlineto{\pgfqpoint{2.897594in}{0.720214in}}%
\pgfpathlineto{\pgfqpoint{2.898462in}{0.767903in}}%
\pgfpathlineto{\pgfqpoint{2.899329in}{0.696468in}}%
\pgfpathlineto{\pgfqpoint{2.900196in}{0.725965in}}%
\pgfpathlineto{\pgfqpoint{2.901063in}{0.699520in}}%
\pgfpathlineto{\pgfqpoint{2.901930in}{0.746774in}}%
\pgfpathlineto{\pgfqpoint{2.902797in}{0.724298in}}%
\pgfpathlineto{\pgfqpoint{2.903664in}{0.867559in}}%
\pgfpathlineto{\pgfqpoint{2.905399in}{0.741872in}}%
\pgfpathlineto{\pgfqpoint{2.906266in}{0.888540in}}%
\pgfpathlineto{\pgfqpoint{2.908000in}{0.777647in}}%
\pgfpathlineto{\pgfqpoint{2.908867in}{0.779320in}}%
\pgfpathlineto{\pgfqpoint{2.909734in}{0.715703in}}%
\pgfpathlineto{\pgfqpoint{2.910601in}{0.718812in}}%
\pgfpathlineto{\pgfqpoint{2.912336in}{0.875440in}}%
\pgfpathlineto{\pgfqpoint{2.914070in}{0.705934in}}%
\pgfpathlineto{\pgfqpoint{2.914937in}{0.737639in}}%
\pgfpathlineto{\pgfqpoint{2.915804in}{0.925881in}}%
\pgfpathlineto{\pgfqpoint{2.916671in}{0.773979in}}%
\pgfpathlineto{\pgfqpoint{2.917538in}{0.868997in}}%
\pgfpathlineto{\pgfqpoint{2.918406in}{1.138121in}}%
\pgfpathlineto{\pgfqpoint{2.919273in}{0.777883in}}%
\pgfpathlineto{\pgfqpoint{2.920140in}{0.859004in}}%
\pgfpathlineto{\pgfqpoint{2.921007in}{0.976538in}}%
\pgfpathlineto{\pgfqpoint{2.921874in}{0.850939in}}%
\pgfpathlineto{\pgfqpoint{2.922741in}{0.952253in}}%
\pgfpathlineto{\pgfqpoint{2.923608in}{0.779517in}}%
\pgfpathlineto{\pgfqpoint{2.924476in}{0.920983in}}%
\pgfpathlineto{\pgfqpoint{2.925343in}{1.445361in}}%
\pgfpathlineto{\pgfqpoint{2.927077in}{0.824289in}}%
\pgfpathlineto{\pgfqpoint{2.927944in}{1.165600in}}%
\pgfpathlineto{\pgfqpoint{2.929678in}{0.837570in}}%
\pgfpathlineto{\pgfqpoint{2.930545in}{0.847437in}}%
\pgfpathlineto{\pgfqpoint{2.931413in}{0.728011in}}%
\pgfpathlineto{\pgfqpoint{2.932280in}{0.767159in}}%
\pgfpathlineto{\pgfqpoint{2.933147in}{0.717507in}}%
\pgfpathlineto{\pgfqpoint{2.934014in}{0.804034in}}%
\pgfpathlineto{\pgfqpoint{2.934881in}{0.758561in}}%
\pgfpathlineto{\pgfqpoint{2.936615in}{0.916906in}}%
\pgfpathlineto{\pgfqpoint{2.937483in}{0.708698in}}%
\pgfpathlineto{\pgfqpoint{2.938350in}{0.815326in}}%
\pgfpathlineto{\pgfqpoint{2.939217in}{0.809235in}}%
\pgfpathlineto{\pgfqpoint{2.940084in}{0.713332in}}%
\pgfpathlineto{\pgfqpoint{2.943552in}{0.905928in}}%
\pgfpathlineto{\pgfqpoint{2.945287in}{0.707939in}}%
\pgfpathlineto{\pgfqpoint{2.946154in}{0.812687in}}%
\pgfpathlineto{\pgfqpoint{2.947021in}{0.808684in}}%
\pgfpathlineto{\pgfqpoint{2.947888in}{0.715515in}}%
\pgfpathlineto{\pgfqpoint{2.948755in}{0.721926in}}%
\pgfpathlineto{\pgfqpoint{2.949622in}{0.712524in}}%
\pgfpathlineto{\pgfqpoint{2.950490in}{0.734423in}}%
\pgfpathlineto{\pgfqpoint{2.951357in}{0.889385in}}%
\pgfpathlineto{\pgfqpoint{2.953091in}{0.742772in}}%
\pgfpathlineto{\pgfqpoint{2.953958in}{0.854182in}}%
\pgfpathlineto{\pgfqpoint{2.954825in}{0.749857in}}%
\pgfpathlineto{\pgfqpoint{2.955692in}{0.768221in}}%
\pgfpathlineto{\pgfqpoint{2.956559in}{0.745025in}}%
\pgfpathlineto{\pgfqpoint{2.957427in}{0.859411in}}%
\pgfpathlineto{\pgfqpoint{2.958294in}{0.811645in}}%
\pgfpathlineto{\pgfqpoint{2.959161in}{0.818525in}}%
\pgfpathlineto{\pgfqpoint{2.960028in}{1.032843in}}%
\pgfpathlineto{\pgfqpoint{2.960895in}{0.808015in}}%
\pgfpathlineto{\pgfqpoint{2.961762in}{0.831062in}}%
\pgfpathlineto{\pgfqpoint{2.962629in}{1.052805in}}%
\pgfpathlineto{\pgfqpoint{2.964364in}{0.721321in}}%
\pgfpathlineto{\pgfqpoint{2.965231in}{0.788456in}}%
\pgfpathlineto{\pgfqpoint{2.966098in}{0.880741in}}%
\pgfpathlineto{\pgfqpoint{2.967832in}{0.710719in}}%
\pgfpathlineto{\pgfqpoint{2.968699in}{0.741775in}}%
\pgfpathlineto{\pgfqpoint{2.970434in}{0.749004in}}%
\pgfpathlineto{\pgfqpoint{2.972168in}{1.099052in}}%
\pgfpathlineto{\pgfqpoint{2.973035in}{0.786748in}}%
\pgfpathlineto{\pgfqpoint{2.974769in}{1.198164in}}%
\pgfpathlineto{\pgfqpoint{2.976503in}{0.722271in}}%
\pgfpathlineto{\pgfqpoint{2.977371in}{0.770917in}}%
\pgfpathlineto{\pgfqpoint{2.979105in}{0.740678in}}%
\pgfpathlineto{\pgfqpoint{2.979972in}{0.924500in}}%
\pgfpathlineto{\pgfqpoint{2.981706in}{0.811158in}}%
\pgfpathlineto{\pgfqpoint{2.982573in}{1.301120in}}%
\pgfpathlineto{\pgfqpoint{2.984308in}{0.763670in}}%
\pgfpathlineto{\pgfqpoint{2.986042in}{0.908704in}}%
\pgfpathlineto{\pgfqpoint{2.986909in}{0.953504in}}%
\pgfpathlineto{\pgfqpoint{2.987776in}{0.708881in}}%
\pgfpathlineto{\pgfqpoint{2.988643in}{1.215105in}}%
\pgfpathlineto{\pgfqpoint{2.989510in}{1.214561in}}%
\pgfpathlineto{\pgfqpoint{2.990378in}{0.731320in}}%
\pgfpathlineto{\pgfqpoint{2.991245in}{0.733363in}}%
\pgfpathlineto{\pgfqpoint{2.992112in}{0.729041in}}%
\pgfpathlineto{\pgfqpoint{2.992979in}{0.814422in}}%
\pgfpathlineto{\pgfqpoint{2.994713in}{1.558516in}}%
\pgfpathlineto{\pgfqpoint{2.995580in}{1.571585in}}%
\pgfpathlineto{\pgfqpoint{2.997315in}{4.012636in}}%
\pgfpathlineto{\pgfqpoint{2.998182in}{1.777842in}}%
\pgfpathlineto{\pgfqpoint{2.999916in}{3.668377in}}%
\pgfpathlineto{\pgfqpoint{3.001650in}{1.033613in}}%
\pgfpathlineto{\pgfqpoint{3.002517in}{1.263222in}}%
\pgfpathlineto{\pgfqpoint{3.004252in}{0.764255in}}%
\pgfpathlineto{\pgfqpoint{3.005986in}{0.744600in}}%
\pgfpathlineto{\pgfqpoint{3.006853in}{1.198779in}}%
\pgfpathlineto{\pgfqpoint{3.007720in}{1.172922in}}%
\pgfpathlineto{\pgfqpoint{3.008587in}{0.752068in}}%
\pgfpathlineto{\pgfqpoint{3.010322in}{0.940679in}}%
\pgfpathlineto{\pgfqpoint{3.011189in}{0.704044in}}%
\pgfpathlineto{\pgfqpoint{3.012056in}{0.849493in}}%
\pgfpathlineto{\pgfqpoint{3.012923in}{0.817766in}}%
\pgfpathlineto{\pgfqpoint{3.013790in}{0.716218in}}%
\pgfpathlineto{\pgfqpoint{3.015524in}{1.147737in}}%
\pgfpathlineto{\pgfqpoint{3.017259in}{0.758986in}}%
\pgfpathlineto{\pgfqpoint{3.018993in}{1.025743in}}%
\pgfpathlineto{\pgfqpoint{3.019860in}{0.701666in}}%
\pgfpathlineto{\pgfqpoint{3.021594in}{1.350071in}}%
\pgfpathlineto{\pgfqpoint{3.022462in}{0.699557in}}%
\pgfpathlineto{\pgfqpoint{3.023329in}{0.976171in}}%
\pgfpathlineto{\pgfqpoint{3.025063in}{0.759405in}}%
\pgfpathlineto{\pgfqpoint{3.025930in}{0.777278in}}%
\pgfpathlineto{\pgfqpoint{3.026797in}{0.702586in}}%
\pgfpathlineto{\pgfqpoint{3.027664in}{0.717494in}}%
\pgfpathlineto{\pgfqpoint{3.028531in}{0.704279in}}%
\pgfpathlineto{\pgfqpoint{3.029399in}{0.728063in}}%
\pgfpathlineto{\pgfqpoint{3.030266in}{0.720830in}}%
\pgfpathlineto{\pgfqpoint{3.031133in}{0.697397in}}%
\pgfpathlineto{\pgfqpoint{3.032000in}{0.711184in}}%
\pgfpathlineto{\pgfqpoint{3.032867in}{0.706456in}}%
\pgfpathlineto{\pgfqpoint{3.033734in}{0.760559in}}%
\pgfpathlineto{\pgfqpoint{3.034601in}{0.704669in}}%
\pgfpathlineto{\pgfqpoint{3.035469in}{0.896013in}}%
\pgfpathlineto{\pgfqpoint{3.037203in}{0.756911in}}%
\pgfpathlineto{\pgfqpoint{3.038070in}{0.819835in}}%
\pgfpathlineto{\pgfqpoint{3.038937in}{0.725385in}}%
\pgfpathlineto{\pgfqpoint{3.039804in}{0.740167in}}%
\pgfpathlineto{\pgfqpoint{3.040671in}{0.708929in}}%
\pgfpathlineto{\pgfqpoint{3.041538in}{0.777008in}}%
\pgfpathlineto{\pgfqpoint{3.042406in}{0.721453in}}%
\pgfpathlineto{\pgfqpoint{3.043273in}{1.044759in}}%
\pgfpathlineto{\pgfqpoint{3.045007in}{0.749680in}}%
\pgfpathlineto{\pgfqpoint{3.045874in}{1.055593in}}%
\pgfpathlineto{\pgfqpoint{3.046741in}{0.980624in}}%
\pgfpathlineto{\pgfqpoint{3.048476in}{0.714397in}}%
\pgfpathlineto{\pgfqpoint{3.049343in}{0.725819in}}%
\pgfpathlineto{\pgfqpoint{3.050210in}{0.712037in}}%
\pgfpathlineto{\pgfqpoint{3.051077in}{0.722413in}}%
\pgfpathlineto{\pgfqpoint{3.051944in}{0.699223in}}%
\pgfpathlineto{\pgfqpoint{3.053678in}{0.740466in}}%
\pgfpathlineto{\pgfqpoint{3.056280in}{0.987957in}}%
\pgfpathlineto{\pgfqpoint{3.057147in}{0.995229in}}%
\pgfpathlineto{\pgfqpoint{3.060615in}{0.730892in}}%
\pgfpathlineto{\pgfqpoint{3.061483in}{0.754648in}}%
\pgfpathlineto{\pgfqpoint{3.063217in}{0.775949in}}%
\pgfpathlineto{\pgfqpoint{3.064084in}{0.854651in}}%
\pgfpathlineto{\pgfqpoint{3.064951in}{0.758788in}}%
\pgfpathlineto{\pgfqpoint{3.066685in}{1.001998in}}%
\pgfpathlineto{\pgfqpoint{3.068420in}{0.734396in}}%
\pgfpathlineto{\pgfqpoint{3.069287in}{0.746941in}}%
\pgfpathlineto{\pgfqpoint{3.070154in}{0.791984in}}%
\pgfpathlineto{\pgfqpoint{3.071888in}{1.082388in}}%
\pgfpathlineto{\pgfqpoint{3.072755in}{0.804209in}}%
\pgfpathlineto{\pgfqpoint{3.073622in}{0.862570in}}%
\pgfpathlineto{\pgfqpoint{3.074490in}{0.707411in}}%
\pgfpathlineto{\pgfqpoint{3.075357in}{0.882293in}}%
\pgfpathlineto{\pgfqpoint{3.076224in}{0.848049in}}%
\pgfpathlineto{\pgfqpoint{3.077091in}{0.742656in}}%
\pgfpathlineto{\pgfqpoint{3.077958in}{1.107000in}}%
\pgfpathlineto{\pgfqpoint{3.080559in}{0.718685in}}%
\pgfpathlineto{\pgfqpoint{3.081427in}{0.773183in}}%
\pgfpathlineto{\pgfqpoint{3.082294in}{0.743243in}}%
\pgfpathlineto{\pgfqpoint{3.084028in}{0.900046in}}%
\pgfpathlineto{\pgfqpoint{3.084895in}{0.735205in}}%
\pgfpathlineto{\pgfqpoint{3.085762in}{0.961934in}}%
\pgfpathlineto{\pgfqpoint{3.086629in}{0.945032in}}%
\pgfpathlineto{\pgfqpoint{3.087497in}{0.730022in}}%
\pgfpathlineto{\pgfqpoint{3.089231in}{0.926423in}}%
\pgfpathlineto{\pgfqpoint{3.090098in}{0.746775in}}%
\pgfpathlineto{\pgfqpoint{3.090965in}{1.083272in}}%
\pgfpathlineto{\pgfqpoint{3.091832in}{1.053673in}}%
\pgfpathlineto{\pgfqpoint{3.092699in}{0.807474in}}%
\pgfpathlineto{\pgfqpoint{3.093566in}{0.829045in}}%
\pgfpathlineto{\pgfqpoint{3.094434in}{0.742076in}}%
\pgfpathlineto{\pgfqpoint{3.095301in}{0.744018in}}%
\pgfpathlineto{\pgfqpoint{3.097035in}{1.149389in}}%
\pgfpathlineto{\pgfqpoint{3.098769in}{0.767179in}}%
\pgfpathlineto{\pgfqpoint{3.099636in}{0.900221in}}%
\pgfpathlineto{\pgfqpoint{3.100503in}{0.860667in}}%
\pgfpathlineto{\pgfqpoint{3.101371in}{0.736708in}}%
\pgfpathlineto{\pgfqpoint{3.103105in}{1.069254in}}%
\pgfpathlineto{\pgfqpoint{3.103972in}{0.808067in}}%
\pgfpathlineto{\pgfqpoint{3.104839in}{1.115713in}}%
\pgfpathlineto{\pgfqpoint{3.105706in}{1.030315in}}%
\pgfpathlineto{\pgfqpoint{3.106573in}{0.719018in}}%
\pgfpathlineto{\pgfqpoint{3.107441in}{1.041561in}}%
\pgfpathlineto{\pgfqpoint{3.109175in}{0.775314in}}%
\pgfpathlineto{\pgfqpoint{3.110042in}{0.854710in}}%
\pgfpathlineto{\pgfqpoint{3.111776in}{0.780682in}}%
\pgfpathlineto{\pgfqpoint{3.112643in}{1.244919in}}%
\pgfpathlineto{\pgfqpoint{3.114378in}{0.716231in}}%
\pgfpathlineto{\pgfqpoint{3.115245in}{0.716840in}}%
\pgfpathlineto{\pgfqpoint{3.116112in}{0.741409in}}%
\pgfpathlineto{\pgfqpoint{3.116979in}{0.709566in}}%
\pgfpathlineto{\pgfqpoint{3.117846in}{1.021290in}}%
\pgfpathlineto{\pgfqpoint{3.118713in}{0.998498in}}%
\pgfpathlineto{\pgfqpoint{3.119580in}{0.717921in}}%
\pgfpathlineto{\pgfqpoint{3.120448in}{1.167820in}}%
\pgfpathlineto{\pgfqpoint{3.122182in}{0.721903in}}%
\pgfpathlineto{\pgfqpoint{3.123916in}{0.777773in}}%
\pgfpathlineto{\pgfqpoint{3.124783in}{0.759011in}}%
\pgfpathlineto{\pgfqpoint{3.125650in}{0.892426in}}%
\pgfpathlineto{\pgfqpoint{3.127385in}{0.739213in}}%
\pgfpathlineto{\pgfqpoint{3.128252in}{0.785291in}}%
\pgfpathlineto{\pgfqpoint{3.129119in}{0.790356in}}%
\pgfpathlineto{\pgfqpoint{3.129986in}{0.769826in}}%
\pgfpathlineto{\pgfqpoint{3.130853in}{0.793309in}}%
\pgfpathlineto{\pgfqpoint{3.132587in}{0.697558in}}%
\pgfpathlineto{\pgfqpoint{3.134322in}{0.850304in}}%
\pgfpathlineto{\pgfqpoint{3.136056in}{0.701203in}}%
\pgfpathlineto{\pgfqpoint{3.137790in}{0.892752in}}%
\pgfpathlineto{\pgfqpoint{3.139524in}{0.759902in}}%
\pgfpathlineto{\pgfqpoint{3.140392in}{0.828102in}}%
\pgfpathlineto{\pgfqpoint{3.142126in}{0.703375in}}%
\pgfpathlineto{\pgfqpoint{3.142993in}{0.747789in}}%
\pgfpathlineto{\pgfqpoint{3.144727in}{0.707189in}}%
\pgfpathlineto{\pgfqpoint{3.145594in}{0.710564in}}%
\pgfpathlineto{\pgfqpoint{3.147329in}{0.726254in}}%
\pgfpathlineto{\pgfqpoint{3.148196in}{0.703391in}}%
\pgfpathlineto{\pgfqpoint{3.149063in}{0.779260in}}%
\pgfpathlineto{\pgfqpoint{3.149930in}{0.758001in}}%
\pgfpathlineto{\pgfqpoint{3.150797in}{0.755058in}}%
\pgfpathlineto{\pgfqpoint{3.151664in}{0.810700in}}%
\pgfpathlineto{\pgfqpoint{3.152531in}{0.730710in}}%
\pgfpathlineto{\pgfqpoint{3.153399in}{0.744774in}}%
\pgfpathlineto{\pgfqpoint{3.154266in}{0.749081in}}%
\pgfpathlineto{\pgfqpoint{3.155133in}{0.700788in}}%
\pgfpathlineto{\pgfqpoint{3.156000in}{0.721630in}}%
\pgfpathlineto{\pgfqpoint{3.156867in}{0.705215in}}%
\pgfpathlineto{\pgfqpoint{3.157734in}{0.839408in}}%
\pgfpathlineto{\pgfqpoint{3.159469in}{0.733404in}}%
\pgfpathlineto{\pgfqpoint{3.160336in}{0.909423in}}%
\pgfpathlineto{\pgfqpoint{3.161203in}{0.803605in}}%
\pgfpathlineto{\pgfqpoint{3.162070in}{0.866198in}}%
\pgfpathlineto{\pgfqpoint{3.162937in}{0.793945in}}%
\pgfpathlineto{\pgfqpoint{3.163804in}{0.800724in}}%
\pgfpathlineto{\pgfqpoint{3.164671in}{0.762414in}}%
\pgfpathlineto{\pgfqpoint{3.166406in}{0.897110in}}%
\pgfpathlineto{\pgfqpoint{3.167273in}{0.697844in}}%
\pgfpathlineto{\pgfqpoint{3.168140in}{0.951817in}}%
\pgfpathlineto{\pgfqpoint{3.169007in}{0.911538in}}%
\pgfpathlineto{\pgfqpoint{3.170741in}{0.716987in}}%
\pgfpathlineto{\pgfqpoint{3.171608in}{0.723502in}}%
\pgfpathlineto{\pgfqpoint{3.173343in}{1.067693in}}%
\pgfpathlineto{\pgfqpoint{3.174210in}{0.786558in}}%
\pgfpathlineto{\pgfqpoint{3.175077in}{0.980102in}}%
\pgfpathlineto{\pgfqpoint{3.175944in}{0.725627in}}%
\pgfpathlineto{\pgfqpoint{3.176811in}{0.851487in}}%
\pgfpathlineto{\pgfqpoint{3.177678in}{0.807014in}}%
\pgfpathlineto{\pgfqpoint{3.178545in}{0.953197in}}%
\pgfpathlineto{\pgfqpoint{3.179413in}{0.920916in}}%
\pgfpathlineto{\pgfqpoint{3.180280in}{0.696208in}}%
\pgfpathlineto{\pgfqpoint{3.181147in}{0.741850in}}%
\pgfpathlineto{\pgfqpoint{3.182881in}{0.711184in}}%
\pgfpathlineto{\pgfqpoint{3.184615in}{0.770407in}}%
\pgfpathlineto{\pgfqpoint{3.186350in}{0.702039in}}%
\pgfpathlineto{\pgfqpoint{3.187217in}{0.734242in}}%
\pgfpathlineto{\pgfqpoint{3.188084in}{0.732960in}}%
\pgfpathlineto{\pgfqpoint{3.188951in}{0.741831in}}%
\pgfpathlineto{\pgfqpoint{3.189818in}{0.711856in}}%
\pgfpathlineto{\pgfqpoint{3.190685in}{0.826280in}}%
\pgfpathlineto{\pgfqpoint{3.191552in}{0.738155in}}%
\pgfpathlineto{\pgfqpoint{3.193287in}{1.157527in}}%
\pgfpathlineto{\pgfqpoint{3.194154in}{0.705094in}}%
\pgfpathlineto{\pgfqpoint{3.195021in}{0.933061in}}%
\pgfpathlineto{\pgfqpoint{3.195888in}{0.723468in}}%
\pgfpathlineto{\pgfqpoint{3.196755in}{0.942930in}}%
\pgfpathlineto{\pgfqpoint{3.197622in}{0.782177in}}%
\pgfpathlineto{\pgfqpoint{3.199357in}{1.050531in}}%
\pgfpathlineto{\pgfqpoint{3.200224in}{0.706407in}}%
\pgfpathlineto{\pgfqpoint{3.201091in}{0.764078in}}%
\pgfpathlineto{\pgfqpoint{3.201958in}{0.720918in}}%
\pgfpathlineto{\pgfqpoint{3.202825in}{0.749790in}}%
\pgfpathlineto{\pgfqpoint{3.204559in}{1.006351in}}%
\pgfpathlineto{\pgfqpoint{3.205427in}{0.975063in}}%
\pgfpathlineto{\pgfqpoint{3.206294in}{1.018696in}}%
\pgfpathlineto{\pgfqpoint{3.208028in}{0.899614in}}%
\pgfpathlineto{\pgfqpoint{3.208895in}{1.019746in}}%
\pgfpathlineto{\pgfqpoint{3.210629in}{0.756133in}}%
\pgfpathlineto{\pgfqpoint{3.211497in}{0.839126in}}%
\pgfpathlineto{\pgfqpoint{3.212364in}{0.743580in}}%
\pgfpathlineto{\pgfqpoint{3.213231in}{0.870241in}}%
\pgfpathlineto{\pgfqpoint{3.214098in}{0.846714in}}%
\pgfpathlineto{\pgfqpoint{3.214965in}{0.733881in}}%
\pgfpathlineto{\pgfqpoint{3.215832in}{0.942148in}}%
\pgfpathlineto{\pgfqpoint{3.216699in}{0.885384in}}%
\pgfpathlineto{\pgfqpoint{3.217566in}{0.737280in}}%
\pgfpathlineto{\pgfqpoint{3.218434in}{0.775419in}}%
\pgfpathlineto{\pgfqpoint{3.219301in}{0.780431in}}%
\pgfpathlineto{\pgfqpoint{3.220168in}{0.732047in}}%
\pgfpathlineto{\pgfqpoint{3.221902in}{0.752741in}}%
\pgfpathlineto{\pgfqpoint{3.222769in}{0.729426in}}%
\pgfpathlineto{\pgfqpoint{3.223636in}{0.733101in}}%
\pgfpathlineto{\pgfqpoint{3.224503in}{0.894137in}}%
\pgfpathlineto{\pgfqpoint{3.226238in}{0.754857in}}%
\pgfpathlineto{\pgfqpoint{3.227105in}{0.779752in}}%
\pgfpathlineto{\pgfqpoint{3.227972in}{0.770657in}}%
\pgfpathlineto{\pgfqpoint{3.228839in}{0.917836in}}%
\pgfpathlineto{\pgfqpoint{3.230573in}{0.722250in}}%
\pgfpathlineto{\pgfqpoint{3.231441in}{0.728404in}}%
\pgfpathlineto{\pgfqpoint{3.233175in}{0.845034in}}%
\pgfpathlineto{\pgfqpoint{3.234909in}{0.737645in}}%
\pgfpathlineto{\pgfqpoint{3.236643in}{0.871707in}}%
\pgfpathlineto{\pgfqpoint{3.238378in}{0.720911in}}%
\pgfpathlineto{\pgfqpoint{3.239245in}{0.829036in}}%
\pgfpathlineto{\pgfqpoint{3.240112in}{0.723456in}}%
\pgfpathlineto{\pgfqpoint{3.240979in}{0.736934in}}%
\pgfpathlineto{\pgfqpoint{3.241846in}{0.779681in}}%
\pgfpathlineto{\pgfqpoint{3.242713in}{0.716192in}}%
\pgfpathlineto{\pgfqpoint{3.243580in}{0.758369in}}%
\pgfpathlineto{\pgfqpoint{3.244448in}{0.703772in}}%
\pgfpathlineto{\pgfqpoint{3.245315in}{0.752924in}}%
\pgfpathlineto{\pgfqpoint{3.246182in}{0.702548in}}%
\pgfpathlineto{\pgfqpoint{3.247049in}{0.762738in}}%
\pgfpathlineto{\pgfqpoint{3.247916in}{0.716866in}}%
\pgfpathlineto{\pgfqpoint{3.249650in}{0.842573in}}%
\pgfpathlineto{\pgfqpoint{3.250517in}{0.759014in}}%
\pgfpathlineto{\pgfqpoint{3.251385in}{0.840829in}}%
\pgfpathlineto{\pgfqpoint{3.252252in}{0.705582in}}%
\pgfpathlineto{\pgfqpoint{3.253986in}{0.976994in}}%
\pgfpathlineto{\pgfqpoint{3.255720in}{0.712613in}}%
\pgfpathlineto{\pgfqpoint{3.258322in}{0.875813in}}%
\pgfpathlineto{\pgfqpoint{3.259189in}{0.747607in}}%
\pgfpathlineto{\pgfqpoint{3.260923in}{0.885387in}}%
\pgfpathlineto{\pgfqpoint{3.261790in}{0.731459in}}%
\pgfpathlineto{\pgfqpoint{3.262657in}{0.804032in}}%
\pgfpathlineto{\pgfqpoint{3.263524in}{0.696942in}}%
\pgfpathlineto{\pgfqpoint{3.265259in}{1.123870in}}%
\pgfpathlineto{\pgfqpoint{3.266993in}{0.792230in}}%
\pgfpathlineto{\pgfqpoint{3.267860in}{0.701703in}}%
\pgfpathlineto{\pgfqpoint{3.268727in}{0.718168in}}%
\pgfpathlineto{\pgfqpoint{3.269594in}{0.734940in}}%
\pgfpathlineto{\pgfqpoint{3.270462in}{1.027455in}}%
\pgfpathlineto{\pgfqpoint{3.272196in}{0.806001in}}%
\pgfpathlineto{\pgfqpoint{3.273063in}{0.843235in}}%
\pgfpathlineto{\pgfqpoint{3.273930in}{0.704149in}}%
\pgfpathlineto{\pgfqpoint{3.274797in}{0.720932in}}%
\pgfpathlineto{\pgfqpoint{3.275664in}{0.708756in}}%
\pgfpathlineto{\pgfqpoint{3.276531in}{0.781422in}}%
\pgfpathlineto{\pgfqpoint{3.278266in}{1.228303in}}%
\pgfpathlineto{\pgfqpoint{3.280000in}{0.829241in}}%
\pgfpathlineto{\pgfqpoint{3.280867in}{1.162733in}}%
\pgfpathlineto{\pgfqpoint{3.282601in}{0.698016in}}%
\pgfpathlineto{\pgfqpoint{3.283469in}{0.706554in}}%
\pgfpathlineto{\pgfqpoint{3.284336in}{0.749088in}}%
\pgfpathlineto{\pgfqpoint{3.285203in}{0.704566in}}%
\pgfpathlineto{\pgfqpoint{3.286070in}{0.863786in}}%
\pgfpathlineto{\pgfqpoint{3.286937in}{0.831401in}}%
\pgfpathlineto{\pgfqpoint{3.287804in}{0.724441in}}%
\pgfpathlineto{\pgfqpoint{3.288671in}{0.983381in}}%
\pgfpathlineto{\pgfqpoint{3.291273in}{0.771632in}}%
\pgfpathlineto{\pgfqpoint{3.293007in}{0.865606in}}%
\pgfpathlineto{\pgfqpoint{3.294741in}{0.748311in}}%
\pgfpathlineto{\pgfqpoint{3.295608in}{0.984724in}}%
\pgfpathlineto{\pgfqpoint{3.297343in}{0.700344in}}%
\pgfpathlineto{\pgfqpoint{3.299077in}{0.798414in}}%
\pgfpathlineto{\pgfqpoint{3.299944in}{0.709515in}}%
\pgfpathlineto{\pgfqpoint{3.300811in}{1.216141in}}%
\pgfpathlineto{\pgfqpoint{3.301678in}{1.215650in}}%
\pgfpathlineto{\pgfqpoint{3.302545in}{0.820950in}}%
\pgfpathlineto{\pgfqpoint{3.303413in}{1.018771in}}%
\pgfpathlineto{\pgfqpoint{3.304280in}{0.828496in}}%
\pgfpathlineto{\pgfqpoint{3.305147in}{0.928711in}}%
\pgfpathlineto{\pgfqpoint{3.306014in}{0.911829in}}%
\pgfpathlineto{\pgfqpoint{3.306881in}{0.723286in}}%
\pgfpathlineto{\pgfqpoint{3.308615in}{0.858905in}}%
\pgfpathlineto{\pgfqpoint{3.310350in}{0.698116in}}%
\pgfpathlineto{\pgfqpoint{3.311217in}{0.699390in}}%
\pgfpathlineto{\pgfqpoint{3.312084in}{0.711668in}}%
\pgfpathlineto{\pgfqpoint{3.313818in}{0.796436in}}%
\pgfpathlineto{\pgfqpoint{3.314685in}{0.700882in}}%
\pgfpathlineto{\pgfqpoint{3.315552in}{0.867995in}}%
\pgfpathlineto{\pgfqpoint{3.316420in}{0.739081in}}%
\pgfpathlineto{\pgfqpoint{3.318154in}{0.950645in}}%
\pgfpathlineto{\pgfqpoint{3.319021in}{0.784066in}}%
\pgfpathlineto{\pgfqpoint{3.319888in}{0.896555in}}%
\pgfpathlineto{\pgfqpoint{3.320755in}{0.802839in}}%
\pgfpathlineto{\pgfqpoint{3.321622in}{0.807158in}}%
\pgfpathlineto{\pgfqpoint{3.323357in}{0.732144in}}%
\pgfpathlineto{\pgfqpoint{3.324224in}{0.767163in}}%
\pgfpathlineto{\pgfqpoint{3.325091in}{0.715842in}}%
\pgfpathlineto{\pgfqpoint{3.326825in}{0.924332in}}%
\pgfpathlineto{\pgfqpoint{3.328559in}{0.711847in}}%
\pgfpathlineto{\pgfqpoint{3.331161in}{0.836182in}}%
\pgfpathlineto{\pgfqpoint{3.332028in}{1.076158in}}%
\pgfpathlineto{\pgfqpoint{3.333762in}{0.837410in}}%
\pgfpathlineto{\pgfqpoint{3.334629in}{0.905639in}}%
\pgfpathlineto{\pgfqpoint{3.336364in}{0.704596in}}%
\pgfpathlineto{\pgfqpoint{3.337231in}{0.699752in}}%
\pgfpathlineto{\pgfqpoint{3.338098in}{0.708389in}}%
\pgfpathlineto{\pgfqpoint{3.339832in}{0.801709in}}%
\pgfpathlineto{\pgfqpoint{3.340699in}{0.709538in}}%
\pgfpathlineto{\pgfqpoint{3.341566in}{0.945797in}}%
\pgfpathlineto{\pgfqpoint{3.342434in}{0.923280in}}%
\pgfpathlineto{\pgfqpoint{3.343301in}{0.716699in}}%
\pgfpathlineto{\pgfqpoint{3.345035in}{0.822781in}}%
\pgfpathlineto{\pgfqpoint{3.346769in}{0.698885in}}%
\pgfpathlineto{\pgfqpoint{3.347636in}{0.696014in}}%
\pgfpathlineto{\pgfqpoint{3.348503in}{0.697471in}}%
\pgfpathlineto{\pgfqpoint{3.349371in}{0.705656in}}%
\pgfpathlineto{\pgfqpoint{3.351105in}{0.879312in}}%
\pgfpathlineto{\pgfqpoint{3.351972in}{0.731269in}}%
\pgfpathlineto{\pgfqpoint{3.352839in}{0.780415in}}%
\pgfpathlineto{\pgfqpoint{3.353706in}{0.904549in}}%
\pgfpathlineto{\pgfqpoint{3.354573in}{0.788164in}}%
\pgfpathlineto{\pgfqpoint{3.355441in}{0.817389in}}%
\pgfpathlineto{\pgfqpoint{3.357175in}{0.703092in}}%
\pgfpathlineto{\pgfqpoint{3.358909in}{0.739952in}}%
\pgfpathlineto{\pgfqpoint{3.359776in}{0.702461in}}%
\pgfpathlineto{\pgfqpoint{3.360643in}{0.782720in}}%
\pgfpathlineto{\pgfqpoint{3.361510in}{0.779926in}}%
\pgfpathlineto{\pgfqpoint{3.362378in}{0.739507in}}%
\pgfpathlineto{\pgfqpoint{3.363245in}{0.831748in}}%
\pgfpathlineto{\pgfqpoint{3.364979in}{0.699735in}}%
\pgfpathlineto{\pgfqpoint{3.365846in}{0.723611in}}%
\pgfpathlineto{\pgfqpoint{3.366713in}{0.711383in}}%
\pgfpathlineto{\pgfqpoint{3.367580in}{0.773700in}}%
\pgfpathlineto{\pgfqpoint{3.368448in}{1.164641in}}%
\pgfpathlineto{\pgfqpoint{3.370182in}{0.876150in}}%
\pgfpathlineto{\pgfqpoint{3.371049in}{1.388528in}}%
\pgfpathlineto{\pgfqpoint{3.371916in}{1.031439in}}%
\pgfpathlineto{\pgfqpoint{3.372783in}{1.096472in}}%
\pgfpathlineto{\pgfqpoint{3.373650in}{1.086915in}}%
\pgfpathlineto{\pgfqpoint{3.374517in}{0.696257in}}%
\pgfpathlineto{\pgfqpoint{3.375385in}{0.943215in}}%
\pgfpathlineto{\pgfqpoint{3.376252in}{0.767965in}}%
\pgfpathlineto{\pgfqpoint{3.377119in}{0.811650in}}%
\pgfpathlineto{\pgfqpoint{3.377986in}{1.077460in}}%
\pgfpathlineto{\pgfqpoint{3.380587in}{0.760086in}}%
\pgfpathlineto{\pgfqpoint{3.382322in}{0.988808in}}%
\pgfpathlineto{\pgfqpoint{3.383189in}{0.721916in}}%
\pgfpathlineto{\pgfqpoint{3.384056in}{0.788716in}}%
\pgfpathlineto{\pgfqpoint{3.384923in}{0.830079in}}%
\pgfpathlineto{\pgfqpoint{3.385790in}{0.722201in}}%
\pgfpathlineto{\pgfqpoint{3.386657in}{0.724189in}}%
\pgfpathlineto{\pgfqpoint{3.387524in}{0.755812in}}%
\pgfpathlineto{\pgfqpoint{3.388392in}{0.752299in}}%
\pgfpathlineto{\pgfqpoint{3.389259in}{0.747947in}}%
\pgfpathlineto{\pgfqpoint{3.390126in}{0.729430in}}%
\pgfpathlineto{\pgfqpoint{3.391860in}{0.923738in}}%
\pgfpathlineto{\pgfqpoint{3.392727in}{0.863436in}}%
\pgfpathlineto{\pgfqpoint{3.393594in}{0.721896in}}%
\pgfpathlineto{\pgfqpoint{3.395329in}{0.879676in}}%
\pgfpathlineto{\pgfqpoint{3.396196in}{0.836965in}}%
\pgfpathlineto{\pgfqpoint{3.397063in}{0.914316in}}%
\pgfpathlineto{\pgfqpoint{3.398797in}{0.741025in}}%
\pgfpathlineto{\pgfqpoint{3.399664in}{0.806430in}}%
\pgfpathlineto{\pgfqpoint{3.400531in}{0.705803in}}%
\pgfpathlineto{\pgfqpoint{3.401399in}{0.747012in}}%
\pgfpathlineto{\pgfqpoint{3.402266in}{0.699764in}}%
\pgfpathlineto{\pgfqpoint{3.404000in}{0.916098in}}%
\pgfpathlineto{\pgfqpoint{3.404867in}{0.720954in}}%
\pgfpathlineto{\pgfqpoint{3.405734in}{0.754993in}}%
\pgfpathlineto{\pgfqpoint{3.406601in}{0.768882in}}%
\pgfpathlineto{\pgfqpoint{3.408336in}{0.700278in}}%
\pgfpathlineto{\pgfqpoint{3.409203in}{0.701685in}}%
\pgfpathlineto{\pgfqpoint{3.410070in}{0.727615in}}%
\pgfpathlineto{\pgfqpoint{3.410937in}{0.704717in}}%
\pgfpathlineto{\pgfqpoint{3.413538in}{0.858552in}}%
\pgfpathlineto{\pgfqpoint{3.415273in}{0.726635in}}%
\pgfpathlineto{\pgfqpoint{3.417007in}{0.946297in}}%
\pgfpathlineto{\pgfqpoint{3.418741in}{0.697831in}}%
\pgfpathlineto{\pgfqpoint{3.420476in}{0.823195in}}%
\pgfpathlineto{\pgfqpoint{3.421343in}{0.814010in}}%
\pgfpathlineto{\pgfqpoint{3.423077in}{0.728211in}}%
\pgfpathlineto{\pgfqpoint{3.425678in}{0.696108in}}%
\pgfpathlineto{\pgfqpoint{3.427413in}{0.875313in}}%
\pgfpathlineto{\pgfqpoint{3.428280in}{0.721583in}}%
\pgfpathlineto{\pgfqpoint{3.430014in}{1.019251in}}%
\pgfpathlineto{\pgfqpoint{3.431748in}{0.705520in}}%
\pgfpathlineto{\pgfqpoint{3.432615in}{0.734544in}}%
\pgfpathlineto{\pgfqpoint{3.434350in}{1.069559in}}%
\pgfpathlineto{\pgfqpoint{3.435217in}{0.990159in}}%
\pgfpathlineto{\pgfqpoint{3.436084in}{0.754954in}}%
\pgfpathlineto{\pgfqpoint{3.436951in}{0.800566in}}%
\pgfpathlineto{\pgfqpoint{3.437818in}{0.789774in}}%
\pgfpathlineto{\pgfqpoint{3.438685in}{0.733272in}}%
\pgfpathlineto{\pgfqpoint{3.439552in}{0.785036in}}%
\pgfpathlineto{\pgfqpoint{3.440420in}{0.780152in}}%
\pgfpathlineto{\pgfqpoint{3.441287in}{0.701217in}}%
\pgfpathlineto{\pgfqpoint{3.442154in}{0.776274in}}%
\pgfpathlineto{\pgfqpoint{3.443888in}{0.703615in}}%
\pgfpathlineto{\pgfqpoint{3.445622in}{0.803999in}}%
\pgfpathlineto{\pgfqpoint{3.446490in}{0.698444in}}%
\pgfpathlineto{\pgfqpoint{3.447357in}{0.909502in}}%
\pgfpathlineto{\pgfqpoint{3.448224in}{0.857195in}}%
\pgfpathlineto{\pgfqpoint{3.449091in}{0.703831in}}%
\pgfpathlineto{\pgfqpoint{3.449958in}{0.725866in}}%
\pgfpathlineto{\pgfqpoint{3.450825in}{0.716773in}}%
\pgfpathlineto{\pgfqpoint{3.452559in}{0.758154in}}%
\pgfpathlineto{\pgfqpoint{3.453427in}{0.872178in}}%
\pgfpathlineto{\pgfqpoint{3.456895in}{0.701428in}}%
\pgfpathlineto{\pgfqpoint{3.457762in}{0.711155in}}%
\pgfpathlineto{\pgfqpoint{3.459497in}{0.759150in}}%
\pgfpathlineto{\pgfqpoint{3.460364in}{0.735023in}}%
\pgfpathlineto{\pgfqpoint{3.461231in}{0.742937in}}%
\pgfpathlineto{\pgfqpoint{3.463832in}{0.801107in}}%
\pgfpathlineto{\pgfqpoint{3.465566in}{1.162833in}}%
\pgfpathlineto{\pgfqpoint{3.466434in}{1.047281in}}%
\pgfpathlineto{\pgfqpoint{3.468168in}{0.724791in}}%
\pgfpathlineto{\pgfqpoint{3.471636in}{1.177658in}}%
\pgfpathlineto{\pgfqpoint{3.472503in}{1.220280in}}%
\pgfpathlineto{\pgfqpoint{3.474238in}{0.907198in}}%
\pgfpathlineto{\pgfqpoint{3.475972in}{0.878583in}}%
\pgfpathlineto{\pgfqpoint{3.477706in}{0.937292in}}%
\pgfpathlineto{\pgfqpoint{3.480308in}{0.730637in}}%
\pgfpathlineto{\pgfqpoint{3.481175in}{0.711645in}}%
\pgfpathlineto{\pgfqpoint{3.482042in}{0.768109in}}%
\pgfpathlineto{\pgfqpoint{3.482909in}{1.069593in}}%
\pgfpathlineto{\pgfqpoint{3.483776in}{1.028285in}}%
\pgfpathlineto{\pgfqpoint{3.486378in}{0.767309in}}%
\pgfpathlineto{\pgfqpoint{3.488112in}{1.101491in}}%
\pgfpathlineto{\pgfqpoint{3.489846in}{0.776830in}}%
\pgfpathlineto{\pgfqpoint{3.491580in}{1.003482in}}%
\pgfpathlineto{\pgfqpoint{3.493315in}{0.720982in}}%
\pgfpathlineto{\pgfqpoint{3.495049in}{1.079739in}}%
\pgfpathlineto{\pgfqpoint{3.496783in}{0.727620in}}%
\pgfpathlineto{\pgfqpoint{3.497650in}{0.952138in}}%
\pgfpathlineto{\pgfqpoint{3.498517in}{0.847512in}}%
\pgfpathlineto{\pgfqpoint{3.499385in}{0.878699in}}%
\pgfpathlineto{\pgfqpoint{3.501119in}{0.712729in}}%
\pgfpathlineto{\pgfqpoint{3.501986in}{0.789431in}}%
\pgfpathlineto{\pgfqpoint{3.502853in}{0.743880in}}%
\pgfpathlineto{\pgfqpoint{3.503720in}{0.840606in}}%
\pgfpathlineto{\pgfqpoint{3.504587in}{1.176579in}}%
\pgfpathlineto{\pgfqpoint{3.505455in}{1.136805in}}%
\pgfpathlineto{\pgfqpoint{3.508056in}{0.771273in}}%
\pgfpathlineto{\pgfqpoint{3.508923in}{0.867088in}}%
\pgfpathlineto{\pgfqpoint{3.509790in}{1.103508in}}%
\pgfpathlineto{\pgfqpoint{3.510657in}{1.077888in}}%
\pgfpathlineto{\pgfqpoint{3.512392in}{0.720372in}}%
\pgfpathlineto{\pgfqpoint{3.514126in}{1.104700in}}%
\pgfpathlineto{\pgfqpoint{3.514993in}{0.835309in}}%
\pgfpathlineto{\pgfqpoint{3.515860in}{0.877488in}}%
\pgfpathlineto{\pgfqpoint{3.516727in}{1.062425in}}%
\pgfpathlineto{\pgfqpoint{3.518462in}{0.740223in}}%
\pgfpathlineto{\pgfqpoint{3.519329in}{0.803194in}}%
\pgfpathlineto{\pgfqpoint{3.521063in}{0.737789in}}%
\pgfpathlineto{\pgfqpoint{3.522797in}{0.927065in}}%
\pgfpathlineto{\pgfqpoint{3.525399in}{0.700315in}}%
\pgfpathlineto{\pgfqpoint{3.527133in}{1.453545in}}%
\pgfpathlineto{\pgfqpoint{3.528867in}{0.779419in}}%
\pgfpathlineto{\pgfqpoint{3.529734in}{0.985225in}}%
\pgfpathlineto{\pgfqpoint{3.530601in}{0.978534in}}%
\pgfpathlineto{\pgfqpoint{3.531469in}{0.782822in}}%
\pgfpathlineto{\pgfqpoint{3.533203in}{1.690515in}}%
\pgfpathlineto{\pgfqpoint{3.534937in}{0.979912in}}%
\pgfpathlineto{\pgfqpoint{3.536671in}{0.708954in}}%
\pgfpathlineto{\pgfqpoint{3.537538in}{0.723621in}}%
\pgfpathlineto{\pgfqpoint{3.538406in}{0.858018in}}%
\pgfpathlineto{\pgfqpoint{3.539273in}{0.843349in}}%
\pgfpathlineto{\pgfqpoint{3.541007in}{0.739859in}}%
\pgfpathlineto{\pgfqpoint{3.541874in}{0.780708in}}%
\pgfpathlineto{\pgfqpoint{3.542741in}{0.956316in}}%
\pgfpathlineto{\pgfqpoint{3.543608in}{0.953460in}}%
\pgfpathlineto{\pgfqpoint{3.544476in}{1.006714in}}%
\pgfpathlineto{\pgfqpoint{3.545343in}{1.339192in}}%
\pgfpathlineto{\pgfqpoint{3.547077in}{0.788163in}}%
\pgfpathlineto{\pgfqpoint{3.548811in}{0.977263in}}%
\pgfpathlineto{\pgfqpoint{3.550545in}{0.756457in}}%
\pgfpathlineto{\pgfqpoint{3.551413in}{0.863642in}}%
\pgfpathlineto{\pgfqpoint{3.552280in}{0.813402in}}%
\pgfpathlineto{\pgfqpoint{3.554014in}{1.203796in}}%
\pgfpathlineto{\pgfqpoint{3.555748in}{0.776676in}}%
\pgfpathlineto{\pgfqpoint{3.556615in}{1.018282in}}%
\pgfpathlineto{\pgfqpoint{3.558350in}{0.713444in}}%
\pgfpathlineto{\pgfqpoint{3.559217in}{0.748572in}}%
\pgfpathlineto{\pgfqpoint{3.560951in}{0.979226in}}%
\pgfpathlineto{\pgfqpoint{3.562685in}{0.702060in}}%
\pgfpathlineto{\pgfqpoint{3.563552in}{0.763372in}}%
\pgfpathlineto{\pgfqpoint{3.564420in}{0.939757in}}%
\pgfpathlineto{\pgfqpoint{3.565287in}{0.901727in}}%
\pgfpathlineto{\pgfqpoint{3.566154in}{0.702203in}}%
\pgfpathlineto{\pgfqpoint{3.567021in}{0.800016in}}%
\pgfpathlineto{\pgfqpoint{3.568755in}{0.698984in}}%
\pgfpathlineto{\pgfqpoint{3.569622in}{0.700355in}}%
\pgfpathlineto{\pgfqpoint{3.570490in}{0.860797in}}%
\pgfpathlineto{\pgfqpoint{3.571357in}{0.842379in}}%
\pgfpathlineto{\pgfqpoint{3.572224in}{0.781960in}}%
\pgfpathlineto{\pgfqpoint{3.573091in}{1.019282in}}%
\pgfpathlineto{\pgfqpoint{3.573958in}{0.869593in}}%
\pgfpathlineto{\pgfqpoint{3.575692in}{1.069234in}}%
\pgfpathlineto{\pgfqpoint{3.576559in}{0.706881in}}%
\pgfpathlineto{\pgfqpoint{3.578294in}{0.946078in}}%
\pgfpathlineto{\pgfqpoint{3.580028in}{0.716263in}}%
\pgfpathlineto{\pgfqpoint{3.580895in}{0.739743in}}%
\pgfpathlineto{\pgfqpoint{3.582629in}{0.719499in}}%
\pgfpathlineto{\pgfqpoint{3.583497in}{0.797559in}}%
\pgfpathlineto{\pgfqpoint{3.585231in}{0.705106in}}%
\pgfpathlineto{\pgfqpoint{3.586098in}{0.777108in}}%
\pgfpathlineto{\pgfqpoint{3.586965in}{0.757898in}}%
\pgfpathlineto{\pgfqpoint{3.587832in}{0.705934in}}%
\pgfpathlineto{\pgfqpoint{3.588699in}{0.766654in}}%
\pgfpathlineto{\pgfqpoint{3.589566in}{0.759079in}}%
\pgfpathlineto{\pgfqpoint{3.591301in}{0.707378in}}%
\pgfpathlineto{\pgfqpoint{3.592168in}{0.757470in}}%
\pgfpathlineto{\pgfqpoint{3.593035in}{0.699067in}}%
\pgfpathlineto{\pgfqpoint{3.593902in}{0.817254in}}%
\pgfpathlineto{\pgfqpoint{3.594769in}{0.793677in}}%
\pgfpathlineto{\pgfqpoint{3.595636in}{0.745442in}}%
\pgfpathlineto{\pgfqpoint{3.597371in}{1.072696in}}%
\pgfpathlineto{\pgfqpoint{3.599105in}{0.780379in}}%
\pgfpathlineto{\pgfqpoint{3.599972in}{0.715334in}}%
\pgfpathlineto{\pgfqpoint{3.601706in}{1.146438in}}%
\pgfpathlineto{\pgfqpoint{3.602573in}{1.039320in}}%
\pgfpathlineto{\pgfqpoint{3.603441in}{1.093277in}}%
\pgfpathlineto{\pgfqpoint{3.604308in}{0.790679in}}%
\pgfpathlineto{\pgfqpoint{3.605175in}{0.899717in}}%
\pgfpathlineto{\pgfqpoint{3.606042in}{0.835691in}}%
\pgfpathlineto{\pgfqpoint{3.606909in}{0.855199in}}%
\pgfpathlineto{\pgfqpoint{3.607776in}{0.932756in}}%
\pgfpathlineto{\pgfqpoint{3.608643in}{0.700551in}}%
\pgfpathlineto{\pgfqpoint{3.610378in}{1.324999in}}%
\pgfpathlineto{\pgfqpoint{3.611245in}{0.839111in}}%
\pgfpathlineto{\pgfqpoint{3.612112in}{0.918488in}}%
\pgfpathlineto{\pgfqpoint{3.612979in}{0.961906in}}%
\pgfpathlineto{\pgfqpoint{3.613846in}{0.709890in}}%
\pgfpathlineto{\pgfqpoint{3.614713in}{0.715022in}}%
\pgfpathlineto{\pgfqpoint{3.615580in}{0.716887in}}%
\pgfpathlineto{\pgfqpoint{3.616448in}{0.937871in}}%
\pgfpathlineto{\pgfqpoint{3.617315in}{0.888977in}}%
\pgfpathlineto{\pgfqpoint{3.618182in}{0.698702in}}%
\pgfpathlineto{\pgfqpoint{3.619049in}{0.741500in}}%
\pgfpathlineto{\pgfqpoint{3.619916in}{0.735820in}}%
\pgfpathlineto{\pgfqpoint{3.621650in}{0.804251in}}%
\pgfpathlineto{\pgfqpoint{3.624252in}{0.696988in}}%
\pgfpathlineto{\pgfqpoint{3.627720in}{0.951043in}}%
\pgfpathlineto{\pgfqpoint{3.629455in}{0.722906in}}%
\pgfpathlineto{\pgfqpoint{3.630322in}{0.789331in}}%
\pgfpathlineto{\pgfqpoint{3.631189in}{0.703734in}}%
\pgfpathlineto{\pgfqpoint{3.632923in}{0.904560in}}%
\pgfpathlineto{\pgfqpoint{3.633790in}{0.709162in}}%
\pgfpathlineto{\pgfqpoint{3.634657in}{0.751019in}}%
\pgfpathlineto{\pgfqpoint{3.635524in}{0.775261in}}%
\pgfpathlineto{\pgfqpoint{3.637259in}{0.709496in}}%
\pgfpathlineto{\pgfqpoint{3.638993in}{0.765155in}}%
\pgfpathlineto{\pgfqpoint{3.639860in}{0.700893in}}%
\pgfpathlineto{\pgfqpoint{3.640727in}{0.733576in}}%
\pgfpathlineto{\pgfqpoint{3.641594in}{0.700483in}}%
\pgfpathlineto{\pgfqpoint{3.643329in}{0.888007in}}%
\pgfpathlineto{\pgfqpoint{3.644196in}{0.818710in}}%
\pgfpathlineto{\pgfqpoint{3.645063in}{0.855629in}}%
\pgfpathlineto{\pgfqpoint{3.645930in}{1.026459in}}%
\pgfpathlineto{\pgfqpoint{3.647664in}{0.751547in}}%
\pgfpathlineto{\pgfqpoint{3.650266in}{0.843162in}}%
\pgfpathlineto{\pgfqpoint{3.651133in}{0.761065in}}%
\pgfpathlineto{\pgfqpoint{3.652000in}{0.763529in}}%
\pgfpathlineto{\pgfqpoint{3.652867in}{0.816226in}}%
\pgfpathlineto{\pgfqpoint{3.653734in}{0.800934in}}%
\pgfpathlineto{\pgfqpoint{3.654601in}{0.704555in}}%
\pgfpathlineto{\pgfqpoint{3.655469in}{0.772462in}}%
\pgfpathlineto{\pgfqpoint{3.657203in}{0.702259in}}%
\pgfpathlineto{\pgfqpoint{3.658937in}{0.696126in}}%
\pgfpathlineto{\pgfqpoint{3.660671in}{0.716253in}}%
\pgfpathlineto{\pgfqpoint{3.661538in}{0.698840in}}%
\pgfpathlineto{\pgfqpoint{3.662406in}{0.712438in}}%
\pgfpathlineto{\pgfqpoint{3.663273in}{0.697921in}}%
\pgfpathlineto{\pgfqpoint{3.664140in}{0.716153in}}%
\pgfpathlineto{\pgfqpoint{3.665007in}{0.700476in}}%
\pgfpathlineto{\pgfqpoint{3.665874in}{0.701385in}}%
\pgfpathlineto{\pgfqpoint{3.666741in}{0.705954in}}%
\pgfpathlineto{\pgfqpoint{3.668476in}{0.822740in}}%
\pgfpathlineto{\pgfqpoint{3.669343in}{0.708171in}}%
\pgfpathlineto{\pgfqpoint{3.671077in}{1.380949in}}%
\pgfpathlineto{\pgfqpoint{3.671944in}{0.839634in}}%
\pgfpathlineto{\pgfqpoint{3.672811in}{0.904800in}}%
\pgfpathlineto{\pgfqpoint{3.673678in}{0.924032in}}%
\pgfpathlineto{\pgfqpoint{3.676280in}{0.713637in}}%
\pgfpathlineto{\pgfqpoint{3.678014in}{0.897989in}}%
\pgfpathlineto{\pgfqpoint{3.678881in}{0.716820in}}%
\pgfpathlineto{\pgfqpoint{3.679748in}{0.956489in}}%
\pgfpathlineto{\pgfqpoint{3.680615in}{0.749273in}}%
\pgfpathlineto{\pgfqpoint{3.682350in}{1.075242in}}%
\pgfpathlineto{\pgfqpoint{3.683217in}{0.838325in}}%
\pgfpathlineto{\pgfqpoint{3.684084in}{1.026338in}}%
\pgfpathlineto{\pgfqpoint{3.685818in}{0.754022in}}%
\pgfpathlineto{\pgfqpoint{3.686685in}{0.932837in}}%
\pgfpathlineto{\pgfqpoint{3.688420in}{0.738183in}}%
\pgfpathlineto{\pgfqpoint{3.689287in}{0.799504in}}%
\pgfpathlineto{\pgfqpoint{3.690154in}{0.696075in}}%
\pgfpathlineto{\pgfqpoint{3.691021in}{0.772241in}}%
\pgfpathlineto{\pgfqpoint{3.691888in}{0.714120in}}%
\pgfpathlineto{\pgfqpoint{3.692755in}{0.729606in}}%
\pgfpathlineto{\pgfqpoint{3.693622in}{0.708536in}}%
\pgfpathlineto{\pgfqpoint{3.694490in}{0.779218in}}%
\pgfpathlineto{\pgfqpoint{3.695357in}{1.004274in}}%
\pgfpathlineto{\pgfqpoint{3.696224in}{0.941982in}}%
\pgfpathlineto{\pgfqpoint{3.697091in}{0.724683in}}%
\pgfpathlineto{\pgfqpoint{3.698825in}{0.939037in}}%
\pgfpathlineto{\pgfqpoint{3.700559in}{0.712261in}}%
\pgfpathlineto{\pgfqpoint{3.702294in}{0.698073in}}%
\pgfpathlineto{\pgfqpoint{3.704028in}{0.698004in}}%
\pgfpathlineto{\pgfqpoint{3.704895in}{0.722940in}}%
\pgfpathlineto{\pgfqpoint{3.706629in}{0.940680in}}%
\pgfpathlineto{\pgfqpoint{3.707497in}{0.697102in}}%
\pgfpathlineto{\pgfqpoint{3.708364in}{1.006561in}}%
\pgfpathlineto{\pgfqpoint{3.709231in}{0.941904in}}%
\pgfpathlineto{\pgfqpoint{3.710098in}{0.728870in}}%
\pgfpathlineto{\pgfqpoint{3.710965in}{0.793685in}}%
\pgfpathlineto{\pgfqpoint{3.712699in}{0.716156in}}%
\pgfpathlineto{\pgfqpoint{3.713566in}{0.716923in}}%
\pgfpathlineto{\pgfqpoint{3.714434in}{0.755237in}}%
\pgfpathlineto{\pgfqpoint{3.715301in}{0.743330in}}%
\pgfpathlineto{\pgfqpoint{3.716168in}{0.764886in}}%
\pgfpathlineto{\pgfqpoint{3.717035in}{0.881250in}}%
\pgfpathlineto{\pgfqpoint{3.717902in}{0.780097in}}%
\pgfpathlineto{\pgfqpoint{3.718769in}{0.781462in}}%
\pgfpathlineto{\pgfqpoint{3.719636in}{0.952264in}}%
\pgfpathlineto{\pgfqpoint{3.720503in}{0.734363in}}%
\pgfpathlineto{\pgfqpoint{3.721371in}{0.940221in}}%
\pgfpathlineto{\pgfqpoint{3.722238in}{0.823198in}}%
\pgfpathlineto{\pgfqpoint{3.723105in}{0.904774in}}%
\pgfpathlineto{\pgfqpoint{3.724839in}{0.751565in}}%
\pgfpathlineto{\pgfqpoint{3.725706in}{0.758906in}}%
\pgfpathlineto{\pgfqpoint{3.727441in}{1.144878in}}%
\pgfpathlineto{\pgfqpoint{3.728308in}{0.836510in}}%
\pgfpathlineto{\pgfqpoint{3.729175in}{1.192634in}}%
\pgfpathlineto{\pgfqpoint{3.730909in}{0.746504in}}%
\pgfpathlineto{\pgfqpoint{3.731776in}{0.943127in}}%
\pgfpathlineto{\pgfqpoint{3.733510in}{0.759788in}}%
\pgfpathlineto{\pgfqpoint{3.735245in}{0.704905in}}%
\pgfpathlineto{\pgfqpoint{3.736979in}{0.782984in}}%
\pgfpathlineto{\pgfqpoint{3.737846in}{0.699128in}}%
\pgfpathlineto{\pgfqpoint{3.739580in}{0.829534in}}%
\pgfpathlineto{\pgfqpoint{3.741315in}{0.709168in}}%
\pgfpathlineto{\pgfqpoint{3.742182in}{0.713217in}}%
\pgfpathlineto{\pgfqpoint{3.743049in}{0.701720in}}%
\pgfpathlineto{\pgfqpoint{3.743916in}{0.755321in}}%
\pgfpathlineto{\pgfqpoint{3.744783in}{0.698321in}}%
\pgfpathlineto{\pgfqpoint{3.745650in}{0.745188in}}%
\pgfpathlineto{\pgfqpoint{3.746517in}{0.704110in}}%
\pgfpathlineto{\pgfqpoint{3.748252in}{0.906922in}}%
\pgfpathlineto{\pgfqpoint{3.749119in}{0.700418in}}%
\pgfpathlineto{\pgfqpoint{3.749986in}{0.728186in}}%
\pgfpathlineto{\pgfqpoint{3.750853in}{0.721193in}}%
\pgfpathlineto{\pgfqpoint{3.751720in}{0.743922in}}%
\pgfpathlineto{\pgfqpoint{3.753455in}{1.064247in}}%
\pgfpathlineto{\pgfqpoint{3.754322in}{0.742632in}}%
\pgfpathlineto{\pgfqpoint{3.755189in}{0.792270in}}%
\pgfpathlineto{\pgfqpoint{3.756056in}{0.837387in}}%
\pgfpathlineto{\pgfqpoint{3.757790in}{0.696458in}}%
\pgfpathlineto{\pgfqpoint{3.758657in}{0.726621in}}%
\pgfpathlineto{\pgfqpoint{3.759524in}{0.803125in}}%
\pgfpathlineto{\pgfqpoint{3.760392in}{0.748325in}}%
\pgfpathlineto{\pgfqpoint{3.761259in}{0.752625in}}%
\pgfpathlineto{\pgfqpoint{3.762993in}{0.928915in}}%
\pgfpathlineto{\pgfqpoint{3.764727in}{0.698840in}}%
\pgfpathlineto{\pgfqpoint{3.765594in}{0.700247in}}%
\pgfpathlineto{\pgfqpoint{3.767329in}{0.927339in}}%
\pgfpathlineto{\pgfqpoint{3.768196in}{0.864280in}}%
\pgfpathlineto{\pgfqpoint{3.769930in}{0.765157in}}%
\pgfpathlineto{\pgfqpoint{3.770797in}{0.826122in}}%
\pgfpathlineto{\pgfqpoint{3.772531in}{0.698737in}}%
\pgfpathlineto{\pgfqpoint{3.773399in}{0.733719in}}%
\pgfpathlineto{\pgfqpoint{3.774266in}{0.769503in}}%
\pgfpathlineto{\pgfqpoint{3.775133in}{0.756246in}}%
\pgfpathlineto{\pgfqpoint{3.776000in}{0.765082in}}%
\pgfpathlineto{\pgfqpoint{3.777734in}{0.827542in}}%
\pgfpathlineto{\pgfqpoint{3.779469in}{0.958904in}}%
\pgfpathlineto{\pgfqpoint{3.780336in}{0.732936in}}%
\pgfpathlineto{\pgfqpoint{3.781203in}{0.829824in}}%
\pgfpathlineto{\pgfqpoint{3.782070in}{1.204329in}}%
\pgfpathlineto{\pgfqpoint{3.783804in}{0.715334in}}%
\pgfpathlineto{\pgfqpoint{3.784671in}{0.828348in}}%
\pgfpathlineto{\pgfqpoint{3.786406in}{0.704565in}}%
\pgfpathlineto{\pgfqpoint{3.787273in}{0.722263in}}%
\pgfpathlineto{\pgfqpoint{3.788140in}{0.867241in}}%
\pgfpathlineto{\pgfqpoint{3.789874in}{0.704961in}}%
\pgfpathlineto{\pgfqpoint{3.790741in}{0.769955in}}%
\pgfpathlineto{\pgfqpoint{3.791608in}{0.709695in}}%
\pgfpathlineto{\pgfqpoint{3.792476in}{0.795905in}}%
\pgfpathlineto{\pgfqpoint{3.793343in}{1.048000in}}%
\pgfpathlineto{\pgfqpoint{3.795077in}{0.712397in}}%
\pgfpathlineto{\pgfqpoint{3.795944in}{0.736708in}}%
\pgfpathlineto{\pgfqpoint{3.796811in}{0.712507in}}%
\pgfpathlineto{\pgfqpoint{3.799413in}{0.841959in}}%
\pgfpathlineto{\pgfqpoint{3.800280in}{0.848760in}}%
\pgfpathlineto{\pgfqpoint{3.801147in}{0.762522in}}%
\pgfpathlineto{\pgfqpoint{3.802014in}{1.008698in}}%
\pgfpathlineto{\pgfqpoint{3.803748in}{0.720853in}}%
\pgfpathlineto{\pgfqpoint{3.804615in}{0.771691in}}%
\pgfpathlineto{\pgfqpoint{3.806350in}{0.722056in}}%
\pgfpathlineto{\pgfqpoint{3.807217in}{0.769690in}}%
\pgfpathlineto{\pgfqpoint{3.808084in}{0.883164in}}%
\pgfpathlineto{\pgfqpoint{3.809818in}{0.735801in}}%
\pgfpathlineto{\pgfqpoint{3.810685in}{0.728570in}}%
\pgfpathlineto{\pgfqpoint{3.811552in}{0.864755in}}%
\pgfpathlineto{\pgfqpoint{3.812420in}{0.696654in}}%
\pgfpathlineto{\pgfqpoint{3.813287in}{0.946009in}}%
\pgfpathlineto{\pgfqpoint{3.814154in}{0.830624in}}%
\pgfpathlineto{\pgfqpoint{3.815021in}{0.981246in}}%
\pgfpathlineto{\pgfqpoint{3.815888in}{0.786247in}}%
\pgfpathlineto{\pgfqpoint{3.817622in}{1.297611in}}%
\pgfpathlineto{\pgfqpoint{3.818490in}{0.738723in}}%
\pgfpathlineto{\pgfqpoint{3.819357in}{0.958972in}}%
\pgfpathlineto{\pgfqpoint{3.820224in}{0.740105in}}%
\pgfpathlineto{\pgfqpoint{3.821091in}{0.822206in}}%
\pgfpathlineto{\pgfqpoint{3.821958in}{0.772501in}}%
\pgfpathlineto{\pgfqpoint{3.822825in}{0.822580in}}%
\pgfpathlineto{\pgfqpoint{3.823692in}{0.772555in}}%
\pgfpathlineto{\pgfqpoint{3.824559in}{0.791442in}}%
\pgfpathlineto{\pgfqpoint{3.825427in}{0.732440in}}%
\pgfpathlineto{\pgfqpoint{3.827161in}{0.850474in}}%
\pgfpathlineto{\pgfqpoint{3.828028in}{0.740705in}}%
\pgfpathlineto{\pgfqpoint{3.828895in}{0.768803in}}%
\pgfpathlineto{\pgfqpoint{3.829762in}{0.763442in}}%
\pgfpathlineto{\pgfqpoint{3.831497in}{0.696665in}}%
\pgfpathlineto{\pgfqpoint{3.833231in}{0.702774in}}%
\pgfpathlineto{\pgfqpoint{3.834098in}{0.715861in}}%
\pgfpathlineto{\pgfqpoint{3.834965in}{0.767085in}}%
\pgfpathlineto{\pgfqpoint{3.835832in}{0.703410in}}%
\pgfpathlineto{\pgfqpoint{3.836699in}{0.802416in}}%
\pgfpathlineto{\pgfqpoint{3.837566in}{0.779893in}}%
\pgfpathlineto{\pgfqpoint{3.838434in}{0.711937in}}%
\pgfpathlineto{\pgfqpoint{3.839301in}{0.748670in}}%
\pgfpathlineto{\pgfqpoint{3.840168in}{0.698086in}}%
\pgfpathlineto{\pgfqpoint{3.841902in}{0.724652in}}%
\pgfpathlineto{\pgfqpoint{3.842769in}{0.705151in}}%
\pgfpathlineto{\pgfqpoint{3.843636in}{0.709959in}}%
\pgfpathlineto{\pgfqpoint{3.844503in}{0.709354in}}%
\pgfpathlineto{\pgfqpoint{3.846238in}{0.771969in}}%
\pgfpathlineto{\pgfqpoint{3.847105in}{0.710532in}}%
\pgfpathlineto{\pgfqpoint{3.847972in}{0.733968in}}%
\pgfpathlineto{\pgfqpoint{3.849706in}{1.025877in}}%
\pgfpathlineto{\pgfqpoint{3.851441in}{0.743857in}}%
\pgfpathlineto{\pgfqpoint{3.852308in}{0.899176in}}%
\pgfpathlineto{\pgfqpoint{3.854042in}{0.712306in}}%
\pgfpathlineto{\pgfqpoint{3.854909in}{0.760778in}}%
\pgfpathlineto{\pgfqpoint{3.855776in}{0.989547in}}%
\pgfpathlineto{\pgfqpoint{3.856643in}{0.764341in}}%
\pgfpathlineto{\pgfqpoint{3.857510in}{0.807971in}}%
\pgfpathlineto{\pgfqpoint{3.858378in}{0.999257in}}%
\pgfpathlineto{\pgfqpoint{3.859245in}{0.749442in}}%
\pgfpathlineto{\pgfqpoint{3.860112in}{0.797850in}}%
\pgfpathlineto{\pgfqpoint{3.860979in}{0.811647in}}%
\pgfpathlineto{\pgfqpoint{3.861846in}{0.696866in}}%
\pgfpathlineto{\pgfqpoint{3.863580in}{0.784685in}}%
\pgfpathlineto{\pgfqpoint{3.865315in}{0.696297in}}%
\pgfpathlineto{\pgfqpoint{3.866182in}{0.701675in}}%
\pgfpathlineto{\pgfqpoint{3.867049in}{0.737901in}}%
\pgfpathlineto{\pgfqpoint{3.867916in}{0.710599in}}%
\pgfpathlineto{\pgfqpoint{3.869650in}{0.776066in}}%
\pgfpathlineto{\pgfqpoint{3.870517in}{0.770854in}}%
\pgfpathlineto{\pgfqpoint{3.871385in}{0.825792in}}%
\pgfpathlineto{\pgfqpoint{3.872252in}{0.719283in}}%
\pgfpathlineto{\pgfqpoint{3.873119in}{0.861183in}}%
\pgfpathlineto{\pgfqpoint{3.873986in}{0.724761in}}%
\pgfpathlineto{\pgfqpoint{3.874853in}{1.044093in}}%
\pgfpathlineto{\pgfqpoint{3.876587in}{0.732800in}}%
\pgfpathlineto{\pgfqpoint{3.877455in}{0.942292in}}%
\pgfpathlineto{\pgfqpoint{3.879189in}{0.791523in}}%
\pgfpathlineto{\pgfqpoint{3.880056in}{0.767332in}}%
\pgfpathlineto{\pgfqpoint{3.880923in}{0.705248in}}%
\pgfpathlineto{\pgfqpoint{3.882657in}{0.748910in}}%
\pgfpathlineto{\pgfqpoint{3.883524in}{0.725132in}}%
\pgfpathlineto{\pgfqpoint{3.884392in}{0.734259in}}%
\pgfpathlineto{\pgfqpoint{3.885259in}{0.718972in}}%
\pgfpathlineto{\pgfqpoint{3.886126in}{0.793387in}}%
\pgfpathlineto{\pgfqpoint{3.886993in}{0.727932in}}%
\pgfpathlineto{\pgfqpoint{3.888727in}{0.850192in}}%
\pgfpathlineto{\pgfqpoint{3.889594in}{0.716623in}}%
\pgfpathlineto{\pgfqpoint{3.890462in}{0.793442in}}%
\pgfpathlineto{\pgfqpoint{3.891329in}{0.725071in}}%
\pgfpathlineto{\pgfqpoint{3.892196in}{0.750106in}}%
\pgfpathlineto{\pgfqpoint{3.893063in}{0.918972in}}%
\pgfpathlineto{\pgfqpoint{3.894797in}{0.717130in}}%
\pgfpathlineto{\pgfqpoint{3.895664in}{0.837629in}}%
\pgfpathlineto{\pgfqpoint{3.897399in}{0.698526in}}%
\pgfpathlineto{\pgfqpoint{3.898266in}{0.708855in}}%
\pgfpathlineto{\pgfqpoint{3.899133in}{0.774034in}}%
\pgfpathlineto{\pgfqpoint{3.900000in}{0.754884in}}%
\pgfpathlineto{\pgfqpoint{3.901734in}{0.780509in}}%
\pgfpathlineto{\pgfqpoint{3.903469in}{0.698050in}}%
\pgfpathlineto{\pgfqpoint{3.904336in}{0.765595in}}%
\pgfpathlineto{\pgfqpoint{3.906070in}{0.724365in}}%
\pgfpathlineto{\pgfqpoint{3.906937in}{0.842375in}}%
\pgfpathlineto{\pgfqpoint{3.908671in}{0.697166in}}%
\pgfpathlineto{\pgfqpoint{3.910406in}{0.790904in}}%
\pgfpathlineto{\pgfqpoint{3.911273in}{0.744584in}}%
\pgfpathlineto{\pgfqpoint{3.912140in}{0.760865in}}%
\pgfpathlineto{\pgfqpoint{3.913007in}{0.722154in}}%
\pgfpathlineto{\pgfqpoint{3.913874in}{0.771239in}}%
\pgfpathlineto{\pgfqpoint{3.914741in}{0.943425in}}%
\pgfpathlineto{\pgfqpoint{3.915608in}{0.911833in}}%
\pgfpathlineto{\pgfqpoint{3.916476in}{1.044721in}}%
\pgfpathlineto{\pgfqpoint{3.917343in}{0.709040in}}%
\pgfpathlineto{\pgfqpoint{3.918210in}{1.020751in}}%
\pgfpathlineto{\pgfqpoint{3.919077in}{0.950688in}}%
\pgfpathlineto{\pgfqpoint{3.919944in}{0.872196in}}%
\pgfpathlineto{\pgfqpoint{3.920811in}{0.935674in}}%
\pgfpathlineto{\pgfqpoint{3.921678in}{0.718124in}}%
\pgfpathlineto{\pgfqpoint{3.922545in}{0.896468in}}%
\pgfpathlineto{\pgfqpoint{3.923413in}{0.696671in}}%
\pgfpathlineto{\pgfqpoint{3.924280in}{0.930016in}}%
\pgfpathlineto{\pgfqpoint{3.925147in}{0.879316in}}%
\pgfpathlineto{\pgfqpoint{3.926014in}{0.769552in}}%
\pgfpathlineto{\pgfqpoint{3.926881in}{0.781513in}}%
\pgfpathlineto{\pgfqpoint{3.928615in}{0.707502in}}%
\pgfpathlineto{\pgfqpoint{3.929483in}{0.705750in}}%
\pgfpathlineto{\pgfqpoint{3.930350in}{0.720151in}}%
\pgfpathlineto{\pgfqpoint{3.931217in}{0.696038in}}%
\pgfpathlineto{\pgfqpoint{3.932951in}{0.756855in}}%
\pgfpathlineto{\pgfqpoint{3.933818in}{0.746882in}}%
\pgfpathlineto{\pgfqpoint{3.934685in}{0.714306in}}%
\pgfpathlineto{\pgfqpoint{3.935552in}{0.724226in}}%
\pgfpathlineto{\pgfqpoint{3.937287in}{0.812634in}}%
\pgfpathlineto{\pgfqpoint{3.939021in}{0.700346in}}%
\pgfpathlineto{\pgfqpoint{3.939888in}{0.706900in}}%
\pgfpathlineto{\pgfqpoint{3.940755in}{0.746913in}}%
\pgfpathlineto{\pgfqpoint{3.941622in}{0.707978in}}%
\pgfpathlineto{\pgfqpoint{3.942490in}{0.839945in}}%
\pgfpathlineto{\pgfqpoint{3.943357in}{0.814961in}}%
\pgfpathlineto{\pgfqpoint{3.944224in}{0.707842in}}%
\pgfpathlineto{\pgfqpoint{3.945091in}{0.814089in}}%
\pgfpathlineto{\pgfqpoint{3.946825in}{0.703478in}}%
\pgfpathlineto{\pgfqpoint{3.947692in}{0.749773in}}%
\pgfpathlineto{\pgfqpoint{3.948559in}{0.703385in}}%
\pgfpathlineto{\pgfqpoint{3.949427in}{0.793894in}}%
\pgfpathlineto{\pgfqpoint{3.950294in}{0.784440in}}%
\pgfpathlineto{\pgfqpoint{3.951161in}{0.743152in}}%
\pgfpathlineto{\pgfqpoint{3.952028in}{0.882060in}}%
\pgfpathlineto{\pgfqpoint{3.952895in}{0.697951in}}%
\pgfpathlineto{\pgfqpoint{3.953762in}{0.820269in}}%
\pgfpathlineto{\pgfqpoint{3.954629in}{0.721115in}}%
\pgfpathlineto{\pgfqpoint{3.955497in}{0.737901in}}%
\pgfpathlineto{\pgfqpoint{3.956364in}{0.700913in}}%
\pgfpathlineto{\pgfqpoint{3.957231in}{0.761250in}}%
\pgfpathlineto{\pgfqpoint{3.958965in}{0.715194in}}%
\pgfpathlineto{\pgfqpoint{3.959832in}{0.748564in}}%
\pgfpathlineto{\pgfqpoint{3.960699in}{0.719063in}}%
\pgfpathlineto{\pgfqpoint{3.961566in}{0.784448in}}%
\pgfpathlineto{\pgfqpoint{3.962434in}{0.704441in}}%
\pgfpathlineto{\pgfqpoint{3.964168in}{1.016079in}}%
\pgfpathlineto{\pgfqpoint{3.965035in}{0.698393in}}%
\pgfpathlineto{\pgfqpoint{3.965902in}{1.002532in}}%
\pgfpathlineto{\pgfqpoint{3.966769in}{0.962706in}}%
\pgfpathlineto{\pgfqpoint{3.967636in}{0.732901in}}%
\pgfpathlineto{\pgfqpoint{3.968503in}{0.743555in}}%
\pgfpathlineto{\pgfqpoint{3.969371in}{0.793397in}}%
\pgfpathlineto{\pgfqpoint{3.970238in}{0.711401in}}%
\pgfpathlineto{\pgfqpoint{3.971105in}{0.726719in}}%
\pgfpathlineto{\pgfqpoint{3.971972in}{0.710810in}}%
\pgfpathlineto{\pgfqpoint{3.972839in}{0.814827in}}%
\pgfpathlineto{\pgfqpoint{3.973706in}{0.704146in}}%
\pgfpathlineto{\pgfqpoint{3.974573in}{0.925007in}}%
\pgfpathlineto{\pgfqpoint{3.975441in}{0.906520in}}%
\pgfpathlineto{\pgfqpoint{3.976308in}{0.830902in}}%
\pgfpathlineto{\pgfqpoint{3.977175in}{0.928396in}}%
\pgfpathlineto{\pgfqpoint{3.978042in}{0.735869in}}%
\pgfpathlineto{\pgfqpoint{3.978909in}{0.759486in}}%
\pgfpathlineto{\pgfqpoint{3.979776in}{0.848495in}}%
\pgfpathlineto{\pgfqpoint{3.980643in}{0.729334in}}%
\pgfpathlineto{\pgfqpoint{3.982378in}{0.944025in}}%
\pgfpathlineto{\pgfqpoint{3.984112in}{0.723496in}}%
\pgfpathlineto{\pgfqpoint{3.984979in}{0.729134in}}%
\pgfpathlineto{\pgfqpoint{3.987580in}{0.865730in}}%
\pgfpathlineto{\pgfqpoint{3.988448in}{1.009312in}}%
\pgfpathlineto{\pgfqpoint{3.989315in}{0.728413in}}%
\pgfpathlineto{\pgfqpoint{3.991049in}{1.484980in}}%
\pgfpathlineto{\pgfqpoint{3.992783in}{0.775319in}}%
\pgfpathlineto{\pgfqpoint{3.993650in}{0.972424in}}%
\pgfpathlineto{\pgfqpoint{3.994517in}{0.765344in}}%
\pgfpathlineto{\pgfqpoint{3.995385in}{0.875529in}}%
\pgfpathlineto{\pgfqpoint{3.996252in}{0.729125in}}%
\pgfpathlineto{\pgfqpoint{3.997986in}{0.909386in}}%
\pgfpathlineto{\pgfqpoint{3.998853in}{0.699001in}}%
\pgfpathlineto{\pgfqpoint{3.999720in}{0.770489in}}%
\pgfpathlineto{\pgfqpoint{4.000587in}{0.698167in}}%
\pgfpathlineto{\pgfqpoint{4.001455in}{0.743652in}}%
\pgfpathlineto{\pgfqpoint{4.002322in}{0.736980in}}%
\pgfpathlineto{\pgfqpoint{4.003189in}{0.722198in}}%
\pgfpathlineto{\pgfqpoint{4.004056in}{0.726008in}}%
\pgfpathlineto{\pgfqpoint{4.004923in}{0.759301in}}%
\pgfpathlineto{\pgfqpoint{4.005790in}{0.867340in}}%
\pgfpathlineto{\pgfqpoint{4.006657in}{0.736325in}}%
\pgfpathlineto{\pgfqpoint{4.009259in}{1.334735in}}%
\pgfpathlineto{\pgfqpoint{4.011860in}{3.849175in}}%
\pgfpathlineto{\pgfqpoint{4.013594in}{2.631114in}}%
\pgfpathlineto{\pgfqpoint{4.014462in}{2.643823in}}%
\pgfpathlineto{\pgfqpoint{4.018797in}{1.203085in}}%
\pgfpathlineto{\pgfqpoint{4.019664in}{1.564279in}}%
\pgfpathlineto{\pgfqpoint{4.021399in}{4.056000in}}%
\pgfpathlineto{\pgfqpoint{4.026601in}{0.791279in}}%
\pgfpathlineto{\pgfqpoint{4.027469in}{0.863782in}}%
\pgfpathlineto{\pgfqpoint{4.028336in}{0.828708in}}%
\pgfpathlineto{\pgfqpoint{4.029203in}{0.713923in}}%
\pgfpathlineto{\pgfqpoint{4.030070in}{0.903589in}}%
\pgfpathlineto{\pgfqpoint{4.030937in}{0.744770in}}%
\pgfpathlineto{\pgfqpoint{4.032671in}{1.293033in}}%
\pgfpathlineto{\pgfqpoint{4.033538in}{0.721686in}}%
\pgfpathlineto{\pgfqpoint{4.034406in}{1.081232in}}%
\pgfpathlineto{\pgfqpoint{4.035273in}{1.072049in}}%
\pgfpathlineto{\pgfqpoint{4.036140in}{0.708042in}}%
\pgfpathlineto{\pgfqpoint{4.037007in}{0.872403in}}%
\pgfpathlineto{\pgfqpoint{4.038741in}{0.734572in}}%
\pgfpathlineto{\pgfqpoint{4.039608in}{0.757153in}}%
\pgfpathlineto{\pgfqpoint{4.040476in}{0.733187in}}%
\pgfpathlineto{\pgfqpoint{4.042210in}{0.827634in}}%
\pgfpathlineto{\pgfqpoint{4.043077in}{0.725484in}}%
\pgfpathlineto{\pgfqpoint{4.043944in}{0.729754in}}%
\pgfpathlineto{\pgfqpoint{4.044811in}{0.815423in}}%
\pgfpathlineto{\pgfqpoint{4.046545in}{0.732396in}}%
\pgfpathlineto{\pgfqpoint{4.047413in}{0.946456in}}%
\pgfpathlineto{\pgfqpoint{4.049147in}{0.766527in}}%
\pgfpathlineto{\pgfqpoint{4.050014in}{0.925316in}}%
\pgfpathlineto{\pgfqpoint{4.053483in}{0.708882in}}%
\pgfpathlineto{\pgfqpoint{4.055217in}{0.764642in}}%
\pgfpathlineto{\pgfqpoint{4.056084in}{0.708617in}}%
\pgfpathlineto{\pgfqpoint{4.056951in}{0.822153in}}%
\pgfpathlineto{\pgfqpoint{4.057818in}{0.752183in}}%
\pgfpathlineto{\pgfqpoint{4.058685in}{0.825754in}}%
\pgfpathlineto{\pgfqpoint{4.059552in}{0.744069in}}%
\pgfpathlineto{\pgfqpoint{4.061287in}{1.259837in}}%
\pgfpathlineto{\pgfqpoint{4.062154in}{0.787844in}}%
\pgfpathlineto{\pgfqpoint{4.063021in}{1.405977in}}%
\pgfpathlineto{\pgfqpoint{4.064755in}{0.714724in}}%
\pgfpathlineto{\pgfqpoint{4.065622in}{0.960322in}}%
\pgfpathlineto{\pgfqpoint{4.067357in}{0.727770in}}%
\pgfpathlineto{\pgfqpoint{4.068224in}{0.742430in}}%
\pgfpathlineto{\pgfqpoint{4.069958in}{0.713346in}}%
\pgfpathlineto{\pgfqpoint{4.071692in}{0.743300in}}%
\pgfpathlineto{\pgfqpoint{4.072559in}{0.795559in}}%
\pgfpathlineto{\pgfqpoint{4.073427in}{1.014579in}}%
\pgfpathlineto{\pgfqpoint{4.074294in}{1.010161in}}%
\pgfpathlineto{\pgfqpoint{4.075161in}{0.762308in}}%
\pgfpathlineto{\pgfqpoint{4.076028in}{0.939614in}}%
\pgfpathlineto{\pgfqpoint{4.076895in}{0.930937in}}%
\pgfpathlineto{\pgfqpoint{4.078629in}{0.731926in}}%
\pgfpathlineto{\pgfqpoint{4.080364in}{0.701411in}}%
\pgfpathlineto{\pgfqpoint{4.081231in}{0.776123in}}%
\pgfpathlineto{\pgfqpoint{4.082965in}{0.736524in}}%
\pgfpathlineto{\pgfqpoint{4.083832in}{0.737334in}}%
\pgfpathlineto{\pgfqpoint{4.084699in}{0.706681in}}%
\pgfpathlineto{\pgfqpoint{4.085566in}{0.734156in}}%
\pgfpathlineto{\pgfqpoint{4.087301in}{0.895141in}}%
\pgfpathlineto{\pgfqpoint{4.089035in}{0.757663in}}%
\pgfpathlineto{\pgfqpoint{4.090769in}{0.898170in}}%
\pgfpathlineto{\pgfqpoint{4.092503in}{0.762074in}}%
\pgfpathlineto{\pgfqpoint{4.095105in}{0.875268in}}%
\pgfpathlineto{\pgfqpoint{4.096839in}{0.742278in}}%
\pgfpathlineto{\pgfqpoint{4.099441in}{0.827114in}}%
\pgfpathlineto{\pgfqpoint{4.100308in}{0.813854in}}%
\pgfpathlineto{\pgfqpoint{4.101175in}{0.704646in}}%
\pgfpathlineto{\pgfqpoint{4.102909in}{0.965770in}}%
\pgfpathlineto{\pgfqpoint{4.104643in}{0.705314in}}%
\pgfpathlineto{\pgfqpoint{4.105510in}{0.776750in}}%
\pgfpathlineto{\pgfqpoint{4.107245in}{0.699132in}}%
\pgfpathlineto{\pgfqpoint{4.109846in}{0.726152in}}%
\pgfpathlineto{\pgfqpoint{4.110713in}{0.961861in}}%
\pgfpathlineto{\pgfqpoint{4.112448in}{0.731686in}}%
\pgfpathlineto{\pgfqpoint{4.113315in}{0.853218in}}%
\pgfpathlineto{\pgfqpoint{4.114182in}{0.761390in}}%
\pgfpathlineto{\pgfqpoint{4.115049in}{0.779622in}}%
\pgfpathlineto{\pgfqpoint{4.116783in}{0.733229in}}%
\pgfpathlineto{\pgfqpoint{4.117650in}{0.749649in}}%
\pgfpathlineto{\pgfqpoint{4.118517in}{0.717614in}}%
\pgfpathlineto{\pgfqpoint{4.119385in}{0.760425in}}%
\pgfpathlineto{\pgfqpoint{4.120252in}{0.719946in}}%
\pgfpathlineto{\pgfqpoint{4.121119in}{0.757645in}}%
\pgfpathlineto{\pgfqpoint{4.121986in}{0.756439in}}%
\pgfpathlineto{\pgfqpoint{4.122853in}{0.697395in}}%
\pgfpathlineto{\pgfqpoint{4.124587in}{0.787350in}}%
\pgfpathlineto{\pgfqpoint{4.126322in}{0.704768in}}%
\pgfpathlineto{\pgfqpoint{4.127189in}{0.711940in}}%
\pgfpathlineto{\pgfqpoint{4.128056in}{0.738623in}}%
\pgfpathlineto{\pgfqpoint{4.128923in}{0.698855in}}%
\pgfpathlineto{\pgfqpoint{4.129790in}{0.746894in}}%
\pgfpathlineto{\pgfqpoint{4.130657in}{0.711067in}}%
\pgfpathlineto{\pgfqpoint{4.132392in}{0.854798in}}%
\pgfpathlineto{\pgfqpoint{4.133259in}{0.737917in}}%
\pgfpathlineto{\pgfqpoint{4.134126in}{0.840547in}}%
\pgfpathlineto{\pgfqpoint{4.135860in}{0.712978in}}%
\pgfpathlineto{\pgfqpoint{4.136727in}{0.832947in}}%
\pgfpathlineto{\pgfqpoint{4.137594in}{0.810503in}}%
\pgfpathlineto{\pgfqpoint{4.138462in}{0.711721in}}%
\pgfpathlineto{\pgfqpoint{4.139329in}{0.940111in}}%
\pgfpathlineto{\pgfqpoint{4.140196in}{0.769725in}}%
\pgfpathlineto{\pgfqpoint{4.141063in}{0.807410in}}%
\pgfpathlineto{\pgfqpoint{4.141930in}{1.015669in}}%
\pgfpathlineto{\pgfqpoint{4.142797in}{0.734795in}}%
\pgfpathlineto{\pgfqpoint{4.144531in}{0.999895in}}%
\pgfpathlineto{\pgfqpoint{4.146266in}{0.755557in}}%
\pgfpathlineto{\pgfqpoint{4.147133in}{0.947397in}}%
\pgfpathlineto{\pgfqpoint{4.148867in}{0.748529in}}%
\pgfpathlineto{\pgfqpoint{4.149734in}{0.955539in}}%
\pgfpathlineto{\pgfqpoint{4.150601in}{0.808027in}}%
\pgfpathlineto{\pgfqpoint{4.151469in}{0.848914in}}%
\pgfpathlineto{\pgfqpoint{4.152336in}{1.041134in}}%
\pgfpathlineto{\pgfqpoint{4.153203in}{0.723661in}}%
\pgfpathlineto{\pgfqpoint{4.154070in}{1.183566in}}%
\pgfpathlineto{\pgfqpoint{4.154937in}{1.098129in}}%
\pgfpathlineto{\pgfqpoint{4.155804in}{0.708945in}}%
\pgfpathlineto{\pgfqpoint{4.156671in}{1.089476in}}%
\pgfpathlineto{\pgfqpoint{4.158406in}{0.698877in}}%
\pgfpathlineto{\pgfqpoint{4.159273in}{0.721145in}}%
\pgfpathlineto{\pgfqpoint{4.160140in}{0.723252in}}%
\pgfpathlineto{\pgfqpoint{4.161007in}{0.835721in}}%
\pgfpathlineto{\pgfqpoint{4.161874in}{0.746923in}}%
\pgfpathlineto{\pgfqpoint{4.163608in}{1.211437in}}%
\pgfpathlineto{\pgfqpoint{4.164476in}{0.713338in}}%
\pgfpathlineto{\pgfqpoint{4.165343in}{0.983685in}}%
\pgfpathlineto{\pgfqpoint{4.166210in}{0.951405in}}%
\pgfpathlineto{\pgfqpoint{4.167077in}{1.037226in}}%
\pgfpathlineto{\pgfqpoint{4.167944in}{0.915159in}}%
\pgfpathlineto{\pgfqpoint{4.168811in}{0.937984in}}%
\pgfpathlineto{\pgfqpoint{4.169678in}{1.231202in}}%
\pgfpathlineto{\pgfqpoint{4.171413in}{0.787396in}}%
\pgfpathlineto{\pgfqpoint{4.173147in}{0.728580in}}%
\pgfpathlineto{\pgfqpoint{4.174014in}{0.749598in}}%
\pgfpathlineto{\pgfqpoint{4.174881in}{0.957525in}}%
\pgfpathlineto{\pgfqpoint{4.175748in}{0.770058in}}%
\pgfpathlineto{\pgfqpoint{4.176615in}{0.829025in}}%
\pgfpathlineto{\pgfqpoint{4.177483in}{0.994653in}}%
\pgfpathlineto{\pgfqpoint{4.179217in}{0.711156in}}%
\pgfpathlineto{\pgfqpoint{4.180084in}{0.783322in}}%
\pgfpathlineto{\pgfqpoint{4.180951in}{0.763326in}}%
\pgfpathlineto{\pgfqpoint{4.181818in}{0.883242in}}%
\pgfpathlineto{\pgfqpoint{4.183552in}{0.715189in}}%
\pgfpathlineto{\pgfqpoint{4.184420in}{0.857239in}}%
\pgfpathlineto{\pgfqpoint{4.186154in}{0.720621in}}%
\pgfpathlineto{\pgfqpoint{4.187021in}{0.720870in}}%
\pgfpathlineto{\pgfqpoint{4.188755in}{0.701238in}}%
\pgfpathlineto{\pgfqpoint{4.189622in}{0.701596in}}%
\pgfpathlineto{\pgfqpoint{4.190490in}{0.704960in}}%
\pgfpathlineto{\pgfqpoint{4.192224in}{0.755707in}}%
\pgfpathlineto{\pgfqpoint{4.193091in}{0.849882in}}%
\pgfpathlineto{\pgfqpoint{4.194825in}{0.732469in}}%
\pgfpathlineto{\pgfqpoint{4.195692in}{0.811869in}}%
\pgfpathlineto{\pgfqpoint{4.197427in}{0.738450in}}%
\pgfpathlineto{\pgfqpoint{4.198294in}{0.742723in}}%
\pgfpathlineto{\pgfqpoint{4.199161in}{0.763395in}}%
\pgfpathlineto{\pgfqpoint{4.200028in}{0.761091in}}%
\pgfpathlineto{\pgfqpoint{4.202629in}{0.697286in}}%
\pgfpathlineto{\pgfqpoint{4.205231in}{0.791262in}}%
\pgfpathlineto{\pgfqpoint{4.206098in}{0.955025in}}%
\pgfpathlineto{\pgfqpoint{4.207832in}{0.747665in}}%
\pgfpathlineto{\pgfqpoint{4.208699in}{0.834223in}}%
\pgfpathlineto{\pgfqpoint{4.209566in}{0.697476in}}%
\pgfpathlineto{\pgfqpoint{4.211301in}{0.950905in}}%
\pgfpathlineto{\pgfqpoint{4.213035in}{1.130820in}}%
\pgfpathlineto{\pgfqpoint{4.213902in}{1.043236in}}%
\pgfpathlineto{\pgfqpoint{4.214769in}{0.804553in}}%
\pgfpathlineto{\pgfqpoint{4.215636in}{0.853406in}}%
\pgfpathlineto{\pgfqpoint{4.216503in}{0.844139in}}%
\pgfpathlineto{\pgfqpoint{4.218238in}{0.698466in}}%
\pgfpathlineto{\pgfqpoint{4.219105in}{0.805376in}}%
\pgfpathlineto{\pgfqpoint{4.220839in}{0.741558in}}%
\pgfpathlineto{\pgfqpoint{4.221706in}{0.851859in}}%
\pgfpathlineto{\pgfqpoint{4.222573in}{0.712009in}}%
\pgfpathlineto{\pgfqpoint{4.224308in}{0.889978in}}%
\pgfpathlineto{\pgfqpoint{4.225175in}{0.748871in}}%
\pgfpathlineto{\pgfqpoint{4.226042in}{0.836919in}}%
\pgfpathlineto{\pgfqpoint{4.226909in}{0.772293in}}%
\pgfpathlineto{\pgfqpoint{4.227776in}{0.893824in}}%
\pgfpathlineto{\pgfqpoint{4.229510in}{0.801322in}}%
\pgfpathlineto{\pgfqpoint{4.230378in}{1.115119in}}%
\pgfpathlineto{\pgfqpoint{4.231245in}{1.040988in}}%
\pgfpathlineto{\pgfqpoint{4.232979in}{0.847987in}}%
\pgfpathlineto{\pgfqpoint{4.233846in}{0.893358in}}%
\pgfpathlineto{\pgfqpoint{4.235580in}{0.711331in}}%
\pgfpathlineto{\pgfqpoint{4.237315in}{0.731960in}}%
\pgfpathlineto{\pgfqpoint{4.238182in}{0.700841in}}%
\pgfpathlineto{\pgfqpoint{4.239049in}{0.703631in}}%
\pgfpathlineto{\pgfqpoint{4.240783in}{0.783124in}}%
\pgfpathlineto{\pgfqpoint{4.241650in}{0.779411in}}%
\pgfpathlineto{\pgfqpoint{4.245986in}{0.702576in}}%
\pgfpathlineto{\pgfqpoint{4.247720in}{0.749202in}}%
\pgfpathlineto{\pgfqpoint{4.248587in}{0.729876in}}%
\pgfpathlineto{\pgfqpoint{4.250322in}{0.881597in}}%
\pgfpathlineto{\pgfqpoint{4.251189in}{0.706837in}}%
\pgfpathlineto{\pgfqpoint{4.252923in}{0.912928in}}%
\pgfpathlineto{\pgfqpoint{4.253790in}{0.722237in}}%
\pgfpathlineto{\pgfqpoint{4.254657in}{0.855494in}}%
\pgfpathlineto{\pgfqpoint{4.256392in}{0.725568in}}%
\pgfpathlineto{\pgfqpoint{4.257259in}{0.815954in}}%
\pgfpathlineto{\pgfqpoint{4.259860in}{0.730392in}}%
\pgfpathlineto{\pgfqpoint{4.260727in}{0.736921in}}%
\pgfpathlineto{\pgfqpoint{4.261594in}{0.708720in}}%
\pgfpathlineto{\pgfqpoint{4.262462in}{0.912553in}}%
\pgfpathlineto{\pgfqpoint{4.263329in}{0.877164in}}%
\pgfpathlineto{\pgfqpoint{4.265063in}{0.756245in}}%
\pgfpathlineto{\pgfqpoint{4.265930in}{0.772662in}}%
\pgfpathlineto{\pgfqpoint{4.266797in}{0.890885in}}%
\pgfpathlineto{\pgfqpoint{4.268531in}{0.790462in}}%
\pgfpathlineto{\pgfqpoint{4.269399in}{0.858194in}}%
\pgfpathlineto{\pgfqpoint{4.270266in}{0.851669in}}%
\pgfpathlineto{\pgfqpoint{4.272000in}{0.703929in}}%
\pgfpathlineto{\pgfqpoint{4.272867in}{0.751632in}}%
\pgfpathlineto{\pgfqpoint{4.273734in}{0.744423in}}%
\pgfpathlineto{\pgfqpoint{4.274601in}{0.747650in}}%
\pgfpathlineto{\pgfqpoint{4.275469in}{0.761448in}}%
\pgfpathlineto{\pgfqpoint{4.277203in}{1.011275in}}%
\pgfpathlineto{\pgfqpoint{4.278937in}{0.810988in}}%
\pgfpathlineto{\pgfqpoint{4.279804in}{0.818888in}}%
\pgfpathlineto{\pgfqpoint{4.280671in}{0.847582in}}%
\pgfpathlineto{\pgfqpoint{4.281538in}{1.171365in}}%
\pgfpathlineto{\pgfqpoint{4.283273in}{0.842705in}}%
\pgfpathlineto{\pgfqpoint{4.284140in}{0.916998in}}%
\pgfpathlineto{\pgfqpoint{4.285007in}{0.858610in}}%
\pgfpathlineto{\pgfqpoint{4.285874in}{1.061420in}}%
\pgfpathlineto{\pgfqpoint{4.286741in}{0.722326in}}%
\pgfpathlineto{\pgfqpoint{4.288476in}{0.922353in}}%
\pgfpathlineto{\pgfqpoint{4.289343in}{0.696430in}}%
\pgfpathlineto{\pgfqpoint{4.290210in}{0.743283in}}%
\pgfpathlineto{\pgfqpoint{4.291077in}{0.780440in}}%
\pgfpathlineto{\pgfqpoint{4.291944in}{0.763852in}}%
\pgfpathlineto{\pgfqpoint{4.292811in}{0.701891in}}%
\pgfpathlineto{\pgfqpoint{4.294545in}{0.886262in}}%
\pgfpathlineto{\pgfqpoint{4.295413in}{0.718065in}}%
\pgfpathlineto{\pgfqpoint{4.297147in}{1.555239in}}%
\pgfpathlineto{\pgfqpoint{4.298014in}{0.704215in}}%
\pgfpathlineto{\pgfqpoint{4.299748in}{1.384039in}}%
\pgfpathlineto{\pgfqpoint{4.300615in}{0.708033in}}%
\pgfpathlineto{\pgfqpoint{4.301483in}{0.865833in}}%
\pgfpathlineto{\pgfqpoint{4.302350in}{0.790157in}}%
\pgfpathlineto{\pgfqpoint{4.303217in}{0.798623in}}%
\pgfpathlineto{\pgfqpoint{4.304084in}{0.751190in}}%
\pgfpathlineto{\pgfqpoint{4.305818in}{0.902818in}}%
\pgfpathlineto{\pgfqpoint{4.306685in}{0.700977in}}%
\pgfpathlineto{\pgfqpoint{4.307552in}{0.877988in}}%
\pgfpathlineto{\pgfqpoint{4.309287in}{0.700555in}}%
\pgfpathlineto{\pgfqpoint{4.310154in}{0.810885in}}%
\pgfpathlineto{\pgfqpoint{4.311021in}{0.779868in}}%
\pgfpathlineto{\pgfqpoint{4.311888in}{0.698902in}}%
\pgfpathlineto{\pgfqpoint{4.313622in}{0.767167in}}%
\pgfpathlineto{\pgfqpoint{4.315357in}{0.718687in}}%
\pgfpathlineto{\pgfqpoint{4.316224in}{0.747056in}}%
\pgfpathlineto{\pgfqpoint{4.317958in}{0.890257in}}%
\pgfpathlineto{\pgfqpoint{4.318825in}{0.762335in}}%
\pgfpathlineto{\pgfqpoint{4.320559in}{1.104243in}}%
\pgfpathlineto{\pgfqpoint{4.321427in}{0.753186in}}%
\pgfpathlineto{\pgfqpoint{4.323161in}{1.124533in}}%
\pgfpathlineto{\pgfqpoint{4.324895in}{0.725008in}}%
\pgfpathlineto{\pgfqpoint{4.325762in}{0.865947in}}%
\pgfpathlineto{\pgfqpoint{4.326629in}{1.202359in}}%
\pgfpathlineto{\pgfqpoint{4.328364in}{0.713058in}}%
\pgfpathlineto{\pgfqpoint{4.329231in}{0.846995in}}%
\pgfpathlineto{\pgfqpoint{4.330098in}{1.071503in}}%
\pgfpathlineto{\pgfqpoint{4.331832in}{0.783015in}}%
\pgfpathlineto{\pgfqpoint{4.332699in}{0.812454in}}%
\pgfpathlineto{\pgfqpoint{4.333566in}{0.936153in}}%
\pgfpathlineto{\pgfqpoint{4.334434in}{0.719006in}}%
\pgfpathlineto{\pgfqpoint{4.336168in}{1.023076in}}%
\pgfpathlineto{\pgfqpoint{4.337035in}{0.715902in}}%
\pgfpathlineto{\pgfqpoint{4.338769in}{0.965528in}}%
\pgfpathlineto{\pgfqpoint{4.341371in}{0.705074in}}%
\pgfpathlineto{\pgfqpoint{4.342238in}{0.748992in}}%
\pgfpathlineto{\pgfqpoint{4.343105in}{0.955680in}}%
\pgfpathlineto{\pgfqpoint{4.343972in}{0.921747in}}%
\pgfpathlineto{\pgfqpoint{4.344839in}{0.721224in}}%
\pgfpathlineto{\pgfqpoint{4.345706in}{0.729711in}}%
\pgfpathlineto{\pgfqpoint{4.346573in}{0.839128in}}%
\pgfpathlineto{\pgfqpoint{4.347441in}{0.800981in}}%
\pgfpathlineto{\pgfqpoint{4.348308in}{0.699546in}}%
\pgfpathlineto{\pgfqpoint{4.349175in}{0.717661in}}%
\pgfpathlineto{\pgfqpoint{4.350042in}{0.708023in}}%
\pgfpathlineto{\pgfqpoint{4.350909in}{0.729771in}}%
\pgfpathlineto{\pgfqpoint{4.351776in}{0.726273in}}%
\pgfpathlineto{\pgfqpoint{4.352643in}{0.708005in}}%
\pgfpathlineto{\pgfqpoint{4.353510in}{0.801467in}}%
\pgfpathlineto{\pgfqpoint{4.354378in}{0.773538in}}%
\pgfpathlineto{\pgfqpoint{4.356112in}{0.701038in}}%
\pgfpathlineto{\pgfqpoint{4.356979in}{0.698206in}}%
\pgfpathlineto{\pgfqpoint{4.357846in}{0.726114in}}%
\pgfpathlineto{\pgfqpoint{4.359580in}{0.820568in}}%
\pgfpathlineto{\pgfqpoint{4.360448in}{0.819326in}}%
\pgfpathlineto{\pgfqpoint{4.362182in}{0.722718in}}%
\pgfpathlineto{\pgfqpoint{4.363049in}{0.774356in}}%
\pgfpathlineto{\pgfqpoint{4.364783in}{0.711571in}}%
\pgfpathlineto{\pgfqpoint{4.365650in}{0.729768in}}%
\pgfpathlineto{\pgfqpoint{4.366517in}{0.718149in}}%
\pgfpathlineto{\pgfqpoint{4.367385in}{0.721007in}}%
\pgfpathlineto{\pgfqpoint{4.369119in}{0.847624in}}%
\pgfpathlineto{\pgfqpoint{4.369986in}{0.704159in}}%
\pgfpathlineto{\pgfqpoint{4.370853in}{0.776052in}}%
\pgfpathlineto{\pgfqpoint{4.372587in}{0.725715in}}%
\pgfpathlineto{\pgfqpoint{4.373455in}{0.803033in}}%
\pgfpathlineto{\pgfqpoint{4.375189in}{0.723485in}}%
\pgfpathlineto{\pgfqpoint{4.376056in}{0.857403in}}%
\pgfpathlineto{\pgfqpoint{4.377790in}{0.732222in}}%
\pgfpathlineto{\pgfqpoint{4.378657in}{1.012841in}}%
\pgfpathlineto{\pgfqpoint{4.380392in}{0.716158in}}%
\pgfpathlineto{\pgfqpoint{4.382126in}{0.783537in}}%
\pgfpathlineto{\pgfqpoint{4.382993in}{0.737132in}}%
\pgfpathlineto{\pgfqpoint{4.384727in}{1.030450in}}%
\pgfpathlineto{\pgfqpoint{4.386462in}{0.714611in}}%
\pgfpathlineto{\pgfqpoint{4.388196in}{0.884096in}}%
\pgfpathlineto{\pgfqpoint{4.389063in}{0.805841in}}%
\pgfpathlineto{\pgfqpoint{4.390797in}{1.036072in}}%
\pgfpathlineto{\pgfqpoint{4.391664in}{0.815459in}}%
\pgfpathlineto{\pgfqpoint{4.392531in}{0.873460in}}%
\pgfpathlineto{\pgfqpoint{4.393399in}{0.858437in}}%
\pgfpathlineto{\pgfqpoint{4.395133in}{0.712852in}}%
\pgfpathlineto{\pgfqpoint{4.396867in}{0.769522in}}%
\pgfpathlineto{\pgfqpoint{4.397734in}{0.696156in}}%
\pgfpathlineto{\pgfqpoint{4.399469in}{0.742356in}}%
\pgfpathlineto{\pgfqpoint{4.400336in}{0.708141in}}%
\pgfpathlineto{\pgfqpoint{4.401203in}{0.712986in}}%
\pgfpathlineto{\pgfqpoint{4.402070in}{0.741989in}}%
\pgfpathlineto{\pgfqpoint{4.402937in}{0.697635in}}%
\pgfpathlineto{\pgfqpoint{4.405538in}{0.780070in}}%
\pgfpathlineto{\pgfqpoint{4.406406in}{0.736070in}}%
\pgfpathlineto{\pgfqpoint{4.407273in}{0.788765in}}%
\pgfpathlineto{\pgfqpoint{4.408140in}{0.707632in}}%
\pgfpathlineto{\pgfqpoint{4.409007in}{0.735811in}}%
\pgfpathlineto{\pgfqpoint{4.409874in}{0.705670in}}%
\pgfpathlineto{\pgfqpoint{4.410741in}{0.987707in}}%
\pgfpathlineto{\pgfqpoint{4.412476in}{0.825805in}}%
\pgfpathlineto{\pgfqpoint{4.413343in}{1.029307in}}%
\pgfpathlineto{\pgfqpoint{4.415944in}{0.741917in}}%
\pgfpathlineto{\pgfqpoint{4.416811in}{0.847487in}}%
\pgfpathlineto{\pgfqpoint{4.418545in}{0.696258in}}%
\pgfpathlineto{\pgfqpoint{4.420280in}{0.779921in}}%
\pgfpathlineto{\pgfqpoint{4.422014in}{0.700843in}}%
\pgfpathlineto{\pgfqpoint{4.422881in}{0.808923in}}%
\pgfpathlineto{\pgfqpoint{4.423748in}{0.781789in}}%
\pgfpathlineto{\pgfqpoint{4.424615in}{0.701143in}}%
\pgfpathlineto{\pgfqpoint{4.425483in}{0.719118in}}%
\pgfpathlineto{\pgfqpoint{4.426350in}{0.788533in}}%
\pgfpathlineto{\pgfqpoint{4.427217in}{0.724232in}}%
\pgfpathlineto{\pgfqpoint{4.428951in}{0.938929in}}%
\pgfpathlineto{\pgfqpoint{4.429818in}{0.699869in}}%
\pgfpathlineto{\pgfqpoint{4.430685in}{0.860821in}}%
\pgfpathlineto{\pgfqpoint{4.433287in}{0.708663in}}%
\pgfpathlineto{\pgfqpoint{4.434154in}{0.721632in}}%
\pgfpathlineto{\pgfqpoint{4.436755in}{0.839976in}}%
\pgfpathlineto{\pgfqpoint{4.437622in}{0.744978in}}%
\pgfpathlineto{\pgfqpoint{4.438490in}{0.914247in}}%
\pgfpathlineto{\pgfqpoint{4.439357in}{0.910087in}}%
\pgfpathlineto{\pgfqpoint{4.440224in}{0.704713in}}%
\pgfpathlineto{\pgfqpoint{4.441091in}{1.085238in}}%
\pgfpathlineto{\pgfqpoint{4.441958in}{1.052371in}}%
\pgfpathlineto{\pgfqpoint{4.442825in}{0.702668in}}%
\pgfpathlineto{\pgfqpoint{4.444559in}{0.975198in}}%
\pgfpathlineto{\pgfqpoint{4.445427in}{0.772097in}}%
\pgfpathlineto{\pgfqpoint{4.446294in}{0.919912in}}%
\pgfpathlineto{\pgfqpoint{4.447161in}{0.714825in}}%
\pgfpathlineto{\pgfqpoint{4.448028in}{0.795332in}}%
\pgfpathlineto{\pgfqpoint{4.448895in}{0.740473in}}%
\pgfpathlineto{\pgfqpoint{4.449762in}{0.742394in}}%
\pgfpathlineto{\pgfqpoint{4.450629in}{0.769037in}}%
\pgfpathlineto{\pgfqpoint{4.453231in}{0.712808in}}%
\pgfpathlineto{\pgfqpoint{4.454098in}{0.706119in}}%
\pgfpathlineto{\pgfqpoint{4.455832in}{0.841433in}}%
\pgfpathlineto{\pgfqpoint{4.456699in}{0.732115in}}%
\pgfpathlineto{\pgfqpoint{4.457566in}{0.918778in}}%
\pgfpathlineto{\pgfqpoint{4.459301in}{0.759657in}}%
\pgfpathlineto{\pgfqpoint{4.460168in}{0.888907in}}%
\pgfpathlineto{\pgfqpoint{4.461902in}{0.710037in}}%
\pgfpathlineto{\pgfqpoint{4.462769in}{0.790538in}}%
\pgfpathlineto{\pgfqpoint{4.463636in}{0.771011in}}%
\pgfpathlineto{\pgfqpoint{4.464503in}{0.700866in}}%
\pgfpathlineto{\pgfqpoint{4.465371in}{0.791496in}}%
\pgfpathlineto{\pgfqpoint{4.466238in}{0.716983in}}%
\pgfpathlineto{\pgfqpoint{4.467105in}{0.751254in}}%
\pgfpathlineto{\pgfqpoint{4.467972in}{0.892635in}}%
\pgfpathlineto{\pgfqpoint{4.469706in}{0.705208in}}%
\pgfpathlineto{\pgfqpoint{4.470573in}{0.709882in}}%
\pgfpathlineto{\pgfqpoint{4.473175in}{0.832097in}}%
\pgfpathlineto{\pgfqpoint{4.474042in}{0.801073in}}%
\pgfpathlineto{\pgfqpoint{4.474909in}{0.696315in}}%
\pgfpathlineto{\pgfqpoint{4.475776in}{0.778891in}}%
\pgfpathlineto{\pgfqpoint{4.476643in}{0.771088in}}%
\pgfpathlineto{\pgfqpoint{4.478378in}{0.701935in}}%
\pgfpathlineto{\pgfqpoint{4.479245in}{0.773066in}}%
\pgfpathlineto{\pgfqpoint{4.480112in}{0.718106in}}%
\pgfpathlineto{\pgfqpoint{4.481846in}{1.109546in}}%
\pgfpathlineto{\pgfqpoint{4.482713in}{0.782382in}}%
\pgfpathlineto{\pgfqpoint{4.483580in}{0.899727in}}%
\pgfpathlineto{\pgfqpoint{4.484448in}{1.169365in}}%
\pgfpathlineto{\pgfqpoint{4.486182in}{0.818253in}}%
\pgfpathlineto{\pgfqpoint{4.487049in}{0.903520in}}%
\pgfpathlineto{\pgfqpoint{4.487916in}{0.780046in}}%
\pgfpathlineto{\pgfqpoint{4.488783in}{0.783333in}}%
\pgfpathlineto{\pgfqpoint{4.490517in}{0.725111in}}%
\pgfpathlineto{\pgfqpoint{4.491385in}{0.742246in}}%
\pgfpathlineto{\pgfqpoint{4.493119in}{0.696530in}}%
\pgfpathlineto{\pgfqpoint{4.494853in}{0.837225in}}%
\pgfpathlineto{\pgfqpoint{4.495720in}{0.705893in}}%
\pgfpathlineto{\pgfqpoint{4.497455in}{1.152789in}}%
\pgfpathlineto{\pgfqpoint{4.499189in}{0.762847in}}%
\pgfpathlineto{\pgfqpoint{4.500056in}{0.961953in}}%
\pgfpathlineto{\pgfqpoint{4.500923in}{0.919367in}}%
\pgfpathlineto{\pgfqpoint{4.502657in}{0.699852in}}%
\pgfpathlineto{\pgfqpoint{4.503524in}{0.698015in}}%
\pgfpathlineto{\pgfqpoint{4.505259in}{0.715424in}}%
\pgfpathlineto{\pgfqpoint{4.506993in}{0.821537in}}%
\pgfpathlineto{\pgfqpoint{4.508727in}{0.705127in}}%
\pgfpathlineto{\pgfqpoint{4.510462in}{0.834217in}}%
\pgfpathlineto{\pgfqpoint{4.511329in}{0.737639in}}%
\pgfpathlineto{\pgfqpoint{4.512196in}{1.004865in}}%
\pgfpathlineto{\pgfqpoint{4.513063in}{0.954617in}}%
\pgfpathlineto{\pgfqpoint{4.515664in}{0.733406in}}%
\pgfpathlineto{\pgfqpoint{4.516531in}{0.831749in}}%
\pgfpathlineto{\pgfqpoint{4.517399in}{1.102677in}}%
\pgfpathlineto{\pgfqpoint{4.519133in}{0.785520in}}%
\pgfpathlineto{\pgfqpoint{4.520000in}{0.950619in}}%
\pgfpathlineto{\pgfqpoint{4.521734in}{0.697488in}}%
\pgfpathlineto{\pgfqpoint{4.523469in}{0.960425in}}%
\pgfpathlineto{\pgfqpoint{4.525203in}{0.737256in}}%
\pgfpathlineto{\pgfqpoint{4.526070in}{0.732640in}}%
\pgfpathlineto{\pgfqpoint{4.526937in}{0.743908in}}%
\pgfpathlineto{\pgfqpoint{4.527804in}{0.707477in}}%
\pgfpathlineto{\pgfqpoint{4.529538in}{0.895935in}}%
\pgfpathlineto{\pgfqpoint{4.530406in}{0.704987in}}%
\pgfpathlineto{\pgfqpoint{4.532140in}{0.969662in}}%
\pgfpathlineto{\pgfqpoint{4.533007in}{0.697093in}}%
\pgfpathlineto{\pgfqpoint{4.533874in}{0.757681in}}%
\pgfpathlineto{\pgfqpoint{4.534741in}{0.804433in}}%
\pgfpathlineto{\pgfqpoint{4.535608in}{0.732369in}}%
\pgfpathlineto{\pgfqpoint{4.537343in}{0.949392in}}%
\pgfpathlineto{\pgfqpoint{4.538210in}{0.702768in}}%
\pgfpathlineto{\pgfqpoint{4.539077in}{0.986695in}}%
\pgfpathlineto{\pgfqpoint{4.539944in}{0.747378in}}%
\pgfpathlineto{\pgfqpoint{4.540811in}{0.841200in}}%
\pgfpathlineto{\pgfqpoint{4.541678in}{0.839272in}}%
\pgfpathlineto{\pgfqpoint{4.542545in}{0.696065in}}%
\pgfpathlineto{\pgfqpoint{4.543413in}{0.720061in}}%
\pgfpathlineto{\pgfqpoint{4.544280in}{0.738875in}}%
\pgfpathlineto{\pgfqpoint{4.545147in}{0.785739in}}%
\pgfpathlineto{\pgfqpoint{4.546881in}{0.956071in}}%
\pgfpathlineto{\pgfqpoint{4.547748in}{0.813672in}}%
\pgfpathlineto{\pgfqpoint{4.548615in}{0.861273in}}%
\pgfpathlineto{\pgfqpoint{4.549483in}{0.850649in}}%
\pgfpathlineto{\pgfqpoint{4.550350in}{0.757546in}}%
\pgfpathlineto{\pgfqpoint{4.551217in}{0.777955in}}%
\pgfpathlineto{\pgfqpoint{4.552951in}{0.714557in}}%
\pgfpathlineto{\pgfqpoint{4.554685in}{0.729375in}}%
\pgfpathlineto{\pgfqpoint{4.555552in}{0.719620in}}%
\pgfpathlineto{\pgfqpoint{4.556420in}{0.865069in}}%
\pgfpathlineto{\pgfqpoint{4.557287in}{0.739911in}}%
\pgfpathlineto{\pgfqpoint{4.558154in}{0.749894in}}%
\pgfpathlineto{\pgfqpoint{4.559021in}{0.803917in}}%
\pgfpathlineto{\pgfqpoint{4.560755in}{0.717658in}}%
\pgfpathlineto{\pgfqpoint{4.561622in}{0.756405in}}%
\pgfpathlineto{\pgfqpoint{4.563357in}{0.892232in}}%
\pgfpathlineto{\pgfqpoint{4.564224in}{0.868489in}}%
\pgfpathlineto{\pgfqpoint{4.565091in}{1.091457in}}%
\pgfpathlineto{\pgfqpoint{4.566825in}{0.697740in}}%
\pgfpathlineto{\pgfqpoint{4.567692in}{0.778066in}}%
\pgfpathlineto{\pgfqpoint{4.571161in}{0.696538in}}%
\pgfpathlineto{\pgfqpoint{4.572028in}{0.712664in}}%
\pgfpathlineto{\pgfqpoint{4.572895in}{0.701247in}}%
\pgfpathlineto{\pgfqpoint{4.574629in}{0.753108in}}%
\pgfpathlineto{\pgfqpoint{4.575497in}{0.701800in}}%
\pgfpathlineto{\pgfqpoint{4.576364in}{0.857454in}}%
\pgfpathlineto{\pgfqpoint{4.577231in}{0.855129in}}%
\pgfpathlineto{\pgfqpoint{4.578098in}{0.725197in}}%
\pgfpathlineto{\pgfqpoint{4.578965in}{0.872550in}}%
\pgfpathlineto{\pgfqpoint{4.579832in}{0.742095in}}%
\pgfpathlineto{\pgfqpoint{4.580699in}{0.900604in}}%
\pgfpathlineto{\pgfqpoint{4.581566in}{0.737030in}}%
\pgfpathlineto{\pgfqpoint{4.583301in}{1.085341in}}%
\pgfpathlineto{\pgfqpoint{4.584168in}{0.720726in}}%
\pgfpathlineto{\pgfqpoint{4.585035in}{0.893115in}}%
\pgfpathlineto{\pgfqpoint{4.588503in}{0.737248in}}%
\pgfpathlineto{\pgfqpoint{4.589371in}{0.743858in}}%
\pgfpathlineto{\pgfqpoint{4.590238in}{0.880253in}}%
\pgfpathlineto{\pgfqpoint{4.591105in}{0.780000in}}%
\pgfpathlineto{\pgfqpoint{4.592839in}{0.868625in}}%
\pgfpathlineto{\pgfqpoint{4.593706in}{0.703095in}}%
\pgfpathlineto{\pgfqpoint{4.594573in}{0.835844in}}%
\pgfpathlineto{\pgfqpoint{4.595441in}{0.740764in}}%
\pgfpathlineto{\pgfqpoint{4.596308in}{0.773642in}}%
\pgfpathlineto{\pgfqpoint{4.597175in}{0.888954in}}%
\pgfpathlineto{\pgfqpoint{4.598042in}{0.725025in}}%
\pgfpathlineto{\pgfqpoint{4.598909in}{0.882697in}}%
\pgfpathlineto{\pgfqpoint{4.600643in}{0.719586in}}%
\pgfpathlineto{\pgfqpoint{4.601510in}{0.756194in}}%
\pgfpathlineto{\pgfqpoint{4.602378in}{0.727090in}}%
\pgfpathlineto{\pgfqpoint{4.604112in}{0.753524in}}%
\pgfpathlineto{\pgfqpoint{4.604979in}{0.696730in}}%
\pgfpathlineto{\pgfqpoint{4.606713in}{0.998321in}}%
\pgfpathlineto{\pgfqpoint{4.608448in}{0.812170in}}%
\pgfpathlineto{\pgfqpoint{4.609315in}{0.801314in}}%
\pgfpathlineto{\pgfqpoint{4.610182in}{0.888342in}}%
\pgfpathlineto{\pgfqpoint{4.611916in}{0.723953in}}%
\pgfpathlineto{\pgfqpoint{4.612783in}{0.713072in}}%
\pgfpathlineto{\pgfqpoint{4.615385in}{0.897057in}}%
\pgfpathlineto{\pgfqpoint{4.616252in}{0.715199in}}%
\pgfpathlineto{\pgfqpoint{4.617119in}{0.758827in}}%
\pgfpathlineto{\pgfqpoint{4.617986in}{0.717830in}}%
\pgfpathlineto{\pgfqpoint{4.618853in}{0.907362in}}%
\pgfpathlineto{\pgfqpoint{4.619720in}{0.708504in}}%
\pgfpathlineto{\pgfqpoint{4.621455in}{1.229043in}}%
\pgfpathlineto{\pgfqpoint{4.622322in}{0.727086in}}%
\pgfpathlineto{\pgfqpoint{4.623189in}{0.879453in}}%
\pgfpathlineto{\pgfqpoint{4.624923in}{0.699658in}}%
\pgfpathlineto{\pgfqpoint{4.625790in}{0.759394in}}%
\pgfpathlineto{\pgfqpoint{4.626657in}{0.912710in}}%
\pgfpathlineto{\pgfqpoint{4.627524in}{0.762718in}}%
\pgfpathlineto{\pgfqpoint{4.629259in}{0.893949in}}%
\pgfpathlineto{\pgfqpoint{4.630993in}{0.708052in}}%
\pgfpathlineto{\pgfqpoint{4.631860in}{0.724392in}}%
\pgfpathlineto{\pgfqpoint{4.632727in}{0.865010in}}%
\pgfpathlineto{\pgfqpoint{4.633594in}{0.863544in}}%
\pgfpathlineto{\pgfqpoint{4.634462in}{0.749994in}}%
\pgfpathlineto{\pgfqpoint{4.635329in}{0.758543in}}%
\pgfpathlineto{\pgfqpoint{4.636196in}{0.721231in}}%
\pgfpathlineto{\pgfqpoint{4.637063in}{0.738574in}}%
\pgfpathlineto{\pgfqpoint{4.637930in}{0.704494in}}%
\pgfpathlineto{\pgfqpoint{4.638797in}{0.843368in}}%
\pgfpathlineto{\pgfqpoint{4.639664in}{0.738791in}}%
\pgfpathlineto{\pgfqpoint{4.640531in}{0.786032in}}%
\pgfpathlineto{\pgfqpoint{4.641399in}{1.042806in}}%
\pgfpathlineto{\pgfqpoint{4.643133in}{0.720125in}}%
\pgfpathlineto{\pgfqpoint{4.644000in}{0.748660in}}%
\pgfpathlineto{\pgfqpoint{4.645734in}{0.919983in}}%
\pgfpathlineto{\pgfqpoint{4.647469in}{0.842542in}}%
\pgfpathlineto{\pgfqpoint{4.648336in}{0.718689in}}%
\pgfpathlineto{\pgfqpoint{4.649203in}{0.721525in}}%
\pgfpathlineto{\pgfqpoint{4.650070in}{0.747580in}}%
\pgfpathlineto{\pgfqpoint{4.651804in}{0.830903in}}%
\pgfpathlineto{\pgfqpoint{4.652671in}{0.729870in}}%
\pgfpathlineto{\pgfqpoint{4.653538in}{0.753676in}}%
\pgfpathlineto{\pgfqpoint{4.654406in}{0.706579in}}%
\pgfpathlineto{\pgfqpoint{4.655273in}{0.771329in}}%
\pgfpathlineto{\pgfqpoint{4.657007in}{0.701589in}}%
\pgfpathlineto{\pgfqpoint{4.658741in}{0.745882in}}%
\pgfpathlineto{\pgfqpoint{4.659608in}{0.766618in}}%
\pgfpathlineto{\pgfqpoint{4.660476in}{1.267481in}}%
\pgfpathlineto{\pgfqpoint{4.661343in}{1.132901in}}%
\pgfpathlineto{\pgfqpoint{4.662210in}{0.782884in}}%
\pgfpathlineto{\pgfqpoint{4.663077in}{1.274559in}}%
\pgfpathlineto{\pgfqpoint{4.663944in}{1.115366in}}%
\pgfpathlineto{\pgfqpoint{4.664811in}{0.749881in}}%
\pgfpathlineto{\pgfqpoint{4.665678in}{0.921863in}}%
\pgfpathlineto{\pgfqpoint{4.666545in}{0.725843in}}%
\pgfpathlineto{\pgfqpoint{4.668280in}{0.984279in}}%
\pgfpathlineto{\pgfqpoint{4.669147in}{0.943595in}}%
\pgfpathlineto{\pgfqpoint{4.670014in}{1.022408in}}%
\pgfpathlineto{\pgfqpoint{4.670881in}{0.717443in}}%
\pgfpathlineto{\pgfqpoint{4.671748in}{1.099643in}}%
\pgfpathlineto{\pgfqpoint{4.673483in}{0.802674in}}%
\pgfpathlineto{\pgfqpoint{4.674350in}{0.914369in}}%
\pgfpathlineto{\pgfqpoint{4.676084in}{0.708298in}}%
\pgfpathlineto{\pgfqpoint{4.676951in}{0.782086in}}%
\pgfpathlineto{\pgfqpoint{4.678685in}{0.709947in}}%
\pgfpathlineto{\pgfqpoint{4.679552in}{0.712801in}}%
\pgfpathlineto{\pgfqpoint{4.680420in}{0.698968in}}%
\pgfpathlineto{\pgfqpoint{4.682154in}{0.807782in}}%
\pgfpathlineto{\pgfqpoint{4.683021in}{0.745168in}}%
\pgfpathlineto{\pgfqpoint{4.683888in}{0.809316in}}%
\pgfpathlineto{\pgfqpoint{4.684755in}{0.800629in}}%
\pgfpathlineto{\pgfqpoint{4.685622in}{0.699518in}}%
\pgfpathlineto{\pgfqpoint{4.688224in}{0.788426in}}%
\pgfpathlineto{\pgfqpoint{4.689091in}{0.697894in}}%
\pgfpathlineto{\pgfqpoint{4.690825in}{0.818138in}}%
\pgfpathlineto{\pgfqpoint{4.691692in}{0.710811in}}%
\pgfpathlineto{\pgfqpoint{4.692559in}{0.737578in}}%
\pgfpathlineto{\pgfqpoint{4.693427in}{0.696489in}}%
\pgfpathlineto{\pgfqpoint{4.694294in}{0.743014in}}%
\pgfpathlineto{\pgfqpoint{4.696028in}{0.701394in}}%
\pgfpathlineto{\pgfqpoint{4.696895in}{0.697631in}}%
\pgfpathlineto{\pgfqpoint{4.697762in}{0.721459in}}%
\pgfpathlineto{\pgfqpoint{4.698629in}{0.709245in}}%
\pgfpathlineto{\pgfqpoint{4.699497in}{0.715738in}}%
\pgfpathlineto{\pgfqpoint{4.701231in}{0.873684in}}%
\pgfpathlineto{\pgfqpoint{4.702965in}{0.759546in}}%
\pgfpathlineto{\pgfqpoint{4.703832in}{0.983498in}}%
\pgfpathlineto{\pgfqpoint{4.704699in}{0.929983in}}%
\pgfpathlineto{\pgfqpoint{4.705566in}{0.701369in}}%
\pgfpathlineto{\pgfqpoint{4.707301in}{0.898689in}}%
\pgfpathlineto{\pgfqpoint{4.709035in}{0.712963in}}%
\pgfpathlineto{\pgfqpoint{4.709902in}{0.867407in}}%
\pgfpathlineto{\pgfqpoint{4.711636in}{0.710123in}}%
\pgfpathlineto{\pgfqpoint{4.712503in}{0.709044in}}%
\pgfpathlineto{\pgfqpoint{4.713371in}{0.740909in}}%
\pgfpathlineto{\pgfqpoint{4.714238in}{0.842617in}}%
\pgfpathlineto{\pgfqpoint{4.715972in}{0.738533in}}%
\pgfpathlineto{\pgfqpoint{4.716839in}{0.739787in}}%
\pgfpathlineto{\pgfqpoint{4.718573in}{1.182936in}}%
\pgfpathlineto{\pgfqpoint{4.720308in}{0.709271in}}%
\pgfpathlineto{\pgfqpoint{4.721175in}{0.794611in}}%
\pgfpathlineto{\pgfqpoint{4.722909in}{0.699042in}}%
\pgfpathlineto{\pgfqpoint{4.724643in}{0.822365in}}%
\pgfpathlineto{\pgfqpoint{4.725510in}{0.729882in}}%
\pgfpathlineto{\pgfqpoint{4.726378in}{0.896199in}}%
\pgfpathlineto{\pgfqpoint{4.727245in}{0.875575in}}%
\pgfpathlineto{\pgfqpoint{4.728112in}{0.762977in}}%
\pgfpathlineto{\pgfqpoint{4.728979in}{1.018024in}}%
\pgfpathlineto{\pgfqpoint{4.730713in}{0.713837in}}%
\pgfpathlineto{\pgfqpoint{4.731580in}{0.761622in}}%
\pgfpathlineto{\pgfqpoint{4.733315in}{0.735660in}}%
\pgfpathlineto{\pgfqpoint{4.734182in}{0.786067in}}%
\pgfpathlineto{\pgfqpoint{4.735049in}{0.769840in}}%
\pgfpathlineto{\pgfqpoint{4.735916in}{0.697624in}}%
\pgfpathlineto{\pgfqpoint{4.736783in}{0.812485in}}%
\pgfpathlineto{\pgfqpoint{4.738517in}{0.732394in}}%
\pgfpathlineto{\pgfqpoint{4.739385in}{0.951625in}}%
\pgfpathlineto{\pgfqpoint{4.740252in}{0.785181in}}%
\pgfpathlineto{\pgfqpoint{4.741986in}{0.916166in}}%
\pgfpathlineto{\pgfqpoint{4.742853in}{0.731396in}}%
\pgfpathlineto{\pgfqpoint{4.743720in}{0.741044in}}%
\pgfpathlineto{\pgfqpoint{4.744587in}{0.696564in}}%
\pgfpathlineto{\pgfqpoint{4.746322in}{0.712084in}}%
\pgfpathlineto{\pgfqpoint{4.747189in}{0.779144in}}%
\pgfpathlineto{\pgfqpoint{4.748056in}{0.706894in}}%
\pgfpathlineto{\pgfqpoint{4.748923in}{0.736397in}}%
\pgfpathlineto{\pgfqpoint{4.749790in}{0.811602in}}%
\pgfpathlineto{\pgfqpoint{4.750657in}{0.788669in}}%
\pgfpathlineto{\pgfqpoint{4.751524in}{0.700040in}}%
\pgfpathlineto{\pgfqpoint{4.753259in}{0.743062in}}%
\pgfpathlineto{\pgfqpoint{4.754126in}{0.726113in}}%
\pgfpathlineto{\pgfqpoint{4.755860in}{0.830066in}}%
\pgfpathlineto{\pgfqpoint{4.756727in}{0.728170in}}%
\pgfpathlineto{\pgfqpoint{4.758462in}{0.888412in}}%
\pgfpathlineto{\pgfqpoint{4.760196in}{0.698244in}}%
\pgfpathlineto{\pgfqpoint{4.761063in}{0.707058in}}%
\pgfpathlineto{\pgfqpoint{4.761930in}{0.754808in}}%
\pgfpathlineto{\pgfqpoint{4.763664in}{0.725982in}}%
\pgfpathlineto{\pgfqpoint{4.764531in}{0.736090in}}%
\pgfpathlineto{\pgfqpoint{4.766266in}{0.707867in}}%
\pgfpathlineto{\pgfqpoint{4.768000in}{0.787688in}}%
\pgfpathlineto{\pgfqpoint{4.769734in}{0.696027in}}%
\pgfpathlineto{\pgfqpoint{4.770601in}{0.709989in}}%
\pgfpathlineto{\pgfqpoint{4.771469in}{0.704860in}}%
\pgfpathlineto{\pgfqpoint{4.772336in}{0.733838in}}%
\pgfpathlineto{\pgfqpoint{4.774937in}{0.894623in}}%
\pgfpathlineto{\pgfqpoint{4.776671in}{0.740847in}}%
\pgfpathlineto{\pgfqpoint{4.777538in}{1.085857in}}%
\pgfpathlineto{\pgfqpoint{4.778406in}{1.019225in}}%
\pgfpathlineto{\pgfqpoint{4.779273in}{0.854044in}}%
\pgfpathlineto{\pgfqpoint{4.780140in}{1.039753in}}%
\pgfpathlineto{\pgfqpoint{4.781007in}{1.030211in}}%
\pgfpathlineto{\pgfqpoint{4.781874in}{0.773700in}}%
\pgfpathlineto{\pgfqpoint{4.782741in}{0.857140in}}%
\pgfpathlineto{\pgfqpoint{4.783608in}{1.583630in}}%
\pgfpathlineto{\pgfqpoint{4.787077in}{0.722160in}}%
\pgfpathlineto{\pgfqpoint{4.787944in}{0.722084in}}%
\pgfpathlineto{\pgfqpoint{4.789678in}{1.128302in}}%
\pgfpathlineto{\pgfqpoint{4.791413in}{0.696788in}}%
\pgfpathlineto{\pgfqpoint{4.793147in}{1.080535in}}%
\pgfpathlineto{\pgfqpoint{4.794881in}{0.703303in}}%
\pgfpathlineto{\pgfqpoint{4.795748in}{0.728177in}}%
\pgfpathlineto{\pgfqpoint{4.796615in}{0.696369in}}%
\pgfpathlineto{\pgfqpoint{4.797483in}{0.725680in}}%
\pgfpathlineto{\pgfqpoint{4.798350in}{0.699114in}}%
\pgfpathlineto{\pgfqpoint{4.800084in}{0.879503in}}%
\pgfpathlineto{\pgfqpoint{4.801818in}{0.704092in}}%
\pgfpathlineto{\pgfqpoint{4.802685in}{0.802667in}}%
\pgfpathlineto{\pgfqpoint{4.804420in}{0.703126in}}%
\pgfpathlineto{\pgfqpoint{4.806154in}{0.806565in}}%
\pgfpathlineto{\pgfqpoint{4.807021in}{0.792475in}}%
\pgfpathlineto{\pgfqpoint{4.807888in}{0.723119in}}%
\pgfpathlineto{\pgfqpoint{4.808755in}{0.804186in}}%
\pgfpathlineto{\pgfqpoint{4.809622in}{0.748617in}}%
\pgfpathlineto{\pgfqpoint{4.810490in}{0.792291in}}%
\pgfpathlineto{\pgfqpoint{4.811357in}{0.939817in}}%
\pgfpathlineto{\pgfqpoint{4.812224in}{0.697273in}}%
\pgfpathlineto{\pgfqpoint{4.813091in}{0.880624in}}%
\pgfpathlineto{\pgfqpoint{4.815692in}{0.707511in}}%
\pgfpathlineto{\pgfqpoint{4.817427in}{0.801597in}}%
\pgfpathlineto{\pgfqpoint{4.818294in}{0.755768in}}%
\pgfpathlineto{\pgfqpoint{4.819161in}{0.819001in}}%
\pgfpathlineto{\pgfqpoint{4.820895in}{0.699358in}}%
\pgfpathlineto{\pgfqpoint{4.821762in}{0.706722in}}%
\pgfpathlineto{\pgfqpoint{4.823497in}{0.909095in}}%
\pgfpathlineto{\pgfqpoint{4.824364in}{0.727340in}}%
\pgfpathlineto{\pgfqpoint{4.825231in}{0.746110in}}%
\pgfpathlineto{\pgfqpoint{4.826098in}{0.710550in}}%
\pgfpathlineto{\pgfqpoint{4.826965in}{0.725843in}}%
\pgfpathlineto{\pgfqpoint{4.827832in}{0.724896in}}%
\pgfpathlineto{\pgfqpoint{4.829566in}{1.033351in}}%
\pgfpathlineto{\pgfqpoint{4.830434in}{0.737144in}}%
\pgfpathlineto{\pgfqpoint{4.831301in}{0.761211in}}%
\pgfpathlineto{\pgfqpoint{4.832168in}{0.727766in}}%
\pgfpathlineto{\pgfqpoint{4.833035in}{0.733263in}}%
\pgfpathlineto{\pgfqpoint{4.835636in}{0.911351in}}%
\pgfpathlineto{\pgfqpoint{4.836503in}{0.876376in}}%
\pgfpathlineto{\pgfqpoint{4.837371in}{0.766021in}}%
\pgfpathlineto{\pgfqpoint{4.839105in}{0.854322in}}%
\pgfpathlineto{\pgfqpoint{4.839972in}{0.724856in}}%
\pgfpathlineto{\pgfqpoint{4.840839in}{0.758516in}}%
\pgfpathlineto{\pgfqpoint{4.841706in}{0.892803in}}%
\pgfpathlineto{\pgfqpoint{4.843441in}{0.716188in}}%
\pgfpathlineto{\pgfqpoint{4.844308in}{0.804765in}}%
\pgfpathlineto{\pgfqpoint{4.845175in}{0.703747in}}%
\pgfpathlineto{\pgfqpoint{4.847776in}{0.955272in}}%
\pgfpathlineto{\pgfqpoint{4.848643in}{0.940295in}}%
\pgfpathlineto{\pgfqpoint{4.850378in}{0.808347in}}%
\pgfpathlineto{\pgfqpoint{4.851245in}{0.726066in}}%
\pgfpathlineto{\pgfqpoint{4.852112in}{0.751391in}}%
\pgfpathlineto{\pgfqpoint{4.852979in}{0.854566in}}%
\pgfpathlineto{\pgfqpoint{4.853846in}{0.756804in}}%
\pgfpathlineto{\pgfqpoint{4.855580in}{0.956353in}}%
\pgfpathlineto{\pgfqpoint{4.856448in}{0.775455in}}%
\pgfpathlineto{\pgfqpoint{4.857315in}{0.780459in}}%
\pgfpathlineto{\pgfqpoint{4.858182in}{0.717184in}}%
\pgfpathlineto{\pgfqpoint{4.859049in}{0.857840in}}%
\pgfpathlineto{\pgfqpoint{4.861650in}{0.716606in}}%
\pgfpathlineto{\pgfqpoint{4.862517in}{0.864806in}}%
\pgfpathlineto{\pgfqpoint{4.864252in}{0.786673in}}%
\pgfpathlineto{\pgfqpoint{4.865119in}{1.063094in}}%
\pgfpathlineto{\pgfqpoint{4.865986in}{0.767385in}}%
\pgfpathlineto{\pgfqpoint{4.866853in}{0.838167in}}%
\pgfpathlineto{\pgfqpoint{4.867720in}{0.995606in}}%
\pgfpathlineto{\pgfqpoint{4.868587in}{0.719723in}}%
\pgfpathlineto{\pgfqpoint{4.869455in}{0.772819in}}%
\pgfpathlineto{\pgfqpoint{4.870322in}{0.753493in}}%
\pgfpathlineto{\pgfqpoint{4.871189in}{0.701827in}}%
\pgfpathlineto{\pgfqpoint{4.872056in}{0.755099in}}%
\pgfpathlineto{\pgfqpoint{4.873790in}{0.708519in}}%
\pgfpathlineto{\pgfqpoint{4.874657in}{0.703899in}}%
\pgfpathlineto{\pgfqpoint{4.877259in}{0.743850in}}%
\pgfpathlineto{\pgfqpoint{4.878126in}{0.801648in}}%
\pgfpathlineto{\pgfqpoint{4.878993in}{0.760560in}}%
\pgfpathlineto{\pgfqpoint{4.879860in}{0.877824in}}%
\pgfpathlineto{\pgfqpoint{4.880727in}{0.705043in}}%
\pgfpathlineto{\pgfqpoint{4.881594in}{0.969584in}}%
\pgfpathlineto{\pgfqpoint{4.882462in}{0.778202in}}%
\pgfpathlineto{\pgfqpoint{4.883329in}{0.806699in}}%
\pgfpathlineto{\pgfqpoint{4.884196in}{0.878216in}}%
\pgfpathlineto{\pgfqpoint{4.885063in}{0.744916in}}%
\pgfpathlineto{\pgfqpoint{4.885930in}{0.831388in}}%
\pgfpathlineto{\pgfqpoint{4.886797in}{0.718844in}}%
\pgfpathlineto{\pgfqpoint{4.887664in}{0.835576in}}%
\pgfpathlineto{\pgfqpoint{4.888531in}{0.802753in}}%
\pgfpathlineto{\pgfqpoint{4.889399in}{0.697299in}}%
\pgfpathlineto{\pgfqpoint{4.890266in}{0.735783in}}%
\pgfpathlineto{\pgfqpoint{4.891133in}{0.712920in}}%
\pgfpathlineto{\pgfqpoint{4.892000in}{0.717062in}}%
\pgfpathlineto{\pgfqpoint{4.892867in}{0.733613in}}%
\pgfpathlineto{\pgfqpoint{4.894601in}{0.879676in}}%
\pgfpathlineto{\pgfqpoint{4.895469in}{0.746716in}}%
\pgfpathlineto{\pgfqpoint{4.897203in}{0.970305in}}%
\pgfpathlineto{\pgfqpoint{4.898937in}{0.706435in}}%
\pgfpathlineto{\pgfqpoint{4.902406in}{0.806857in}}%
\pgfpathlineto{\pgfqpoint{4.904140in}{0.721603in}}%
\pgfpathlineto{\pgfqpoint{4.905007in}{0.812152in}}%
\pgfpathlineto{\pgfqpoint{4.905874in}{0.798506in}}%
\pgfpathlineto{\pgfqpoint{4.906741in}{0.859096in}}%
\pgfpathlineto{\pgfqpoint{4.908476in}{0.724859in}}%
\pgfpathlineto{\pgfqpoint{4.909343in}{0.738193in}}%
\pgfpathlineto{\pgfqpoint{4.910210in}{0.732319in}}%
\pgfpathlineto{\pgfqpoint{4.911077in}{0.748621in}}%
\pgfpathlineto{\pgfqpoint{4.911944in}{0.746709in}}%
\pgfpathlineto{\pgfqpoint{4.912811in}{0.750188in}}%
\pgfpathlineto{\pgfqpoint{4.914545in}{0.768525in}}%
\pgfpathlineto{\pgfqpoint{4.915413in}{0.701405in}}%
\pgfpathlineto{\pgfqpoint{4.916280in}{0.714868in}}%
\pgfpathlineto{\pgfqpoint{4.917147in}{0.698182in}}%
\pgfpathlineto{\pgfqpoint{4.918014in}{0.730456in}}%
\pgfpathlineto{\pgfqpoint{4.918881in}{0.701209in}}%
\pgfpathlineto{\pgfqpoint{4.919748in}{0.726253in}}%
\pgfpathlineto{\pgfqpoint{4.920615in}{0.720074in}}%
\pgfpathlineto{\pgfqpoint{4.922350in}{0.986687in}}%
\pgfpathlineto{\pgfqpoint{4.923217in}{0.788748in}}%
\pgfpathlineto{\pgfqpoint{4.926685in}{1.019168in}}%
\pgfpathlineto{\pgfqpoint{4.928420in}{0.732172in}}%
\pgfpathlineto{\pgfqpoint{4.929287in}{0.730242in}}%
\pgfpathlineto{\pgfqpoint{4.931021in}{0.711918in}}%
\pgfpathlineto{\pgfqpoint{4.932755in}{1.018713in}}%
\pgfpathlineto{\pgfqpoint{4.933622in}{0.755938in}}%
\pgfpathlineto{\pgfqpoint{4.935357in}{0.872718in}}%
\pgfpathlineto{\pgfqpoint{4.937091in}{0.915156in}}%
\pgfpathlineto{\pgfqpoint{4.937958in}{0.783797in}}%
\pgfpathlineto{\pgfqpoint{4.938825in}{0.825705in}}%
\pgfpathlineto{\pgfqpoint{4.940559in}{1.032678in}}%
\pgfpathlineto{\pgfqpoint{4.942294in}{0.722178in}}%
\pgfpathlineto{\pgfqpoint{4.943161in}{0.789647in}}%
\pgfpathlineto{\pgfqpoint{4.944028in}{0.722277in}}%
\pgfpathlineto{\pgfqpoint{4.944895in}{0.754683in}}%
\pgfpathlineto{\pgfqpoint{4.945762in}{0.717723in}}%
\pgfpathlineto{\pgfqpoint{4.947497in}{1.043194in}}%
\pgfpathlineto{\pgfqpoint{4.949231in}{0.836734in}}%
\pgfpathlineto{\pgfqpoint{4.950098in}{0.891332in}}%
\pgfpathlineto{\pgfqpoint{4.950965in}{0.872535in}}%
\pgfpathlineto{\pgfqpoint{4.951832in}{0.699369in}}%
\pgfpathlineto{\pgfqpoint{4.952699in}{0.720776in}}%
\pgfpathlineto{\pgfqpoint{4.953566in}{0.702853in}}%
\pgfpathlineto{\pgfqpoint{4.956168in}{0.858694in}}%
\pgfpathlineto{\pgfqpoint{4.957902in}{0.755356in}}%
\pgfpathlineto{\pgfqpoint{4.958769in}{0.964316in}}%
\pgfpathlineto{\pgfqpoint{4.959636in}{0.773947in}}%
\pgfpathlineto{\pgfqpoint{4.960503in}{0.817138in}}%
\pgfpathlineto{\pgfqpoint{4.961371in}{0.881420in}}%
\pgfpathlineto{\pgfqpoint{4.962238in}{0.711371in}}%
\pgfpathlineto{\pgfqpoint{4.963105in}{0.790883in}}%
\pgfpathlineto{\pgfqpoint{4.965706in}{0.704708in}}%
\pgfpathlineto{\pgfqpoint{4.967441in}{0.933347in}}%
\pgfpathlineto{\pgfqpoint{4.968308in}{0.705507in}}%
\pgfpathlineto{\pgfqpoint{4.970042in}{1.008689in}}%
\pgfpathlineto{\pgfqpoint{4.971776in}{0.704663in}}%
\pgfpathlineto{\pgfqpoint{4.972643in}{0.741778in}}%
\pgfpathlineto{\pgfqpoint{4.974378in}{0.705454in}}%
\pgfpathlineto{\pgfqpoint{4.975245in}{0.737392in}}%
\pgfpathlineto{\pgfqpoint{4.976979in}{0.708416in}}%
\pgfpathlineto{\pgfqpoint{4.977846in}{0.710053in}}%
\pgfpathlineto{\pgfqpoint{4.979580in}{0.706464in}}%
\pgfpathlineto{\pgfqpoint{4.980448in}{0.715313in}}%
\pgfpathlineto{\pgfqpoint{4.982182in}{0.696991in}}%
\pgfpathlineto{\pgfqpoint{4.983049in}{0.698180in}}%
\pgfpathlineto{\pgfqpoint{4.983916in}{0.714455in}}%
\pgfpathlineto{\pgfqpoint{4.984783in}{0.707215in}}%
\pgfpathlineto{\pgfqpoint{4.985650in}{0.709960in}}%
\pgfpathlineto{\pgfqpoint{4.986517in}{0.699251in}}%
\pgfpathlineto{\pgfqpoint{4.989986in}{0.748654in}}%
\pgfpathlineto{\pgfqpoint{4.990853in}{0.735239in}}%
\pgfpathlineto{\pgfqpoint{4.991720in}{0.826412in}}%
\pgfpathlineto{\pgfqpoint{4.992587in}{0.819647in}}%
\pgfpathlineto{\pgfqpoint{4.993455in}{0.714518in}}%
\pgfpathlineto{\pgfqpoint{4.994322in}{0.750935in}}%
\pgfpathlineto{\pgfqpoint{4.995189in}{0.743246in}}%
\pgfpathlineto{\pgfqpoint{4.996056in}{0.697244in}}%
\pgfpathlineto{\pgfqpoint{4.998657in}{0.821922in}}%
\pgfpathlineto{\pgfqpoint{4.999524in}{0.814747in}}%
\pgfpathlineto{\pgfqpoint{5.000392in}{0.742197in}}%
\pgfpathlineto{\pgfqpoint{5.001259in}{0.831955in}}%
\pgfpathlineto{\pgfqpoint{5.002993in}{0.704187in}}%
\pgfpathlineto{\pgfqpoint{5.003860in}{0.731518in}}%
\pgfpathlineto{\pgfqpoint{5.005594in}{0.854068in}}%
\pgfpathlineto{\pgfqpoint{5.006462in}{0.724550in}}%
\pgfpathlineto{\pgfqpoint{5.008196in}{0.993628in}}%
\pgfpathlineto{\pgfqpoint{5.009930in}{0.796011in}}%
\pgfpathlineto{\pgfqpoint{5.010797in}{0.794290in}}%
\pgfpathlineto{\pgfqpoint{5.011664in}{0.964593in}}%
\pgfpathlineto{\pgfqpoint{5.013399in}{0.714057in}}%
\pgfpathlineto{\pgfqpoint{5.014266in}{0.730569in}}%
\pgfpathlineto{\pgfqpoint{5.015133in}{0.712314in}}%
\pgfpathlineto{\pgfqpoint{5.016000in}{0.753091in}}%
\pgfpathlineto{\pgfqpoint{5.016867in}{0.728679in}}%
\pgfpathlineto{\pgfqpoint{5.017734in}{0.824043in}}%
\pgfpathlineto{\pgfqpoint{5.021203in}{0.713278in}}%
\pgfpathlineto{\pgfqpoint{5.022070in}{0.728506in}}%
\pgfpathlineto{\pgfqpoint{5.022937in}{0.725891in}}%
\pgfpathlineto{\pgfqpoint{5.023804in}{0.704175in}}%
\pgfpathlineto{\pgfqpoint{5.025538in}{0.875251in}}%
\pgfpathlineto{\pgfqpoint{5.026406in}{0.711264in}}%
\pgfpathlineto{\pgfqpoint{5.028140in}{0.849893in}}%
\pgfpathlineto{\pgfqpoint{5.029874in}{0.736820in}}%
\pgfpathlineto{\pgfqpoint{5.030741in}{0.736292in}}%
\pgfpathlineto{\pgfqpoint{5.032476in}{0.702555in}}%
\pgfpathlineto{\pgfqpoint{5.035077in}{0.909525in}}%
\pgfpathlineto{\pgfqpoint{5.036811in}{0.732229in}}%
\pgfpathlineto{\pgfqpoint{5.037678in}{0.928164in}}%
\pgfpathlineto{\pgfqpoint{5.039413in}{0.706167in}}%
\pgfpathlineto{\pgfqpoint{5.040280in}{0.754602in}}%
\pgfpathlineto{\pgfqpoint{5.042014in}{0.703562in}}%
\pgfpathlineto{\pgfqpoint{5.042881in}{0.755305in}}%
\pgfpathlineto{\pgfqpoint{5.043748in}{0.696837in}}%
\pgfpathlineto{\pgfqpoint{5.044615in}{0.799579in}}%
\pgfpathlineto{\pgfqpoint{5.045483in}{0.799155in}}%
\pgfpathlineto{\pgfqpoint{5.046350in}{0.698011in}}%
\pgfpathlineto{\pgfqpoint{5.047217in}{0.767746in}}%
\pgfpathlineto{\pgfqpoint{5.048084in}{0.746033in}}%
\pgfpathlineto{\pgfqpoint{5.048951in}{0.753330in}}%
\pgfpathlineto{\pgfqpoint{5.049818in}{0.713890in}}%
\pgfpathlineto{\pgfqpoint{5.050685in}{0.717257in}}%
\pgfpathlineto{\pgfqpoint{5.051552in}{0.758661in}}%
\pgfpathlineto{\pgfqpoint{5.052420in}{0.708101in}}%
\pgfpathlineto{\pgfqpoint{5.053287in}{0.723310in}}%
\pgfpathlineto{\pgfqpoint{5.054154in}{0.717980in}}%
\pgfpathlineto{\pgfqpoint{5.055021in}{0.721133in}}%
\pgfpathlineto{\pgfqpoint{5.055888in}{0.718341in}}%
\pgfpathlineto{\pgfqpoint{5.058490in}{0.842282in}}%
\pgfpathlineto{\pgfqpoint{5.059357in}{0.716291in}}%
\pgfpathlineto{\pgfqpoint{5.061091in}{0.902635in}}%
\pgfpathlineto{\pgfqpoint{5.061958in}{0.697516in}}%
\pgfpathlineto{\pgfqpoint{5.062825in}{0.711878in}}%
\pgfpathlineto{\pgfqpoint{5.063692in}{0.713663in}}%
\pgfpathlineto{\pgfqpoint{5.065427in}{0.751611in}}%
\pgfpathlineto{\pgfqpoint{5.066294in}{0.885238in}}%
\pgfpathlineto{\pgfqpoint{5.068028in}{0.737646in}}%
\pgfpathlineto{\pgfqpoint{5.069762in}{0.719166in}}%
\pgfpathlineto{\pgfqpoint{5.070629in}{0.712692in}}%
\pgfpathlineto{\pgfqpoint{5.071497in}{0.797358in}}%
\pgfpathlineto{\pgfqpoint{5.072364in}{0.699820in}}%
\pgfpathlineto{\pgfqpoint{5.073231in}{0.833169in}}%
\pgfpathlineto{\pgfqpoint{5.074098in}{0.708992in}}%
\pgfpathlineto{\pgfqpoint{5.075832in}{0.976607in}}%
\pgfpathlineto{\pgfqpoint{5.076699in}{0.792830in}}%
\pgfpathlineto{\pgfqpoint{5.078434in}{1.008086in}}%
\pgfpathlineto{\pgfqpoint{5.079301in}{0.719150in}}%
\pgfpathlineto{\pgfqpoint{5.080168in}{0.770153in}}%
\pgfpathlineto{\pgfqpoint{5.081035in}{0.773802in}}%
\pgfpathlineto{\pgfqpoint{5.081902in}{0.825975in}}%
\pgfpathlineto{\pgfqpoint{5.083636in}{0.752209in}}%
\pgfpathlineto{\pgfqpoint{5.084503in}{0.759653in}}%
\pgfpathlineto{\pgfqpoint{5.085371in}{0.711739in}}%
\pgfpathlineto{\pgfqpoint{5.087105in}{0.946305in}}%
\pgfpathlineto{\pgfqpoint{5.087972in}{0.777455in}}%
\pgfpathlineto{\pgfqpoint{5.088839in}{0.804431in}}%
\pgfpathlineto{\pgfqpoint{5.089706in}{0.826818in}}%
\pgfpathlineto{\pgfqpoint{5.090573in}{0.707389in}}%
\pgfpathlineto{\pgfqpoint{5.093175in}{0.826907in}}%
\pgfpathlineto{\pgfqpoint{5.094042in}{0.752381in}}%
\pgfpathlineto{\pgfqpoint{5.094909in}{0.798976in}}%
\pgfpathlineto{\pgfqpoint{5.095776in}{0.700857in}}%
\pgfpathlineto{\pgfqpoint{5.098378in}{0.802668in}}%
\pgfpathlineto{\pgfqpoint{5.099245in}{0.711538in}}%
\pgfpathlineto{\pgfqpoint{5.100112in}{0.782864in}}%
\pgfpathlineto{\pgfqpoint{5.100979in}{0.737747in}}%
\pgfpathlineto{\pgfqpoint{5.101846in}{0.740515in}}%
\pgfpathlineto{\pgfqpoint{5.102713in}{0.721285in}}%
\pgfpathlineto{\pgfqpoint{5.103580in}{0.787729in}}%
\pgfpathlineto{\pgfqpoint{5.104448in}{0.759223in}}%
\pgfpathlineto{\pgfqpoint{5.105315in}{0.870757in}}%
\pgfpathlineto{\pgfqpoint{5.106182in}{0.781663in}}%
\pgfpathlineto{\pgfqpoint{5.107049in}{0.784979in}}%
\pgfpathlineto{\pgfqpoint{5.107916in}{0.728465in}}%
\pgfpathlineto{\pgfqpoint{5.108783in}{0.772601in}}%
\pgfpathlineto{\pgfqpoint{5.109650in}{0.934034in}}%
\pgfpathlineto{\pgfqpoint{5.111385in}{0.764603in}}%
\pgfpathlineto{\pgfqpoint{5.112252in}{0.891227in}}%
\pgfpathlineto{\pgfqpoint{5.114853in}{0.703388in}}%
\pgfpathlineto{\pgfqpoint{5.116587in}{0.741944in}}%
\pgfpathlineto{\pgfqpoint{5.117455in}{0.740092in}}%
\pgfpathlineto{\pgfqpoint{5.118322in}{0.845574in}}%
\pgfpathlineto{\pgfqpoint{5.119189in}{0.814599in}}%
\pgfpathlineto{\pgfqpoint{5.120056in}{0.726230in}}%
\pgfpathlineto{\pgfqpoint{5.120923in}{0.742660in}}%
\pgfpathlineto{\pgfqpoint{5.122657in}{0.697496in}}%
\pgfpathlineto{\pgfqpoint{5.123524in}{0.716983in}}%
\pgfpathlineto{\pgfqpoint{5.124392in}{0.800284in}}%
\pgfpathlineto{\pgfqpoint{5.125259in}{0.726126in}}%
\pgfpathlineto{\pgfqpoint{5.126993in}{0.862354in}}%
\pgfpathlineto{\pgfqpoint{5.127860in}{0.702087in}}%
\pgfpathlineto{\pgfqpoint{5.128727in}{0.854252in}}%
\pgfpathlineto{\pgfqpoint{5.129594in}{0.708877in}}%
\pgfpathlineto{\pgfqpoint{5.130462in}{0.871235in}}%
\pgfpathlineto{\pgfqpoint{5.131329in}{0.856822in}}%
\pgfpathlineto{\pgfqpoint{5.132196in}{0.716490in}}%
\pgfpathlineto{\pgfqpoint{5.133063in}{0.834488in}}%
\pgfpathlineto{\pgfqpoint{5.133930in}{0.700206in}}%
\pgfpathlineto{\pgfqpoint{5.134797in}{0.768064in}}%
\pgfpathlineto{\pgfqpoint{5.135664in}{0.701826in}}%
\pgfpathlineto{\pgfqpoint{5.136531in}{0.832092in}}%
\pgfpathlineto{\pgfqpoint{5.138266in}{0.751505in}}%
\pgfpathlineto{\pgfqpoint{5.139133in}{0.776584in}}%
\pgfpathlineto{\pgfqpoint{5.140000in}{0.705511in}}%
\pgfpathlineto{\pgfqpoint{5.140867in}{0.738610in}}%
\pgfpathlineto{\pgfqpoint{5.141734in}{0.825048in}}%
\pgfpathlineto{\pgfqpoint{5.142601in}{0.733371in}}%
\pgfpathlineto{\pgfqpoint{5.144336in}{0.847320in}}%
\pgfpathlineto{\pgfqpoint{5.145203in}{0.706954in}}%
\pgfpathlineto{\pgfqpoint{5.146070in}{0.728401in}}%
\pgfpathlineto{\pgfqpoint{5.146937in}{0.710172in}}%
\pgfpathlineto{\pgfqpoint{5.147804in}{0.812595in}}%
\pgfpathlineto{\pgfqpoint{5.148671in}{0.753429in}}%
\pgfpathlineto{\pgfqpoint{5.149538in}{1.088108in}}%
\pgfpathlineto{\pgfqpoint{5.151273in}{0.769172in}}%
\pgfpathlineto{\pgfqpoint{5.152140in}{0.952757in}}%
\pgfpathlineto{\pgfqpoint{5.154741in}{0.698732in}}%
\pgfpathlineto{\pgfqpoint{5.155608in}{0.797880in}}%
\pgfpathlineto{\pgfqpoint{5.156476in}{0.720229in}}%
\pgfpathlineto{\pgfqpoint{5.157343in}{0.755457in}}%
\pgfpathlineto{\pgfqpoint{5.158210in}{0.839098in}}%
\pgfpathlineto{\pgfqpoint{5.159944in}{0.696720in}}%
\pgfpathlineto{\pgfqpoint{5.163413in}{0.726613in}}%
\pgfpathlineto{\pgfqpoint{5.164280in}{0.707916in}}%
\pgfpathlineto{\pgfqpoint{5.165147in}{0.719675in}}%
\pgfpathlineto{\pgfqpoint{5.166881in}{0.797313in}}%
\pgfpathlineto{\pgfqpoint{5.167748in}{0.710520in}}%
\pgfpathlineto{\pgfqpoint{5.168615in}{0.818297in}}%
\pgfpathlineto{\pgfqpoint{5.169483in}{0.816345in}}%
\pgfpathlineto{\pgfqpoint{5.170350in}{0.708133in}}%
\pgfpathlineto{\pgfqpoint{5.171217in}{0.723057in}}%
\pgfpathlineto{\pgfqpoint{5.172951in}{0.700103in}}%
\pgfpathlineto{\pgfqpoint{5.174685in}{0.720706in}}%
\pgfpathlineto{\pgfqpoint{5.175552in}{0.741004in}}%
\pgfpathlineto{\pgfqpoint{5.176420in}{0.738563in}}%
\pgfpathlineto{\pgfqpoint{5.177287in}{0.702743in}}%
\pgfpathlineto{\pgfqpoint{5.179021in}{0.760423in}}%
\pgfpathlineto{\pgfqpoint{5.179888in}{0.704537in}}%
\pgfpathlineto{\pgfqpoint{5.180755in}{0.902001in}}%
\pgfpathlineto{\pgfqpoint{5.181622in}{0.861816in}}%
\pgfpathlineto{\pgfqpoint{5.182490in}{0.705128in}}%
\pgfpathlineto{\pgfqpoint{5.183357in}{0.851179in}}%
\pgfpathlineto{\pgfqpoint{5.185091in}{0.697978in}}%
\pgfpathlineto{\pgfqpoint{5.186825in}{0.707402in}}%
\pgfpathlineto{\pgfqpoint{5.187692in}{0.729941in}}%
\pgfpathlineto{\pgfqpoint{5.188559in}{0.696665in}}%
\pgfpathlineto{\pgfqpoint{5.190294in}{0.898335in}}%
\pgfpathlineto{\pgfqpoint{5.192028in}{0.722975in}}%
\pgfpathlineto{\pgfqpoint{5.192895in}{0.777528in}}%
\pgfpathlineto{\pgfqpoint{5.193762in}{0.728618in}}%
\pgfpathlineto{\pgfqpoint{5.194629in}{0.734038in}}%
\pgfpathlineto{\pgfqpoint{5.195497in}{0.724592in}}%
\pgfpathlineto{\pgfqpoint{5.196364in}{0.866700in}}%
\pgfpathlineto{\pgfqpoint{5.197231in}{0.856911in}}%
\pgfpathlineto{\pgfqpoint{5.199832in}{0.751575in}}%
\pgfpathlineto{\pgfqpoint{5.202434in}{0.824209in}}%
\pgfpathlineto{\pgfqpoint{5.203301in}{0.709020in}}%
\pgfpathlineto{\pgfqpoint{5.204168in}{0.753731in}}%
\pgfpathlineto{\pgfqpoint{5.205035in}{0.701161in}}%
\pgfpathlineto{\pgfqpoint{5.206769in}{0.901630in}}%
\pgfpathlineto{\pgfqpoint{5.208503in}{0.967542in}}%
\pgfpathlineto{\pgfqpoint{5.209371in}{0.771213in}}%
\pgfpathlineto{\pgfqpoint{5.210238in}{0.874846in}}%
\pgfpathlineto{\pgfqpoint{5.211105in}{1.117270in}}%
\pgfpathlineto{\pgfqpoint{5.213706in}{0.730304in}}%
\pgfpathlineto{\pgfqpoint{5.214573in}{0.702255in}}%
\pgfpathlineto{\pgfqpoint{5.215441in}{0.717919in}}%
\pgfpathlineto{\pgfqpoint{5.216308in}{0.772571in}}%
\pgfpathlineto{\pgfqpoint{5.217175in}{0.731941in}}%
\pgfpathlineto{\pgfqpoint{5.218909in}{0.899464in}}%
\pgfpathlineto{\pgfqpoint{5.219776in}{0.725656in}}%
\pgfpathlineto{\pgfqpoint{5.220643in}{1.015622in}}%
\pgfpathlineto{\pgfqpoint{5.221510in}{0.944502in}}%
\pgfpathlineto{\pgfqpoint{5.222378in}{0.721973in}}%
\pgfpathlineto{\pgfqpoint{5.223245in}{0.759719in}}%
\pgfpathlineto{\pgfqpoint{5.224112in}{0.750811in}}%
\pgfpathlineto{\pgfqpoint{5.224979in}{0.725106in}}%
\pgfpathlineto{\pgfqpoint{5.226713in}{1.077338in}}%
\pgfpathlineto{\pgfqpoint{5.227580in}{1.011628in}}%
\pgfpathlineto{\pgfqpoint{5.229315in}{0.753650in}}%
\pgfpathlineto{\pgfqpoint{5.230182in}{0.900075in}}%
\pgfpathlineto{\pgfqpoint{5.231916in}{0.727105in}}%
\pgfpathlineto{\pgfqpoint{5.233650in}{0.915002in}}%
\pgfpathlineto{\pgfqpoint{5.235385in}{0.729907in}}%
\pgfpathlineto{\pgfqpoint{5.237119in}{0.778033in}}%
\pgfpathlineto{\pgfqpoint{5.239720in}{1.748617in}}%
\pgfpathlineto{\pgfqpoint{5.240587in}{2.395578in}}%
\pgfpathlineto{\pgfqpoint{5.242322in}{1.636047in}}%
\pgfpathlineto{\pgfqpoint{5.243189in}{2.071361in}}%
\pgfpathlineto{\pgfqpoint{5.244056in}{2.070884in}}%
\pgfpathlineto{\pgfqpoint{5.247524in}{0.739303in}}%
\pgfpathlineto{\pgfqpoint{5.248392in}{0.703685in}}%
\pgfpathlineto{\pgfqpoint{5.250993in}{0.834305in}}%
\pgfpathlineto{\pgfqpoint{5.251860in}{1.108230in}}%
\pgfpathlineto{\pgfqpoint{5.253594in}{0.702377in}}%
\pgfpathlineto{\pgfqpoint{5.254462in}{0.794865in}}%
\pgfpathlineto{\pgfqpoint{5.255329in}{0.803325in}}%
\pgfpathlineto{\pgfqpoint{5.256196in}{0.748345in}}%
\pgfpathlineto{\pgfqpoint{5.257063in}{0.763479in}}%
\pgfpathlineto{\pgfqpoint{5.257930in}{0.803433in}}%
\pgfpathlineto{\pgfqpoint{5.259664in}{0.728577in}}%
\pgfpathlineto{\pgfqpoint{5.260531in}{0.855723in}}%
\pgfpathlineto{\pgfqpoint{5.261399in}{0.817138in}}%
\pgfpathlineto{\pgfqpoint{5.262266in}{0.919063in}}%
\pgfpathlineto{\pgfqpoint{5.263133in}{0.888177in}}%
\pgfpathlineto{\pgfqpoint{5.264000in}{0.706956in}}%
\pgfpathlineto{\pgfqpoint{5.264867in}{0.872370in}}%
\pgfpathlineto{\pgfqpoint{5.265734in}{0.720294in}}%
\pgfpathlineto{\pgfqpoint{5.266601in}{0.762053in}}%
\pgfpathlineto{\pgfqpoint{5.267469in}{0.741879in}}%
\pgfpathlineto{\pgfqpoint{5.269203in}{0.895619in}}%
\pgfpathlineto{\pgfqpoint{5.270070in}{0.747295in}}%
\pgfpathlineto{\pgfqpoint{5.270937in}{0.921221in}}%
\pgfpathlineto{\pgfqpoint{5.272671in}{0.730064in}}%
\pgfpathlineto{\pgfqpoint{5.273538in}{0.742879in}}%
\pgfpathlineto{\pgfqpoint{5.274406in}{0.700749in}}%
\pgfpathlineto{\pgfqpoint{5.276140in}{0.740231in}}%
\pgfpathlineto{\pgfqpoint{5.277007in}{0.726960in}}%
\pgfpathlineto{\pgfqpoint{5.277874in}{0.776792in}}%
\pgfpathlineto{\pgfqpoint{5.279608in}{0.706491in}}%
\pgfpathlineto{\pgfqpoint{5.281343in}{1.099588in}}%
\pgfpathlineto{\pgfqpoint{5.283077in}{0.714548in}}%
\pgfpathlineto{\pgfqpoint{5.283944in}{0.963191in}}%
\pgfpathlineto{\pgfqpoint{5.287413in}{0.733704in}}%
\pgfpathlineto{\pgfqpoint{5.288280in}{0.794902in}}%
\pgfpathlineto{\pgfqpoint{5.289147in}{0.785174in}}%
\pgfpathlineto{\pgfqpoint{5.290014in}{0.724794in}}%
\pgfpathlineto{\pgfqpoint{5.290881in}{0.803726in}}%
\pgfpathlineto{\pgfqpoint{5.291748in}{0.750052in}}%
\pgfpathlineto{\pgfqpoint{5.292615in}{0.764997in}}%
\pgfpathlineto{\pgfqpoint{5.293483in}{0.866167in}}%
\pgfpathlineto{\pgfqpoint{5.294350in}{0.714576in}}%
\pgfpathlineto{\pgfqpoint{5.295217in}{0.718829in}}%
\pgfpathlineto{\pgfqpoint{5.296951in}{0.701372in}}%
\pgfpathlineto{\pgfqpoint{5.297818in}{0.709540in}}%
\pgfpathlineto{\pgfqpoint{5.298685in}{0.759387in}}%
\pgfpathlineto{\pgfqpoint{5.299552in}{0.701443in}}%
\pgfpathlineto{\pgfqpoint{5.301287in}{0.872107in}}%
\pgfpathlineto{\pgfqpoint{5.302154in}{0.704256in}}%
\pgfpathlineto{\pgfqpoint{5.303888in}{0.863098in}}%
\pgfpathlineto{\pgfqpoint{5.305622in}{0.721208in}}%
\pgfpathlineto{\pgfqpoint{5.306490in}{0.799103in}}%
\pgfpathlineto{\pgfqpoint{5.307357in}{0.791516in}}%
\pgfpathlineto{\pgfqpoint{5.309091in}{0.731887in}}%
\pgfpathlineto{\pgfqpoint{5.309958in}{0.981468in}}%
\pgfpathlineto{\pgfqpoint{5.310825in}{0.851779in}}%
\pgfpathlineto{\pgfqpoint{5.311692in}{0.891444in}}%
\pgfpathlineto{\pgfqpoint{5.312559in}{1.091125in}}%
\pgfpathlineto{\pgfqpoint{5.313427in}{0.731054in}}%
\pgfpathlineto{\pgfqpoint{5.314294in}{0.750307in}}%
\pgfpathlineto{\pgfqpoint{5.315161in}{0.810609in}}%
\pgfpathlineto{\pgfqpoint{5.316028in}{0.781266in}}%
\pgfpathlineto{\pgfqpoint{5.316895in}{0.710819in}}%
\pgfpathlineto{\pgfqpoint{5.317762in}{0.794555in}}%
\pgfpathlineto{\pgfqpoint{5.318629in}{0.718611in}}%
\pgfpathlineto{\pgfqpoint{5.319497in}{0.746895in}}%
\pgfpathlineto{\pgfqpoint{5.320364in}{0.696837in}}%
\pgfpathlineto{\pgfqpoint{5.321231in}{0.828996in}}%
\pgfpathlineto{\pgfqpoint{5.322965in}{0.755477in}}%
\pgfpathlineto{\pgfqpoint{5.323832in}{0.714769in}}%
\pgfpathlineto{\pgfqpoint{5.324699in}{0.821536in}}%
\pgfpathlineto{\pgfqpoint{5.325566in}{0.752703in}}%
\pgfpathlineto{\pgfqpoint{5.326434in}{0.776876in}}%
\pgfpathlineto{\pgfqpoint{5.327301in}{0.942998in}}%
\pgfpathlineto{\pgfqpoint{5.328168in}{0.730427in}}%
\pgfpathlineto{\pgfqpoint{5.329035in}{0.802858in}}%
\pgfpathlineto{\pgfqpoint{5.329902in}{1.020069in}}%
\pgfpathlineto{\pgfqpoint{5.331636in}{0.726776in}}%
\pgfpathlineto{\pgfqpoint{5.334238in}{0.851132in}}%
\pgfpathlineto{\pgfqpoint{5.335105in}{0.731364in}}%
\pgfpathlineto{\pgfqpoint{5.335972in}{0.928159in}}%
\pgfpathlineto{\pgfqpoint{5.337706in}{0.715348in}}%
\pgfpathlineto{\pgfqpoint{5.338573in}{0.700839in}}%
\pgfpathlineto{\pgfqpoint{5.340308in}{0.886951in}}%
\pgfpathlineto{\pgfqpoint{5.342909in}{0.753097in}}%
\pgfpathlineto{\pgfqpoint{5.343776in}{0.756708in}}%
\pgfpathlineto{\pgfqpoint{5.344643in}{0.755649in}}%
\pgfpathlineto{\pgfqpoint{5.346378in}{0.944081in}}%
\pgfpathlineto{\pgfqpoint{5.348112in}{0.698409in}}%
\pgfpathlineto{\pgfqpoint{5.349846in}{0.950764in}}%
\pgfpathlineto{\pgfqpoint{5.351580in}{0.702399in}}%
\pgfpathlineto{\pgfqpoint{5.353315in}{0.813099in}}%
\pgfpathlineto{\pgfqpoint{5.354182in}{0.778819in}}%
\pgfpathlineto{\pgfqpoint{5.355049in}{1.380775in}}%
\pgfpathlineto{\pgfqpoint{5.356783in}{0.839040in}}%
\pgfpathlineto{\pgfqpoint{5.357650in}{0.982184in}}%
\pgfpathlineto{\pgfqpoint{5.358517in}{0.714645in}}%
\pgfpathlineto{\pgfqpoint{5.359385in}{0.786285in}}%
\pgfpathlineto{\pgfqpoint{5.360252in}{0.701780in}}%
\pgfpathlineto{\pgfqpoint{5.361119in}{0.815863in}}%
\pgfpathlineto{\pgfqpoint{5.361986in}{0.741327in}}%
\pgfpathlineto{\pgfqpoint{5.363720in}{0.762811in}}%
\pgfpathlineto{\pgfqpoint{5.364587in}{0.703388in}}%
\pgfpathlineto{\pgfqpoint{5.365455in}{0.775934in}}%
\pgfpathlineto{\pgfqpoint{5.367189in}{0.708257in}}%
\pgfpathlineto{\pgfqpoint{5.368923in}{0.801234in}}%
\pgfpathlineto{\pgfqpoint{5.370657in}{0.722216in}}%
\pgfpathlineto{\pgfqpoint{5.371524in}{0.724190in}}%
\pgfpathlineto{\pgfqpoint{5.372392in}{0.719076in}}%
\pgfpathlineto{\pgfqpoint{5.373259in}{0.778326in}}%
\pgfpathlineto{\pgfqpoint{5.374126in}{0.929219in}}%
\pgfpathlineto{\pgfqpoint{5.375860in}{0.736524in}}%
\pgfpathlineto{\pgfqpoint{5.376727in}{0.773659in}}%
\pgfpathlineto{\pgfqpoint{5.378462in}{0.728662in}}%
\pgfpathlineto{\pgfqpoint{5.380196in}{0.818362in}}%
\pgfpathlineto{\pgfqpoint{5.381063in}{0.722080in}}%
\pgfpathlineto{\pgfqpoint{5.381930in}{0.780577in}}%
\pgfpathlineto{\pgfqpoint{5.382797in}{0.709952in}}%
\pgfpathlineto{\pgfqpoint{5.383664in}{0.733500in}}%
\pgfpathlineto{\pgfqpoint{5.384531in}{0.721640in}}%
\pgfpathlineto{\pgfqpoint{5.385399in}{0.747737in}}%
\pgfpathlineto{\pgfqpoint{5.386266in}{0.811433in}}%
\pgfpathlineto{\pgfqpoint{5.387133in}{0.725155in}}%
\pgfpathlineto{\pgfqpoint{5.388000in}{0.741954in}}%
\pgfpathlineto{\pgfqpoint{5.388867in}{0.740019in}}%
\pgfpathlineto{\pgfqpoint{5.389734in}{0.715245in}}%
\pgfpathlineto{\pgfqpoint{5.390601in}{0.737313in}}%
\pgfpathlineto{\pgfqpoint{5.391469in}{0.717138in}}%
\pgfpathlineto{\pgfqpoint{5.392336in}{0.722827in}}%
\pgfpathlineto{\pgfqpoint{5.393203in}{0.760950in}}%
\pgfpathlineto{\pgfqpoint{5.394070in}{0.701030in}}%
\pgfpathlineto{\pgfqpoint{5.395804in}{1.028233in}}%
\pgfpathlineto{\pgfqpoint{5.397538in}{0.723169in}}%
\pgfpathlineto{\pgfqpoint{5.398406in}{0.796538in}}%
\pgfpathlineto{\pgfqpoint{5.399273in}{0.754622in}}%
\pgfpathlineto{\pgfqpoint{5.400140in}{0.784560in}}%
\pgfpathlineto{\pgfqpoint{5.401007in}{0.713112in}}%
\pgfpathlineto{\pgfqpoint{5.402741in}{0.930968in}}%
\pgfpathlineto{\pgfqpoint{5.403608in}{0.723463in}}%
\pgfpathlineto{\pgfqpoint{5.404476in}{0.758667in}}%
\pgfpathlineto{\pgfqpoint{5.405343in}{0.698605in}}%
\pgfpathlineto{\pgfqpoint{5.406210in}{0.770601in}}%
\pgfpathlineto{\pgfqpoint{5.407077in}{0.726280in}}%
\pgfpathlineto{\pgfqpoint{5.408811in}{0.847306in}}%
\pgfpathlineto{\pgfqpoint{5.409678in}{0.704097in}}%
\pgfpathlineto{\pgfqpoint{5.410545in}{0.752344in}}%
\pgfpathlineto{\pgfqpoint{5.411413in}{0.697248in}}%
\pgfpathlineto{\pgfqpoint{5.414014in}{0.839453in}}%
\pgfpathlineto{\pgfqpoint{5.415748in}{0.708839in}}%
\pgfpathlineto{\pgfqpoint{5.416615in}{0.733999in}}%
\pgfpathlineto{\pgfqpoint{5.417483in}{0.726901in}}%
\pgfpathlineto{\pgfqpoint{5.419217in}{0.864085in}}%
\pgfpathlineto{\pgfqpoint{5.420084in}{0.717810in}}%
\pgfpathlineto{\pgfqpoint{5.421818in}{0.838317in}}%
\pgfpathlineto{\pgfqpoint{5.422685in}{0.718825in}}%
\pgfpathlineto{\pgfqpoint{5.423552in}{0.735214in}}%
\pgfpathlineto{\pgfqpoint{5.424420in}{0.792384in}}%
\pgfpathlineto{\pgfqpoint{5.426154in}{0.747537in}}%
\pgfpathlineto{\pgfqpoint{5.427021in}{0.910780in}}%
\pgfpathlineto{\pgfqpoint{5.428755in}{0.697529in}}%
\pgfpathlineto{\pgfqpoint{5.429622in}{0.768205in}}%
\pgfpathlineto{\pgfqpoint{5.430490in}{0.708252in}}%
\pgfpathlineto{\pgfqpoint{5.431357in}{0.709276in}}%
\pgfpathlineto{\pgfqpoint{5.432224in}{0.712207in}}%
\pgfpathlineto{\pgfqpoint{5.433958in}{0.879602in}}%
\pgfpathlineto{\pgfqpoint{5.434825in}{0.744681in}}%
\pgfpathlineto{\pgfqpoint{5.435692in}{0.897453in}}%
\pgfpathlineto{\pgfqpoint{5.437427in}{0.742919in}}%
\pgfpathlineto{\pgfqpoint{5.438294in}{1.124439in}}%
\pgfpathlineto{\pgfqpoint{5.440028in}{0.775499in}}%
\pgfpathlineto{\pgfqpoint{5.440895in}{1.081044in}}%
\pgfpathlineto{\pgfqpoint{5.441762in}{0.777955in}}%
\pgfpathlineto{\pgfqpoint{5.442629in}{0.802733in}}%
\pgfpathlineto{\pgfqpoint{5.443497in}{0.928222in}}%
\pgfpathlineto{\pgfqpoint{5.444364in}{0.767697in}}%
\pgfpathlineto{\pgfqpoint{5.445231in}{0.789338in}}%
\pgfpathlineto{\pgfqpoint{5.446098in}{0.746269in}}%
\pgfpathlineto{\pgfqpoint{5.446965in}{0.813135in}}%
\pgfpathlineto{\pgfqpoint{5.447832in}{0.966854in}}%
\pgfpathlineto{\pgfqpoint{5.449566in}{0.702254in}}%
\pgfpathlineto{\pgfqpoint{5.451301in}{0.809996in}}%
\pgfpathlineto{\pgfqpoint{5.452168in}{0.809336in}}%
\pgfpathlineto{\pgfqpoint{5.453035in}{0.808500in}}%
\pgfpathlineto{\pgfqpoint{5.453902in}{0.718579in}}%
\pgfpathlineto{\pgfqpoint{5.454769in}{0.734448in}}%
\pgfpathlineto{\pgfqpoint{5.455636in}{0.739283in}}%
\pgfpathlineto{\pgfqpoint{5.456503in}{0.711598in}}%
\pgfpathlineto{\pgfqpoint{5.457371in}{0.714380in}}%
\pgfpathlineto{\pgfqpoint{5.458238in}{0.832032in}}%
\pgfpathlineto{\pgfqpoint{5.459105in}{1.236748in}}%
\pgfpathlineto{\pgfqpoint{5.459972in}{0.826008in}}%
\pgfpathlineto{\pgfqpoint{5.460839in}{0.831027in}}%
\pgfpathlineto{\pgfqpoint{5.461706in}{1.075331in}}%
\pgfpathlineto{\pgfqpoint{5.463441in}{0.723150in}}%
\pgfpathlineto{\pgfqpoint{5.464308in}{0.744868in}}%
\pgfpathlineto{\pgfqpoint{5.465175in}{0.714996in}}%
\pgfpathlineto{\pgfqpoint{5.466909in}{0.862171in}}%
\pgfpathlineto{\pgfqpoint{5.467776in}{0.720611in}}%
\pgfpathlineto{\pgfqpoint{5.468643in}{0.724362in}}%
\pgfpathlineto{\pgfqpoint{5.469510in}{0.747029in}}%
\pgfpathlineto{\pgfqpoint{5.470378in}{0.703041in}}%
\pgfpathlineto{\pgfqpoint{5.471245in}{0.708856in}}%
\pgfpathlineto{\pgfqpoint{5.472112in}{0.709710in}}%
\pgfpathlineto{\pgfqpoint{5.473846in}{0.761974in}}%
\pgfpathlineto{\pgfqpoint{5.474713in}{0.739346in}}%
\pgfpathlineto{\pgfqpoint{5.475580in}{0.876633in}}%
\pgfpathlineto{\pgfqpoint{5.476448in}{0.845689in}}%
\pgfpathlineto{\pgfqpoint{5.477315in}{0.700496in}}%
\pgfpathlineto{\pgfqpoint{5.478182in}{0.784367in}}%
\pgfpathlineto{\pgfqpoint{5.479049in}{0.715678in}}%
\pgfpathlineto{\pgfqpoint{5.479916in}{0.749134in}}%
\pgfpathlineto{\pgfqpoint{5.480783in}{0.739489in}}%
\pgfpathlineto{\pgfqpoint{5.481650in}{0.767222in}}%
\pgfpathlineto{\pgfqpoint{5.482517in}{0.921939in}}%
\pgfpathlineto{\pgfqpoint{5.484252in}{0.710579in}}%
\pgfpathlineto{\pgfqpoint{5.485119in}{0.708503in}}%
\pgfpathlineto{\pgfqpoint{5.485986in}{0.702321in}}%
\pgfpathlineto{\pgfqpoint{5.486853in}{0.715726in}}%
\pgfpathlineto{\pgfqpoint{5.488587in}{0.919483in}}%
\pgfpathlineto{\pgfqpoint{5.489455in}{0.718340in}}%
\pgfpathlineto{\pgfqpoint{5.490322in}{0.752038in}}%
\pgfpathlineto{\pgfqpoint{5.492923in}{0.702775in}}%
\pgfpathlineto{\pgfqpoint{5.494657in}{0.903625in}}%
\pgfpathlineto{\pgfqpoint{5.495524in}{0.705098in}}%
\pgfpathlineto{\pgfqpoint{5.496392in}{0.820344in}}%
\pgfpathlineto{\pgfqpoint{5.499860in}{0.697169in}}%
\pgfpathlineto{\pgfqpoint{5.501594in}{0.936139in}}%
\pgfpathlineto{\pgfqpoint{5.502462in}{0.817561in}}%
\pgfpathlineto{\pgfqpoint{5.503329in}{0.833857in}}%
\pgfpathlineto{\pgfqpoint{5.505063in}{0.942258in}}%
\pgfpathlineto{\pgfqpoint{5.505930in}{0.698829in}}%
\pgfpathlineto{\pgfqpoint{5.506797in}{0.930753in}}%
\pgfpathlineto{\pgfqpoint{5.508531in}{0.707974in}}%
\pgfpathlineto{\pgfqpoint{5.509399in}{0.752202in}}%
\pgfpathlineto{\pgfqpoint{5.510266in}{0.759405in}}%
\pgfpathlineto{\pgfqpoint{5.511133in}{0.784999in}}%
\pgfpathlineto{\pgfqpoint{5.512000in}{0.731255in}}%
\pgfpathlineto{\pgfqpoint{5.512867in}{0.947214in}}%
\pgfpathlineto{\pgfqpoint{5.514601in}{0.701199in}}%
\pgfpathlineto{\pgfqpoint{5.515469in}{0.728632in}}%
\pgfpathlineto{\pgfqpoint{5.516336in}{0.726732in}}%
\pgfpathlineto{\pgfqpoint{5.518070in}{0.956939in}}%
\pgfpathlineto{\pgfqpoint{5.518937in}{0.753733in}}%
\pgfpathlineto{\pgfqpoint{5.519804in}{0.804811in}}%
\pgfpathlineto{\pgfqpoint{5.520671in}{0.794740in}}%
\pgfpathlineto{\pgfqpoint{5.522406in}{0.732123in}}%
\pgfpathlineto{\pgfqpoint{5.523273in}{0.723687in}}%
\pgfpathlineto{\pgfqpoint{5.524140in}{0.738951in}}%
\pgfpathlineto{\pgfqpoint{5.525007in}{0.725185in}}%
\pgfpathlineto{\pgfqpoint{5.525874in}{0.741269in}}%
\pgfpathlineto{\pgfqpoint{5.526741in}{0.804752in}}%
\pgfpathlineto{\pgfqpoint{5.527608in}{0.701711in}}%
\pgfpathlineto{\pgfqpoint{5.530210in}{0.923069in}}%
\pgfpathlineto{\pgfqpoint{5.531077in}{0.839425in}}%
\pgfpathlineto{\pgfqpoint{5.531944in}{0.879410in}}%
\pgfpathlineto{\pgfqpoint{5.532811in}{0.875143in}}%
\pgfpathlineto{\pgfqpoint{5.533678in}{0.705469in}}%
\pgfpathlineto{\pgfqpoint{5.534545in}{0.763306in}}%
\pgfpathlineto{\pgfqpoint{5.534545in}{0.763306in}}%
\pgfusepath{stroke}%
\end{pgfscope}%
\begin{pgfscope}%
\pgfsetrectcap%
\pgfsetmiterjoin%
\pgfsetlinewidth{0.803000pt}%
\definecolor{currentstroke}{rgb}{0.000000,0.000000,0.000000}%
\pgfsetstrokecolor{currentstroke}%
\pgfsetdash{}{0pt}%
\pgfpathmoveto{\pgfqpoint{0.800000in}{0.528000in}}%
\pgfpathlineto{\pgfqpoint{0.800000in}{4.224000in}}%
\pgfusepath{stroke}%
\end{pgfscope}%
\begin{pgfscope}%
\pgfsetrectcap%
\pgfsetmiterjoin%
\pgfsetlinewidth{0.803000pt}%
\definecolor{currentstroke}{rgb}{0.000000,0.000000,0.000000}%
\pgfsetstrokecolor{currentstroke}%
\pgfsetdash{}{0pt}%
\pgfpathmoveto{\pgfqpoint{5.760000in}{0.528000in}}%
\pgfpathlineto{\pgfqpoint{5.760000in}{4.224000in}}%
\pgfusepath{stroke}%
\end{pgfscope}%
\begin{pgfscope}%
\pgfsetrectcap%
\pgfsetmiterjoin%
\pgfsetlinewidth{0.803000pt}%
\definecolor{currentstroke}{rgb}{0.000000,0.000000,0.000000}%
\pgfsetstrokecolor{currentstroke}%
\pgfsetdash{}{0pt}%
\pgfpathmoveto{\pgfqpoint{0.800000in}{0.528000in}}%
\pgfpathlineto{\pgfqpoint{5.760000in}{0.528000in}}%
\pgfusepath{stroke}%
\end{pgfscope}%
\begin{pgfscope}%
\pgfsetrectcap%
\pgfsetmiterjoin%
\pgfsetlinewidth{0.803000pt}%
\definecolor{currentstroke}{rgb}{0.000000,0.000000,0.000000}%
\pgfsetstrokecolor{currentstroke}%
\pgfsetdash{}{0pt}%
\pgfpathmoveto{\pgfqpoint{0.800000in}{4.224000in}}%
\pgfpathlineto{\pgfqpoint{5.760000in}{4.224000in}}%
\pgfusepath{stroke}%
\end{pgfscope}%
\end{pgfpicture}%
\makeatother%
\endgroup%
}}
  \caption{Received signal after demodulation and pulse compression.}
  \label{fig:task2}
\end{figure}
\section{Task 3}\label{sec:3}
The first peak of the interpolated signal can be seen in Figure~\ref{fig:task3}. The $\SI{3}{\deci\bel}$ width was calculated to be $\SI{9.4}{\nano\second}$ using \textit{findbw} in ``ha1utils.pyx.''
The theoretical range resolution should be $\SI{1}{\kilo\metre}$ since the echos have to be separated by the pulse width in order to be able to separated after pulse compresison, and the pulse width is $\SI{6.7}{\micro\second}$.
The theoretical range resolution seems unrealistically high but the pulse is also very long.

The time-bandwidth product after pulse compression is 0.7 ($t_p=\SI{9.4}{\nano\second}$, $B$ from Table~\ref{tab:task1}). The theoretical time-bandwidth product should because 0.5 since the modulation allows for double spectral efficienty. The simulated time-bandiwdth product is significantly lower (and close to the theoretical value) after pulse compresison.
\begin{figure}[h]
  \centering
  \noindent\makebox[\textwidth]{\scalebox{0.90}{%% Creator: Matplotlib, PGF backend
%%
%% To include the figure in your LaTeX document, write
%%   \input{<filename>.pgf}
%%
%% Make sure the required packages are loaded in your preamble
%%   \usepackage{pgf}
%%
%% Figures using additional raster images can only be included by \input if
%% they are in the same directory as the main LaTeX file. For loading figures
%% from other directories you can use the `import` package
%%   \usepackage{import}
%% and then include the figures with
%%   \import{<path to file>}{<filename>.pgf}
%%
%% Matplotlib used the following preamble
%%   \usepackage{fontspec}
%%   \setmainfont{DejaVu Serif}
%%   \setsansfont{DejaVu Sans}
%%   \setmonofont{DejaVu Sans Mono}
%%
\begingroup%
\makeatletter%
\begin{pgfpicture}%
\pgfpathrectangle{\pgfpointorigin}{\pgfqpoint{6.400000in}{4.800000in}}%
\pgfusepath{use as bounding box, clip}%
\begin{pgfscope}%
\pgfsetbuttcap%
\pgfsetmiterjoin%
\definecolor{currentfill}{rgb}{1.000000,1.000000,1.000000}%
\pgfsetfillcolor{currentfill}%
\pgfsetlinewidth{0.000000pt}%
\definecolor{currentstroke}{rgb}{1.000000,1.000000,1.000000}%
\pgfsetstrokecolor{currentstroke}%
\pgfsetdash{}{0pt}%
\pgfpathmoveto{\pgfqpoint{0.000000in}{0.000000in}}%
\pgfpathlineto{\pgfqpoint{6.400000in}{0.000000in}}%
\pgfpathlineto{\pgfqpoint{6.400000in}{4.800000in}}%
\pgfpathlineto{\pgfqpoint{0.000000in}{4.800000in}}%
\pgfpathclose%
\pgfusepath{fill}%
\end{pgfscope}%
\begin{pgfscope}%
\pgfsetbuttcap%
\pgfsetmiterjoin%
\definecolor{currentfill}{rgb}{1.000000,1.000000,1.000000}%
\pgfsetfillcolor{currentfill}%
\pgfsetlinewidth{0.000000pt}%
\definecolor{currentstroke}{rgb}{0.000000,0.000000,0.000000}%
\pgfsetstrokecolor{currentstroke}%
\pgfsetstrokeopacity{0.000000}%
\pgfsetdash{}{0pt}%
\pgfpathmoveto{\pgfqpoint{0.800000in}{0.528000in}}%
\pgfpathlineto{\pgfqpoint{5.760000in}{0.528000in}}%
\pgfpathlineto{\pgfqpoint{5.760000in}{4.224000in}}%
\pgfpathlineto{\pgfqpoint{0.800000in}{4.224000in}}%
\pgfpathclose%
\pgfusepath{fill}%
\end{pgfscope}%
\begin{pgfscope}%
\pgfsetbuttcap%
\pgfsetroundjoin%
\definecolor{currentfill}{rgb}{0.000000,0.000000,0.000000}%
\pgfsetfillcolor{currentfill}%
\pgfsetlinewidth{0.803000pt}%
\definecolor{currentstroke}{rgb}{0.000000,0.000000,0.000000}%
\pgfsetstrokecolor{currentstroke}%
\pgfsetdash{}{0pt}%
\pgfsys@defobject{currentmarker}{\pgfqpoint{0.000000in}{-0.048611in}}{\pgfqpoint{0.000000in}{0.000000in}}{%
\pgfpathmoveto{\pgfqpoint{0.000000in}{0.000000in}}%
\pgfpathlineto{\pgfqpoint{0.000000in}{-0.048611in}}%
\pgfusepath{stroke,fill}%
}%
\begin{pgfscope}%
\pgfsys@transformshift{1.341477in}{0.528000in}%
\pgfsys@useobject{currentmarker}{}%
\end{pgfscope}%
\end{pgfscope}%
\begin{pgfscope}%
\pgftext[x=1.341477in,y=0.430778in,,top]{\sffamily\fontsize{10.000000}{12.000000}\selectfont 8.740}%
\end{pgfscope}%
\begin{pgfscope}%
\pgfsetbuttcap%
\pgfsetroundjoin%
\definecolor{currentfill}{rgb}{0.000000,0.000000,0.000000}%
\pgfsetfillcolor{currentfill}%
\pgfsetlinewidth{0.803000pt}%
\definecolor{currentstroke}{rgb}{0.000000,0.000000,0.000000}%
\pgfsetstrokecolor{currentstroke}%
\pgfsetdash{}{0pt}%
\pgfsys@defobject{currentmarker}{\pgfqpoint{0.000000in}{-0.048611in}}{\pgfqpoint{0.000000in}{0.000000in}}{%
\pgfpathmoveto{\pgfqpoint{0.000000in}{0.000000in}}%
\pgfpathlineto{\pgfqpoint{0.000000in}{-0.048611in}}%
\pgfusepath{stroke,fill}%
}%
\begin{pgfscope}%
\pgfsys@transformshift{2.323050in}{0.528000in}%
\pgfsys@useobject{currentmarker}{}%
\end{pgfscope}%
\end{pgfscope}%
\begin{pgfscope}%
\pgftext[x=2.323050in,y=0.430778in,,top]{\sffamily\fontsize{10.000000}{12.000000}\selectfont 8.742}%
\end{pgfscope}%
\begin{pgfscope}%
\pgfsetbuttcap%
\pgfsetroundjoin%
\definecolor{currentfill}{rgb}{0.000000,0.000000,0.000000}%
\pgfsetfillcolor{currentfill}%
\pgfsetlinewidth{0.803000pt}%
\definecolor{currentstroke}{rgb}{0.000000,0.000000,0.000000}%
\pgfsetstrokecolor{currentstroke}%
\pgfsetdash{}{0pt}%
\pgfsys@defobject{currentmarker}{\pgfqpoint{0.000000in}{-0.048611in}}{\pgfqpoint{0.000000in}{0.000000in}}{%
\pgfpathmoveto{\pgfqpoint{0.000000in}{0.000000in}}%
\pgfpathlineto{\pgfqpoint{0.000000in}{-0.048611in}}%
\pgfusepath{stroke,fill}%
}%
\begin{pgfscope}%
\pgfsys@transformshift{3.304623in}{0.528000in}%
\pgfsys@useobject{currentmarker}{}%
\end{pgfscope}%
\end{pgfscope}%
\begin{pgfscope}%
\pgftext[x=3.304623in,y=0.430778in,,top]{\sffamily\fontsize{10.000000}{12.000000}\selectfont 8.744}%
\end{pgfscope}%
\begin{pgfscope}%
\pgfsetbuttcap%
\pgfsetroundjoin%
\definecolor{currentfill}{rgb}{0.000000,0.000000,0.000000}%
\pgfsetfillcolor{currentfill}%
\pgfsetlinewidth{0.803000pt}%
\definecolor{currentstroke}{rgb}{0.000000,0.000000,0.000000}%
\pgfsetstrokecolor{currentstroke}%
\pgfsetdash{}{0pt}%
\pgfsys@defobject{currentmarker}{\pgfqpoint{0.000000in}{-0.048611in}}{\pgfqpoint{0.000000in}{0.000000in}}{%
\pgfpathmoveto{\pgfqpoint{0.000000in}{0.000000in}}%
\pgfpathlineto{\pgfqpoint{0.000000in}{-0.048611in}}%
\pgfusepath{stroke,fill}%
}%
\begin{pgfscope}%
\pgfsys@transformshift{4.286196in}{0.528000in}%
\pgfsys@useobject{currentmarker}{}%
\end{pgfscope}%
\end{pgfscope}%
\begin{pgfscope}%
\pgftext[x=4.286196in,y=0.430778in,,top]{\sffamily\fontsize{10.000000}{12.000000}\selectfont 8.746}%
\end{pgfscope}%
\begin{pgfscope}%
\pgfsetbuttcap%
\pgfsetroundjoin%
\definecolor{currentfill}{rgb}{0.000000,0.000000,0.000000}%
\pgfsetfillcolor{currentfill}%
\pgfsetlinewidth{0.803000pt}%
\definecolor{currentstroke}{rgb}{0.000000,0.000000,0.000000}%
\pgfsetstrokecolor{currentstroke}%
\pgfsetdash{}{0pt}%
\pgfsys@defobject{currentmarker}{\pgfqpoint{0.000000in}{-0.048611in}}{\pgfqpoint{0.000000in}{0.000000in}}{%
\pgfpathmoveto{\pgfqpoint{0.000000in}{0.000000in}}%
\pgfpathlineto{\pgfqpoint{0.000000in}{-0.048611in}}%
\pgfusepath{stroke,fill}%
}%
\begin{pgfscope}%
\pgfsys@transformshift{5.267769in}{0.528000in}%
\pgfsys@useobject{currentmarker}{}%
\end{pgfscope}%
\end{pgfscope}%
\begin{pgfscope}%
\pgftext[x=5.267769in,y=0.430778in,,top]{\sffamily\fontsize{10.000000}{12.000000}\selectfont 8.748}%
\end{pgfscope}%
\begin{pgfscope}%
\pgftext[x=3.280000in,y=0.240809in,,top]{\sffamily\fontsize{10.000000}{12.000000}\selectfont Distance [km]}%
\end{pgfscope}%
\begin{pgfscope}%
\pgfsetbuttcap%
\pgfsetroundjoin%
\definecolor{currentfill}{rgb}{0.000000,0.000000,0.000000}%
\pgfsetfillcolor{currentfill}%
\pgfsetlinewidth{0.803000pt}%
\definecolor{currentstroke}{rgb}{0.000000,0.000000,0.000000}%
\pgfsetstrokecolor{currentstroke}%
\pgfsetdash{}{0pt}%
\pgfsys@defobject{currentmarker}{\pgfqpoint{-0.048611in}{0.000000in}}{\pgfqpoint{0.000000in}{0.000000in}}{%
\pgfpathmoveto{\pgfqpoint{0.000000in}{0.000000in}}%
\pgfpathlineto{\pgfqpoint{-0.048611in}{0.000000in}}%
\pgfusepath{stroke,fill}%
}%
\begin{pgfscope}%
\pgfsys@transformshift{0.800000in}{0.545880in}%
\pgfsys@useobject{currentmarker}{}%
\end{pgfscope}%
\end{pgfscope}%
\begin{pgfscope}%
\pgftext[x=0.481898in,y=0.493118in,left,base]{\sffamily\fontsize{10.000000}{12.000000}\selectfont 0.0}%
\end{pgfscope}%
\begin{pgfscope}%
\pgfsetbuttcap%
\pgfsetroundjoin%
\definecolor{currentfill}{rgb}{0.000000,0.000000,0.000000}%
\pgfsetfillcolor{currentfill}%
\pgfsetlinewidth{0.803000pt}%
\definecolor{currentstroke}{rgb}{0.000000,0.000000,0.000000}%
\pgfsetstrokecolor{currentstroke}%
\pgfsetdash{}{0pt}%
\pgfsys@defobject{currentmarker}{\pgfqpoint{-0.048611in}{0.000000in}}{\pgfqpoint{0.000000in}{0.000000in}}{%
\pgfpathmoveto{\pgfqpoint{0.000000in}{0.000000in}}%
\pgfpathlineto{\pgfqpoint{-0.048611in}{0.000000in}}%
\pgfusepath{stroke,fill}%
}%
\begin{pgfscope}%
\pgfsys@transformshift{0.800000in}{1.272014in}%
\pgfsys@useobject{currentmarker}{}%
\end{pgfscope}%
\end{pgfscope}%
\begin{pgfscope}%
\pgftext[x=0.481898in,y=1.219252in,left,base]{\sffamily\fontsize{10.000000}{12.000000}\selectfont 0.2}%
\end{pgfscope}%
\begin{pgfscope}%
\pgfsetbuttcap%
\pgfsetroundjoin%
\definecolor{currentfill}{rgb}{0.000000,0.000000,0.000000}%
\pgfsetfillcolor{currentfill}%
\pgfsetlinewidth{0.803000pt}%
\definecolor{currentstroke}{rgb}{0.000000,0.000000,0.000000}%
\pgfsetstrokecolor{currentstroke}%
\pgfsetdash{}{0pt}%
\pgfsys@defobject{currentmarker}{\pgfqpoint{-0.048611in}{0.000000in}}{\pgfqpoint{0.000000in}{0.000000in}}{%
\pgfpathmoveto{\pgfqpoint{0.000000in}{0.000000in}}%
\pgfpathlineto{\pgfqpoint{-0.048611in}{0.000000in}}%
\pgfusepath{stroke,fill}%
}%
\begin{pgfscope}%
\pgfsys@transformshift{0.800000in}{1.998148in}%
\pgfsys@useobject{currentmarker}{}%
\end{pgfscope}%
\end{pgfscope}%
\begin{pgfscope}%
\pgftext[x=0.481898in,y=1.945386in,left,base]{\sffamily\fontsize{10.000000}{12.000000}\selectfont 0.4}%
\end{pgfscope}%
\begin{pgfscope}%
\pgfsetbuttcap%
\pgfsetroundjoin%
\definecolor{currentfill}{rgb}{0.000000,0.000000,0.000000}%
\pgfsetfillcolor{currentfill}%
\pgfsetlinewidth{0.803000pt}%
\definecolor{currentstroke}{rgb}{0.000000,0.000000,0.000000}%
\pgfsetstrokecolor{currentstroke}%
\pgfsetdash{}{0pt}%
\pgfsys@defobject{currentmarker}{\pgfqpoint{-0.048611in}{0.000000in}}{\pgfqpoint{0.000000in}{0.000000in}}{%
\pgfpathmoveto{\pgfqpoint{0.000000in}{0.000000in}}%
\pgfpathlineto{\pgfqpoint{-0.048611in}{0.000000in}}%
\pgfusepath{stroke,fill}%
}%
\begin{pgfscope}%
\pgfsys@transformshift{0.800000in}{2.724282in}%
\pgfsys@useobject{currentmarker}{}%
\end{pgfscope}%
\end{pgfscope}%
\begin{pgfscope}%
\pgftext[x=0.481898in,y=2.671521in,left,base]{\sffamily\fontsize{10.000000}{12.000000}\selectfont 0.6}%
\end{pgfscope}%
\begin{pgfscope}%
\pgfsetbuttcap%
\pgfsetroundjoin%
\definecolor{currentfill}{rgb}{0.000000,0.000000,0.000000}%
\pgfsetfillcolor{currentfill}%
\pgfsetlinewidth{0.803000pt}%
\definecolor{currentstroke}{rgb}{0.000000,0.000000,0.000000}%
\pgfsetstrokecolor{currentstroke}%
\pgfsetdash{}{0pt}%
\pgfsys@defobject{currentmarker}{\pgfqpoint{-0.048611in}{0.000000in}}{\pgfqpoint{0.000000in}{0.000000in}}{%
\pgfpathmoveto{\pgfqpoint{0.000000in}{0.000000in}}%
\pgfpathlineto{\pgfqpoint{-0.048611in}{0.000000in}}%
\pgfusepath{stroke,fill}%
}%
\begin{pgfscope}%
\pgfsys@transformshift{0.800000in}{3.450416in}%
\pgfsys@useobject{currentmarker}{}%
\end{pgfscope}%
\end{pgfscope}%
\begin{pgfscope}%
\pgftext[x=0.481898in,y=3.397655in,left,base]{\sffamily\fontsize{10.000000}{12.000000}\selectfont 0.8}%
\end{pgfscope}%
\begin{pgfscope}%
\pgfsetbuttcap%
\pgfsetroundjoin%
\definecolor{currentfill}{rgb}{0.000000,0.000000,0.000000}%
\pgfsetfillcolor{currentfill}%
\pgfsetlinewidth{0.803000pt}%
\definecolor{currentstroke}{rgb}{0.000000,0.000000,0.000000}%
\pgfsetstrokecolor{currentstroke}%
\pgfsetdash{}{0pt}%
\pgfsys@defobject{currentmarker}{\pgfqpoint{-0.048611in}{0.000000in}}{\pgfqpoint{0.000000in}{0.000000in}}{%
\pgfpathmoveto{\pgfqpoint{0.000000in}{0.000000in}}%
\pgfpathlineto{\pgfqpoint{-0.048611in}{0.000000in}}%
\pgfusepath{stroke,fill}%
}%
\begin{pgfscope}%
\pgfsys@transformshift{0.800000in}{4.176550in}%
\pgfsys@useobject{currentmarker}{}%
\end{pgfscope}%
\end{pgfscope}%
\begin{pgfscope}%
\pgftext[x=0.481898in,y=4.123789in,left,base]{\sffamily\fontsize{10.000000}{12.000000}\selectfont 1.0}%
\end{pgfscope}%
\begin{pgfscope}%
\pgftext[x=0.426343in,y=2.376000in,,bottom,rotate=90.000000]{\sffamily\fontsize{10.000000}{12.000000}\selectfont Relative (squared) magnitude}%
\end{pgfscope}%
\begin{pgfscope}%
\pgfpathrectangle{\pgfqpoint{0.800000in}{0.528000in}}{\pgfqpoint{4.960000in}{3.696000in}} %
\pgfusepath{clip}%
\pgfsetrectcap%
\pgfsetroundjoin%
\pgfsetlinewidth{1.505625pt}%
\definecolor{currentstroke}{rgb}{0.121569,0.466667,0.705882}%
\pgfsetstrokecolor{currentstroke}%
\pgfsetdash{}{0pt}%
\pgfpathmoveto{\pgfqpoint{0.799816in}{0.582660in}}%
\pgfpathlineto{\pgfqpoint{0.833770in}{0.588391in}}%
\pgfpathlineto{\pgfqpoint{0.867724in}{0.596467in}}%
\pgfpathlineto{\pgfqpoint{0.897906in}{0.605836in}}%
\pgfpathlineto{\pgfqpoint{0.928087in}{0.617466in}}%
\pgfpathlineto{\pgfqpoint{0.958268in}{0.631551in}}%
\pgfpathlineto{\pgfqpoint{0.988450in}{0.648273in}}%
\pgfpathlineto{\pgfqpoint{1.018631in}{0.667805in}}%
\pgfpathlineto{\pgfqpoint{1.048812in}{0.690296in}}%
\pgfpathlineto{\pgfqpoint{1.078994in}{0.715860in}}%
\pgfpathlineto{\pgfqpoint{1.109175in}{0.744572in}}%
\pgfpathlineto{\pgfqpoint{1.139357in}{0.776446in}}%
\pgfpathlineto{\pgfqpoint{1.169538in}{0.811430in}}%
\pgfpathlineto{\pgfqpoint{1.203492in}{0.854335in}}%
\pgfpathlineto{\pgfqpoint{1.241219in}{0.906019in}}%
\pgfpathlineto{\pgfqpoint{1.282718in}{0.966955in}}%
\pgfpathlineto{\pgfqpoint{1.335535in}{1.048733in}}%
\pgfpathlineto{\pgfqpoint{1.437397in}{1.207203in}}%
\pgfpathlineto{\pgfqpoint{1.475124in}{1.261200in}}%
\pgfpathlineto{\pgfqpoint{1.505305in}{1.300753in}}%
\pgfpathlineto{\pgfqpoint{1.531714in}{1.331994in}}%
\pgfpathlineto{\pgfqpoint{1.558123in}{1.359556in}}%
\pgfpathlineto{\pgfqpoint{1.580759in}{1.379909in}}%
\pgfpathlineto{\pgfqpoint{1.603395in}{1.396991in}}%
\pgfpathlineto{\pgfqpoint{1.622258in}{1.408594in}}%
\pgfpathlineto{\pgfqpoint{1.641122in}{1.417722in}}%
\pgfpathlineto{\pgfqpoint{1.659985in}{1.424332in}}%
\pgfpathlineto{\pgfqpoint{1.678848in}{1.428415in}}%
\pgfpathlineto{\pgfqpoint{1.697712in}{1.430002in}}%
\pgfpathlineto{\pgfqpoint{1.716575in}{1.429159in}}%
\pgfpathlineto{\pgfqpoint{1.735438in}{1.425995in}}%
\pgfpathlineto{\pgfqpoint{1.754302in}{1.420660in}}%
\pgfpathlineto{\pgfqpoint{1.776938in}{1.411667in}}%
\pgfpathlineto{\pgfqpoint{1.799574in}{1.400239in}}%
\pgfpathlineto{\pgfqpoint{1.825982in}{1.384497in}}%
\pgfpathlineto{\pgfqpoint{1.867482in}{1.356723in}}%
\pgfpathlineto{\pgfqpoint{1.912754in}{1.326709in}}%
\pgfpathlineto{\pgfqpoint{1.939162in}{1.311792in}}%
\pgfpathlineto{\pgfqpoint{1.958026in}{1.303259in}}%
\pgfpathlineto{\pgfqpoint{1.976889in}{1.297077in}}%
\pgfpathlineto{\pgfqpoint{1.991980in}{1.294170in}}%
\pgfpathlineto{\pgfqpoint{2.007070in}{1.293363in}}%
\pgfpathlineto{\pgfqpoint{2.022161in}{1.294914in}}%
\pgfpathlineto{\pgfqpoint{2.037252in}{1.299077in}}%
\pgfpathlineto{\pgfqpoint{2.052342in}{1.306097in}}%
\pgfpathlineto{\pgfqpoint{2.067433in}{1.316209in}}%
\pgfpathlineto{\pgfqpoint{2.082524in}{1.329635in}}%
\pgfpathlineto{\pgfqpoint{2.097614in}{1.346582in}}%
\pgfpathlineto{\pgfqpoint{2.112705in}{1.367239in}}%
\pgfpathlineto{\pgfqpoint{2.127796in}{1.391776in}}%
\pgfpathlineto{\pgfqpoint{2.142887in}{1.420340in}}%
\pgfpathlineto{\pgfqpoint{2.157977in}{1.453057in}}%
\pgfpathlineto{\pgfqpoint{2.176841in}{1.499940in}}%
\pgfpathlineto{\pgfqpoint{2.195704in}{1.553597in}}%
\pgfpathlineto{\pgfqpoint{2.214567in}{1.614092in}}%
\pgfpathlineto{\pgfqpoint{2.233431in}{1.681412in}}%
\pgfpathlineto{\pgfqpoint{2.256067in}{1.771075in}}%
\pgfpathlineto{\pgfqpoint{2.278703in}{1.870128in}}%
\pgfpathlineto{\pgfqpoint{2.305111in}{1.996918in}}%
\pgfpathlineto{\pgfqpoint{2.331520in}{2.134798in}}%
\pgfpathlineto{\pgfqpoint{2.361701in}{2.304217in}}%
\pgfpathlineto{\pgfqpoint{2.399428in}{2.530075in}}%
\pgfpathlineto{\pgfqpoint{2.452245in}{2.861946in}}%
\pgfpathlineto{\pgfqpoint{2.520153in}{3.287499in}}%
\pgfpathlineto{\pgfqpoint{2.554107in}{3.487216in}}%
\pgfpathlineto{\pgfqpoint{2.584289in}{3.651296in}}%
\pgfpathlineto{\pgfqpoint{2.606925in}{3.763639in}}%
\pgfpathlineto{\pgfqpoint{2.629561in}{3.865096in}}%
\pgfpathlineto{\pgfqpoint{2.648424in}{3.940303in}}%
\pgfpathlineto{\pgfqpoint{2.667288in}{4.006213in}}%
\pgfpathlineto{\pgfqpoint{2.682378in}{4.051787in}}%
\pgfpathlineto{\pgfqpoint{2.697469in}{4.090663in}}%
\pgfpathlineto{\pgfqpoint{2.712560in}{4.122566in}}%
\pgfpathlineto{\pgfqpoint{2.723878in}{4.141774in}}%
\pgfpathlineto{\pgfqpoint{2.735196in}{4.156841in}}%
\pgfpathlineto{\pgfqpoint{2.746514in}{4.167696in}}%
\pgfpathlineto{\pgfqpoint{2.754059in}{4.172563in}}%
\pgfpathlineto{\pgfqpoint{2.761604in}{4.175518in}}%
\pgfpathlineto{\pgfqpoint{2.769150in}{4.176550in}}%
\pgfpathlineto{\pgfqpoint{2.776695in}{4.175652in}}%
\pgfpathlineto{\pgfqpoint{2.784240in}{4.172818in}}%
\pgfpathlineto{\pgfqpoint{2.791786in}{4.168046in}}%
\pgfpathlineto{\pgfqpoint{2.799331in}{4.161337in}}%
\pgfpathlineto{\pgfqpoint{2.810649in}{4.147650in}}%
\pgfpathlineto{\pgfqpoint{2.821967in}{4.129639in}}%
\pgfpathlineto{\pgfqpoint{2.833285in}{4.107343in}}%
\pgfpathlineto{\pgfqpoint{2.844603in}{4.080819in}}%
\pgfpathlineto{\pgfqpoint{2.859694in}{4.039000in}}%
\pgfpathlineto{\pgfqpoint{2.874784in}{3.989996in}}%
\pgfpathlineto{\pgfqpoint{2.889875in}{3.934062in}}%
\pgfpathlineto{\pgfqpoint{2.908738in}{3.854857in}}%
\pgfpathlineto{\pgfqpoint{2.927602in}{3.765967in}}%
\pgfpathlineto{\pgfqpoint{2.950238in}{3.647613in}}%
\pgfpathlineto{\pgfqpoint{2.976646in}{3.495351in}}%
\pgfpathlineto{\pgfqpoint{3.006828in}{3.305951in}}%
\pgfpathlineto{\pgfqpoint{3.044554in}{3.052656in}}%
\pgfpathlineto{\pgfqpoint{3.169053in}{2.201621in}}%
\pgfpathlineto{\pgfqpoint{3.199234in}{2.018683in}}%
\pgfpathlineto{\pgfqpoint{3.221870in}{1.893370in}}%
\pgfpathlineto{\pgfqpoint{3.244506in}{1.780101in}}%
\pgfpathlineto{\pgfqpoint{3.263369in}{1.695980in}}%
\pgfpathlineto{\pgfqpoint{3.282233in}{1.621999in}}%
\pgfpathlineto{\pgfqpoint{3.297323in}{1.570551in}}%
\pgfpathlineto{\pgfqpoint{3.312414in}{1.526286in}}%
\pgfpathlineto{\pgfqpoint{3.327505in}{1.489436in}}%
\pgfpathlineto{\pgfqpoint{3.338823in}{1.466776in}}%
\pgfpathlineto{\pgfqpoint{3.350141in}{1.448449in}}%
\pgfpathlineto{\pgfqpoint{3.361459in}{1.434499in}}%
\pgfpathlineto{\pgfqpoint{3.372777in}{1.424951in}}%
\pgfpathlineto{\pgfqpoint{3.380322in}{1.421038in}}%
\pgfpathlineto{\pgfqpoint{3.387867in}{1.419086in}}%
\pgfpathlineto{\pgfqpoint{3.395413in}{1.419090in}}%
\pgfpathlineto{\pgfqpoint{3.402958in}{1.421044in}}%
\pgfpathlineto{\pgfqpoint{3.410503in}{1.424937in}}%
\pgfpathlineto{\pgfqpoint{3.418049in}{1.430756in}}%
\pgfpathlineto{\pgfqpoint{3.429367in}{1.443056in}}%
\pgfpathlineto{\pgfqpoint{3.440685in}{1.459582in}}%
\pgfpathlineto{\pgfqpoint{3.452003in}{1.480247in}}%
\pgfpathlineto{\pgfqpoint{3.463321in}{1.504953in}}%
\pgfpathlineto{\pgfqpoint{3.478411in}{1.543979in}}%
\pgfpathlineto{\pgfqpoint{3.493502in}{1.589674in}}%
\pgfpathlineto{\pgfqpoint{3.508593in}{1.641688in}}%
\pgfpathlineto{\pgfqpoint{3.527456in}{1.714990in}}%
\pgfpathlineto{\pgfqpoint{3.546319in}{1.796707in}}%
\pgfpathlineto{\pgfqpoint{3.568955in}{1.904572in}}%
\pgfpathlineto{\pgfqpoint{3.595364in}{2.041733in}}%
\pgfpathlineto{\pgfqpoint{3.629318in}{2.231442in}}%
\pgfpathlineto{\pgfqpoint{3.685908in}{2.564022in}}%
\pgfpathlineto{\pgfqpoint{3.734953in}{2.847931in}}%
\pgfpathlineto{\pgfqpoint{3.765134in}{3.011530in}}%
\pgfpathlineto{\pgfqpoint{3.791543in}{3.143629in}}%
\pgfpathlineto{\pgfqpoint{3.814179in}{3.246584in}}%
\pgfpathlineto{\pgfqpoint{3.836815in}{3.338511in}}%
\pgfpathlineto{\pgfqpoint{3.855678in}{3.405765in}}%
\pgfpathlineto{\pgfqpoint{3.874542in}{3.463830in}}%
\pgfpathlineto{\pgfqpoint{3.889632in}{3.503299in}}%
\pgfpathlineto{\pgfqpoint{3.904723in}{3.536306in}}%
\pgfpathlineto{\pgfqpoint{3.916041in}{3.556706in}}%
\pgfpathlineto{\pgfqpoint{3.927359in}{3.573302in}}%
\pgfpathlineto{\pgfqpoint{3.938677in}{3.586043in}}%
\pgfpathlineto{\pgfqpoint{3.949995in}{3.594893in}}%
\pgfpathlineto{\pgfqpoint{3.957540in}{3.598621in}}%
\pgfpathlineto{\pgfqpoint{3.965086in}{3.600606in}}%
\pgfpathlineto{\pgfqpoint{3.972631in}{3.600849in}}%
\pgfpathlineto{\pgfqpoint{3.980176in}{3.599350in}}%
\pgfpathlineto{\pgfqpoint{3.987722in}{3.596117in}}%
\pgfpathlineto{\pgfqpoint{3.999040in}{3.588028in}}%
\pgfpathlineto{\pgfqpoint{4.010358in}{3.576089in}}%
\pgfpathlineto{\pgfqpoint{4.021676in}{3.560347in}}%
\pgfpathlineto{\pgfqpoint{4.032994in}{3.540864in}}%
\pgfpathlineto{\pgfqpoint{4.048084in}{3.509197in}}%
\pgfpathlineto{\pgfqpoint{4.063175in}{3.471219in}}%
\pgfpathlineto{\pgfqpoint{4.078266in}{3.427172in}}%
\pgfpathlineto{\pgfqpoint{4.097129in}{3.364001in}}%
\pgfpathlineto{\pgfqpoint{4.115993in}{3.292382in}}%
\pgfpathlineto{\pgfqpoint{4.138629in}{3.196233in}}%
\pgfpathlineto{\pgfqpoint{4.161265in}{3.090149in}}%
\pgfpathlineto{\pgfqpoint{4.187673in}{2.955651in}}%
\pgfpathlineto{\pgfqpoint{4.217855in}{2.790596in}}%
\pgfpathlineto{\pgfqpoint{4.259354in}{2.550191in}}%
\pgfpathlineto{\pgfqpoint{4.376307in}{1.863591in}}%
\pgfpathlineto{\pgfqpoint{4.410261in}{1.680985in}}%
\pgfpathlineto{\pgfqpoint{4.440442in}{1.530684in}}%
\pgfpathlineto{\pgfqpoint{4.466851in}{1.410099in}}%
\pgfpathlineto{\pgfqpoint{4.489487in}{1.315709in}}%
\pgfpathlineto{\pgfqpoint{4.512123in}{1.230133in}}%
\pgfpathlineto{\pgfqpoint{4.534759in}{1.153722in}}%
\pgfpathlineto{\pgfqpoint{4.553622in}{1.097201in}}%
\pgfpathlineto{\pgfqpoint{4.572485in}{1.047232in}}%
\pgfpathlineto{\pgfqpoint{4.591349in}{1.003789in}}%
\pgfpathlineto{\pgfqpoint{4.610212in}{0.966782in}}%
\pgfpathlineto{\pgfqpoint{4.625303in}{0.941710in}}%
\pgfpathlineto{\pgfqpoint{4.640394in}{0.920559in}}%
\pgfpathlineto{\pgfqpoint{4.655484in}{0.903201in}}%
\pgfpathlineto{\pgfqpoint{4.670575in}{0.889492in}}%
\pgfpathlineto{\pgfqpoint{4.685666in}{0.879265in}}%
\pgfpathlineto{\pgfqpoint{4.700756in}{0.872340in}}%
\pgfpathlineto{\pgfqpoint{4.715847in}{0.868521in}}%
\pgfpathlineto{\pgfqpoint{4.730938in}{0.867598in}}%
\pgfpathlineto{\pgfqpoint{4.746028in}{0.869350in}}%
\pgfpathlineto{\pgfqpoint{4.761119in}{0.873548in}}%
\pgfpathlineto{\pgfqpoint{4.776210in}{0.879954in}}%
\pgfpathlineto{\pgfqpoint{4.795073in}{0.890699in}}%
\pgfpathlineto{\pgfqpoint{4.813936in}{0.904036in}}%
\pgfpathlineto{\pgfqpoint{4.836572in}{0.922785in}}%
\pgfpathlineto{\pgfqpoint{4.866754in}{0.951086in}}%
\pgfpathlineto{\pgfqpoint{4.979934in}{1.061393in}}%
\pgfpathlineto{\pgfqpoint{5.006342in}{1.082195in}}%
\pgfpathlineto{\pgfqpoint{5.028978in}{1.097201in}}%
\pgfpathlineto{\pgfqpoint{5.047842in}{1.107435in}}%
\pgfpathlineto{\pgfqpoint{5.066705in}{1.115425in}}%
\pgfpathlineto{\pgfqpoint{5.085569in}{1.121044in}}%
\pgfpathlineto{\pgfqpoint{5.104432in}{1.124204in}}%
\pgfpathlineto{\pgfqpoint{5.123295in}{1.124852in}}%
\pgfpathlineto{\pgfqpoint{5.142159in}{1.122975in}}%
\pgfpathlineto{\pgfqpoint{5.161022in}{1.118591in}}%
\pgfpathlineto{\pgfqpoint{5.179885in}{1.111754in}}%
\pgfpathlineto{\pgfqpoint{5.198749in}{1.102547in}}%
\pgfpathlineto{\pgfqpoint{5.217612in}{1.091082in}}%
\pgfpathlineto{\pgfqpoint{5.240248in}{1.074536in}}%
\pgfpathlineto{\pgfqpoint{5.262884in}{1.055214in}}%
\pgfpathlineto{\pgfqpoint{5.289293in}{1.029589in}}%
\pgfpathlineto{\pgfqpoint{5.319474in}{0.996922in}}%
\pgfpathlineto{\pgfqpoint{5.357201in}{0.952332in}}%
\pgfpathlineto{\pgfqpoint{5.417563in}{0.876590in}}%
\pgfpathlineto{\pgfqpoint{5.477926in}{0.801750in}}%
\pgfpathlineto{\pgfqpoint{5.515653in}{0.758305in}}%
\pgfpathlineto{\pgfqpoint{5.549607in}{0.722690in}}%
\pgfpathlineto{\pgfqpoint{5.579788in}{0.694414in}}%
\pgfpathlineto{\pgfqpoint{5.606197in}{0.672595in}}%
\pgfpathlineto{\pgfqpoint{5.632606in}{0.653678in}}%
\pgfpathlineto{\pgfqpoint{5.659014in}{0.637753in}}%
\pgfpathlineto{\pgfqpoint{5.681650in}{0.626496in}}%
\pgfpathlineto{\pgfqpoint{5.704286in}{0.617414in}}%
\pgfpathlineto{\pgfqpoint{5.726922in}{0.610437in}}%
\pgfpathlineto{\pgfqpoint{5.753331in}{0.604819in}}%
\pgfpathlineto{\pgfqpoint{5.760876in}{0.603684in}}%
\pgfpathlineto{\pgfqpoint{5.760876in}{0.603684in}}%
\pgfusepath{stroke}%
\end{pgfscope}%
\begin{pgfscope}%
\pgfpathrectangle{\pgfqpoint{0.800000in}{0.528000in}}{\pgfqpoint{4.960000in}{3.696000in}} %
\pgfusepath{clip}%
\pgfsetbuttcap%
\pgfsetroundjoin%
\pgfsetlinewidth{1.003750pt}%
\definecolor{currentstroke}{rgb}{1.000000,0.000000,0.000000}%
\pgfsetstrokecolor{currentstroke}%
\pgfsetdash{}{0pt}%
\pgfsys@defobject{currentmarker}{\pgfqpoint{-0.034722in}{-0.034722in}}{\pgfqpoint{0.034722in}{0.034722in}}{%
\pgfpathmoveto{\pgfqpoint{0.000000in}{-0.034722in}}%
\pgfpathcurveto{\pgfqpoint{0.009208in}{-0.034722in}}{\pgfqpoint{0.018041in}{-0.031064in}}{\pgfqpoint{0.024552in}{-0.024552in}}%
\pgfpathcurveto{\pgfqpoint{0.031064in}{-0.018041in}}{\pgfqpoint{0.034722in}{-0.009208in}}{\pgfqpoint{0.034722in}{0.000000in}}%
\pgfpathcurveto{\pgfqpoint{0.034722in}{0.009208in}}{\pgfqpoint{0.031064in}{0.018041in}}{\pgfqpoint{0.024552in}{0.024552in}}%
\pgfpathcurveto{\pgfqpoint{0.018041in}{0.031064in}}{\pgfqpoint{0.009208in}{0.034722in}}{\pgfqpoint{0.000000in}{0.034722in}}%
\pgfpathcurveto{\pgfqpoint{-0.009208in}{0.034722in}}{\pgfqpoint{-0.018041in}{0.031064in}}{\pgfqpoint{-0.024552in}{0.024552in}}%
\pgfpathcurveto{\pgfqpoint{-0.031064in}{0.018041in}}{\pgfqpoint{-0.034722in}{0.009208in}}{\pgfqpoint{-0.034722in}{0.000000in}}%
\pgfpathcurveto{\pgfqpoint{-0.034722in}{-0.009208in}}{\pgfqpoint{-0.031064in}{-0.018041in}}{\pgfqpoint{-0.024552in}{-0.024552in}}%
\pgfpathcurveto{\pgfqpoint{-0.018041in}{-0.031064in}}{\pgfqpoint{-0.009208in}{-0.034722in}}{\pgfqpoint{0.000000in}{-0.034722in}}%
\pgfpathclose%
\pgfusepath{stroke}%
}%
\begin{pgfscope}%
\pgfsys@transformshift{3.142644in}{2.373438in}%
\pgfsys@useobject{currentmarker}{}%
\end{pgfscope}%
\begin{pgfscope}%
\pgfsys@transformshift{2.373019in}{2.370539in}%
\pgfsys@useobject{currentmarker}{}%
\end{pgfscope}%
\end{pgfscope}%
\begin{pgfscope}%
\pgfsetrectcap%
\pgfsetmiterjoin%
\pgfsetlinewidth{0.803000pt}%
\definecolor{currentstroke}{rgb}{0.000000,0.000000,0.000000}%
\pgfsetstrokecolor{currentstroke}%
\pgfsetdash{}{0pt}%
\pgfpathmoveto{\pgfqpoint{0.800000in}{0.528000in}}%
\pgfpathlineto{\pgfqpoint{0.800000in}{4.224000in}}%
\pgfusepath{stroke}%
\end{pgfscope}%
\begin{pgfscope}%
\pgfsetrectcap%
\pgfsetmiterjoin%
\pgfsetlinewidth{0.803000pt}%
\definecolor{currentstroke}{rgb}{0.000000,0.000000,0.000000}%
\pgfsetstrokecolor{currentstroke}%
\pgfsetdash{}{0pt}%
\pgfpathmoveto{\pgfqpoint{5.760000in}{0.528000in}}%
\pgfpathlineto{\pgfqpoint{5.760000in}{4.224000in}}%
\pgfusepath{stroke}%
\end{pgfscope}%
\begin{pgfscope}%
\pgfsetrectcap%
\pgfsetmiterjoin%
\pgfsetlinewidth{0.803000pt}%
\definecolor{currentstroke}{rgb}{0.000000,0.000000,0.000000}%
\pgfsetstrokecolor{currentstroke}%
\pgfsetdash{}{0pt}%
\pgfpathmoveto{\pgfqpoint{0.800000in}{0.528000in}}%
\pgfpathlineto{\pgfqpoint{5.760000in}{0.528000in}}%
\pgfusepath{stroke}%
\end{pgfscope}%
\begin{pgfscope}%
\pgfsetrectcap%
\pgfsetmiterjoin%
\pgfsetlinewidth{0.803000pt}%
\definecolor{currentstroke}{rgb}{0.000000,0.000000,0.000000}%
\pgfsetstrokecolor{currentstroke}%
\pgfsetdash{}{0pt}%
\pgfpathmoveto{\pgfqpoint{0.800000in}{4.224000in}}%
\pgfpathlineto{\pgfqpoint{5.760000in}{4.224000in}}%
\pgfusepath{stroke}%
\end{pgfscope}%
\end{pgfpicture}%
\makeatother%
\endgroup%
}}
  \caption{Received signal after demodulation, pulse compression and interpolation using an interpolation number of 100.}
  \label{fig:task3}
\end{figure}
\end{document}